\section{ 第二部}


一

后方的人看到一列列军车开往前方的时候,会感到无比喜悦和兴奋,觉得这些大炮,这些新涂了漆的坦克正是担负朝夕盼望的总攻任务的,战争的胜利结局很快就要来到了。

离了预备队登上军车的人心情特别紧张。年轻的排长们仿佛看到了斯大林的密令……当然,老练一些的人根本不考虑这类事,而是喝开水,在小桌上或在靴后跟上捶里海鱼干,谈着少校的风流韵事,谈着到下一个枢纽站可以换到什么货物。久经沙场的人仿佛已看到,部队怎样在前线附近只有德国轰炸机到过的偏僻小站下车,而新兵们一遇到轰炸就会多少失去兴奋的心情……在路上睡肿了眼皮的人再也无法睡觉,日日夜夜行军,没工夫吃,没工夫喝,滚烫的马达不停地轰鸣,震得两鬓隐隐作痛,两手没有力气抓方向盘。指挥员天天收到看不完的密码电报,时时刻刻在无线电报话机里听到训斥和骂娘,司令部要求快点儿把缺口堵住,在这儿再也没有人过问新部队在练习射击中达到什么指标了。“进攻,进攻,进攻!”部队指挥员耳朵里响着的就是这个词儿。于是他进攻,再不怠慢,全力以赴。有时部队在行军中,还没有弄清地势,就径直投入战斗,这时候会有一个疲惫而紧张的声音说:“快点儿进行反击,就在这片高地上,我们都打光啦,可是他们还在拼命往前攻,我们他妈的完蛋啦!”

连日来在路上的轧轧声与轰轰声,在坦克手、报话兵和瞄准手的头脑里,和德国飞机的嗡嗡声、地雷爆炸的喀嚓声混到了一起。

在这里特别能看到战争的疯狂—一个钟头过去,便是一片凄惨景象:一辆辆被烧毁、散了架的坦克冒着烟,炮被打坏,履带被打断。

几个月刻苦的训练哪儿去了?炼钢工、电工们顽强勤奋的劳动哪儿去了?

上级首长为了掩盖让刚刚开到的部队仓促投入战斗的过失,掩盖该部队几乎无益的牺牲,向上面做不痛不痒的汇报:“刚刚开到的预备部队投入战斗,在一定时间里阻止了敌军的推进,使我有可能重新部署兵力。”

假如不是一个劲儿地喊“进攻,进攻”,假如让部队摸清地势,不闯入布雷区,那样的话,坦克即使不起什么决定作用,也会好好打一阵子,给德国人造成很大的不痛快和不方便。

诺维科夫的坦克军向前方开拔着。

没有打过仗的天真的坦克手小伙子们以为,他们正是要参加决定性战役的。尝过战争滋味的人就笑话他们。第一旅旅长马卡罗夫和全军最出色的坦克营营长法托夫就很清楚这一切是怎么一回事儿,他们见识过不只一次了。

持怀疑和悲观态度的人都是很现实的人,有过痛苦经验的人,因为流过血,遭过难,对战争有更多的理解。就这一点来说,他们比那些大大咧咧的幼稚的人好些。但是有过痛苦经验的人错了。诺维科夫上校率领的坦克手们要参加的确实是决定性的战斗,这场战斗决定了战争的命运,也决定了千千万万人战后的生活。

二

诺维科夫接到命令,到达古比雪夫以后,要和总参谋部的代表留京中将取得联系,最高统帅部有许多问题需要了解。

诺维科夫原以为会有人在车站迎接他的,但是担任车站军代表的一名目光粗野、到处乱看,同时又疲惫无神的少校说,没有任何人问起诺维科夫。想在车站给将军打个电话也打不成,将军的电话号码严格保密,没办法打通。

诺维科夫便步行前往军区司令部。

来到车站广场上,他感到很不自在。野战部队的指挥官突然来到陌生的城市环境中,往往有这样的感觉。自己处于生活中心地位的感觉一下子消失了,在这儿既没有电话员给他递话筒,又没有司机为他开着汽车到处跑。

在圆石铺砌的大街上,人们在匆匆忙忙地跑着,跑到配给商店门口去排队:“谁是队尾?……我在您后面……”

对于这些提着叮当响的大桶小桶的人们,似乎再没有什么事比到食品店门口排队更重要了。特别使诺维科夫生气的是他遇到的一些军人,几乎每个人手里都提着小包大包。诺维科夫心想:“真该把他们这些狗崽子都抓起来,装上军车,带到前线去。”

难道他今天能看到她吗?他在街上走着,想着她。叶尼娅,你好!

他和留京将军在军区司令办公室里见面的时间不长。刚开始谈话,总参就给将军打来电话,要他火速飞往莫斯科。

留京向诺维科夫表示了歉意,便拨通了市内电话。

“玛莎,情况变啦。天一亮飞机就起飞,你转告安娜·阿里斯塔尔霍芙娜。土豆咱们来不及带了,农场还有几麻袋……”他那苍白的脸显得不耐烦,难受地皱着眉头,看样子,他打断了像流水一样顺着电话线向他涌来的话,说道:“没办法,总不能向最高统帅部报告说,因为一件女大衣没做成,我不能起飞呀。”

将军放下话筒,对诺维科夫说:

“上校同志,您以为,坦克的传动部分符合我们对设计人员提出的要求吗?”

这次谈话使诺维科夫感到很不舒服。他在坦克军里待了几个月,学会了准确地看人,就是说,看人的实在分量。他一眼就可以准确无误地掂量出到军里来找他的那些代表、特派员,各种委员会的领导人、检查员、指导员的分量。

他知道轻声慢语说出的话“马林科夫同志要我转告您……”的意义;他知道,有些人戴着勋章和将军肩章,又有口才,嗓门儿又大,却没有本事弄到一吨柴油,无权任命一个仓库管理员或者解除一个文书的职务。

留京所占据的不是庞大的国家机构的高层。他是做配角,他的工作只是提供统计数字,了解基本情况,做一般化的解释说明,所以诺维科夫一面和他谈话,一面看起表来。将军把老大的记事本合上。

“上校同志,很遗憾,时候不早了,明天一早我还要赶往总参去呢。不过没什么,总还可以在莫斯科见到您。”

“是的,中将同志,总有一天我会带着我的坦克上莫斯科去。”诺维科夫冷冷地回答说。

他们握手告别。留京请他代为向涅乌多布诺夫问好,过去他们在一块儿工作过的。诺维科夫还在宽敞的办公室的绿色地毯上走着,就听见留京对着话筒说:

“给我接一号农场场长办公室。”

诺维科夫心想:“他要抓紧时间搞土豆。”

他朝叶尼娅的住处走去。他在那个闷热的夏夜曾经走到她在斯大林格勒的家的门口,那是从草原上去的,草原上到处是撤退时的硝烟和灰尘。现在他又去找她了,似乎在那个人与这个人之间有一道深渊,可实际上他依然是那样,他依然是他,是同一个人。

“这一次你是我的了,”他想,“你是我的了。”

三

这是一座两层楼的旧式建筑,是一座气候不随着季节变化的结实楼房,墙壁很厚,到了夏天依然凉丝丝的,而到秋凉时候还保留着窒闷和带灰尘的热气。

他按过门铃,一股热气从打开的门里朝他扑来,他看见叶尼娅站在堆满篓子和箱子的过道里。他看见的是她,既没有看见她头上的白头巾,没有看见那黑色连衣裙,也没有看见她的眼睛和脸、她的手臂和双肩……似乎不是他的眼睛看见了她,而是那颗没有视觉的心看见了她。她啊呀了一声,多少向后退了退,就像很多人因为意外感到吃惊时那样。

他向她问好,她也对他说了一句什么话。

他向她走去,闭上眼睛,又感到活着很幸福,又感到宁愿此时此刻马上死去,也感触到她的温暖。

为了享受他从未体验过的爱情,享受幸福,原来既不需要眼睛,也不需要思想,不需要说话。

她问他话,他一面回答,一面跟着她在黑糊糊的走廊里走,拉着她的手,就好像一个小孩子怕在人群里丢失了。

“这走廊好宽呀,”他想道,“简直可以开坦克了。”

他们走进一间屋子,这间屋子有一个窗户对着邻屋一堵没有窗户的墙。

靠墙有两张床。一张床上铺着灰色被子,有一个压得平平的、皱皱巴巴的枕头;另一张床上罩着白色花边床罩,还有一个打松的枕头。白色床罩上方贴着几张小画片,上面有穿着晚礼服的新年和圣诞节美人,还有刚刚要出鸡蛋壳的小鸡。

桌子上堆满一卷一卷的绘图纸,桌角上有一块面包,半个干蒜头,还有一瓶素油。

“叶尼娅……”他说。

她的目光平常带有嘲笑的意味和注视的神气,这会儿却显得很特别,很奇怪。她说:

“您饿了吧,您是刚刚来到吧?”

她显然是想破坏和打碎已经出现并且已经无法打碎的新东西。他变得有些不同了,不是过去那样了,这个人已经有权统率成百上千的人,统率阴森可怕的战争机器,眼睛却又流露着一个不幸的小伙子那种幽怨的神气。由于这种不相称,她心慌意乱,很想对他抱着一种宽容,甚至怜悯,不去理睬他的魅力。自由曾是她的幸福;现在自由正离她而去,可她也感到幸福。

突然,他开口说道:

“怎么,难道你还不明白!”说完,他又一次再也听不见自己的话和她的话了。他心中又出现了幸福感和一种与此有关的感情:哪怕马上去死,也没有什么遗憾了。她搂住他的脖子,她的头发像温暖的水,洒在他的额头上,他的面颊上,他在这披散的黑发丛中看到了她的眼睛。

她的柔声细语淹没了战争的声音,淹没了坦克的轧轧声……

晚上,他们喝开水,吃面包,叶尼娅说:

“首长已经吃不惯黑面包啦。”

她把放在窗外的一锅荞麦饭端了进来。已经冰凉的老大的荞麦粒已经变成紫色和蓝色。麦粒上还出了一层冷汗。“真像波斯丁香花。”叶尼娅说。诺维科夫尝了尝这波斯丁香花,心想:“这东西真不好吃!”

“首长已经吃不惯啦。”她又说。

他心想:“幸亏没有听格特马诺夫的话,幸亏没有带吃的东西来。”

他说:

“战争开始的时候,我在布列斯特,在空军集团军里。飞行员们朝飞机场奔去,我听到一个波兰妇女高声问:‘这是什么人?’一个波兰小孩子回答说:‘这是俄罗斯人,当兵的。’这时候我特别强烈地感觉到:我是俄罗斯人,俄罗斯人……你要知道,我一直没忘记我是俄罗斯人,可是这时候心里怦怦跳起来:我是俄罗斯人,我是俄罗斯人。说实在的,战前可是用另外一种精神教育我们……今天,也就是这会儿,是我最好的日子,这会儿我看着你,又像那时候一样—我痛苦、我幸福都因为我是俄罗斯人……这就是我想对你说的……”他问:“你怎么了?”

她眼前仿佛闪过克雷莫夫那一头乱发的头。天啊,难道她永远和他分手了吗?正是在这幸福时刻,她觉得永远和他分手是难以忍受的。

有一会儿,似乎她就要把今天,把今天吻她的这个人的话同已经逝去的岁月连接起来,一下子弄清楚自己一生的真正出路,就要看到过去未能看清的东西—自己的心的深处。正是心的深处在决定今后的命运。

“这间屋子是一位德国老奶奶的,”叶尼娅说,“是她让我住在这儿的。这张很洁净的白白的床就是她的。比她更随和、更老实的人我这一辈子还没有见过……说也奇怪,就在和德国人打仗的时候,我还是相信,她是这个城市里最善良的人。奇怪吗?”

“她很快就要回来了吧?”他问。

“不,跟她打的仗已经打完了,把她送走了。”

“那也没办法。”诺维科夫说。

她很想对他说说她是怎样怜悯被她抛弃的克雷莫夫。他连可以通通信的人都没有了,也没有人需要他去看望了,他只有苦恼,无法排遣的苦恼,孤独。

此外她还想谈谈里蒙诺夫,谈谈沙尔戈罗茨基,谈谈与这两个人有联系的很有意思然而不易理解的一些新的说法。想说说小时候亨利逊怎样把沙波什尼科夫家的小孩子们说的一些好笑的话记下来,记录这些话的笔记本就在桌子上,可以看一看。很想说一说报户口的经过,说一说那个户籍股长。但是她还不够信任他,在他面前怕难为情。他要不要听她说的呢?

很奇怪……她就像重新在经历她和克雷莫夫关系的破裂,她的心灵深处一直还以为可以破镜重圆,恢复过去的一切。这一点使她心里得到安慰。这会儿,当她感到有一股力量将她卷起时,她又痛苦,又惶恐:难道这就永远、永远不再恢复了吗?可怜的克雷莫夫,真可怜啊!为什么他这样苦?

“这算怎么回事儿啊?”她说。

“你是我诺维科夫家的人啦。”他随口说。

她笑起来,凝视着他的脸。

“你是陌生人,完全是陌生人嘛。说真的,你是什么人?”

“这我不知道。可是我知道,你是我的人了。”

她已经身不由己了。她一面给他往杯子里倒开水,一面问:

“还要面包吗?”

忽然她又说:

“如果克雷莫夫出什么事,受重伤或者进监狱,我还要回到他身边去。这一点你要考虑。”

“他因为什么要进监狱?”他正色问道。

“哼,进监狱还不容易吗,他过去搞过共产国际,托洛茨基也认识他,看过他一篇文章之后,还说过:‘真精彩!’”

“你试试看,要是再回去,他还要把你赶走呢。”

“你别操心。那就是我的事了。”

他对她说,战后她将成为一座大房子的女主人,房子将是很漂亮的,房子后面还会有花园。

难道就这样定了,就这样一辈子吗?

不知为什么她很希望让诺维科夫明白:克雷莫夫是一个聪明人,一个有才华的人,她对克雷莫夫是有感情的,应该说,是很爱他的。她不希望诺维科夫因为她爱克雷莫夫而产生醋意,但是她所做的一切都是在不自觉地挑动他的醋意。不过她把托洛茨基的话对他说了,这话克雷莫夫只对她一个人说过,现在她也只是对他一个人说。“如果当时还有人知道这件事,克雷莫夫在一九三七年未必能逃脱。”她既然爱诺维科夫,就应该高度信任他,于是,她把一个她对不起的人的命运交给了他。

她的脑子里有各种各样的想法,想将来,想今天,想过去,她时而发呆,时而高兴,羞涩,忐忑,愁闷,害怕,不知道母亲、姐姐、侄子、薇拉,还有不少人会怎样看待她生活中发生的这一变化。如果诺维科夫和里蒙诺夫谈话,听听别人谈诗歌和绘画,又会怎样呢?他不会感到羞惭的,虽然他不知道夏加尔和马蒂斯……他是强者,强者,强者。连她都服从了。战争会结束的。难道,难道她再也见不到克雷莫夫了吗?天啊,天啊,她干的什么事呀。现在就不想这些吧。因为还不知道今后一切会怎么样呢。

“现在我才明白:我还一点不了解你。我不是开玩笑:你是陌生人。房子、花园,干吗要说这些呀?你是当真的吗?”

“你要是愿意,我就复员,到西伯利亚东部什么地方去,到建筑工地上去做一名工长。咱们就住在带家眷的棚屋里。”

这是真心话,他不是开玩笑。

“不一定住带家眷的屋。”

“一定要住。”

“你简直疯啦。为什么要这样?”她心里想:“还有克雷莫夫呢。”

“怎么为什么?”他惊骇地问。

可是他既不想将来,也不想过去。他觉得很幸福。有时想到,过几分钟他们就要分别了,也不觉得可怕。他和她坐在一起,他看着她……她是他诺维科夫的人了……他觉得很幸福。他爱的不是她聪明、漂亮、年轻。他确实一直在爱她。起初他不敢幻想她会成为他的妻子。后来他却幻想了很多年。但就是今天,他依然带着腼腆和胆怯的神气在看她的笑容,听她的一些带有讥笑意味的话。不过,他看出来,新的情况出现了。

她看着他准备动身,便说:

“到时候啦,斯捷潘·拉辛该回到沸沸扬扬的队伍里去,该把我扔进涌来的浪涛里啦。”

等到他开始告别的时候,他明白了,她并不是多么刚强的,女人总归是女人,哪怕她绝顶聪明,而且很会讥笑人。

“有多少话想说啊,可是什么也没有说。”她说。

不过,倒也不是这样。决定人的一生的最重要的事,在他们相会的时候已经定下来了。他的确是爱她的。

四

诺维科夫朝车站走去。

……叶尼娅,她那心慌意乱的低语,赤裸的双脚,亲热的低语,在分别时的眼泪,令他迷恋的魅力,她的贫困与纯洁,她头发的味道,她的可爱的羞涩,她的身体的温暖,他因为意识到自己的工人、士兵式的单纯而感到腼腆,又因为自己带有工人、士兵式的单纯而自豪。

诺维科夫顺着铁路线朝前走去,他的热辣、模糊的思想云团之中扎进来一根尖尖的针—一个当兵的在路途中的恐惧:军车是不是开走了?

他老远看见一节节铁路货车、盖着帆布的一辆辆凸凸棱棱的钢甲坦克、戴着黑色钢盔的岗哨,看见挂着白窗帘的军部车厢。

他从一名立正的哨兵身旁走进车厢。

副官维尔什科夫因为诺维科夫没有带他上市里去,很不高兴,所以一声不响地把统帅部来的密码电报放到小桌上:开往萨拉托夫,然后开上阿斯特拉罕支线……

涅乌多布诺夫将军走进来,也不看诺维科夫的脸,而是看着他手里的电报,说:

“路线定下来了。”

“是的,涅乌多布诺夫同志,”诺维科夫说,“不是路线,是命运已经定了:斯大林格勒!”他又说:“留京中将问候您。”

“啊,啊,啊。”涅乌多布诺夫说。实在弄不清他这冷漠的“啊,啊,啊”是针对什么的:是对将军的问候,还是斯大林格勒的命运?

他是一个奇怪的人,诺维科夫觉得他有些可怕:不论路上出什么事儿—等待对向开来的列车通过,车厢的轴箱发生故障,或者调度员没及时给发车信号—这时候涅乌多布诺夫就来了劲儿,说:

“把名字记下来,记下来,这是有意破坏,应该抓起来,坏蛋。”

诺维科夫在内心深处对于所谓人民敌人、富农和富农帮凶没有仇恨,没有恶感。他从来不曾想过把什么人关进监狱,把什么人送交法庭,或者在大会上揭发什么人。不过他认为,这种好心肠和恨不起来是由于自己政治觉悟不高。

可是诺维科夫却觉得,涅乌多布诺夫一见到人,首先出现和马上出现的便是警惕性,他会抱着怀疑的态度想:“啊呀,亲爱的同志,你不是敌人吗?”昨天他还对诺维科夫和格特马诺夫说过,有一些反革命的建筑师,曾经企图把莫斯科的一些主要街道变为敌人空军的降落场。

“依我看,这是胡说八道,”诺维科夫说,“这是军事上的无知。”

现在涅乌多布诺夫和诺维科夫谈起自己喜欢谈的第二个话题—谈家庭生活。他摸了摸车厢里的暖气管,说起战前不久在他的别墅里安装的暖气设备。

这个话题出乎意外地使诺维科夫很感兴趣,他认为很重要,并且请涅乌多布诺夫画了一张别墅暖气设备的线路图,他把图纸折叠起来,放进军装的内口袋。

“将来会用得着的。”他说。

不久,格特马诺夫走了进来,高高兴兴地大声向诺维科夫表示欢迎:“好哇,我们的军长又回来啦,我们本来还想重新选举首领呢,以为斯捷潘·拉辛把自己的队伍扔掉啦。”他眯缝起眼睛,很和善地看着诺维科夫。诺维科夫听到政委开玩笑,也在笑着,可是他心里出现了已经成为习惯的紧张。

格特马诺夫开的玩笑有一个很奇怪的特点,他似乎知道诺维科夫的很多事情,他开的玩笑正是暗示这方面的事。于是他重复了一遍叶尼娅在分别时说的话,不过这当然是无意的巧合。

格特马诺夫看了看表,说:

“好啦,两位大人,该我上市里去一趟啦,没意见吧?”

“请吧,您走了,我们在这儿也不会感到寂寞。”诺维科夫说。

“这话对,”格特马诺夫说,“军长同志,您在古比雪夫总不会感到寂寞的。”

这句玩笑话就不是巧合了。格特马诺夫已经站到单间门口,问道:

“军长同志,沙波什尼科娃同志身体好吗?”

格特马诺夫是一本正经的,眼中也没有笑意。

“谢谢,很好,工作干得不错。”

诺维科夫说过这话,就想把话引开,于是便问涅乌多布诺夫:

“涅乌多布诺夫同志,您怎么不想到市里去走走?”

“市里我什么没有见过呀?”涅乌多布诺夫回答说。

他们坐在一起。诺维科夫一面听涅乌多布诺夫说话,一面翻看文件,看过了就放到一边,并且不时地说:“噢,噢,噢,您说下去……”诺维科夫一辈子总是向首长汇报,首长在听汇报的时候总是在看文件,一面漫不经心地说:“噢,噢,您说下去……”诺维科夫过去总觉得这是一种侮辱,他认为自己永远也不会这样做。

“是这样,”诺维科夫说,“为了维修,咱们应该早点儿要求补充维修技术人员。修车轮的人咱们有的是,可是修履带的人几乎一个没有。”

“我已经写好了申请表。我想,最好直接交给总指挥,反正总要找他批。”

“噢,噢,噢。”诺维科夫说。他在申请表上签了字,又说:“要检查检査各旅的防空装置,过了萨拉托夫可能会有空袭。”

“我已经在军部里发过指示了。”

“这不管用。应该让各军列指挥官各自负责,让他们在十六点以前汇报情况。要他们亲自检查,亲自汇报。”

涅乌多布诺夫说:

“萨佐诺夫担任旅参谋长的批文已经下来了。”

“真快,简直像电报。”诺维科夫说。

这一次涅乌多布诺夫没有朝旁边看,他笑了笑,知道诺维科夫很懊恼,很不自在。

诺维科夫一向没有胆量坚决维护他认为特别适宜担任指挥职务的一些人。一涉及指挥人员的政治可靠性问题,他就泄了气,就好像人的真正才干一下子就成了无关紧要的。

但是现在他火了。他不想容忍了。他看着涅乌多布诺夫,说:

“我错了,为人事档案牺牲了军事才能。到前线上咱们要改正。总不能靠人事档案作战。一出什么问题,我他妈的马上把他撤了!”

涅乌多布诺夫耸了耸肩膀,说:

“我个人对那个加尔梅克人巴桑戈夫一点意见也没有,不过最好还是要尊重俄罗斯人。各民族友谊是神圣的事,不过,您该了解,在少数民族中,抱敌对态度的人、不可靠的人、面貌不清的人占的比例很大。”

“这一点在一九三七年就该考虑,”诺维科夫说,“我有一个这样的朋友,叫米佳·叶甫谢耶夫。他天天在叫喊:‘我是俄罗斯人,这是最要紧的。’可是他这个俄罗斯人也倒了霉,被关起来了。”

“各个时期有各个时期的情况,”涅乌多布诺夫说,“关的都是坏蛋、敌人。我们是不会无缘无故关人的。过去我们和德国人缔结布列斯特和约,符合布尔什维克主义;现在斯大林同志号召彻底、干净地消灭侵入苏联国土的所有德国人,也符合布尔什维克主义。”

又换成教训的口吻说:

“在我们的时代,布尔什维克首先应该是热爱俄罗斯的人。”

诺维科夫非常气愤:他诺维科夫对俄罗斯的感情是在战火中锤炼出来的,涅乌多布诺夫的俄罗斯感情也许是从诺维科夫不曾跨过的什么办公室里借来的。

他和涅乌多布诺夫谈着,非常恼火,想着很多事情,心里很激动。他两颊通红,好像风吹过或者太阳炙晒过,心咚咚跳着,跳得很激烈,无法平静。

似乎有一个团从他的心上走过,许多靴子齐声响:“叶尼娅,叶尼娅,叶尼娅。”

已经不再怨恨诺维科夫的维尔什科夫探进头来,用恭顺的语调说:

“上校同志,请允许我报告:炊事员不知怎样才好,等您吃饭已经等了两个多钟头了。”

“好的,好的,就是要快一点儿。”

一名满头大汗的炊事员马上带着紧张、幸福和委屈的表情跑进单间里来,摆起一碟碟乌拉尔腌制品。

“给我来一瓶啤酒。”涅乌多布诺夫懒洋洋地说。

“有,有,少将同志。”炊事员得意地说。

诺维科夫觉得,因为很久没开荤,现在突然非常想吃,眼泪都急出来了。

“首长已经吃不惯啦。”他在心里说着,想起刚刚不久前吃的冰冷的波斯丁香。

诺维科夫和涅乌多布诺夫同时朝窗外看了看:一名喝醉的坦克手由一名背枪的民警扶着,歪歪倒倒、踉踉跄跄地在铁路线上走,一面尖声叫着。坦克手想挣开,想打民警,但是民警把他抱得紧紧的,看样子,坦克手已经醉糊涂了,一会儿就忘记了要打人,忽然很亲热地在民警的脸上吻了起来。

诺维科夫对副官说:

“这真不成体统,马上去查清楚,向我汇报。”

“要把这个坏蛋、这个破坏军纪的分子枪毙。”涅乌多布诺夫说着,把窗帘拉上。

在维尔什科夫那单纯的脸上出现了复杂的表情。首先他觉得伤脑筋,这一下子军长要倒胃口了。同时他又同情那名坦克手。这种同情包含各种各样的意味:有苦笑,有鼓励,有同志般的赞赏,有父亲般的疼爱,有难过和担心。

他报告说:“是的,马上调查,汇报。”又编造理由代为开脱说:“他妈妈住在这里,他是俄罗斯人,哪儿知道分寸,心里又难过,很想最后和老母好好话别,所以喝多了一点儿。”

诺维科夫搔了搔后脑勺,把一个碟子拉到自己跟前。“不行,我再也不离开军车上哪儿去了。”他在心里对等待他的那个女子说。

格特马诺夫在快要开车的时候才回来。他满脸通红,十分快活,不吃晚饭了,只是吩咐手下人给他打开一瓶他很喜欢喝的橘子水。他哼哧哼哧地把靴子脱掉,躺到沙发床上,用一只穿袜子的脚把单间的门掩实。

他对诺维科夫说起一位当州委书记的老朋友告诉他的一些消息。那位老朋友昨天刚从莫斯科回来。他在莫斯科得到一个人接见,那个人在节庆日子里有资格登上列宁墓,但还不够跟斯大林一起,站在麦克风旁边。那个透露消息的人当然不是什么都知道,而且当然也不会把他所知道的全都告诉这位州委书记,因为这位州委书记只是在伏尔加河畔一个不大的城市里担任区委指导员时和他熟识的。这位州委书记又在无形的化学天平上称了称谈话的对象,从他听到的消息中拣出不多的一部分对这位坦克军政委说了说。当然,这位坦克军政委对诺维科夫上校说的也只是他从州委书记嘴里听到的不多的一部分……

但是这天晚上他说话用的是特别信任的语气,以前他还没有用这样的语气和诺维科夫说过话。似乎他认为,诺维科夫十分了解马林科夫有很大的实权,知道除了莫洛托夫之外,只有贝利亚能够对斯大林同志称“你”,知道斯大林同志最痛恨擅自行动,知道斯大林同志喜欢苏禄干酪,知道斯大林同志因为牙齿不好常常将面包蘸了酒吃,也知道他脸上的碎麻子是小时候出天花留下的,知道莫洛托夫同志早已不是党内第二号人物,知道斯大林同志近来已经不怎么赏识赫鲁晓夫同志了,不久前甚至在高频电话里把他臭骂了一顿。

在谈到国家最高领导人时那种推心置腹的语调,谈斯大林在和丘吉尔谈话时一面画十字一面笑着说的风趣话,谈斯大林对一位元帅的过失的不满,似乎比那个站在陵墓上的人说的带有一点儿暗示意味的话,也就是诺维科夫心里一直在盼望、在揣测的话—马上就要反攻了!—更为重要。诺维科夫心里想:“哈,我也进入上层的圈子了!”不由得在心里得意地傻笑了一下,笑过了,自己也觉得羞惭,不久军列就开动了,既没有打铃,也没有吹哨。

诺维科夫走到军车的连廊,开了门,凝视着城市上面黑沉沉的天空。又好像有步兵在心里咚咚走过:“叶尼娅,叶尼娅,叶尼娅。”悠扬的《叶尔马克之歌》的歌声透过轧轧声与轰隆声从机车方向飘过来。

车轮轧在钢轨上的隆隆声、驮载着一辆辆钢甲坦克奔赴前方的铁路货车的叮当声、年轻人的歌声、伏尔加河上吹来的冷风、浩瀚的星空,这一切似乎都换了一副面貌进入他的心田,不再像一秒钟以前那样,也不像战争开始以来这整个一年中那样了,他的心中感到有一种强悍的战斗力量,因而泛起一股豪迈的喜悦和剧烈而甜蜜的幸福感,似乎战争的面貌变了,完全不同了,不再是只有痛苦和仇恨的丑陋样子……从黑暗中飘来的惆怅而悲伤的歌声也带有威严和豪迈的意味了。

不过很奇怪,今天的幸福感没有唤起他的善心和宽恕。这种幸福感激发他的仇恨、愤怒,激发他的愿望,希望显示自己的力量,消灭阻挡这种力量的一切。

他回到单间。刚才秋夜是那样迷人,这会儿却是车厢里的滞闷,烟草、烧焦的牛油和鞋油的气味,红光满面的军部人员身上的汗味。格特马诺夫穿着睡衣,露着白白的胸膛,靠在沙发床上。

“喂,玩一会儿骨牌吧,怎么样?将军同意了。”

“没问题,可以打。”诺维科夫回答说。

格特马诺夫轻轻地打了一个饱嗝儿,用忧虑的口气说:

“恐怕我有胃溃疡,一喝酒,肚子就痛得厉害。”

“不应该让医生跟着第二军列先走。”诺维科夫说。

诺维科夫很生自己的气,心想:“我当时想安排达林斯基,费奥多连科一皱眉头,我就改变了主意。我对格特马诺夫和涅乌多布诺夫也说过,他们一皱眉头,说干吗要用受过处分的人,我就害怕了。我推荐巴桑戈夫,他们又说干吗要用非俄罗斯人,我又改变了主意……我究竟有没有自己的主意。”他看着格特马诺夫,心里想着,而且偏偏要往荒唐处想:“今天他拿我的白兰地招待别人,明天我老婆来了,他还想跟我老婆睡觉呢。”

但是他这个有充分信心可以打碎德国战争机器的脊梁骨的人,为什么在同格特马诺夫和涅乌多布诺夫交谈的时候,总感到自己软弱和胆怯?

在这幸福的一天里,他心中涌起一股强烈的愤恨,愤恨过去多年来的生活现实,愤恨这种已成为他的准则的状况:那些军事上无知然而有权有势、吃惯了佳肴美酒、挂满了勋章的人们听他的汇报,恩赐他一间领导人员住房,为他申报奖赏。一些人虽然不知道大炮口径的大小,念不通别人为他们写的讲话稿,看不懂地图,满口的错字别字,然而总是要领导他。他要向他们汇报。他们没有文化,并不因为是工人出身,要知道,他的父亲、祖父、哥哥也是矿工。有时候他觉得,这些人没有文化,正是他们的优点,有了这个优点,就不要文化了。他的知识,他的口才,他喜欢读书,都是他的缺点。在战前他觉得,这些人比他更有毅力和信心。可是战争已经证明了,就在这方面也不是这样。

战争把他推上高级指挥岗位,但实际上仍然不能当家做主。他仍然要服从他一向能感觉到、却不能理解的势力。在他统率之下没有指挥权的这两个人便是这种势力的代表。所以,当格特马诺夫跟他谈起那些权势炙手可热的人物时,他高兴得发了呆。

战争迟早会证明俄罗斯将依靠谁—是依靠他这样的人,还是依靠格特马诺夫这样的人。他所幻想的,已经实现了:他爱了很多年的女子,就要成为他的妻子了……这一天,他的坦克军接到命令,向斯大林格勒进军。

“诺维科夫同志,”格特马诺夫忽然说,“您可知道,今天你上市里去的时候,我和涅乌多布诺夫有一场争论?”

他欠起身来,喝了一口啤酒,说:

“我这人是直肠子,我要直截了当地告诉你:我们谈起了沙波什尼科娃同志。她的哥哥在一九三七年进入……”格特马诺夫朝地下指了指,“原来,那时候涅乌多布诺夫认识他,我也认识她的前夫克雷莫夫,此人得到保全,真可以说是奇迹。他是中央宣讲团里的。所以涅乌多布诺夫说,既然诺维科夫同志得到苏联人民和斯大林同志这样高的信任,就不应该跟社会政治关系不清的人结合。”

“我的个人生活跟他有什么相干?”诺维科夫说。

“就是这话,”格特马诺夫说,“这都是一九三七年遗留的问题,不能把这些问题看得太严重。不,不,您要正确理解我的意思。涅乌多布诺夫是一个很好的人,忠诚无私,是斯大林式的坚定的共产党员。但是他有一个小小的缺点—有时看不见、感觉不到新事物的出现。他认为最主要的是摘引革命导师的著作。至于现实生活所提供的经验教训,他却往往看不见。有时似乎他都不明白他是生活在什么样的国家里,他摘引的又是一些什么。战争教给我们许多新东西。罗科索夫斯基中将、戈尔巴托夫将军、普尔杜斯将军、别洛夫将军都坐过牢嘛。可是斯大林同志认为可以让他们指挥军队。今天,我去拜访的米特里奇就对我说了说罗科索夫斯基从劳改营里直接调任集团军司令的情形:他正在棚屋的洗脸池里洗裹脚布,就有人跑去叫他:‘快点儿!’他以为连脚布都不准他洗了,因为昨天一个头头儿还审讯他,把他打了一顿。谁知,一架飞机把他直接送进了克里姆林宫。我们还是应该从这一点得出一些结论的。可是咱们的涅乌多布诺夫是一九三七年的积极分子,他头脑僵化,立场是不会改变的。不知道沙波什尼科娃这位哥哥犯的是什么罪,如果还活着的话,也许贝利亚同志现在也会把他放出来,让他指挥一个集团军。克雷莫夫还在军队里嘛。人还好好的,还是党员。有什么事呢?”

但是这番话偏偏把诺维科夫惹火了。

“这跟我有屁关系!”他用老大的嗓门儿说。他第一次听到自己的嗓门儿有这样响亮,自己也觉得吃了一惊。“沙波什尼科夫是不是敌人,跟我有什么相干?我连认都不认识他!托洛茨基是对这个克雷莫夫谈过他的文章,说他的文章写得十分精彩。这跟我又有什么相干?精彩就精彩好了。就让托洛茨基,就让雷科夫,就让布哈林,就让普希金拼命赞赏他好了,跟我的生活有什么相干?我又没读过他的精彩文章。这跟沙波什尼科娃又有什么关系?怎么,难道是她一九三七年以前在共产国际工作过?同志们,好好领导作战吧!干点真正的工作!让我告诉你,算了吧!够啦!”

他的两颊火辣辣的,心剧烈地跳着。他的思想是清楚、分明、强烈的,可是脑子里迷迷糊糊:“叶尼娅,叶尼娅,叶尼娅。”

他听着自己在说话,自己感到吃惊:难道这是他,竟敢这样毫无顾忌地在对一位党的大干部说话?他心里觉得痛快,同时克制着后悔和担心的心情,看了看格特马诺夫。

格特马诺夫忽然从沙发床上跳起来,张开两条老粗的胳膊,说:

“诺维科夫同志,让我来拥抱你,你是真正的男子汉。”

诺维科夫愣了一会儿,便和他拥抱,互相吻了吻,格特马诺夫朝着过道里喊道:

“维尔什科夫,把白兰地给我们拿来,军长和政委现在要喝交谊酒啦!”

五

叶尼娅收拾好了房间,心想:“好了,行了。”就好像这一下子房间也洁净了,床也铺平整了,枕头也不打皱了,她的心也不乱了。但是等到床头边再也没有烟灰,最后一个烟头儿也从小架子边上捡走之后,叶尼娅明白了,她一直是想欺骗自己,明白了在这世界上她什么也不需要,就需要诺维科夫。她真想把她生活中发生的这件事对索菲亚·列文顿说说,就要对她说,不是对妈妈,不是对姐姐。她也模模糊糊地知道,为什么她想把这事对索菲亚说说。

“啊,索涅奇卡,索涅奇卡·列文顿尼哈。”叶尼娅把心里想的说出声来。

后来她想到,玛露霞已经不在了。她明白,没有他是不能活下去的,她拿手拼命在桌子上敲了一下。然后她说:“算了,我谁也不需要!”她说过这话,却又在诺维科夫挂军大衣的地方跪下来,说:“你要活下去啊!”

然后她心里想:“真是虚伪,我真是一个水性杨花的女人。”

她故意折磨起自己,不出声地自己对自己说起话来,假托一个又鄙俗又尖刻的人之口,不知是女人还是男人:“哼,这个女人没有男人就受不住,风流惯了,又是在这风风雨雨的年月……已经扔掉一个啦,当然,她怎么会看得起克雷莫夫,他连党内都待不稳。这会儿她要做军长夫人啦。又是那样魁伟的男子汉!哪一个女人都会想的,当然了……他不用花什么力气,她已经什么都给他了,不是吗?不用说,这会儿夜里该睡不着觉了,又担心他被打死,又担心他找上一个十九岁的电话员姑娘。”那个鄙俗而下流的人似乎窥见了连叶尼娅自己也不知道的一个念头,就又说:“没什么,没什么,你很快可以跑去找他嘛。”

她真不懂,为什么她不爱克雷莫夫了。不过这会儿也不需要懂了—她已经感到很幸福了。

忽然间,她不由得想起,是克雷莫夫阻碍着她的幸福。他一直站在她和诺维科夫中间,是他使她快活不起来。他还在毁坏她的生活。为什么她就应当永远痛苦,为什么还要受良心责备?有什么办法,不爱就是不爱!他究竟要她怎样,为什么他要一个劲儿地跟着她?她有权做一个幸福的女人,有权爱她爱的男人。为什么她总觉得克雷莫夫是个弱者,是个没办法、没主意、孤孤单单的人?他并不多么软弱!并不多么善良!

她对克雷莫夫愤恨起来。她决不拿自己的幸福给他做牺牲,决不,决不……他是一个残酷、狭隘的人,是一个顽固的狂热分子。她永远看不惯他对受难遭殃的人那种冷漠态度。这和她,和她妈妈、爸爸多么不同啊……就在俄罗斯和乌克兰农村成千上万的妇女儿童在可怕的饥馑中痛苦死去的时候,他竟说:“富农不值得怜惜。”在亚戈达和叶若夫那时候,他说:“没有罪的人是不会被抓的。”有一次妈妈说,一九一八年在卡梅申,曾经用大船把商人、房产主和他们的家小送到伏尔加河心里,把他们淹死,其中就有玛露霞在中学里的同学,有米纳耶夫家、戈尔布诺夫家、卡萨特金家、萨波什尼科夫家,克雷莫夫听了后,却很激烈地说对待这些仇恨革命的人,您说该怎么办?拿甜饼喂他们吗?”为什么她没有幸福的权利?为什么她就应该痛苦,应该怜惜一个从来不怜惜弱者的人?

但是在她的心的深处,在她痛恨和发狠的同时,她也知道自己是不对的,克雷莫夫并不那么残酷。

她脱下她在古比雪夫集市上换来的厚裙子,穿起她自己夏季穿的裙子,这是斯大林格勒大火后留下来的唯一一条裙子,一天傍晚她就是穿着这条裙子和诺维科夫一起站在斯大林格勒滨河大街霍尔祖诺夫纪念碑前的。

在亨利逊老奶奶被送走前不久,叶尼娅问她,过去是不是爱过什么人。

老人家很不好意思,说:

“是的,爱过一个黄鬈发、蓝眼睛的男孩子。他穿的是丝绒夹克,衬衣领子雪白雪白的。那年我十一岁,我和他不认识。”

这会儿那个穿丝绒夹克的鬈发男孩子在哪儿呢?亨利逊老人家又在哪儿呢?

叶尼娅坐到床上,看了看表。一般在这个时候沙尔戈罗茨基都要到她这儿来的。啊,她今天可不想听什么高深的谈论。

她很快地穿起大衣,扎好头巾。已经没意思了—军车早已开走了。

在车站的墙脚下,许许多多的人坐在提箱和包裹上。叶尼娅在车站的小巷道里漫步走着,有一个女子问她有没有乘车用餐券,另一个女子问她有没有乘车凭证……有些人迷迷糊糊地用怀疑的目光打量她。有一列货车很沉重地从第一道线路开过,车站的墙抖动着,站房的窗玻璃叮叮当当响了起来。似乎她的心也在打颤。擦着车站栏杆滑过的是一台台平板货车,上面是一辆辆的坦克。

她忽然充满了幸福感。一辆又一辆坦克滑过,还有雕塑一样坐在坦克上的一个个头戴盔形帽、斜挎冲锋枪的红军战士。

她像小孩子一样挥着手臂,朝家里走去。她把大衣敞开,看着自己夏季穿的裙子。夕阳忽然把一条条街道照得十分明亮,寒冷阴沉、破破烂烂、尘土飞扬的冬季即将降临的城市,一下子变得喜气洋洋,呈现出鲜亮的玫瑰色。她走进楼房,居民小组长加林娜因为今天在过道里见过前来找叶尼娅的上校,所以露出一副巴结的神气,笑着说:

“有您的信。”

“噢,是我时来运转啦。”叶尼娅心里想着,把信封打了开来。信是从喀山来的,是妈妈写来的。她看过前面几行,就小声叫了起来,惊慌地唤道:

“托里亚呀,托里亚呀!”

六

夜里在大街上突然意外地出现在维克托脑子里的那一想法,成了新理论的基础。他研究了几个星期得出的方程式完全没有扩展物理学家们承认的传统理论,没有成为其补充部分。相反,传统理论本身对于维克托得出的新的普遍结论倒成了部分现象,他的方程式把似乎包罗万像的传统理论包罗进去了。

维克托暂时不再上研究所去,实验室的工作由索科洛夫领导。维克托几乎不出门,只是在房里走来走去,有时在桌边坐一阵子。晚上有时出去散散步,专拣车站附近的偏僻街道走一走,为的是不碰上熟人。他在家里的生活依然和平常一样:吃饭时说说笑话,看报,听新闻广播,逗逗娜佳,向岳母问问工厂的情形,和妻子说说话。

柳德米拉觉得,丈夫在这些日子里和她一样了,做一切事情都是出于习惯,就像上了发条的钟表,心里对外在的生活没有什么感觉,他生活得很轻松,只是因为这生活他已经习惯了。但是这种相似并没有使柳德米拉和丈夫接近起来。这种相似是表面的。实际上是完全相反的原因使他们和家里人在思想上疏远了,完全相反的原因决定着他们对生和死的态度。

维克托不怀疑自己的成果。这样的信心他从来不曾有过。但是恰恰就在这时候,在把他得出的最重要的科学结论表现为公式的时候,他一点也不怀疑其正确性。在他想到一系列方程式,可以重新解释广泛的物理现象的那几分钟里,他不知为什么再也不像平素那样喜欢怀疑和动摇了,立刻就感觉出这一思路是正确的。

就连现在,当他进行的复杂的数学运算快要结束,他一再地检查自己的推论过程的时候,他的信心也没有超过在空荡荡的大街上突然冒出来的猜想使他大吃一惊的那时候。

有时候他想看清楚他走过的道路。从表面看,似乎一切都十分简单。

实验室里进行的试验应该可以证实理论的推断。事实上却没有证实。试验结果与理论的矛盾,很自然地使人怀疑试验的准确性。根据许多研究者几十年的研究得出的理论,而且这一理论也阐明了一些新的研究试验中的许多现象,这样的理论似乎是不可动摇的。反复的试验一次又一次表明,参与核反应的带电粒子出现的偏离,依然完全不符合理论的推断。不论怎样改进试验的准确性,不论怎样校正测量仪器,调制摄取核爆炸图像的感光剂,都不能解释这种完全不相符合的现象。

这时候才清楚,试验结果是不容怀疑的,于是维克托便千方百计修补理论,将一些任意的假设纳入理论中,为的是使实验室中得到的新的试验资料服从于理论。他所做的一切,都由于他承认最基本、最主要的一点:理论来自试验,因此试验不能和理论相矛盾。为了使理论和新的试验相符合,花费了大量的劳动。但是传统的理论,似乎永远不能偏离、不能违背的理论,即使修补过,也仍然不能解释越来越矛盾的试验数据。修补以后仍然无能为力,就和没有修补一样。

就在这个时候出现了新的想法。

旧的理论不再是基础,不再是根本,不再是包罗万象的整体。旧理论不是错误,不是荒唐的迷误,但是却作为局部性答案进入了新的理论……太后起身朝拜起新的王后。这一切都是在转瞬间发生的。

维克托一想到他脑子里出现新理论的情形,就感到意外和惊愕。

在这里,理论与试验相联系的简单逻辑完全不存在了。似乎地上的脚印儿没有了,他看不清他走过的道路。

以前他总认为,理论来自试验;试验产生理论。他认为,理论与新的试验数据的矛盾自然而然地导致包罗性更广的新理论的产生。但是事情很奇怪—他相信,实际情形完全不是那样。他取得成就,偏偏是在他既不想以理论联系试验,也不想以试验联系理论的时候。

新的理论似乎不是来自试验,而是来自维克托的头脑。这一点他理解得十分淸楚。新理论是很自然地出现的。头脑产生了理论。理论的逻辑推理及其因果关系,都和马尔科夫在实验室里进行的试验没有联系。似乎理论是从自由自在的思想游戏中自然而然产生的,这种似乎与试验无关的思想游戏就能够解释所有老的和新的丰富的试验资料。

试验是外部推动力,促使脑子进行思考。但试验不能决定思考的内容。

这是使人吃惊的……

他的脑子里充满了数学关系式、微分方程、概率法则、高等代数定律和数论定律。这些数学关系独立地存在于冥冥之中,超越原子核世界和星际世界,超越电磁场和引力场,超越时间和空间,超越人类历史和地球的地质史。但是却在他的头脑中。

同时他的头脑里也充满了另外一些关系和定律—量子关系,力场,可以判断核反应过程实质的恒量、光的运动、时间与空间的收缩与延伸。事情很奇怪,在一个理论物理学家看来,物质世界的变化过程仅仅是空洞的数学天地中各种定律的反映。在维克托的头脑里,不是数学反映世界,而是世界成了微分方程的投影,世界是数学的映像。

同时他的脑子里也充满了计量器和仪表所显示的数字,在感光剂和照相纸上记录粒子和核爆炸运动的一条条虚线。

同时他的脑子里也有树叶的飒飒声,也有月光,有小米饭和牛奶,有炉火的呼呼声,有乐曲声,有狗吠声,有罗马的元老院,有苏联情报局的战报,有对奴役的仇恨,有对南瓜子的喜好。

理论就是从这种杂七杂八的状态中冒出来,浮上来的,是从它的深处钻出来的,那儿既没有数学,也没有物理,没有物理实验室的试验,没有现实的经验,那儿没有意识,只有下意识的可燃的泥炭……

与现实世界没有联系的数学推理,反映、表现和体现在现实的物理学理论中,而理论忽然又极其精确地化作复杂的虚线状的图案,印在照相纸上。在头脑里产生了这一切的人,看着证实了他所发现的真理的一道道微分方程和一片片照相纸,抽搭起来,不住地揩着往外直涌的幸福的泪水。话又说回来,如果没有那些不成功的试验,如果不出现那些混乱、不合理的情形,他和索科洛夫就勉勉强强修补旧理论了,那他们就错了。

幸亏,不合理就是不合理,没有向他们的固执让步,多么好呀!

话说回来,尽管新的见解产生于头脑,但还是与马尔科夫的试验有关系的。确实,如果世界上没有原子核和原子,在人的头脑里也就不会有其概念,这话是不错的,是的,是的,如果没有精密的仪器,如果没有莫斯科水电站,没有冶金炉和纯质的试剂,那么,数学在理论物理学家的头脑里也无法预测现实。

维克托感到惊异的是,他取得他的最高科学成就,偏偏是在他十分痛苦的时候,在他的脑子天天被愁闷压得非常难受的时候。怎么会出现这种情形?

为什么偏偏在一场使他惴惴不安的危险、大胆而尖锐的谈话,跟他的研究毫不相干的谈话之后,一切未解决的问题忽然在短短的瞬间找到了答案?不过,当然,这是无关紧要的巧合。

要想弄清楚这一切,是很难的……

研究工作完成了,维克托很想谈谈这项研究。在这之前他没有想过可以和什么人谈自己的想法。

他很想看到索科洛夫,想写信给契贝任。他在想象,曼德尔施塔姆、约费、朗道、塔姆、库尔恰托夫等人将怎样看待他的新方程式,局里、科里、实验室的同事们又会是什么态度,新方程会给列宁格勒的人什么样的印象。他开始考虑,用什么标题发表他的著作。他开始思索,伟大的丹麦科学家会怎样对待他的专著,费密[1]会说什么。也许,爱因斯坦会读到他的专著,会写信给他。什么人会表示反对?他的研究有助于解决什么样的问题呢?

他不想跟妻子谈他的研究。一般在寄出公务方面的信件之前,他都要先念给柳德米拉听听。每次他在大街上突然碰到什么熟人,他的第一个念头是:柳德米拉肯定会觉得吃惊。他和研究所长争论,说过一句尖锐的话,马上就会想:“我要对柳德米拉说说,我是怎样骂他的。”他不能想象看电影或者看戏没有柳德米拉坐在一起,或者小声对她说:“天啊,简直是胡诌。”使他动心、使他不安的事,他都要跟她说一说;他还在大学上学的时候就说过:“你知道吗,我觉得,我是个呆子。”

为什么他现在不说了呢?也许,他想跟她谈自己的事是因为相信她对他的事比对自己的事更关心,他的事就是她的事?现在已经不这样相信了。是她不爱他了?也许,是他不再爱她了?

不过他还是对妻子说了说自己在研究方面的情况,虽然他不愿意和她谈。

“你可知道,”他说,“我有一种很奇怪的感觉:现在我不管出什么事,哪怕朝我这心口来一下子,我这一辈子也不算白活了。要知道,正是现在我才第一次不怕死,哪怕马上死也不怕了,这不是,你看,搞出来啦!”

他把桌上写得满满的一页纸指给她看。

“我毫不夸张:这是研究核能量性质的新观点,新原理,是的,是的,这是开启许多关闭的大门的钥匙……你该知道,在小时候,不,不是小时候,不过,有这样一种感觉,就好像从漆黑死寂的水里忽然冒出一朵睡莲,哈,太美了!”

“我太高兴啦,太高兴啦,维克托。”她说着,笑了起来。

他看出,她在想自己的心思,不是在为他高兴和激动。

她也没有把他对她说的事告诉母亲,也没有告诉娜佳,看样子,她已经忘了。

晚上,维克托去找索科洛夫。

他不仅想和索科洛夫谈谈自己的研究。他很想和他叙叙自己的心情。索科洛夫会理解他的。索科洛夫不光是聪明,而且心地善良纯洁。与此同时,他又担心索科洛夫会提起他那晚发表的大胆言论。索科洛夫喜欢解释别人的所作所为,喜欢啰里啰唆地教训人。

他已经很久没上索科洛夫家里来了。大概在这段时间里,在索科洛夫家里已经聚会过三四次了。有一会儿他似乎看见了马季亚罗夫那凸出的眼睛。“这家伙胆子真大。”他想道。奇怪的是,在整个这段时间里他几乎没有想起晚间的聚会。就是现在他也不愿意想。总有一种担忧、恐惧和在劫难逃的感觉跟这种晚间的谈话联系着。是的,他们太肆无忌惮了,说丧气话,可是,你们瞧,斯大林格勒支持住了。德国人被抵挡住了,疏散的人就要回莫斯科去了。

他昨天对柳德米拉说,现在他不怕死,就是马上死也不怕。可是他还是很怕去想他那些牢骚话。马季亚罗夫简直是毫无顾忌。细想起来就更可怕了。卡里莫夫所怀疑的事是十分可怕的。万一马季亚罗夫真的是拿话引话,汇报上去,怎么办?

“是的,是的,死也不怕了,”维克托想道,“不过我这个无产者现在有东西可以丢失了,不光是锁链。”

索科洛夫正穿着家常外衣坐在桌边,在看书。

“玛利亚在哪儿?”维克托惊讶地问道,并且对自己的惊讶感到惊讶。他看到她不在家,心里若有所失,就好像他是准备和她谈理论物理的,不是和索科洛夫。

索科洛夫一面把眼镜往套子里塞,一面笑着说:

“难道玛利亚一定要时时刻刻坐在家里吗?”

维克托对索科洛夫详细讲解自己的想法,并且列出方程式,激动得气喘咳嗽,语无伦次。索科洛夫是了解他想法的第一个人,因此维克托对事情又有新的、完全不同的感觉。

“就是这些。”维克托说。他的声音哆嗦着,他感觉出索科洛夫也很激动。

他们都不作声了。维克托觉得这种沉默是好事。他低头坐着,皱着眉,忧郁地摇着头。最后他胆怯地、很快地看了看索科洛夫—他觉得索科洛夫的眼里有泪水。

在这可怕的、全世界都在打仗的时候,两个人坐在这寒碜的小房间里。在他们和生活在其他国家的人们以及生活在几百年以前的人们之间有着神奇的联系。以前的人们思想纯正,一心想完成人类应当完成的最高尚、最美好的事业。

维克托很希望索科洛夫以后也不说话。这种沉默是天大的好事……他们沉默了很久。后来索科洛夫走到维克托身边,把一只手放在他的肩上,维克托觉得,索科洛夫马上就要哭了。

索科洛夫说:

“太好了,太妙了,太美妙了。我衷心祝贺您。多么带劲儿,多么有说服力,多么漂亮啊!您的论断就是从美学角度来看也是完美无缺的。”

这一下子维克托更是激动不已,他在心里说:“噢,天啊,天啊!不过这是面包,不是美学上的事。”

“哦,您瞧,维克托·帕夫洛维奇,”索科洛夫说,“您原来那样泄气,想把一切停下来,等回到莫斯科再说,真是太不应该了。”他用维克托最讨厌的神学教员的口气说起来:“你的信心太差,耐性太差。这往往对您很有影响……”

“是啊,是啊,”维克托连忙说,“我知道。我一走进死胡同就觉得难受,就闷得受不了。”

可是索科洛夫议论起来,他这会儿说的一切,维克托都不喜欢,虽然他一下子就明白了维克托的成就的意义,并且给予极高的评价。但是维克托觉得任何评价都使人不快,都没有一点意思。

“您的研究预示着了不起的结果。”

什么“预示着”,简直是浑蛋话。不用索科洛夫说,维克托也知道他的研究“预示着”什么。结果干吗还要预示?研究本身就是结果,用不着预示什么。

“您采用的是独特的解决方法。”

没什么独特的……很普通,是面包,黑面包。

维克托特意谈起实验室日常的工作。

“顺便说说,我忘了告诉您,我收到乌拉尔的来信,咱们订购的仪器,交货时间要延期了。”

“瞧,瞧,”索科洛夫说,“等仪器送来,咱们已经在莫斯科了。这也有好的一面。要不然仪器来了咱们在喀山又不能安装,那样肯定会招来批评,说我们不积极完成选题计划。”

他啰里啰唆地谈起实验室的事,谈起完成选题计划的问题。尽管是维克托自己把话题转向研究所的日常事务,现在索科洛夫如此轻易地撇开主要的、重大的话题,他还是感到很不痛快。

此时此刻维克托分外感到自己的孤独。

难道索科洛夫不明白,现在谈的是比一般的研究所选题更大的东西?

这大概是维克托所做出的最重要的科学成果;这一成果将影响物理学家们的理论观点。索科洛夫显然从维克托的脸色看出来,不应该这样轻易地、忙不迭地转向日常事务的话题。

“很有意思,”他说,“您完全从新的角度证实了中子和重原子核的这一问题。”他用手掌做了一个动作,就像是一架雪橇从陡坡上又快又平稳地飞驰下来。“在这方面,新仪器咱们还是用得着的。”

“也许是的,”维克托说,“不过我觉得这是局部性的。”

“噢,可不能这样说,”索科洛夫说,“这种局部够大的,这是巨大的能量,您必须认识到。”

“嗯,随它去吧,”维克托说,“有意思的是,我觉得,对微观能量方面的观点变了。这会使有些人高兴,免得闭着眼睛原地踏步。”

“他们也算不上多么高兴,”索科洛夫说,“就好像有些运动员,看到别人创了纪录,而不是他们创纪录时,表现出的那种高兴。”

维克托没有冋答。索科洛夫触及了不久前在实验室里争论过的问题。

在那次争论的时候,萨沃斯季扬诺夫说,科学家的研究很像运动员的训练,科学家也要进行准备和训练,在解决科学问题时,其紧张程度不次于运动员的紧张。也是在创纪录。

维克托,特别是索科洛夫,听到萨沃斯季扬诺夫这样说,非常生气。

索科洛夫甚至做了长篇发言,把萨沃斯季扬诺夫叫做新的犬儒主义者,从他的发言可以感觉到,似乎科学像宗教一样神圣,似乎人类对神圣天国的向往就表现在科学研究中。

维克托明白,他在争论时生萨沃斯季扬诺夫的气,不只是因为他说的不对。因为他自己有时就感到像运动员那样高兴,那样激动和嫉妒。

但是他知道,紧张、嫉妒、狂热、创纪录的感觉、运动员的激动都不是实质,只是他和科学的关系的表象。他生萨沃斯季扬诺夫的气,不仅因为他说对了,也因为他说的不对。

他在年幼时心中就产生的对科学的真正感情,他对任何人,甚至对妻子都没有说过。他高兴的是,索科洛夫在同萨沃斯季扬诺夫争论中说出了对科学的正确而高尚的看法。

为什么现在索科洛夫忽然说起科学家像运动员呢?他为什么说这话?为什么偏偏在这特别的、对于维克托特别要紧的时候说?他感到慌乱、不快,便很尖锐地向索科洛夫问道:

“索科洛夫同志,既然不是您创的记录,您是不是因为咱们刚才谈的事不高兴呀?”

索科洛夫这时候正在想着,维克托想出的答案是那么简单,不用说,在他索科洛夫的脑子里已经有了,用不了多久,他一定也会说出来的。

索科洛夫说:

“是的,就是这样,就像洛伦兹那样不高兴,因为不是他自己,而是爱因斯坦完成了洛伦兹的方程式。”

他极其坦率地承认了这一点,倒是维克托后悔自己气量小了。

但是索科洛夫马上又说:

“这是开玩笑,当然是开玩笑。这跟洛伦兹毫无共同之处。我没有那样想。不过还是我说的对,不是您说的对,虽然我没有这样想。”

“当然不会,当然不会。”维克托说。不过他的恼火还没有消下去,而且他彻底明白了,索科洛夫就是这样想的。

“今天他不诚实了,”维克托想,“他真是单纯得像个孩­一样,一作假,马上就露了馅儿。”

“索科洛夫同志,”他问道,“到星期六,你们家还像往常一样有人集会吗?”

索科洛夫动了动强盗相的大鼻子,准备说点什么,但是什么也没有说。维克托用询问的目光看了看他。索科洛夫说:

“维克托·帕夫洛维奇,不瞒您说,我已经不喜欢这种茶余闲谈了。”现在他用询问的目光看了看维克托,维克托没有说话。他又说:“您要问为什么?您自己也明白……这不是说着玩儿的。简直是乱说一气。”

“您并没有乱说呀,”维克托说,“您没说什么话嘛。”

“哼,您要知道,问题就在这里呢。”

“好吧,你们都上我家里去吧,我非常欢迎。”维克托说。

真难理解!他也作假了!干吗他要说谎?他在心里也赞同索科洛夫的态度,却为什么要和他争论?他也害怕这样的聚会嘛,现在他还是不希望有这样的聚会。

“为什么上您家里?”索科洛夫问道。“我说的不是这个。我就坦率地告诉您吧:我和我的亲戚,和主要的发言人马季亚罗夫吵了一场。”

维克托很想问:“索科洛夫同志,您相信马季亚罗夫是个忠厚人吗?您能为他担保吗?”但是他却说:

“这有什么?都是自己吓唬自己,好像说一句大胆的话,国家就会垮台。您和马季亚罗夫争吵,倒是很遗憾。我很喜欢他。非常喜欢!”

“在俄罗斯最困难的时候,专挑俄罗斯人的毛病,实在不太好。”索科洛夫说。

维克托又想问:“索科洛夫同志,说正经的,您相信马季亚罗夫不会去汇报吗?”但是他没有提这个问题,只是说:

“对不起,恰好这会儿不那么困难了。斯大林格勒的局面正在好转。我们也造好了迁回的名单。您可记得两个多月以前的情况?脑子里整天想的是上乌拉尔,进原始森林,上哈萨克。”

“那就尤其不应该,”索科洛夫说,“我看不出有什么理由要说丧气活。”

“丧气话?”维克托反问道。

“就是丧气话。”

“您是怎么啦,真的,索科洛夫同志。”维克托说。

他和索科洛夫告过别,可是心里还是有一股困惑和苦闷。

他感到孤独得不得了。从早晨他就心神不定,思索着他怎样和索科洛夫见面。他感到这将是一次不平常的会面。可是,索科洛夫说的一些话,他觉得几乎都是不真诚的,是很庸俗的。

他也很不真诚。他的孤独感依然没有消失,而且更强烈了。他走出门来,走到大门口,有一个不高的女声喊了他一声。他听出这是谁的声音。

玛利亚被路灯照亮的脸,她的两颊和额头,因为有雨水,亮闪闪的。她穿着旧大衣,头上裹着毛头巾,这位科学院士和教授的夫人简直成了战争疏散时期贫困的化身。

“真像一个售货员。”他想道。

“柳德米拉怎么样?”她问道。她那黑黑的眼睛里的凝视的目光却盯着维克托的脸。

他把手一挥,说:“还是那样子。”

“明天我早一点儿上您家去。”她说。

“就这样您已经是她的守护天使了,”维克托说,“幸亏,索科洛夫能忍耐,他是孩子,没有您,一个钟头也不能过,可是您却离不了柳德米拉。”

她还在若有所思地看着他,似听见又似没听见他的话,说:

“维克托·帕夫洛维奇,今天您的脸和往常完全不同。您有什么好事儿吧?”

“为什么您认为是这样?”

“您的眼睛和往常不一样,”她忽然说,“您的研究取得了好结果,是吗?哦,您瞧,可是您还以为山穷水尽了呢。”

“您这是从哪儿知道的?”他问道,并且在心里说:“哼,娘们儿就是藏不住话,一定是柳德米拉对她说的。”他把自己的气愤掩藏在取笑的口气中,问道:

“您究竟在我的眼睛里看到了什么?”

她思索他的话,有一会儿没有作声。她没有理会他的取笑口气,只是说:

“在您的眼睛里总是有一种苦闷的神气,可是今天没有了。”

于是他忽然对她说起来:

“玛利亚,事情多么奇怪呀,我觉得,我现在完成了我一生的大事。因为科学是面包,是精神面包。而且要知道,这是在这样痛苦、这样艰难的时候完成的。多么奇怪,生活中的一切多么难以理解呀。唉,我真想……算了,没什么……”

她听着,还在看着他的眼睛,小声说:

“我要是能够把痛苦赶出你们的家门有多好呀。”

“谢谢,玛利亚。”维克托一面告别,一面说。他心里一下子宁静下来,就好像他就是来看她的,而且也说出了他想说的话。

过了一分钟,他便忘了索科洛夫夫妇,走在昏暗的大街上,寒气从一扇扇大门下往外钻,十字路口的狂风吹得大衣下摆扑扑直抖。维克托耸了耸肩,皱着眉头:难道母亲永远、永远不会知道儿子今天的事情了吗?

七

维克托召集了实验室的同事们,即物理学家马尔科夫、萨沃斯季扬诺夫、安娜·纳乌莫芙娜·魏斯帕比尔,机械师诺兹德林,电工佩列佩里律,対他们说,怀疑仪器不完善是没有根据的。正因为测量特别精确,所以不论试验条件怎样改变,得出的结果都是一样。

维克托和索科洛夫专门从事理论研究,实验室的试验工作由马尔科夫领导。他具有非凡的才能,善于解决试验中的疑难问题,准确无误地掌握复杂的新仪器的原理。

维克托很佩服马尔科夫对待他不熟悉的仪器的信心,他不必看什么说明书,几分钟工夫就能掌握其主要原理和细微零件的功能。他显然把物理仪器当做活物的身体,他认为,只要看见猫,就自然能看见猫的眼睛、尾巴、耳朵、爪子,能摸到猫的心跳,能说出哪一部分是管什么用的。

每当实验室里安装新的仪器,需要做细致精密活儿的时候,性情高傲的机械师诺兹德林就成了王牌中的大王。喜欢说笑话的浅色头发的萨沃斯季扬诺夫在说到诺兹德林时,笑着说:

“等他死的时候,把他的一双手送到脑科研究所去研究研究。”

但是诺兹德林不喜欢开玩笑,他不把从事研究的同事放在眼里,他明白,没有他的一双能干的手,实验室里的事情就干不成。

萨沃斯季扬诺夫是实验室里大家都喜欢的人。不论解决理论问题还是试验中的问题,他都有两下子。他干起任何事情,都是那样轻松,快捷,毫不吃力。

即使在最阴暗的秋天,他那发亮的小麦色头发也好像沐浴在阳光里。维克托每看到他,心里就想,他的头发放光是因为他的智慧也是明亮剔透的。索科洛夫也很器重萨沃斯季扬诺夫。

“是的,你我这样的丑角和书呆子,都比不上他,他能抵得上你、我,再加上马尔科夫。”维克托对索科洛夫说。

实验室里爱说俏皮话的人管安娜·纳乌莫芙娜叫“母鸡加公马”。她有非凡的工作能力和耐性。有一次,为了考察感光乳剂的变化,她守着显微镜坐了十八个小时。

很多研究所部门的领导人认为维克托很幸运—他的实验室工作人员配搭得很好。维克托也常常开玩笑说:“每个主任都有跟他般配的工作人员……”

“以前我们一块儿操心,一块儿发愁,”维克托说,“现在我们可以一块儿高兴了。马尔科夫教授进行试验是没有话说的。在这里面,当然也有机械小组的功劳,也有试验员们的功劳,他们进行了大量的观察,做过几百、几千次计算。”

马尔科夫很快地咳嗽了几声,说:

“维克托·帕夫洛维奇,很希望您尽量把您的观点说详细点儿。”

他放低了声音,又说道:

“我听说,科契库罗夫在邻近领域的研究使人们在实践方面产生了希望。我听说,莫斯科方面已经来询问他的研究成果了。”

马尔科夫一般都了解各种各样事件的底细。当军车载着研究所的工作人员往外疏散的时候,马尔科夫总能给车厢里打听来各种消息:线路阻塞,更换车头,一路上有多少食品供应站,等等。胡子拉碴的萨沃斯季扬诺夫故作忧虑地说:

“遇到这种事儿,我一个人要把实验室的酒精喝光了。”

安娜·纳乌莫芙娜是个大社交家,她说:

“瞧,咱们多走运,可是在基层工会的生产会议上已经有人说咱们犯了死罪啦。”

机械师抚摩着瘪下去的两颊,没有说话。

一条腿的电工佩列佩里津的脸颊慢慢红了,他没有说一句话,拐杖叭的一声掉在地上。维克托这一天非常愉快,非常高兴。上午,年轻的所长皮敏诺夫就和维克托通了电话,对他说了不少好话。

皮敏诺夫乘飞机上莫斯科去了—正在做最后的准备工作,研究所几乎所有的部门就要回莫斯科去了。

“维克托·帕夫洛维奇,”皮敏诺夫最后说,“咱们很快就要在莫斯科见面了。我很幸运,我感到自豪,就在我担任所长期间,您完成了您了不起的研究项目。”

在实验室工作人员大会上,一切情形都使维克托感到愉快。马尔科夫常常嘲笑实验室的情况,他说:“咱们的博士、教授有一个团,咱们的副博士和初级研究员有一个营,可是士兵只有诺兹德林一个!这是对理论物理学家信不过。我们像一座奇怪的金字塔。”他接着解释说:“塔顶又宽又大,往下越来越细。所以咱们摇摇晃晃,很不牢稳,应当让基础宽大,最好有一个团的诺兹德林。”

维克托做过报告之后,马尔科夫又说:

“嘿,瞧我们这个团,瞧我们的金字塔。”

一直宣扬科学像体育的萨沃斯季扬诺夫,听过维克托的报告以后,眼睛显得格外好看,露出又幸福又和善的神气。

维克托觉得,萨沃斯季扬诺夫这会儿看待他不是像运动员看待教练,而是像教徒看待圣徒了。

他想起不久前他和索科洛夫的谈话,想起索科洛夫和萨沃斯季扬诺夫的争论,在心里说:“也许,我在核能量方面能想出点儿什么,可是在人的方面一窍不通。”

快到下班的时候,安娜·纳乌莫芙娜来办公室里找到维克托,说:

“维克托·帕夫洛维奇,新来的人事处长没把我列入复员名单。我刚才看到名单了。”

“我知道,知道,”维克托说,“用不着犯愁,复员的名单有两份,您是第二批走,只不过晚几个星期。”

“可是在您这一组里偏偏就我一个人不是第一批。疏散日子我过够了,恐怕我要发疯了。每天夜里我都梦见莫斯科。再说,到莫斯科安装仪器,没有我怎么行?”

“是的,是的,的确是这样。不过您要知道,名单已经批过了,要改变,十分困难。磁力实验室的斯维琴已经为鲍·里斯·伊斯莱列维奇说过,他的情况也和您一样,可是结果还是很难改变。您最好也等些时候吧。”

他忽然上了火,叫起来:

“谁知道他妈的是怎么考虑的,他们把一些闲人塞进名单里,像您,进行安装就马上需要的人,他们却不知为什么偏忘了。”

“不是把我忘了,”安娜·纳乌莫芙娜说着,眼睛里涌出了泪水,“比忘了更糟糕……”

安娜·纳乌莫芙娜迅速地用一种奇怪而胆怯的目光回头看了看半张着的门,说:

“维克托·帕夫洛维奇,不知为什么从名单里划掉的只是一些犹太人,人事处的秘书莉玛还告诉我,在乌法,在乌克兰科学院的名单中几乎把所有的犹太人都去掉了,只留下一些科学院院士。”

维克托半张着嘴,惘然失措地看了她一会儿,后来哈哈大笑起来:

“您怎么啦,好同志,您疯啦!我们谢天谢地,不是生活在沙皇俄国。您从哪儿学来这种狭隘的怪毛病?赶快把这些乱七八糟的糊涂想法扔远点儿吧!”

八

友谊!有各种各样的友谊。

劳动中建立的友谊,革命工作中形成的友谊,长途跋涉中的友谊,共同战斗过的友谊。羁押犯人的监狱中,尽管囚友们在这儿相识与分手间隔只有两三天,可是这几天的印象却要保留很多年。安乐中的友谊,患难中的友谊。平等的友谊,不平等的友谊。

究竟什么是友谊?友谊的实质是否仅仅存在于共同的劳动和共同的厄运中?要知道,有些人本是一个党的党员,却因为观点有微小的分歧,产生的仇恨竟超过他们对党的敌人的仇恨。有时候,有些并肩战斗的人彼此憎恨,超过他们对共同敌人的仇恨。甚至有的时候,囚徒之间的宿怨更甚于他们对监狱看守的愤恨。

当然,更多的还是在同命运、同职业、有共同思想的人中间交到朋友,不过还是不能说,类似的共同性是友谊的决定因素。

不喜欢自己职业的人彼此也会有友谊,有时也会成为朋友。结成朋友的不仅是战斗英雄和劳动模范,还有战场上的逃兵和劳动中的懒汉。不过,这样或那样友谊的基础都是共同性。

两个性格相反的人能不能成为朋友?当然可以!

有时友谊是一种无私的关系。

有时友谊是为了一己私欲,有时友谊是自我牺牲,但奇怪的是,利己主义的友谊却能无私地给朋友带来好处,而自我牺牲的友谊的基础却是利己主义。

友谊是一面镜子,人在其中看到自己。有时候,你在同朋友谈心的时候,可以认识自己—等于自己同自己谈心,自己同自己交往。

友谊是平等和相似。但同时友谊又是不平等和不相似。

友谊有时是有实际目的、实际作用的,如共同劳动中的友谊,共同为了生存、为了面包而斗争的友谊。

有为了崇高理想的友谊,有意气相投、彼此谈得来的友谊,有职业各不相同,然而对现实有共同看法的人的友谊。

也许,最高层次的友谊便是实用的友谊,劳动、斗争的友谊与谈得来的友谊的结合体。

朋友往往是彼此用得着的,但朋友从友谊中得到的东西并不总是相等。朋友希望从友谊中得到的并不总是同样的东西。有的在交游中授人以经验,有的则在交游中丰富自己的经验。有的在帮助软弱和没有经验的年轻朋友时,感到了自己的成熟和能力,有的则在朋友身上看到自己的理想,希望自己也像那样成熟,有能力,有经验。就这样,有的在友谊中奉献,有的得到礼物。

有时朋友是无言的裁判,一个人借助这种裁判可以和自己对话,在自己的思想中得到欢乐,因为自己的想法在朋友的心中得到共鸣和回响,这些想法也就有了声音,能听见,能看见。

理性的、观察思辨的、哲学意味的友谊要求人的观点一致,但这种一致不是无所不包的。有时友谊出现在争论中,出现于朋友之间的差异中。

如果朋友们在各方面都相似,如果朋友们互相成为彼此的映像,那么,同朋友争论便等于同自己争论。

能够谅解你的弱点、毛病甚至过错的人,能够肯定你的正确、才能和功绩的人,才是朋友。

用爱护的态度指出你的弱点、毛病和过错的人,才是朋友。

所以,友谊的基础是相似,其表现却是分歧、矛盾、不一致。所以,有的人在交游中一心想从朋友身上得到自己所没有的东西。有的人又在交游中一心想把自己所有的东西慷慨赠与别人。

喜欢交朋友是人的天性。不善于和人交朋友的人,就和动物交朋友—和狗、马、猫、老鼠、蜘蛛。

绝对强大者不需要友谊。恐怕,只有上帝是这样的。

真正的友谊,与你的朋友身居高位,势衰落魄,还是身陷囹圄毫不相干;真正的朋友看重心灵内在的实质,把荣耀与外在的权势置之度外。

友谊的形式是各种各样的,友谊的内容也是多种多样的,但它的牢固基础只有一个—那就是相信朋友的忠诚,以及对朋友忠诚。所以,在人为自由事业效力的地方,友谊特别珍贵。在为了最高利益可以牺牲朋友的地方,在一个人被认作最高理想的敌人而众叛亲离,却相信他没有失去唯一的朋友的地方,友谊特别珍贵。

九

维克托回到家里,看到一件熟悉的大衣挂在衣架上—是卡里莫夫来了。

卡里莫夫放下报纸。维克托心想,看样子,柳德米拉不愿意陪客人说话呢。

卡里莫夫说:“我是从集体农庄上这儿来的,在那儿作报告的。”又补充说:“不过,请放心,我在农庄里吃得很饱。要知道,我们的人民是特别好客的。”

维克托心想,柳德米拉都没有问卡里莫夫要不要喝茶。

维克托只是在对卡里莫夫那宽鼻子的、布满皱纹的脸仔细端详了一阵子之后,才看出他的脸和一般的俄罗斯人以至斯拉夫人的脸型微微有些不同。有时在突然转头的短短瞬间里,这些细微的区别一齐表露出来,他的脸变成蒙古人的脸。

就像这样,有时维克托在大街上能猜出一些浅色头发、眼睛明亮、鼻子上翘的人是犹太人。有一些隐隐约约的特点可以说明这些人是犹太人出身:有时是笑容,有时是皱眉头表示惊讶的神气,眯眼睛的神气,有时是耸肩膀的姿态。

卡里莫夫说起他见到的一位中尉,那位中尉是受伤后回村里看望父母的。显然,卡里莫夫就是为了说说这事儿来到维克托家的。

“真是个好小伙子,”卡里莫夫说,“他说话非常直率。”

“说的是鞑靼语吗?”维克托问。

“当然。”卡里莫夫说。

维克托心想,如果他遇到这样的受伤的犹太中尉,是无法跟他说犹太语的;他懂得的犹太词语不超过十个,而且都是在开玩笑的时候使用的。

那名中尉一九四一年秋天在刻赤附近被俘。德国人叫他去收割埋在雪下没有收割的庄稼喂马。中尉瞅准机会,在冬日暮霭的掩护下逃跑了。俄罗斯和鞑靼居民把他掩藏起来。

“我现在完全有希望再见到妻子和女儿了,”卡里莫夫说,“原来德国人也和咱们一样,有各种各类的证件。”

“我过去上大学的时候,爬过克里木的山。”维克托说,并且想起母亲汇钱让他去旅游的事。“那位中尉看到犹太人了吗?”

柳德米拉朝门里探了探头,说:

“妈妈到现在没有回来,我很担心。”

“是呀,是呀,她这是哪儿去啦?”维克托心不在焉地说。

等柳德米拉把门掩上,他又问道:

“那位中尉有没有说起犹太人?”

“他看到把一家犹太人拉去枪毙,有一个老奶奶,两个姑娘。”

“天啊!”维克托说。

“哦,此外,他还听说在波兰有一些集中营,把犹太人赶进去,杀掉,把尸体分割开,就像屠宰场里那样。不过显然这是瞎猜想。我专门问过他有关犹太人的情况,我知道您关心这方面的事。”

“为什么偏偏只有我关心?”维克托想。“难道别人都不关心?”

卡里莫夫沉思了一会儿,又说:

“哦,我忘啦,他还对我说,德国人好像下命令要把吃奶的孩子送到警备司令部去,他们往小孩子嘴上抹一种无色的药剂,小孩子马上就死。”

“是刚生下的婴儿吗?”维克托反问道。

“我以为,这都是瞎想,就跟集中营分割尸体的说法一样,都不可信。”

维克托在房间里踱了一会儿,然后说:

“当你想到今天还在杀害婴儿的时候,一切文化建树似乎都毫无意义了。哼,歌德和巴赫教人的是什么?杀起婴儿来了!”

“是啊,可怕呀。”卡里莫夫说。

维克托看出卡里莫夫的同情心,但也看出他的高兴和兴奋:那名中尉的话增强了他同妻子相会的希望。可是维克托知道,战后他再也不能见到母亲了。

卡里莫夫要回家了,维克托舍不得和他分别,便决定送他一下。

“您要知道,”维克托忽然说,“我们苏联科学家都是一些幸福的人。正直的德国物理学家或化学家,明知自己的发明对希特勒有好处,会有什么感觉呢?您是否能想象,一个犹太物理学家,他的亲人被这样杀害,就像宰杀疯狗一样,而他却幸存,在进行创造发明,他的发明却违反他的心意,在为法西斯增强军事实力?他什么都能看见,什么都明白,可是依然不能不为自己的发明感到高兴—实在可怕!”

“是呀,是啊,”卡里莫夫说,“可是要知道,动惯了脑筋的人没办法不动脑筋呀。”

他们来到街上,卡里莫夫说:

“您送我,我不敢当。天气这样冷,您回到家里才不久,就又上外面来。”

“没关系,没关系,”维克托回答说,“我只把您送到街口。” 他看了看同伴的脸,又说:“虽然天气这么冷,我和您在大街上走一走,感到很愉快。”

“您不久就要回莫斯科了,咱们就要分别了。我很珍惜你我的知遇。”

“是的,是的,是的,说实在的,我也是这样。”维克托说。

维克托朝家里走去,竟没有注意,有人喊他。

马季亚罗夫拿黑黑的眼睛看着他。他的大衣领子竖立着。

“怎么回事儿?”他问道。“咱们的盛会停止啦?您的影子也见不到啦,索科洛夫在生我的气呢。”

“是啊,当然啦,很遗憾,”维克托说,“不过咱们在他家凭一时的激动胡乱说了不少。”

马季亚罗夫说:“谁又会注意凭一时激动说出的话呢?”

他把脸凑到维克托跟前,他那睁得大大的、神情忧愁的大眼睛显得更忧愁了,他说:

“咱们的聚会停止了,倒也好。”

维克托问:“怎么回事儿?”

马季亚罗夫一面呼哧呼哧喘着,一面说:

“应当告诉您,我觉得,卡里莫夫老头子是有任务的。懂吗?您好像跟他常常会面吧?”

“胡扯,我永远不会相信!”维克托说。

“您却没有想想,他所有的朋友,所有的朋友的朋友,已经化成灰土有十年了,跟他在一起的那一伙子连影子都没有了,只有他一个留下来,而且青云直上,当了院士。”

“这有什么?”维克托问。“我也是院士,您也是院士嘛。”

“就是这话。您想想这命运中的蹊跷吧。我想,先生,您也不是小孩子。”

十

“维克托,妈妈刚刚才回来。”柳德米拉说。

弗拉基米罗芙娜披着披肩坐在桌旁。她把一杯茶端到自己面前,却马上又推到一边,说:

“是这样,我和一个人谈了谈。那人在战争开始前见过米佳。”

她很激动,因此用分外平静、从容的语气说,她们车间实验室有一位同事,邻居家里来了一位乡亲,要在这儿住几日。那位同事在来客面前偶然提到了弗拉基米罗芙娜的姓,那人就问,在这位弗拉基米罗芙娜家里有没有人叫米佳。

下班后,弗拉基米罗芙娜去了同事家里。才知道那人是不久前才从劳改营里释放出来的。他原是报社的校对员。排字工人在排一篇社论时,把斯大林同志的姓氏排错了一个字母,他没有校对出来,结果坐了七年牢。战前又以不守纪律为由,把他从科米自治共和国的劳改营转押到远东,那里属于湖泊区劳改营系统,是对外严格保密的劳改营。在那里和他住同一棚屋的有一个人姓沙波什尼科夫。

“一听他的话,我就知道那是米佳。他说:‘他躺在床铺上,老是吹口哨:小黄雀,斑海雀,你在哪儿……’米佳在被捕前上我这儿来,我问他什么,他总是笑笑,总是在吹口哨:‘小黄雀……’今天晚上那人就要搭载货汽车上莱舍沃去了,他的家在那儿。他说,米佳有病,是坏血病,心脏也不大好。还说,米佳不相信自己能获释。米佳跟他说过我,说过谢廖沙。米佳在厨房里干活儿,这被认为是上等的工作。”

“是啊,要干这种活儿,得上两次大学呢。”维克托说。

“这事儿可不能轻易相信,万一是派的人来暗地里试探呢?”柳德米拉说。

“谁需要试探一个老婆子?”

“不过,维克托是在很重要的单位里,自有人想知道他的情况。”

“算啦,柳德米拉,这是胡思乱想。”维克托生气地说。

“他为什么得到释放,他说了吗?”娜佳问道。

“他说的一切,都让人觉得不可思议。那里有许许多多人,我觉得,那是个不可理解的世界。他好像是从另一个国度来的。他们有自己的风俗,自己的中世纪和新世纪历史,自己的谚语……

“我问他为什么获释,他很吃惊,说‘您怎么不明白,给我定案啦’。我还是不懂。原来,放出来的都是些身体太弱、快要死的人。他们劳改营内部有这样的分类:有的是做苦力的,有的是糊涂虫,有的是看守的狗腿……我问,一九三七年有许多人被判十年没有通信自由,是怎么回事儿?他说,他换过几十个劳改营,没遇到一个人是这样判的。那些人又到哪儿去了呢?他说,不知道,劳改营里反正没有。

“伐木,超期服刑,迁徙转移……他说得我直心疼。米佳也在那里面,那里有苦力、糊涂虫、狗腿……他还说到了自杀的方法:在科雷马沼地上,不吃东西,一连几天光是喝水,就这样死于水肿,他们把这叫做‘喝水’、‘开始喝水’,当然,心脏有毛病才用这种死法。”

她注意到维克托神情紧张而痛苦,女儿眉头紧皱。

她非常激动,觉得头很疼,嘴里发干,但她继续说下去:

“他说,在路上和军车里,比在劳改营里更可怕。刑事犯作威作福,剥衣服,抢吃的东西,拿政治犯的性命当赌注,输了就用刀杀人,被杀的人直到死也不知道自己的命是别人的赌注。还有更可怕的:劳改营里刑事犯处处占据着领导地位,棚屋大组长、采伐队长都是刑事犯,政治犯丝毫无权,拿他们不当人看,刑事犯还管米佳叫‘法西斯分子’。”

弗拉基米罗芙娜放大了声音,像对着人群讲话一样说:

“后来,这个人又从米佳那个劳改营,转押到瑟克特夫卡尔。在战争的第一年,中央派了一个姓卡什科津的人到米佳所在的那一类劳改营里去,布置杀害了好几万犯人。”

“哎哟,我的天呀,”柳德米拉说,“我很想明白:斯大林是不是了解这种可怕的事?”

“哎哟,我的天呀,”娜佳很气愤地学着妈妈的语调说,“难道你不明白吗?他们是斯大林下命令杀的呀。”

“娜佳,”维克托说,“住嘴!”

维克托就像有些人一样,感觉内心的虚弱被旁人识破了似的,忽然发起火来,朝娜佳吼道:

“你别忘了,斯大林是最高统帅,正率领军队同法西斯作战,你的祖母直到生命的最后一天都指望着斯大林,我们生活、呼吸,都因为有斯大林和红军……你还是先学学揩鼻涕,再去评论斯大林,是斯大林在斯大林格勒挡住了法西斯。”

“斯大林住在莫斯科,在斯大林格勒挡住法西斯的,你也知道是谁,”娜佳说,“真不知道你是怎么一回事儿,你从索科洛夫家回来,也说过我说的这话……”

他对娜佳的气更大了,他觉得这股气一辈子都消不了。

“我从索科洛夫家回来,根本没说过类似的话,你别胡扯。”他说。

柳德米拉说:“就在苏联的孩子们纷纷为国战死的时候,干吗要提这些可怕的事?”

但是娜佳也马上说出她所理解到的爸爸心中的隐秘和弱点。

“哼,当然啦,你什么也没有说,”她说,“现在嘛,现在你在研究中取得了那样的成就,在斯大林格勒也把德国人挡住了……”

“你怎么能,”维克托说,“你怎么能怀疑爸爸虚伪!柳德米拉,你听见没有?”

他希望得到妻子的支持,但柳德米拉无动于衷。

“你有什么大惊小怪的?”她说。“你说的话她听了不少。这都是你和你那个卡里莫夫说的,和那个讨人嫌的马季亚罗夫说的。玛利亚也常对我说起你们谈的话。而且你自己在家里也说了不少。唉,还是快点儿回到莫斯科去吧。”

“够啦,”维克托说,“我早就知道你要对我说什么样的痛快话了。”

娜佳没有再说话。她的脸变得像老太婆一样委顿、难看,她扭过头,背着爸爸,但是他还是看到了她的眼神,她用那样痛恨的眼神看他,他吃了一惊。

气氛显得非常窒闷,空气中包含了太多沉重的东西,让人喘不过气来。

几乎在每一个家庭,一年年暗地生长着的东西,可能作怪,可能平息,但因为相爱和信任而被压抑着的东西,现在冲了出来,浮到表面上,漫开去,充塞在生活中,似乎在父亲、母亲和女儿之间仅仅存在着不了解、怀疑、气恼和责难了。

难道他们共同经历的命运,产生的只有分歧和隔阂吗?

“外婆!”娜佳唤道。

维克托和柳德米拉同时看了看弗拉基米罗芙娜。她坐在那里,用手紧紧按着额头,好像头疼得不得了。

她是那样软弱无力,似乎她和她的痛苦谁也不稀罕,只能妨碍别人,使人生气,使家里人不和,她这个一辈子刚强、坚毅的人,这会儿坐在那里,那样孤单,那样软弱—这一切流露着一种说不出的可怜意味。

娜佳忽然跪下,把额头贴到外婆的腿上,说:

“外婆,亲爱的外婆……”

维克托走到墙边,打开收音机,硬纸板做的喇叭嘶哑地响起来,发出呻吟和喘息。好像广播的是秋夜的雨雪天气。在战场的前沿阵地,在战火烧毁的村庄,在阵亡士兵的坟头,在科雷马和沃尔库塔,在野战机场,在冷雨和初雪打湿了的卫生营帆布篷顶,今夜将是一片雨急风狂、雪花漫舞的景象。

维克托看了看妻子愁眉不展的脸,便走到岳母跟前,抓起她的手,吻起手来。

然后,他俯下身去,抚摩娜佳的头。

似乎在这几分钟里一切都没有变化,房里依然是这几个人,他们依然十分痛苦,他们的命运依然如故。只有他们自己知道,他们的痛苦不堪的心在这几分钟里充满了多么神奇的温暖……

忽然一个很响的声音冲进房间:

“一天来,我军在斯大林格勒地区、图阿普谢西北和纳尔奇克地区同敌人继续进行战斗。其他战线没有任何变化。”

十 一

德军中尉别捷尔·巴赫因为肩部被子弹打伤,进了军医院。他的伤势不重,送他上救护车的同伴们祝贺他走运。

巴赫怀着一种幸福感,同时疼得哼哼着,由卫生员搀扶着前去洗澡。

一接触到热水,真是说不出的快活。

“比在战壕里舒服吧?”卫生员问道。他希望对伤员说点儿快活的,就又说:“等您出院的时候,大概那儿全都收拾好了。”他朝那个方向指了指,那边不停地传来响成一片的轰隆声。

“您来这儿不久吧?”巴赫问。

卫生员用树皮擦子给中尉擦了几下脊背之后,说:

“您为什么断定我来这儿不久?”

“这儿已经没有人认为战事会很快结束。这儿的人都认为战事很快结束不了。”

卫生员看了看澡盆里光着身子的中尉。巴赫想起来,军医院工作人员有责任汇报伤员的思想,而他的话流露出他对德军威力的不信任。于是他一字一顿地又说了一遍:“是啊,卫生员同志,这事怎样结束,目前还没有人知道呢。”他为什么把这句危险的话重说一遍?这是只有生活在极权制帝国的人才能明白的。他重说一遍,是因为他很生气,不该在说过第一遍之后就害怕了。他重说一遍,也带有防备的目的—想骗骗他所设想的这个告密者,表示自己有口无心。

过了一会儿,他为了消除有关自己的反对立场的不好印象,又说:

“我们在这里集中这样多的兵力,可能自从战争开始以来还不曾有过。请相信我的话,卫生员同志。”

后来他厌烦了这种又复杂又伤脑筋的把戏,一心一意玩起儿童游戏:把浸透了肥皂水的海绵攥在手里,使劲攥,那肥皂水一会儿射到澡盆沿上,一会儿射到巴赫自己的脸上。

“喷火器就是这样喷射。”他对卫生员说。

他痩了多少啊!他看着自己光光的两臂和胸膛,想起两天以前吻他的那个俄罗斯年轻女子。他何曾想到,在斯大林格勒会跟一个俄罗斯女子有这样一段艳史?当然,这还很难叫做艳史。只不过是偶然的战地艳遇。那是一种很不平常、难以想象的环境,他们在地下室里相遇,他在一片瓦砾中向她走去,一阵阵爆炸的火光映照在他身上。那在小说中也是一种十分精彩的场面。昨天他应该去找她的。她大概以为他已经牺牲了。等他康复后,一定还要去找她。真想知道,是谁填补了他的位子呢?自然界是不兴留空缺的呀……

洗过澡以后,很快把他带到X光室,医师让他站到X光透视机前。

“中尉,那边不好过吧?”

“俄国人比我们更不好过。”巴赫回答说。他想给医生一点儿好印象,希望得到很好的诊断,动起手术也会轻快些,少受点罪。

外科医生走了进来。两位医生看了看巴赫的内脏,可以看清已经在胸腔里钙化了的过去的各种病灶。

外科医生抓住巴赫的胳膊,把他转来转去,一会儿拉着他贴到荧光屏上,一会儿把他拉远一点儿。他注意的是弹片伤,至于伤的是一个受过高等教育的年轻人,那是无关紧要的情况。

两位医生说起话来,夹杂着拉丁语和开玩笑的德国粗话,于是巴赫明白了,他的伤情不严重,胳膊还能保得住。

“请你们准备给中尉做手术,”外科医生说,“我还要在这儿看一个复杂的病例,是颅部重伤。”

卫生员脱去巴赫的伤员服,一名外科护士叫他坐到凳子上。

“见鬼,”巴赫苦笑着说,并且因为自己光着身子感到不好意思,“小姐,应该先把凳子弄暖和一点儿,再让斯大林格勒大战参加者的光屁股坐到上面。”

她连笑也没笑,回答他说:

“我们没有这样的任务。”

她说过这话,便把手术用具从玻璃橱里一样一样往外拿,巴赫一看到就觉得害怕。可是摘除弹片的手术进行得又快又轻松。巴赫甚至生起医生的气,认为医生是在向伤员散布瞧不起小手术的思想。

那位外科护士问巴赫,要不要把他送到病房里去。

“我自己能走。”他说。

“您在我们这儿不会待很久的。”她用安慰的语调说。

“太好啦,”他说,“我已经开始无聊了。”

她笑了。

这位护士显然是按照报纸通讯来想象伤员的。作家和记者们在通讯里写的伤员,总是偷偷地从军医院跑出去,跑回自己的营里和连里;他们一定要向敌人开枪开炮,不这样就不能过日子。

也许,记者们在军医院里也碰见过这样的人,不过当巴赫躺在铺了干净被单的床上,吃了一碗米饭,又抽了一支烟(在病房里严禁抽烟),和邻床的人聊起来的时候,他可是感到快活得不得了。

病房里有四名伤员:三名是前方下来的军官,第四名是文官,凹进去的胸脯,凸出来的肚子,是从后方来办公事,在古姆拉克地区遭遇车祸。在他仰面躺着,把两手放在肚子上的时候,就好像有人和这位大叔开玩笑,往他的被窝里塞了一个足球。

显然,他就是因为这种伤得了个外号“守门员”。

守门员在所有的人当中,是唯一表示遗憾的,因为受伤不能报效国家。他常常用慷慨激昂的语调谈起祖国、军队、天职,说他因为在斯大林格勒受伤感到光荣。

为民族流过血的前方军官们,常常嘲笑他的爱国主义。其中有一位侦察连长克拉普,因为屁股受伤,天天趴在床上,苍白的脸,厚嘴唇,棕色的凸眼睛,他对守门员说:

“看样子,您这样的守门员不仅能把球挡回去,也会把球踢进去。”

这位侦察连长是个色情狂,他主要谈的是两性关系。守门员想讽刺一下对方,问道:

“为什么您没有晒黑呀?您大概是在办公室工作吧?”

克拉普可没在办公室工作过。

“我是夜里的鸟儿,”他说,“我打食儿都是在夜里。我跟娘们儿睡觉是在白天,和您不一样。”

在病房里常常骂官僚,他们一到晚上就坐小汽车从柏林上别墅去;骂那些军需官,他们得勋章比作战的人都便当;谈作战的官兵家庭的贫困,不少人家里的房子都被炸毁了;骂后方的浪荡子勾引军人的妻子;骂前方的小货摊光卖香水和刮脸刀片。

睡在巴赫旁边的是耶内中尉。巴赫原以为他是贵族出身,谁知他却是个农民,是德国国家社会主义党政变中涌现的人物之一。他担任一个团的副参谋长,在夜晚空袭中被弹片炸伤。

守门员被送去做手术的时候,躺在角落里的憨厚的上尉弗列谢尔说:

“我从一九三九年就打仗,可是我从来没有夸耀过自己的爱国主义。给我吃,给我喝,给我穿,我就打仗。没有什么道理好说。”

巴赫说:“不对,不能那样说。打过仗的人嘲笑守门员的虚伪,这里面就有自己的道理。”

“是这样啊!”耶内说。“请问,这究竟是什么样的道理?”

他那很不和善的眼神,巴赫早就习惯了。他感觉到,耶内恨那些希特勒上台以前的知识分子。巴赫耳闻目睹许多言论,说旧知识分子倾慕美国财阀,暗地倾向犹太旧教和犹太观念,在绘画和文学方面喜欢犹太风格。巴赫感到非常气愤。现在,当他愿意向这些新势力的粗暴低头的时候,为什么还拿阴沉的、像狼那样的怀疑目光看他呢?难道他不是和他们一样,也挨过虱子咬,挨过冻吗?他们竟不把他这个前沿阵地的军官当成德国人!巴赫闭上眼睛,转身朝着墙。

“你为什么问得这样恶毒?”他在心里生气地说。

耶内会带着鄙夷和优越的笑容说:

“您好像没有明白吧?”

他会被这话激怒,说:“我跟你讲过,我是没有明白。”然后补充说:“我要想想。”

耶内当然笑了。

“你怀疑我阳一套阴一套?”他高声喊道。

“就是,就是阳一套阴一套!”耶内的声音显得很快活。

“精神阳痿?”

这时候弗雷塞尔会哈哈大笑起来。克拉普用胳膊肘支起身子,非常不客气地看看巴赫。

“你们这群退化的败类,”巴赫会用打雷一样的声音喊道,“耶内,您已经是介乎猴子和人之间了……咱们说真的。”

他恨得打了一个寒颤,闭紧了本来就阖上的眼睑,在心里继续说:

“你们只要就任何小问题写出一个小册子,马上就仇恨起为德国科学奠定基础和砌墙的人。你们只要写进一本薄薄的小说,马上就瞧不起有光荣传统的德国文学。你们是否以为科学和艺术有点儿像官场,老一辈的官员妨碍你们晋升?你们和你们的书越来越没有出路了,科赫、能斯特、普朗克和凯勒曼已经在挡你们的路了……科学和艺术不是官场,是无垠的天空下的帕耳纳斯山,永远是宽阔的,整个人类历史长河中所有的天才在那儿都有足够的地方可以生存,只是容不得你们和你们的恶果。不是没有地方,只是那儿不是你们待的。可是你们还在忙着清除场地。你们那可怜的、吹不起来的汽球不会因此就升高一点儿。你们赶走爱因斯坦,你们永远不能填补他的位置。是的,是的,爱因斯坦,他当然是犹太人,不过,对不起,他确实是天才。世界上还没有那样大的权力,能够帮助你们接替他的位置。你们想想吧,值不值得花那样大的力量来消灭那些人,那些人的位置是永远无法填补的。如果你们不够格,不能走希特勒开辟的道路,那也只能怪你们自己,不能恼恨够格的人。在文化方面动用警察,煽动仇恨,这种办法是毫无用处的!你们瞧,希特勒和戈培尔对这一点认识得多么深刻?他们以身作则在教导我们。他们在对待德国科学、绘画、文学方面表现得多么喜爱,多有耐心,多有策略。就要学他们的样子,走团结的道路,不能给我们德国的共同事业造成分裂!”

巴赫不出声地说完这番话,睁开眼睛。旁边的人都还躺在被窝里。

弗雷塞尔说:“伙计们,往这儿看!”

他像变戏法一样从枕头底下抽出一瓶意大利白兰地。耶内的喉咙里发出一种奇怪的声音。只有真正的酒徒,而且只有农村里的真正酒徒看到酒瓶才会露出这样的神情。

“他这人不坏嘛,从各方面看,他不坏。”巴赫想道。并且为自己没有说出的歇斯底里的话感到不好意思起来。

就在这时候,弗雷塞尔用一条腿蹦着,往几个床头小柜上的玻璃杯里斟酒。

“您真是野兽。”侦察连长笑着说。

“这可是能征惯战的中尉。”耶内说。

弗雷塞尔说:“有个医官发现了我的酒瓶,问:‘您这报纸里包的是什么?’我回答说:‘这是我母亲的来信,我一直带着不离身。’”

他举起杯,说:“来吧,中尉弗雷塞尔向你们致敬!”

大家一饮而尽。

耶内马上就想再喝一杯,就说:

“噢,应该还要留一杯给守门员呀。”

“守门员去他妈的吧,你说是吗,中尉?”克拉普问道。

“让他为祖国效劳吧,咱们喝咱们的。”弗雷塞尔说。

“每个人都希望活着嘛。”

“我现在来劲儿了,”侦察连长说,“这会儿顶好再来一个不胖不瘦的娘们儿。”

大家都轻松、快活起来。

“好,再来一杯。”耶内举起杯来。

大家又喝干了。

“咱们能住到一个病房里,太好啦。”

“我一看,马上就断定:‘这才是真正的伙伴,都是上过火线的。’”

“可是说实话,我怀疑过巴赫,”耶内说,“我心想:‘哼,这是党里的人。’”

“不,我不是党里的。”

他们掀开被子,躺了下来。大家都觉得热起来。谈起前方的事。

弗雷塞尔原来在右翼,在奥卡托夫镇一带作战。

“谁他妈的知道,”他说,“苏联人简直不会打进攻仗。可是到十一月初,我们还停在那儿。我们八月里喝了多少伏特加呀,天天举杯祝贺:‘但愿战后不要失去联系,要成立攻克斯大林格勒老战士协会。’”

“他们进攻的本领不算差,”在工厂区作过战的侦察连长说,“他们不会固守。他们只要把我们从楼房里打出来,就马上要么睡觉,要么吃起东西。俄国军官就爱喝酒。”

“他们都是一些野蛮人,”弗雷塞尔说着,挤了挤眼睛,“我们在这些斯大林格勒野蛮人身上耗费的钢铁,比在整个欧洲耗费的还要多。”

“不光是耗费钢铁,”巴赫说,“在我们团里有一些人,常常无缘无故地哭,像公鸡一样扯开嗓子又哭又喊。”

“如果到冬天事情还不能解决,”耶内说,“那就要真的陷入僵局了。像那样打来打去,毫无意思。”

侦察连长小声说:

“我告诉你们,咱们正准备在工厂区发动攻势,调集的兵力超过以前任何时候。近几天就要打响了。到十一月二十日,咱们都可以跟萨拉托夫的姑娘们睡觉了。”

在挂了窗帘的窗户外面响起低沉的隆隆炮声和夜袭的飞机的轰轰声。

“苏联飞机出动了,”巴赫说,“他们的飞机在这时候进行轰炸。有些人管它们叫‘锯神经的锯子’。”

“在我们团部里管它们叫‘值班士官’。”耶内说。

“别作声!”侦察连长竖起一个手指头。“你们听,这是重型炮!”

“可是我们却在轻伤员病房里喝酒呢。”弗雷塞尔说。

于是他们在这一天里第三次快活起来。

他们谈起苏联的女人。每个人都有可谈的。巴赫一向不喜欢谈这些事。

但是在军医院的这天晚上,巴赫却说起住在被炸毁的楼房的地下室里的季娜,说得很带劲儿,大家都在笑。

卫生员走进来,打量了一下一张张笑脸,就动手收拾守门员床上的被单。

“这个柏林来的祖国的卫士出院了吧?受伤是装的吧?”弗雷塞尔问。

“卫生员,你怎么不说话?”耶内说。“我们都是男子汉嘛,他要是有什么情况,就对我们说说。”

“他死了,”卫生员说,“心肌麻痹。”

“你们瞧,满嘴爱国主义,落了个这样的结果。”耶内说。

巴赫说:“这样说死人,可不大好。他并不是说假话,他用不着在咱们面前说假话。就是说,他是真心实意的。伙计们,这样不好。”

“哦,”耶内说,“怪不得我觉得这位中尉是奉党的命令上我们这儿来的。我一下子就明白了,他可是有新思想的。”

十 二

夜里,巴赫睡不着,他太舒服了。想起掩蔽所,想起一起作战的伙伴,想起莱纳德的到来,他甚至还和他一起透过掩蔽所开着的门眺望落日,一起抽烟,喝暖水瓶里的咖啡—他感到非常奇怪。

昨天,他要上救护车的时候,他还用没有受伤的胳膊抱着莱纳德,他们对视一眼,笑了起来。他何曾想到,他会在斯大林格勒的土室里同这个纳粹分子共饮,在炮火照耀的瓦砾场上去找自己的俄罗斯情人。

他的变化异常奇怪。多年来他一直痛恨希特勒。当他听到无耻的白发苍苍的教授说,法拉第、达尔文、爱迪生是一伙儿偷窃德国科学的盗贼,而希特勒才是古今各国最伟大的学者的时候,他怀着幸灾乐祸的心情想:“哼,算啦,这都是腐朽不堪的东西,这一切统统要完蛋。”还有那些小说,用惊人的虚伪笔调描写没有缺点的人,描写高尚的工人和农民的幸福,描写英明的党的教育工作,同样引起他的反感。哼,杂志上发表的那些诗多么不像样子。这一点使他特别生气。他在中学里就写诗了。

可是现在在斯大林格勒,他想入党了。当他是小孩子的时候,他怕父亲在争论中把他说服,常常用手捂住耳朵,喊:“我不愿意听,不听,就是不听……”可是现在他听了!世界绕着轴心转了个身。

他还像过去一样非常厌恶平庸的戏剧和电影。也许,人们在几年、十几年中读不到好的诗歌,又有什么办法呢?不过就是在今天也有可能写出真理!因为德国精神就是主要的真理,是世界的理想。要知道,文艺复兴时期的大师们即便是根据王公和主教的指示,写出的作品也能表现最伟大、可贵的精神。

侦察连长克拉普还在睡着,他一面参加夜战,一面大声叫喊着,他的喊声大概在外面都能听得见:“手榴弹!手榴弹!”他想爬,就很别扭地翻了个身,疼得叫了起来,后来又睡着了。打起鼾来。

甚至过去使他胆战心惊的排犹行为,这会儿从新的角度重新出现在他的脑际。啊,如果他有权,他马上就下令制止对犹太人的大批屠杀。不过,虽然他有不少犹太朋友,他还是要实实在在地说:德国人有德国人的性格与精神,而犹太人有犹太人的性格与精神。

马克思主义破产了!对于一个父母当年都是社会民主党人的人来说,是很难想到这一点的。

马克思就像一个物理学家,将物质构造理论的基础建立在互相排斥的力量上,却忽视了万有引力。他为阶级互相排斥的力量下了定义,他是人类有史以来将这种力量研究得最透彻的。但是他也和一些有伟大发现的人一样,片面地认为,他所证实的阶级斗争力量是唯一能决定社会发展和历史进程的。他没有看到超阶级的民族团结的强大力量,他这种社会物理学忽视了民族万有引力的规律,因此是荒谬的。

国家不是后果,国家是前因!

有一种神秘而奇特的规律决定着民族国家的诞生。国家是一种有机的结合体,只有国家能够代表千百万人特别珍视的、长远的东西,能够代表德国人的性格、德国的源流、德国人的意志和牺牲精神。

巴赫闭着眼睛躺了好一会儿。为了能睡着,他想象出一群羊:一头白羊,一头黑羊;又是一头白羊,一头黑羊;又是一头白羊,一头黑羊……

吃过早饭以后,巴赫给母亲写信。他皱着眉头,叹着气,知道母亲看到他写的内容不会高兴。但是,他应该把近来的感觉对母亲说说。他在回去度假的时候,什么也没有对她说。但她看出他的焦躁,看出他不愿意听她没完没了地回忆父亲的事—如今依然是这样。

她会想,他背叛父亲的信仰了。可是他没有。他恰恰是不肯背叛。

伤员们经过早晨的治疗,都疲乏了,所以都静静地躺着。夜里抬来一名重伤员,放在原来守门员的床上。他还在昏迷状态中,无法弄清他是哪个部队的。

怎么能向母亲说清楚,今天新德国的人比小时候的朋友和他更亲近?

卫生员走进来,问道:“谁是巴赫中尉?”

“是我。”巴赫说着,拿手盖住开了头的信。

“中尉先生,有一个苏联女人打听您。”

“打听我?”巴赫吃惊地问。他马上想到,这是他在斯大林格勒的情人季娜来了。她怎么会知道他在哪儿呢?可是他马上明白了,这是连里的救护车司机告诉她的。他很高兴,很感动:因为这要摸黑走出来,要搭顺路汽车,还要步行七八公里。于是他好像看到了她那大大的眼睛、苍白的脸,她那细细的脖子、头上的灰头巾。

病房里哈哈大笑起来。

“瞧咱们的巴赫中尉!”耶内说。“这是他在当地居民中干出的成绩。”

弗雷塞尔两只手摆动了几下,就好像要抖掉手指头上的水,说:

“卫生员,叫她到这儿来吧。中尉的床够宽的。我们就让他们成亲。”

侦察连长克拉普说:“女人和狗一样,男人到哪儿,她到哪儿。”

忽然巴赫生起气来。她是怎么想的?她怎么能上军医院里来?因为严禁军官和苏联女人有什么关系。万一在军医院里工作的有他家的人或者他的朋友福斯特家的人呢?只有那么一点不怎么样的关系,即使是一个德国女子,也未必敢来找他。

那个昏迷中的重伤员好像正在厌恶地冷笑呢。

“请告诉那个女人,我不能出去见她。”他阴沉地说。为了不参与他们的说笑,他马上拿起铅笔,念起已经写好的几行:

“……奇怪的是,多年来我认为国家压制着我。可是现在我明白了,正是国家代表着我的心意。我不希望命运一帆风顺。如果有必要的话,我可以同老朋友断绝关系。我知道,我要投奔的一些人永远不会真正拿我当自己人。但为了最主要的目标,我可以牺牲我的一切……”

病房里依然在高声说笑。

“安静点儿,别打搅他。他在给未婚妻写信呢。”耶内说。

巴赫笑起来。有时压抑着的笑很像抽泣,于是他心里想,他现在可以笑,也可以哭。

十 三

有些将军和军官们,不是经常能见到第六步兵集团军司令弗里德里希·保卢斯的,都认为这位上将的思想和心情没有发生什么变化。举止的风度、发布命令的口气、听取细小意见和重大报告时的笑容,都证明这位上将依然驾驭着战争的局面。

只有和司令特别接近的一些人,如他的副官阿丹斯上校、集团军参谋长施密特将军,才了解保卢斯在斯大林格勒这段时间里的变化有多大。

他依然显得很风趣,很宽厚,雍容自若,依然亲切地关怀下属的生活情形,依然牢牢操纵着指挥各团各师作战的大权,依然决定着将领们的任免升降,批准奖赏,依然在抽自己习惯了的纸烟……但是他的内心深处却在一天一天地发生变化,而且正准备彻底变化。

他渐渐失去了那种驾驭局面和时机的感觉。不久前,他见到司令部侦察科的报告,还只是用平静的目光匆匆扫一扫:苏军有什么打算,他们的后备兵力的调动有什么目的,都没有什么大不了的。

现在阿丹斯发现:每天早上他把一叠报告和文件放到司令的桌子上的时候,司令首先拿起的是有关苏军夜间行动的侦察报告。

有一次,阿丹斯改变了叠放文件的顺序,把侦察科的报告放在最上面。保卢斯打开公文夹,看了看放在上面的报告。他那长长的眉毛扬了起来,接着就把公文夹合上了。

阿丹斯上校明白了,他的做法很不聪明。保卢斯那种一闪即逝的、似乎很悲哀的目光使他大吃一惊。过了几天,保卢斯看过了按往常顺序叠放的报告和文件之后,笑了笑,对自己的副官说:

“革新者先生,您显然是一个细心人。”

在这个寂静的秋日黄昏,施密特将军怀着几分得意的心情前去向保卢斯报告。

施密特顺着小镇宽阔的街道朝司令住的房子走去,快活地呼吸着寒冷的空气,空气冲洗着夜里抽烟抽得发燥的喉咙。他抬头望了望,只见天空被草原落日的模糊色彩染得斑斑斓斓。他的心里非常宁静,他想到绘画,想到午饭后的打嗝已经停止,不那么难受了。

他走在寂静而空旷的黄昏的大街上,在他的头脑里,在沉甸甸的大沿帽底下,装着全部设想,那是在最残酷的激战时必须说出来的,而在斯大林格勒战役时期这样的激战早晚会到来的。当司令请他坐下,准备好听他报告的时候,他就这样说了:

“当然,在我们作战的历史上,为了进攻确实动员过大量的军事装备。不过,在这样小的作战地区,在陆地和空中火力密集到这样的程度,我个人还从来不曾遇到过。”

保卢斯佝偻着身子坐着听参谋长报告,似乎失去了大将军的风度,他的头匆忙地随着施密特那指着图表线条和地图方块的手指头转悠。这次进攻是保卢斯筹划的。保卢斯已经定出进攻的兵力数据。但是现在,听着跟他共事多年的这位才华出众的参谋长的意见,他觉得,在未来作战计划的细节方面,他的一些想法是不现实的。

施密特似乎不是在陈述已经变为作战计划的保卢斯的设想,而是把自己的意见硬加给保卢斯,他与保卢斯的意见相反,准备使用步兵、坦克、工兵营发动进攻。

“是啊,是啊,密度太大了,”保卢斯说,“如果和咱们左翼的空虚相比,那就太明显了。”

“没办法呀,”施密特说,“东方的土地太大了,咱们徳国的兵不够用。”

“不光是我担心这一点,冯·魏克斯也对我说:‘咱们打人不是用拳头,而是张开手指,分散在无边无际的东方土地上。’担心这一点的不光是魏克斯。不光是……”他没有说完。

一切情况在意料之中,又在意料之外。

近几个星期的战斗中出现了偶然的情况和一些小小的失利,似乎从中就可以看出战局出现了新的变化,令人悲观绝望的真相。

侦察队不断地送来有关苏军在西北面集结的情报,空军无力阻止。魏克斯无法向保卢斯集团军的两翼补充后备兵力。他在罗马尼亚军队中设置德军广播电台,想迷惑苏军。但罗马尼亚人并没有因此就成为德国人。

一开始对非洲的远征似乎所向无敌。在敦刻尔克,在挪威和希腊,痛击英军,结果仍没有占领英伦三岛。在东方取得了巨大胜利,长驱几千公里直抵伏尔加河边,结果并没有彻底击溃苏军。总以为大局已定,即使尚未彻底胜利,那这也只是偶然的不顺利,微不足道……

他与伏尔加河之间这几百米距离,这毁了一半的工厂,这一座座烧焦的楼房的空壳,与夏季攻势以来攻占的广大地区相比,又算得了什么?……但是在埃及的沃土地带与隆美尔将军之间,也还有几千公里的沙漠。为了在已占领的法国取得完全胜利,还差敦刻尔克的几公里,几小时……不论哪里总是差几公里,不能彻底打垮敌人。不论哪里两翼总是空虚,所向无敌的军队背后总是留下广大的地区,后备兵力总是不足。

今年夏天是何等气势!那些日子里他的感觉,恐怕一生中只能有一次。他感到自己的脸上已经有印度的气息。如果排山倒海的狂涛巨澜能够感受的话,那么这狂涛的感受,就是他的感受。

这些日子他曾闪过一种想法,认为德国人的耳朵已经习惯了弗里德里希这个名字。当然,这是一种开玩笑的、不认真的想法,但他毕竟有这种想法。可就在这些日子里,在他脚下—或者说牙齿中间—出现了几粒不怀好意的很硬的砂石。在司令部里依然是一片胜利和幸福的紧张气氛。他在接收各部指挥官的书面报告,听取口头报告、无线电报告、电话报告。似乎这不是繁重的作战工作,而是德国胜利的象征性表现……保卢斯拿起话筒。“上将大人……”他从声音听出这是谁在说话。战时用惯了的语调跟电话中的嗡嗡声很不协调。

师长维德列尔报告说,苏军在他的地段上发动了进攻,他们的一支步兵,大约有一个加强营,冲到了西边,占领了斯大林格勒火车站。这桩看似微不足道的小事,让他开始感到焦虑的刺痛。

施密特念完了一道作战命令的草稿,微微舒展肩膀,抬起下巴,表示他还没有失去下属应有的恭敬,虽然他和司令之间的私人关系很好。

突然,上将放低了声音,既不用军人的语调,更不用大将军口气,说了几句很奇怪的、使施密特大惑不解的话:

“我相信能取胜。但是您知道吗,咱们在这个城市打仗没有必要,毫无意义。”

“真有点儿意外,进攻斯大林格勒部队的司令会说出这话。”施密特说。

“您以为意外吗?斯大林格勒已经不再是交通中心和重工业中心。既然这样,咱们在这儿又能干什么呢?高加索方面军的东北翼可以由阿斯特拉罕至卡拉奇这条战线掩护。斯大林格勒在这方面不起什么作用。施密特,我相信能取胜,我们能够拿下拖拉机工厂。但是这并不能掩护我们的侧翼。冯·魏克斯认为苏军一定会反攻。虚张声势吓不住他们。”

“随着战局的变化,战事的意义也会变化,不过元首一向是不达目的,决不罢休啊。”施密特说。

保卢斯认为,问题就在于最光辉的胜利都没有带来什么结果,因为都没有坚决、顽强地进行到底;同时他又认为,一位统帅的真正价值,就在于能够拒绝执行已经失去意义的任务。

但是,他看着施密特那聪颖、锐利的眼神,说:

“我们不能把自己的意志强加于伟大的元首。”

他拿过桌子上发起进攻的命令,签了字。

“考虑到特别保密,这个文件只有一式四份。”施密特说。

十 四

达林斯基从草原的集团军司令部来到一支部队,这支部队在斯大林格勒战线的东南翼,在里海地区缺水的沙漠地带。

现在达林斯基觉得那紧靠着河水和湖水的草原有点儿像仙土福地了,那儿有芦苇,有马嘶,有些地方还有树。

在沙漠化的荒原上住着几千人,他们习惯了潮湿的空气、清晨的露水、沙沙作响的干草。沙子击打着皮肤,往耳朵里直钻,在小米饭和面包里咯咯直响,食盐里有沙子,枪栓里有沙子,手表里有沙子,战士的梦里也有沙子……人的身体、鼻孔、喉咙、小腿肚子在这儿都很难受。人生活在这儿,就好像一辆大车离开了平坦的车辙,在烂泥里咯吱咯吱地慢慢挣扎。

整个一天,达林斯基都在炮兵阵地上转,和人谈话,做记录,制图,查看大炮、弹苀¯仓库。快到傍晚时候,他筋疲力尽,头嗡嗡响,腿也疼,在松软的沙地上走路实在太不习惯了。

达林斯基早就发现,在撤退的日子里将军们往往特别关心下属的生活需要;司令员和军委委员们都很大方地表现他们的自我批评精神、怀疑精神和谦逊。

在仓皇撤退的时期,当敌人节节取胜,最高统帅部愤怒追查失职官兵的时候,部队里就会出现许多无所不知的聪明人。

但是在这里,在沙漠里,人们却懒洋洋的,对一切都很淡漠。司令部里的军官和队列军官们似乎认定,在这世界上没有什么事需要他们关心,明天,后天,一年之后,沙子反正还是沙子。

炮兵团参谋长鲍瓦中校请达林斯基到他那儿去过夜。这位中校虽然姓的是英雄故事中鲍瓦王子的姓,身子却佝偻着,秃顶,一只耳朵听力很差。他有一次奉命到方面军炮兵司令部去,他的非凡的记忆力使大家吃了一惊。似乎在他那安在又窄又佝偻的肩膀上的秃脑袋里,装的全是数字、炮兵连和营的番号、驻地名称、指挥员的姓名、高地的标志。

鲍瓦住的是一座木板小屋,墙上抹了黄泥和牛粪,地上铺了破碎的油毡。这座小屋和散布在沙漠上的其他军官的住处没有任何不同。

“哈,您好!”鲍瓦说着,使劲握了握达林斯基的手。“很好吧,嗯?”他朝着墙指了指。“这儿就是住在抹了牛粪的狗窝里过冬。”

“是啊,这房子不坏!”达林斯基说着,就看到文静的鲍瓦再也不文静了,感到很惊讶。

他请达林斯基坐在原来装美国罐头的一个空箱子上,给他倒了一玻璃杯酒,玻璃杯黏糊糊的,边上还沾满了牙粉,又把放在一张泡软的报纸上的一个青色的渍蕃茄推了过来。

“请吧,中校同志,这就是我的葡萄酒和水果了!”他说。

达林斯基像一切不会喝酒的人一样,小心翼翼地喝了一小口,就把杯子放到离自己远些的地上,向鲍瓦问起军队中的事。但是鲍瓦偏要谈别的,不谈正事。

“唉,中校同志,”他说,“我满脑子都是军事,从来不想别的,我们在乌克兰的时候,那儿的娘们儿才漂亮呢,在库班,就更不用说了……简直是心甘情愿送上门,只要你挤挤眼睛就行!可是我这个傻瓜待在那儿动也不动,后来醒悟过来,已经在沙漠里了!”

达林斯基起初有点生气,因为鲍瓦不愿谈每公里战线的平均密度问题和在沙漠地区迫击炮优于大炮的问题,可他终于还是对新的话题有了兴趣。

“当然啦,”他说,“乌克兰的女子确实漂亮得不得了。在一九四一年,司令部驻扎在基辅的时候,我遇到一个乌克兰女子,是一位检察院工作人员的妻子,简直美极啦!”

他欠起身来,举起一只手,手指头碰了碰矮矮的顶棚,又说:

“至于库班,我的看法也和您一样。库班在这方面也是数一数二,十个中就有九个是美人儿。”

达林斯基的话鼓起了鲍瓦的劲头儿。他骂了一声娘,用哭腔叫了起来:

“可是,您瞧瞧加尔梅克娘们儿那模样儿吧!”

“可不能这么说!”达林斯基打断他的话,并且头头是道地说起黑皮肤、高颧骨、带有野蒿气味和草原烟味的女子的美。他想起了草原的集团军司令部里的阿拉,就总结了一下自己的长篇议论:“总而言之,您说的不对,到处有漂亮娘们儿。沙漠里没有水,这是对的,可是漂亮娘们儿还是有的。”

但是鲍瓦却没有接他的话。这时达林斯基发现,鲍瓦睡着了。他这才想到,主人已经喝醉了。

鲍瓦睡觉打鼾,鼾声很像垂危病人的呻吟。他的头从床上耷拉下去。达林斯基怀着俄罗斯男子对待醉汉的那种特别的耐心和善意,把鲍瓦的头放到枕头上,又在他腿下垫了一张报纸,擦了擦他嘴上的唾沫,这才四下里看了看,考虑自己在哪儿睡。

达林斯基把鲍瓦的大衣铺在地上,又把自己的大衣扔在鲍瓦的大衣上,拿自己鼓鼓囊囊的军用包当枕头,这军用包在出差期间又是他的办公桌,又是给养仓库和盥洗用具箱。

他走到外面,呼吸了几口夜晚的冷空气,看到黑黑的亚洲天空的星光,高兴得啊呀了一声,解了一下小便,依然在望着星星,心里说:“宇宙好大呀!”便回来睡觉。

他躺在主人的大衣上,把自己的大衣盖在身上,却没有合上眼睛,反而把眼睛睁得大大的—有一种凄凉感,使他大吃一惊。

四周黑沉沉,空荡荡,好不凄凉!瞧,他就睡在地上,看到的是渍蕃茄的残渣,还有一个硬纸箱,里面大概有一条带有老大的黑色商标的方格短毛巾、皱巴巴的衬领、手枪的空皮套、压瘪的肥皂盒。

秋天他曾在上波格罗姆内的一所小房子住过,现在他觉得那儿是很阔绰的了。过一年之后,今天这间可怜的小屋也许又成了豪华的了,将来有一天住到地窖里,既没有刮脸刀,又没有提箱,没有破裹脚布的时候,又会想起这小屋的。

在炮兵司令部工作的这几个月,他的心里发生了很大的变化。如饥似渴地要求工作的心愿已经满足了。他已经不因为自己在工作而感到幸福。因为天天能吃饱的人并不感觉自己是幸福的。

达林斯基工作能力很强,领导很器重他。起初一段时期这使他非常高兴,因为他难得有被人看重、被人珍视的时候。多年来他习惯了相反的情形。

达林斯基没有想过,为什么他心中产生的优越感,没有使他对同事产生宽容的态度—宽容是真正强者的特点。不过,显然他不是强者。

他常常发火,叫嚷,骂人,然后很难过地看着被他骂的人,不过他从来不请求被他骂的人原谅。有些人恼恨他,但不认为他是坏人。在斯大林格勒方面军司令部,对他的看法也许比过去在西南方面军司令部对诺维科夫的看法还要好些。据说,在一些大人物向莫斯科的一些更大的人物汇报时,常常整页整页地使用他的报告文稿。原来,在困难时期他的才智和工作都是有用的和有益的。战前五年他妻子离开了他,因为她认为他是人民的敌人,认为他巧妙地向她隐瞒了自己的本质,毫无志气,是个两面派。他常常因为出身不好找不到工作—父亲和母亲的出身都不太好。起初他听说,不让他干的工作,却让极其愚蠢或者无知的人干了,他非常生气。后来他觉得,的确不能让他做重大的工作。他从劳改营里出来以后,索性觉得自己各方面都不行了。

可是,在可怕的战争时期,情况就不是这样了。

他把大衣朝肩膀上拉了拉,这样一来两条腿马上感觉到从门缝儿钻进来的冷风,他心想,就在他的知识和本事用得上的时候,他却躺在这鸡窝里的地上,听着骆驼的刺耳的叫声,希求的不是疗养地和别墅,而是一条干净衬裤,希望能弄到一块肥皂头,洗个澡。

他引以自豪的是,他地位的提高和物质方面没有任何联系。但同时这也使他很气愤。他在自信和自负的同时,在生活要求上却总是表现得很胆怯。他觉得,优越的生活条件永远不是他应该得到的。他从小就习惯了这种不敢希求什么的感觉,习惯了已经成为习惯的总是没有钱的状况,习惯了经常感觉自己穿着寒碜的旧衣服。

就是在今天,在他一帆风顺的时候,他依然有这样的感觉。

他一想到,他要是上军委食堂去,服务员会说:“中校同志,您应该在一般部队食堂用餐。”他就觉得害怕。有时在什么地方参加会议,有的将军会开玩笑,眨眨眼睛,说:“怎么样,中校同志,就在军委食堂喝碗加油甜菜汤吧?”他也觉得不自在。他看到,不仅是将军们,就连报社的记者们都像当家的那样笃定地在他们不应该得到享用的地方又吃又喝,要汽油,要服装,要香烟,这总是使他感到十分惊讶。

过去的日子一直是这样过的,他的父亲年年找不到工作,长年赡养一家人的是做速记员的母亲。

到半夜时候,鲍瓦的鼾声停止了,达林斯基听到他在床上一点声息也没有,担心起来。

突然,鲍瓦问道:“中校同志,您没有睡吗?”

“没有,睡不着。”达林斯基回答说。

“真对不起,没有把您安排好,我喝醉了,”鲍瓦说,“现在我头脑清醒了,就像一点酒也没有喝。这会儿我躺在这儿,在想:咱们怎么来到这样的鬼地方啦?是谁让咱们来到这鬼地方的?”

“还能是谁,德国佬呗。”达林斯基回答。

“您到床上来睡,我睡地上。”鲍瓦说。

“不用,我在这儿挺好。”

“有点儿不像话,主人睡在床上,客人睡在地上,按照高加索风俗,可不应该这样。”

“没关系,没关系,咱们又不是高加索人。”

“差不多算高加索人啦,就在高加索山脚下嘛。您说,是德国佬让我们这样的,可是,您要知道,不光是德国佬,还有我们自己人。”

看样子,鲍瓦欠起身来了:他的床咯吱响了几声。

“嗯,是啊……”他说。

“是啊,是啊。”达林斯基在地上说。

鲍瓦一下子把谈话推向特别的异常的轨道,两个人都沉默下来,都在考虑,该不该和不知底细的人谈这样的事。看样子,他们考虑之后,得出的结论是:不应该同不知道底细的人谈这类的话。

鲍瓦抽起烟来。

擦着火柴的时候,达林斯基看到了他的脸。觉得这脸很不舒展,显得阴郁、陌生。

达林斯基也抽起烟来。

火光闪亮的瞬间,鲍瓦也看到了用胳膊肘支着身子的达林斯基的脸,他的脸看起来淡漠、冷酷、陌生。

在这之后,不知怎的,偏偏谈起了不应该谈的话。

“是的。”鲍瓦说。不过这一次没有拉长声音,而是又短又干脆。“是官僚作风和官僚让我们来到这儿的。”

“官僚作风是很坏的事,”达林斯基说,“我的司机说:战前在农村里的官僚作风十分严重,没有酒在农庄里别想弄到证明。”

“您别笑,这没有什么好笑的,”鲍瓦说,“您要知道,官僚作风可不是开玩笑的,官僚作风在和平时期把人折腾够了。在前方打仗的时候,官僚作风害起人来更够呛。在空军部队里有这样一件事:一架歼击机被击中,飞行员从着了火的飞机里跳出来,人好好儿的,裤子却烧坏了。可是,就是不发给他裤子!真荒唐,总务科副科长不肯发,说是还不到穿破的时候!飞行员三天没穿裤子,一直弄到集团军司令那儿才解决。”

“这事儿荒唐是荒唐,”达林斯基说,“不过只是有的浑蛋不发裤子,不会因此就从布列斯特退到里海地区的沙漠上来。”

鲍瓦酸溜溜地哼哧了一声,说:

“难道我说是因为不发裤子?我再对你说一件事:有一个步兵排被包围了,没有东西吃。空军得到命令,要用降落伞向他们空投食品。可是军需处不发给食品,说是需要领用人在发货单上签字,如果从飞机上把这些东西给他们投下去,他们在下面怎么能签字呢?军需官就是不发。后来靠上面命令,才勉强发了。”

达林斯基笑了笑。

“有一件可笑的事,不过也是小事。只顾形式,不顾实际。在前方,官僚作风一表现出来就特别可怕。您可知道有一道‘不准后退一步’的命令?有一次,敌人对准几百人轰击,只要把人带到对面山坡上,人也安全,战略上也不吃亏,装备也能保住。可是有‘不准后退一步’的命令,所以就让待在炮火之下,人也完了,装备也完了。”

“就是,就是,一点不错,”鲍瓦说,“在一九四一年,从莫斯科派来两位上校,来我们集团军里检查‘不准后退一步’这道命令的执行情况。他们没有汽车,我们在三昼夜之间从戈梅利往后跑了两百公里。我让两位上校坐到我们的吨半汽车里,免得他们落到德国人手里。他们在汽车里直打哆嗦,还一个劲儿地要求我:‘有关执行不准后退一步命令的情况,给我们提供一些材料。’他们要汇报,有什么办法呢?”

达林斯基往胸中吸了一大口气,就好像要潜入水深处,看样子,他确实潜入了深处,说:

“有一名红军战士,是一个机枪手,保卫一处高地,一个人对七十个德国人,把敌人打退了,他也牺牲了,全军都向他表示敬意,可是他那害肺痨的妻子却被人从房子里赶出来,区苏维埃主席骂她:不要脸的女人,滚出去!这种官僚作风真可怕。有时候,让一个人填二十四张履历表,可是到末了他自己在大会上承认:‘同志们,我不是你们的人。’您要知道,这也是官僚制度问题。要是一个人说:是的,是的,国家是工人农民的,可是我的爸爸妈妈都是贵族,是不劳动的分子,你们把我撵走,那就好了。这也是官僚制度问题。”

“可是我不认为这是官僚制度问题,”鲍瓦反驳说,“事实如此,国家是工农的,是工农在管理国家。这有什么不好的?这很好嘛。资产阶级国家不会让穷人来领导。”

达林斯基愣了,看样子,对方完全想到别的方面去了。

鲍瓦擦着了火柴,却没有点烟,而是用火柴朝着达林斯基照了照。

达林斯基眯起眼睛,感觉就像在战场上落到了敌人的探照灯灯光下。

可是鲍瓦说:

“我是地地道道的工人家庭出身,父亲是工人,祖父也是工人。我的出身历史都是清白的。可是我在战前也不受重用。”

“您究竟为什么不受重用?”达林斯基问。

“如果在工农的国家里,用慎重的态度对待贵族,我不认为是官僚作风。可是为什么我这样一个工人在战前要受压抑呢?不是往果品蔬菜公司的仓库搬运土豆,就是扫街,我都不在乎。可是我用阶级观点发表了一点意见,批评了一下领导,说他们的日子过得太阔气了,我一下子就倒了霉。依我看,如果一个工人在自己的国家里都要吃苦受难的话,官僚作风的主要根源就在这里面。”

达林斯基马上感觉出来,对方这番话触及了非常重大的问题,并且因为他还不习惯谈这些激动人心、使心里火辣辣的事情,也不习惯听别人谈这种话,所以心里感到说不出的畅快。毫无顾虑、毫无恐惧地发表意见,争论那些令人激动不安的问题,实在是一种幸福。正因为这种议论特别使人激动难安,他从来没有同任何人谈过这些事。

在这里,在这小屋的地上,同这个朴实的军人在一起夜谈,这个人醉后又醒来。他感觉到自己周围都是从西乌克兰撤到这沙漠上的人,一切都是另一种境况。于是出现了一种很自然、很朴素的期待—然而又是很难理解、很难想象的情形:人与人真诚地谈了起来!

“您的话又对又不对,”达林斯基说,“穷光蛋进不了资产阶级的参议院,这样说是对的,但是穷光蛋如果成了百万富翁,就能进参议院了。福特就是工人出身。我们不让资产阶级和地主占据领导岗位,这是对的。但是如果给老老实实工作的人也打上犯罪印记,仅仅因为他的父亲或祖父是富农或者神甫,那就完全是另一回事儿了。这不算阶级观点。您以为我在劳改营里受折腾的时候没有遇到普梯洛夫工厂的工人和顿涅茨矿工吗?要多少有多少!我们的官僚制度很可怕,因为这不是国家身上的赘疣,赘疣是可以割掉的。这种官僚制度所以特别可怕,因为官僚制度就是国家。在战争时期,没有任何人愿意为了人事处长去牺牲。在申请书上批一个‘不同意’或者把士兵的遗孀赶出办公室,任何一个无能的奴才都能办得到。可是要把德国佬赶出去,就需要刚强的、真正的好汉了。”

“这话很对。”鲍瓦说。

“我不抱怨。我很感激,非常感激。非常感谢!我是幸福的!不过另一点就很不好:为了我能幸福,能为国家贡献自己的力量,还要再来那样可怕的时期,那就糟了。那我再也不要这种幸福。去他妈的!”

达林斯基觉得,他还是没有深挖到主要的、他们所谈的问题的真正实质,一针见血地阐明现实问题的东西,不过他这一下子想了、说了平时不敢想、不敢说的事情,这使他感到非常高兴。他对自己的交谈者说:

“您要知道,这一生今后不论出现什么情况,我都不懊悔今天夜里同您的长谈。”

十 五

莫斯托夫斯科伊在隔离室里过了三个多星期。给他吃得很好,党卫军的医生给他检查过两次,还开了处方,给他注射葡萄糖。

刚被关起来的时候,他一直等待着传讯,一个劲儿地埋怨自己:真不该同伊康尼科夫交谈;一定是那个糊涂老头子,在搜查之前塞给他那几张可能有问题的纸,把他害了。

一天天过去,却没有传讯他。他思索着同犯人们进行政治谈话的题目,考虑可以吸收什么人参加工作。夜里睡不着的时候,他为传单打腹稿,挑选营里人交谈用的一些字眼儿,好让各种不同民族的人更容昜打交道。

他想起了在奸细告密的情况下可以防止全面失败的一些秘密活动的传统办法。

他很想向叶尔绍夫和奥西波夫问问建立组织的最初几个步骤;他相信能够使奥西波夫消除对叶尔绍夫的偏见。

他觉得,又仇恨布尔什维克又盼望红军胜利的切尔涅佐夫实在可怜。他想到面临的审讯,心里几乎是平静的。

夜里,他的心脏病发作。他躺着,把头抵在墙上,难受得要命,只有在监狱里的快要死的人才会这样难受。他疼得昏迷了一阵子。等他苏醒过来,不怎么疼了,胸膛、脸上、手上都出了一层汗。头脑里也出现了一种似是而非的、虚假的清醒状态。

他想到他和意大利神甫议论世界性罪恶的那番话,联想起小时候有一天忽然下起雨来,他跑进妈妈做针线活儿的房间时那种幸福感;又联想起当年去叶尼塞流放地看他的妻子,想起她那哭湿了的幸福的眼睛;又联想起面色苍白的捷尔任斯基,他在一次党的会议上向捷尔任斯基问起社会革命党一个可爱的小伙子的下落。捷尔任斯基回答说:“枪毙了。”他想起基里洛夫少校那苦闷的眼睛……想起雪橇拖着的朋友的尸体,用被单盖着。朋友在列宁格勒被围的日子里,没有得到他的帮助。

他那像小孩子一样的乱蓬蓬的头充满了幻想,他那老大的秃头顶贴在粗糙的集中营板墙上。

过了一阵子,遥远的事渐渐远去,越来越淡,渐渐失去色彩。他似乎慢慢沉入凉爽的水里。他睡着了,为的是在晨曦中重新听到笛声,迎接新的一天。

下午,把他带到浴室里。他很不痛快地吸着气,打量着自己的胳膊和瘪瘪的胸膛。

“是啊,老了。”他想道。

等到带他来洗澡的士兵在手里捏着纸烟走出门去,一个正在用拖把擦洗水泥地的窄肩膀麻脸囚犯对莫斯托夫斯科伊说:

“叶尔绍夫要我向您报告一个消息:在斯大林格勒地区我军把德国佬所有的坦克打退啦。他要我告诉您,一切情况正常。他要您写传单,下一次洗澡的时候交给我。”

莫斯托夫斯科伊正想说,他没有铅笔和纸,但这时候一名看守走了进来。莫斯托夫斯科伊在穿衣服的时候,摸到口袋里有一个纸包。里面有十块糖、一块用破布包着的奶油、一张白纸和一个铅笔头儿。

莫斯托夫斯科伊感到非常高兴。他希望有的东西全有了!可以不是在毫无意义地担心血管硬化、胃病、心绞痛的状态中结束生命了。

他把糖块和铅笔头儿紧紧按在胸口。

夜里,有一名党卫军的士官把他押出来,押着他顺着街道往前走。一阵阵冷风吹在他的脸上。他回头朝一座座沉睡的棚屋看了看,在心里说:“没什么,没什么,你们的莫斯托夫斯科伊同志神经不那么脆弱,同志们,你们好好儿地睡吧。”

他们走进集中营管理处大门。这里已经闻不到集中营里那种氨水气味,可以闻到冰冷的烟草气息。莫斯托夫斯科伊发现地上有一根老大的烟头儿,他真想捡起来。

他们上了二楼,又上了三楼,那士官叫莫斯托夫斯科伊在擦脚垫上把脚擦干净,士官自己也把鞋底擦了老半天。莫斯托夫斯科伊爬楼已经累得上气不接下气,这会儿尽可能平息一下气喘。

他们顺着铺在走廊里的长条地毯走去。

一盏盏半透明的郁金香形小灯,灯罩里透出柔和、宁静的灯光。他们经过一扇打磨得锃亮的门,门上挂着一个不大的木牌“警备长办公室”,来到另一扇同样富丽堂皇的门前站住,门上的牌子是“党卫军少校利斯办公室”。

莫斯托夫斯科伊常常听到这个名字,这是秘密警察总头子希姆莱在集中营管理处的代表。莫斯托夫斯科伊觉得好笑的是,古济将军曾经很生气,因为奥西波夫是利斯亲自审讯的,而审讯他古济的却只是利斯的一名助手。他认为这是对队列指挥人员的轻视。

奥西波夫说过,利斯在审讯他的时候不用翻译,因为他原来是苏联里加市的德国人,精通俄语。

从里面走出一名年轻军官,对押解的士官说了几句话,便叫莫斯托夫斯科伊进办公室去,门依然开着。

办公室里没有人。铺着地毯,花瓶里插着鲜花,墙上还有一幅画:树林的边缘,红瓦顶的农舍。

莫斯托夫斯科伊心想,他来到屠宰场场主的办公室里了—旁边是要死的牲畜在哼哧,内脏在冒热气,屠宰手的身上溅满了血,可是场主这里却这样宁静,地毯这样干净,只有桌上的黑色电话机说明屠宰场和这间办公室是联系着的。

敌人!多么简单明了的字眼儿!又想起切尔涅佐夫的话—人的命运在“狂飙突进运动”时代是多么可怜。不过他是戴着小山羊皮白手套的。于是莫斯托夫斯科伊看了看自己的手掌和手指头。

办公室里面的门开了。通向走廊的门也马上吱扭响了一下,看样子,是值班军官看到利斯来到办公室,把门掩上了。莫斯托夫斯科伊皱紧眉头站着,等待着。

“您好。”这个灰军服袖子上带着党卫军标志的小个子低声说。

利斯的脸上没有任何狰狞的地方,因此莫斯托夫斯科伊觉得看到这张脸特别可怕。这是一张鹰钩鼻子的脸,黑灰色眼睛神情专注,宽大的额头,苍白瘦削的两腮,显露出一副恪尽职守、清心寡欲的神气。

利斯等到莫斯托夫斯科伊咳嗽过了,说:

“我想和您谈谈。”

“可是我不想和您谈。”莫斯托夫斯科伊说过这话,侧眼朝远处的角落里看了看,估计利斯手下的刽子手们会从那边过来打他的耳光。

“我完全能理解您,”利斯说,“请坐吧。”

他让莫斯托夫斯科伊坐在安乐椅上,自己也紧挨着坐下来。他说的俄语是一种没有特色、没有生活气息的冰冷语言,是科普小册子里使用的语言。

“您身体不大好吧?”

莫斯托夫斯科伊耸了耸肩膀,什么也没有说。

“是的,是的,我知道。我派医生给您看了,他对我说过。我深更半夜里打扰您了。不过我实在想和您谈谈。”

“可不是嘛。”莫斯托夫斯科伊在心里说。他回答道:

“我是来受审的。咱们没有什么好谈的。”

“为什么?”利斯问道。“您看着我穿着制服。但我不是生来就穿这制服的。领袖和党分派穿制服,于是就穿上了,成了党的士兵。我一直是党内的理论家,我对哲学和历史问题很感兴趣,不过我是党员罢了。难道你们内务部的每个工作人员都赞赏卢比扬卡监狱吗?”

莫斯托夫斯科伊注视着利斯的脸。他心里想,这张苍白的、高额头的脸应该画在人类学图表的最低栏内,其进化程度相当于原始的尼安德特人。

“如果党中央派您去加强肃反委员会的工作,您能拒绝吗?您只能放下黑格尔的书,去工作。所以我们也放下了黑格尔的书。”

莫斯托夫斯科伊侧眼看了看说话的人,觉得这张肮脏的嘴说出黑格尔的名字,实在很奇怪,简直是亵渎……在拥挤的电车里,一个可怕的、老练的贼走到他跟前,要和他搭话。他听着,一心一意注视着贼的手,只要看到划包的刀片一闪,就照着眼睛打过去。此刻他就是这样的心情。

可是利斯抬起两手,朝手上看了看,说:

“我们的手和你们的手一样,它们喜欢干大事,不怕弄脏。”

莫斯托夫斯科伊眉头紧锁。利斯说出的话连同他的手势,令他觉得难以忍受。

利斯很带劲儿地说起来,说得很快,就好像从前就和莫斯托夫斯科伊谈过,现在能够把那次中断的话说完,十分高兴。

“只要坐二十个钟头的飞机,您就可以到苏联的马加丹市,可以坐在自己的办公室里的椅子上了。您在我们这儿,可以和在自己家里一样,不过您不走运。你们的宣传机构竟和财阀的宣传机构一块儿丑化我们党的司法,我很痛心。”

他摇了摇头。接着又很快地说起令人吃惊、意外,又可怕又荒唐的话:

“在我们面对面互相看着的时候,我们看到的不仅是仇恨的面孔,我们是在照镜子。这是我们时代的悲剧。难道您没有在我们身上看到你们自己,看到你们的意志?难道在你们来说,世界不就是你们的意志,难道谁能够使你们动摇,使你们停止?”

利斯的脸凑近了莫斯托夫斯科伊的脸。

“您明白我的意思吗?我的俄语说得不太好,但我希望您能明白我的意思。您以为,您是在痛恨我们,但这是表象,实际上您是通过恨我们恨你们自己。很可怕,是吗?您明白吗?”

莫斯托夫斯科伊打定主意不说话,利斯也不一定要他说话。

莫斯托夫斯科伊有一会儿觉得,这个盯着他的眼睛的人并不想欺骗他,而是实心实意聚精会神地在说语,挑选着字眼儿。似乎他是在倾诉烦恼,请人帮他弄清使他苦恼的问题。

莫斯托夫斯科伊感到非常难受。似乎有一根针在扎他的心。

“您明白吗,明白吗?”利斯很快地说。他已经看不见莫斯托夫斯科伊了,他心里十分慌乱。“我们打你们的军队,但我们也是在打自己。我们的坦克冲击的不光是你们的国境,也是我们的国境,我们的坦克履带辗压的是德国的国家社会主义。真可怕,简直是梦里自杀。我们有可能失败得很惨。明白吗?如果我们胜利了,又会怎样?我们胜利了,我们就没有了你们,我们就要单独对抗痛恨我们的另外一个世界。”

这个人的话很容易驳倒。他的眼睛离莫斯托夫斯科伊更近了一些。但是有一种什么东西比这个老练的党卫军间谍的话更坏、更危险。这个东西有时在莫斯托夫斯科伊的心里和脑子里活动,并且吱咯吱咯地响,有时畏畏缩缩,有时躁动得很厉害。这是一种很坏的、见不得人的怀疑情绪,莫斯托夫斯科伊不是在异己者的话里发现的,而是在自己心里发现的。

就好比一个人怕生病,怕恶性肿瘤,却又不找医生,尽量不理会自己的病疼,不和家里人谈自己的病。现在有人对他说:“您瞧,您常常这样疼,一般是在上午,一般是在……是的,是的……”

“您明白我的意思吗,老师?”利斯问道。“有一个德国人,您是非常了解他的判断能力的,他说,拿破仑一生的悲剧就在于他表现了英国精神,而英国正是他的死敌。”

“噢呀,这比打耳光都厉害,”莫斯托夫斯科伊心里想道,并且在心里说,“他这是说的斯宾格勒[2]。”

利斯抽起烟来,并且把烟盒递给莫斯托夫斯科伊。

莫斯托夫斯科伊生硬地说:

“不想抽。”

他想到,世界上所有的宪兵,不论四十年前审讯过他的那些宪兵,还是现在大谈黑格尔和斯宾格勒的这一个,都使用同样的笨拙办法:请被审讯的人抽烟。他一想到这一点,就比较坦然了。是的,说实话,这都是因为神经紊乱,由于意外:本来以为会挨耳光的,谁知却听到一番荒唐的、令人厌恶的话。不过,有些沙皇时代的宪兵也研究政治问题,其中也有一些真正有文化的人,有一个人还研究过《资本论》。可是不知道研究马克思的宪兵是否有这样的情况:突然在内心深处出现这样的念头—也许马克思是对的呢?在这种情况下那个宪兵有什么样的感觉呢?不过,不论怎样,宪兵不会成为革命者。他踩灭自己的怀疑,仍然做宪兵……我也是在踩灭自己的怀疑。不过我是仍然要做革命者。

利斯却没有注意莫斯托夫斯科伊已经拒绝抽烟,还在说:

“是的,是的,请吧,不错,这烟很好。”

他把烟盒合上,并且很难过地说:

“我的话为什么使您这样惊讶?您以为我不会说出这样的话吗?难道在你们的卢比扬卡监狱里工作的,就没有受过高等教育的人吗?就没有人能够和巴甫洛夫院士,和奥尔登堡院士谈谈吗?不过他们是有目的的。我可没有什么隐秘的目的。我可以向您保证。你们思考的问题,我也在思考。”

他笑了笑,补充说:

“一个盖世太保的保证,这可不是开玩笑的。”

莫斯托夫斯科伊在心里一遍又一遍地说:“不说话,就是不说话,不和他说什么话,不反驳。”

利斯继续说下去,他又好像把莫斯托夫斯科伊忘记了。

“两个极端!当然是这样!假如不完全是这样的话,今天就不会有这样可怕的战争。我们是你们的死敌,是的,是的。但我们的胜利也就是你们的胜利。明白吗?如果你们胜利了,那我们又会完蛋,又会依靠你们的胜利活下去。这好像是奇谈怪论:我们打输了,也是打赢了,我们将换一种形式继续发展下去,实质还是一样。”

为什么这个权势显赫的利斯不去看缴获的电影,不喝酒,不给希姆莱写报告,不看养花的书,不看女儿的来信,不去玩弄刚刚从军列上挑选来的年轻姑娘,不去服用增强新陈代谢的药品,到他那宽敞的卧室里睡觉,却在深更半夜里把这个浑身散发着集中营臭气的苏联老布尔什维克找了来?

他打算干什么?他为什么掩盖自己的目的,他想探问的是什么?

现在莫斯托夫斯科伊不怕用刑审讯了。可怕的倒是有一种想法:万一这个德国人说的不是假话,而是实在话呢?一个人有时就是想说说话嘛。

有一种使他非常厌恶的想法:他们两个都是病人,两个人害的都是一种病,但是一个人憋不住,说出来了,和别人分一分痛苦,另外一个人却不说,瞒着,可是听着,听别人说。

利斯好像终于要回答莫斯托夫斯科伊没有说出口的问题似的,把桌上放着的公文夹打了开来,带着厌恶的神气用两个手指头把一叠肮脏的纸抽了出来。莫斯托夫斯科伊马上认出来,这就是伊康尼科夫塞给他的那几张纸。

利斯显然以为,莫斯托夫斯科伊一看到伊康尼科夫给他的这几张纸,会惊慌失措的……

但是莫斯托夫斯科伊没有惊慌失措。他几乎是很高兴地看着伊康尼科夫写满了字的这几张纸:一切都明朗了,就像警察审讯时常有的情况一样,丝毫不客气,直截了当。

利斯把伊康尼科夫写的字推到桌子边上,后来又拉到自己跟前。他忽然用德语说起来:

“您看,这是从您那儿搜出来的。我看了几个字,就知道这种乱七八糟的玩意儿不是您写的,虽然我不认识您的笔迹。”

莫斯托夫斯科伊没有说话。利斯用一个指头在纸上敲着,请他说话—是很客气地、善意地、一再地请他说话。可是莫斯托夫斯科伊没有说话。

“我说错了吗?”利斯惊讶地问道。“不会的!我不会错。你们和我们都十分厌恶这上面写的东西。你们和我们是站在一起的,另一边才是这些乱七八糟的玩意儿!”他指了指伊康尼科夫那几张纸。

“好吧,好吧,”莫斯托夫斯科伊急急忙忙地、很不耐烦地说,“咱们就把事情谈谈吧。这几张纸吗?是的,是的,是从我那儿拿来的。您想知道这是谁交给我的吗?您别问这事儿吧。也许,是我写的。也许,是您叫您的走狗暗暗塞到我的褥垫底下的。明白吗?”

有一会儿,似乎利斯就要接受挑战,就要发作起来,叫喊:“我有办法叫您说出来!”莫斯托夫斯科伊非常希望这样,这样事情就简单了,就好办了。“敌人”是多么简单明了的字眼儿。可是利斯却说:

“这几张破烂的纸算什么?谁写的,还不是一样?我知道:不是您,也不是我。我是多么难过呀。难过得不得了!如果不是战争,如果我们的集中营里关的不是战俘,这些集中营里会是一些什么人呢?如果不是战争的话,我们的集中营里关的会是党的敌人、人民的敌人。您熟悉的一些人现在就在你们的劳改营里。如果在和平时期,我们的帝国保安局也会把你们的犯人关进德国的监狱,我们决不会释放的。你们的犯人,也就是我们的犯人。”

他笑了笑,又说:

“我们在集中营里关过的那些德国共产党人,你们在一九三七年也关进了劳改营。叶若夫关他们,帝国首领希姆莱也关他们……老师,您要相信黑格尔的话。”

他朝莫斯托夫斯科伊挤了挤眼睛,又继续说下去:

“我想,外语的用处在你们的集中营里不会比在我们的集中营里小些。今天我们对犹太人的仇恨使你们害怕。也许,明天你们就要采取我们的经验。到后天,我们就会显得很宽松了。我走过了很长的道路,是一位伟人领我走的。你们也有一位伟人领导着,你们也走过很长、很艰难的路。您相信布哈林是奸细吗?只有伟人能够领导你们走这条路。我也认识勒姆,我相信他。可是就应该枪毙他。我真不懂,你们实行恐怖政策,杀了几百万人,全世界竟只有我们德国人能理解:应该这样!完全正确!您一定要理解,就像我理解你们一样。这次战争想必使你们害怕了。拿破仑本来不应该打英国。”

这一新的说法使莫斯托夫斯科伊十分吃惊。他甚至眯起眼睛,不知是因为眼睛突然受到刺激,还是他想回避这种使人不快的说法。要知道,他的怀疑也许并不是软弱无力的表现,并不是可鄙的怀疑动摇的表现,不是疲惫和无信心的表现。也许,他这种时强时弱的怀疑正是他的最真诚、最纯洁之处。可是他却拼命压制、排斥、痛恨这种怀疑。也许,这里面就有革命真理的种子?这里面就有自由的炸药!

要想摆脱利斯,摆脱他那又滑又黏的手指头,只要不再痛恨切尔涅佐夫,不再瞧不起傻子伊康尼科夫就行了!不过,不行,还不止这样!还要否定终生的信仰,要批判自己一直在维护、在主张的东西。

可是,不行,还不止这样!不只是批判,而是要全心全意,用全部革命激情痛恨集中营、卢比扬卡监狱,痛恨沾满鲜血的叶若夫、亚戈达、贝利亚!不过还不够,还要痛恨斯大林和他的专制!

可是,不行,还不止这样!还要批判列宁!直到深谷的边缘!

但那将是利斯的胜利,不是在战场上进行的战争的胜利,而是在这种充满了蛇毒的、不用枪炮的战争中的胜利,这会儿这个秘密警察头目就是在同他进行这种战争。

他似乎马上就要发疯了。可是他忽然轻松愉快地舒了一口气。一时间令他为之恐惧、迷乱的想法化为灰尘,显得可笑又可鄙。他迷惑了几秒钟。可是,他对伟大事业的正确性能够真的怀疑吗,哪怕一秒钟,哪怕一秒的十分之一?利斯看了看他,咬了咬嘴巴,继续说:

“一些人看到我们就害怕,难道看到你们就喜欢,就对你们抱着希望吗?请您相信吧,看到我们害怕的人,看到你们也害怕。”

现在莫斯托夫斯科伊什么也不怕了。现在他知道了自己的怀疑的代价。他们不像他原来猜想的那样,是要他到泥淖里去,而是要他进可怕的深谷!

利斯拿起伊康尼科夫那几张纸。

“您为什么要和这些人打交道?这种可恨的战争把什么都搞乱了,混杂了。唉,如果我能做得到的话,真想把混乱的东西分分清楚。”

利斯先生,并没有混乱。一切都很清楚,很简单。我们打败你们,用不着联合伊康尼科夫和切尔涅佐夫。我们有足够的力量对付你们,对付他们。

莫斯托夫斯科伊看出来,利斯把一切阴暗险恶的东西拉到了一起。垃圾坑的气味是一样的,所有的残屑、木片、碎瓦全都一样。不应该在垃圾里寻找区别或相似,而应当在建筑者的构思、在他的意图中去找。

于是他理直气壮地愤恨起来,不仅愤恨利斯和希特勒,而且愤恨那个问他对马克思主义的意见的浅色眼睛的英国军官,愤恨独眼龙孟什维克的可恶言论,愤恨窝窝囊囊、却做了警察内线的神甫。这些浑蛋怎么会认为社会主义国家和法西斯帝国有什么相同之处呢?只有这个秘密警察头目利斯才看得上他们的烂货。这时候莫斯托夫斯科伊比以往任何时候都了解法西斯与其代言人的真正联系。

莫斯托夫斯科伊心里想,斯大林的天才是否就在于此:在痛恨和消灭这一类人的时候,只有他看到法西斯和伪善者、虚伪的自由的宣扬者之间的秘密联盟。他觉得这个道理显而易见,他真想对利斯说一说,说明他的理论的荒谬性。但他只是笑了笑,他是老练的,他可不像傻瓜戈尔登别尔那样,跟高等法院检察长胡乱谈民意党的事。

他用眼睛直盯着利斯,大声说(大概站在门口的警卫也能听到他的声音):

“我劝您,不要在我身上浪费时间了。快把我枪毙,或者马上把我勒死,把我杀了吧。”

利斯赶紧说:

“谁也不想杀您。请放心吧。”

“我没什么不放心的,”莫斯托夫斯科伊快活地说,“我不想操心什么。”

“应该,应该操心!让我的失眠变成您的失眠吧。我们相互为敌的原因何在,我真不明白……希特勒不是元首,是斯廷内斯和克虏伯[3]家的仆人?你们没有个人土地所有权吗?你们的工厂和银行是属于人民吗?你们是国际主义者,而我们鼓吹民族仇恨吗?是我们放了火,你们在千方百计灭火吗?全人类都在仇恨我们,都在用期望的目光望着你们的斯大林格勒吗?你们是这样说吗?胡说!瞎扯!全是胡诌出来的。咱们的政体实质是一样的,都是党统治的国家。我们的资本家不是主人。国家给他们计划和规格。国家征收他们的产品和利润。他们留下百分之六的利润作为他们的工资。你们的党领导的国家也制订计划、要点,征收产品。你们叫做主人的工人,也从你们的党的国家手里领取工资。”

莫斯托夫斯科伊望着利斯,心里想:“难道就是这种卑劣的胡扯曾经使我困惑过一阵子吗?难道我会在这种又毒又臭的泥水中呛死吗?”

利斯失望地摇了摇手。

“你们的人民的国家打的是工人的红旗,我们也号召建立民族功绩和劳动功绩,号召团结,我们也说……党代表着德国工人的理想。’你们也说:‘民族性。劳动。’你们和我们一样,都知道:民族主义是二十世纪的主要力量。民族主义是时代灵魂。一个国家的社会主义是民族主义的最高表现!

“我认为咱们没有理由互相为敌。但是德国人民的天才领袖和导师、我们的父亲、德国母亲们的最好的朋友、最伟大和最英明的统帅发动了这场战争。不过我相信希特勒!我相信,你们的斯大林的头脑也并没有因为愤怒和头疼而糊涂了。他能够透过战争的硝烟和炮火看到真理。他了解自己的敌人是谁。他了解,很了解,即便他正在和敌人讨论应对我们的战略,在为敌人的健康干杯。世界上有两位伟大的革命家:斯大林和我们的领袖。是他们的意志产生了国家的民族社会主义。

“我认为,同你们联合,比起为了东方的辽阔土地而进行的战争更为重要。我们在建筑两座楼,两座楼应当在一起。老师,我希望您单独平静地生活一些时候,希望您想一想,好好想一想,下一次咱们再谈。”

“干什么?瞎扯!无聊!荒谬!”莫斯托夫斯科伊说。“干吗要这种莫名其妙的称呼‘老师’?”

“噢,这称呼可不是莫名其妙的,您和我应该明白:未来的命运不是在战场上决定的。您是了解列宁的。他创立了新型的党。是他第一个懂得了,只有党和领袖能反映民族的动机,所以取消了立宪会议。不过,就像麦克斯韦在物理上推翻牛顿力学的时候,他想的还是证实牛顿力学,列宁在创立二十世纪伟大的民族主义的时候,却认为自己是国际主义的创造者。后来斯大林教给我们很多东西。为了在一个国家实行社会主义,必须取消农民种地和做买卖的自由,于是斯大林毫不手软,消灭了几百万农民。我们的希特勒看出来:妨碍我们德国民族的社会主义运动的敌人是犹太人。于是他决定消灭几百万犹太人。不过希特勒不只是学生,他是天才!你们在一九三七年的清党,是斯大林从我们清除勒姆中看到的,看到希特勒也没有手软……您应该相信我。我在说话,您却不作声,不过我知道,我对您来说是外科手术上的镜子。”

莫斯托夫斯科伊说:

“镜子?你说的这一切,从头到尾都是胡说八道。我不想降低我的身份,驳斥你这些肮脏、发臭的无耻谰言。你是镜子吗?怎么,一点没有知觉吗?斯大林格勒会叫你恢复知觉的。”

利斯站起身来,莫斯托夫斯科伊慌乱、欣喜、愤恨地想:“这一下他要枪毙我了……完了!”

但是利斯好像没听见他的话似的,毕恭毕敬地向他深深鞠了一个躬。

“老师,”他说,“你们时时刻刻教导我们,也时时刻刻向我们学习。咱们所想的会完全一致的。”

他的脸是忧伤和严肃的,眼睛却在笑着。又好像有一根很毒的针扎了一下莫斯托夫斯科伊的心。利斯看了看表。

“时间不会白白过去的,”他按了按铃,低声说,“如果您需要的话,就把这写的东西拿去吧。咱们不久还要见面的。再见。”

莫斯托夫斯科伊自己也不知道为什么,拿起桌上的纸,塞进口袋里。他被带出了管理处的大楼。他吸了一口冷空气。在这湿乎乎的夜晚,离开秘密警察头目的办公室,不再听国家社会主义党理论家那低沉的声音,听着晨曦中的汽笛声,心里多么舒畅呀。

他被带到隔离室跟前,有一辆带紫色车灯的小汽车从肮脏的柏油路上开过。莫斯托夫斯科伊明白,这是利斯回去休息了。他又感到十分苦恼。押解兵把他送进隔离室,把门锁上。

他坐在铺上,心想:“如果我信仰上帝的话,就可以断定,这个可怕的交谈者是上帝派来惩罚我的,就因为我怀疑。”

他睡不着。新的一天已经开始了。他背靠在粗糙的杉木板墙上,看起了伊康尼科夫写的东西。

十 六

世间大多数人都不想为“善”下个定义。什么是善?什么人需要善?什么人行善?有没有通用的善,可以施之于一切人、一切民族、一切情况?或者,对我是善,对你就是恶,对我的民族是善,对你的民族就是恶?善是不是永久的、永远不变的,还是昨天的善今天就成为恶,昨天的恶今天就是善?

最后审判的时刻总是要到的,思考善与恶的不应只是哲学家和传教士,应该是所有的人,有知识的人和没有知识的人。

几千年来人类有关善的概念是否有过变化?有没有像福音书的圣徒所说的,不分希腊人与犹太人,不分阶级、民族、国家,对于所有的人都一样的这种概念?也许,这一概念的范围还要广泛些,适用于动物、树木、苔藓,也就是被释迦牟尼及其佛经列入善的概念的那种广义的概念?就是那个释迦牟尼,为了使人生充满善和爱,才得出人生一切皆空的结论。

我看到,几千年来,人类在哲学和道德方面的领袖产生的一些观念,使善的概念越来越狭窄。

与释迦牟尼相隔五世纪的耶稣的观念,使施善对象的范围变狭窄了。不是所有的生物,只是人!

早期基督徒的善,即所有的人的善,又变成只为基督徒的善,与之并存的是穆斯林的善。

但是过了几个世纪,基督徒的善又分裂为天主教徒的善、新教徒的善、东正教的善。在东正教的善中出现了旧教的善和新教的善。

同时存在的有富人的善和穷人的善,同时出现的有黄种人的善、黑种人的善、白种人的善。

而且,分裂了,又分裂,善已经被划进了宗派、种族、阶级的圈子,在圈子以外的一切人已经进不了善的圈子了。

于是人们看到,因为这种小的、不善的善,而同这种小善认为恶的一切东西进行斗争,流的血实在太多了。

有时这种善的概念本身会成为人生的灾难,成为比恶更恶的恶。

这种善是一种空壳,神圣的种子已经从其中脱出,失落。谁能把失落的种子还给人类呢?

究竟什么是善?有人曾经这样说:善—就是意愿和与意愿相连的能够使人类、家庭、民族、国家、阶级、信仰兴旺发达的行动。

为了个人的好处而奋斗的人,总是尽力给人为了大家的假象。所以他们说:我的好处和大家的好处是一致的,我的好处不仅对我有利,对大家都有利。我为自己做好事,其实是为大家做好事。

所以,善失去其公共性之后,一个宗派、阶级、民族、国家的善总是尽可能使自己带上虚伪的公共性,披上无私为公的外衣,实则打击自己认为恶的东西。

不过,就连残暴的希律一世进行血腥屠杀也不是为恶,而是为他的残暴者的善。因为新的力量来到世上,将会给他,他的家族、亲人、朋友,他的王国和军队带来灭亡的威胁。

但出现的不是恶,出现的是基督教。人类从来没听到过这样的话:“不可判断人,免得你们被判断。你们怎样判断人,也必怎样被判断;你们用什么标准衡量人,也必照样被衡量……当爱你们的仇敌,为迫害你们的祈祷……你们愿意人怎样待你们,你们也要怎样待人,这是律法和先知的总纲。”[4]

这条和平与爱的教义给人类带来的是什么?

拜占庭的圣像破坏运动,宗教法庭的刑讯,法国、意大利、佛兰德、德国的反异教运动,新教和天主教的斗争,僧侣会的阴谋诡计,尼康和阿瓦库姆的斗争,很多世纪以来对科学和自由的压制,基督徒对塔斯马尼亚异教居民的大屠杀,焚烧非洲黑人村庄的歹徒。所有这一切造成的灾难,超过了强盗和歹徒为作恶而作恶犯下的罪恶。

人类的人道主义学说本身的命运也是这样使人震惊,使人焦虑,人道主义学说没有逃脱共同的命运,也分裂为一个个局部的、小圈子的善。现实的残酷使一些伟人的心里产生了善,他们使善回到现实中来,一心想按照他们心中的善的模式改造现实。但是,现实并没有按照善的概念的模式变化,而是善的概念陷进了现实的泥淖中,渐渐分裂,失去原有的公共性,为当前的现实效劳,而不是按照自己的美好的、无定形的模式塑造现实。

人们往往认为现实的变化就是善与恶的斗争,但实际不是这样。希望人类善良的人,无法消除现实的恶。

需要有伟大的思想,能够开辟新的渠道,把石头推开,把暗礁消除,把森林移开,需要有公共的善的理想,好使伟大的流水和谐地流动。假如大海一旦有了思想,那么,每次风暴来临时,海水会产生幸福的思想和理想,每一股海浪在岩石上碎裂时,会以为它是为海水的好处牺牲的,就不会想到这是风把它吹起来的,尽管在这之前的千千万万股海浪都是风吹起的,今后风还会吹起千千万万股海浪。

很多书写了怎样同恶作斗争,写了什么是善,什么是恶。

但是这一切毫无疑问都是可悲的。其可悲就在于:哪里有善的曙光升起—这种善是永恒的,并且永远不会被恶所战胜,当然那种恶本身也是永恒的,也永远胜不过善—哪里就会流血,就会有大批儿童和老人死于非命。不但是人,就连上帝也无法消除现实的恶。

“在拉玛听见有声音,是痛哭、极大哀号的声音;拉结为她的儿女哀哭,不肯受安慰,因为他们都不在了。”[5]至于圣人认为什么是善,什么是恶,对于失去孩子的她来说,都无所谓了。

不过,也许,现实就是恶?

我看到我国产生的社会的善这一思想具有不可动摇的力量。我在普遍集体化时期看到了这种力量,在一九三七年也看到了这种力量。我看到,为了善的思想—这种思想极其美好,极其人道,就像耶稣教的理想那样—为了这种思想消灭了许多人。我看到整村整村的人死于饥饿,我看到农民的孩子死在西伯利亚的雪地里,我看到一列列军车把成千成万男人和女人从莫斯科、列宁格勒和苏联其他城市送往西伯利亚,因为他们被划为社会的善这种光辉伟大思想的敌人。这种思想是美好的和伟大的,所以要杀掉一些人,摧残一些人的生活,要使妻子离开丈夫,使孩子离开父母。

今天德国法西斯的巨大恐怖笼罩了世界。到处可以听到就死者的哀号和呻吟声。到处弥漫着焚尸炉的烟,天空黑暗,日月无光。可是,就连这样的罪行,这种全世界人类不曾见过的罪行也是借了善的名义。

当年我住在北方森林里的时候,曾经想过,善不在人类中,不在动物和昆虫的相互残杀的世界中,而是在默默无言的树木的世界里。可是,不对!我见到过森林的骚动,见过树木为争夺土地,阴险毒辣地同青草和灌木进行搏斗。千千万万种子飞播开去,生根发芽,渐渐把青草弄死,把友好的灌木扼杀。成千成万幸存下来的幼芽开始优胜劣汰,相互搏斗。只有那些活下来的树木组成了统一的新的林冠,彼此缔结势均力敌的同盟,分享温暖的阳光。云杉和山毛榉则在这林冠之下昏暗的苦役牢里冻得瑟瑟发抖。但是独占阳光的树木总有衰老的时候,高大的云杉就从它们的林冠底下冲出来,冲向阳光,又将赤杨和白桦扼杀。

树木就是这样永远生活在你争我夺中。只有瞎子才认为树木和草的世界是善的世界。难道生存就是恶?

善不在自然界,不在传教士和圣人的说教中,不在伟大的社会学家和人民领袖的学说中,不在哲学家的道德中……倒是一些普通人心里怀着对活物的爱,很自然地、不由自主地珍爱和怜惜生命,喜欢在劳动一天之后享受一下炉灶的温暖,不在场地上烧火堆和放火。

所以,除了可怕的大的善,还有平常的人的善良。一个老奶奶拿一块面包给俘虏吃,一个士兵把壶里的水给受伤的敌人喝,年轻人怜惜老年人,农民把犹太老头子藏在草垛里,这都是善良。有的看守人员冒着个人失去自由的危险,把囚犯或俘虏的信件传送出来,不是给志同道合的同伴,而是给母亲和妻子们,这也是善良。

这是个人之间偶尔为之的善良,是无需证明的、没有用心的小善良。可以叫做无意识的善良。是宗教的善和社会的善之外的善良。

但是,我们只要一想就可以看出来,这种无意识的、个人间的、偶然性的善良是永恒不灭的。这种善良可以施于一切生物,甚至一只老鼠,一根树枝都可以受到这种善的恩泽—有时行人会忽然站下来,扶一扶受伤的树枝,让它更容易重新长到树干上。

在可怖的时代,在以国家民族光荣为名义、以对全世界行善为名义而进行疯狂残杀的时候,在人已经不像人,而只是像树枝一样荡来荡去,又像一块块石头填进山沟和土坑的时候,就是在这种可怖和疯狂的时代,这种没有用心的、可怜的、像镭粒子一样分散在生活中的善良也没有消失。

有一些德国兵来到村子里。昨天在路上有两名德国兵被打死。晚上把一些妇女赶出去,叫她们在树林边挖坑。有几名士兵住到一个上了年纪的女人家里。她的丈夫被带到警察所去了,那里还关着二十个农民。她一夜没有睡,德国兵在地下室里搜到一筐鸡蛋和一瓶蜂蜜,他们自己生起炉子,炒鸡蛋,喝酒。有一个年纪大些的吹起口琴,其余的人又跺脚又唱歌。他们对女房东连看也不看,好像她不是一个人,而是一只猫。早晨,天亮了,他们开始检查自己的枪。那个年纪大些的士兵很笨拙地拉了一下枪机,一颗子弹打进自己的肚子里。大家一齐叫起来,乱成一片。几个德国兵草草地给他包扎了一下,就把他放到床上。接着他们都被叫走了。他们临走向女房东打了打手势,叫她照应受伤的德国兵。女房东看到,要把他掐死不费吹灰之力。他一会儿嘟哝,一会儿闭上眼睛,又哭又咂吧嘴。后来忽然睁开眼睛,很清楚地说:“妈妈,给我水。”女房东说:“哼,你这该死的东西,把你掐死才好呢。”可是她还是给他端来了水。他抓住她的手,叫她把他扶起来,因为血堵得他不能喘气。她把他扶起来,他用两手勾住她的脖子,支撑着身子。这时村子里响起一片枪声,她吓得直打哆嗦。

后来她说起当时的情形,但是谁也无法理解,她也无法解释。

这是一种善良。有一则寓言说一个修士让蛇在怀里暖和身子,就是指责这种善良没有意义。这种善良,就好比爱惜咬死小孩的毒狼蛛。这是一种不理智的、有害的、荒唐的善良!

人们乐于援引寓言中的例证,记住这种没有意义的善良带来的(和可能带来的)害处。不必害怕!如果怕这种善良,就好比一条淡水鱼偶然从河里来到水咸的大海里,感到害怕。

没有意义的善良有时给社会、阶级、民族、国家造成的害处,与天生善良的人发出的光相比,是会黯然失色的。这种没有意义的善良正是人的人性,它就是人和其他一切的区别,它就是人的精神所能达到的最高境界。它说明,生存并不就是恶。

这种善良是没有言语、没有用心的。它是本能的。是盲目的。一旦耶稣教把它变为教堂神甫的教义,它就变得暗淡了,种子就变成了空壳。当善良没有言语、没有心思、没有用意的时候,当善良隐藏在人心里的时候,当善良没有成为传教士的武器和商品,当矿金没有炼成神的金币的时候,善良是有生命力的。它就像生活一样实实在在。就连耶稣的说教,也使善良失去其生命力。善良的生命力在人心的不言不语中。

但是,我怀疑人类的善,也怀疑善良。我很惋惜它没有生命力!它既然没有什么感染力,又有什么好处呢?

我以为,它没有生命力。美好而又没有生命力,简直就像露水。

怎么能不使它枯死,不使它丢失,而使它变得有力呢?教会就是使它枯死了,将它丢失了。当善良不是什么力量的时候,它才是有生命力的。只要人想把善良变为力量,它就失去本色,就会暗淡,失去光彩,消失。

现在我看到恶的真正力量。天国是空的。地上只有人。拿什么来扑灭恶呢?拿人类的善良,拿这样几滴露水?但是要知道,这种火用所有的海洋里的水和所有云层的水都是扑不灭的,从福音书的时代直到今天的钢铁时代所汇集起来的一点点可怜的露水也扑不灭……

我再也不相信能够在上帝身上、在自然界找到善,就这样,我再也不相信善良。

但是,法西斯的黑暗在我面前暴露得越多,越广,我就越加看清:人性总是存在的,是泯灭不了的,即使在浸透了血的黄土的旁边,在毒气室的门口。

我在地狱里锻炼了信心。我的信心是从焚尸炉里出来的,是穿过了毒气室的水泥墙的。我看出来,不是人在同恶的斗争中软弱无力,是强大的恶在同人的斗争中软弱无力。毫无意义的善良永远不灭的秘密,就在于它的无力。这种善良是不可战胜的。这种善良越傻,无力,没有意义,就越是巨大。恶对它无可奈何。圣人、传教士、宗教改革家、首领、领袖,在它面前无可奈何。它是一种不看什么、不说什么的爱,是人的本义。

人类的历史不是善极力要战胜恶的搏斗,人类的历史是巨大的恶极力要辗碎人性的种子的搏斗。但是,如果人性就是现在仍没有被摧残殆尽的话,那么,恶已经不可能取得胜利了。

莫斯托夫斯科伊看完之后,半闭起眼睛,坐了好几分钟。

是的,这是一个受了震动的人写的。一个可怜的灵魂的悲剧!

这个蔫了的人竟说,天国是空的……他把人生看作一切人对一切人的战争。到末了他玩弄起旧的铃铛,玩弄起老奶奶的善良,还打算用灌肠的喷嘴扑灭世界的大火。这一切多么无聊呀!

莫斯托夫斯科伊望着隔离室的灰墙,想起了天蓝色的安乐椅,想起他和利斯的谈话,感到十分沉重。头并不难受,是心里难过起来,呼吸也困难了。看样子,他怀疑伊康尼科夫,是错了。这个呆子写的东西,不仅引起他的鄙视,也引起夜里和他谈话的那个讨厌的家伙的鄙视。他又想了想自己对切尔涅佐夫的感觉,想了想利斯谈到这一类人时鄙夷和仇恨的口气。他的模模糊糊的苦恼似乎比肉体的痛苦更使他难受。

十 七

谢廖沙·沙波什尼科夫指着背囊旁边砖堆上的一本书,说:

“你看过吗?”

“看了好几遍啦。”

“喜欢吗?”

“我更喜欢狄更斯。”

“嘿,狄更斯。” 他用讥笑的、傲慢的口气说。

“你喜欢《巴马修道院》吗?”

“不怎么喜欢。”他想了想,回答说。又补充道:“今天我要跟步兵一起把旁边一座小屋的德国佬打出去。”他看到她的目光,又说:“当然,是格列科夫的命令。”

“别的迫击炮手呢,琴佐夫呢?”

“他们不去,就我一个去。”

他们沉默了一会儿。

“他老是缠着你吗?”

她点了点头。

“你怎么样?”

“你知道嘛。”

“我觉得,我今天可能被打死。”

“为什么叫你跟步兵一起去?你是迫击炮手啊。”

“为什么他要把你留在这儿?报话机已经打成碎片。早就该把你送回团里去,上左岸去。你在这儿无事可干,成了流浪女了。”

“不过咱们可以天天见面呀。”

他摆了摆手,就走开了。

卡佳向周围看了看。彭丘克在二楼望着,笑着。显然,谢廖沙也看到了彭丘克,所以突然走开了。

德军用大炮轰这座楼房,一直轰到黄昏时候。有三个人受轻伤,有一段内墙倒塌下来,把地下室的出口堵住了。大家把出口处打通,一颗炮弹又炸倒一段墙,地下室出口又被堵住,大家又开始挖。

安齐费罗夫朝灰尘飞扬的幽暗处瞅了瞅,问道:

“喂,报话员同志,您活着吗?”

“是的。”卡佳在幽暗处回答说。她打了一个喷嚏,啐‡º一口红色的痰。

“祝您健康。”一名工兵说。

等到天黑下来,德军打出照明弹,用机枪扫射起来,有几架轰炸机飞来,扔下爆破弹。谁也没有睡觉。格列科夫亲自打起机枪,步兵有两次大声骂着娘,用工兵的锹掩护着脸,冲上去把德国佬打退。

德国佬似乎觉得,他们不久前占领的这座无主的楼房,马上就要遭到进攻。

当枪炮声停息的时候,卡佳能听到他们吵吵嚷嚷说话的声音,就连他们的笑声也能听得很清楚。

德国佬说话非常难听,发音完全不像外语课教师教的那样。她看到小猫从垫子上爬了下来。小猫后面两个爪子不能动了,只用两个前爪在爬,正急急忙忙朝卡佳爬来。

后来小猫不爬了,嘴张了几下,就闭上了……卡佳拨了拨小猫合上的眼皮。“死了。”她在心里说,蓦地浮起一股厌恶感。忽然她明白了,这已经半麻痹的小猫是预感到要死了,所以想到她,向她爬来……她把已死的小猫放进一个坑里,上面撒了一些碎砖渣子。

地下室里充满了照明弹的光,她觉得似乎地下室里没有空气,似乎她呼吸的是一种带血的液体,这种液体从天花板上往下流,从每一块砖里往外渗。

眼看着德国佬从远处的角落里爬出来了,正在朝她爬,马上就会把她抓住,把她带走。已经很近了,他们就在跟前打枪。也许,德国佬要扫荡二楼?也许,他们不从下面来,而是从上面,从天花板的窟窿里跳下来?

为了让自己镇定,卡佳尽量回想钉在门上的小卡片:“季霍米罗夫家—按一下,茨加家—按两下,契列穆什金家—按三下,芬别尔格家—按四下,文格罗夫家—按五下,安德留先科家—按六下,彼果夫家—长长的一下……”她拼命回想芬别尔格家放在煤气炉上、盖着胶合板的大锅子,回想阿纳斯塔西娅·斯捷潘诺芙娜·安德留先科家蒙着大罩子的木盆、季霍米罗夫家挂在绳扣上的掉了瓷的脸盆。她想象她在给自己铺床,把妈妈的棕色头巾、一块棉绒、开了绽的夹大衣都垫到弹簧坏了的褥垫底下。

然后她就想“6—1”楼房。这会儿,当希特勒的匪徒步步逼近,从地上爬过来的时候,那些粗野的骂娘话似乎也不可恼了,格列科夫的目光她也不怕了,以前她看到那目光,不仅脸会红,连脖子,连军装里面的肩膀都会红的。在参军后的这几个月里,她听了多少下流话!当秃顶的中校龇着金牙暗示她可以留在河那边的通讯站时,她用“无线电”和他进行了多么糟糕的通话呀……她想起有些女孩子小声唱的伤心的歌儿:

有一个秋夜里

指挥官亲自将她温存

唤了一夜小亲亲

从此她就自暴自弃……

她不是胆小鬼,只不过出现了这样的心情。

她第一次看到谢廖沙,是在他念诗的时候,她心里想:“真是一个呆子。”后来他有两天不见人影,她也不好意思打听他,心里老是在想,他是不是被打死了。后来他在夜里突然出现了,她并且听见他对格列科夫说,他是从司令部的掩蔽所里偷跑回来的。

“好,”格列科夫说,“你这是开小差跟着我们朝阴间跑。”

谢廖沙在离开格列科夫从卡佳身边走过的时候,没有看,也没有回头。她很难过,后来生起气来,又在心里说:“傻瓜!”

后来她听到楼房里的人的谈话。他们说的是,谁最有可能第一个和卡佳睡觉。有一个说:

“不用说,是格列科夫。”

另外一个说:

“这可不一定。不过,谁的名次排在最后面,我倒是可以说说,那就是迫击炮手谢廖沙。女孩子越是年轻,越喜欢老练的男子。”

后来,她发现几乎没有人跟她逗着玩儿、开玩笑了。格列科夫毫不掩饰别人接触卡佳时他的不快心情。

有一次,大胡子祖巴廖夫喊她:

“喂,楼长夫人。”

格列科夫并不着急,但是他显然很有信心,而且她也感觉得到他自己很有把握。在报话机被炸成碎片之后,他叫她躲到很深的地下室的一个隔间里。昨天他对她说:“我这一辈子还没有见过像你这样的姑娘。”又补充说:“我要是在战前遇到你,一定会娶你。”她想说,要娶她还得问问她呢,可是她没有说,她不敢说。他对她没有任何不好的行为,没有对她说过任何粗野的下流话,但是她想到他,就觉得可怕。

也是昨天,他很忧愁地对她说:

“德国佬很快就要发动进攻了。咱们这里面的人未必有谁能活下来。德国佬钉住我们的楼房不肯放。”

他用缓慢而凝神的目光把她打量了一下,卡佳害怕了,不是因为想到了德国佬即将发动的进攻,而是因为看到这缓慢而镇静的目光。

“我会上你这儿来的。”他说。似乎这话和他说的在德国佬发动进攻之后未必有谁能活下去的话没有什么联系,但联系是有的,而且卡佳也明白了。

他不像她在科特卢班山下看到的那些指挥员。他和人说话既不高声大叫,也不吓唬,可是大家都听他的。他坐在那里,又抽烟,又说笑,又听别人说笑,跟士兵没有任何区别。可是他的威信很高。

她和谢廖沙几乎不说话。她有时觉得,他爱上她了,可是也和她一样,在又喜欢又怕的人面前非常胆怯。谢廖沙又胆小,又没有经验,可是她真想请求他保护,对他说:“来我这儿坐坐吧。”有时她还想安慰安慰他。跟他在一块儿说话,感觉真是奇怪,就好像根本没有打仗,也没有这“6—1”楼房。他也好像感觉到这一点,就有意尽量表现得粗野些,有一次他还在她面前骂过娘。

就这会儿她也觉得,她模模糊糊的想法和感情与格列科夫派谢廖沙去攻打德国佬占的房子这件事有一种无情的联系。她听着枪声,想象着,谢廖沙躺在红红的砖堆上,已经死去的蓬乱的头耷拉下去。

她感到对他心疼得不得了。五光十色的夜晚的战火,对格列科夫的害怕,对他的钦佩,钦佩他敢于凭借一片瓦砾向德军的钢铁队伍发动进攻,还有对母亲的想念—这一切在她心里交织在一起了。

她想,只要能看到谢廖沙活着回来,她愿意奉献她的一切。

“要是有人问,要妈妈还是要他,怎么办?”她心里想道。

后来她听见一个人的脚步声。她用手指头抓住一块砖,仔细听着。

枪声停了,一切都静下来。她的脊背、肩膀、小腿都痒起来,但是她怕挠痒,怕弄出响声。有人问巴特拉科夫,为什么他老是挠痒,他回答说:“这是神经性的。” 可是昨天他说:“我在身上逮了十一个虱子。”于是科洛密采夫笑着说:“神经性的虱子咬巴特拉科夫啦。”

等到她被打死了,大家把她抬到坑边,会说:

“这可怜的姑娘浑身都是虱子啦。”

也许,这真是神经性的?于是她明白了,黑暗中有一个人向她走来了,不是虚幻的、臆想的人,是从沙沙声中,从一片片亮光、一片片黑暗,从紧张的心跳中出现的。卡佳问:

“是谁?”

“是我,自己人。”黑影回答说。

十 八

“今天不发动进攻了。格列科夫决定改在明天夜里。今天德国佬一个劲儿地在进攻……我想顺便说说,那本叫《修道院》的小说,我从来没看过。”

她没有回答。

他很想在黑暗中看清她的神情,一阵爆炸的火光顺应他的愿望,把她的脸照得透亮。过了一秒钟,又黑了下来,于是他们又无声地商量好,等待下一次爆炸和闪光。谢廖沙握住她的手。他攥住她的手指头。他平生第一次把姑娘的手握在自己手里。

生满虱子的肮脏的姑娘静静地坐着,她的脖子在黑暗中发亮了。突然闪起照明弹的亮光,他们把头挨在一起。他把她抱住,她眯起眼睛,他们都知道学校里有一个说法:谁睁着眼睛接吻,谁就不是真爱。

“这不是开玩笑,是吗?”他问道。

她用两手捧住他的两鬓,把他的头转过来朝着自己。

“一生一世,永不变心。”他说得很慢。

“太好了,”她说,“我就是怕,忽然有什么人来。可是以前不论谁来,不论是里亚霍夫、科洛密采夫、祖巴廖夫……我有多么高兴呀。”

“还有格列科夫。”他提醒说。

“哎呀,不。”她说。

他吻起她的脖子,并且解开她军装上的扣子,拿嘴去吻她那瘦削的锁骨,却不敢吻她的胸脯。她抚摩着他那硬扎扎的、没有洗过的头发,就好像他是一个小孩子,她已经知道,这一切现在是不可避免的了,这都是应该有的事了。

他看了看发光的表盘。

“明天谁带你们去进攻?”她问道。“是格列科夫吗?”

“你问这干什么?我们自己去,用不着谁带我们。”

他又把她抱住,忽然他的手指头发凉,由于下了决心,情绪激动,胸中也发起凉来。她半躺在军大衣上,似乎连气也不喘了。他一会儿碰着她那粗糙的、好像蒙着灰土似的军服和裙子,一会儿碰着她那扎手的充革布高筒靴。他的手感觉到她的身体的温暖。她想坐起来,但是他吻起她来。忽然一阵亮光闪起,刹那间照亮了落在砖堆上的卡佳的军帽,照亮了她的脸,在这几秒钟里,他觉得她的脸和往常大不一样。可是马上又黑了下来,而且不知为什么特别黑……

“卡佳!”

“怎么了!”

“没什么,就是想听听你的声音。你为什么不看我?”

“别这样,别这样,冷静点儿!”

她又想起他和她母亲,想着她应该更喜欢谁。

“原谅我。”她说。

他没有明白她的意思,就说:

“你别怕,我一辈子不变心,只要能活下去的话。”

“我这是想起了妈妈。”

“可是我的妈妈死了。我现在才明白,她是因为我爸才被流放的。”

他们互相拥抱着,在军大衣上睡着了。楼长走到他们跟前,看了看他们的睡态:迫击炮手谢廖沙的头枕在报话员姑娘的肩上,他的一只手搂着她的腰,他好像怕把她丢了。格列科夫觉得他们两个都死了,因为他们躺在那里一动也不动,那样安静。

黎明时候,里亚霍夫朝地下室的隔间里瞅了瞅,喊道:

“喂,沙波什尼科夫,喂,文格罗娃,楼长叫你们,要快点儿,麻利点儿!”

在朦胧而寒冷的晨曦中的格列科夫的脸是阴沉的、严峻的。他的一个宽大的肩膀靠在墙上,蓬乱的头发耷拉在窄窄的前额上。

他们站在他面前,倒换着两只脚,没有觉察到他们是手挽手站着。

格列科夫动了动他那扁平的狮鼻的大鼻孔,说:

“是这样,沙波什尼科夫,你马上到团部去,我派你去。”

谢廖沙感觉到姑娘的手指在抖动,就把她的手指头攥住,于是她也感觉到他的手指在抖动。他吸了一口气,感到舌头和上腭发干发燥。

多云的天空和大地一片寂静。盖着军大衣胡乱躺在地上的人似乎都没有睡,都在等待着,连气也不喘。

周围的一切都很好,都很可亲,谢廖沙心想:“要把他从天堂赶出去,要像拆散农奴一样把我们拆散了。”他怀着祈求和仇恨的心情望着格列科夫。

格列科夫眯起眼睛,凝视着姑娘的脸,谢廖沙觉得他的目光很讨厌、很无情、很放肆。

“好吧,就这样,”格列科夫说,“报话员同志跟你一块儿去,没有报话机,她在这儿无事可干,你把她送回团部去。”他笑了笑。“以后你们上哪儿,到时候你们自己知道。这是调派信,我把你们两个人写在一起了,我不喜欢写字。明白吗?”

谢廖沙忽然看到,一双透着亲切、精明然而又忧伤的眼睛正望着他,这样的眼睛他还从来不曾见过。

十 九

步兵团政委皮沃瓦罗夫没有到过“6—1”楼房。和楼房的无线电联系中断了,不知是报话机坏了,还是上级的严厉训斥让楼房里的头头儿格列科夫大尉厌烦了。

有一段时间,可以通过一名党员迫击炮手得到有关被围大楼里的情况的报告。他报告说,楼长作风散漫,对士兵们信口开河,胡说八道。不过,格列科夫同敌人作战是很勇敢的,这一点汇报人也不否认。

就在皮沃瓦罗夫准备进入“6—1”楼房的这天夜里,团长别廖兹金害起重病。他躺在掩蔽所里,脸烧得通红,睁着失神的、透明的、茫然的眼睛。

医生看了看别廖兹金,慌了。他治惯了打断的胳膊腿和打裂的头盖骨,现在却是一个人本身害起病来。医生说:

“要拔火罐,可是上哪儿去弄罐子呀?”

皮沃瓦罗夫决定向上级报吿团长的病情,可是师政委打电话给皮沃瓦罗夫,要他立刻到师部去。

当皮沃瓦罗夫喘着粗气(他遇到炮弹爆炸,曾经两次卧倒)走进师政委的掩蔽所时,师政委正在和从左岸来的一位营政委说话。皮沃瓦罗夫听说这个人常常给驻扎在各个工厂里的部队作报告。

皮沃瓦罗夫大声报告说:“奉命来到。”并且马上就报告了别廖兹金的病情。

“啊……伤脑筋,”师政委说,“皮沃瓦罗夫同志,您得担当起团的指挥任务了。”

“被围困的楼房怎么办?”

“您不用管了,”师政委说,“这座被围的楼房惹出大麻烦。这事儿弄到方面军司令部去了。”

他把一张密码电报对着皮沃瓦罗夫晃了晃。

“我就是为这事叫您来的。这不是,克雷莫夫同志接到方面军政治部的命令,要他进入被困的楼房,建立布尔什维克党的秩序,在那里做作战政委,如有必要,就解除那个格列科夫的职务,自己担任指挥……因为这是在你们团的地段上,所以你们要给予一切必要的供应,要负责帮助进入被困楼房,负责今后的联系。明白吗?”

“明白了,”皮沃瓦罗夫说,“一定做到。”

说过这话以后,他改变了谈公事的腔调,用平时谈家常的语气问道:

“营政委同志,跟这样一些小伙子打交道,是您的专长吗?”

“正是我的专长,”从左岸来的政委笑着说,“一九四一年夏天我带领二百人突围,在乌克兰到处转,那时候见惯了游击习气。”

师政委说:“好吧,克雷莫夫同志,那您就去干吧。多跟我联系。国中有国是很不好的。”

“是啊,那里面的人还和报话员姑娘有一些不干不净,”皮沃瓦罗夫说,“我们的别廖兹金一直在担心。他们的报话机又叫不通。那里面的小伙子又是那种样子,他们什么事儿都会干出来。”

“好啦,到里面您就清楚了,要好好地整一整,祝您成功。”师政委说。

二 十

格列科夫送走谢廖沙和卡佳之后,过了一天,克雷莫夫便在一名士兵护送下,前往被德军围困的著名大楼。

他是在明亮而寒冷的黄昏时候从步兵团团部出发的。克雷莫夫一进入斯大林格勒拖拉机厂铺了沥青的院子,就比任何时候更清楚、更强烈地感觉到死亡的危险。

同时,他的振奋和喜悦依然没有消失。突然收到的方面军司令部的密码电报向他证实了,在斯大林格勒这地方,一切都不一样,这里是另外一种关系,另外一种评价标准,对人有另外一种要求。克雷莫夫又是克雷莫夫了,不是残废队的残废人,而是布尔什维克的作战政委了。危险而困难的任务并没有使他感到害怕。在师政委和皮沃瓦罗夫的眼里他又看到了过去党内同志常常对他流露的那种神情,感到何等愉快,何等甜蜜。

在被炸得坑坑洼洼的沥青地上,炸坏的迫击炮旁边,躺着一名被打死的红军战士。

现在,就在克雷莫夫心里充满了希望,兴高采烈的时候,这具尸体的样子,不知为什么令他大吃一惊。他见过许多死人,对死人已经没什么感觉了。可是现在他哆嗦起来—已经僵了的尸体像鸟儿一样软弱无力地躺着,蜷着两条腿,好像怕冷。

一个身穿歪歪扭扭的灰斗篷的政治指导员提着鼓鼓的图囊从旁边跑过,几名红军士兵用帆布裹着防坦克地雷和大面包,拖着往前走。

死人不需要面包和武器,也不希望收到忠诚的妻子的来信。他并没有因为死就强大起来,他是最弱小的,像一只死麻雀,连小蚊子、小虫儿都不怕他。

在车间的一个墙豁口里,炮兵们正在安置团里的一门炮,并且和一挺重机枪的机枪手争吵。从争吵者的手势可以清楚地看出来,他们吵的是什么。

“你要知道,我们的机枪在这儿待了多久啦?你们还在河那边逛荡的时候,我们就在这儿打起来了!”

“真不要脸,你们算什么人!”

空中一声尖啸,一颗炮弹在车间角落里爆炸了。炮弹片打在墙上。走在克雷莫夫前面的士兵回头看了看,看看政委是不是被炸死了。等到克雷莫夫走到跟前,他说:

“政委同志,您别怕,我们认为,这儿是第二梯队,是大后方。”

过了不长时间,克雷莫夫就明白了,车间墙外的院子确实算是很平安的地方。

他们又跑,又卧倒,把脸埋在地里,然后又跑,又卧倒。他们有两次跳进步兵所在的战壕里,他们在烧毁的房屋中间跑,这一带已经没有人了,只有钢铁的呼啸与尖叫声……那名士兵为了安慰克雷莫夫,又说:

“这不算什么,顶要紧的是飞机没有轰炸。”但接着又提议说:“来,政委同志,咱们下到这个弹坑里避避。”

克雷莫夫溜进弹坑里,朝上面看了看:蓝天还在头顶上,头也没有掉下来,依然长在肩膀上。只有死神在前后左右,在头顶上啸叫和狞笑的时候,才感觉到人的存在是很奇怪的。

在死神挖出的坑里有一种安全感,也是很奇怪的。那士兵不等他喘息过来,就说:“跟我进去!”他爬进了坑底一个黑咕隆咚的通道口。克雷莫夫跟着他钻进去,低矮的通道口变宽了,顶也变高了,他们进了地道。

在地下可以听到地上大战的隆隆声,穹顶在颤动,隆隆声在地道里滚动着。在铁管特别密集、手臂粗的黑电缆纵横交叉的地方,墙上用红颜料写着“马霍夫是头驴”。那士兵用电筒照了照,说:

“咱们头顶上就是德国佬了。”

一会儿他们拐进一条窄窄的通道,朝着一个隐约可见的灰色光点走去。通道深处的光点越来越清楚,越来越亮,传来的爆炸声和机枪射击声也越来越激烈。

有一小会儿,克雷莫夫觉得他这是朝死刑台走去。但是等他们来到地面上,克雷莫夫看到的首先是一张张人的脸;他觉得这一张张脸像圣像一样安详。

克雷莫夫感到一种说不出的高兴和轻松。他甚至感到,这疯狂的战争不像是生与死的可怕关头,而是年轻、强壮、充满生命力的行路人头顶上的雷雨。

他清楚地感觉到一种坚定的自信,相信他现在时来运转了。他好像在这一天的光明中看到了自己的未来—他又可以充分发挥自己的才干、志向和布尔什维克的抱负了。

跟这种年轻的豪情壮志交织在一起,他又想起了离他而去的妻子。他觉得她是无比可爱的。

现在他觉得并没有永远失去她。她会跟着他的力量,跟着以前的生活一起回到他这里的。他离不开她。

有个老兵把军帽扣在额头上,站在一堆火旁边,用刺刀翻着在洋铁瓦上烙的土豆饼;土豆饼烙好了,他就放到钢盔里。他一看到这个联络员,很快地问道:“谢廖沙在哪儿?”

联络员一本正经地说:“首长来啦!”

“老爹,多大岁数了?”克雷莫夫问。

“六十了。”老头子回答说,又解释说:“我是从工人民兵里来的。”

他又侧眼看了看联络员。

“谢廖沙在哪儿?”

“他不在团里,看样子,他到友邻部队去了。”

“唉,”老头子懊丧地说,“他要完啦。”

克雷莫夫向大家问好,向周围看了看,又去看了地下室里板壁只剩一半的隔间。有一处安放着团里的一门炮,炮口从墙上打的一个窟窿伸出去。

“就像在战列舰上。”克雷莫夫说。

“是的,不过水太少啦。”那个士兵说。

再往前,在石头坑里和夹缝里安放着迫击炮。在地上放着一些带尾巴的地雷。再过去一点儿,防雨布上放着一架手风琴。

“咱们‘6—1’号楼撑住了,没有向法西斯屈服,”克雷莫夫大声说,“全世界千千万万人都会为这感到高兴。”

大家都没有说话。波里亚科夫老头子把装满土豆饼的钢盔端到克雷莫夫面前。

“关于波里亚科夫怎样烙饼,不会报导吧?”

“你们光知道笑,”波里亚科夫说,“可是我们的谢廖沙被赶走了。”

这个迫击炮手问道:

“还没有开辟第二战场吗?一点消息也没有吗?”

“还没有。”克雷莫夫回答说。

有一个穿着汗衫、敞着军服上衣的人说:

“有一次伏尔加河那边的重炮朝我们轰,一阵气浪把科洛密采夫打倒,他爬进来就说:‘好啦,同志们,开辟第二战场啦。’”

一个黑头发的小伙子说:

“干吗要瞎说,假如没有重炮的话,咱们在这儿也待不住。德国佬早把咱们吃掉啦。”

“可是,指挥员在哪儿呀?”克雷莫夫问。

“那不是,躺在最前沿上呢。”

这支队伍的指挥官正躺在高高的砖堆上,用望远镜在瞭望。

克雷莫夫唤他一声,他很不情愿地转过脸来,带着警告的神气调皮地把一个指头放到嘴上,又用望远镜了望起来。过了一会儿,他的肩膀抖动起来,他笑了。他从上面爬下来,笑着说:

“比下棋还不如呢。”

他打量了一下克雷莫夫军服上的绿杠和政工人员军星,说:“营政委同志,欢迎光临寒舍。”并且自我介绍说:“我是楼长格列科夫。您是从我们的地道里来的吗?”

他的一切—他的目光,他的快动作,他的扁鼻子的大鼻孔—都是很粗野的,本身就是粗野。

“没什么,没什么,我会让你服帖的。”克雷莫夫在心里说。

克雷莫夫开始向他询问情况。格列科夫懒洋洋地、心不在焉地回答着,一面打呵欠,一面四处张望,好像克雷莫夫的问话打扰了他,使他不能回想真正重要的、有意义的事情。

“要是把您撤掉呢?”克雷莫夫问。

“为什么?”格列科夫回答说。“顶好用小教练机送点儿黄烟来,当然,还要迫击炮弹、手榴弹,如果舍得的话,再弄点酒和吃的东西来……”他扳着手指头数算着。

“这么说,您不准备离开了?”克雷莫夫生气又不满地端详着格列科夫很不好看的脸,问道。

他们都不说话了,在这短短的沉默时间里,克雷莫夫战胜了自己要在精神上制服被困大楼里的人的心情。

“您写作战日记吗?”他问道。

“我没有纸,”格列科夫回答说,“没地方写,而且也没有工夫,也没有必要。”

“您是在一七六步兵团团长领导下呀。”克雷莫夫说。

“是,营政委同志。”格列科夫回答说。又用冷笑的口吻说:“在这块地段被截断,我在这座楼房里把人和武器集合起来,打退三十次进攻,烧毁八辆坦克的时候,没有什么人领导我。”

“现有人员的准确数字,您知道吗?检查过吗?”

“我用不着检查,我又不申报队列人员名单,又不到行政管理处和补给站领给养。我们有烂土豆吃,有臭水喝就行了。”

“这楼里有女人吗?”

“政委同志,您好像是在对我进行审问呀?”

“你们的人有被俘的吗?”

“没有,没有人被俘。”

“那么,你们的女报话员哪儿去啦?”

格列科夫咬了咬嘴唇,两道眉毛皱到了一起,他回答说:

“那个姑娘是德国间谍,她发展了我,后来我把她强奸了,后来我又把她枪杀了。”他伸直脖子,问道:“您是要我这样回答吗?”又用冷笑的口吻说:“我看出来,这儿有惩戒营的气味了,是这样吗,首长同志?”

克雷莫夫一声不响地看了他一会儿,说:

“格列科夫呀,格列科夫,您的头发昏啦。我也被围困过,当时也受过询问。”

他看了看格列科夫,慢慢地说:

“我奉上级的指示,必要时解除您的指挥职务,亲自指挥这批人员。干吗您自己要往叉子上闯,非要我走这一步呢?”

格列科夫没有说话,想了想,侧耳听了听,然后说:

“没有声音了。德国佬停止进攻了。”

二十一

“那好吧,咱们两个人坐一会儿,”克雷莫夫说,“研究一下情况。”

“干吗要两个人坐坐,”格列科夫说,“我们这儿打仗都是大家一块儿,研究情况也是大家在一块儿。”

克雷莫夫很喜欢格列科夫的粗鲁,但同时又很生气。他很想对格列科夫说说在乌克兰被围困的事,说说自己在战前的情形,使格列科夫不把他看成官僚。但是他觉得,说这类的事,就表示自己软弱。他到这座楼里来是表现自己的力量的,不是表现软弱。他本来就不是政治部门的官僚,他是作战政委。

他在心里说:“没什么,政委又不丢脸。”

在一片寂静中,大家在砖堆上坐下来或半躺下来。

格列科夫说:“今天德国佬不会再来了。”他向克雷莫夫建议说:“政委同志,咱们来吃点儿东西吧。”

克雷莫夫和格列科夫一起在休息的人们当中蹲下来。

“我看着你们大家,”克雷莫夫说,“脑子里有一个想法老是转悠着:俄罗斯人总能打败普鲁士人。”

有一个不高的、懒洋洋的声音应声说:“是嘛!”

在这一声“是嘛”中,流露出很明显的对这种陈词滥调的勉强附和与嘲笑的意味,所以大家一齐轻轻笑了起来。他们比那个第一次说出“俄罗斯人总能打败普鲁士人”的人更了解,俄罗斯人消耗着多大的力量,而他们本身就是这种力量的直接代表。而且他们也知道和明白,普鲁士人打到伏尔加河边,打到斯大林格勒,完全不是因为俄罗斯人总能打败他们。

这时候克雷莫夫发生了奇怪的变化。他一向不喜欢政治工作人员颂扬俄罗斯古代将领,他的革命的心灵十分厌恶《红星报》社论中摘引德拉戈米罗夫[6]的话,他认为没有必要以苏沃洛夫[7]、库图佐夫[8]和博赫丹·赫梅利尼茨基[9]的名义设立勋章。革命就是革命,革命的队伍只需要一面旗帜,那就是红旗。可是为什么偏偏就在今天,在他重新呼吸到往日列宁主义的革命空气的时候,却出现了这种感触和想法?

一名士兵用嘲笑的、懒洋洋的语气说的那一声“是嘛”刺得他很疼。

“同志们,怎样打仗,用不着教导你们,”克雷莫夫说,“在这方面,你们可以教导任何人。可是,前总指挥部为什么认为有必要派我上你们这儿来呢?或者说,我上你们这儿来干什么呢?”

“是来喝菜汤,为了喝菜汤吧?”有一个人很亲热地小声推测说。

但是听众迎接这小声推测的笑声就不小了。克雷莫夫看了看格列科夫。

格列科夫和大家一起在笑。

“同志们!”克雷莫夫说。他气得两边腮都红了。“同志们,严肃点儿,我是党派到你们这儿来的。”

这是怎么回事儿?是偶然出现的情绪,还是造反?是不是因为觉得自己有本事、有经验,不愿听政委的?也许,听众的开心没有任何反叛的意味,只是因为感觉到真正的平等,这种感觉在斯大林格勒是很强烈的。

可是为什么以前克雷莫夫很赞赏的这种真正平等的感觉现在却引起他的气愤,他要把它压下去,打下去呢?

克雷莫夫在这里同这些人的关系不融洽,不是因为他们受压抑、张皇失措、胆怯。这儿的人感觉自己是强者,是有信心的,难道他们这种强者的感觉影响他们和政委克雷莫夫的关系,引起他和他们之间的疏远和仇视?

烙饼子的那个老头子说:

“我早就想问问党里的人。政委同志,听说,到了共产主义社会,大家都各取所需,那么,如果每个人都按照需要,一天喝到晚,可怎么办呀?”

克雷莫夫朝老头子转过脸去,看到他脸上一副真正担心的神气。可是格列科夫在笑,他的眼睛也在笑,大大的鼻孔笑得更大了。头上缠着血糊糊的肮脏绷带的一名工兵问道:

“政委同志,集体农庄怎么办?战后最好把集体农庄取消。”

“这个报告题目倒是不坏。”格列科夫说。

“我到你们这里不是来作报告的,”克雷莫夫说,“我是作战政委,我到这里来,为的是消灭你们的严重的游击习气。”

“那您就来消灭消灭吧,”格列科夫说,“可是,谁又来消灭德国佬呢?”

“会有人的,不用您操心。我不是为喝汤来的,不像你们说的那样,我是来让你们尝尝布尔什维克的饭的。”

“好吧,您就来消灭消灭,”格列科夫说,“来让我们尝尝吧。”

克雷莫夫一面笑着,同时又很严肃地说:

“如有必要,格列科夫,我们连您一起吃下去。”

这会儿克雷莫夫镇定了,有信心了。原来拿不定主意,不知道怎样办最正确,这会儿主意拿定了。应该解除格列科夫的指挥职务。

克雷莫夫现在已经清楚地看出格列科夫的敌对思想和异己思想,发生在被困楼房里的英雄事迹既不能减弱,更不能消除这种思想。他知道,他能制服格列科夫。

等到天完全黑下来,克雷莫夫走到楼长跟前,说:

“格列科夫,咱们来认真地、开诚布公地谈一谈。您想要什么?”

格列科夫很快地、从下面朝上(他坐着,克雷莫夫站着)看了看他,快活地说:

“我想要自由,我就是为自由作战。”

“我们都要自由。”

“算了吧,”格列科夫把手一甩,“你们要自由干什么?你们只要能打败德国佬就行了。”

“格列科夫同志,不要开玩笑,”克雷莫夫说,“有的战士说出不正确的政治主张,您为什么不制止呢?嗯?您有威信,您可以制止,不次于任何一个政委。可是我有一种印象,大家一面说怪话,一面看着您,似乎在等待您的赞许。那个说到集体农庄的战士就是这样。您为什么要支持他呢?我干脆了当地告诉您:咱们一起来把这种情形整顿整顿吧。如果您不愿意,我也干脆地告诉您:我不会开玩笑的。”

“说说集体农庄,这有什么?实际上,没人喜欢集体农庄吧,这一点您也不是不知道。”

“您怎么,格列科夫,想改变历史的进程吗?”

“您想把一切拉回老的轨道上去吗?”

“‘一切’是什么意思?”

“就是一切。全面的强制劳动。”

他用懒懒的口吻说着,毫不客气,一面冷笑着。他忽然欠起身来,说:

“政委同志,算啦。我什么也没有想。我是随便说说,逗逗您。我是和您一样的苏联人。不相信我,我可要生气啦。”

“那咱们别开玩笑,格列科夫,咱们来认真谈谈,如何克服这种不好的、不是苏联人应有的游击情绪。这是您滋生出来的,您帮助我把它消灭吧。您还要光荣地进行战斗呀。”

“我很想睡觉。您也该休息了。您会看到,天一亮就睡不成了。”

“好吧,格列科夫,那就明天谈吧。我反正又不想离开你们这儿,我哪儿也不去。”

格列科夫大笑起来:

“看样子,咱们能谈得好。”

“情况很清楚了,”克雷莫夫想道,“我不能用顺势疗法。我要用手术刀。政治上的驼背靠劝说是不能抻直的。”

格列科夫忽然说:

“您的眼睛很深沉。您很苦恼。”

克雷莫夫因为感到意外,把两手一摊,什么也没有说。可是格列科夫好像听到了对方承认他的话,就又说:

“您要知道,我也有苦恼。不过这算不了什么,是个人的事。这种事儿在报告里也是不值得写的。”

夜里,在睡着了的时候,克雷莫夫被一颗流弹打伤了头部。子弹打掉一块头皮,在颅骨上划了一下。伤势不重,但是头晕得厉害,克雷莫夫站不住了。老是想呕吐。

格列科夫吩咐准备担架,就在黎明前的寂静时刻,把受伤的克雷莫夫送出了被围困的楼房。克雷莫夫躺在担架上,头又发晕又嗡嗡作响,鬓角咚咚地响,一阵阵地刺痛。

格列科夫把担架送到地道口。

“政委同志,您真不走运。”他说。

克雷莫夫脑子里忽然出现了一种猜想:“是不是格列科夫夜里朝他开的枪?”

快到黄昏时候,克雷莫夫开始呕吐,头疼加剧了。

他在师部卫生营里躺了两天,然后被转送到左岸,住进集团军野战医院。

二十二

团政委皮沃瓦罗夫来到卫生营狭小的地下室里,看到情况很不好—伤员们都横七竖八地躺着。他在卫生营里没有见到克雷莫夫,昨天夜里把他送到左岸去了。

“他怎么一去就受伤了呢?”皮沃瓦罗夫想道。“也许是他不走运,也许是他走运。”

皮沃瓦罗夫同时很想做个决定,该不该把生病的团长送进卫生营。他好不容易回到团部掩蔽所(他在路上差一点被德军的迫击炮打死),对士兵格鲁什科夫说,卫生营里没有任何条件为病人治病。到处是成堆的血糊糊的纱布、绷带、棉花,走到跟前都害怕。格鲁什科夫听到政委这样说,就说:

“当然嘛,政委同志,在自己的掩蔽所里总要好些。”

“是啊,”政委点头说,“在那儿简直就分不清,谁是团长,谁是士兵,大家都躺在地上。”

于是,按军衔应该躺在地上的格鲁什科夫说:“是啊,这怎么像话呀。”

“团长说什么了吗?”皮沃瓦罗夫问。

“没有,”格鲁什科夫摇了摇手,“政委同志,他哪儿还能说什么,给他送去妻子的来信,信还放在那儿,他连看也没看。”

“你说什么?”皮沃瓦罗夫说。“他病成这样啦!连信也不看,这事儿真可怕。”

他把信拿起来,在手里掂量掂量,把信拿到别廖兹金面前,一本正经地用提醒的口吻说:

“别廖兹金同志,您的夫人来的信。”

等了一会儿,又换了另外一种口气说:

“老兄,这是你妻子的信呀,你难道不明白吗,嗯?”

但是别廖兹金没有明白。

他的脸通红通红的,玻璃球似的眼睛亮晶晶地、茫然地望着皮沃瓦罗夫。

这一天,战争带着一股顽强的劲头撞击着生病的团长的掩蔽所。从夜里起,几乎所有的电话联系都中断了,偏偏别廖兹金掩蔽所里的电话一直很正常,各处都通过这条线打来电话:接通师部,接通集团军司令部作战科,和古尔耶夫师的一位团长通话,还有别廖兹金手下的营长鲍丘法罗夫和德尔金。掩蔽所里一直有人来来往往,门不停地吱扭着,格鲁什科夫挂在门口的帆布不停地呼呼啦响。从清早起,人们就惶惶不安,等待着。这一天与往常不同,大炮懒洋洋地发射着,飞机稀稀拉拉、漫无目的地胡乱扔着炸弹,正因为这样,很多人产生了极其苦恼的认识,认定德国人要发动突击了。这一苦恼的认识同样折磨着崔可夫和团政委皮沃瓦罗夫,同样折磨着“6—1”楼房里的人,折磨着一大早就在斯大林格勒拖拉机厂烟囱旁边喝酒为自己过生日的一名步兵排排长。

每次在别廖兹金的掩蔽所里谈起有趣的事或者特别可笑的事的时候,大家都要回头看看团长:难道他连这都听不见吗?

连长赫连诺夫因为夜里伤了风,用沙哑的嗓子对皮沃瓦罗夫说,黎明前他从他的地下指挥所里走出来,蹲在石头上,听听德国佬有没有什么动静。忽然空中响起又生气又发狠的声音:

“唉,赫连[10],怎么连灯也不点?”

赫连诺夫愣了一会儿:这是谁在天上唤他呀?他害怕了。后来才弄清楚,这是小飞机飞行员关了马达,在头顶上滑翔,看样子是想给“6—1”楼房空投食品,看到没亮出标志就生气了。

在掩蔽所里的人都回头看了看别廖兹金,看他是不是笑了。但是只有格鲁什科夫觉得,在病人那像玻璃球一样发亮的眼睛里似乎出现了一点生气。吃午饭的时间到了,掩蔽所里空了。别廖兹金静静地躺着,格鲁什科夫在叹气:别廖兹金躺在那里,旁边就是盼了很久的信。皮沃瓦罗夫和接替已牺牲的科申科夫的新的少校参谋长去吃饭了,喝美味的甜菜汤和好酒。

炊事员已经请格鲁什科夫喝过这种很好喝的甜菜汤了。可是当家的团长却什么也不吃,只是用茶缸喂他几口水……

格鲁什科夫打开信,径直走到床边,清清楚楚地、慢慢地低声念道:

“你好,我的亲爱的万尼亚,你好,我的心肝儿,你好,我心爱的。”

格鲁什科夫皱起眉头,继续念信上的话。

他为昏迷中的团长念妻子的信。已经由军事检查机关检查人员看过的这封信充满柔情蜜意,充满惆怅之情。这信世界上只有一个人有资格看,那就是别廖兹金。

当别廖兹金转过头来并且说“给我”,又伸过手来的时候,格鲁什科夫并没有觉得十分惊讶。

信上一行行的字在哆嗦着的粗大的手指头中间哆嗦着:

“……万尼亚,这里很美,万尼亚,太想念你了。柳芭老是问,为什么爸爸不和我们在一起。我们住在湖边,房子里很暖和,房东有奶牛,有奶喝。我们有你寄来的钱。我早晨出门去,寒冷的水里漂着黄的、红的枫叶,周围已经到处是雪了,显得水特别蓝,天也特别蓝,树叶黄的格外黄,红的格外红。柳芭还问:你为什么哭?万尼亚,万尼亚,我的亲爱的,谢谢你,因为你的一切,谢谢你,因为你的一切,一切,因为你的善良。我为什么哭,怎么解释呢?我哭,因为我活着。我哭,因为斯拉瓦不在了,我却活着,很难受。我哭,因为你活着,我很幸福。我哭,因为我想起妈妈和姐妹们。我哭,因为我看到了早晨的阳光,因为周围这样美,而我和所有的人都这样痛苦。万尼亚,万尼亚,我的亲爱的,我心爱的……”

头脑一个劲儿在打转,周围一切都在打转,手指在哆嗦,信和灼热的空气一起在哆嗦。

“格鲁什科夫,”别廖兹金说,“今天一定要给我治好(塔玛拉可不希望他生病)。怎么样,开水炉子没有打坏吧?”

“开水炉子好好儿的。一天怎么能给您治好呀?您发烧有四十度,一下子怎么能好起来?”

几名士兵轰隆轰隆地把一个空汽油桶滚进了掩蔽所里。往桶里倒了半桶热腾腾的浑浊的河水。水是用锅子和帆布桶往里倒的。格鲁什科夫帮别廖兹金脱光衣服,把他扶到桶边。

“中校同志,太烫啦,”格鲁什科夫摸了摸桶外面,马上把手抽回来,说,“会把您烫坏的。我叫过政委同志,他在师长那儿开会呢,咱们最好等政委同志来。”

“等他干什么?”

“如果您出什么事儿,我就自杀。我也许自个儿下不了手,那就请政委皮沃瓦罗夫同志向我开枪。”

“来,帮我下去。”

“请原谅,至少我要把参谋长叫来。”

“嗯。”别廖兹金说。虽然这一声又短又沙哑的“嗯”出自一个脱得光光的、勉强站得住的人之口,但是格鲁什科夫不再犟了。别廖兹金爬进水里之后,哼哼起来,又哎哟又乱动,格鲁什科夫看着他,也哼哼起来,围着桶转起圈子。

“就像在产科医院里啦。”不知为什么他这样想道。

别廖兹金昏迷了一会儿,军事上的担心和生病的发烧在迷糊状态中搅在了一起。忽然心不动了,不乱跳了,滚烫的水也不那样烫得难受。后来他清醒过来,对格鲁什科夫说:

“要把地上的水扫一扫。”

但是格鲁什科夫没有看到桶里的水漫出来。团长通红的脸开始变白了,嘴半张开,剃得光光的头上冒出老大的汗珠子,格鲁什科夫觉得汗珠子好像是蓝色的。别廖兹金又开始昏迷,但是等格鲁什科夫试图把他拖出来时,他清清楚楚地说:

“还不到时候。”

他咳嗽起来。等到一阵咳嗽过去,别廖兹金不等喘过气来就说:

“再加一些开水。”

他终于从水里爬了出来。格鲁什科夫看着他,心里非常不是滋味。他帮别廖兹金擦干身子,躺到床上,盖上被子和军大衣,然后又把掩蔽所里所有的一切破旧的东西,如雨衣、棉袄、棉裤,全都盖上去。

等到皮沃瓦罗夫回来,掩蔽所里已经收拾好了。只是空气中还有湿乎乎的像澡堂里的气味。别廖兹金静静地躺着,睡着了。皮沃瓦罗夫在他身边站了一会儿。

“他的脸色很好,”皮沃瓦罗夫想道,“他倒是没写过揭发材料。”

这一整天他惴惴不安,因为他想起他在五年前揭发过和他一起上过两年大学的同学什梅廖夫。今天,出现了这种不祥的、使人苦恼难受的寂静状态的时候,什么样乱七八糟的事都浮现在头脑里,什梅廖夫也浮现在头脑里,他仿佛看到:什梅廖夫脸上带着又可怜又痛苦的表情,侧眼望着,听着大会上宣读他的好朋友皮沃瓦罗夫写的揭发材料。

夜里十二点左右,崔可夫打来电话,没有通过师长,而是直接打到驻守在拖拉机厂的团里,因为他很为这个团担心:侦察队多次报告,说德军的坦克和步兵一个劲儿往这一地区集中。

“喂,你们那里怎么样?”他很焦急地说。“你们团究竟是谁在指挥?巴秋克告诉我,说团长害了什么肺炎,要把他送到左岸去。”

一个沙哑的声音回答说:

“这个团是我在指挥,我是别廖兹金中校。是有一点儿伤风,不过现在好了。”

“我听到啦,”崔可夫好像有些幸灾乐祸地说,“你沙哑得厉害呢,德国佬就要给你喝点儿热牛奶啦,已经准备好了,你要注意,他们就要给你来一下子啦。”

“懂了,一号同志。”别廖兹金说。

“啊,懂啦,”崔可夫带着吓唬口吻说,“那你就注意,如果想后退,那我就给你糖拌生蛋黄,不比德国佬的牛奶差!”

二十三

波里亚科夫和克里莫夫约好夜里要去一趟团部,老头子想打听一下谢廖沙的下落。波里亚科夫把自己的想法对格列科夫说了说,格列科夫很高兴。

“快去吧,快去吧,老爹,你到后方可以多少休息一下,还可以对我们说说他们在那儿怎么样。”

“是说卡佳怎么样吧?”波里亚科夫猜到格列科夫为什么赞成他的想法,就问道。

“他们已经不在团里了,”克里莫夫说,“我听说,团长派他们上伏尔加河那边去了。他们大概已经在阿赫图巴户口登记处登记了。”

波里亚科夫是一个不肯饶人的老头子,他就问格列科夫:

“要是这样的话,是不是就不让我们去啦,或者您写信去?”

格列科夫很快地看了他一眼,但是很平静地说:

“好啦,去吧。已经说过了嘛。”

“当然啦。”波里亚科夫在心里说。

早晨四点多钟,他们顺着地道爬去。波里亚科夫的头时不时碰到支架上,不时地骂谢廖沙两句,他又生气又觉得不好意思,因为他竟想念起这个小伙子。

地道宽一些了,他们坐下来多少休息一下。克里莫夫笑着说:

“你怎么不带点儿礼物呀?”

“去他的吧,乳臭未干的孩子,”波里亚科夫说,“要带就带一块砖头,敲他几下子。”

“当然啦,”克里莫夫说,“你就是为这去的嘛,还准备过河到那边去呢。也许,老人家,你是想看看卡佳吧。吃醋了吧?”

“走吧。”波里亚科夫说。

不多一会儿,他们就来到地面上,走在没有人的地段,四周静悄悄的。

“是不是仗打完啦?”波里亚科夫想道。他马上清清楚楚地想象自家的屋:桌上摆了一碟子热汤,老伴儿在刮他钓来的鱼。他都觉得身上发热了。

就是这天夜里,保卢斯将军发出向斯大林格勒拖拉机厂地区进攻的命令。

两个步兵师要进入空袭、炮轰和坦克冲击过的大门。从半夜起,香烟卷的红色火光就在士兵们无所事事的手里晃动着了。

在黎明前一个半小时,“容克”轰炸机的马达声在工厂各车间的上空响了起来。轰炸开始之后,就没有停顿和休歇了。如果在这连成一片的轰隆声中还有短暂的间隙的话,那这间隙里也充满了炸弹的呼啸声,一颗颗炸弹正拼足了自己沉重的钢铁力量朝地上冲。这连成一片的轰隆声似乎能和钢铁一样,敲碎人的头颅,打断人的脊梁骨。

天开始放亮了,可是工厂区上空依然黑沉沉的。

似乎大地自动在喷射电光、轰隆声、硝烟和黑色灰尘。

尤为强大的攻击对准了别廖兹金团和“6—1”号楼房。

在整个团的防地上,被震聋了的人们都像发疯似的跳起来,明白了这是德国佬开始了新的、空前强大的杀人勾当。

克里莫夫和老头子遇到了轰炸,便连忙朝无人地段奔去,在九月末重磅炸弹在那儿炸了不少大坑。朝无人地段跑的还有刚刚从轰塌的战壕里跳出来的鲍丘法罗夫营的战士。

德军战壕与苏军战壕之间的距离很近,所以一部分炸弹落到德军前沿阵地上,炸死炸伤德军打头进攻的一个师的部分士兵。

波里亚科夫觉得好像是从下游阿斯特拉罕来的风在波涛汹涌的伏尔加河上呼啸。他有好几次被气浪冲倒,他在倒下的时候,忘记了他是在阳间还是阴间,忘记了他是年老还是年轻,忘记了哪儿是上,哪儿是下。但是克里莫夫一直拉着他走—快点,快点!他们终于倒进一个深坑里,滚到潮漉漉、黏糊糊的坑底。这儿有三重黑暗,就是说,这黑暗是由夜的黑暗、硝烟和尘土的黑暗和深坑的黑暗交织成的。

他们躺在一起,这年老的和年轻的脑子里都留着一线希望的光,活命的祈求。这种微光,这种感人的祈求不仅燃烧在所有人的脑子里和心里,而且也燃烧在鸟兽的最简单的心里。

波里亚科夫小声骂着娘,认为一切灾难全是谢廖沙招来的,嘴里嘟哝着“搞成这样都怪谢廖沙”,可内心里仍然在为他祈祷。

这种连成一片的爆炸不可能持续很久,因为已经是超负荷的了。但是时间分分秒秒过去,强烈的轰隆声依然没有减弱,黑黑的烟幕依然没有放亮,而是越来越浓,天和地更加混沌了。

克里莫夫摸了摸波里亚科夫的粗糙的干活儿的手,握了握,他的手动了动,那是善意的回答,这对于处在未埋土的坟墓里的克里莫夫是一种暂时的安慰。近处的爆炸把土块和碎石甩进坑里来;碎砖块打在老头子的背上。等到一片片的土从坑壁上往下溜,他们就感到恶心起来。坑已经不像坑了,而且再也看不见光了,德国人把一切从天上往下撒,要把周围填平。

克里莫夫平常在侦察的时候,不喜欢有搭档,喜欢快点儿溜进黑暗中去,就像冷静而老练的游泳者喜欢快点儿离开岸边岩石,泅进辽阔的大海黑郁郁的深处。然而在这土坑里,他却很高兴有波里亚科夫躺在一起。

时间不再均匀地前进,而是疯狂起来,像爆炸的气浪一样朝前冲,有时忽然凝冻起来,被卷成了羊角形。

但是终于坑里的人抬起头来,头顶上出现了模模糊糊的亮光,硝烟和灰尘渐渐被风吹散……大地安静下来,连成一片的轰隆声变成零零落落的爆炸声。令人感到苦闷、疲惫,似乎心里的一切生命力都被挤压光了,只剩下愁闷。

克里莫夫欠起身来,在他旁边躺着的竟是一个德国兵,身上盖了一层灰土,从帽子到靴子,浑身都被战争磨破、咬烂了。克里莫夫一向不怕德国人,他一向相信自己的力量,相信自己有本事神出鬼没地抢在敌人之前一秒钟扣响扳机,扔出手榴弹,用刺刀捅出去或者用枪托子打过去。

可是现在他茫然失措了,他吃惊的是,在听不见也看不见的时候,他感觉到这个德国兵在旁边竟因此得到安慰,他竟把德国兵的手当成波里亚科夫的手。他们互相对望着。他们被同样一种力量控制着,无法摆脱这一力量。这一力量不保护他们中任何一个,而是同样威胁着两个人。

这两个战场上的敌手都没有作声。

他们所具有的准确无误的机械性能—杀人,没有发挥出来。

波里亚科夫坐在稍远些的地方,也在看着满脸胡茬的德国兵。尽管波里亚科夫不喜欢长时间不说话,可是这会儿也没有说话。

活着是可怕的。他们的眼睛深处闪现出一股沮丧的洞察力,仿佛看到:战争过去,那股驱使他们来到这坑里、让他们趴在泥地上的力量,还会在那儿等着他们,不管是战败者,还是战胜者。

他们就像商量好了一样,从坑里往外爬,尽管自己的脊背和脑壳很容易受到枪击,但是都毫不犹豫地相信自己没有危险。

波里亚科夫直往下滑,但是在旁边爬的德国兵没有帮助他,老头子滚了下去,一面咒骂着天和地,可是又仍然顽强地朝地面上爬。克里莫夫和那个德国兵爬到地面上,两个人都望了望,一个朝东面望,一个朝西面望:上级是不是看到他们从一个坑里爬出来,谁也没有打死谁。他们都没有回头,各自朝自己的战壕走去,跨过被炸翻过来、还在冒烟的土地上的一个个土包和一道道沟坎。

“咱们的大楼没有了,炸平了!”克里莫夫恐怖地对跟上来的波里亚科夫说。“弟兄们,难道你们都死了吗?”

这时候,大炮和机枪响了起来,呼啸声,咆哮声。德军发动了强大的攻势。这是斯大林格勒最沉重的一天。

“都是浑小子谢廖沙搞的。”波里亚科夫嘟哝说。他还不明白是怎么一回事儿,不明白“6—1”大楼里的人已经全部牺牲了,他看到克里莫夫在抽搭,在哀叹,还生气呢。

二十四

在飞机轰炸的时候,一颗炸弹落在营指挥所所在的地下煤气管道的检修处上面,把此刻正在里面的团长别廖兹金、营长德尔金和营里的报话员埋住。别廖兹金处在一片漆黑中,耳朵也被震聋了,被石头粉灰呛得喘不上气来,起初他以为自己已经完了,但是德尔金在短暂的寂静时刻里打了一个喷嚏,问:

“中校同志,您活着吗?”

别廖兹金就回答说:

“活着。”

德尔金听到团长的声音,高兴起来,多年来没有离开过他的好情绪马上又回到他心中。

“既然活着,那就是一切情况正常。”他说,虽然他被灰土呛得喘不过气来,咳嗽着往外吐唾沫,显然情况并不怎么正常。德尔金和报话员被碎石头埋住,还不知道骨头断了没有,因为无法动弹知觉还没有恢复。一根铁梁悬在他们的头上,使他们直不起腰来,但是,看样子,正是这根铁梁救了他们。德尔金拧亮了手电筒,他才真正害怕起来。在一片灰尘中,一块块石头、压弯的铁梁、鼓起来的抹了润滑油的混凝土、炸碎的电缆都悬在头顶上。看样子,只要再有爆炸一震动,铁和石头合拢来,这狭窄的空隙就不存在了,几个人也就没有了。

他们安静了一阵子,缩着身子,一种疯狂的力量冲打着一个个车间。别廖兹金心想,这些车间在以自己僵死的躯体参加保卫战呢,因为要打碎混凝土和钢筋是很难的。

后来他们到处敲敲碰碰,摸索着,就明白了,要自己爬出去是不可能的。电话机还好好的,但是哑了,因为电话线被炸断了。

他们彼此几乎不能说话,因为爆炸的轰隆声掩盖了他们的声音,他们被灰尘呛得直咳嗽。

前一天还在发高烧的别廖兹金,现在并不觉得没有力气。他的力量在战斗中往往能带动指挥人员,带动战士们,不过这力量的实质不是军事性与战斗性的,这是一种通情达理的人性的力量。能保持这种力量并且能够在残酷的战斗中表现出这种力量的,只有很少一些人,正是这些人,这些平易近人、通人情、有理性的人,才是战争的真正主人。

但是轰炸停止了,被埋住的几个人又听到钢铁的隆隆响声。别廖兹金揩了揩鼻子,咳嗽了几声,说:

“狼群叫起来了,坦克朝拖拉机厂冲来了。”又补充说:“咱们正好在他们的路上。”

也许由于彻底绝望了,德尔金忽然用难以形容的嗓门儿大声唱了起来,一面咳嗽,一面唱起电影歌曲:

嘿伙计们,活着就好,活着就好,

跟头领在一起咱们不用烦恼……

报话员心想,营长准是疯了,可是他也一面咳嗽一面吐,跟着唱了起来:

老婆会伤心,会嫁给别人,

一嫁给别人,就把我忘了……

这时候在地面上,在充满了硝烟、灰尘和坦克吼声的隆隆作响的车间废墟上,格鲁什科夫不顾血糊糊的手上的皮肉,拼命地扒石头、混凝土块、断钢筋,他用一股疯狂的劲头干着,正是这股疯劲儿帮助他扭动沉重的铁梁,干几十个人才能干的事情。

别廖兹金又看到了带有硝烟与尘土的朦胧的光线,这光线中还混杂着爆炸声、德军坦克的吼声、大炮声与机枪声。不管怎么说,那是一种微弱的亮光了,别廖兹金一看到这亮光,首先就在心里说:“你瞧,塔玛拉,你不该为我担心嘛,我对你说过,没有什么了不起的。”格鲁什科夫一双强壮有力的手把他抱住。

德尔金用号哭的声音叫道:

“团长同志,向您报告,我的一个营全完了!”

他用手朝周围指了指。

“万尼亚死了!我们的万尼亚死了!”

他指了指侧着身子躺在黑色的血泊与机油中的营政委的尸体。团指挥所倒是比较平安,只有桌子和床上撒了一层土。

皮沃瓦罗夫一看见别廖兹金,就高兴得骂起娘来,并且跑了过来。

别廖兹金就问起来:

“和各营有联系吗?被围的大楼怎样了?鲍丘法罗夫怎么样?我刚才和德尔金就像落进老鼠夹子里,不见光,也没有联系。谁活着,谁死了,我们的人在哪儿,德国佬在哪儿,我一点儿也不知道,快把情况说一说!你们在战斗的时候,我们还在那儿唱着歌。”

皮沃瓦罗夫说起伤亡情况,说“6—1”大楼里的人都完了,全牺牲了,那个好捣乱的格列科夫也死了,只活下来两个人,一名侦察兵和一个民兵老头子。

但是这个团经住了德军的打击,活下来的人还活着。

这时候电话机发出声音,团部里的人看了看电话员,从他的脸色看出来,这是斯大林格勒最高指挥官打来的电话。

电话员把话筒递给别廖兹金,听得很清楚,掩蔽所里安静下来的人都听出了崔可夫那粗大而低沉的声音:

“是别廖兹金吗?你们的师长负伤了,副师长和参谋长都牺牲了,我命令您担任师长职务。”

稍停之后他用又慢又重的声音说:

“你在空前艰难、危险的情况下率领全团作战,顶住了进攻。谢谢你。好同志,我拥抱你。祝你成功。”

在拖拉机厂各车间里的战斗开始了。活着的人还活着。

“6—1”楼房无声无息。再也听不到从瓦砾堆里打出来的枪声。显然,空袭的主要力量对准了这座楼房,断垣残壁倒塌了,石头堆被扫平了。德军坦克借这座破楼的瓦砾堆做掩护,向鲍丘法罗夫营开了火。

不久前还在残酷无情地打击德军,使德国人感到害怕这座楼的废墟,如今却成了他们的安全地带。

从远处看,那一个个红红的砖堆很像是一块块老大的冒热气的生肉,身穿灰绿军服的德国兵嚷叫着,很起劲地在被摧毁的楼房的砖堆中间跑来跑去。

“你指挥这个团吧。”别廖兹金对皮沃瓦罗夫说。又说:“整个战争期间上级都对我很不满意。可是现在,我在地下闲待了一阵子,又唱了歌儿,可是你瞧,又得到崔可夫的感谢,又捞到师长头衔,这可不是玩儿的。现在我可是不能放过你。”

但是德国佬冲过来了,没工夫开玩笑了。

二十五

在寒冷的下雪的日子里,维克托带着妻子和女儿来到莫斯科。弗拉基米罗芙娜不愿意厂里的化验工作中断,就留在了喀山,虽然维克托已经在奔走,设法把她安置在卡尔波夫研究院。

这些天是很奇怪的—心里又高兴又惶惶不安。似乎德国人依然很可怕,很强大,他们正准备新的猛烈的进攻。

战争似乎还未见转机。但是人们想回莫斯科已经是自然而然的事了,政府开始组织一些单位复员回莫斯科,也是合乎情理的了。

人们已经隐约感觉出战争的春天即将到来的信息。不过,首都在战争的第二个冬天里依然显得冷清,凄凉。

人行道上肮脏的雪堆像一座座小山。郊区的街巷间,一条条小道像乡间小径一样,连接着从居民家门口到电车站与商店的通路。很多窗子里伸出冒烟的罗马尼亚式铁烟囱,墙上覆盖了一层熏得黄黄的冰凌。

身穿小皮袄、头上裹围巾的莫斯科人显得很土气,很像乡下人。

在从车站回家的路上,维克托坐在货车车厢里的行李上,打量着坐在旁边的娜佳阴沉的脸,问道:

“怎么,小姐,你在喀山想象的莫斯科不是这种样子吧?”

娜佳因为爸爸摸到了她的心思,很生气,就什么也没有回答。

维克托就给她讲解起来:

“人类不懂得,他们建起的城市并不是大自然本来就有的一部分。人类为了保护文明,必须驱除野狼,清除冰雪,铲除杂草,因此就不能放下武器、铁锹和扫帚。如果他们马虎大意,闲散一两年,那可就糟了,野狼会从森林里跑出来,杂草到处生长,城市会被冰雪堵塞,到处是灰尘。已经有多少大城市被尘土、积雪和荒草淹没了啊。”

维克托很希望跟捞外快的司机一起坐在驾驶室里的柳德米拉也能听到他的高论,就把身子探到车厢拦板外面,对着开了一半的小窗孔问道:

“柳德米拉,你坐得舒服吗?”

娜佳说:

“不过是扫院子的人没有扫雪,这跟毁灭文化有什么关系?”

“你这傻孩子,”维克托说,“你看看这一堆堆的冰。”

汽车很猛烈地颠簸了一下,车厢里所有的箱子和包裹一下子蹦了起来,维克托和娜佳也跟着蹦了一下。他们对看了一下,笑了起来。

奇怪,很奇怪。他何曾想到,在战争的痛苦年月里,在喀山逃难的时候,他会取得他最大、最重要的成就?

他们在进入莫斯科的时候,似乎只能感到得意和兴奋,也许只有怀念安娜·谢苗诺芙娜、托里亚、玛露霞,怀念几乎每个家庭都有的牺牲者的痛苦心情,会和归来的喜悦心情交织在一起,填满人的心灵。

然而,一切并不像想象的那样。在火车里,维克托常常因为一些小事发火。他生气的是,柳德米拉老是睡觉,也不看看窗外她儿子保卫过的土地。她在睡梦中大声打呼噜。一名伤兵从车厢里走过,听到她的呼噜声,说:“哎哟,打得真带劲儿!”

他很生娜佳的气:妈妈专拣她吃剩的东西吃,她也就毫不客气地在包里挑选烤得最好的饼子。在火车里她学会了对爸爸使用一种戏弄和嘲笑的腔调。维克托听到她在旁边一个单间里说:“我爸是个老大的音乐迷,自己也能胡乱弹一弹钢琴。”

同车厢的人谈莫斯科的下水道和暖气设备,谈到那些无忧无虑的人不必按莫斯科的转帐单付钱,无需像没有公房住的人那样付房租,还谈到往莫斯科带什么样的食品比较合算。维克托听到谈生活问题就生气,可是他也谈了房屋管理和自来水问题,在夜里睡不着的时候,他又想到在莫斯科登记领取供应品的问题,又想到电话是不是已经被拆除了。

一个很凶恶的女列车员在打扫车厢的时候,从座位下面扫出维克托扔的一根鸡骨头,就说:

“哼,简直是猪,还自以为是有文化的人呢。”

在穆罗姆,维克托和娜佳在站台上散步,从两个身穿羊羔皮领子大衣的年轻人身边走过。其中一个年轻人说:

“大英雄疏散回来啦。”

另一个解释说:

“大英雄要赶回去领取保卫莫斯科奖章呢。”

在卡纳什车站,火车在迎面开来的一列装运犯人的军车旁边停下来。押车兵在军车旁边走来走去,犯人们将一张张苍白的脸贴在小小的、装了铁栏杆的窗户上,叫喊着:“抽烟……”,“给点儿黄烟吧……”押车兵骂着,把犯人从窗口赶开去。

黄昏时候,维克托走到索科洛夫夫妇所在的车厢里。玛利亚头上裹着花头巾,正在铺床,让丈夫睡下铺,自己睡上铺。她很担心丈夫是不是舒服,维克托问她什么,她回答得牛头不对马嘴,她甚至都没有问柳德米拉身体好不好。

索科洛夫打着呵欠,说是车厢里太闷,弄得他一点精神也没有了。维克托看到索科洛夫没有因为他的到来表示高兴,而是一副心不在焉的样子,不知为什么感到特别生气。

维克托说:

“我这辈子头一次看到,丈夫让妻子爬上铺,自己睡下铺。”

他说这话用的是很气愤的口气,连他自己也觉得奇怪,这种情况为什么使他这样生气。

“我们一直是这样,”玛利亚说,“他在上铺总感到气闷,我倒是无所谓。”她吻了吻索科洛夫的鬓角。

“好啦,我走了。”维克托说。索科洛夫夫妇没有挽留他,他又很生气。

夜里车厢里很闷。想起喀山,想起卡里莫夫、弗拉基米罗芙娜,想起和马季亚罗夫谈的话,想起在大学里的小小的房间……过去维克托上索科洛夫家去,议论政治的时候,玛利亚的眼睛多么亲切,多么动情啊。不像今天在车厢里这样漠然,这样疏远。

“鬼才知道这是怎么一回事儿,自己睡在下面,下面又舒服又凉爽。这算什么道理?”他在心里说。

他一向认为玛利亚在他认识的女人当中是最好的女人,又温柔,又善良。现在他生她的气了,就在心里想道:“就像是一只红鼻子母兔。索科洛夫是一个难以相处的人,又懦弱,又拘谨,同时又自负得不得了,城府很深,又爱记仇。是的,实在够她受的。”

他怎么也睡不着,试着想想即将和朋友们,和契贝任见面的情形—很多人已经知道他的研究成果了嘛。他见到的将是什么样的情形呢?他是胜利归来的啊。古列维奇和契贝任会对他说什么呢?

他想,能够详详细细地掌握新的试验装备性能的马尔科夫再过一个星期才能到莫斯科来,他不来还不能开始工作。糟糕的是,索科洛夫和我都是瘸子:只能动脑子,不能动手……

唉,好一个胜利者,胜利者!

但是这些想法懒懒地接续着,渐渐断了。

他眼前出现了叫喊着“要抽烟”,“给点儿黄烟”的人们,出现了管他叫“大英雄”的两个年轻人。波斯托耶夫当着他的面对索科洛夫说了一句很奇怪的话—索科洛夫说了说年轻物理学家兰杰斯曼的研究情况,波斯托耶夫就说:“兰杰斯曼又算什么,维克托·帕夫洛维奇的第一流发现才真正能震动世界呢。”他把索科洛夫抱住,又说:“不过最主要的还是,咱们是苏联人。”

电话还通吗,煤气还有吗?难道一百多年前的人在躲避拿破仑之后回莫斯科的时候,也想这些乱七八糟的事吗?……

汽车在楼房大门口停下来。于是维克托一家人又看到了自家的一套住房的四个窗户,窗玻璃上还保留着去年夏天贴的蓝色纸条,又看到了大门,看到人行道边的菩提树,看到“牛奶店”的招牌、房管处门上的牌子。

“电梯恐怕还没有开,”柳德米拉说,并且转脸朝着司机问道,“同志,您能不能帮我们把东西送到三楼?”

司机回答说:

“怎么不行,可以。不过,您要给我一些面包,算是脚力。”

把汽车上的东西卸下来,留下娜佳看东西,维克托和妻子朝楼上走去。他们慢慢地朝上走,感到很惊奇,因为一切都没有什么变化,二楼那包了黑漆布的门、那熟悉的邮箱都是老样子。多么奇怪啊,街道、房屋,几乎已经忘记的许多东西都没有消失,这不是,这一切又出现在眼前,人又置身其中了。

有一次,托里亚不愿等电梯,跑上三楼,从上面对着维克托叫喊:“哈,我已经到家了!”

维克托对妻子说:

“咱们在楼梯口歇一会儿,你都喘不上气来了。”

“天啊,”柳德米拉说,“这楼梯脏成什么样子啦。明天我就找房管处,叫瓦西里·伊万诺维奇组织人打扫打扫。”

终于他们夫妻两人站到自己的家门口了。

“也许,你想亲手开开门吧?”维克托问。

“不,不,你开吧,你是户主嘛。”

他们走进房里,没有脱大衣,在各个房间里走了一遍。她用手试了试暖气片,拿起电话筒,吹了吹,说:

“电话还能打通!”

然后她走到厨房里,说:

“也有自来水,这么说,卫生间还能用。”

她走到煤气炉跟前,试了试煤气炉开关,煤气是关着的。

天啊,天啊,一切都还在。敌人被挡住了。他们回到自己家里来了。一九四一年六月二十一日那个星期六,好像就是昨天。好像一切都没变,好像一切都变了!是另外一些人回到家里,他们已经是另外一种心情,另外一种命运,他们生活在另外一个时代。为什么这样心神不宁,这样平淡无味?为什么已经逝去的战前生活显得那样美好,那样幸福?为什么要这样操心明天的事—凭票供应,户口登记,用电限额,电梯开不开,订报纸?……到夜里又可以在自己的床上听熟悉的钟声了。

他跟在妻子后面走着,忽然想起他在夏天来莫斯科的情形,想起和他在一起喝酒的俊俏的尼娜,空酒瓶现在还放在厨房里的水槽旁边呢。

他想起他看过诺维科夫上校带来的妈妈的信之后的那个夜晚,想起自己突然上契里亚宾斯克的情形。他就是在这儿吻尼娜的,她有一只发卡掉下来,他们怎么找也找不到。他心慌起来,担心那只发卡现在出现在地板上,也说不定,尼娜把口红和香粉盒忘在这里了。

但是这时候,司机呼哧呼哧喘着粗气,把箱子放下来,打量了一下房间,问道:

“整个这一套房都是你们家住的吗?”

“是的。”维克托很不好意思地回答说。

“我们家六口人才住八平米呢,”司机说,“我老婆在白天趁大家都去干活儿的时候睡觉,夜里她就在椅子上坐着。”

维克托走到窗前,看到娜佳站在汽车旁堆行李的地方,又蹦又跳,还用嘴呵着手指头。

好娜佳,可怜的女儿,这就是你的家。

司机把装食物的口袋和装被褥的大布袋扛进来,就在椅子上坐下来,卷起烟卷儿。看样子,他当真关心居住问题,一再地和维克托谈起卫生设备和区房管局的人贪污受贿。这时厨房里的锅子响了几声。

“这就烧饭啦。”司机说,并且朝维克托挤了挤眼睛。维克托又朝窗外看了看。

“这就好了,好了,”司机说,“可是等到在斯大林格勒打垮了德国佬,大家都从疏散的地方回来,房子就更不够住了。不久前我们有一个工人受过两次伤以后回到工厂里,不用说,房子被炸毁了,他带着一家人住到没人住的地下室里,老婆怀着孩子,两个孩子都害肺病。地下室里灌进了水,水到了膝盖以上。他们把木板铺在板凳上,从床上到桌子边,从桌子边到炉边,都从木板上走。于是他到处要求解决住房问题,党委会、区委会都找过,也给斯大林写过信。都答应解决,答应只是答应。一天夜里他带上老婆、孩子和破烂东西住进五楼一个房间,是区苏维埃的机动房间。房间有八点四三平方米。这一下子事情闹大了!检察长把他传了去:要么在二十四小时内搬出去,要么判五年徒刑,两个孩子交保育院。这一来,他怎么办?他在战争中得过五颗勋章,现在他把五颗勋章扎在胸膛上,扎进肉里,就在中午休息的时候在车间里上了吊。大伙儿发现了,马上把绳子割断。救护车把他送进医院。这一来,马上给他发了住房证,他目前还在医院里呢,不过总算他走运,房间虽小,可是好歹有了个窝儿。结果还不坏。”

司机刚说完他的故事,娜佳就走了进来。

“要是东西被偷了,谁负责任?”司机问。

娜佳耸了耸肩膀,就一面呵着冻僵的手指头,在几个房间里转悠起来。

娜佳一走进房间来,就惹爸爸生气了。“你哪怕把领子放下来也好。”维克托说。

但是娜佳没有理睬,却朝着厨房叫道:

“妈妈,我饿死啦!”

这一天柳德米拉表现出非凡的精力和干劲儿,维克托简直觉得,她如果把这股劲头儿用在军事上,德国佬一定会从莫斯科后退一百公里。

管道工接通了暖气,管道完全正常,虽然不怎么热。找煤气工人却很不容易。柳德米拉打电话给煤气管道主任,管道主任从抢修队派来一名工人。柳德米拉把所有的煤气炉都点着了,把烙铁放上去,虽然火力不大,但是坐在房里可以不穿大衣了。在司机、管道工、煤气工忙活过一阵子之后,装面包的口袋就轻飘飘的了。

柳德米拉做家务事一直忙到很晚时候。她把破布缠到刷子上,把天花板和墙上的灰土都扫干净了。又把吊灯架上的灰土揩干净了,把干枯了的花拿到黑黑的过道里,清扫出很多垃圾、旧纸、破布;娜佳也一面嘟哝着,帮着提出去三桶脏水。

柳德米拉把厨房和餐室里的家什都洗了一遍,维克托也在她的指挥下擦洗碟子、叉子和刀子,茶具却不放心让他擦洗。她又开始擦洗浴室,在炉子上炼油,挑拣从喀山带来的土豆。

维克托给索科洛夫打了个电话,接电话的是玛利亚,她说:

“我叫他睡了,一路上他很疲乏,不过,如果有什么急事,我把他叫醒。”

“不,不,我没有事,只是想和他聊聊。”维克托说。

“我觉得太幸福啦,”玛利亚说,“一个劲儿想哭呢。”

“上我们家来玩儿吧,”维克托说,“您怎么样,晚上有空吗?”

“今天哪儿行啊,”玛利亚笑着说,“柳德米拉有多少事儿,我也有多少事儿。”

她问了问用电限额和自来水管道方面的事,他忽然很不礼貌地说:

“我马上把柳德米拉叫来,让她来和您谈自来水问题。”马上又故意用开玩笑的口吻说:“您不来,真遗憾,实在遗憾,要不然咱们可以念念福楼拜的长诗《马克斯和莫里茨》了。”

但是她没有理睬他的玩笑,说:

“我等一会儿再给您打电话。柳德米拉收拾房间有多么忙,我也有多么忙。”

维克托明白,她听到他的不礼貌的腔调生气了。他忽然很想上喀山去。

人究竟有多么奇怪啊?维克托打电话找波斯托耶夫,他们家的电话却打不通。他打电话找物理学博士古列维奇,邻居接电话说,古列维奇上索科里尼基妹妹家去了。他打电话找契贝任,却没有人接电话。

忽然电话铃响起来,有一个男孩子的声音要娜佳接电话,但是这时候娜佳倒垃圾桶去了。

“是谁找她?”维克托一本正经地问。

“没要紧事儿,是一个熟人。”

“维克托,别在电话里闲扯吧,来帮我把柜子搬一搬。”柳德米拉喊道。

“我跟谁闲扯?在莫斯科还没人跟我闲扯呢,”维克托说,“你最好还是给我弄点儿吃的。索科洛夫已经吃过饭,睡觉了。”

似乎柳德米拉把家里搞得更乱了:到处堆着衣服,从橱子里拿出来的家什摆在地板上;又是锅子,又是盆,又是口袋,想在各个房间里和走廊里走走,却走不通。

维克托以为柳德米拉开头会有一段时间不上托里亚的屋里去,他估计错了。她的眼里流露着操心的神气,脸红红的,她说:

“维克托,你把这只中国花瓶放到托里亚的屋里,放到书橱上,我洗干净了。”

电话铃又响了,他听到娜佳说:

“你好……我哪儿也没有去,刚才我妈叫我倒垃圾去了。”

柳德米拉催促他说:

“维克托,帮帮我吧,别睡觉,还有这么多事情!”

女人有多么强大的本能,这种本能多么顽强又多么单纯。

到晚上,一切整理就绪了,房间里暖和了,又呈现出战前原有的样子。

晚饭是在厨房里吃的。柳德米拉烙了饼,又用下午烧的米饭当馅做了馅饼。

“刚才是谁给你打电话?”维克托问娜佳。

“噢,是一个男孩子,”娜佳回答说,并且笑了起来,“他打电话已经打了四天了,终于打通了。”

“你怎么,是在和他通信吗?事先告诉他了你要回来吗?”柳德米拉问道。

娜佳气得皱了皱眉头,一个肩膀动了动。

“可是,哪怕有一只狗给我打打电话也好啊。”维克托说。

夜里,维克托醒了。柳德米拉穿着内衣站在开着的托里亚的房间门前说:

“你瞧,我的托里亚,我一下子都收拾好了,你的屋里也收拾好了,就跟没有打仗一样,我的好孩子……”

二十六

复员回来的科学家们汇集在科学院的一座大厅里。

这些人有年老的,有年轻的,有面色苍白的,有秃顶的,有大眼睛的,有眼睛小而锐敏的,有宽额头的,有窄额头的,大家汇集在一起之后,就回味着过去那段生活中曾经存在的那种崇高的诗意,散文的诗意。

长久放在没有生炉子的房子里的发潮的资料和书页,竖起大衣领子做科学报告,用冻僵冻红的手指头抄写公式,用几颗土豆和烂白菜叶子做的莫斯科杂烩汤,拥挤着领饭票,在配给咸鱼和补贴素油的名册上恼人地签上自己的名字—这一切一下子退居次要位置了。老同事见了面,问候声响成一片。

维克托看到契贝任和院士希沙科夫在一起。

“德米特里·佩特罗维奇!德米特里·佩特罗维奇!”维克托看着他的亲热的脸,一连喊了两遍。契贝任把他抱住。

“您的孩子们从前方给您来信吗?”维克托问道。

“他们都很好,来信的,来信的。”

契贝任却没有笑,而是皱起眉头,维克托从他这种神气看出来,他已经知道托里亚牺牲了。

“维克托·帕夫洛维奇,”他说,“请代我向您的夫人表示敬意,衷心的敬意。我的敬意和内人的敬意。”

契贝任接着又说:

“我看过您的论文了,很有意义,很重要,比一般认识到的还要重要。您要知道,其重要性将超过我们现在所能想象到的。”

他吻了吻维克托的额头。

“哪里,哪里,这算不了什么。”维克托说。他觉得又不好意思,又高兴。他来开会的路上,还惴惴不安地想着:有谁看过他的论文,会怎样评价他的论文?要是根本没有人看过呢?

他听了契贝任的话,马上就充满了信心:他和他的论文在这里要成为唯一的话题了。

希沙科夫站在旁边,可是维克托有很多话要对契贝任说,这些话是不能当着别人的面说的,尤其不能当着希沙科夫的面说。

维克托看见希沙科夫,常常想起格列布·乌斯宾斯基的一句滑稽的话:“金字塔形水牛。”

希沙科夫那肉乎乎的方脸,傲慢的厚嘴唇,指甲泛着油光的胖手指,密密实实的银灰色平头,维克托一看到就觉得不痛快。他每次遇到希沙科夫,心里都要出现疑问:“他认识我吗?会跟我打招呼吗?”每当希沙科夫用肥厚的嘴唇慢慢地说出好像也是肉乎乎的、牛肉似的话时,他却一面生自己的气,一面感到高兴。

“是一头傲慢的公牛!”维克托在谈到希沙科夫时,对索科洛夫这样说。“我一见到他就害怕,就像小镇上的犹太人见了骑兵上校。”

“有什么了不起的!”索科洛夫说。“谁都知道,他都不知道摄影图像出现时的正电子。每一个研究生都知道,希沙科夫院士却不知道。”

索科洛夫很少说别人坏话,不知是由于谨慎,还是由于那种不能责难别人的宗教式感情。可是希沙科夫总是使他非常生气,所以他常常骂希沙科夫,嘲笑希沙科夫,忍也忍不住。

大家谈起战争。

“咱们在伏尔加河上把德国人挡住了,”契贝任说,“伏尔加河真了不起呀。真是活命水,活命水。”

“是斯大林格勒,斯大林格勒,”希沙科夫说,“斯大林格勒之战反映出我们战略的光辉和我们人民的坚强。”

“阿列克谢·阿列克谢耶维奇,您知道维克托·帕夫洛维奇最近的论文吗?”契贝任问。

“当然听说过,不过还没有看过。”

从希沙科夫脸上看不出他是否真的听说过维克托的论文。

维克托对着契贝任的眼睛看了很长的一眼:让他的老朋友和老师看到他经受的痛苦吧,让契贝任知道他的损失和疑虑吧。可是维克托的眼睛也看出了契贝任的悲哀、他的痛苦的思绪、他的暮年的疲惫感。

索科洛夫走过来,就在契贝任和他握手的时候,希沙科夫院士不大客气地拿眼睛扫了扫他的旧上衣。等波斯托耶夫走到跟前,希沙科夫绽开他那大脸上所有的肉高兴地笑了笑,说:

“你好,你好,我的好朋友,我见到你真高兴。”

这两个又高又粗的魁梧汉子谈起身体健康、老婆、孩子、别墅。

维克托低声问索科洛夫:

“你们家收拾好了吗?家里暖和吗?”

“目前还不比在喀山好。玛利亚一再要我问候你们。可能明天下午她上你们家去。”

“那太好啦,”维克托说,“我们已经想她了,在喀山天天见面,我们已经习惯了。”

“是啊,天天见面,”索科洛夫说,“据我看,玛利亚一天上你们家三趟。我早就劝她搬到你们家去啦。”

维克托笑起来,心里想,自己的笑不是完全自然的。这时候数学家列昂季耶夫院士来到大厅里。列昂季耶夫大鼻子,大脑袋剃得光光的,戴着黄镜框的大眼镜。过去他们住在加斯普拉的时候,有一次上雅尔塔去,在酒店里喝了很多酒,唱着黄色小调来到加斯普拉的食堂,弄得食堂工作人员不知如何是好,惹得所有休养的人捧腹大笑。列昂季耶夫现在一看见维克托,就笑起来。维克托微微垂下眼睛,等待着列昂季耶夫谈他的论文。

但是看样子,列昂季耶夫想起了加斯普拉的趣事,把手一挥,高声说:

“噢,怎么样,维克托·帕夫洛维奇,咱们再喝几杯?”

进来一位穿黑西装的黑头发年轻人,维克托发现,希沙科夫马上向他鞠了一个躬。

苏斯拉科夫走到年轻人跟前。苏斯拉科夫是在主席团里分管重要而不为人所知的事情的;大家只知道,借助他的力量比借助主席团的力量更容易把一位科学博士从阿拉木图调到喀山,更容易分到住房。这是一个面容疲惫、习惯于夜晚工作、脸颊像灰色面团一样苍白的人,是大家时时都用得着的人。

大家都习惯了,苏斯拉科夫在开会时抽“巴尔米拉”牌高级香烟,院士们抽黄烟和土烟,在走出科学院大门以后,不是科学界名人们对他说:“来,坐我的车吧。”而是他一面朝自己的小汽车走,一面对科学家们说:“来,我把您带着。”

现在维克托观察着苏斯拉科夫和那个黑头发的年轻人说话,看出来,那个年轻人丝毫无求于苏斯拉科夫。不论请求的方式多么斯文典雅,总能看出,谁是求人的,谁是被人求的。相反,那个年轻人倒是希望快点儿结束同苏斯拉科夫的谈话。年轻人特意带着恭敬的神气向契贝任鞠了一个躬,但是在这种恭敬之中有一种不易觉察、但不知为什么还是可以觉察到的漫不经心的神气。

“请问,这位年轻的大人物是谁?”维克托问。

波斯托耶夫低声说:

“他最近调到中央委员会科学处工作。”

“您要知道,”维克托说,“我有一种很奇怪的感觉。我觉得,我们在斯大林格勒的不屈不挠精神—这就是牛顿的不屈不挠精神,爱因斯坦的不屈不挠精神。在伏尔加河上的胜利标志着爱因斯坦思想的胜利,总而言之,我就是有这样的感觉。”

希沙科夫带着无法理解的神气笑了笑,轻轻摇了摇头。

“阿列克谢·阿列克谢耶维奇,难道您不理解我的意思吗?”维克托说。

“是啊,是不能理解,”科学处的年轻人来到旁边笑着说,“看样子,只有所谓相对论才能帮助找出俄罗斯的伏尔加河与爱因斯坦之间的联系。”

“所谓相对论?”维克托吃惊地说。他看到对他表示的这种不友好的嘲笑态度,不禁皱了一下眉头。

他看了看希沙科夫,想寻求支持,但是看样子,这位金字塔形水牛那种不屑一顾的蔑视态度也推广到爱因斯坦身上了。

维克托立刻觉得十分懊恼,又难受,又气愤。他有时候就会这样,一生起气来,费很大力气才能忍住。回到家里以后,才会在大晚上慷慨激昂地反驳欺侮他的人。有时他忘乎所以,又叫喊,又打手势,通过想象中的发言维护自己的所爱,嘲笑敌人。柳德米拉就对娜佳说:

“你爸爸又发表高论了。”

这会儿他感到受了侮辱,不仅是因为对待爱因斯坦的轻蔑态度。他认为,每一个熟人都应该和他谈谈他的论文,他应该成为与会者注意的中心。他觉得自己受了欺负,受了凌辱。他知道,为这类的事生气是很可笑的,但是他生气了。只有契贝任和他谈起他的论文。

维克托用温和的口气说:

“法西斯分子赶走了天才的爱因斯坦,他们的物理学就成了猢狲的物理学。可是,谢天谢地,我们挡住了法西斯的进攻。于是这一切就在一起了:伏尔加河,斯大林格勒,还有我们时代首屈一指的天才人物爱因斯坦,还有最落后的村庄,没有文化的老农妇,还有大家都盼望的自由。这一切都连在一起了。我好像说得很乱,不过,恐怕没有什么比这种乱更清楚了。”

“维克托·帕夫洛维奇,我觉得您对爱因斯坦的颂扬太过分了。”

“总的来说,”波斯托耶夫快活地说,“可以说,是有些过分。”

科学处的年轻人带着不快活的神气看了看维克托。

“嗯,施特鲁姆同志[11],”他说,于是维克托又感觉出他的口气的不善,“在我国人民的生死一线的紧急关头,您认为在自己心里把爱因斯坦和伏尔加河联系起来是很自然的事,可是在这些日子里,与您观点不同的同志们心里却出现的是另外的想法。各人的心是各人的,这没有什么好争论的。不过,至于如何评价爱因斯坦,倒是可以争论争论,因为,我认为,用唯心主义理论冒充最高的科学成就是不应该的。”

“您别来这一套吧,”维克托打断他的话,又用傲慢的、教训的口吻说,“阿列克谢·阿列克谢耶维奇,现代物理学离开爱因斯坦,就是猢狲的物理学。我们不应该拿爱因斯坦、伽利略、牛顿的名字开玩笑。”

他动了一下手指头,警告希沙科夫,他看到希沙科夫眨巴了一下眼睛。过了一小会儿,维克托就站在窗前,声音忽大忽小地把这次偶然发生的冲突说给索科洛夫听。

“您刚才就站在旁边,竟然什么也没有听见,”维克托说,“契贝任也好像有意走了开去,没有听见。”

他皱起眉头,不再说话了。他还想今天自己会成为大家注意的中心呢,想得多么天真,多么孩子气啊。谁知,大家的激动情绪是上级机关的一个年轻人的到来引起的。

“您知道这个年轻后生姓什么吗?”索科洛夫就好像猜到了他的心思,忽然问道。“他是什么人家里的?”

“我一点也不知道。”维克托说。

索科洛夫把嘴巴凑到维克托耳朵上,小声说起来。

“您说什么!”维克托叫起来。他想起当时他很不理解的金字塔水牛和苏斯拉科夫对待这位大学生年龄的小伙子的态度,不禁拉长声音说:

“怪……不……得……呢……我还觉得奇怪呢。”

索科洛夫微微笑着对他说:

“您回来第一天就在科学处和科学院领导层为自己建立起良好关系啦。您就像马克·吐温小说里那个人物,在税务检查官面前夸起自己的收入。”

但是维克托不喜欢这种俏皮话,他问道:

“您刚才站在我旁边,当真没有听见我们的争论吗?还是不愿意参与我和税务检查官的谈话?”

索科洛夫那小小的眼睛对着维克托笑了笑,那双眼睛显得很善良,因此也显得很好看了。

“维克托·帕夫洛维奇,”他说,“您别不好受,难道您以为,希沙科夫会重视您的论文吗?哼,我的天啊,我的天啊,这儿有多少荣华富贵的事要忙活,您的论文可是实在事情呀。”

他的眼神和声调中流露出真诚和温暖,这正是维克托在喀山那个秋日黄昏去找他时希望得到的。那时候在喀山维克托没有得到。

大会开始了。发言的一些人谈到科学在危难的战争时期的任务,谈到自己愿意为人民的事业贡献出全部力量,要帮助军队战胜德国法西斯。谈到科学院各研究所的研究工作,谈到党中央对科学家的帮助,谈到斯大林同志在领导军队和人民的同时,还要抽时间关心科学问题,还说科学家们要不辜负党和斯大林同志本人的信任。

谈到在新的环境中势必进行组织上的改变。物理学家们很吃惊地了解到,发言人对该研究所的科学研究计划很不满意:过分注重纯理论问题了。大家都在大厅里小声传说着苏斯拉科夫的话:“研究所脱离实际。”

二十七

党中央委员会研究了国内科研工作的状况。都说,党现在要把主要的注意力放在物理学、数学和化学的发展上。

中央委员会认为,科学应当面向生产,应当接近现实,同现实有更密切的联系。

据说,斯大林同志参加了会议,他像往常一样,一只手握着烟斗,在大厅里走来走去,不时地带着沉思的神气停下来,不知是倾听与会者的发言,还是倾听自己心里的话。

与会者尖锐地批评了唯心主义和轻视本国哲学和科学的倾向。斯大林在会议上有两次插话。当谢尔巴科夫发言,赞成对科学院的预算进行限制的时候,斯大林摇了摇头,说:

“搞科学不是做肥皂。我们对科学院不进行限制。”

第二次插话是在会上有人谈到唯心主义理论的害处和一部分科学家过分崇拜西方科学的时候。斯大林点点头,说:

“应当好好保护我们的人,决不能实行专制残暴统治。”

被邀参加这次会议的科学家们,对朋友们说了说斯大林的情形,叫朋友们保证不要说出去。过了三天,整个莫斯科科学界人士便在几十个家庭和朋友圈子里小声议论起会议上的情形。

很多人小声传说着,说斯大林已经白了头,说他的嘴里一口黑牙,牙齿已经坏了,说他的手很好看,手指头细细的,因为出过天花,脸上还有麻子。

听到这些话的人警告未成年人说:

“小心,你要是乱说,不仅要害了自己,还会害了咱们全家。”

大家都认为,科学家们的状况将会大大地改善。斯大林说的关于专制残暴制度的话,使人产生很大的希望。

过了几天,著名的植物遗传学家切特韦里科夫被逮捕了。关于他被捕的原因有各种各样的传说:有的说他是间谍;有的说他在出国期间常常和俄国流亡分子会面;有的说他的德国裔妻子在战前常常和住在柏林的妹妹通信;有的说他企图推广小麦的有害品种,以造成病害和歉收;有的认为,他的被捕与他说的有关食指的一句话有关系;有的认为,他被捕是因为他对小时候的伙伴说过一桩政治方面的笑话。

在战争时期不常听到政治性的逮捕,所以许多人,包括维克托在内,就以为这种可怕的事永远不会有了。

维克托又想起了一九三七年,那时候几乎每天都可以说出夜里被捕的人的名字。想起那时候怎样在电话里互相报告这方面的事:“昨天夜里安娜·安德列耶芙娜的丈夫病了……”想起邻居在电话里怎样回答有关被捕者的情况:“他离开了,不知道什么时候能够回来……”想起当时常听到的逮捕人的情形:有的正在给孩子洗澡,就被抓走了,有的是在工作,在看戏,在深夜里被抓走。想起有人说过:“搜查了两天两夜,什么都搜了,甚至把地板都撬起来……几乎什么都没看,为了做样子,随便翻了翻书……”

想起一去不复返的几十个人的名字:瓦维洛夫院士……维捷院士……诗人曼德尔施塔姆、作家巴别尔……鲍里斯·皮利尼亚克……梅耶霍德……细菌学家科尔叔诺夫和兹拉托戈罗夫……普列特尼奥夫教授……列文博士……

但问题并不在于被捕者是杰出人物和社会名流,问题在于,不论是名人还是毫不出众的普通人,全都没有罪,都是老老实实工作的人。

难道这一切又要开始了?难道到了战后,听到夜里的脚步声和汽车声还是要心惊肉跳?

多么难把争取自由的战争和这种事联系在一起啊……是啊,是啊,我们在喀山真不该那样乱说啊。

切特韦里科夫被捕之后,过了一个星期,契贝任声明离开物理研究所,接替他的位子的是希沙科夫。

科学院主席团的人上契贝任家里去过。据说,不知是贝利亚,还是马林科夫召见过契贝任,好像契贝任不肯改变研究所的选题计划。

据说,考虑到他的巨大的科学成就,暂时不想对他采取极端措施。同时被解除职务的还有分管行政工作的所长、年轻的自由主义分子皮敏诺夫,认为他不称职。

希沙科夫院士担任了所长职务和契贝任原来担任的学术领导职务。

有传闻说,契贝任在这些事情之后,心脏病发作。维克托马上就准备去看他,往他家里打了个电话。接电话的保姆说,契贝任最近确实身体不大好,遵照医生意见同夫人一起上外地去了,过两三个星期才能回来。

维克托对柳德米拉说:

“这种情形,就好比把一个小孩子从电车门口往下推,还要把这叫做保护,让他不受专制残暴制度的危害。契贝任是马克思主义者,还是佛教徒、喇嘛教徒,这跟物理有什么关系?契贝任建立了一个学派。契贝任是卢瑟福的朋友。契贝任方程式每一个管院子的人都知道。”

“哼,关于管院子的,爸爸,你算了吧。”娜佳说。

维克托说:

“小心,你要是乱说,不仅要害了自己,还会害了我们全家。”

“我知道,这种话只能对家里人说。”

维克托用温和的口气说:

“唉,娜佳,我有什么办法能改变中央的决议?能用头去撞墙吗?而且契贝任是自己声明愿意离职的呀。况且,据说大家都不满意他的工作。”

柳德米拉对丈夫说:

“用不着这样激动。再说,你自己也常常和契贝任争论嘛。”

“如果不争论,就没有真正的友谊。”

“就是了,”柳德米拉说,“瞧着吧,你那样喜欢乱说,也会把你的实验室领导职务撤掉。”

“我倒不担心这个,”维克托说,“娜佳说得不错,的确,我所有的话都是说给自家人听的,等于在口袋里做手势。你打个电话给切特韦里科夫的夫人,去看看她!你们是朋友嘛。”

“现在这样不合适,再说,我们也不是多么亲近的朋友,”柳德米拉说,“我一点也帮助不了她。她现在也用不着我。以往出了这种事之后,你给谁打过电话吗?”

“依我看,应该。”娜佳说。

维克托皱了皱眉头。

“就是打个电话,实质上还是等于‘在口袋里打手势’。”

他想和索科洛夫谈谈契贝任的离职,这种事不能和老婆孩子谈。但是他硬压制着自己不给索科洛夫打电话,这种事不能在电话里谈。

还是很奇怪。为什么让希沙科夫当所长?很明显,维克托最近发表的论文是科学界的大事。契贝任在学术会议上说,这是苏联物理理论界十年来最重大的事件。可是却让希沙科夫做研究所的领导。这是闹着玩儿的吗?他看着几百张照片,看到电子的痕迹往左偏转,忽然又看到照片上同样的痕迹、同样的粒子往右偏转。可以说,把正电子握住了。这是年轻的萨沃斯季扬诺夫也会明白的。可是希沙科夫却撅起嘴,把照片推到一边,认为照片有毛病。谢里凡说:“唉,这就是向右呀,你简直不知道哪边是右,哪边是左。”

最奇怪的是,谁也不觉得这样的事奇怪。这样的事也就不知不觉变成理所当然的了。维克托的朋友们、他的妻子和他自己也就认为这种情况是合理合法的了。维克托不适合做所长,希沙科夫适合做所长。

波斯托耶夫是怎么说的?哦,他说:“最主要的是,我们都是苏联人。”

不过,要做一个比契贝任更爱苏联的苏联人,恐怕很难。

早晨,在去研究所的路上,维克托想象着,所里的工作人员,从院士到试验员,一定都在谈着契贝任了。研究所门口停着一辆小汽车,司机是一个戴眼镜的上了年纪的人,正在看报。门房老头子夏天常常和维克托一块儿在实验室里喝茶,今天在楼梯上碰到他,说:“新官上任啦。”又伤心地说:“咱们的老所长呢,嗯?”

在大厅里,试验员们在谈设备安装的事。试验设备是昨天从喀山运来的。试验大厅里摆满一个个大箱子。在乌拉尔定做的新仪器同旧的设备一起运到。诺兹德林站在一个老大的木板箱旁边,维克托觉得他的脸上似乎流露着一副不可一世的神气。

佩列佩里津腋下夹着拐杖,用一条腿在这个大箱子周围蹦跳着。

安娜·斯捷潘诺芙娜指着一个个箱子,说:

“您看,维克托·帕夫洛维奇!”

“这么大的东西连瞎子也会看到。”佩列佩里津说。

但安娜·斯捷潘诺芙娜说的不是箱子。

“看见啦,看见啦,当然看见啦。”维克托说。

“再过一个小时,工人们就来了,”诺兹德林说,“我已经跟马尔科夫教授说好了。”

他是用当家人的平静而缓慢的口气说这话的。轮到他说话算数的时候了。

维克托走进自己的办公室。马尔科夫和萨沃斯季扬诺夫坐在长沙发上,索科洛夫站在窗前,旁边的磁实验室主任斯维琴坐在写字台前抽着自己卷的烟卷儿。

维克托一走进来,斯维琴站起来,要把椅子让给他:

“这是主人的位子嘛。”

“不用,不用,请坐吧。”维克托说。接着又问:“最高会议上谈的是什么?”

马尔科夫说:

“关于限额问题。每位院士的限额要提高到一千五,一般的人限额提高到五百,和人民演员,和列别杰夫—库马奇那样的伟大诗人一样。”

“咱们要开始安装设备了,”维克托说,“可是契贝任不在所里了。正如俗话说的:房屋失火,时钟还在走。”

但是坐在办公室里的人都没有接他的话。

萨沃斯季扬诺夫说:

“昨天我有个堂弟来了,他是出了医院上前方去,从这儿路过,家里没有酒,我向邻居家买了一瓶,花了三百五十卢布。”

“真不得了!”斯维琴说。

“搞科学不是做肥皂。”萨沃斯季扬诺夫快活地说。但是从几个人的脸色可以看出来,他这个玩笑开得很不合适。

“新官已经到任啦。”维克托说。

“是一个劲头儿十足的人呢。”斯维琴说。

“咱们有希沙科夫当头头儿,就有办法啦,”马尔科夫说,“他是日丹诺夫同志家里的座上客。”

马尔科夫是个很奇怪的人,他与人交往似乎不多,但总是什么事都知道:知道旁边的实验室里的副博士加布里切芙斯卡娅怀了孕,知道清洁工丽达的丈夫又进了军医院,也知道最高学位评委会没有批准斯莫罗金采夫的博士学位申请报告。

“可不是吗,”萨沃斯季扬诺夫说,“他的出了名的错误我们都是知道的。不过,总的说,他这人也不坏。诸位可知道,好人与坏人的区别在哪儿?好人做卑鄙事不是心甘情愿的。”

“错误不过是错误,”磁实验室主任说,“不过,一个人凭错误当不了院士。”

斯维琴是研究所党委委员,他是一九四一年秋天入党的,虽然参与党的活动不久,但和很多人一样,非常顶真,用宗教式的虔诚对待党的使命。

“维克托·帕夫洛维奇,”他说,“我正有事要找你,党委请您在大会上发言,谈谈您对新的任务的看法。”

“要我谈领导的错误,批判契贝任吗?”维克托很气愤地问道。他本不希望这样,可是一谈起来就控制不住了。“我不知道我是好人还是坏人,但是要我干卑鄙的事,我不会心甘情愿。”

他转脸朝着实验室的同事们,问道:

“比如说,同志们,你们赞成契贝任离职吗?”

他原本相信会得到他们支持的,可是看到萨沃斯季扬诺夫态度暧昧地耸了耸肩膀并且说“人老了,不中用了”的时候,他觉得很尴尬。

斯维琴说:

“契贝任已经声明,他不再安排任何新的研究工作。有什么办法呢?再说,是他自己辞职的呀,而且还挽留过他呢。”

“那么,阿拉克切耶夫呢?”维克托问。“哼,终于露底了。”

马尔科夫压低了声音说:

“维克托·帕夫洛维奇,据说,当初卢瑟福曾经发誓不研究中子,担心中子可以造成巨大的爆炸力。这是很高尚的,但又是一种毫无意义的清高。据说,契贝任就常常谈一些类似的带有浸礼派教会精神的话。”

维克托心想:“天啊,他怎么全知道呀?”

维克托对索科洛夫说:

“彼得·拉甫连季耶维奇,可见,您和我不是大多数。”

索科洛夫摇了摇头,说:

“维克托·帕夫洛维奇,我认为,在这样的时期,个人主义和执拗是不容许的。战争时期嘛。在领导同志和契贝任谈话的时候,他就不应该考虑自己,不应该考虑自己的利益。”

“哎哟,还有你吗,布鲁特斯?”[12]维克托说。他用这样一句挖苦话掩盖自己的慌乱。

不过说也奇怪:他不光是慌乱,好像也很高兴。他想:“哼,当然啦,我早就知道嘛。”但有什么“哼,当然”的?因为他并没有料到索科洛夫会这样回答。就算料到的话,又有什么可高兴的?

“您应该发言,”斯维琴说,“您不一定要批评契贝任。哪怕说几句话,谈谈党中央的决议和您的研究的关系。”

战前,维克托常常在音乐学院的交响乐音乐会上和斯维琴见面。据说,斯维琴青年时代在物理数学系上学的时候,常常写未来主义派诗歌,在扣眼里别一朵菊花。可是现在斯维琴说起党委的决定,就像说的是亘古不变的真理的定义。

维克托有时想对他挤挤眼睛,拿手指头轻轻朝他的腰上捅一捅,说:“喂,老伙计,咱们干干脆脆地谈谈吧。”

但是他知道,现在不能和斯维琴敞开心扉地谈什么了。不过,他因为听了索科洛夫的话感到非常吃惊,还是索性谈起来。他问道:

“把切特韦里科夫抓起来,也和新的任务有关系吗?老瓦维洛夫坐监牢,也和这有关系吗?恕我斗胆说一句,我认为,契贝任在物理学方面有更大的发言权,其权威性超过日丹诺夫同志,超过中央科学处处长,甚至超过……”

在座的人都看着他,以为他就要说出斯大林的名字。他看到他们的眼神,就把手一挥,说:

“好啦,算啦,咱们上实验大厅里去吧。”

从乌拉尔运来的一些装着新仪器的箱子已经打了开来,从锯屑、碎纸和撬开的木板中已经小心翼翼地取出有大半吨重的仪器主要部件。维克托把手放在光溜溜的金属表面上。从这个金属肚子里将产生急速的粒子束,就像谢利格尔湖边的小教堂下面涌出一条伏尔加河那样。这时候,人的眼睛是很舒服的。当你感觉到世界上竟有这样神奇的机器时,是很愉快的。还要什么呢?下班以后,实验室里只剩下维克托和索科洛夫两个人。

“维克托·帕夫洛维奇,您为什么像只公鸡一样直蹦直跳?您真沉不住气。我对玛利亚说了说您在科学院大会上的成就:您竟然在半小时之内破坏了同新所长,同科学处年轻的大人物的关系!玛利亚吓得提心吊胆,夜里都睡不好觉。您要知道咱们生活在什么时代。我看到了您看着仪器时的脸。这一切都要为几句空话牺牲了。”

“够啦,够啦,”维克托说,“连气都不能喘啦。”

“啊,等一等,”索科洛夫打断他的话,“在研究方面谁也不会干涉你。你可以痛痛快快地喘气。”

“您听我说,我的好朋友,”维克托说着,苦笑了一下,“您对我是好意,我非常感谢。请允许我也以好意相报。比如,说实在的,您为什么忽然当着斯维琴的面那样说契贝任?在喀山有过一阵子思想自由之后,我见到这种事不知道为什么这样难受。至于我……非常遗憾,我并不是那样天不怕地不怕的人。正如咱们在学生时代常说的,我并不是丹东。”

“噢,您不是丹东,真谢天谢地。说实在的,我一向认为,政治演说家恰恰是那些在创造方面无所作为的人。而你我是可以有所作为的。”

“噢,是这样啊,”维克托说,“那么,法国的伽罗华怎么样呢?基巴利契奇又怎么样?”

索科洛夫把椅子推开,说:

“您该知道,基巴利契奇上了绞刑台。不过我指的是乱说废话。就像马季亚罗夫说的那些话。”

维克托问:“这么说,我也是乱说废话了?”

索科洛夫一声不响地耸了耸肩膀。

他们过去有多次争执和口角都被忘记了,看样子这次也会被忘记的。可是不知为什么这次短暂的争执没有就这样过去,没有被忘记。当一个人和另一个人相处十分融洽的时候,他们有时也争吵,有时吵得很没有道理,他们彼此的怨气还是会消失得无影无踪。但如果在人们之间出现了内在的分歧而又不了解这种内在分歧的话,那么,即使偶然的一句话,彼此间一点小的疏忽,也会变成一把尖刀,对友谊是致命的。

而且内在的分歧往往隐藏得很深,永远不暴露出来,人们也就永远认识不到。于是人们就认为,一次无关紧要的大声争论、冲口而出的不好听的话是破坏多年友谊的不幸原因。

不是的,伊凡·伊凡诺维奇和伊凡·尼基福罗维奇争吵不是因为公鹅![13]

二十八

大家都说研究所副所长卡西扬·捷连季耶维奇·科甫琴科是“希沙科夫的准确无误的底片”。科甫琴科和蔼可亲,说话有时带几个乌克兰词儿。他以惊人的速度分到了房子和专用小汽车。

马尔科夫知道院士们和科学院领导人的很多事情。他说,科甫琴科获得斯大林奖金,是因为他生平第一次宣读的一篇已经发表的论文,而他之所以成为论文作者之一,仅仅因为他搞到紧缺的试验材料并使论文很快地在各级通过。

希沙科夫责成科甫琴科组织选聘人员,填补新的空缺。要招聘一些高级科研人员,还要填补真空实验室主任和低温度实验室主任两个空缺。

军事部门调拨了材料和人力,机械厂在改建,研究所大楼在装修,莫斯科水电站向研究所供应无限额的电力,保密工厂拨给研究所一些紧缺材料。这些事也都是科甫琴科操办的。

通常每当一个单位里来了新的领导人,大家都会用尊敬的口气说:“他上班比大家都早,下班比大家都晚。”大家也是这样说科甫琴科的。但是,如果大家说新的领导“上任已经有两个星期了,可是只来过一次,只呆过半小时。简直见不到他这个人”,这样的新领导会引起下属更大的尊敬。因为这就说明,领导人正在攀登新的阶梯,正在高级领导层中活动。

开头一段时期在研究所里大家就是这样谈论希沙科夫院士的。

话说契贝任到城外别墅里去,如他自己说的,到试验小屋里搞研究去了。著名的心脏病医生法因加尔特教授劝他不要做剧烈动作,不要拿重物。契贝任在别墅里又劈柴,又挖沟,自我感觉良好,他写信给医生说,是严格遵守治疗方法帮助了他。

在饥饿而寒冷的莫斯科,研究所似乎是一块温暖而富饶的绿洲。所里的工作人员夜里在潮湿的住房里冻得发抖,早晨一来上班,就很满意地把手放在热乎乎的暖气片上。

研究所里的人特别喜欢设在半地下室里的新食堂。食堂有小卖部,卖酸牛奶、甜咖啡和香肠。售货员在卖食品的时候,不收食品供应卡上的肉票和油票,这是研究所里的人最看重的。

食堂伙食分六个等级:供应各学科博士的,供应高级研究员的,供应初级研究员的,供应高级试验员的,供应技术人员的,供应服务人员的。

主要的纠纷是围绕着两种高级伙食发生的,二者的差别仅在于第三道菜,一种是干果做的果羹,一种是干粉做的羹。发生纠纷,还与发给博士、各科主任家里的食品袋有关系。

萨沃斯季扬诺夫说,当年议论哥白尼的理论,还没有现在议论这些食品袋这样激烈。

有时会觉得,参与创立这种神秘的分配等级制的不光是院委会和党委会,还有更高、更神秘的机构。

一天晚上柳德米拉说:

“今天我收到发给你的食品包,不过真是奇怪,斯维琴在研究方面一点本事也没有,可是领到二十个鸡蛋,不知为什么只给你十五个。我还看了看名单。给你和索科洛夫都是十五个。”

维克托开玩笑地说:

“鬼才知道是怎么一回事儿!众所周知,我们的科学家是分等级的:最伟大的,伟大的,著名的,优秀的,最后,是高级的。因为最伟大和伟大的已经不在人世了,所以不用发给他们鸡蛋。其余的科学家都按学术分量发给白菜、碎麦米和鸡蛋。可是我们全乱了:有的人对社会无益,却能主持马克思主义讨论会,讨得院领导喜欢。一切都乱了套。科学院汽车库主任的待遇和泽林斯基[14]一样:二十五个鸡蛋。昨天斯维琴的实验室里有一位很和蔼的女员工甚至气得放声大哭起来,像甘地一样绝食了。”

娜佳听了爸爸的话哈哈大笑,随后却说:

“你要知道,爸爸,你们这些人当着清洁工的面吃煎肉排而不觉得难为情,是很奇怪的。外婆无论如何不会赞成。”

“知道吗,”柳德米拉说,“这就是按劳分配的原则嘛。”

“哼,简直荒唐。这种食堂连一点儿社会主义气味也没有。”维克托说。接着又补充:“哼,算了吧,我看这一切都是胡闹。”忽然又说:“你们可知道,今天马尔科夫对我说什么?他说,不仅是我们所里的人,而且数学研究所和力学研究所里的人都用打字机把我的论文打出来,在互相传阅。”

“就像传阅曼德尔施塔姆的诗一样吗?”娜佳问。

“你不要笑,”维克托说,“一些大学的高年级学生还希望我去给他们做专题报告。”

“就是嘛,”娜佳说,“就连波斯托耶夫家的阿尔珈也对我说:你爸爸成了天才啦。”

“噢,不一定吧,我离天才还远着呢。”维克托说。

他朝自己的房间走去,但马上又转回来对妻子说:

“我真想不通,会有这样浑蛋的事,发给斯维琴二十个鸡蛋。我们这儿真会侮辱人!”

索科洛夫在名单上和他排在一个等级,他也感到很不痛快,虽然也觉得不好意思。当然嘛,应该表示表示维克托的成就大些,哪怕多一个鸡蛋也好,比如说,给索科洛夫十四个,少一点点儿,只是表示表示。

他觉得自己很可笑,但是,不知为什么他觉得他和索科洛夫分得一样多,比起斯维琴分得比他多更可气。斯维琴的情形是很简单的:他是党委委员,他的优势是在党国方面。维克托对这一点是不生气的。

可是索科洛夫的情形就涉及科研能力和科学家的成就。在这方面维克托就不能平心静气了。他从内心里感到气愤,感到难受。但这种评价的表现方式是很可笑又可怜的。他很明白这一点。但是如果一个人并不总是很伟大,而是通常会很可怜,那又有什么办法呢?

上床就寝的时候,维克托想起不久前和索科洛夫谈起契贝任的那一场谈话,很生气地骂道:

“一副奴才相![15]”

“你说谁?”正在被窝里看书的柳德米拉问道。

“说的是索科洛夫,”维克托说,“真是个奴才!”

柳德米拉把一个手指头夹在书里,也没有转过头来,说:

“你瞧着吧,说不定还要把你从研究所赶出去呢,全是因为你乱说一气。又爱发火,对什么人都不满意……跟什么人都吵过了,现在我看出来,你还要跟索科洛夫吵一场呢。过不了多久,就没一个人肯上咱们家来了。”

维克托说:

“噢,用不着,用不着,柳德米拉,亲爱的。噢,怎么给你解释呢?你要知道,现在又像战前那样为了每一句话提心吊胆了,又像那样没有一点儿正气了。你瞧瞧契贝任!柳德米拉,这可是一个了不起的人!我以为全研究所里的人会一齐叫起来的,谁知只有一个看门的老头子对他表示同情。波斯托耶夫竟对索科洛夫说:‘最主要的是,我们都是苏联人。’他说这话管什么用?”

他很想和柳德米拉多谈一会儿,对她说说自己的一些想法。他不知不觉地关心起这些事,关心起发食品的事,感到很惭愧。为什么会这样?为什么回到莫斯科以后,他好像老了,没有劲头了,关心起生活琐事、庸俗的问题、官场上的事?为什么在喀山的时候他的精神生活更深厚、更有意义、更纯洁?为什么就连他主要的科研兴趣、他的欢乐也模糊了,同许多渺小、虚荣的念头混到了一起?

“柳德米拉,我真不痛快,处境艰难。喂,你怎么不说话?柳德米拉?”

柳德米拉没有说话。她睡着了。

他轻轻地笑起来。他觉得真好笑:一个女人听说他得罪了人,担心得睡不好觉,另一个女人却睡着了。他仿佛看到了玛利亚那瘦削的脸,于是便把刚才的话又重复了一遍,但不是对妻子:

“你理解我吗?嗯,玛利亚?”

“见鬼,什么乱七八糟的都往脑子里钻了。”他想道,一面沉沉入睡。

乱糟糟的东西确实钻进了他的脑子。

二十九

维克托的手不巧。家里的电熨斗烧坏了,电灯短路了,一般都是柳德米拉修理。

在他们共同生活的头几年,他的无用使她感到可亲可爱。但是近来她开始生他的气。有一次,他把空空的茶壶放到火上,她就说:

“你的手简直是泥巴做的,笨透啦!”

在研究所里开始安装仪器的时候,维克托常常想起这一句使他又生气又懊恼的话。

在实验室里当家做主的是马尔科夫和诺兹德林。萨沃斯季扬诺夫首先感觉到这一点,有一次在生产会议上说:

“除了马尔科夫教授和诺兹德林,这里没有上帝,也没有上帝的代表!”

马尔科夫的古板和稳重不见了。维克托很赞赏他的思想的大胆,能够随时随地解决突然出现的问题。维克托觉得马尔科夫简直像一名外科医生,在纵横交错的血管与神经结中间得心应手地操纵着手术刀。一个有着高度智慧和灵敏感觉的聪明物种似乎正在他的刀下诞生。似乎这个新的、在世界上第一次出现的金属有机体也有心脏,也有感觉,也会高兴和痛苦,和制造它的人完全一样。

维克托总觉得马尔科夫那种坚定不移的自信心有些可笑,他坚信自己的工作、自己设计的仪器比释迦牟尼和穆罕默德干的那些无聊的事或者托尔斯泰和陀思妥耶夫斯基写的书更为重要。

托尔斯泰怀疑过自己的伟大创作是否有益。天才的作家并不坚信自己在做有益于人类的事。但是物理学家们就不怀疑自己的研究对人类是否有用。马尔科夫就不怀疑。

但是现在维克托不觉得马尔科夫的这种信心可笑了。维克托喜欢看诺兹德林拿锉刀、钳子、螺丝刀干活儿,或者细心地调理一缕缕的电线,帮助电工将引线上的电流通向新的装置。

地上放着一捆捆的电线和许多青灰色的铅片。大厅当中的钢板上放着从乌拉尔运来的新装置的基本部件,带有不少方的和圆的镗孔。这种用于超精密的物质研究的金属庞然大物,蕴藏着一种惊心动魄的美。

一两千年以前,在海边有几个人用粗木头做木筏,用绳子捆,用扒钉钉。海边沙滩上放着绞车、木工台,用瓦罐在火上熬松脂……出海的时刻越来越近了。

晚上,做木筏的人回到家里,呼吸呼吸家庭生活的气息,烤烤火,听听老婆的唠叨和笑声,有时也和家里人吵吵嘴,打打孩子,和邻居吵一架。到夜里,在温暖的黑暗中会听到大海的波涛声,会预感到未来航程的惊险,心会紧紧收缩起来。

索科洛夫在看别人做事情的时候,一般不说话。维克托在回头看的时候,一般都要碰到他那严肃的、凝视的眼神,似乎往常他们之间良好的、重要的关系依然存在。

维克托很想开诚布公地和索科洛夫谈谈。事实上,一切都是很奇怪的。就如天天想着票证、限额,想着荣誉的分量、领导的照顾,都是有损心灵的。这不是,心灵里也还有与领导、与职务高低、与奖金无关的东西。

他现在又觉得喀山的那些晚上很美好,很有年轻人的气氛,有点儿像革命前的大学生晚间集会。可能马季亚罗夫是一个十分清白的人。真奇怪:卡里莫夫怀疑马季亚罗夫,马季亚罗夫也怀疑卡里莫夫……两个人都是十分清白的。他相信这一点。不过,也许像海涅说的,“两个都臭”呢?

他有时想起和契贝任谈发面桶的一番话。为什么他现在回到莫斯科,一切渺小、卑微的东西都在心里浮现出来?为什么他不尊敬的一些人都浮到了面上?为什么他认为有本事、有才能、忠诚可靠的一些人如此无用呢?要知道契贝任谈过希特勒德国,契贝任说错了啊。

“很奇怪,”维克托对索科洛夫说,“各个实验室的人都来看咱们安装设备,就是希沙科夫没有来看过,一次也没有来。”

“他的事情很多呀。”索科洛夫说。

“当然,当然。”维克托连忙表示同意。

是啊,回到莫斯科以后,很难和索科洛夫推心置腹地谈谈了。真不知道是怎么回事。

说也奇怪,他再也不和索科洛夫争论任何问题了,倒是希望能避开争论。但是要避开争论也不容易。有时争论会突然发生,出乎维克托的意料。

维克托慢悠悠地说:

“我想起咱们在喀山说的许多话……哦,马季亚罗夫怎么样,有信给您吗?”

索科洛夫摇了摇头。

“不知道,不知道马季亚罗夫怎么样。我对您说过嘛,直到离开喀山,我们都没有再见面。想起那时候咱们谈的一些话,我越来越觉得不痛快。咱们因为灰心丧气,就想把战争时期的暂时困难说成是苏维埃制度的所谓缺陷造成的。一切被看做苏维埃制度的缺陷的,恰恰是其优越性。”

“比如说,一九三七年也是优越性吗?”维克托问道。

索科洛夫说:

“维克托·帕夫洛维奇,近来咱们不论谈什么,您都要使谈话变成争论。”

维克托很想对他说,恰恰相反,他倒是不希望争吵,是索科洛夫有火气,这种火气就使他一有什么缘由就争论起来。可是他却说:

“可能这是因为我的脾气太坏,而且越来越坏。不光是您这样说,柳德米拉也这样说。”

他说过这话,心里想:“我多么孤单。在家里,在外面,都很孤单。”

三 十

帝国党卫军首领希姆莱要召开会议,研究帝国保安总部推行的特别措施。这次会议受到特别重视,这和希姆莱前往元首的行营有关系。

党卫军少校利斯接到柏林来的命令,要他汇报集中营管理处附近一项特别工程建筑的进展情况。

利斯在视察这项工程之前,先要到福斯公司的机械厂和为保安部生产订货的化学工厂去一趟。在这之后,利斯再去柏林向主持筹备会议的党卫军少校艾希曼汇报情况。

利斯因为有机会去柏林,感到很高兴。老是住在集中营里,天天和野蛮、愚昧的人打交道,他感到受不了。

他在上汽车的时候,想起了莫斯托夫斯科伊。

大概老头子在隔离室里日日夜夜拼命猜想,利斯传他去有什么目的,正在紧张地等待着呢。实际上不过是他要检验一下自己的一些想法,希望写一篇论文《敌人的意识形态及其代表人物》。

多么有意思的性格!事实上,如果有谁进入原子核,不仅会受到排斥力的作用,也会受到吸引力的作用。

小汽车出了集中营的大门,利斯也就把莫斯托夫斯科伊忘记了。

第二天一早,利斯来到福斯公司的工厂。

吃过早饭以后,利斯在福斯的办公室里和设计师普拉什凯谈了谈,然后和指导生产的几个工程师谈了谈,在办事处营业主任和他谈了谈所订的成套设备的成本计算。他在工厂的各个车间里待了几个小时,在机器的隆隆声中转来转去,到傍晚,他就十分疲乏了。

福斯的工厂生产的是保安部订货的重要部分,利斯看了十分满意:企业领导者对事情考虑得很周密,技术条件执行得很精确,机械工程师们改进了传送结构,热力工程师设计出最经济的焚化炉操作图。

在工厂辛苦地转悠了一天之后,来到福斯家里读过的夜晚特别愉快。

对化学工厂的视察却让利斯非常失望:计划生产的化学产品只完成了百分之四十多一点儿。

尤其使利斯生气的是,化学工厂的人有很多怨言:生产又复杂,又变化无常;在空袭的时候炸坏了通风装置,车间里有许多工人中毒;稳定生产所需要的硅藻土供应很不稳定;密闭的容器常常在铁路运输中耽搁……

不过,化学股份公司经理处的人非常清楚保安部订货的意义。股份公司的化学总工程师基利赫加尔津对利斯说,保安部的订货任务一®š会如期完成。经理处已经采取措施,推迟完成军火部的订货任务,这是从一九三九年九月以来不曾有过的事。

利斯没有去观看化学合成实验室的一次重要试验,但是查看了有生理学家、化学家和生物化学家签名的记录。

这一天,利斯会见了进行试验的科学工作者。这是一些年轻的科学家。有两个女的(一个是生理学家,一个是生物化学家),一名病理解剖医生,一名低沸点有机化合专家,还有领导试验的毒物学家菲舍尔教授。参加会议的人给利斯留下良好的印象。虽然他们因为自己制定的研究方案受到称赞都很高兴,但是他们也没在利斯面前掩盖工作中的薄弱环节和对自己的质疑。

第三天,利斯和奥伯施泰因安装公司的一名工程师一起乘飞机前往建筑工地。他心情很好,这一次外出他很开心。接下去就是最开心的事:视察过工程之后,就要和工程的技术领导人一起飞往柏林,到保安总部去汇报情况。

天气很坏,下着十一月的冷雨。飞机好不容易在集中营的中央机场着陆—在低空机翼就开始结冰,地面上还笼罩着一层雾。黎明时候下过雪,有的地方的土块上还有一点一点又湿又滑的积雪,没有被雨水冲洗掉。

工程师们的呢帽帽檐浸透了沉甸甸的雨水,耷拉了下来。

新铺的铁路通到建筑工地上,这铁路直接与主要干线相连接。

铁路附近有一些仓库的库房,于是就从仓库开始视察。敞棚底下正在对物品进行分类:有各种各样的机械零件、溜槽和滑轮传送装置的各个部件、各种直径的管子、鼓风和通风装置、粉碎骨头的球磨机、尚未装上架子的测量气体和测量电力的仪器、一捆捆的电缆、水泥、自动翻斗车、一堆堆的钢轧,还有办公室的家具。

有一些特别库房由党卫军把守着,这种库房有许多排气装置,通风机嗡嗡地响着,用来储藏已经开始生产的化学化合产品。里面有许多带有红色阀门的气瓶和贴了红蓝色标签的十五公斤大罐,远看很像一罐罐保加利亚果酱。

从这座半地下库房里走出来,利斯和他的陪同者迎面碰上刚刚乘火车从柏林来的公司总设计师什塔尔干克教授,还有工程主任冯·赖内克。赖内克是个高大的男子,穿着黄色的皮夹克。

什塔尔干克呼哧呼哧地喘着气,潮湿的空气引起他的哮喘病发作。他周围的工程师们都在责怪他不爱惜自己的身体;他们都知道,什塔尔干克的设计图册就在希特勒的私人图书室里。

建筑工地和二十世纪中期一般的巨大建筑工地没有任何不同。在一处处基坑周围可以听到哨兵的哨声、挖土机的轧轧声、吊车的移动声和机车的尖叫声。

利斯及其陪同者走到一座没有窗子的四方形灰色建筑物跟前。所有的工业建筑物、一座座红砖炉、粗大的烟囱、装了玻璃顶的调度塔和警卫塔,都跟这座没有窗子、没有挂牌子的灰色建筑物有关系。

筑路工人正在一条路上铺沥青,热腾腾的灰烟从压路机下面往上冒,和灰色的冷雾混合到一起。

赖内克对利斯说,在检查一号工程的密闭性的时候,结果不能令人满意。什塔尔干克忘记了自己的哮喘,用激动的嘶哑声音向利斯说明新建筑物的设计思想。

一般的工业水轮机看起来很简单,体积又小,却是巨大的能量和速度的中心,在水轮机的旋转中水的地质能量变为功。

这座建筑物就是根据水轮机的原理建造的。它能使生命和与生命有关的各种能量变为无机物。在这种新形式的轮机中,要消除心理功能、神经功能、呼吸功能、心脏功能、肌肉功能、造血功能。水轮机原理、屠宰机原理和焚烧垃圾机原理将联合于新建筑之中。必须把这几种特性联合于一个简单的设计方案之中。

“众所周知,”什塔尔干克说,“我们的敬爱的元首在视察最平常的工业工程的时候,也不会忘记设计形式。”

然后他放低了声音,只让利斯一个人能听见。

“您是知道的,帝国元首看到华沙附近的集中营设计在形式上过分讲求神秘感,非常不高兴。这一切也必须考虑到。”

水泥建筑的内部结构是与高速度大量生产的工业时代相适应的。

生命和水一样,一进入下水道,就不能停止,也不能往回流了。生命在水泥通道里的移动速度可以用斯托克斯关于液体在管子里移动的公式来表示,就是说,其移动的速度取决于其浓度、比重、黏性、摩擦力和温度。一盏盏电灯嵌在棚顶上,都用很厚的半透明玻璃保护着。

越往前走,电灯越亮,走到密闭室门口,更是亮得刺眼。密闭室安着光滑的钢门。

视察的人来到门口,显得特别激动,建筑工人和安装工人在新的成套设备要开工时往往会这样的。

一些做粗活的工人在用水龙带冲洗地面。一名穿白大褂的化学工程师在关闭的门口测量压力。赖内克吩咐打开密闭室的门。走进带有低矮水泥顶的宽敞的密闭大厅之后,有几名工程师摘下帽子。密闭大厅的地面是用可移动的沉甸甸的钢板拼成的,钢板都装了钢框,一块块钢板之间不见缝隙。在调度人员开动机械装置的时候,地面的钢板就一齐竖立起来,密闭大厅里所有的一切都会进入地下室。掉下去的有机物要经过口腔科人员检查,摘去装在口腔里的贵金属。然后,通向火化炉的传送带开始运转。已经失去知觉的有机物到了火化炉里就在热能的作用下受到进一步的破坏—变为磷肥、石灰、氨肥、二氧化碳和二氧化硫。

一名联络官走到利斯跟前,递给他一封电报。大家都看到,这位党卫军少校看过电报之后,脸色阴沉下来。

电报通知利斯,说党卫军少校艾希曼今天夜里来工地上和他见面。艾希曼已经乘汽车上了慕尼黑的公路干线。

利斯不能去柏林了。他本来明天夜里就要回到自己的别墅,生病的妻子就住在别墅里,天天盼望着他。他本来可以在睡觉之前穿着软软和和的便鞋,在安乐椅上坐一会儿,在温暖与舒适中暂时忘却这严峻的时代。夜里在郊外别墅的被窝里听着柏林防空部队高射炮远远的轰鸣声,多么愉快啊。

做过汇报之后,在上郊外之前,在傍晚没有空袭的安静时候,他还可以去看望哲学研究所里的一个年轻女子,只有她才知道他有多么难过,心里多么慌乱。为了和那女子相会,他在公文包里还带了一瓶白兰地和一盒巧克力。现在这一切成了泡影。

工程师们、化学家们、设计师们都一齐望着他:是什么样的烦恼事使保安总部的这位视察要员如此不快呢?谁又能知道呢?

在场的人有一会儿曾经以为,密闭室已经不属于建设者了,已经活了,就要凭自己的水泥特性生活,要满足自己的水泥的饥渴,就要开始分泌毒液,用钢铁的大嘴开始咀嚼,开始消化食物了。

什塔尔干克朝赖内克挤了挤眼睛,小声说:

“大概利斯是接到通知。那位党卫军少校要在这儿听他的汇报,这我在早晨就知道了。他原本要在家里休息休息,也许还要和一位心爱的女士相会,这一来就落空了。”

三十一

利斯和艾希曼在夜里见了面。艾希曼有三十五岁左右。手套、帽子、靴子,这三样表现徳国武装力量的神气、高傲和优越性的东西,跟党卫军领袖希姆莱所穿戴的完全一样。

利斯在战前就认识艾希曼一家。他们是同乡。利斯在柏林大学上学的时候,在报社以及后来在哲学杂志编辑部工作的时候,有时回故乡去看看,常常见到中学时期的同学。有些人在社会浪潮中得势了,后来浪潮过去,就消沉了,荣誉和物质享受又被别人捞去。可是年轻的艾希曼一直生活得很不起眼,很单调。凡尔登城下的炮声,曾经似乎要来的胜利,失败和通货膨胀,国会里的政治斗争,绘画、戏剧、音乐中左的和超左的流派的冲击,新风尚的兴起和衰落—一切都没有改变他的单调生活。

他做过外地一家公司的代理人。无论在家里还是对待外人,他从不过分粗暴也不过分殷勤。人生的条条大路都被闹哄哄的、指手画脚的、敌视他的人群堵塞着。到处可以看到排挤他的又敏捷又机警的人,他们灵活老练,闪动着发亮的深沉的眼睛,带着傲慢的神气朝他冷笑……在柏林中学毕业之后,他没有找到工作。柏林一些公司的经理和业主对他说,没有空缺,可是艾希曼从旁边了解到,有的公司没接收他,却接收了一个很不像样的不知是什么民族的人,也许是波兰人,也许是意大利人。他想上大学,但是大学里对人的态度很不公正,他上不了。他看到,考试人员一看见他的浅色眼睛和圆圆的脸、浅色的平头、又短又直的鼻子,就没有劲了。似乎他们喜欢的是长脸、黑眼睛、佝偻腰、窄肩膀的人,喜欢没出息的人。回到外省老家的人不只是他一个。这是很多人的命运。柏林一直有一类人,这一类的人在社会各个阶层都有。但是这一类人大多数是在崇尚世界主义、失去民族特点的知识分子中间,他们不分德国人和意大利人,不分德国人和波兰人。

这是很特殊的一类人,是一个很奇怪的种族,他们最聪明,最有学问,最能冷眼旁观。这类人所发出的朝气蓬勃的、非侵略性的思想威力给予人的强烈感觉是可怕的。这种威力表现在这些人的奇怪的爱好中,表现在他们的日常生活中,他们在生活中注意时髦,却又不修边幅,似乎不看重时髦;表现在他们对动物的热爱中,喜爱动物却与他们纯粹的城市生活方式相结合;表现在他们的抽象思维能力方面,他们善于抽象思维的同时,却又十分喜欢艺术和生活中粗犷的东西……这些人推进了德国的染料化学和氮合成化学,推进了强射线研究和优质钢的生产。就因为他们,外国的学者、艺术家、哲学家和工程师们纷纷来到德国,但正是这些人最不像德国人,他们在全世界到处游荡,他们的友好交往完全不是德国需要的,他们的德国人特征太不鲜明。

一个外地公司的职员怎么能出人头地呢,能够填饱肚子就不错了。可是现在你瞧他手里的文件,这文件在世界上只有三个人知道,那就是希特勒、希姆莱、卡尔津布伦涅尔[16]。他把文件锁进保险柜,走出自己的办公室。一部老大的黑色轿车正在门口等着他。卫兵向他敬礼,副官给他打开车门,党卫军少校艾希曼上了车。司机开大了油门,这部大马力的警察要员专用车便飞驰起来,一路上只见城里的警察恭恭敬敬对汽车行礼,急急忙忙打开绿灯,汽车穿过一条条柏林街道,便上了公路干线。冷雨,晨雾,喇叭声,公路缓缓地盘旋转弯。

此刻,在斯莫列维奇,在果树丛中是一座座幽静的小房子,人行道上长着青草。在加尔季切夫商场的街道上,涂了紫色或红色记号的肮脏的黄色爪子的母鸡在灰土中跑来跑去。在基辅的波多尔区和瓦西里科夫,在有很多肮脏的玻璃窗的多层楼房里,楼梯被孩子和老人千万次的步履磨得光光溜溜。

在敖德萨,院子里长着花皮悬铃木,晒着花连衣裙、褂子和裤子,煮果酱的铜盆在火盆上冒着热气,还没见过太阳的黑皮肤婴儿在摇篮里啼哭。

在华沙,狭窄的六层楼房里住着裁缝、装订工人、家庭教师、夜酒吧和咖啡馆的歌手、大学生、钟表匠。

在斯大林道尔弗,傍晚农舍里生起炉火,风从彼列科普方向吹来,夹带着盐味和暖和的尘土味,老牛哞哞叫着,晃悠着沉重的大头……

在布达佩斯,在法斯托夫,在维也纳,在梅利托波尔和在阿姆斯特丹,在玻璃窗明净如镜的别墅里,在工厂烟雾笼罩的房屋中,居住着犹太族的人们。

集中营的铁丝网、毒气室的墙、防坦克壕的黄土把千千万万人联接在一起,他们属于各种各样的年龄和职业,使用各种各样的语言,具有各种各样的生活和精神爱好,有信神的宗教狂热分子,也有无神论的坚定信徒,有工人,有游手好闲的人,有医生和商人,有聪明人,有白痴,有小偷,有喜欢空想的人,有冷眼旁观者,有好心人,有圣洁的人,也有卑劣的人,死神在等待着他们。

警察要员的大马力轿车一路奔驰着,在秋天的公路干线上不停地转着弯儿。

三十二

他们是在夜里见面的。艾希曼一面往办公室走,一面很快地询问着,径直走进办公室,坐到安乐椅上。

“我的时间不多,最迟在明天我要上华沙去。”

他已经去过集中营警备队,和建筑工地主任谈过。

“工厂的情况怎样,您对福斯这个人的印象如何,据您看,这些化学家有水平吗?”他很快地询问着。

艾希曼用他那长着粉红色大指甲的白胖的手指翻阅着桌上的文件,不时地用自来水笔做记号。利斯觉得,艾希曼并不认为这事与其他事有什么不同,虽然这种事情即便铁石心肠的人也要发冷发怵的。

利斯这几天喝了很多酒。气喘病加剧了,每天夜里他感到心跳得厉害。但是他认为,酒精对身体的害处不如神经紧张的害处大,而他是时时刻刻处在神经紧张状态中的。

他很希望重新去研究那些敌视国家社会主义的著名活动家的思想,解答那些冷酷、复杂然而不用流血的问题。到那时候他就不再喝酒了,一天顶多抽上两三支香烟。所以不久前一天夜里他把一个苏联的老布尔什维克叫了来,跟他下了一盘政治棋,他回到卧室以后,没用安眠药就睡着了,一直睡到上午九点多钟。

在夜间视察毒气室的时候,建设者们为艾希曼和利斯安排了一次别出心裁的小宴会。在毒气室中间放一张小桌,摆上酒和菜,赖内克请艾希曼和利斯饮酒。

艾希曼一见到这别出心裁的酒宴,就笑起来,说:

“我乐意从命。”

他把帽子交给自己的卫兵,就在桌旁坐下来。他的一张大脸忽然露出踌躇满志的样子,就像千千万万喜欢吃喝的男子坐上摆满山珍海味的宴席那样。

赖内克站着斟好了酒,大家都端起酒杯,等着埃·希曼致祝酒词。

在这水泥密闭室的寂静中,在斟得满满的酒杯里,有一种异常紧张的气氛,利斯觉得,他的心简直要经受不住了。他很希望高声祝愿德国理想早日实现的祝酒词打破紧张的气氛。但是紧张气氛非但没有打破,反而越来越紧张了。因为艾希曼正在吃火腿面包。

“先生们,你们怎么啦?”艾希曼问。“这火腿太好了。”

“我们在等待您的祝酒词呢。”利斯说。

艾希曼端起酒杯。

“祝咱们为党国效劳取得更大胜利,依我看,这是最值得祝贺的。”

只有他一个人几乎没喝,而是吃了很多。

早晨艾希曼穿着裤衩在打开的窗户前做了一会儿早操。晨雾中露出一排排整齐的集中营棚屋。火车汽笛声传来。利斯一向不羡慕艾希曼。利斯没有很高的职务,却有很高的地位—在帝国保安总部里都认为他是一个聪明人。希姆莱很喜欢和他交谈。上层的人在大多数情况下尽可能不在他面前显示自己官位高。他习惯于不仅在保安部门博得尊敬。到处都有帝国保安总部的影响和势力:在大学里,在儿童疗养院院长的签字中,在歌剧院招收年轻演员的考试中,在为春季画展评选作品的时候,在国会选举的候选名单里。

这里是生活的轴心。党之所以永远正确,党的道理或者没有道理之所以能战胜其他任何道理,党的哲学之所以能战胜其他一切哲学,主要靠国家秘密警察的工作。这真是一根魔杖!要是失手掉落了,魔力就消失了,伟大的演说家就会变为牛皮大王,学术巨著就会变为异端邪说。万万不能放下这根魔杖。

利斯这天早晨看着艾希曼,生平第一次感到自己萌发了嫉妒心理。艾希曼在离开之前几分钟说:

“利斯,咱们是同乡呀。”

他们谈起他们喜欢去的故乡城市的一些街道、饭馆、电影院。

“当然,有的地方我也没有去过。”艾希曼说。并且提到一个俱乐部,那地方他这个小业主的儿子过去是不能去的。

利斯想换个话题,就问道:

“请问,能不能大致地有个数,准备处理多少犹太人?”

他以为,他的问题问得过头了,也许,除了元首和希姆莱,世界上只有三个人能够回答他的问题。但是,在艾希曼回忆他年轻时在民主和风行世界主义的时代不得志的情形之后,利斯问他这种事,承认自己不知情,正是最恰当的时候。

艾希曼回答了他。

利斯非常震惊,又问一遍:

“是几百万吗?”

艾希曼耸了耸肩膀。

他们沉默了一阵子。

“咱们在学生时代不曾相识,非常遗憾,”利斯说,“如歌德说的,最好的是大学生时代。”

“我没有做过柏林的大学生,我是在外地上学的,您用不着感到遗憾。”艾希曼说。又补充说:“老乡,这个数目我是第一次说出来。如果算上在贝希特斯加登[17]、帝国内阁和元首府那几次,那这个数目总共说过七次或者八次。”

“我明白,我们不会在明天的报纸上看到这个数目的。”

“我指的就是报纸。”艾希曼说。

他带着冷笑的神气看了看利斯,利斯感到惶恐不安,因为他觉得艾希曼比他更聪明。艾希曼却说:

“除了咱们都是一个绿树丛中的宁静小城的同乡以外,我对您说出这个数目,还有一个原因。我希望,它能使我们在今后的共同工作中很好地配合。”

“非常感谢,”利斯说,“应当好好考虑考虑,事情是十分重大的。”

“当然啦。这主意不光是我的。”艾希曼竖起一个指头朝着上面。“如果您能跟我合作,万一希特勒失败了。那咱们就一起上吊。”

“前景是十分美好的,值得考虑。”利斯说。

“可以设想,两年后我们再坐在这房间里的舒适的小桌旁,就可以说:我们用二十个月的时间解决了人类用二十个世纪没有解决的问题!”

他们告别了。利斯目送着汽车。

他对于人与人在国家中的关系有自己的观点。在实行国家社会主义的国家中,生活不能自由发展,生活的每一步都必须加以控制。

为了指导人的呼吸、母亲的感情,指导如何读书、唱歌、夏天旅游,领导工厂和军队,就需要有许多领导者。因为生活不能像野草一样随便生长,不能像大海一样随便翻腾。利斯认为,领导者可以分为四种性格类型。

第一种类型:性格单纯的人,一般缺乏敏锐的智慧和分析的能力。这些人从报纸和杂志上摘取口号和公式,从希特勒的讲话、戈培尔的文章、佛朗哥和罗森堡的书中寻找理论根据。一旦感到失去支柱,就会不知所措。他们不考虑各种现象的联系,在任何问题上都表现得激烈和偏执。他们不论对待哲学、国家社会主义的科学、似是而非的新发现,还是对待新戏剧的成就、新的音乐、国会选举运动,都十分顶真。他们像小学生一样,读书死记硬背,听报告、看书都要做笔记。他们的个人生活一般都十分简朴,有时甚至很贫困,他们往往比其他类型的人更积极地响应党的号召,离开家庭。

利斯起初以为艾希曼正是属于这种类型。

第二种性格类型:聪明的无耻之徒。这些人知道魔杖是存在的。他们在可靠的朋友圈子里讥笑很多人,讥笑新博士和硕士不学无术,讥笑各级长官的错误和习性。他们不讥笑的只有领袖和崇高理想。这些人一般生活都很阔绰,他们有的是酒喝。这些人在党内占据高位的比职位低的多。在下层当权的主要是第一种性格类型的人。

利斯认为,在最高层掌权的是第三类性格的人。最高领导层掌权的不过八九个人,再有十五至二十人相配合。那儿另是一番天地,不再有什么信条,可以自由地裁判一切。那儿不再有理想,只看是否有利于我,只求称我心意,翻云覆雨,心狠手辣,不惜任何手段。

有时候利斯觉得,在德国所发生的一切都是为了他们和他们的利益。

利斯发现,头脑简单的人出现在最高层,往往标志着不祥事件的开端。这少数翻云覆雨的高手们提拔一些恪守信条的人,为的是让他们干特别血腥的事情。恪守信条的老实人暂时会受到最高层的赏识和犒劳,但是等到完成了任务,一般都要销声匿迹,有时会落得和自己的牺牲者一样的下场。最上层又是只有几个翻云覆雨的高手了。

第一种性格类型的老实人具有特别可贵的品质:他们具有人民性。他们不光摘引国家社会主义大师们的语句,也说人民的语言。他们的粗暴是人民的粗暴,农民的粗暴。他们说的笑话会在工人大会上引起一阵阵笑声!

第四种性格类型:奉命行事的人。他们对信条、思想、哲学丝毫不感兴趣,但也没有什么分析能力。国家社会主义党给他们薪俸,他们就为党效劳。他们追求的唯一的、最高的目标就是吃、穿、别墅、珠宝、家具、小汽车、冷气设备。他们不大喜欢金钱,不相信金钱的可靠性。

利斯向往最高领导层,希望和最高领导者交往,和他们接近,在高层里,在玩弄心计、进行文的较量的地方,他感到得心应手,轻松自如,非常得意。

但是利斯看到,在可怕的高层,在一些最高的领导者之上,在那一层之上还有一个隐隐约约、模模糊糊、不易理解、不依逻辑行事的世界,领袖希特勒就在这个最高世界里。

不知为什么,许多无法结合的特点汇集于希特勒一身:他是许多高手的头儿,是超级技师,特等装修工,总监工,其阴险毒辣甚至超过他所有的亲密助手的总和。利斯害怕的正是这一点。况且,在希特勒身上还有教条式的狂热、宗教式的信仰和盲目性,又像老牛一样的不讲道理,这些特点利斯只是在最低层的党的领导者中间见到过。他是魔杖的创作者,是头号圣人,同时又是极其愚昧和狂热的信徒。

现在,利斯目送着汽车渐渐远去的时候,他觉得艾希曼忽然使他隐隐产生了一种又害怕又羡慕的感觉,过去使他产生这种感觉的只有一个人,那就是德国人的领袖希特勒。

三十三

重新建立起来的部队在夜间秘密地朝斯大林格勒前线移动。

在斯大林格勒西北,顿河中游,新战线的兵力越来越密集。一列列军车就在草原上停靠,部队在重新铺好的铁路沿线上下车。

天一开始放亮,夜里如奔腾的河流似的铁路线就安静下来,只有淡淡的尘雾笼罩在草原上。白天,炮身用干枯的野草和麦秸掩盖着,似乎世界上再没有什么东西比这些与秋日的原野融为一体的炮身更沉静的了。一架架飞机张着翅膀,像僵死的昆虫似的停在机场上,上面覆盖着网状掩蔽物。

在那幅全世界只有几个人能看到的地图上,三角符号、菱形符号和圆圈一天比一天稠密,标志番号的数字也越来越稠密。这是新的西南战线—也就是现在的进攻战线—各部队在编队,聚集,开向出发的地界。

坦克兵团和炮兵师避开硝烟弥漫的斯大林格勒,顺着伏尔加右岸空旷的盐碱地带朝南开去,开向一处处安静的河湾。军队渡过伏尔加河以后,在加尔梅克草原上,在湖汊之间的盐碱地上驻扎下来,成千上万的俄罗斯人说起他们都觉得奇怪的话……这是在战场南边,在加尔梅克草原上集结兵力,面对德军的右翼。苏军最高指挥部正准备包围保卢斯的斯大林格勒集团军。

一艘艘轮船、渡船和驳船在秋日的星光下,在黑沉沉的夜色中,把诺维科夫的坦克军渡向斯大林格勒以南的右岸。

成千上万的人看到用白漆涂在钢甲上的俄罗斯古代将领的姓氏:“库图佐夫”、“苏沃洛夫”、“亚历山大·涅夫斯基”。

成千上万的人看到,苏联的重炮、火箭炮和从盟国租借来的武器一齐向斯大林格勒涌去。

虽然千百万人看到了这样的调动,集结大量兵力准备进攻斯大林格勒西北面和南面的行动还是在秘密中进行着。

怎么会出现这种事呢?德国人也知道这种大规模的调动。要遮掩是不可能的,就好比一个人走在草原上,遮不住草原上的风。

德国人都知道苏军在向斯大林格勒调动,可是进攻斯大林格勒对于他们依然是秘密。每一个德军的尉官只要看到地图上标出的苏军集结地点,都会猜出只有斯大林、朱可夫和华西列夫斯基知道的苏方的最高军事机密。

可是,德军在斯大林格勒地区被围,不论对德军尉官们还是对德军元帅们,都是非常突然的。

这怎么可能呢?

斯大林格勒依然没有失守,虽然投入了大量兵力,德军多次进攻依然没有取得决定性的胜利。而在消耗殆尽的苏军的一些团里,也只剩下几十名战士。这承担起残酷战斗的超级重负的少数人正是使德国人思想产生迷乱的原因。

敌人不能设想,他们强大的兵力会被一小堆人打碎。在他们看来,苏军的后备力量似乎只是在准备增援苏联守军。在伏尔加河畔抗击保卢斯集团军进攻的战士们成了斯大林格勒进攻战的战略家。

而历史的无情的魔力隐藏得还要深些。在这里面,自由是可以产生胜利的。自由仍然是战争的目的,而一旦触碰到历史有魔力的手指,它便成了历史得心应手的工具。

三十四

一个老妇人抱着一捆干芦苇朝家门口走去,她的阴沉的脸流露出一副操心的神气。她从一部落满灰尘的吉普车旁边走过,又从军部的一辆坦克旁边走过,坦克上盖着帆布,一个角紧靠着房子的板墙。她瘦得皮包骨头,样子很不起眼,似乎再没有什么比这个从她家门前的坦克旁边走过的老妇人更平常的了。可是,这个老妇人,还有此时在棚子底下挤牛奶的模样平平的女儿,还有把一个指头杵到鼻孔里、看着牛奶从奶头里往外窜的她的浅色头发的外孙,却和驻扎在草原上的军队有重要关系,其重要程度超过世界上一切大事。

所有这些军队上的人:军部、集团军司令部的少校,坐在黑糊糊的乡下圣像下面抽香烟的将军,在俄罗斯炉灶上烧羊肉的将军们的炊事员,躲在仓库里用子弹和钉子做发卷儿的电话员姑娘,在院子里对着洋铁洗脸盆刮脸、一只眼看着镜子、一只眼看着天空留意着敌机的坦克手们—这钢铁、电力和汽油组成的整个战争世界,已成为一座座草原村庄长期生活的不可分割的一部分。

对于老妇人来说,这里还有一种不可分割的关系:她看到今天在坦克上的小伙子们,就想起夏天那些疲惫无神的小伙子,那些小伙子步行来到这里求宿,一个劲儿担惊害怕,夜里都不睡,不时地到外面观望。

加尔梅克草原村落里的这个老妇人,和在乌拉尔给后备坦克军军部送铜茶炊的老妇人,和六月间在沃罗涅日把麦秸铺在地上让上校睡觉、一面望着窗外红红的火光画着十字的老妇人,都有不可分割的关系。不过这种关系已经习以为常,所以不论是要回屋里生炉灶的老妇人,还是走出门来的上校,谁都没有注意到。

加尔梅克草原上异常宁静,使人心旷神怡。这天早晨在柏林大街上走来走去的人是否知道,俄罗斯在这里已经把自己的脸转向西方,准备进攻和出击了?

诺维科夫在台阶上唤来司机哈里托诺夫:

“把我和政委的大衣带上,咱们要很晚才能回来。”

格特马诺夫和涅乌多布诺夫走出门来。

“涅乌多布诺夫同志,”诺维科夫说,“要是有什么情况,您打电话给卡尔波夫,下午三点以后,就打电话给别洛夫和马卡罗夫。”

涅乌多布诺夫说:“会有什么情况呢?”

“那可说不定,也许司令员一下子来了呢。”诺维科夫说。

从太阳那边出现了两架铁鸟,朝村子飞去。飞得越来越快,响声越来越大,草原的安静一下子就被打破了。哈里托诺夫从汽车里跳出来,朝仓房的墙根下跑去。

“傻瓜,怎么,躲起自己的飞机来啦?”格特马诺夫喊道。

这时候其中一架飞机用机枪朝村子扫射起来,另一架飞机投下一枚炸弹。呼啸声,轰隆声。妇女尖叫起来,小孩子哭起来,爆炸掀起的土块纷纷往地上落。

诺维科夫听到炸弹下落的啸声,弯了弯身子。有一小会儿,一切都笼罩在灰尘与硝烟中,他能看见的只有和他站在一起的格特马诺夫。接着涅乌多布诺夫的身影也从灰尘与硝烟中露了出来。他直着身子、昂着头站在那里,像是木雕的,只有他没有弯下身子。

格特马诺夫脸色有些灰白,但是又兴奋,又快活,一面打裤子上的灰土,一面带着洋洋得意的自夸口气说:

“没什么,还行,裤子还没有湿,咱们的将军甚至连动都没有动呢。”

然后格特马诺夫和涅乌多布诺夫去看炸弹坑周围的土飞得多么远。他们吃惊的是,远处房屋上的玻璃大都碎了,最近的房屋上的玻璃却好好的。他们又看了看倒下的篱笆。

诺维科夫觉得这两个第一次看到炸弹爆炸的人很有意思,看样子他们吃惊的是,把这枚炸弹造出来,带上天空又扔到地上,目的只有一个:炸死格特马诺夫的孩子的父亲和涅乌多布诺夫的孩子的父亲。原来,人在战场上就干这种事儿。

格特马诺夫坐上汽车以后,一个劲儿在谈这次空袭,后来自己打断自己的话,说:

“诺维科夫同志,你听我说这些话,也许觉得好笑,你遇到上千次轰炸,我这是头一回呀。”接着又换了话题,问道:“我问你,那个克雷莫夫好像被俘过吧?”

诺维科夫说:“克雷莫夫吗?你问他干什么?”

“我在方面军司令部听到说起过他,说得很有意思。”

“他被围困过,至于被俘,好像没有。说他什么了?”

格特马诺夫没听到诺维科夫的话,捅了捅司机的肩膀,说:

“顺着这条大路可以到第一旅旅部,不用过那条沟。你瞧,我在战场上也是有眼力的。”

诺维科夫已经习惯了,格特马诺夫在交谈时从来不跟着对方走:一会儿他自己说,一会儿提问题,一会儿又是他说,一会儿又问起什么。似乎他的思想走的是没有规律的曲线。不过,看起来好像是这样,实际上却不是这样。格特马诺夫常常谈起自己的老婆和孩子,随身带着很厚的一摞家人的照片,两次派人上乌法去送东西。可是他马上就爱上了卫生所那个很凶的黑发女医生塔玛拉·巴甫洛芙娜,而且爱得很深。有一天早晨维尔什科夫很痛心地对诺维科夫说:“上校同志,女医生夜里在政委那儿睡的,天快亮时候才出来。”

诺维科夫说:

“维尔什科夫,这不是您管的事。您别偷偷拿我的水果糖就好了。”

格特马诺夫不隐瞒他和塔玛拉·巴甫洛芙娜的关系,就是这会儿在草原上,他也把肩膀靠在诺维科夫身上,小声说:

“诺维科夫同志,有一个小伙子爱上咱们的女医生啦。”他带着亲热和惆怅的神气看了看诺维科夫。

“那是个政委。”诺维科夫说着,拿眼睛瞟了瞟司机。

“这也没什么,布尔什维克又不是和尚,”格特马诺夫小声说,“你要知道,我这个老糊涂蛋爱上她啦。”

他们沉默了几分钟。格特马诺夫又说起话来,似乎刚才说那一番推心置腹、亲密无间的话的不是他。

“诺维科夫同志,你到了你熟悉的前方环境里,一点没有瘦。可是,就拿我来说,我天生是做党的工作的材料。我是在最艰难的一年到州党委工作的,如果是别人,会累出肺痨病的:粮食计划没有完成,斯大林同志两次打电话找我,可是我即使有点儿不自在,照样发胖,就像在疗养院一样。你现在就是这样。”

“鬼知道我是干什么的材料,”诺维科夫说,“也许,我当真是打仗的材料吧。”

他笑起来。

“我发现,一看到什么有趣的事儿,我首先就想,别忘了对叶尼娅说说。刚才德国佬向你和涅乌多布诺夫扔下第一颗炸弹,我就想:一定要对她说说。”

“要作政治汇报吗?”格特马诺夫问道。

“就是,就是。”诺维科夫说。

“老婆嘛,当然啦,”格特马诺夫说,“老婆总是最亲近的。”

他们来到第一旅驻地,下了汽车。

在诺维科夫的脑子里经常有一长串的人、姓名、地名、大大小小的任务、明白的事和不明白的事、下达的和取消的指示。

夜里他有时忽然醒来,犯起愁来,他很怀疑:该不该进行超出瞄准器射程标尺刻度的远程射击?在行进中射击是否合适?各排排长是否能迅速而准确地判断战局的变化,独立决策,瞬息间发出命令?

然后他想象,一队一队的坦克冲破德军和罗马尼亚军队的战线,冲进缺口,进行追击,和强击航空大队、自行炮队、摩托化步兵和工兵联合在一起,不断地向西推进,夺取渡口、桥梁,绕过布雷区,攻向敌人防御中心。他高兴激动得把两条光光的腿从床上荡下来,坐在黑暗中,兴奋得喘粗气。

他从来不想把夜里自己的一些想法告诉格特马诺夫。

他在草原上比在乌拉尔的时候更经常对格特马诺夫和涅乌多布诺夫感到恼火。

他在心里说:“你们是专拣甜饼子吃的。”

他已经不是一九四一年那样子了。他比以前喝酒喝得多。他常常骂娘,常常发火。有一次他差点对燃料供应处处长动手。

他看到,有些人很怕他。

“他妈的谁知道我是不是天生打仗的材料,”他说,“不过顶好还是跟自己喜欢的娘们儿住在森林小屋里。白天去打打野味,晚上回来。她做好了吃的,吃过就睡觉。战争可是不能养活人。”

格特马诺夫侧歪着头,仔细看了看他。

第一旅旅长卡尔波夫上校圆滚滚的脸,红头发,晶亮的蓝眼睛,这样的眼睛只有头发很红的人才有。他在战地无线电台旁边碰到了诺维科夫和格特马诺夫。

他的作战经历有一段时间和西北战线的战斗有关系;在那里,卡尔波夫不止一次把自己的坦克埋到土里,把坦克变成固定的火力点。

他和诺维科夫、格特马诺夫一起朝第一团驻地走去,那神气就好像他是主要首长,他的动作是那样从容。

从他的体质来看,似乎他应该是一个喜欢喝酒和美食的和气人。但他却是另外一种性格:不爱说话,对人很冷淡,器量又小,又多疑。他从不热情招待客人,是一个出了名的小气鬼。

格特马诺夫称赞了他们为坦克和大炮挖掘掩蔽所的认真态度。

这位旅长什么都考虑到了,既考虑了坦克威胁的方向,又考虑到侧翼进逼的可能性,他只是没有考虑到,即将开始的战斗可能让他带领全旅迅速地冲进缺口,转向追击。

诺维科夫看到格特马诺夫又点头又说话表示赞许,十分生气。

可是卡尔波夫就好像故意给诺维科夫火上浇油似的,说:

“上校同志,请允许我来说说。在敖德萨我们就隐蔽得很好。那天傍晚我们发起反攻,狠狠打了罗马尼亚人一顿,到夜里遵照集团军司令员的命令,我军像一个人似的进入海港,上了轮船。罗马尼亚人到上午十点钟才猛醒过来,急忙进攻已被我们遗弃的战壕,可是我们已经在黑海上的轮船上了。”

“你们现在面对的不是罗马尼亚人的空战壕啊。”诺维科夫说。

卡尔波夫能不能在进攻时期日日夜夜地往前冲,把敌人的作战部队、防御中心抛在后面?……能不能不顾自己的前方后背、左右侧翼,一心只想着追击,一直往前冲?他不是那种性格,不是的。

周围的一切依然带着已经过去的暑热的痕迹;奇怪的是,空气如此凉爽。坦克手们干着士兵们的家常事:有的把小镜子搁在炮塔上,坐在钢甲上刮脸,有的在擦枪,有的在写信,有的在地上铺了帆布,在上面打扑克牌,有一大堆小伙子闲着没有事儿,围着一位卫生员姑娘说笑。在辽阔的天空下、广袤的大地上的这幅平常的画面,充满了黄昏前的惆怅情调。

这时候,一位营长朝着走到跟前的三位首长跑来,一面跑一面抻平制服上衣,尖声喊着:

“全营立正!”

诺维科夫就像和他作对似的,回答说:

“稍息!稍息!”

在政委随便说着话走过的地方响起笑声,坦克手们互相看了看,他们的脸显得更快活了。政委问一些人,离开乌拉尔的姑娘,心里什么滋味;又问,是不是一写信就写很多张纸;还问,在草原上能不能天天收到《红星报》。

政委狠狠批评了军需官。

“弟兄们今天吃的什么?昨天吃的什么?前天吃的什么?你这三天也是吃大麦米加青番茄汤吗?好吧,把炊事员叫来,”他在坦克手们的一片笑声中说,“让他说说,他给军需官做什么吃的。”

他一再询问坦克手们的生活条件和生活情形,好像是责备队列军官不关心士兵生活:

“你们这是怎么回事儿,光知道操心战术,战术。”

军需官是一个痩痩的人,穿着落满灰土的胶布靴子,一双手通红通红的,好像洗衣妇的手,刚刚在冷水里涮过衣服。他站在格特马诺夫面前,不住地咳嗽。

诺维科夫可怜起他来,就说:

“政委同志,咱们是不是一块儿从这儿上别洛夫那儿去?”

格特马诺夫从战前起,就不愧是一个很好的群众工作者和领导者。他一开始说话,人们就开始笑,他的话简单明了,生动活泼,还常常带上几句粗话,一下子就会抹掉州委书记和穿着肮脏工装的普通人之间的界限。

他常常关心生活问题:是不是能按时领到工资,乡村商店和工人合作社有没有次货,宿舍里暖气设备好不好,田间宿营地是否筑好了炉灶?

他和上了年纪的工厂女工和农庄女庄员说话特别随便,特别和善,大家都很欣喜地看到,书记是人民的勤务员,他常常严厉地批评管供应的人,批评公共宿舍的保卫人员,如果工厂厂长和农机站站长不关心干活儿的人,他也一样毫不留情地谴责。他是农民的儿子,自己也在工厂里做过钳工,工人们都能感觉到这一点。但是他在自己的州党委办公室里操心的却总是他对国家负的责任,莫斯科的忧虑是他的主要忧虑;关于这一点,大工厂的厂长们知道,农村区委书记们也知道。

“你在破坏国家的计划,明白吗?党证你想要不要?你可知道,党委托给你的是什么?还有什么说的?”

在他的办公室里,没有人笑,没有人说玩笑话,也不谈公共宿舍里的开水或者车间的绿化。在他的办公室里批准硬性的生产计划,谈的是提高生产定额,谈的是住房建筑暂缓进行,要把腰带勒得更紧些,更坚决地降低成本、提高零售商品价格。

当他在州党委主持会议的时候,特别能显示出这个人的本事。在这些会议上常常会出现一种感觉,所有的人不是带着自己的想法和要求到他的办公室里来的,而是为了来帮助格特马诺夫,整个会议进程事先已经由格特马诺夫的毅力、智慧和意志安排定了。

他说话声音不高,从容不迫,他相信听他说话的人都在专心地听着。

“你说说你那个区的情形,同志们,咱们让农业专家发发言。如果你,彼得·米海洛维奇,能补充补充,就更好啦。让拉齐科说说吧,他在这方面不是十分顺利。你,罗季昂诺夫,我看出来啦,也想发发言;同志们,依我看,问题很清楚啦,可以做结论啦,我想,不会有什么反对意见。同志们,这儿有一份决议草案,罗季昂诺夫,你念念吧。”

罗季昂诺夫本来想表示怀疑,甚至想争论争论的,这一来就很用心地念起决议,一面侧眼看着会议主席,担心自己是不是念错了字句。

“就这样吧,同志们都没有意见。”

不过,最了不起的是,格特马诺夫在要求各个区委书记完成计划的时候,在削减农庄劳动日可怜的报酬的时候,在降低工人工资的时候,在要求降低成本、提高零售价格的时候,在很感动地和农村妇女谈话,表示同情她们生活困难的时候,在看到工人住房拥挤表示难过的时候,他都能显得很真诚,很自然。

这是很难理解的。不过,难道现实中所有的事情都那么容易理解吗?在诺维科夫和格特马诺夫走到汽车跟前的时候,格特马诺夫对送他们的卡尔波夫开玩笑说:

“我们只有在别洛夫那儿吃午饭了,您和您的军需官的午饭我们就吃不成了。”

卡尔波夫说:

“政委同志,目前还没有让军需官动用前方仓库的东西。至于他本人,顺便说说,他什么也不吃,正在害胃病。”

“害胃病,哎呀呀,那可真糟。”格特马诺夫说着,打了一个呵欠,把手一挥。“好啦,我们走啦。”

别洛夫旅与卡尔波夫旅相比,向西挺进了很远。

别洛夫瘦瘦的,大鼻子,两条腿弯弯的,又长又粗。他头脑灵活机敏,说话像开机关枪一样。诺维科夫很喜欢他。

诺维科夫认为他是生就的坦克军里猛冲快攻的好手。

虽然参加战斗的时间不长,他博得的评价是很好的。十二月里他在莫斯科附近对敌人后方进行过坦克袭击。

可是现在诺维科夫很不放心,只看这位旅长的毛病:酗酒,放荡,追逐女人,健忘,得不到下属的爱戴。别洛夫没有采取防御措施。看样子,别洛夫不关心这个旅的物质技术供应问题。他关心的只是燃料和弹药的供应。至于如何修理坦克,如何从战场上撤出受损伤的坦克,他也不够关心。

“您这是怎么啦,别洛夫同志,不管怎么说,这不是在乌拉尔,是在草原上呀。”诺维科夫说。

“是啊,就像一群茨冈人,营地太不像样子了。”格特马诺夫补充说。

别洛夫马上回答说:

“在防空方面,我采取了措施;至于地面的敌人,并不可怕。我认为,在这样的后方,敌人不可能来。”

他吸了一口气,说:

“不希望防守,一心想往前冲。等着心里憋得难受,上校同志。”

格特马诺夫说:

“好样的,别洛夫,好样的。真是当今的苏沃洛夫,真正的大将之材。”然后把称呼换成“你”,用亲热的口气小声说:“政治部主任告诉我,好像你和卫生所的一位护士勾搭上啦,是真的吗?”

别洛夫因为听到格特马诺夫的亲热口气,一下子没有明白问题的严重性,就问道:

“对不起,他说什么了?”

不过,不等对方重复,那句话就进入了他的意识,他不好意思起来。

“我也是个男子汉呀,没办法,政委同志,天天在野地里嘛。”

“可是你有老婆,还有一个孩子呀。”

“三个。”别洛夫带着忧愁的神气纠正说。

“噢,你瞧,三个孩子呢。指挥部撤掉了第二旅的一名很好的营长布兰诺维奇,采取了严厉措施,在出发之前派科贝林接替了他,不过就是因为这样的事儿呀。你给下属做的什么样子,嗯?还是苏联军官,是三个孩子的父亲呢。”

别洛夫恼了,大声说:

“这事儿怪不得哪一个,因为我没有强迫她。做这种榜样的有您,有我,也有您的爹。”

格特马诺夫没有提高嗓门儿,却把称呼又换成“您”,说:

“别洛夫同志,别忘了您是党员。在上级首长和您说话的时候,要好好地站着。”

别洛夫换成军人的完全像木头一样的姿势,说:

“对不起,政委同志,我当然明白,当然能认识到。”

格特马诺夫对他说:

“我相信你在军事上是有成绩的,军长也相信你,只是不要在个人生活上出问题。”他看了看表。“诺维科夫同志,我要回军部去,不能和你一起上马卡罗夫那儿去了。我借用一下别洛夫的汽车。”

等他们走出掩蔽所,诺维科夫憋不住,问道:

“怎么,想塔玛拉了吗?”

格特马诺夫带着使人不解的神气用冷冷的眼睛看了看他,用不满意的口气说:

“方面军军委委员有事找我呢。”

诺维科夫在回军部之前,又去看了他很喜欢的第三旅旅长马卡罗夫。

他们一块儿朝湖边走去。有一个营驻扎在湖边。

马卡罗夫脸色苍白,眼睛流露着忧郁的神气,似乎这样的眼睛不可能属于一个重型坦克旅旅长,他对诺维科夫说:

“上校同志,在德国佬赶着我们在芦苇丛里到处跑的时候,白俄罗斯那片沼地,您还记得吗?”

诺维科夫记得白俄罗斯那片沼地。

他想了想卡尔波夫和别洛夫。显然,问题不仅在于经验,还在于天性。应该让指挥员们取得他们所缺乏的经验。但是无论如何不应该压制他们的天性。不能把歼击航空兵调为工兵。不是所有的人都像马卡罗夫一样,既能守,又善攻。

格特马诺夫说自己天生是做党的工作的材料。那么,马卡罗夫就是当兵的材料。不能派错了用场。马卡罗夫呀,马卡罗夫,真是一员好战将!

诺维科夫不希望听马卡罗夫汇报。他喜欢和他商量,和他交换意见。在进攻中怎样配合步兵和摩托化步兵,配合工兵,配合自行炮炮兵?在进攻开始后,他们对敌人的意图和行动的推测是否彼此相符?他们对敌人防坦克力量的估计是否一致?怎样才能正确地确定展开兵力的界线?

他们来到营指挥所。

指挥所在一条不深的干沟里。营长法托夫一看到诺维科夫和旅长,就觉得不好意思,因为他觉得营部的掩蔽所太不像样子,不配接待这样的高级客人。而且还有一名战士拿火药撒在木柴上生火,炉子里哧啦哧啦响着,好像有意使人难堪。

“同志们,咱们要记住,”诺维科夫说,“咱们这个军将担负的是整个前线最重要的一部分任务,我又把其中最困难的部分交给了马卡罗夫,据我所知,马卡罗夫又把自己任务中最复杂的部分交给了法托夫。至于怎样完成任务,这是你们自己需要考虑的。我在战斗中不会把自己的决定强加给你们。”

他向法托夫询问了怎样跟团部和各连连长进行联系的问题、电台工作情况、弹药数量问题、发动机检修问题、燃料质量问题。

在分手之前,诺维科夫说:

“马卡罗夫,全准备好了吗?”

“没有,上校同志,还没有完全准备好。”

“再有三天能行吗?”

“上校同志,能行。”

诺维科夫坐上汽车以后,对司机说:

“哈里托诺夫,怎么样,马卡罗夫这儿好像一切都像个样子吧?”

哈里托诺夫侧眼看了看诺维科夫,回答说:

“上校同志,这儿的样子吗,当然啦,一个个都像样得很。食品供应处处长喝得醉醺醺的,营里有人来领压缩食品,可是他睡觉去了,把钥匙带走了。等到把他找了来,他又找不到钥匙了。一位司务长对我说,连长把弟兄们的酒都领了去,给自己过命名日,把酒全喝光了。我想把备用车胎补一补,可是他们连胶水都没有。”

三十五

涅乌多布诺夫将军在军部的房屋里朝窗外看了看,在一团灰尘中看到了军长的吉普军,非常高兴。

在他小时候,有一天大人都出门去了,他觉得一个人在家里没有人管束了,十分高兴,可是,把门一关上,他就觉得好像有贼,好像失火了,于是他从门口到窗口来来回回地走着,呆呆地听着,拿鼻子嗅着,闻闻有没有烟味。

现在他也体验到这种束手无策的感觉,过去他管理大事的一些方法,在这里全用不上。

万一敌人突然来了呢?要知道,从军部到前方也只有六十公里。在这儿不能用撤职来吓唬坦克,不能谴责坦克和阶级敌人有关系。要是坦克一个劲儿地猛冲过来,拿什么来阻挡坦克呢?这种显而易见的事情,却使涅乌多布诺夫感到十分惊讶—国家愤怒的威力曾经使千千万万人服服帖帖,心惊胆战,现在,在这前线上,在德国人冲过来的时候,竟一钱不值了。德国人不填写履历表,不在大会上交代自己的历史,也不必因为父母在革命前的经历担惊受怕。

他所喜欢、所依靠的一切,他的命运和他的孩子们的命运,已经不在伟大而威严、他觉得可亲可爱的国家保护之下了。于是他第一次带着不好意思和友好的心情想到诺维科夫。

诺维科夫一走进军部的房子,就说:

“将军同志,我看到了,马卡罗夫是好样的!他在任何情况下都能够独立地解决突然出现的问题。别洛夫可以不顾一切地往前冲,别的事他不懂。至于卡尔波夫,则是一个慢性子、没有冲劲儿的人,需要督促。”

“是啊,是啊,干部决定一切嘛。要时时考察干部,这是斯大林同志教导我们的。”涅乌多布诺夫说。又很快地说:“我一直在想,这小镇上有德国间谍,今天早晨一定是这暗藏的坏家伙招引飞机来轰炸咱们军部。”

涅乌多布诺夫在对诺维科夫说起军部的一些事情时,说:

“现在有友邻部队和加强部队的一些指挥官要上咱们这儿来,没什么特别事儿,只是来认识认识,拜访拜访。”

“很遗憾,格特马诺夫上方面军司令部去了。谁知道他去干什么?”诺维科夫说。

他们约定一起吃午饭。诺维科夫便朝自己的住处走去,洗了脸,换换落了许多灰尘的上衣,宽宽的小镇街道上空荡荡的,只有炸弹坑旁边站着一个老头子,正是诺维科夫的房东老大爷。老人家伸着两条胳膊在弹坑旁边测量着,就好像这弹坑是挖出来派什么用场的。诺维科夫走到他跟前,问道:

“老大爷,您在这儿干什么?”

老人家像当兵的那样行了一个军礼,说:

“首长同志,一九一五年我做过德国人的俘虏,在德国给一个女主人干过活儿。”他指了指弹坑,然后又指了指天空,挤了挤眼睛。“这一定是那一家的少爷,狗崽子,飞来啦,来看我呢。”

诺维科夫大笑起来:

“哎哟,您这老人家!”

他朝格特马诺夫住的房子看了看,看到那面窗子上的护窗还关着。他朝台阶上的岗哨点了点头,忽然想道:“格特马诺夫上方面军司令部去干他妈的什么?他究竟有什么事?”他心中闪过一个惴惴不安的念头:“真是一个伪君子,他怎么能责备别洛夫行为不端呢,他自己就和塔玛拉有事嘛,真是可怕。”

但是诺维科夫马上就觉得这种想法是没有根据的了,他不是生性多疑的。他拐过屋角,看到一块空地上有几十个小伙子,可能是区兵役局动员的新兵,正在水井旁边休息。

带领这些小伙子的一名士兵,因为走累了,用军帽蒙着脸,睡着了,在他旁边是堆得像小山一样的包裹和提箱。小伙子们显然走了不少路,腿脚累了,有几个小伙子脱光了鞋袜。他们的头还没有剃光,远看很像一群农村的学生,正在课间休息。他们瘦瘦的脸、细细的脖子、淡黄的头发、用父亲的上衣和裤子改做的带补丁的衣服,所有这一切都带有孩子气。有几个人在玩着孩子们的传统游戏,当年这位军长也玩过的:在远处挖一个小坑,眯起一只眼睛,瞄一瞄,拿铜板朝小坑里扔。其余的小伙子在看着他们玩儿。只有他们的眼睛不像小孩子的眼睛,流露着惶惶不安和忧愁的神气。

他们发现了诺维科夫,就朝睡觉的士兵看了看,看样子,是想问问他,在这位军队首长从他们旁边走过的时候,他们能不能扔铜板,能不能照样坐着。

“玩吧,小伙子们,玩吧。”诺维科夫用温和的声音说着,并且朝他们招了招手,便走了过去。

他心中涌起一股剧烈的怜悯,这股感情来得异常猛烈,他甚至因此感到张皇失措。大概是这一张张痩瘦的、大眼睛的孩子气的脸,这寒碜的农村服装,一下子干脆了当地说明白了:这都是一些孩子,一些小孩子……在军队里,孩子气和天性往往隐藏在军帽底下,隐藏在军姿中,靴子的吱咯声和经过磨练的动作言语中。现在这一切却赤裸裸地表露在外面。

他走进房里。奇怪的是,在今天的一些复杂不安的想法和观感之中,最使他忧虑的是他看到了这些孩子新兵。

“有生力量,”诺维科夫自言自语说,“这就叫有生力量呀,有生力量。”

他在军队里这么多年,只知道害怕上级责备他损失技术装备和弹药,责备他延误时机,责备他不爱护机器、马达、燃料,责备他擅自放弃制高点和要道口……还没有见到过上级领导听说战斗中损失了大量有生力量而真正动气的。有时候一个领导者把大批的人推到炮火下,为的是免得上级领导发火,并且可以为自己辩护,把两手一摊,说没有办法呀,我已经把一半人力用上去,可是还是无法夺取指定的阵地。”

有生力量啊,有生力量。

他有几次看到,有些领导把有生力量赶到炮火下,甚至不是为了逃避责任或者形式主义地执行命令,而是为了逞雄,固执己见。战争的秘密及其悲剧性,就是一个人有权力叫另一个人去死。这种权力所依靠的基础是:人们为了共同事业,可以赴汤蹈火。

诺维科夫有一个朋友,本是一个通情达理的指挥员,他在前沿观察所的时候也不愿改变自己的习惯,每天要喝新鲜牛奶。每天早晨都有第二梯队的士兵冒着敌人的炮火用暖水瓶给他送牛奶。有时德军把送牛奶的士兵打死了,诺维科夫的那个朋友,那个好人,就没有牛奶喝了。到第二天,又派另外的士兵冒着炮火用暖水瓶给他送牛奶。这个通情达理、关怀下属的好人心安理得地喝他的牛奶,他手下的士兵都称他父亲。这种事,实在令人难以理解。

不一会儿,涅乌多布诺夫就来找诺维科夫。诺维科夫一面对着小镜子匆忙而细心地梳理头发,一面说:

“将军同志,是啊,战争总归是很可怕的事!把一些小孩子赶来补充兵力了,您看到吗?”

涅乌多布诺夫说:

“是啊,这样的部队太嫩,太年轻了。我把那个带队的兵叫醒了,我说要把他送到惩戒连里去。他也不管管他们。不像什么军队,乱糟糟的,简直是乌合之众。”

在屠格涅夫的小说里有时写道,一个地主新来安家,邻近的地主纷纷前来拜访。天黑时有两部小汽车来到军部门前,主人便出来迎接客人:来客是炮兵师师长、榴弹炮团团长和火箭炮旅旅长。

……亲爱的读者,咱们手挽着手,一同去我的芳邻达吉雅娜·鲍里索芙娜的庄园吧……[18]

诺维科夫已经从前方的一些故事和指挥部的通报中熟悉了上校炮兵师长,甚至能清清楚楚地想象出他的外表:紫红色脸膛,圆圆的脑袋。可是,他原来已经上了年纪,而且腰背也佝偻了。

上校那一双愉快的眼睛似乎错误地安到了一张忧郁的脸上。有时他的眼睛笑得那样有神,似乎这双眼睛才是上校的灵魂,而那皱纹、那弯腰弓背本来就不应该和这双眼睛连接在一起。

榴弹炮团团长洛帕津不仅可以被看做炮兵师长的儿子,甚至可以被看做他的孙子。

火箭炮旅旅长马基德是一个黑脸汉子,翘翘的上嘴唇上有一抹黑黑的小胡子,因为过早地谢顶,额头显得很高,他是一个能说会道、喜欢俏皮话的人。

诺维科夫把客人带进屋里,桌上已经摆好了酒菜。

“请尝尝乌拉尔口味。”他指着碟子里的腌蘑菇和醋渍蘑菇说。本来做出很优美的姿势站在餐桌旁的炊事员,一下子红了脸,噢呀一声,便走开了,他觉得难为情。

维尔什科夫凑到诺维科夫耳朵上,指着桌上,小声说:

“来吧,把酒瓶打开。”

炮兵师师长莫罗佐夫用指甲比着玻璃杯上四分之一往上一点儿的地方,说:

“无论如何不能再多,我的肝不好。”

“您呢,中校同志?”

“我身体好着呢,斟满吧,没问题。”

“我们的马基德可是好样儿的。”

“少校同志,您的肝怎么样?”

榴弹炮团团长洛帕津用手捂着自己的杯子,说:

“谢谢,我不喝酒。”

他把手移开,又说:

“象征性地斟一点点儿吧,咱们好碰杯。”

“洛帕津是学前儿童,喜欢吃糖。”马基德说。

他们祝贺共同作战取得胜利,一齐把杯干了。于是,像常有的场合一样,大家谈起和平时期彼此都相识的大学和中学里的同学。

大家又谈到前线的领导,谈到驻扎在秋季寒冷的草原上何等凄凉。

“怎么样,快结婚了吧?”洛帕津问道。

“是要结婚了。”诺维科夫说。

“是啊,是啊,我们的‘卡秋莎’到哪儿,哪儿就可以举行婚礼。”马基德说。

马基德坚信他指挥的火箭炮具有决定性作用。一杯酒下肚之后,他流露出一副强者爱护弱者的神气,话里话外嘲讽,怀疑,自视颇高,这令诺维科夫十分反感。

诺维科夫近来常常在心里估量,叶尼娅会怎样看待前方这个人或那个人,他在前方的这个或那个战友如果和叶尼娅在一起,会说些什么,会有什么样的表现。

诺维科夫觉得,如果马基德见了叶尼娅,一定会缠住不放,装腔作势,又吹牛,又说笑话。诺维科夫感到不安,感到有妒意,似乎马基德在拼命向叶尼娅卖弄聪明,似乎叶尼娅正在听他的俏皮话。他也想向她显示显示自己的聪明,他想说说,了解和认识同自己并肩战斗的人,事先能判断出他们在战斗环境中的所作所为,有多么重要。他想说说,对卡尔波夫就需要督促,对别洛夫就需要劝阻,至于马卡罗夫,不论进攻或防守,都是一样地迅速、灵活,应付裕如。

毫无意思的闲谈引起了争论。在不同兵种的指挥官之间常常会出现这样的争论。争论虽然很热烈,不过从实质上说,也是没有多大意思的。

“是啊,人需要的是指引和教导,强迫其改变心意是不应该的。”莫罗佐夫说。

“人需要的是坚定不移的领导,”涅乌多布诺夫说,“不应该怕负责任,应该把责任承担起来。”

洛帕津说:

“谁没有到过斯大林格勒,就根本算不上见过战争。”

“不过,对不起,”马基德反驳说,“斯大林格勒又怎么样?英勇,顽强,坚决,这我不抬杠,抬杠是好笑的!我虽然没有到过斯大林格勒,但是我可以大胆地说,我见过战争。我是进攻的军官,参加过三次进攻,可以说,我亲自冲锋,亲自冲进突破口。我的火箭炮发挥了威力,不仅超越了步兵,而且超越了坦克,也可以说,超越了空军。”

“哼,中校同志,说什么超越坦克,您算了吧,”诺维科夫恼火地说,“坦克是运动战的主人,这是没有话说的。”

“还有一种十分简单的办法,”洛帕津说,“在胜利的时候把一切归于自己。在失败的时候把一切推给友邻部队。”

莫罗佐夫说:

“唉,友邻呀,友邻,有一次,步兵部队的一位将军请求我用炮火支援他。‘快,朋友,请向那边的高地发炮。’‘用多大口径的?’他却骂起娘来,说:‘开炮就是了,别管那一套!’后来才了解,原来他既不知道口径,也不知道射程,而且连地图也看不明白,只知道:‘开炮,开炮,打他妈的……’对下属只知道叫喊:‘往前冲,要不然把你的牙打掉,老子枪毙你!’可是却自认为掌握了战争的全部奥妙。这也算友邻部队长官,就请您多多关照吧。而且你还要归他统制呢,他是将军嘛。”

“唉,对不起,您说的话和我们的情况毫不相干,”涅乌多布诺夫说,“在苏联部队里没有这样的指挥官,更没有这样的将军!”

“怎么没有?”莫罗佐夫说。“打了一年仗,我遇到的这种自作聪明的人有多少呀,他们只知道拿手枪吓唬人,骂娘,毫无意义地把人赶到炮火下面。就比如不久前,有一位营长简直哭着说:‘我干吗要赶着人去叫机枪扫?’一位将军师长握起拳头对着这位营长吆喝:‘要么你马上带人去冲,要么我把你当狗一样打死。’于是他带着人冲上去,就好像带着牲口上屠宰场。”

“是啊,是啊,这就叫做:为所欲为,”马基德说,“将军们为所欲为不光在这方面,他们随随便便糟蹋电话员姑娘。”

“他们写两个字至少要有五个错误。”洛帕津说。

“就是,就是,”莫罗佐夫没有听清楚就说,“跟他们在一起作战就要多流血。他们的本事就在于不怜惜人。”

莫罗佐夫的话引起诺维科夫的同感。他在军队里这么多年,经常遇到这类的事情。

他忽然说:

“怎么能怜惜人呢?如果一个人怜惜人的话,他就不应该来打仗了。”

今天他看到那些孩子新兵,心里十分难受,他很想说说他们的事。可是他并没有说出他的一片好心的话,而是带着一股突如其来的、连自己也莫名其妙的恼恨和粗暴劲儿又接着说:

“这怎么能怜惜人呢?战争所以是战争,就是不能怜惜自己,也不能怜惜别人,主要的问题是:不等把人训练好就编进军队,就把重要的装备交给他们。请问,该怜惜谁呢?”

涅乌多布诺夫拿眼睛很快地打量了一遍大家的脸。

涅乌多布诺夫曾经毁掉不少好人,就像此刻坐在桌旁的这样的人。诺维科夫忽然产生一种使他吃惊的想法:此人可能制造的不幸,也许不次于在前沿阵地上等待着莫罗佐夫,等待着他诺维科夫,等待着马基德、洛帕津和今天在小镇上休息的农村小伙子们的不幸。

涅乌多布诺夫用教训的口气说:

“这不符合斯大林的教导。斯大林同志教导我们说,最宝贵的是人,是我们的干部。我们最宝贵的财产是干部,是人,应当像爱护眼珠一样爱护他们。”

诺维科夫看到,大家听了涅乌多布诺夫的话,露出赞许的表情。他心里想:“这就有意思了。我在他们眼里成了禽兽中的禽兽,涅乌多布诺夫却成了怜惜人的人。很遗憾,格特马诺夫不在这儿,他可是更像一位圣人。我和他们在一起,总是这样。”

他打断涅乌多布诺夫的话,已经是非常粗暴、非常恼恨地说:

“咱们的人是很多的,装备却很少。任何一个笨蛋都会造人,不像造坦克、造飞机。如果要怜惜人的话,就别担任指挥官!”

三十六

斯大林格勒方面军司令叶廖缅科上将召见坦克军的领导人诺维科夫、格特马诺夫、涅乌多布诺夫。

昨天叶廖缅科上各旅里去过,但是没有去军部去。

应召前来的几个人坐在这里,侧眼看着叶廖缅科,不知道他要和他们谈什么。叶廖缅科发现格特马诺夫在打量小床上皱皱巴巴的枕头,就说:“脚疼得厉害。”并且用粗话骂起自己的脚。大家都没有说话,一齐看着他。

“总的说,你们军准备工作做得不坏,已经准备好了。”叶廖缅科说。

他在说这话的时候,看了看诺维科夫,可是诺维科夫听到司令员的称赞并没有露出喜色。叶廖缅科觉得有点儿奇怪:一位军长受到难得夸奖人的司令员的夸奖,反应竟如此淡漠。

“上将同志,”诺维科夫说,“我已经向您报告过,集中在草原干沟地带、准备加入本军编制的一三七坦克旅,一连两天遭到我们的强击航空部队的轰炸。”

叶廖缅科眯起眼睛,在揣测他的用心:是想撇清自己呢,还是在控告空军指挥官?

诺维科夫皱起眉头,又说:

“幸亏没有击中。他们不会轰炸。”

叶廖缅科说:

“那也罢了。他们还要支援你们的,他们会弥补自己的过失。”

格特马诺夫插话说:

“司令员同志,我们当然不会和斯大林的空军发生什么争执。”

“就是,就是,格特马诺夫同志。”叶廖缅科说,并且问:“噢,怎么样,您见过赫鲁晓夫吗?”

“赫鲁晓夫同志吩咐我明天去。”

“他是在基辅认识您的吗?”

“司令员同志,我和赫鲁晓夫同志一起工作差不多有两年。”

“请问,将军同志,是不是有一次我在季齐安·彼得罗维奇家里看到过你?”

“是的,”涅乌多布诺夫回答说,“那一次是季齐安·彼得罗维奇把您和沃罗诺夫元帅一起叫去的。”

“不错,不错。”

“上将同志,我有一段时期依照季齐安·彼得罗维奇的要求暂时担任人民委员。所以我常常上他家里去。”

“就是嘛,我看着面熟嘛。”叶廖缅科说。他想对涅乌多布诺夫表示一下自己的好意,就又说:“将军同志,你在草原上不觉得寂寞吧,我想,居住条件不坏吧?”

他还没有听到回答,就满意地点了点头。

等到三个人要出门的时候,叶廖缅科又唤了诺维科夫一声:

“上校,你过来。”

诺维科夫从门口转回来,叶廖缅科欠起身来,把他那发了胖的农民的身体抬高到桌子上方,唠叨说:

“你瞧,一个和赫鲁晓夫在一起工作过,一个和季齐安·彼得罗维奇一起工作过,可是你,是大兵出身,狗崽子,要记住:你要带领全军完成突破任务。”

三十七

在一个寒冷而阴暗的早晨,克雷莫夫出了医院。他不回驻地,径直去见方面军政治部主任托谢耶夫将军,汇报自己这次来斯大林格勒的情形。

克雷莫夫很走运—托谢耶夫从早晨起就在自己的衬了灰色木板的办公室里,并且立即接见了克雷莫夫。

这位政治部主任的外表与他的姓氏相符[19]。他侧眼看着不久前晋升将军后穿上的新的将军服,抽着鼻子,闻着来人身上发出的医院的石碳酸气味。

“因为负伤,我没有完成‘6—1’号楼的任务,”克雷莫夫说,“现在我可以再上那里面去。”

托谢耶夫用不满的目光狠狠看了看克雷莫夫,说:

“不用了,您给我写一份详细的报告吧。”

他没有提任何问题,对于克雷莫夫的汇报既不表示赞成,也不指责。正如往常一样,在这寒碜的农舍里,将军服和勋章显得十分奇怪。不过,奇怪的不光是这一点。

克雷莫夫无法理解,他有什么地方使上级领导这样阴沉,这样不满意。克雷莫夫来到政治部总务处领取饭票,交验食品供应卡,办理出差回来的手续,补办住医院的手续。在办公室里的人为他办理手续的时候,他坐在凳子上,打量着男男女女工作人员的一张张脸。

这里没有人对他感兴趣,他从斯大林格勒回来,他的负伤、他的所见所闻、他经历的一切都没有什么意义,什么也算不上。总务处的人都忙着办事情。打字机嘀嗒嘀嗒,办公纸沙啦沙啦,工作人员的眼睛在克雷莫夫的身上微微一扫,就又埋进打开的文件夹和堆在桌上的文件里。

有多少皱得紧紧的额头!一双双眼睛里流露着多么紧张的思考神情,多么专心致志,那翻阅文件的手,动作多么从容、多么熟练!

偶尔突然焦躁不安地打一个呵欠,偷偷地很快看一眼手表(是不是快到午饭时间了?),这双或那双眼睛里有时会出现淡淡的灰色阴影—只有这些现象能说明在这沉闷的办公室里,这些人有多无聊和苦闷。

克雷莫夫熟识的政治部第七科的一位指导员来办公室里看了看。克雷莫夫便和他一起到过道里抽烟。“您回来啦?”指导员问。“是的,回来啦。”

因为指导员没有问他在斯大林格勒见到什么和干了一些什么,他便开口问道:

“你们政治部有什么新闻?”

主要的新闻是,旅级政委在重新评定中终于得到了将军头衔。这位指导员带着嘲笑的口气说,托谢耶夫盼望这新的队列头衔,都急得生病了,因为他早就请军队里最好的裁缝做好了将军服,可是等呀等呀,莫斯科老是不给他将军头衔。有一种可怕的说法,说是在重新评定中有些团级和一级营政委将得到大尉和上尉头衔。

“您想想看,”这位指导员说,“像我这样,在部队的政工机关干了八年,得一个尉官头衔,能想得通吗?”

还有一些新闻。政治部情报科副科长奉命回到莫斯科,回到总政治部,得到提拔,被任命为卡里宁前方面军司令部政治部副主任。

政治部的所有一级指导员以前是在科长级食堂就餐的,现在根据军委委员指示,待遇与一般指导员相同,在普通食堂就餐。还有一道指示,要出差的人交出就餐券,也不发给他们干粮。曾经为前线报社的诗人卡茨和塔拉拉耶夫斯基申请红星勋章,但是根据谢尔巴科夫的新指示,前方新闻工作人员的奖励必须通过总政治部,所以两位诗人的材料又送到莫斯科,这时候前方的获奖名单已经由司令员批过了,被批准的名单上的获奖人已经在举杯庆祝自己得政府奖了。

“您还没有吃饭吧?”这位指导员问道。“咱们一块儿去吃饭。”

克雷莫夫说,他还在等着办手续。

“那我先去了。”指导员说。并且在临走时很随便地开玩笑说:“要抓紧时间,要不然咱们就要上军人商店食堂去拼命,去和非军职人员,和打字员姑娘们一起吃饭了。”

一会儿,克雷莫夫也办好了手续,来到外面,吸了一口秋天的潮湿空气。

为什么政治部主任用那样阴沉的脸色迎接他?有什么地方使这位主任不满意?是克雷莫夫没有完成任务?是政治部主任不相信克雷莫夫负伤,怀疑他胆怯?是因为克雷莫夫越过顶头上司直接来见他,而且不是在接待时间,所以他生气?是因为克雷莫夫两次称呼他“旅级政委同志”,而没有称呼他“少将同志”?也许,这与克雷莫夫无关,而是因为别的什么事?是因为托谢耶夫没有得到库图佐夫勋章?是收到了告知妻子生病的家信?谁又能知道,为什么政治部主任这天上午心情这样坏?

克雷莫夫在斯大林格勒待了几个星期,已经不习惯这阿赫图巴河中游地方的情形。政治部领导人和同事们的冷漠目光,食堂服务员们的冷漠目光,他已经很不习惯了。在斯大林格勒可不是这样!

黄昏时候他回到自己住的屋子。主人家的狗非常热情地欢迎他。那狗好像是由不同的两半拼成的:后面一半的毛是棕红色的,而长长的头是黑白相间的。狗的两半都在表示欢迎:棕红色的毛茸茸的尾巴不住地摇着,黑白相间的头扎到克雷莫夫的手里,用和善的棕色眼睛很亲热地看着他,在朦胧的暮色中,似乎是两只狗在和克雷莫夫亲热。狗和他一起进了过道。正在过道里忙活的女房东很生气地对狗说:“该死的,滚出去!”然后才像政治部主任那样,阴沉着脸和克雷莫夫打招呼。

住过了斯大林格勒那可亲可爱的、用防雨布做门的土室,那潮湿的、烟气腾腾的掩蔽所,他觉得这安静的小屋、这罩了白枕套的枕头、这挑花窗帘是那么不舒服,那么冷清。

克雷莫夫坐到桌旁,开始写报告。他写得很快,偶尔查看一下在斯大林格勒的记录。最不容易写的是有关“6—1”号楼的情况。他站起来,在屋子里踱了一会儿,又坐下来,马上又站起来,走到过道里,咳嗽了几声,听了听:鬼老婆子难道连茶水都不供应了?然后他用小罐子从桶里舀了一些水,水很好喝,比斯大林格勒的水好多了。他回到屋里,坐下来,手里握着钢笔,想了一会儿。然后他躺到床上,合上眼睛。

究竟是怎么一回事儿?是格列科夫对他开了枪!

在斯大林格勒,他和人们的联系和亲近感总是越来越强,他在斯大林格勒呼吸非常轻松。在那儿没有阴沉的、对他冷淡的目光。他进入“6—1”楼房,似乎更强烈地感受到列宁的气息。可是他到那里面以后,马上就觉得他们对他嘲笑,不怀好意,他就生起气来,要纠正他们的思想,吓唬他们。他为什么要说起苏沃洛夫?格列科夫对他开了枪!他今天感到特别孤独,看到一些人的傲慢和高人一等的态度就受不了,他认为这些人不过是半文盲,是不干正事的家伙,在党内不过是乳臭未干的小儿。在托谢耶夫面前立正站着有多难受啊!可以感觉出他那气愤的、时而露出嘲笑、时而露出蔑视意味的目光。要知道,论党内资格,托谢耶夫连同他的官衔和勋章,还不抵克雷莫夫一个手指头。都是一些和列宁传统无关的、在党内得势的小人!他们之中有许多人是在一九三七年爬上来的,靠的是写秘密报告,揭发人民敌人。他忽然想起他在地道里朝一点阳光走去时那种美好的刚强、自信、轻松的感觉。

他甚至气愤得喘不上气来,他认为是格列科夫不叫他过那种理想的生活。他在去那座楼房的时候,觉得自己时来运转,十分高兴。他觉得,列宁的传统就在那座楼房里。格列科夫却朝列宁式的布尔什维克开了枪!是他让克雷莫夫回到阿赫图巴河边的办公室,回到龌龊的生活中!可恨的家伙!

克雷莫夫又在桌边坐下来。他写的没有半句谎话。

他把写好的文字看了一遍。不用说,托谢耶夫会把他的报告交给特别科。格列科夫从政治上瓦解了一个战斗的排,并且进行暗杀活动,向党代表和政委开枪。会把克雷莫夫传去作证,和被捕的格列科夫对质。

他想象着格列科夫坐在侦查官桌子前面的样子:胡子老长,脸色黄中带灰,连腰带也没有。

格列科夫说的“你很苦恼”,怎么办,在报告里不好写啊。

马克思列宁主义党的总书记被公认为是绝对正确的,几乎是神圣的!在一九三七年斯大林毫不怜惜老资格的列宁式的战士。他破坏了党的民主与铁的纪律相结合的列宁主义精神。

那样残酷地杀害列宁主义党的党员,能够设想吗,这对吗?不过,格列科夫是要当众枪毙的。杀自己人是可怕的,而格列科夫不是自己人,他是敌人。

克雷莫夫从不怀疑党有权使用专政之剑,从不怀疑革命具有消灭一切敌人的神圣权力。他也从来不同情反对派!他从来不认为布哈林、雷科夫、季诺维也夫和加米涅夫走的是列宁主义路线。托洛茨基虽然智慧过人,虽然具有光辉的革命气质,可是依然不能根除过去的孟什维克观点,没有提高到列宁主义的高度。真正有本事的是斯大林!所以大家都称他主人。他的手从来不发抖,他没有布哈林那种知识分子的优柔寡断性格,列宁缔造的党粉碎一批又一批敌人,跟着斯大林不断前进。格列科夫的军功算不上什么。跟人民敌人没什么可争论的,不必去听他们的什么道理。可是,不论克雷莫夫怎样激发自己的仇恨,此时此刻他对格列科夫再也恨不起来了。他又想起了,“您很苦恼”。“这算什么,”克雷莫夫想道,“怎么,我这不是告密吗?尽管不是捏造,但总是告密……没办法呀,好同志,你是党员嘛……那就尽党员的责任吧。”

第二天上午,克雷莫夫把自己写的报告送交方面军司令部政治部。

过了两天,政治部宣传鼓动科科长、团级政委奥基巴洛夫代替政治部主任召见了他。托谢耶夫在接见刚从前方来的坦克军政委,所以不能亲自接见他。面色苍白、大鼻子、精明而干练的团级政委奥基巴洛夫对克雷莫夫说:

“克雷莫夫同志,过一两天,您还要上右岸去走一趟,这一次是上舒米洛夫的六十四集团军去。凑巧,我们有一部汽车要上州党委指挥所去,您再从州党委指挥所过河上舒米洛夫那儿去。州党委书记要上别克托夫镇去参加庆祝十月革命节大会。”

他不慌不忙地向克雷莫夫交代了派他去六十四集团军政治处的任务。任务非常琐碎,非常乏味,包括收集书面材料,不是实际工作需要的材料,而是供办公室统计数字用的。

“是不是还去做作报告?”克雷莫夫问道。“我遵照您的指示准备了十月革命的报告,想到部队里去做几次报告。”

“暂时缓一缓吧。”奥基巴洛夫说。并且说了说为什么暂时不要克雷莫夫作报告。在克雷莫夫准备要走的时候,奥基巴洛夫对他说:

“您的报告在这里,竟有这样的事,政治部主任把情况对我说了。”

克雷莫夫的心发起怵来:大概,格列科夫的案子已经交办了。这时奥基巴洛夫又说:

“你们那位好汉格列科夫很走运,昨天第六十二集团军政治处主任向我们报告,格列科夫在德国人进攻拖拉机厂的时候牺牲了,和他手下所有的弟兄一起牺牲了。”

他为了安慰克雷莫夫,又说:

“集团军司令提请追认他为苏联英雄,不过现在很明显,我们会把这事压下来的。”

克雷莫夫把两手一摊,好像在说:“好啦,走运倒是走运,反正没办法了。”奥基巴洛夫压低了声音说:

“特别科科长认为,他可能还活着。可能跑到敌人那边去了。”

克雷莫夫回到家,看到一张纸条:要他上特别科去。看样子,格列科夫的案子还没有了结。克雷莫夫决定等出差回来再去特别科进行这场不愉快的谈话。反正人已经死了,没什么可以着急的了。

三十八

在斯大林格勒南部的别克托夫镇上,州党委决定在造船厂举行隆重的集会,庆祝十月革命二十五周年。

十一月六日清早,斯大林格勒州党委的一些领导人来到伏尔加左岸的橡树林里,在州党委的地下指挥所里会齐。州委第一书记、各部门书记、州党委委员们吃完了三道菜点的热腾腾的早饭,便坐上汽车,出了橡树林,上了通向伏尔加河的大路。

坦克和大炮在夜间前往图马克南渡口走的就是这一条路。被战争的炮火打得坑洼不平的草原上,到处是冻实的黄泥块和结了冰的水洼,景象十分凄凉。伏尔加河里漂流着冰块,冰块的沙沙声在离岸边几十米以外的地方都能听得见。正刮着下游来的狂风,在这样的日子乘坐无遮无盖的铁驳船渡过伏尔加河不是什么快活事儿。

等待渡河的红军战士穿着被伏尔加河的冷风吹得鼓起来的军大衣,坐在驳船上,一个个紧紧靠在一起,尽可能不挨到冷冰冰的钢铁。牙齿咯咯地敲打着,腿蜷缩着,等到阿斯特拉罕方向的强劲冷风一吹过来,人就冻僵了,连呵手指头、揉自己的腰、揩鼻涕的劲儿都没有了。驳船烟囱里冒出来的烟被撕成一片一片的,铺在伏尔加的上空。那烟因为有冰做底衬,显得特别黑,那冰也因为有驳船的烟做低低的天幕,显得特别白。流冰从斯大林格勒的岸边带来战争的声音。

一只大头乌鸦停在一大块冰上沉思着。是有些事情值得思考。旁边一大块冰上有一片烧剩的士兵大衣的衣襟,还有一大块冰上有一只冻得像石头一样的毡靴,一支卡宾枪,弯弯的枪筒子冻进了冰里。州委书记和党委委员们的一部部小汽车在朝驳船上开。书记和委员们下了汽车,站在船边上,看着缓缓流动的冰块,听着冰块的沙沙声。驳船的老船长嘴唇发青,戴着红军士兵皮帽,穿着黑色小皮袄。他走到州委分管运输的书记拉克季昂诺夫面前,用河上的潮湿、多年的老酒和土烟磨练出来的非同寻常的嗄哑声音说:

“书记同志,早晨我们第一趟开船过河的时候,看到冰上躺着一个水手,同志们想把他弄下来,差点儿和他一起沉下去,只好用铁棍凿。那就是,在河岸上,用帆布盖着。”

老船长用肮脏的手套朝岸边指了指。拉克季昂诺夫看了看,没有看见从冰里凿出来的死者,他想用粗暴而不客气的问话掩盖自己的不自在,就指着天空问道:

“你们管他干什么?特别现在这是在什么时候?”

老船长把手一挥,说:

“现在是轰炸得很厉害呀。”

老船长骂了一声暂时没有轰炸的德国人,在骂德国人的时候,他的声音忽然一点儿也不嗄哑了,又响亮,又清脆。拖船拖着驳船缓缓地朝别克托夫镇和斯大林格勒之间的河岸驶去,那河岸好像不是战时的河岸,而是平时的河岸,挤满了仓库、棚屋和房舍。前去庆祝革命节的书记和委员们在冷风里站腻了,于是他们又坐进汽车。红军战士们隔着玻璃看着他们,就像在参观玻璃缸里的金鱼。坐在小汽车里的斯大林格勒州党委领导者们在抽烟,挠痒,聊天……隆重的庆祝会在夜里举行。铅印的请柬与和平时期的请柬的不同之处,只是在于易碎的灰色纸质地太差,请柬上也没有印出集会地点。

斯大林格勒州党委领导者们、从六十四集团军来的客人们、附近一些企业的工程师和工人们进入会场都是由熟悉道路的人带领着:“这儿拐弯,再拐弯,小心,这儿有弹坑,钢轨,小心点儿,这儿有一个石灰坑……”

在黑暗中到处可以听到说话声、脚步声。克雷莫夫白天过河后已经到了六十四集团军政治处,现在和六十四集团军的代表一起来参加庆祝会了。这些人在漆黑的夜里,在迷宫似的工厂区走着,像这样秘密而分散地进行活动,有点儿像在沙皇俄国庆祝革命节日。

克雷莫夫激动得喘着粗气,他知道,此时此刻他不用准备就可以作报告,他凭一个老练的群众宣传员的直觉可以感觉出来:大家和他一样激动,一样高兴,因为在斯大林格勒的英勇战斗很像俄国工人的革命斗争。

是的,是的,是的。动员起全民族的巨大力量的战争是为革命而进行的战争。他在被围困的楼房里谈起苏沃洛夫,并不是背离革命。斯大林格勒、塞瓦斯托波尔、拉季谢夫的命运、马克思宣言的威力、列宁在芬兰车站装甲车上的号召都是一致的。

他看到了普里亚欣。普里亚欣像往常那样慢悠悠的,不慌不忙。说来有些奇怪,他想和普里亚欣谈谈,却怎么也谈不成。

他到了州党委的地下指挥所,就马上去找普里亚欣,他有许多话要和他谈谈。但是却谈不成,电话铃声几乎响个不停,不时有人来找第一书记。普里亚欣忽然向克雷莫夫问道:

“有一位格特马诺夫,你认识吗?”

“我认识,”克雷莫夫回答说,“在乌克兰,在党中央,做过中央委员。怎么啦?”

但是普里亚欣什么也没有说。后来就忙着准备上车了。克雷莫夫不高兴的是,普里亚欣没有请他坐自己的汽车。他们有两次面对面碰到一起,普里亚欣就好像不认识他了,那一双眼睛又冷,又淡漠。

两位军人顺着明亮的走廊走来—一位是肥胖的、肩宽腰圆的集团军司令舒米洛夫,一位是棕色鼓眼睛的小个子西伯利亚人、集团军军委委员阿勃拉莫夫将军。克雷莫夫觉得,在两位将军经过的穿着军装、棉袄、皮袄的热腾腾的男子汉人群中有一股纯朴的民主气息,这种气息便是革命初期的精神,列宁精神。一踏上斯大林格勒的河岸,克雷莫夫又感触到这一点。

主席团就座。斯大林格勒市苏维埃主席皮克辛和所有的大会主席一样,把两手撑在桌子上,慢慢地朝着嚷嚷得最厉害的地方咳嗽了几声,就宣布斯大林格勒市苏维埃、党市委与部队代表、斯大林格勒工厂工人代表联合举行的庆祝伟大的十月革命二十五周年的大会开始了。

从硬邦邦的掌声中可以听出来,在这儿鼓掌的全是男子汉的手,士兵的手和工人的手。然后,大块头、大脑门、动作缓慢的第一书记普里亚欣开始作报告。他说不出早已过去的事情和今天的事情之间有什么联系。似乎普里亚欣在和克雷莫夫进行争论,他以自己思想的平缓地反驳克雷莫夫的激动。本州的企业正在按照国家计划进行生产。左岸的各农业区完成了国家的收购任务,尽管多少迟了一点儿,但基本上是令人满意的。在市内和市北的一些企业没有完成国家的任务,因为这些企业在交战地区。

就是这个人,当年曾经和克雷莫夫一起站在前线的群众大会上,从头上摘下帽子,高声叫喊:“战士同志们,弟兄们,制止血腥的战争!自由万岁!”现在他看着大厅里的人,说本州向国家交售的粮食数量减少了很多,是因为季莫夫区和科捷尔区无法完成交售任务,这两个区是战场,还有卡拉奇区和上库尔莫亚尔区全部或部分地被敌人占领了。

然后他又说,本州的群众一面为完成国家的任务继续劳动着,一面广泛地参加了反抗德国侵略者的战斗。他列举了劳动者参加民兵队伍的数字,又报了报因为出色地完成指挥部的任务并且在执行任务中表现英勇顽强而得奖的斯大林格勒人的人数,而且说明,这个数字是不够完全的。

克雷莫夫听着第一书记平静的声音,明白了,他的思想、感情与他所说的本州的工业和农业完成国家计划的话惊人地不一致,这不是毫无意义的,而是表现出他的人生目的。

普里亚欣用石头一样的冰冷口气在强调国家肯定无疑会取得胜利,却不知国家正依靠人民的苦难和向往自由的热衷而被保卫着。

一张张工人和军人的脸严肃而阴沉。

他想起斯大林格勒的人们,想起塔拉索夫、巴秋克,想起自己和被围的“6—1”号楼里的士兵的谈话,是多么奇怪而又令人痛心。想想死在被围楼房瓦砾中的格列科夫,心情是多么沉重啊。

格列科夫对他说那些难听的话,究竟是什么用心?格列科夫竟向他开枪。这位斯大林格勒州党委第一书记、这位老同志普里亚欣的话为什么这样不入耳,这样冰冷?多么奇怪而复杂的感情。

普里亚欣的报告快要结束了,他说:

“我们有幸可以向伟大的斯大林汇报,本州的劳动者完成了苏维埃国家交给自己的任务……”

听完报告之后,克雷莫夫一面随着人群朝门口移动,一面用眼睛寻找普里亚欣。在斯大林格勒鏖战的日子里,普里亚欣不应该这样作报告,不应该这样。

克雷莫夫忽然看到了他:普里亚欣从主席台上下来之后,和六十四集团军司令站在一起,用专注而阴沉的目光直直地朝克雷莫夫望着。他发现克雷莫夫也在朝他看,就慢慢把眼睛转过去。

“这是怎么一回事?”克雷莫夫想道。

三十九

庆祝大会散场之后,当天夜里克雷莫夫就搭乘顺路汽车来到斯大林格勒发电站。

这天夜里发电站的景象显得十分凄惨。昨天德军重轰炸机刚刚轰炸过发电站。炸得到处是大坑,掀起一堆一堆的土块。车间的窗玻璃一块也没有了,有的车间震塌了,三层的办公大楼也炸得不成样子。

油变压器烟气腾腾地燃烧着,懒洋洋地冒着牙齿似的不高的火焰。

担任门卫的一个格鲁吉亚小伙子领着克雷莫夫在院子里走着,院子里有火光照耀着。克雷莫夫发现,在抽烟的门卫小伙子的手指头打着哆嗦。重型炸弹不仅炸得石头楼房倒塌、燃烧,炸得人心里也乱了,也燃烧起来。

克雷莫夫自从得到前来别克托夫镇的命令那一刻起,就想着和斯皮里多诺夫见面的事。也许叶尼娅在这儿,在斯大林格勒发电站?也许,斯皮里多诺夫知道她的下落,也许他还收到过她的信,她在信的结尾写着:“您是不是知道克雷莫夫的什么情形?”他又激动又高兴。也许斯皮里多诺夫会说:“叶尼娅一直在想您呢。”也许他会说:“您要知道,她老是在哭呢。”从早晨起,他就急不可待地要上斯大林格勒发电站来。他很希望在白天来看看斯皮里多诺夫,哪怕待几分钟也好。但是他还是控制住自己,上六十四集团军指挥所去了,虽然集团军政治处一位指导员小声提醒过他:“您这会儿不必急着去见军委委员。他今天一早就喝醉了。”

果然不错,克雷莫夫不该急着去见将军,而没有在白天来看斯皮里多诺夫。他坐在地下指挥所等待接见的时候,听到军委委员在胶合板隔壁那边向打字员口述给友邻集团军司令崔可夫的祝贺信。

他在慷慨激昂地口述着:

“瓦西里·伊万诺维奇,好战士,好朋友!”

将军口述到这里,哭了起来,并且又抽搭着重复了好几遍:

“好战士,好朋友,好战士,好朋友……”

接着他厉声问道:

“你打的是什么?”

“瓦西里·伊万诺维奇,好战士,好朋友。”女打字员念道。

看样子,军委委员觉得她的平淡的语调很不合适,于是纠正她,用高亢的声音说:

“瓦西里·伊万诺维奇,好战士,好朋友!”

他又动了感情,嘟哝起来:“好战士,好朋友,好战士,好朋友……”

后来将军憋住泪水,又厉声问:

“你打的是什么?”

“瓦西里·伊万诺维奇,好战士,好朋友。”女打字员说。

克雷莫夫明白了,不必急着见他了。

此刻院子里的火光很不明亮,照不清道路,倒是把道路弄得混乱了,似乎这火是从地下钻出来的;也许是大地本身在燃烧—这低低的火焰是这样潮湿,这样沉重。他们走到发电站站长的地下指挥所跟前。落在不远处的炸弹炸起一座座高高的土丘,隐隐约约有一条还没有踩实的小路通向指挥所入口。门卫小伙子说:“您来得很巧,今天过节。”

克雷莫夫心里想,他想说的话不能当着别人的面对斯皮里多诺夫说,不能当着别人的面问。他让门卫小伙子把站长叫到外面来,就说方面军司令部有一个政委来了。等到剩下他一个人,他激动起来,怎么也镇定不下来。

“这是怎么了?”他在心里说。“我以为已经断了呢。难道战争也不能把感情冲干净?我这是干什么?”

“走吧,走吧,走吧,快走,要不然就完了!”他自言自语地说。

但是没有力气走,没有力气离开。

斯皮里多诺夫从地下指挥所走了出来。

“同志,有何事见教?”他用不高兴的口气说。

克雷莫夫问道:

“斯捷潘·费多罗维奇,不认识我啦?”

斯皮里多诺夫忐忑不安地说:

“这是谁呀?”

他盯着克雷莫夫的脸,忽然叫了起来:

“尼古拉,尼古拉·格里高力耶维奇!”

他使出猛劲儿用双臂搂住克雷莫夫的脖子。

“尼古拉,我的好兄弟。”他说着,鼻子酸了。

这次在瓦砾堆中的见面使克雷莫夫十分感动。他感觉到斯皮里多诺夫在哭。还是那样,还是那样……他从斯皮里多诺夫的信任和高兴中感觉出自己和叶尼娅一家的亲近,又在这种亲近中重新衡量了自己内心的痛苦。为什么,为什么她要走,为什么带给他这样大的痛苦?她怎么能这样做?

斯皮里多诺夫说:

“都是战争,战争毁了我的一切。我的玛露霞死了。”

他说起薇拉,说她在几天以前终于离开发电站,上伏尔加左岸去了。他说:

“她真是个傻孩子。”

“她丈夫在哪儿?”克雷莫夫问道。

“大概早已不在人世了。他是一个歼击机飞行员。”

克雷莫夫再也憋不住,问道:

“叶尼娅怎么样,还活着吗,在哪儿?”

“活着,不是在古比雪夫,就是在喀山。”

他看着克雷莫夫,又说:

“这可是最要紧的:活着!”

“是的,是的,当然,这是最要紧的。”克雷莫夫说。

可是他以前就不知道什么是最要紧的。他只知道自己心里还非常痛苦。他知道,和叶尼娅有关的一切,都会引起他的痛苦。不论他听说她心情愉快,无牵无挂,不论他听说她心情痛苦,遭遇不幸,他都一样难受。

斯皮里多诺夫说了说弗拉基米罗芙娜的情形,又说了说谢廖沙的情形、柳德米拉的情形,克雷莫夫只是不住地点头,小声嘟哝着说:

“是啊,是啊,是啊……是啊,是啊,是啊……”

“尼古拉,咱们走,”斯皮里多诺夫说,“上我家里去。我现在没有别的家了。就这儿是我的家。”

油灯的亮光照不亮摆满了床铺、橱柜、电话机、玻璃瓶和一袋袋面粉的地下指挥所。在贴墙的板凳、床铺、箱子上坐的都是人。在窒闷的空气中回响着嗡嗡的说话声。

斯皮里多诺夫给各人的玻璃杯、茶缸、饭盒盖子里斟满酒精。大家都安静下来,用一种特别的目光注视着他。这种目光深沉而严肃,毫无担心的意味,只有信任:相信他的公正。

克雷莫夫打量着在座的人的脸,心里想:“最好格列科夫也在这儿。最好也给他斟一杯。”可是格列科夫已经饮完了他应该喝的酒。他不能在人世上再喝了。

斯皮里多诺夫端着酒杯站了起来,克雷莫夫心想:“这一下糟了,他要像普里亚欣那样发表长篇大论了。”

可是斯皮里多诺夫拿酒杯在空中画了一个“8”字形,说:

“来吧,伙计们,干杯。祝大家节日快乐。”

玻璃杯叮当响,铁茶缸叮当响,干杯的人哼哧着,还把头直摇晃。

这儿有各种各样的人,国家在战前把他们安插在不同的地方,他们没有聚在一起饮过酒,没有互相拍过肩膀,没有说:“喂,你听着,我来对你说说。”但是在这里,在炸毁的发电站和燃烧的土地下面,却产生了纯正的兄弟情谊,为了这种情谊不惜牺牲生命。担任夜间打更的一个白发老头子唱起一支古老的歌儿,在革命前察里津的一家法国工厂里,小伙子们很喜欢这支歌儿。

他唱得很清脆,很响亮,依然是年轻时候的声音,因为他自己也觉得年轻时的声音太陌生,所以他听着自己的声音露出好笑和惊讶的神气,就好像在听别人唱。

还有一个黑头发的老头子,把眉头皱得紧紧的,很认真地听着这支倾诉爱情和爱的痛苦的歌儿。

是的,能听到歌声是很愉快的,这样的美好而可怕的时刻,像这样把站长、战地面包房的驼夫、更夫、门卫,将加尔梅克人、俄罗斯人、格鲁吉亚人联结成一体的时刻,是令人愉快的。

那个黑头发老头子等到老更夫把倾诉爱情的歌儿一唱完,又皱了皱深锁的眉头,慢慢地、无腔无调地唱了起来:

我们要打倒旧世界,把旧世界的灰烬从我们的脚上抖干净……

党委书记大笑起来,摇晃起脑袋,斯皮里多诺夫也边笑边摇头。克雷莫夫也笑了笑,向斯皮里多诺夫问道:

“这位老头儿大概以前是孟什维克吧?”

斯皮里多诺夫完全了解安德列耶夫的情况,他当然也可以对克雷莫夫说一说,但他怕的是尼古拉耶夫听到,而且纯朴的兄弟情感也暂时消退了,于是斯皮里多诺夫打断歌声,喊道:

“巴维尔·安德列耶维奇,离题太远啦!”

安德列耶夫马上就不唱了,看了看,然后说:

“我还以为没有离题呢。迷糊啦。”

担任门卫的格鲁吉亚小伙子让克雷莫夫看了看他的脱了皮的手。

“这是挖我的好朋友弄成这样的,他叫谢廖沙·沃罗比约夫。”他的一双黑眼睛亮起来。他喘着粗气,就像尖声喊叫似的说:“我喜欢谢廖沙,比亲兄弟还亲。”

老更夫已经喝醉了,满脸是汗,缠着党委书记尼古拉耶夫说:

“喂,你还是听我说,马库拉泽说他喜欢谢廖沙·沃罗比约夫,比亲兄弟还亲,这算什么!你可知道,我以前在煤矿里干活儿,东家有多么喜欢我,多么看得起我。他和我一块儿喝酒,我唱歌给他听。他当面对我说,你虽然是普通矿工,可是我拿你当亲兄弟看待。我们常在一块儿聊天,在一块儿吃饭。”

“那是一个格鲁吉亚人吧?”尼古拉耶夫问道。

“我才不管他是不是格鲁吉亚人。东家姓沃斯克列辛斯基,所有的矿都是他的。你可知道他多么看得起我呀。他有百万家产,可是为人真不坏。你懂吗?”

尼古拉耶夫和克雷莫夫交换了一下眼色,两个人都很幽默地挤了挤眼睛,摇了摇头。

“嘿,”尼古拉耶夫说,“这话不错。活到老,学到老嘛。”

“那你就学学吧。”老头子没有听出嘲笑意味,就认真地说。

这天晚上过得格外好。到了很晚的时候,等到大家都开始走了,斯皮里多诺夫对克雷莫夫说:

“尼古拉,不要穿大衣,别走了,在我这儿睡吧。”

他不慌不忙地给克雷莫夫铺床,一面考虑着底下铺什么:被子,棉衣,还是防雨布?克雷莫夫走出地下室,望着轻轻晃动的火焰,在黑暗中站了一会儿,又回到地下室里,可是斯皮里多诺夫还在给他铺床。

等克雷莫夫脱了靴子,躺下来,斯皮里多诺夫问道:

“怎么样,还舒服吗?”

他抚摩了一下克雷莫夫的头,亲热地、醉醺醺地笑了笑。

克雷莫夫看到上面燃烧着的火焰,不知为什么想起了一九二四年一月为列宁送葬的时候,夜里在志愿队伍里燃起的篝火。留在地下室里过夜的人好像都已经睡着了,漆黑漆黑的,什么也看不见。

克雷莫夫睁着眼睛躺着,没有注意黑暗,他想着,想着,回忆着……

那是冰天雪地的严寒的日子。受难教堂顶上是黑沉沉的冬日天空,许许多多的人头戴皮帽、布琼尼式军帽,身穿军大衣和皮夹克。受难广场忽然变成一片白,那是千万张纸,是政府的通告。

用农民的雪橇把列宁的尸体从哥尔克运往火车站。雪橇的滑铁哧哧响着,马匹打着响鼻。跟在棺材后面的是头戴毛皮圆帽、扎着灰头巾的克鲁普斯卡娅,列宁的妹妹安娜和玛利亚,他的好友们,哥尔克村的农民。在农村,为善良的脑力劳动者,为地方自治局派任医生和农艺师送葬,往往就是这样。

哥尔克村的列宁住宅静悄悄的。壁炉的瓷砖闪着亮光,在铺了白色被单的床边有一架小橱,小橱里摆满了带标签的小瓶,散发着各种各样的药味。一位穿白衣的上了年纪的女医生走进空空的房间里。她依然习惯性地踮着脚走路。女医生从床边走过,捡起凳子上的一段小绳子和捆在上面的一小片报纸,睡在椅子上的一只小猫听到玩具的熟悉的沙沙声,很快地抬起头来,看了看空空的床,便又打着呵欠躺下了。

走在棺材后面的亲人和亲近的同志们怀念着死者。两位妹妹回想着那个浅色头发的男孩,他的性子很执拗,有时爱讥笑人,对人要求很苛刻,但是他心肠是好的,他很爱妈妈和弟弟妹妹们。

妻子回忆着:在苏黎世,列宁蹲在地上,和女房东的小孙女季莉说话儿。女房东带着很可笑的瑞士口音用俄语说:“你们应该生几个孩子啦。”他带着幽默的神气,很快地朝上面看了看克鲁普斯卡娅。

“狄纳莫”厂的工人来到哥尔克,列宁出去迎接他们,一时忘了自己的病,想说话,可是像诉苦一样发出一些含混不清的声音,摆了摆手;工人们站在他周围,看到他在哭,也都哭了。还有那临终时的目光,好像是恐惧,好像有苦要诉说,很像小孩子看着妈妈的目光。

远处出现了车站的建筑物,机车和高耸的烟囱在雪地里显得分外黑。

伟大列宁的战友雷科夫、加米涅夫、布哈林在雪橇后面走着,胡子上冻结了白霜,他们漫不经心地看着一个穿着长大衣和软筒皮靴的黑脸膛的有麻子的人。他们常常带着嘲笑和容忍的神气打量他那高加索人的装束。斯大林如果知趣的话,他就不应该上哥尔克来,因为在这儿聚会的是伟大列宁的亲属和最亲近的朋友。他们却没有想到,正是这个人将成为列宁的继承人,他会把他们所有的人,包括最亲密的战友,统统打翻,甚至不准列宁的妻子继承列宁的遗产。

列宁的真理不在布哈林、雷科夫和季诺维也夫手里。也不在托洛茨基手里。他们都错了。他们谁也不能成为列宁事业的继承人。不过,就是列宁直到生命的最后一刻也不知道、不明白,列宁的事业会成为斯大林的事业。

一部农村的木架子雪橇拖着一个解决了俄罗斯、欧洲、亚洲和全人类命运的人的尸体去车站的那一天,已经过去将近二十年了。

克雷莫夫的思想总是萦绕着那个时候,他回忆着一九二四年一月里那些严寒的日子,夜间篝火的噼啪声,克里姆林宫外挂着冰雪的墙,千万张痛哭的脸,撕心裂腑的工厂汽笛声,站在木台上宣读告人民书的叶甫多基莫夫的宏亮的声音,紧紧靠在一起的一堆人抬着棺材走向仓促钉成的木头陵墓的情景。

克雷莫夫走上铺了地毯的工会大厦的台阶,看到旁边的一面面大镜子都披了黑红两色的绸带,充满松针气味的暖和的空气中回荡着哀乐声。他走进大厅,看到他在斯莫尔尼宫和老广场的主席台上常常看到的一些人都垂着头。后来,在一九三七年,他又在工会大厦看到这些垂着的头。大概这些后来被定为罪犯的人听着维辛斯基那冷酷而响亮的声音,会想起当年他们跟在雪橇后面,站在列宁的棺材旁边,听着哀乐的情景。

为什么在庆祝革命节的时候,在斯大林格勒发电站忽然想起那年一月里的一些日子?几十个和列宁一起缔造布尔什维克党的人竟成了奸细,成了外国间谍收买的代理人,只有一个人,虽然一直在党内不占重要地位也不是著名理论家,却成为党的事业的救星,成为真理的化身,他们怎么会承认呢?

最好别想这一切。但是这天夜里克雷莫夫偏偏想着这一切。他们怎么会承认?我为什么不说话?克雷莫夫心想,我不说话,是没有胆量说:“我不相信布哈林是破坏者、凶手、奸细。”而且在表决时我还举了手。以后又签了名。以后又作报告,写文章。我自己觉得我的义愤是发自内心的。那时我的怀疑和焦虑哪儿去了呢?这是怎么一回事儿?一个人有两种意识吗?还是一个人就是两个不同的人,各有各的意识?怎么理解呢?不过这种情况是常见的,不光是我,很多人都是这样。

格列科夫说出了很多人心里暗暗感觉到的问题,这些问题秘藏在心底,使人忧虑,引人关心,有时对克雷莫夫有很大的吸引力。但是,这秘藏的问题一说出来,克雷莫夫就觉得有恶意和敌意,就想把格列科夫压倒和制服。如果必要的话,他还会毫不犹豫地把格列科夫枪毙。

普里亚欣却用官腔官调的冷漠语调说话,他代表国家又谈完成计划的百分比,又谈粮食交售,又谈各种各样的任务。克雷莫夫听到这官气十足的、毫无热诚的话,见到说这话的官气十足、毫无热诚的人,一向十分反感,十分讨厌,但是他和这些人步调一致,他们现在是他的上级同志。列宁的事业造就了斯大林,列宁的事业通过这些人,通过国家得到体现。克雷莫夫愿意毫不犹豫地为这事业的荣誉献出自己的生命。

就连老布尔什维克莫斯托夫斯科伊也不例外。他从来没有为他相信忠于革命的一些人说过话,没有维护过他们。他什么也不说。他究竟为什么不说话呢?

再拿那个诚实可爱的小伙子科洛斯科夫来说。他是高级新闻训练班的学员,克雷莫夫当时给他们讲过课。科洛斯科夫是从农村里来的,他对克雷莫夫说了不少集体化的情况,说区里有些坏蛋,看中了谁家的房子或者果园就把谁划成富农。他说到农村的饥饿,说到怎样残忍地把所有的粮食全部弄走,一粒不剩……他说起农村里一个很好的老头子,为了救活老伴和小孙女,自己走上绝路,他说到这里还哭了。可是不久克雷莫夫就在墙报上看到科洛斯科夫写的文章,说富农把粮食埋到地里,说富农对新生政权怀着刻骨的仇恨。

这个真正动情地哭过的科洛斯科夫为什么这样写?莫斯托夫斯科伊为什么不说话?难道仅仅因为胆小怕事?克雷莫夫有多少次心口不一啊。但是当他说和写的时候,却觉得他正是这样想的,他也相信他说的正是他所想的。有时候他对自己说:“有什么办法呢,这是革命需要的呀。”

各种各样的情况都有过,有过,什么都有过。克雷莫夫没有好好维护自己的朋友,尽管他相信他们是无罪的。有时他不说话,有时他说几句含含糊糊的话,有时更坏些:他说话,而且说的不是含糊话。有时把他传到党委去,到区委,到市委,到州委,有时把他传到保安机关,向他询问他熟悉的一些人、一些党员的情况。他从来没有诬陷过朋友,从来不曾诽谤什么人,他没有写过密报,没有告过人……

且住,格列科夫呢?格列科夫是敌人。对待敌人克雷莫夫是从来不客气的,从来不怜悯的。

但是为什么他和被镇压的同志的家属们断绝关系呢?他不再上他们家去,不再给他们打电话;不过,他在大街上遇到被镇压的同志家里的人,从来不曾转到另一边人行道上去,而是依然和他们打招呼。

可是更有一些不同的人,这些人通常是老妇人,家庭女工,党外的平民,常常通过他们往劳改营里送东西,从劳改营里发出来的信也写他们的地址,他们不知为什么却不怕。有时这些老妇人,这些家庭女工和没有文化的保姆,充满了宗教观念,她们收养被捕的父母留下的孤儿,免得这些孩子进收容所和保育院。可是党员们却害怕沾到这些孤儿,就像怕火一样。难道这些老妇人,这些平民,这些没有文化的保姆比列宁式的布尔什维克莫斯托夫斯科伊和克雷莫夫更清白,更有骨气?

人能够战胜恐怖,所以小孩子能够在黑暗中走路,士兵能够投入战斗,一个小伙子可以前进,可以在高空跳伞。

可是有一种恐怖却很特殊,很厉害,千千万万人都不能战胜这种恐怖;这就是在莫斯科的灰暗的冬日天空,用不祥的、变幻莫测的红色字母写出的恐怖—国家恐怖……

不对,不对!恐怖本身不能起这样大的作用。革命的目的以道德的名义摆脱了道德,借口为了未来,证明今天的伪君子、告密者、两面三刀的人是正确的,还要宣传,为什么一个人为了人民的幸福应该把无罪的人推入陷坑。这股势力叫人不要理睬进入劳改营的人的孩子,也是以革命的名义。这股势力还在说,如果一个妻子不揭发自己的清白无辜的丈夫,就必须离开孩子,在劳改营里关十年,这都是革命的需要。

革命的势力与死亡的恐怖、对刑讯的恐惧、感受到远方劳改营气息的人的痛苦结成了联盟。以前人们走向革命的时候,知道等待着自己的是监狱、苦役、成年累月的流浪和无家可归、断头台。

而现在最糟糕、最令人不安的是,为了换取对革命的忠诚,换取对伟大目标的信仰,今天要付出的是优厚的待遇、克里姆林宫的酒宴、人民委员的任命书、专用汽车、疗养证、国际车厢。

“尼古拉,你没有睡吗?”斯皮里多诺夫在黑暗中问道。

克雷莫夫回答说:

“差不多睡着啦,正要睡呢。”

“噢,对不起,我不打搅你了。”

四 十

自从那天夜里把莫斯托夫斯科伊传去和党卫军少校利斯谈过话之后,又是一个多星期过去了。

忐忑不安的等待和紧张变成了难以承受的苦恼。

莫斯托夫斯科伊有时候觉得,朋友和敌人永远把他忘记了,朋友和敌人都认为他已经成了一个无用的、老糊涂的老头子,成了稻草人,成了废物。

一个晴和的早晨,一名党卫军看守带他去洗澡。这一次这名看守没有进澡堂,而是坐在台阶上,把枪放在旁边,抽起烟来。这一天天气晴朗,阳光照在身上很暖和,这名士兵当然不愿意到潮湿的澡堂里去。

管澡堂的一名战俘走到莫斯托夫斯科伊跟前。

“您好,亲爱的莫斯托夫斯科伊同志。”

莫斯托夫斯科伊惊愕得叫了起来:站在他面前的竟是穿着制服上衣、戴着勤务臂章、手里晃悠着抹布的旅政委奥西波夫。他们拥抱在一起。奥西波夫急急忙忙地说:

“我在澡堂里弄到这点儿差事,现在去替换固定的清洁工,我想和您见见面。科季科夫、将军、兹拉托克雷列茨都叫我问候您。您先说说您的情况,您身体怎么样,他们想要您怎样?您一面脱衣服,一面说。”

莫斯托夫斯科伊把那天夜里传他去谈话的情形说了说。奥西波夫用凸出的黑眼睛看着他,说:

“他们想劝诱您,真是妄想。”

“为什么呢?什么目的?目的何在?”

“可能他们想搜集历史方面的资料,想评价党的创始人和领袖,也许,他们想找材料发表什么宣言、文告、公开信。”

“这种打算永远不能得逞。”莫斯托夫斯科伊说。

“莫斯托夫斯科伊同志,他们不会善罢甘休的。”

“他们的打算永远不会得逞,痴心妄想。”莫斯托夫斯科伊又说了一遍,然后问道:“您说说,你们怎么样?”

奥西波夫小声说:

“比预料的情况要好些。最要紧的是,已经和在工厂里工作的人取得了联系,已经开始向我们输送武器,有自动步枪,有手榴弹。有人把零件送来,夜里我们进行装配。当然,目前数量还有限。”

“这是叶尔绍夫安排的,他真有两下子!”莫斯托夫斯科伊说。他脱去衬衣,看了看自己的胸膛,看到自己的衰老,很懊恼、难过地摇了摇头。

奥西波夫说:

“您是党的老同志,我应该告诉您:叶尔绍夫已经不在咱们的集中营里了。”

“什么,怎么不在了?”

“把他送走了,送到布痕瓦尔德集中营去了。”

“你们怎么了?”莫斯托夫斯科伊叫起来。“他是个出色的小伙子呀。”

“他就是到了布痕瓦尔德,依然可以是出色的小伙子。”

“这究竟怎么搞的,为什么会出这种事?”

奥西波夫阴沉地说:

“在领导人员中很快就出现了分裂。许多人自发地倾向叶尔绍夫,这就冲昏了他的头脑。他怎么也不服从领导核心的指挥。他是一个身份不明的人,一个异己分子。情况越来越混乱。地下工作的第一训条就是铁的纪律。可是我们却出现了两个核心:一个党的核心,一个党外核心。我们讨论了情况,通过了决议。一位在办公室工作的捷克同志把他的卡片放进为布痕瓦尔德挑出来的一部案卷里,这样就很自然地把他列入了名单。”

“真是再简单不过了。”莫斯托夫斯科伊说。

“这是共产党员一致通过的决议。”奥西波夫说。

他穿着自己的寒碜的衣服站在莫斯托夫斯科伊面前,手里拿着抹布,神气又严肃,又坚定,相信自己绝对正确,相信自己的权力比上帝的权力更大、更威严,更有权将他所从事的事业提交人类命运的最高法官。

而脱得光光的、瘦瘦的老头子,伟大的党的创始人之一,坐在那里,把两个瘦瘦的、干瘪的肩膀耸得高高的,头垂得低低的,一声不响。他眼前又浮现出那一夜在利斯的办公室里的情景。他又觉得十分可怕:难道利斯说的不是假话,难道他真的没有什么秘密的宪兵式的目的,真的是想和他谈谈?他挺起腰来,又像往常那样,像十年前集体化时期那样,像当年把他年轻时的同志一个个送上断头台的政治恐怖时期那样,说:

“我作为一名党员,服从这一决议,承认这一决议。”

他从放在板凳上的上衣里子里抽出几片纸,这是他草拟的传单。忽然在他眼前浮现出伊康尼科夫的脸,他那像牛眼一样的眼睛,莫斯托夫斯科伊又想听听这个又傻又善良的教士的声音。

“我想问问伊康尼科夫的情形,”莫斯托夫斯科伊说,“那位捷克同志没有把他的卡片塞进那里面去吧?”

“那个老傻子,您说的那个脓包吗?他被处决了。他拒绝上工,不肯去修杀人集中营。凯泽奉命把他枪毙了。”

这天夜里,在集中营的棚屋的一面面墙上,贴了不少莫斯托夫斯科伊拟定的有关斯大林格勒战役的传单。

四十一

战争结束以后不久,在慕尼黑的秘密警察档案室里发现了西德一座集中营里地下组织一案的侦讯材料。在案卷的最后一页中写着,对案犯的判决已经执行,尸体已经火化。名单中的第一名便是莫斯托夫斯科伊。

研究了侦讯材料之后,还是无法判断出卖了同志的内奸是谁。可能,秘密警察把他和被他出卖的人一起处死了。

四十二

在监督队的宿舍里,很暖和,很安宁。监督队是监督毒气室、毒药仓库和火化炉的工作的。

德国人给长期为一号工程工作的囚犯创造了很好的生活条件。每一张床前都有一张小桌,有热水瓶,床与床之间的走道上铺了地毯。

为毒气室干活儿的人没有人看押,而且在特别的食堂吃饭。监督队里的德国人吃饭像在饭店里那样,每个人都可以随便点菜;可以拿到额外的工资,几乎相当于相应级别的现役军人工资的三倍;他们的家属在住房方面享受着优待,得到的粮食供应是高标准的,在受到空袭威胁的地区他们有权最先疏散。

士兵罗捷在观察窗口值班。等到一道程序快结束的时候,他就下命令把毒气室里的人卸下去。此外,他还要监视牙科医生们,看他们干得是否认真仔细。他几次向工程主任卡里特卢夫特报告他同时执行两项任务的困难:有时候他在注视上面放毒气,就不能观察下面牙科医生找金牙,在将人推上输送带的地方,工人们就会偷懒。

罗捷习惯了自己的工作,已经不像最初几天那样面对着观察窗口惶惶不安了。他的前任有一天因为一件事情被打死了,那件事情应该是一个十二岁的孩子干的,不应该是一个执行特殊任务的党卫军士兵干的。罗捷起初不明白同事们在说话中暗指的是什么不体面的事,到后来他才明白了。

罗捷不喜欢这新的工作,虽然他已经习惯了。他对于周围的人对他的尊敬,很不习惯,感到很不安。食堂里的女侍者们常常问他为什么脸色那样苍白。自从他记事起,妈妈就经常哭。不知为什么父亲经常被解雇,好像他有工作的时候不如失业的时候多。他学会了父母那种轻盈、柔和、不会惊扰任何人的步子,学会了对邻居、房东、房东的小猫、校长和站在路口的警察的那种惶恐而亲切的笑。温和与亲切似乎是他性格的基本特点。所以他自己也觉得奇怪,他心中竟有那么多仇恨,怎么过去多年中没有表露出来。

他进了监督队;善于识别人的队长很了解他的软弱、温柔的性格。

看着犹太人在毒气室里抽搐,一点意思也没有。罗捷对于那些喜欢干这种事的士兵很厌恶。特别使人厌恶的是在毒气室门口值早班的战俘茹琴科。他的脸上一直带着一种孩子般的,因而特别令人厌恶的笑容。罗捷不喜欢自己的工作,但是他知道干这种工作有明显的以及潜在的好处。

每天下班的时候,很有气派的牙科医生都要交给罗捷一个小小的纸包,里面包几颗金牙。这小小的纸包只是每天交给集中营管理处的贵金属的微不足道的一部分,但是罗捷已经有两次把一公斤左右的金子交给妻子。这是他们的美好的未来,可以帮他们实现安度晩年的理想。他在年轻时又软弱又胆小,没能够好好地为生活奋斗。他从来不怀疑党的目的只有一个,那就是为弱小的人争取幸福。他已经亲身体验到希特勒的政策的良好结果,因为他就是弱小的人,而他和他一家的生活现在又好,又快活,和以前无法比了。

四十三

安东·赫麦尔科夫有时从心底里对自己的工作感到害怕。晚上,他躺在床上,听着特罗菲姆·茹琴科的笑声,感到发冷,难受,心慌。

茹琴科的手指头又粗又长,正是这双手天天关上毒气室的密闭的门。他的手好像从来没有洗过,当他伸手到面包篮子里去拿面包的时候,实在令人感到厌恶。

茹琴科每天早晨出去值班,等着人群排着队从铁路那边走来的时候,感到无比的兴奋。他总觉得人流移动得慢得不得了,常常扯着嗓子发出尖细的、焦急的叫声,上下颌轻轻哆嗦着,就好像小猫注视着玻璃窗外的麻雀。

此人便是赫麦尔科夫心里不安的原因。当然,赫麦尔科夫也可以喝酒,也可以醉醺醺地拿站队等候的女人取乐。有一处狭窄的通道,监督队的工作人员可以从这里进脱衣室去挑选女人。男人毕竟是男人。赫麦尔科夫有时也挑选一个大姑娘或者小媳妇,带到无人的棚屋隔间里,过半个钟头再带回去交给押解人员。他不说话,女人也不说话。不过,他来到这里,不是为了女人和酒,不是为了华达呢马裤和细皮的军官靴。

在一九四一年七月的一天,他被俘了。德国人用枪托子劈头盖脸地打他,他害痢疾,穿着破靴子被赶着在雪地里走,给他喝黄黄的漂着机油的水,他用手指头撕死马身上发黑发臭的肉,他吃臭大头菜和烂土豆皮。他所选择的只有一点—活下去,他再也不想别的,他躲过了十来次死亡,没有饿死,没有冻死,他不想死于痢疾,不想头上带着九克重的弹头倒下去,不想害浮肿,让水肿从脚下一直攻入自己的心脏。他不是罪犯,他是刻赤市的一名理发师,不论亲戚、同院的邻居、同行,还是和他一起喝酒、吃熏鱼、打牌的朋友,从来没有谁认为他不好。他也认为,他和茹琴科没有任何共同之处。但是有时他觉得,他和茹琴科之间的区别是微不足道的;干的反正都是一样的事情,至于怀着什么心情去干,一个高兴,一个不高兴,又有什么要紧?

可是他却不知道,茹琴科使他惶惶不安,不是因为茹琴科的罪恶最大。他所以觉得茹琴科可怕,是因为茹琴科的天生的、可怕的变态在为他的行为辩护。而赫麦尔科夫却不是变态人,他是正常的人。

他模模糊糊地懂得,在法西斯时期,对于一个还想做人的人来说,比活命更容易做到的选择—就是死。

四十四

工程主任兼监督队长卡里特卢夫特要求调度总站每天晚上把第二天火车到达的时间报上来。卡里特卢夫特可以事先向手下的工作人员布置工作,把车厢的总数、运到的人数告诉他们;另外,还要根据从哪一国来的火车,就调哪一国的战俘前来协助执行—有剃头的,有带路的,有卸人的。

卡里特卢夫特·¥作认真。他不喝酒,看到下属喝醉了,他也不生气。只有一次大家看到他很快活、很兴奋;那一天他要回家过复活节,已经坐上汽车,他把党卫军上尉加恩叫到跟前,把女儿的相片给他看,那女孩大脸盘,大眼睛,长得很像父亲。

卡里特卢夫特很喜欢工作,不愿意白白浪费时间。晚饭后他不上俱乐部,不打牌,也不看电影。过圣诞节的时候,在监督队里举行了枞树晚会,有业余合唱队演出,吃晚饭的时候无偿地发给每两个人一瓶法国白兰地。卡里特卢夫特来俱乐部待了半个小时,大家都看到他的手指头上还有新鲜的墨水痕迹,说明他在圣诞节晚上还在工作。

过去他住在乡下父母的房子里,看来,他的一生就要在这座房子里度过了,因为他喜欢乡下的安静,不怕干活儿。他想振兴父亲的家业,不过他认为,不论他养猪和做大头菜、小麦买卖赚多少钱,他一辈子都要住在父亲又舒适又安静的房子里。可是人生多变。在第一次世界大战快要结束的时候,他上了前线,走上命运为他铺好的道路。似乎命运决定了他从一个农村小伙子成为士兵,又从战壕进入司令部警卫队,又从办公室到副官处,从帝国保安总部到集中营管理处,最后,在杀人营里担任了监督队队长。

如果卡里特卢夫特将来到天国受审,他会为自己的灵魂辩护,会理直气壮地对审判官说,是命运把他推上刽子手的道路,杀了五十九万人。他面对着强大的力量,面对着世界大战、巨大的民族运动、不可违抗的党国的暴力,又有什么办法呢?谁又能按自己的心意行事?他是一个人,他本来可以在父母的房子里住下去的。不是要走这条路,是推着他走,不是他愿意走,是牵着他走,他就像一个小小的孩子,命运牵着他的手走路。他派去工作的人和派他来工作的人如果面对天国审判官,也会这样或者大致这样为自己辩护。

卡里特卢夫特不需在天国为自己的灵魂辩护。所以上帝也不需要向卡里特卢夫特证实世界上没有罪人。

有天国的审判,有国家与社会的审判,但是还有最高审判,那就是罪犯对罪犯的审判。有罪的人掂量了极权制国家的威力,知道国家是无比强大的。这种可怕的力量用宣传、饥饿、孤苦、集中营、死的威胁、落魄和屈辱把人的意志束缚住。但是,一个人在贫困、饥饿、集中营和死亡的威胁下走的每一步,在受制约的同时,也表现出一个人的不受约束的意志。在这位监督队长走过的道路上,从乡村到战壕,从党外的平民到自觉的国家社会主义党党员,到处都有他的意志的痕迹。命运带着人走什么路,一个人跟着走,是因为他愿意;他也可以不愿意。命运带领着一个人,这个人会成为毁灭性力量的工具,但是他可以从中捞到便宜,而不是吃亏。他知道这一点,于是他便去捞便宜;可怕的命运和人有不同的目的,但是二者走的道路是一条。

不是无罪和慈悲的天国审判官,不是英明的、以国家和社会利益为准绳的国家最高审判庭,不是圣人,不是教士,而是可怜的、受到法西斯压迫的肮脏而有罪的人,亲身体验过极权制国家的恐怖政策的人,自己已经倒下过、已经弯下腰、畏畏缩缩、低三下四的人,这样的人在宣布判决。

他会说:

“在可怕的世界上,罪人是有的!我就有罪!”

四十五

行程的最后一天到了。一节节车厢哐啷哐啷,制动器发出刺耳的吱嘎声,然后静了下来,响起门闩的叮当声,响起口令声:

“全体下车!”

人们纷纷走出来,来到新雨后潮漉漉的站台上。

一张张熟悉的脸,出了黑暗的车厢,显得多么奇怪啊!

大衣和头巾比人的变化要小些;女褂和连衫裙使人想起当初在里面穿衣的房间和对着试衣服的镜子。

出了车厢的人挤成一堆一堆的,紧紧地靠在一起,就有一种习惯的、可以使人放心的气氛;在熟悉的气味、熟悉的热气和熟悉的痛苦的脸上和眼睛里,在从四十二节货车车厢里走出来、紧紧靠在一起的巨大人群中,洋溢着这样的气氛。两名穿长大衣的党卫军巡逻兵慢慢走着,那钉了铁掌的靴子敲得水泥地当当响。他们带着一副傲慢和沉思的神气,不看那两个抬出一个白脸上披着白发的死老婆子的犹太小伙子,不看那个趴着在水洼里喝水的卷发小矮子,也不看那个掀起裙子系裤带的驼背女人。

两名党卫军士兵有时交换一下眼色,说一两句话。他们在水泥地上走着,那神气就像太阳在天上走。太阳并不注视风、云彩、海浪和树叶的动静,但是它在从容自若的移动中知道,大地上因为有了阳光,一切事情在正常地进行着。

一些身穿蓝色工装、头戴大檐便帽、袖子上带着白色臂章的人在叫喊着,催促从车上下来的人,他们用的是奇怪的语言,是俄语、德语、犹太语、波兰语和乌克兰语的大杂烩。

穿蓝色工装的人快速、熟练地调理着站台上的人群,挑出站也站不住的人,让比较强壮的人把这些半死不活的人抬上汽车,让乱糟糟的人群站成队伍,让队伍移动,指明移动的方向和目的。

队伍中每排有六个人,在队伍里传着一个消息:

“上澡堂去,先上澡堂去!”

似乎慈悲的上帝再也想不出更慈善的主意了。

“好啦,犹太人,咱们走吧。”

一个头戴便帽的押解队的头头儿叫喊着,一面打量着人群。男人和女人们都提起包裹,孩子们抓住妈妈的裙子或父亲的衣襟。

“上澡堂去……上澡堂去……”

这话催眠般地填满人的意识。

那个戴便帽的大个子身上有一股平易近人、招人喜欢的神气,似乎他和这些不幸的人亲近,而不是和那些身穿灰大衣、头戴钢盔的人亲近。

一个老奶奶带着祈祷时的小心神气用指尖抚摩着他的工装袖子,问:

“是去洗澡吗?”

“是的,是的,大娘,是去洗澡!”

他忽然用嘶哑的嗓门大声发出口令:

“开步走!”

站台空了,一些穿工装的人在打扫水泥地上的破布片、绷带、有人扔掉的破套鞋、孩子们丢下的拼字方块,还有人在轰隆轰隆地关车厢的门。一节节车厢上的钢铁叮叮当当响动起来,像波浪似的扩展开去。空空的列车动了,前去消毒。

服务队干完了活儿,经过公务大门回到集中营里。东方来的列车是最糟糕的,在里面最多的是死人、病人,在车厢里可以找到的是虱子,可以闻到的是臭气。这些列车跟匈牙利或者荷兰,或者比利时来的列车不同,在里面找不到一瓶香水、一包可可、一听炼乳。

四十六

人群往前走着,前面出现了一座巨大的城市。城市的西边沉没在雾气中。远处工厂烟囱里冒出来的黑烟和雾气混合在一起,像棋盘一样的一排排棚屋罩着轻烟,一条条笔直的集中营街道和雾气合在一起,显得很奇怪。

东北方升起高高的黑红色火光,似乎是潮湿的秋日天空燃烧过以后,还在发红。有时从潮湿的火光中冒出火焰,又慢,又不清晰,缓缓地爬动着。

旅途困顿的人们来到宽阔的广场上。广场中央有一座用木头搭起的高台,在大众游艺场上常常有这样的高台。上面站着几十个人。这是乐队。这些人就像他们的乐器一样,模样个个不同。有些人打量着渐渐走近的人群。但是有一个身穿浅色外套的白头发的人说了一句什么话,于是在高台上的人一齐拿起自己的乐器。就好像有一只鸟又胆怯又放肆地叫了起来,于是,被铁丝网和警笛声撕得支离破碎、散发着臭味和油烟味的空气里充满了音乐声。就好像一阵被太阳晒得暖和的夏日的好雨,光闪闪地落到大地上。

集中营里的人、监狱里的人、冲出监狱的人,乃至走向刑场的人,都能感受音乐的震撼人心的力量。

谁也不像进过集中营和监狱的人,不像走向死亡的人对音乐的感受那样强烈。

音乐声一触及濒临死亡的人,在人们心中突然重新唤醒的不是思想,不是希望,而只是一种模糊而强烈的生命奇迹。人群队伍里响起一片号哭声。似乎一切都变了样子,一切都合成一个整体,一切分散的,房屋,天地,童年,道路,车轮声,饥渴,恐怖和这罩在雾中的城市,这暗红色的火光,这一切一下子全都汇合起来了,不是汇合在脑海里,不是在画面上,而是汇合在过往生活的一种模糊、热烈、醉心的感情中。在这里,在火化炉的火光中,在集中营的广场上,人们感觉到,生命大于幸福,因为生命也是痛苦。自由不光是幸福。自由是艰难的,有时也是痛苦的,因为自由就是生命。

音乐挑起心灵的最后震动,使得心灵在模模糊糊的心的深处将一生中感受到的一切,将生的欢乐与痛苦,与这雾茫茫的早晨、这头顶上的火光汇合到一起。但也许不是这样。也许,音乐只是一把开启人的感情的钥匙,不是音乐充满了人的心灵,而是它在这个可怕的时刻把人的内心打了开来。

有时候,一支儿歌能够使一个老头子哭起来。但这不是老头子为儿歌哭,儿歌只是一把钥匙,打开了心灵在寻找的东西。

人们还在广场上画着弧形,从集中营的大门里出来一部奶油色小汽车。一名身穿皮领军大衣、戴眼镜的党卫军军官从汽车里走出来,打了一个不耐烦的手势,正在注视着他的乐队指挥马上忙不迭地把手垂了下来,乐队演奏一下子停止了。

广场上很多声音一连声地叫喊:

“立正!”

军官从一排排的人旁边走过。他用手指头指着谁,押队的人就把那人从队伍里拉出来。军官用冷冷的目光打量着被拉出来的人,押队的头头儿生怕妨碍了军官思考,用小声问着:

“年龄?职业?”

被挑出来的有三十来个人。

一排一排的人群旁边响起另一口令:

“内科医生,外科医生,出列!”

没有人应声。

“内科医生,外科医生,出列!”

依然没有人应声。

那军官对站在广场上的上千人失去了兴趣,便朝汽车走去。

挑出来的人五人一排,命令他们转过身去,朝着带有标语牌的集中营大门,标语牌上写着:“劳动使人自由。”[20]

人群队伍里有一个小孩子叫起来,一些妇女像发了疯似的尖声叫起来。被挑出来的人垂着头,一声不响地站着。

可是,谁又能描写出一个人放开妻子的手时那种心情,最后一次匆匆看一眼亲人的脸的那种目光?想起在默默吿别的时候,你的眼睛在一瞬间眨巴着,为了掩饰自己保得一命的可耻的窃喜。人有过这种残忍的记忆,以后还怎么活下去呢?

妻子把小包袱交给丈夫,包袱里有结婚戒指,还有几块糖和干粮,这个时刻,他会忘记吗?看到天空又闪起新的火光,知道那里烧的是他吻过的手、他心醉的眼睛、他在黑暗中凭气味也能闻出来的头发,知道那是在烧他的孩子、妻子、母亲,难道还能活下去吗?难道他还会为了在棚屋里得到更靠近炉火的铺位而计较吗?还能捧着饭钵去接长柄勺子舀来的一升灰黑的汤糊吗?还能自己把掉下来的鞋掌钉到鞋上吗?怎么能拿铁钎干活儿?怎么还能呼吸?还能喝水?孩子的叫声、母亲的哭号还在耳朵里啊。

继续活下去的人被赶着朝集中营的大门走去。他们听着后面的叫喊声,他们自己也在叫喊,撕扯他们胸前的衣襟,前面是另一种生活等待着他们:通电的铁丝网,架着机枪的混凝土守望塔,棚屋,脸色煞白的妇女在铁丝网外面望着他们,他们胸前带着红的、黄的、蓝的布条子标记,排着队去干活儿。

乐队又演奏起来。被挑出来为集中营干活儿的人走进建筑在沼地上的集中营。黑糊糊的水阴沉无声地在黏腻的水泥板和沉甸甸的大石块中间流着。这水呈黑红色,散发着腐烂的气息。这水里有一团团绿色的化学物质的泡沫、一块块脏布、从集中营手术室里扔出来的一块块血淋淋的肉。这水流入集中营的地下,然后又钻出地面,然后又流入地下。不过,水是要继续流下去的,这集中营里出来的阴沉的水早晚会成为海浪,成为早晨的露水。

可是不幸的人们就要去死了。

四十七

索菲亚·奥西波芙娜走着均匀而沉重的步子,一个男孩子拉着她的手。小孩子的另一只手抚摩着口袋里的火柴盒,火柴盒里的脏棉花里有一只深褐色的蚕蛹,是在车厢里刚刚从茧里钻出来的。旁边是钳工拉萨尔·扬凯列维奇,一面走,一面嘟哝,他的妻子杰鲍拉·萨穆伊洛芙娜抱着一个小孩子。列维卡·布赫曼在背后嘟哝着:“唉,上帝,唉,上帝,唉,上帝。”这一排的第五个人是图书管理员穆霞·鲍里索芙娜。她的头发梳得好好的,衣领还显得很白。她在路上有几次用她领到的面包换半锅子温水。这个穆霞·鲍里索芙娜从来不对谁抱怨什么,在车厢里大家都把她看作圣女,一些见过世面的老奶奶都在吻她的衣服。前面的一排只有四个人,因为那个军官在挑人的时候一下子就挑出去两个,就是斯列波依父子,他们在回答什么职业问题的时候,一齐说:“牙科医生。”军官点了点头。斯列波依父子猜到:可以保命了。这一排里留下来的三个人悠荡着手,看来,他们的手没有用场了;第四个人把领子支得高高的,两手插在口袋里,昂着头,毫不在乎地走着。前面,往前四五排,有一个很突出的高大老头子,戴着红军士兵的暖帽。

在索菲亚·奥西波芙娜背后走的是穆霞·维诺库尔,她在火车上度过了十四岁生日。

死神!死神竟变得乐于交际,他像个老伙伴一样,不请自来,进入人们的院子和车间;他到市场上找家庭主妇,把她和菜篮子一起带走;他和孩子们一起玩耍;女装裁缝们在成衣店里唱着歌儿为委员的妻子赶做女大衣,他也走进去;有人排队买粮食,他也来站队;老妇人补袜子,他也来跟前坐一坐。死神干着自己平常的事情,人们也干着自己的事情。有时死神让人把烟抽完,把饭吃完,有时他像个好朋友一样,粗声粗气地哈哈大笑着拍拍人的肩膀,把人拉住。

人似乎终于对死神有所了解了,死神已经向人显示出他的平常和孩子般的单纯。这种转变和过渡太容易了,就好像过一条小河,小河上有小小的木桥,从这边炊烟袅袅的小屋到对岸空旷的草地上,不过五六步。就这么一回事儿!有什么好怕的?瞧,小牛吧嗒着蹄子从小桥上走过去了,瞧,孩子们也吧嗒着光脚丫跑过去了。

索菲亚听到了音乐声。她第一次听到这乐曲是在小时候,后来上大学的时候,年轻时做医生的时候,她也听过。这支乐曲充满了对未来的生气勃勃的预感,她听着总是非常激动。

音乐欺骗了她。索菲亚已经没有未来了,只有已经过去的生活。

她顿时感触到自己已经过去的与众不同的生活,这种感触一时间遮住了面前的现实—遮住了生命断崖的边沿。

这是所有感触中最奇特的。它无法表达,即使是对最亲近的人,父母兄弟、妻子儿女、挚交好友。它是心灵的秘密。不管心灵多么热切地想要说出自己的秘密,它也无法做到。一个人会把自己一生的感触带走,至死无法与任何人分享:一个与众不同的独立的人,在意识和下意识中汇集了一切好的和坏的,从小到老,一切可笑、可爱、可耻、可怜、羞涩、温柔、胆怯、惊愕的—这一切在个体对自己的生命的隐秘而沉默的孤独感中奇迹般地融为一体。

当乐队开始演奏的时候,达维德想掏出口袋里的火柴盒,为了不让蛹冻伤了风,他把火柴盒打开一点点儿,好让它看看奏乐的人。但是走了几步之后,他就不再觉得高台上有人,只剩下天上的火光和音乐了。悲哀而洪亮的乐曲声把对妈妈的思念灌入他心中,灌得满满的,就像灌满了一个碗。妈妈好静,身体很弱,一直觉得被丈夫抛弃是件不体面的事。她给达维德做了一件衬衫,邻居们在走廊里笑,笑话达维德的衬衫是花布做的,而且袖子也缝斜了。妈妈是他唯一的保护人和希望。他一直坚定不移地、一心一意地指望着她。可是,也许现在是音乐起了作用:他不再指望妈妈了。他爱妈妈,可是妈妈软弱,无能为力,就和现在跟他走在一起的这些人一样。音乐声悠忽而缓慢,他觉得就像小小的波浪,他在迷糊状态中看到过,那时候他发着高烧,梦到从滚热的枕头上爬下来,躺到热乎乎、湿漉漉的沙地上。

乐队声音高起来,一个嗄哑的大嗓门儿大叫起来。

他害咽峡炎的时候,梦见从水里冒出来一堵黑糊糊的墙。现在那墙又悬在他的头顶上,遮住整个天空。

一切曾经使他心悸的东西全都汇合到一起,连成一片。小羊羔没有觉察到枞树丛中狼的影子,他看到那幅画就害怕,他还怕市场上被宰的小牛的头,那眼睛是蓝色的,他怕死去的奶奶,布赫曼家被勒死的小姑娘,还有他第一次梦魇,不要命地尖叫起来喊妈妈—全都来到面前。死神睁大两眼站着,有天那么高大,小达维德迈着小小的步子朝死神走去。周围只有音乐声,既不能抓住作依靠,又不能在上面把头撞碎。

没有翅膀、没有爪子、没有胡须、没有眼睛的蛹还睡在火柴盒里,很信赖地傻等着。

既然是犹太人,那就完了!

他打嗝,透不过气。如果他有力气的话,他会把自己掐死的。音乐声停了。他的一双小腿和另外几十双小腿在急急忙忙地跑着。他没有什么想法,他既不能哭,又不能叫。汗湿的手指头紧紧捏住口袋里的火柴盒,但是他已经不记得有蛹了。只有小小的腿在走着,走着,急急忙忙地跑着。

如果他的恐惧再持续几分钟,他会带着碎裂的心跌倒的。音乐声停止以后,索菲亚擦干了眼睛,气愤地说:

“哼,来这一套!”

她转头看到了这孩子的脸,脸上是那样惊惧的表情,即使在这里也显得十分突出。

“你怎么啦?怎么一回事儿?”索菲亚叫了起来,并且猛地扯了扯他的手。“你怎么啦,怎么一回事儿,咱们这是去洗澡呀。”

在德国人挑外科医生的时候,她没有作声,因为她痛恨敌人。

钳工的妻子在旁边走着,她抱着可怜的大脑袋婴儿,婴儿用纯真和若有所思的目光看着周围的一切。这位钳工妻子为了孩子夜里偷了一个同车妇女的一小把糖。那个被偷的妇女也是非常虚弱的。有一个姓拉比杜斯的老头子为她抱不平。那个老头子身子底下尿湿了,所以谁也不愿意坐在他身边。

这会儿钳工的妻子杰鲍拉心事重重地走着,手里抱着孩子。那孩子本来日日夜夜都在啼哭,现在却不作声。这女人的黑眼睛流露出那样的悲伤神情,她那难看的肮脏的脸和苍白干枯的嘴唇也就不多么显眼了。

“圣母啊。”索菲亚在心里说。

战争爆发前两年,有一天她看到从天山山峦背后升上来的太阳照得山顶积雪亮晶晶的,可是湖水还在黑暗中,就像用蓝宝石雕成的。那时她心想,如果在哪一座寒碜、黑暗、低矮的小屋里有一双孩子的手把她搂住,那世界上再没有什么人不羡慕她了,于是她的五十岁的心里顿时涌出一股十分强烈的感情:为了那孩子,她可以死而无怨。

小达维德勾起她非同一般的慈爱之情,这样的感情她对孩子们还不曾有过,虽然她一直非常爱孩子。在车厢里她拿出自己的一部分面包给他吃,常常在昏暗中把他的脸转过来朝着自己,她想哭,想把他紧紧搂在怀里,想吻他,就像妈妈们吻小宝宝那样,吻得又快又急。为了不让他听得太仔细,她小声说:

“吃吧,我的好儿子,吃吧。”

她很少和这孩子说话,一种奇怪的羞涩使她尽力掩盖她心中产生的母爱。但是她发现:如果她走到车厢的另一边,这孩子就惴惴不安地注视着她;等她来到他身边,他就放下心来。

她自己不愿意承认,为什么叫外科医生离开队伍的时候,她没有应声,继续留在队伍里,为什么在这几分钟里她的心情格外激动。

人群队伍从铁丝网旁边,壕沟旁边,从架着旋转机枪的混凝土守望塔旁边走过。这些早已忘记自由的人觉得,那铁丝网和机枪不是为了防备集中营里的人逃跑,而是为了不让那些将死的人躲进苦役集中营里。

人群队伍离开集中营的铁丝网,朝几座又矮又大的平顶建筑物走去。远远看去,达维德觉得这些没有窗户的灰色方形建筑物很像大型的拼图方块。

达维德从转弯的几排人的空隙中看到敞开大门的建筑物,也不知为什么,从口袋里掏出装着蛹的火柴盒,也没有和蛹告别,就把火柴盒扔到一边。让它活着吧!

“德国人好气派呀。”走在前面的一个人说。就好像德国警备队能听到他的奉承话,会看重他的奉承话似的。

那个支着领子的人不知为什么很奇怪、很特别地耸了耸肩膀,这在旁边看得很清楚;他朝右边看了看,又朝左边看了看,顿时变得又高又大,就像张开了翅膀,突然很轻盈地一跳,一拳打在一名党卫军押队兵的脸上,把他打倒在地。索菲亚凄厉地叫了一声,也跟着朝前冲去,但是踉跄了一下,跌倒了。马上有几只手把她抓住,帮她站了起来。后面的人挤了上来,达维德一面回头看着,怕被挤倒,无意中看到押队的德国兵把一个男子拉到了一边。

在索菲亚试图朝德国兵扑去的一刹那间,她忘记了小孩子。现在她又牵住他的手。达维德看到,一个人在片刻间感到有自由的希望时,眼睛会有多么明亮,多么有神,多么好看。

这时候,前面的几排人已经走上澡堂大门前面的沥青场地,就要进入大敞着的门,人们的脚步声音开始变了。

四十八

在潮湿而暖和的更衣间里,幽暗而宁静,还有若干长方形小窗户。

一排排带着红漆编号的、厚实的白木头板凳朝幽暗中伸去。大厅中间有一道不高的隔墙,一直延伸到大门对面的墙壁,隔墙的一边是男子脱衣的地方,另一边是女人和小孩子脱衣的地方。

像这样分隔开,没有引起人不安,因为人们依然能互相看到,互相召唤:“玛尼娅,玛尼娅,你在那儿呀?”“是的,是的,我看见你啦。”

有人在喊:“马季尔达,你把擦子带过来,给我搓搓背!”

几乎所有的人都感到放心了。

有一些穿工作服脸色严肃的人在人群中来来回回走着,在维持秩序,说的都是一些合乎情理的话,比如,要把袜子和包脚布塞到靴筒里,一定要记住哪一排、哪一个位子的编号。

许多人的声音低低地、嗡嗡地响着。

当一个人渐渐脱光的时候,他也就渐渐接近自己。天啊,胸膛上的毛更硬了,更密了。而且有那么多白毛呢。指甲有多么难看呀。脱光了的人看着自己,只能得到一个结论:“这就是我。”一个人会认出自己,确定自己这个“我”,因为“我”永远只有一个。一个小孩子把细细的手臂交叉在露着肋骨的胸前,看着自己蛤蟆似的身体,会认出:“这就是我。”等他再过五十年,打量着自己腿上骨骨棱棱的青筋,打量着自己的肥胖下垂的肚子,也会认出自己:“这就是我。”

但是却有一种奇怪的感情惊动了索菲亚。在这儿年轻的身体和年老的身体都裸露着:看到大鼻子的瘦小孩子的身体,老妇人们会摇头说:“瘦得可怜的。”十四岁姑娘的身体,即使在这里,几百双眼睛也在欣赏。残缺、衰弱的老头子和老太婆的身体,引起人们的同情和敬重。强壮的男子汉毛茸茸的脊背,女人们肉滚滚的大腿和丰满的乳房—这一切都是人的身体,原本被破衣烂衫遮盖起来的人的裸体。索菲亚觉得,她所感到的“这就是我”不光是她自己,而是人类。这是光光的人类的身体,有年轻的,也有年老的,有充满生气的、正在成长的、强壮的,也有衰老的、带有鬈发和白发的,有好看的,有难看的,有强壮的,有软弱无力的。她看着自己圆圆的雪白的肩膀,还没有人吻过呢,只有在小时候妈妈吻过,然后她带着一派柔情把目光转到孩子身上。难道在几分钟之前她竟忘记了他,像醉汉一样疯狂地扑向党卫军吗?“那真是个犹太小傻瓜,”她心里想道,“还有那个俄罗斯老傻瓜[21],也宣传不以暴力抗恶呢。他们那时候没有法西斯嘛。”索菲亚再不因为她这个处女心中萌发了母爱而觉得羞耻,俯下身去,用自己干活儿的大手捧住达维德的小脸,她觉得自己已经把他那亲热的眼睛握在手里,她吻了吻他。

“是的,是的,好孩子,”她说,“这不是,咱们来到澡堂里了。”

在混凝土脱衣间的幽暗中,似乎一下闪现出弗拉基米罗芙娜·沙波尼什科娃的眼睛。她还活着吗?她们分别了。索菲亚·奥西波芙娜就要走了,这不是,她完了。安娜·施特鲁姆也完了。

钳工的妻子想让丈夫看看脱得光光的小儿子,但是丈夫在隔墙那边,于是她把用布半裹着的孩子递给索菲亚,很得意地说:

“一把他脱光,他就不哭了。”

隔墙那边有一个长着黑黑的大胡子、里面穿着破烂睡裤的男子,闪动着发亮的眼睛和金牙叫道:

“玛尼娅,这儿还卖游泳衣呢,买不买?”

穆霞·鲍里索芙娜听到这句玩笑话,用手捂着从宽大的衬衣豁口里露出来的乳房,笑了笑。索菲亚早已懂得,被判决的人说俏皮话,并不能产生精神力量,然而当弱者和怯懦者对恐怖取笑的时候,恐怖就不那么可怕了。

列维卡·布赫曼那张好看的脸很消瘦,热辣辣的大眼睛故意不看周围的人,偷偷解开沉甸甸的发辫,把戒指和耳环藏到里面去。

她现在有一股盲目的、强烈的求生的劲头。虽然她是不幸的,是软弱无力的,但是法西斯已经把她折磨够了,再也没有谁能够消除她求生的欲望。现在,在她藏戒指的时候,她已经不记得,因为怕孩子哭会暴露阁楼上的藏身处,正是用这双手把自己的孩子掐死的。

但是,就在列维卡·布赫曼像终于躲进安全密林的野兽似的慢慢舒了一口气的时候,她看到一个穿工作服的女人在用剪刀剪穆霞·鲍里索芙娜头上的辫子。旁边还有一个女人在剪一个小姑娘的辫子。光溜溜的黑头发无声地落在水泥地上。一堆堆头发散在地上,就好像妇女们在又黑又亮的水里洗脚。

一个女人不慌不忙地把列维卡护着头的手拉开,抓住脑后的头发,剪刀尖儿碰到了藏在头发里的戒指,那女人也不停止工作,用手指头摸出缠在头发里的戒指,凑到列维卡的耳朵上说:“都要还给您的。”又用更小的声音说:“德国人在这儿。别作声。”列维卡没有记住这个穿工作服的女人的脸,她没有眼睛、嘴巴,只有露出青筋的黄黄的手。

在隔墙的那边有一个歪鼻子上歪戴着眼镜、很像一个可怜的病鬼的白发男子,他用眼睛扫了扫一排排的板凳,用和聋子说话的那种清清楚楚、一字一顿的语调问道:

“妈妈,妈妈,妈妈,你感觉怎么样?”

一个满脸皱纹的小老婆子忽然在嗡嗡的几百人的声音中听出儿子的声音,猜到了他常常问的问题,便很亲热地朝他笑了笑,回答说:

“脉搏很好,很好,跳得很好,你放心吧。”

索菲亚旁边有一个人说:

“这是盖尔曼,有名的内科医生。”

有一个脱得精光的年轻女人,拉着一个穿白裤衩的厚嘴唇小姑娘的手,高声大叫着:

“要杀咱们啦,要杀咱们啦,要杀咱们啦!”

“别嚷嚷,别嚷嚷,你们别叫这个疯女人嚷嚷。”穿工作服的女人说。她们回头看看,看不到押解队了。耳朵和眼睛在幽暗和寂静中得到休息。脱去被污垢和汗水浸得像木头一样硬邦邦的衣服,脱去快要腐烂的袜子和包脚布,有多快活啊,好几个月没尝到这种快活滋味了。穿工作服的几个女人剪完头发,走了,人们更自由地舒了一口气。有些人打起盹儿,有些人在衣缝儿里逮虱子,有些人在小声说话儿。有一个人说:

“可惜没有扑克牌,要不然咱们可以来捉捉傻瓜。”

可是这时候监督队队长一面吸着香烟,拿起电话筒,仓库管理员便把一个个像果酱罐子一样的贴了红色标签的罐子装上带马达的小车,坐在办公室里的特别科值班人员看着墙上:红色信号灯就要亮了。

“起立!”

脱衣间各个角落里忽然响起口令声。

一排排板凳的两头都站着穿黑制服的德国人。人们走进一条宽阔的走廊,走廊的顶上嵌着一盏盏不太明亮的电灯,电灯都用厚厚的椭圆形玻璃罩护着。在这儿可以看出吞吸着人流的、缓缓弯曲的混凝土的肌肉力量。很静,只有光着脚走路的人们沙沙的脚步声。

在战前有一次索菲亚对叶尼娅·沙波什尼科娃说:“如果一个人注定了被另一个人杀死,那么,看着他们怎样渐渐碰到一起,是很有意思的。起初他们也许离得非常远,比如,我在帕米尔高原上采杜鹃花,我走我的路,将来要杀死我的人这时候却在八千俄里之外,放学之后在小河里逮鲈鱼。我要去参加音乐会,他这一天却在车站买票,要上姑娘家去。不过反正早晚我们会碰到一起,就要出事了。”现在索菲亚想起了那一番很奇怪的话。她看了看廊道的顶:头上有这样厚的混凝土,她再也听不见沉雷,看不见暴雨了……她光着脚朝廊道的弯曲处走着,廊道也无声无息地、亲切地迎着她漂流过来;自然而然地移动着,没有强制,就好像迷迷糊糊地滑动,就好像从里到外都抹了甘油,所以都在自然而然地滑动。

密闭室的大门突然渐渐打开了。人流慢慢地滑动着。有一个老头子和一个老太婆,在一起生活了五十年,在脱衣服的时候分开了,现在又走在一起了。钳工的妻子抱着醒了的孩子,妈妈和儿子都朝人群头顶上看着,不是想看看空间,是想看看时间。内科医生的脸闪了一下,旁边又闪过善良的穆霞·鲍里索芙娜的眼睛,又闪过列维卡·布赫曼的恐惧的目光。再就是柳霞·什捷林塔尔,真无法掩盖、无法减弱她那青春的眼睛、轻轻呼吸的鼻孔、脖子、半张着的嘴唇的美,旁边走着的便是嘴巴又发青又干瘪的拉比杜斯老头子。索菲亚又紧紧搂住达维德的肩膀。这种对人的柔情在她心中还从来不曾有过。

走在旁边的列维卡叫了起来。她的叫声极其可怕,这是一个人面临死亡时的叫声。

在毒气室门口站着一个人,手里拿着一段引水管。他穿的是带拉链的棕色短袖衬衫。列维卡·布赫曼就是看到他那隐隐约约的孩子般狂喜的狞笑,才这样可怕地叫起来。

那人的一双眼睛在索菲亚的脸上扫了一下:就是他,终于见面了!

她觉得,她的手指头应该扼住从敞开的领子里伸出来的那根脖子。但是那个狞笑的人又快又利落地扬了一下棒子。她在钟声与玻璃响声中听到那人在喊:

“狗崽子们,别磨蹭了!”

她硬撑着没有跌倒,并且迈着沉甸甸的步子和达维德一起慢慢跨过铁门坎。

四十九

达维德用手摸了摸钢门框,觉得冷冰冰的。他在这钢镜子里看到一个模模糊糊的淡灰色的点儿,那是他的脸。他的光脚丫感觉到,这厅里的地面比廊道里的地面要凉些,因为不久前才冲洗过。

他迈着小小的步子,在这个矮顶的混凝土大箱子里慢慢走着。他看不到灯,但是这厅里有灰灰的亮光,就好像阳光透过混凝土盖顶射了进来,这冷冷的亮光似乎不是为活人照亮的。

人们原来在一起的,现在散开了,彼此看不见了。闪过柳霞·什捷林塔尔的脸。在火车上达维德每看到她,总有一种迷恋的感觉,又甜蜜,又惆怅。但是过了一会儿,在柳霞原来的地方却出现了一个不露脖子的矮个子女人。接着这地方又出现了一个蓝眼睛白头发的老头儿。马上又出现了一个年轻男子睁得大大的、呆滞不动的眼睛。

这种移动不是人类的活动。也不是低等生物的活动。既无用心,也无目的,表现不出活人的意志。人流朝大厅里流动着,正要进来的人推挤着已经进来的人,这些人推挤着那些人,从无数胳膊肘、肩膀、肚子的小的推挤中产生了运动,这种运动和生物学家布朗发现的分子运动没有任何区别。达维德觉得,有人带他走,他就应该走。他走到墙边,先是膝盖、然后是胸膛碰到了冷冰冰的墙,再也没有路了。索菲亚靠在墙上站着。

有一会儿他们看着从门口移动过来的人群。门离得很远。但是可以看出门在哪儿。因为人在进门的时候紧紧挤在一起,人体的白影子特别密集,等到进了宽敞的毒气室,就松散开了。

达维德看到一张张人脸。早晨下了火车之后,他看到的一直是许多脊背,现在好像一列火车的人都面对着他。索菲亚忽然变得和以前不同了;她的声音在这又平又宽敞的混凝土大厅里变了腔调,她一进入这大厅,整个样子都变了。她在说“我的好孩子,紧紧靠住我”的时候,他觉得她很怕丢了他,剩下她一个人。可是他们无法紧紧靠在墙上,而是离开了墙,又迈着碎步挪动起来。达维德觉得他比索菲亚挪动得快些。她的手攥住他的手,朝她跟前拉。可是有一种柔软的、渐渐增强的力量把达维德朝另一方向拉,索菲亚的手指头渐渐松了。

毒气室里的人群越来越密集,移动越来越慢,人的步子越来越小。没有人指挥这混凝土箱子里的移动。人在这毒气室里站着不动,还是漫无目的地在绕弯儿、转圈子,德国人觉得已经无所谓了。光光的孩子漫无目的地迈着小小的步子。他的又轻又小的身体的曲曲弯弯的移动路线和索菲亚那又大又重的身体的曲曲弯弯的移动路线渐渐不一致了,于是他们分开了。牵着他的手是拉不住他的,应该像那两个女的,那个妈妈和姑娘一样,脸贴着脸,胸膛贴着胸膛,哆哆嗦嗦地、死死地抱在一起,成为一个不可分的身体。

人越来越多,分子运动随着分子的密集渐渐偏离阿伏伽德罗定律[22]。小孩子丢掉了索菲亚的手,叫了起来。但是索菲亚马上成为过去。要紧的是现在,眼前。一张张人嘴在一起呼吸着,人的身体紧紧挨在一起,人的思想和感情也连成一体。

达维德落进了一部分旋转的人流中,这些人离开墙壁,朝门口倒流过去。达维德看到三个人紧紧抱在一起:两个男子保护着老妈妈,老妈妈保护着两个孩子。忽然在达维德旁边出现了新的人流,朝新的方向移动。响声也不同了,不是沙沙声和嘟哝声了。

“让开路!”有一个胳膊强劲有力、粗脖子、低着头的人想穿过紧紧靠在一起的人体。他想从沉闷的混凝土节奏中冲出去,他的身体就像鱼的身体在厨房案台上那样,在盲目地、没有目的地挣扎。他很快就喘不上气来,安静下来,倒换着脚,像大家一样了。

因为他的搅动,人流的移动有所改变,达维德又来到索菲亚身边。她使足劲儿把孩子紧紧搂在怀里,这种劲儿死亡集中营里的工人们是发现过的,也知道这种劲儿有多么大,他们在清理毒气室的时候,从来不想把抱在一起的亲人的尸体分开。

门口响起叫喊声。后面的人看到挤得紧紧的人群已经把毒气室塞得满满的,便不肯跨进敞着的门。

达维德看到门是怎样关上的:那钢门就好像被磁石吸引着,又从容又平稳地渐渐接近了钢门框,门与门框合在一起,成为一体。

达维德发现,在墙的上部,在一个方形的金属网罩里,有一个活物动了起来,他以为那是一只灰老鼠,不过他马上明白了,那是风扇转了起来。感觉到有一种淡淡的、甜丝丝的气味。

脚步声停止了,偶尔可以听到含糊不清的话、呻吟声、叫声。说话已经于人无益了,行动已经没有意义了,行动是为了未来,在毒气室里没有未来了。达维德的头和脖子不停扭动着,索菲亚却没有朝那活物的方向看看的念头。

她那双眼睛看过荷马史诗,看过《消息报》、迈因·里德的作品、黑格尔的《逻辑学》,看过许多很好的人和很坏的人,看过库尔斯克青草地上的鹅,在普尔科沃天文台看过星星,看过外科器械的亮光,在罗浮宫看过《蒙娜·丽莎》,看过市场上的番茄和芜菁,看过伊塞克湖的碧波,现在这眼睛没有用场了。这会儿要是有人把她的眼睛弄瞎,她也不会觉得是损失。

她呼吸着,但呼吸已成为一项沉重的工作,她使出所有的劲儿来进行呼吸工作。她想在震耳欲聋的钟声中聚精会神地最后想一想。但是什么也想不成。索菲亚一声不响地站着,也没有闭上什么也看不见的眼睛。

小孩子的动作常常使她心中充满怜惜之情。她对这孩子的感情极其单纯,不用说话,也不需要用眼睛看。这个垂死的孩子在呼吸着,但是他吸进的空气不是延长他的生命,而是毁灭他的生命。他的头转来转去,他还想看看。他看到倒在地上的人,他看到张开的没有牙的嘴,看到张开的露出白牙和金牙的嘴,看到从鼻孔里流出来的一道道鲜血。他看到隔着玻璃朝毒气室里看着的好奇的眼睛。罗捷那观望的眼睛有一小会儿和达维德的眼睛碰到一起。他还要说话,他还想问问索菲亚阿姨,那双眼睛为什么像狼的眼睛。他还要想一想。他在这世界上只走了几步,他见过孩子的光脚丫在热乎乎的土地上走出的脚印儿,他的妈妈还住在莫斯科,月亮朝下看着,眼睛可以从下面看到月亮,煤气炉上烧着开水,白头母鸡跑来跑去,他抓住蛤蟆的后腿,叫蛤蟆跳舞,还有早晨的牛奶—他依然想着这一切。

一双有劲的、火热的手臂一直搂抱着达维德。这孩子还不明白,他的眼睛黑了,心里咚咚响了一阵,就不响了,脑子里枯寂了,模糊了。他被杀死了,他不再存在了。

索菲亚·奥西波芙娜·列文顿感觉到,孩子的身体在她怀里软瘫了。她又失去了他。在地下坑道进行毒气试验的时候,用作毒气试剂的小鸟和老鼠一下子就会死去,因为小鸟和老鼠的身体很小。这孩子的身体小得像鸟儿一样,比她先走了一步。

“我做妈妈了。”她想道。

这是她最后一个念头。

可是她的心还活着:心在紧缩,疼痛,在怜惜你们,活着的和死去的人们。索菲亚感到一阵恶心,就把达维德,已经成了尸体的孩子紧紧搂在怀里,她也成了死人,成了尸体。

五 十

人死了,就是从自由的世界进入奴隶的王国。生命也就是自由,所以死的过程就是渐渐失去自由的过程;起初是知觉渐渐微弱,然后是渐渐消失;在失去知觉的肌体里,生命进程在一定时间内依然延续着,血液还在循环,还在呼吸,新陈代谢还在进行着。但这种向奴隶王国的败退是不可扭转的,因为知觉已经消失,自由的火已经熄灭。

夜空的星星暗淡了,银河不见了,太阳熄灭了,金星、火星、木星熄灭了,海洋寂然不动了,千千万万树枝寂然不动,风也寂然不动了,花儿不鲜艳也不芳香,粮食消失了,水也消失,空气的凉爽与闷热都消失了。人心中的宇宙不再存在了。这个宇宙和不依靠人而存在的唯一的宇宙惊人地相似。这个宇宙和依然存在于千千万万活人头脑中的宇宙惊人地相似。但是这个宇宙的特别惊人之处,是它有一种东西,这种东西使这个宇宙的海洋的涛声、鲜花的香味、树叶的沙沙声、花岗石的色彩、秋日田野的凄凉与存在于或者曾经存在于别人头脑里的宇宙的一切,与不依靠人而永久存在的那个宇宙的一切都不相同。一个生命的灵魂保持其独特性,便是自由。宇宙在人的意识中的反映是人的力量的基础,但是,只有当一个人作为一个在时间的长河中永远无人可以摹仿的世界而存在时,人生才是幸福,才是自由,才是最高的目的。只有这样,一个人才能感到自由和善良的幸福,才能在别人身上找到在自己身上找到的东西。

五十一

和莫斯托夫斯科伊、索菲亚·奥西波芙娜·列文顿一起被俘的司机谢苗诺夫,在靠近前线地区的集中营里忍饥挨饿过了十个星期之后,同一大批被俘的红军在一起,被押往西部边境。

在靠近前线的集中营里,他从来没有挨过拳头和枪托子,也没有挨过踢。

集中营里用饥饿惩罚。

水在小河里缓缓流动,哗哗响着,叹息着,拍打着岸边,可是,瞧,水轰轰响起来,狂号起来,翻滚着巨石,冲走大树,就像冲着麦秸一样,当你看到被挤压在狭窄河道里的河水震撼着山崖,当你觉得这好像不是水,而是许许多多沉重的透明铅块活了,站立起来,发起疯来的时候,会心惊胆战。

饥饿像水一样,永远自然地和生命联系着。所以饥饿有时会一下子成为消灭肉体、摧残扭曲灵魂、毁灭千千万万活物的力量。

饲料缺乏、冰封大地、草原和森林干旱、水灾和瘟疫可以使羊群和马群死亡,可以使狼、狐狸、唱歌的鸟儿、野蜂、骆驼、鲈鱼和毒蛇死去。人在自然灾害时候所受的苦难也和动物差不多。

国家能够按照自己的意愿用堤坝人为地、强制性地约束生活,挤压生活,这时候,可怕的饥饿的力量就像狭窄的河道里的河水一样,可以震动、扭曲、摧残和消灭人、部落、民族。

饥饿可以渐渐榨干人体细胞中的蛋白质和脂肪,饥饿可以使骨头变软,使孩子们的小腿佝偻和弯曲,可以使人贫血,头晕,使肌肉干瘪,破坏神经组织。饥饿可以重重地压在心上,把欢乐与信心赶走,可以消灭思考的能力,可以使人驯顺、低三下四、残忍、绝望和麻木不仁。

人性有时会完全灭绝,这饥饿的生物就会杀人,会吃死尸,会吃人。

国家能够筑起堤坝,把小麦、黑麦和种小麦、黑麦的人隔开,从而引起可怕的大批死亡,这种死亡类似德军围困期间列宁格勒几十万人的死亡,类似希特勒集中营里几百万战俘的死亡。

吃的呀!吃的东西!粮食!调味的佐料!大吃特吃!少吃点也行!有稀汤,有饭菜!油腻的,滋补的,大鱼大肉!营养搭配的伙食!穷家小户的家常菜!丰盛豪华的宴席,精致的佳肴!简单的,乡村的风味!美味的食物。充饥的食物。吃!吃!……

土豆皮、狗肉、蛤蟆、蜗牛、烂菜叶、发霉的甜菜、死马肉、猫肉、乌鸦和寒鸦的肉、腐烂的粮食、皮腰带、皮靴筒、糨糊、从军官食堂里流出来的油糊糊的泔水泡透的泥土—这都是吃的东西。这都是从堤坝里渗透出来的东西。

很多人在想方设法得到这些东西,分享这些东西,交换这些东西,互相偷窃这些东西。

在路上走到第十一天,当火车停在米海洛夫村车站的时候,押解队把昏迷过去的谢苗诺夫从车厢里拖出去,交给车站当局。

上了年纪的德国警备队长对着这个靠在消防棚墙上的半死不活的红军战士看了一会儿。

“让他爬到村子里去吧。要是把他关起来,过一天就会死。枪毙也不值得。”警备队长对翻译官说。

谢苗诺夫爬到了车站附近的一个村子里。第一户人家不让他进去。

“什么也没有,你走吧。”

门里有一个老妇人的声音对他说。

他来到第二家门口,敲门敲了很久,没有人应声,也许这一家已经没有人,也许从里面闩住了。

第三家的门半掩着,他走进过道,没有人喊住他。他走进屋子里,一股暖气朝他扑来。他的头发起晕来,躺到门口一条大板凳上。谢苗诺夫重重地、急促地呼吸着,一面打量着白色的墙壁、圣像、桌子、炉子。他在集中营里过了这么久之后,一见到这一切,十分激动。窗外闪过一个人影,一个妇女走进屋子,一看到谢苗诺夫,叫了起来:

“您是什么人?”

他什么也没有说。他是什么人,那是很清楚的。这一天,不是强大的国家的无情的力量,而是一个人,是赫里斯佳·丘尼娅克老大娘左右着他的生存和命运。

太阳从灰色云块的缝儿里凝望着战火纷飞的大地。在战壕、掩体、集中营的铁丝网、讲坛和特别科之上刮过的风,也来到小屋的窗前低声呼叫。

老大娘给谢苗诺夫端来一茶缸牛奶,他很费劲地、狼吞虎咽地喝了起来。他喝完牛奶,就呕吐起来。吐得肚子要翻出来,眼睛里流着泪水,他好像快要死一样,哧哧地直往里吸气,吐过了又吐。他拼命压制呕吐,脑子里只有一个念头:他浑身又脏又臭,老大娘会把他赶出去的。他用发红的眼睛看着老大娘拿来拖把,拖起地板。

他想对她说,他自己打扫,自己来擦洗,只要她不撵他走。但他只是嘟哝了两句,用哆哆嗦嗦的手指头比划了几下。时间一点一点地过去。老大娘一会儿走进来,一会儿又走出去。她没有撵谢苗诺夫走。也许,她找过邻居,请邻居去叫巡逻队或者警察?

老大娘把一铁锅水放到炉膛里。水烧热了,冒起热气。老大娘的脸露出忧愁的、不和善的神气。

谢苗诺夫心想:“她要把我撵走了,等我走了,她可以进行消毒。”

她从箱子里拿出褂子和裤子。她帮助谢苗诺夫把衣服脱了,把他的衣服包起来。他闻到了自己的肮脏身体的气味,闻到了浸过尿、血和屎的衬裤的气味。

她扶着他坐到一个木盆里。她的粗糙有力的手轻轻擦洗着他被虱子咬遍了的身体。热乎乎的肥皂水在他的胸前背后流着。他忽然哽咽起来,浑身哆嗦起来,一面吞着鼻涕,尖声叫起来:

“妈妈……好妈妈……好妈妈……”

她用灰色的粗麻布手巾揩干他的流泪的眼睛、头发、肩膀。她搀扶着他坐到板凳上,弯下身子,揩干了他那像麻秆一样细的腿,给他穿上褂子和内裤,扣上用布结成的扣子。

她把盆里的水倒进桶里,把又黑又臭的脏水提出去。

她把一张羊皮筒子铺到炕上,上面蒙上带条纹的麻布,又从床上拿来一个大枕头,放好。

然后她像搀一只小鸡一样,轻轻地把谢苗诺夫搀起来,帮助他爬到炕上去。

谢苗诺夫迷迷糊糊地躺着。他的身体感触到难以想象的变化:残酷的世界一心想消灭这受尽折腾的牲畜的企图再也不能实现了。

但是不论在集中营里,还是在火车上,他都没有像现在这样感到难受。两腿麻木,手指酸痛,骨头疼得厉害,恶心,头脑里乱糟糟的,有时忽然轻飘飘、空荡荡的,发起晕来,眼睛刺疼,不住地打嗝儿,眼皮发痒。有时心里发闷,发慌,胸口说不出的难受,好像就要死了。

过了四天。谢苗诺夫下了炕,开始在屋里走动。他感到惊奇的是,好像世界上有许多吃的东西。在集中营里却只有烂甜菜吃。似乎世界上只有稀稀的糊,只有集中营里的发臭的稀汤。

可是现在他看到了小米、土豆、白菜、猪油,他听到了公鸡的叫声。

他像个小孩子一样,觉得世界上好像有两个魔术师,一个善良的魔术师,一个凶恶的魔术师,他很怕凶恶的魔术师又把善良的魔术师打败,那样温暖、有饭吃、善良的世界就要消失,他又要用牙齿啃自己的皮腰带。

他摆弄起一盘手推的磨,因为这手磨的工作效率实在太低。磨几把灰灰的粗面,就要弄得满头大汗。

谢苗诺夫用锉刀和砂纸把传动杆打磨光了,又把连接传动杆与磨盘的栓紧了紧。他这个有文化的莫斯科机械师认为该做的,都做了,对乡下木匠做的粗糙的活儿进行了加工,但是在这之后,手磨更不灵活了。

谢苗诺夫躺在炕上,思考着怎样才能更好地磨面粉。早晨他又把手磨拆开,使用了轮子和旧挂钟的部分零件。

“赫里斯佳大娘,您来看看!”他带着自夸的口气说,并且指了指他安装的双齿轮传动装置。

他们彼此几乎不说什么话。她没有说过她那死于一九三〇年的丈夫,没有说过失去音信的儿子,也没有说过嫁到普里卢基、忘记了妈妈的女儿。她也没有问他,是怎样被俘的,是什么地方人,是乡下人还是城里人。

他怕到外面去。每次在上院子里去之前,先要朝窗外观察半天,而且总是急急忙忙回到屋里。如果关门的响声大了,或者茶缸掉在地上,他就害怕,好像好日子完了,赫里斯佳老大娘再也无能为力了。

有时邻居上赫里斯佳大娘家来,谢苗诺夫就爬到炕上躺着,尽可能不大声喘气,不打喷嚏。不过,邻居不是经常来。

村子里没有驻扎德国兵。他们驻扎在车站附近的铁路工人村里。

他想到周围在进行战争,而自己在这儿过温暖与安宁的日子,并不觉得有愧,他很怕再一次落入集中营和饥饿的世界。

他早晨醒来,很怕马上睁开眼睛,似乎了一夜魔法消失了,他又要看到集中营的铁丝网和警备队,又要听到空饭盒的响声了。

他闭着眼睛躺着,听听赫里斯佳老大娘是不是消失了。

他很少去想不久前的日子,不去回想政委克雷莫夫、斯大林格勒、德国集中营、押送俘虏的火车。但是每天夜里他都在梦里哭和叫。

有一天夜里他从炕上爬下来,在地上爬了一会儿,躲到板床底下,在板床底下睡到天亮。早晨起来,他想不起他梦见了什么样可怕的事。

有几次他看到载重汽车载着土豆和粮食从村里道路上经过,有一天他还看到一部小轿车。马达很好,车轮在泥水里也不打滑。

有时他想象着德国巡逻队在过道里叽哩哇啦说起话来,马上就会冲进屋里来,他的心就会打颤。

他向赫里斯佳老大娘问过德国人。

她回答说:

“有些德国人不坏。在我们这儿打仗的时候,我这屋子里住过两个德国人,一个是大学生,一个是画家。他们常常和孩子们一块儿玩。后来住过一个汽车司机,他还带着一只小猫。他开车回来,小猫就跟他玩儿。小猫好像是从边境上跟他来的。他吃饭时也要把小猫抱在怀里。他对我也很好,给我拉来不少木柴,有一次还给我丢下一口袋面粉。可是有些德国人很坏,杀小孩子,杀老头子,不拿我们当人,随便朝人家里跑,在女人面前光着身子。我们乡下的警察也有这样的,对人很凶。”

“咱们可是没有像德国人那样的野兽。”谢苗诺夫说。接着又问道:“赫里斯佳大娘,我住在您家里,您不害怕吗?”

她摇了摇头,说村子里有很多放回来的俘虏,当然,那都是回自己村子的乌克兰人。不过她可以说,谢苗诺夫是她的外甥,是嫁到了俄罗斯的姐姐的儿子。

谢苗诺夫已经认识了一些邻居和街坊,认识了第一天没有让他进门的那个老妇人。他知道,晚上姑娘们常常去车站看电影,每到礼拜六,车站上有乐队演奏,有舞会。他很想知道,德国人在电影院里放什么样的电影。但是上赫里斯佳大娘家里来的只有老年人,他们不看电影。没有人可以问。

邻居一位大娘拿来女儿的来信,女儿是参加招工上德国去的。信里有好几处地方谢苗诺夫不懂,于是别人解释给他听。那姑娘在信中写着:“万尼亚和格里沙飞来了,窗上安上了玻璃……”这就是说,万尼亚和格里沙是在空军服役,苏联空军轰炸了德国的城市。

那姑娘在另外一处写着:“雨下得很厉害,就像巴赫马奇那样。”这也是指飞机轰炸,因为在战争初期,巴赫马奇车站常常受到很强烈的轰炸。

这天晚上,有一个高高的瘦老头子来到赫里斯佳大娘家。他把谢苗诺夫打量了一遍,便用地道的俄语说:

“好汉,你从哪儿来?”

“我是俘虏。”谢苗诺夫回答说。

老头子说:

“我们都是俘虏。”

他在沙皇时代当过炮兵,炮兵的一些号令他还记得很清楚,并且当着谢苗诺夫的面表演起来。他发号令用俄语,用嗄哑的声音,可是报告结果声音却很响亮,像个年轻人一样,并且还带有乌克兰口音,看样子,他是在模仿几十年前长官的声音和他自己的声音。

后来他骂起德国佬。

他对谢苗诺夫说,起初人们指望德国人解散集体农庄,可是结果德国人想到,集体农庄对他们也是好事情。他们也搞起五户小组、十户小组,和原来的生产小组、生产小队一样。赫里斯佳大娘用长长的、伤心的语调说:

“唉,集体农庄呀,集体农庄!”

谢苗诺夫说:

“集体农庄有什么!谁都知道,咱们到处都有集体农庄。”

赫里斯佳大娘说:

“你住嘴。你可知道,外地人怎样成群成群上我们这儿来的吗?一九三〇年,整个乌克兰都在瞎折腾。天天吃荨麻,吃黄土……把粮食全部弄走,一粒不剩。我男人饿死了,我又是受的什么样的罪呀!我浑身浮肿,话也不能说,路也走不动。”

谢苗诺夫听赫里斯佳大娘说她也和他一样挨过饿,十分吃惊。他总觉得,饥饿和瘟疫和这个善良人家的大娘是无缘的。

“也许,你们家是富农吧?”他问道。

“哪儿是什么富农呀!所有的人都遭殃呀,比战争时期还糟。”

“你是乡下人吗?”老头子问。

“不是,”谢苗诺夫回答说,“我是在莫斯科出生和长大的,我父亲也是在莫斯科出生和长大的。”

“是啊,”老头子带着自夸的口气说,“如果你那时候也参加了集体化,也会完蛋,城里人嘛,说完蛋就完蛋。为什么我活下来啦?我懂得野生草木。你以为我说的是橡子、椴树叶、荨麻、滨藜吧?这些东西大家一下子就吃光了。可是我知道五十六种能吃的野草。所以我活下来了。春天刚刚来到,还看不到一片叶子,我就在地里挖草根吃。伙计,我什么都认识,每一样根、皮、花儿我都认识,每一棵草我都认识。牛、羊、马全死了,可是我没有死,我比牛、羊、马更会吃草。”

“你是莫斯科人吗?”赫里斯佳大娘慢慢地重问了一遍。“我还不知道你是莫斯科人呢。”

老头子走了,谢苗诺夫躺下睡了,可是赫里斯佳大娘用手托着腮坐着,望着黑黑的夜空。那一年是丰收年景。小麦长得密密麻麻,齐齐整整,和她的瓦西里的肩膀一样高,把赫里斯佳连头都遮住。

村里到处可以听到微弱而缓慢的呻吟声,骨瘦如柴的孩子在地上爬着,有气无力地哭着;饿得连喘气也没有劲儿的男子汉拖着水肿的腿在外面晃悠着。妇女们到处找东西吃,什么都吃:荨麻,橡子,椴树叶,掉在外面的马蹄,骨头,牛角,羊角,未加工的羊皮……然而从城里来的小伙子们还在一家一家地转悠着,不管死人,也不管半死不活的人,打开地窖,在棚子里挖坑,拿铁钎子插进地里,寻找和收缴富农藏的粮食。

在一个闷热的夏日里,她的瓦西里死了,停止了呼吸。这时候从城里来的小伙子们又来到屋里,其中有一个蓝眼睛的人,说话带俄罗斯口音,就和谢苗诺夫一样,走到死者跟前,说:

“富农顽抗到底,毫不怜惜自己的命。”

赫里斯佳叹了一口气,画了一个十字,便去铺床。

五十二

维克托·施特鲁姆原以为,他的研究只能得到狭小的理论物理学界的重视。但事实不是这样。近来给他打电话的不只是一些熟识的物理学家,还有一些数学家和化学家。有些人请他解释问题,因为他的数学推论太复杂了。

有的学生会代表到研究所来找他,请他给物理系和数学系高年级学生作报告。他在科学院做过两次报告。马尔科夫和萨沃斯季扬诺夫告诉他,在很多研究所的实验室里都在对他的研究进行争论。

柳德米拉在限额供应商店里听到一位科学家的夫人问另一位夫人:“您站在谁后面?”那位夫人回答说:“这不是,我站在施特鲁姆夫人后面。”原来发问的那位夫人说:“这就是施特鲁姆夫人吗?”

维克托并没有表露出他因为自己的论文引起这样不同寻常的广泛关注而感到高兴。但是他对荣誉不是无动于衷的。在研究所的学术委员会会议上,他的论文被推选为斯大林奖金备选项目。维克托没有出席这次会议,但是这天晚上他一直注视着电话机,等着索洛科夫给他打电话。可会后第一个给他打电话的是萨沃斯季扬诺夫。

往常爱嘲笑人甚至爱说下流话的萨沃斯季扬诺夫,现在说话的口气不一样了。

“这是胜利,了不起的胜利!”他一再地说。

他说了说普拉索洛夫院士的发言。这位老院士说,自从他的研究辐射压力的老朋友列别杰夫去世以后,在物理研究所里还没有出现过这样有分量的论文。

斯维琴教授谈到维克托的数学方法,说这种方法本身就有创新成分。他说,只有苏联人才能在战争环境中这样忘我地为人民的事业贡献自己的力量。

还有很多人发言,马尔科夫也发了言,但是最响亮、最带劲儿的话是古列维奇说的。

“他是好样的,”萨沃斯季扬诺夫说,“他说的话最实在,说话不带框框儿。他说您的著作是经典性的,说应该把您的著作和原子物理奠基人的著作,如普朗克、玻尔、费马的著作,排在同样的位置。”

“真带劲儿。”维克托在心里说。

萨沃斯季扬诺夫打过电话不久,索科洛夫又打来电话。

“今天我不上你们家去了,抽出二十分钟和您在电话里谈一谈吧,我实在太忙了。”他说。

索科洛夫也十分激动,十分高兴。

维克托说:

“我忘记了问萨沃斯季扬诺夫表决的情形。”

索科洛夫说,表示反对的只有从事物理理论研究的加甫罗诺夫教授。他认为,维克托的著作建立在很不科学的基础上,来源于西方物理学家的观点,实际上是不顶用的。

“加甫罗诺夫反对,这倒是好事。”维克托说。

“是啊,也许是好事。”索科洛夫也说。

加甫罗诺夫是一个怪人。大家戏称他“斯拉夫兄弟派”。他带着一股狂热而顽强的劲头千方百计地要证明,物理学的一切成就都和俄国科学家的著作有关系,他把很少有人知道的一些名字,如别特罗夫、乌莫夫、亚可甫列夫,看得比法拉第、麦克斯韦、爱因斯坦还要高。

索科洛夫用开玩笑的口吻说:

“维克托·帕夫洛维奇,您瞧,整个莫斯科都承认您的著作的重大意义了。不久就要为您举行庆祝宴会了。”

玛利亚接过话筒,说:

“恭喜您,请代我向柳德米拉表示祝贺。我为您、为她感到非常高兴。”

维克托说:

“这都算不了什么。”

可˜¯这种“算不了什么”使他非常高兴,非常激动。

夜里,柳德米拉已经在铺床准备睡觉了,马尔科夫打来电话。他是一个熟悉官场情形的人。他用和萨沃斯季扬诺夫、索科洛夫不同的语气说了说学术委员会会议的情形。古列维奇发言以后,科甫琴科在一片笑声中说:“连数学研究所里都敲起钟来,围绕着维克托·帕夫洛维奇的论文闹腾起来。虽然没有什么宗教游行,可是已经有人举起神幡。”

多疑的马尔科夫感觉到科甫琴科的笑话是带有恶意的。他观察到的另外一些情形都和希沙科夫有关系。希沙科夫没有说出自己对维克托的论文的看法。他听着别人发言,只是不时地点点头,也许是表示赞成,也许那意思是:“等着瞧吧。”

希沙科夫极力推荐年轻教授莫洛堪诺夫的著作为斯大林奖金备选项目。他的著作是论述钢的伦琴射线分析的,实用范围很小,只是对于生产优质钢的某些工厂有意义。

马尔科夫又说,散会之后,希沙科夫就走到加甫罗诺夫跟前,和他谈起来。

维克托说:

“马尔科夫同志,您最好到外交部门去工作。”

不善于开玩笑的马尔科夫回答说:

“不,我还是做我的物理试验。”

维克托走到柳德米拉的房间里,说:

“推荐我领取斯大林奖金啦。他们说了不少使我高兴的事情。”

他又对她说了说参加会议的人发言的情形:

“所有这些官方的赞许,都是狗屁不值。不过你要知道,我讨厌透了那种长期形成的莫名其妙的局面。上大厅里去开会,第一排座位常常空着,但是我不敢去坐,总是坐到最后一排,可是希沙科夫、波斯托耶夫却总是毫不犹豫地坐到主席团位子上去。我瞧不起主席团的交椅但是在心里希望自己至少有资格坐这样的交椅。”

“要是托里亚知道了,才高兴呢。”柳德米拉说。

“这事儿我也不能写信向妈妈报告了。”

柳德米拉说:

“维克托,已经十二点了,娜佳还没有回来。昨天她十一点就回来了。”

“会有什么事呢?”

“她说她是上好朋友玛伊卡家里去,可是我很不放心。她说,玛伊卡父亲的汽车有夜晚通行证,他可以把她送到咱们的街口。”

“那还有什么不放心的?”维克托说过这句,心里想道:“真是的,正谈着巨大的成就,谈着国家的斯大林奖金,干吗要拿家庭琐事把这样的谈话打断?”

他没有说出口来,只是轻轻叹了一口气。

在学术会议之后的第三天,他往希沙科夫家里打了一次电话,他想请他为年轻物理学家兰杰斯曼安排工作。科学院管委会和人事处一直拖着不肯办手续。同时他想请希沙科夫设法快一点儿把安娜·纳乌莫芙娜从喀山调回来。现在,在研究所安装新设备的时候,把有技术特长的工作人员留在喀山,是没有意义的。

他早就想和希沙科夫谈谈这些事了,但是他觉得希沙科夫也许会不大客气地说:“您去找副所长谈吧。”所以维克托一直拖着没有谈。

现在,成功的浪波激起了他的劲头。十天之前他还觉得去见希沙科夫是很不合适的,可是今天他觉得往希沙科夫家里打电话是很平常自然的了。

一个女人的声音问道:

“您是谁?”

维克托报了姓名。他报得那样从容,那样镇静,他听着自己的声音感到十分愉快。接电话的女子迟疑了一下,然后很亲切地说:

“请等一会儿。”

可是过了一会儿她又很亲切地说:“对不起,请您明天上午十点钟往研究所打电话。”

“对不起,打搅了。”维克托说。

他浑身感到热辣辣的,很不舒服。

他闷闷不乐地揣度着,恐怕晚上在梦里也摆脱不了这种不舒服的感觉,等早晨醒来,会在心里想:“为什么这样恶心?”然后会想起来:“哦,都是因为这次愚蠢的电话。”他来到柳德米拉房间里,说了说给希沙科夫打电话没有打成。

“是啊,是啊,王牌打得不是地方,就像你妈妈常说我的。”

他又骂起接电话的那个女人:“他妈的,那母狗,我真受不了官腔官调的那一套:先问我是什么人,然后回答说,老爷没有工夫接电话。”

柳德米拉在类似的情况下一般都要生气的,他很想听听她的说法。

“你该记得,”他说,“我曾经说过,希沙科夫态度冷淡是因为他不能靠我的论文捞到什么资本。可是现在他觉得可以捞到资本了,不过捞到的是另一种资本:可以贬低我。因为他知道,上面有人不喜欢我。”

“哎呀,你担心的事太多了,”柳德米拉说,“现在什么时间啦?”

“九点一刻。”

“瞧,娜佳还不回来呢。”

“哎呀,”维克托说,“你担心的事太多了。”

“顺便说说,”柳德米拉说,“今天我在商店里听说:斯维琴也被推荐为奖金备选人了。”

“你看,有这种事,他没有告诉我呀。他凭什么被推荐?”

“好像因为散射理论。”

“真是莫名其妙。他的论文是在战前发表的呀。”

“那有什么关系。过去发表的东西也可以得奖。他会得奖的,你得不到。你就等着瞧吧。这都怪你自己。”

“柳德米拉,你太糊涂了。上面有人不喜欢我呀!”

“你需要的是我母亲。她处处都附和你。”

“我真不明白你为什么有这样的火气。如果当初你对我妈所表现的亲热,能有我对你妈所表现的十分之一就好了。”

“可是你妈从来就没有喜欢过托里亚。”柳德米拉说。

“不是这样,不是这样。”维克托说。他觉得妻子也成了外人,她是那样顽固和不讲理,让人感到可怕。

五十三

第二天早晨,维克托从索科洛夫口里听到一桩新闻。头天晚上,希沙科夫把研究所里一些人请到家里去了。索科洛夫去了,紧接着科甫琴科也坐着小汽车到了。

在被邀请的人当中还有党中央科学处年轻的处长巴季因。

维克托觉得很不自在:显然,他给希沙科夫打电话,正是高朋满座的时候。

他冷冷笑着对索科洛夫说:

“在被邀请的宾客中还有圣热曼伯爵呢,先生们究竟谈了些什么?”

他忽然想起来,在给希沙科夫打电话的时候,还用那样从容的语调报自己的姓名,相信希沙科夫一听到“施特鲁姆”,马上就会高高兴兴地跑了来呢。他想起这一点,甚至懊恼得叫了起来,心里想,狗要抖掉咬得它受不了的虼蚤却抖不掉,就是这样叫的。

“顺便说说,”索科洛夫说,“这次招待得很好,完全不像在战争时期。咖啡,真正的古尔贾尼葡萄酒。人也不多,只有十来个人。”

“很奇怪。”维克托说。索科洛夫马上明白了这意味深长的“很奇怪”指的是什么,他也意味深长地说:

“是啊,不完全清楚。更确切地说,完全不清楚。”

“古列维奇去了吗?”维克托问道。

“古列维奇没有去,好像给他打过电话,他在指导研究生试验。”

“哦,哦,哦。”维克托说着,用手指头敲起桌子。过了一会儿,出乎自己的意料,他忽然向索科洛夫问道:“索科洛夫同志,大家没有说起我的论文吗?”

索科洛夫踌躇了一下,说:

“维克托·帕夫洛维奇,我有这样一种感觉,很多人称赞您,崇拜您,是在帮您的倒忙,因为这样领导很生气。”

“您怎么不明说呢?嗯?”

索科洛夫告诉他,加甫罗诺夫说起维克托的论文,说论文中的观点与列宁主义的物质观相矛盾。

“噢?”维克托说。“那又怎么样呢?”

“是啊,您要知道,加甫罗诺夫是胡说八道,不过总是很不愉快的事。巴季因就支持他的说法。似乎是这样,您的论文尽管有不少独到的见解,但是和那次有名的会议上所定的方针是抵触的。”

他回头朝门口看了看,又朝电话机看了看,然后小声说:

“您要知道,我觉得,因为要开展维护科研的党性的运动,咱们研究所的领导可能有意选定您做替罪羊。您该知道咱们的运动是怎样进行的。选定一个牺牲品,拼命来折腾。这真是可怕呀。您的论文可是真了不起,真难得呀!”

“怎么,就没有人表示不同意见吗?”

“好像没有。”

“您呢?”

“我认为争论是没有意义的。反正无法推翻他们的定论。”

维克托感觉出朋友的尴尬,也不好意思了,就说:

“噢,噢,当然,当然,您说得很对。”

他们都沉默着,但这种沉默并不令人感到轻松。维克托感到毛骨悚然的恐惧,触发了平时隐藏在心中的恐怖感。他害怕国家发怒,怕自己成为国家发怒的牺牲品,国家发起怒来,可以使人变为齑粉。

“是啊,是啊,是啊,”他意味深长地说,“不图发胖,只求活命就行啦。”

“我多么希望您能明白这一切呀。”索科洛夫小声说。

“索科洛夫同志,”维克托也用小声问道,“马季亚罗夫在那儿怎么样,平安无事吗?他有信给您吗?我有时十分担心,自己也不知道因为什么。”

他们突然用低声耳语交谈,好像是在特意表示:人与人之间还有自己的、特别的、人性的、国家以外的关系。

索科洛夫沉着地、一个字一个字地回答说:

“没有,我没有收到喀山方面任何信件。”

他平静而响亮的声音好像在说:这些特别的、人性的、国家以外的关系现在对他们毫无意义了。

马尔科夫和萨沃斯季扬诺夫走进办公室,谈起完全不同的话题。马尔科夫举了一些例子,说明一些妻子搅得丈夫过不好日子。

“有什么样的丈夫,必然有什么样的妻子。”索科洛夫说过这话,看了看表,便走出办公室。

萨沃斯季扬诺夫对着他的背影笑着说:

“如果在电车上只有一个位子,必然是他坐上去,他的玛利亚站着。如果夜里有人来电话,他再也不会从床上起来,而是玛利亚穿了睡衣跑去问:‘您是哪位?’显然,这样的妻子是一个人的好伙伴。”

“我不在幸福者之列,”马尔科夫说,“我常常听到命令:‘你怎么,聋了吗,开门去!’”

维克托忽然生起气来,说:

“哼,您怎么啦,咱们怎么能比得上……索科洛夫是模范丈夫!”

“马尔科夫同志,您怕什么,”萨沃斯季扬诺夫说,“您现在日日夜夜在实验室里,老婆管不到了。”

“您以为,她因为我天天不在家,不骂我吗?”马尔科夫问道。

“当然啦,”萨沃斯季扬诺夫说着,舔了舔嘴唇,已经感觉出自己要说的俏皮话的滋味了,“你应该待在家里!正如俗话说的,我的家就是我的监狱嘛。”

马尔科夫和维克托都笑起来。马尔科夫显然担心这愉快的谈话会拖延下去,便站起来,自言自语地说:

“该干事情了。”

等他走出门去,维克托说:

“这样古板的一个人,动作一向慢条斯理的,现在却像喝醉酒一样了。的确是日日夜夜泡在实验室里。”

“是啊,是啊,”萨沃斯季扬诺夫也承认说,“他就像一只做窝的鸟儿。一头埋进工作里啦!”

维克托笑了笑,说:

“他现在连上流社会的新闻也不关心了,不再传播这种新闻了。是啊,是啊,我很喜欢做窝的鸟儿。”

萨沃斯季扬诺夫猛地转过脸来,朝着维克托。

他那淡黄色眉毛的年轻的脸是严肃的。

“正好,要谈谈上流社会的新闻,”他说,“维克托·帕夫洛维奇,应该说,昨天在希沙科夫家举行酒会,没有请您去,这是令人气愤的事,毫无道理的事……”

维克托皱了皱眉头,他觉得这种同情的话有伤他的尊严。

“您算了吧,别说了!”他不客气地打断了他的话。

“维克托·帕夫洛维奇,”萨沃斯季扬诺夫说,“当然,希沙科夫没有请您,算不了什么。不过,加甫罗诺夫说的话多么可恶,索科洛夫没有对您说过吗?只有丝毫不顾羞耻,才会说您的论文中有犹太教精神,才会说古列维奇称赞您的论文是经典性的,只因为您是犹太人。尤其是在领导者不出声的冷笑中说这些卑鄙的话。好一个‘斯拉夫兄弟’!”

在午休的时候,维克托没有上食堂去,他在自己的办公室里来来回回地踱着。他何曾想到,人世间有这样多卑鄙龌龊的东西!萨沃斯季扬诺夫倒是有头脑!可原来还以为他只会说说俏皮话,天天带着姑娘的泳装照片,是个头脑简单的小伙子呢。是啊,总的说,这一切都是小事。加甫罗诺夫的胡说八道根本算不了什么,他是一个精神变态的人,是一个爱嫉妒的小人。没有人反驳他,是因为他说的话太荒唐,太可笑。

可是这些小事、微不足道的事还是使他很不安,很难受。希沙科夫怎么能不请他呢?的确很不礼貌,很没有道理。特别有伤自尊心的是,平庸无才的希沙科夫和他的宾客们丝毫不把他放在眼里。他非常痛苦,就好像出了不幸的事,这一生都无法挽回了。他知道这是胡思乱想,可是自己拿自己没有办法。哼,哼,还想比索科洛夫多分一两个鸡蛋呢。休想!但是有一件事实实在在地使他伤心。他真想对索科洛夫说:“我的朋友,您怎么不羞愧?加甫罗诺夫那样诬蔑我,您怎么瞒着我?您在那儿不说话,也不对我说。您真不应该,真不应该啊!”

可是,尽管还在生气,他马上自己对自己说:“不过,你也没说话嘛。你也没有对朋友索科洛夫说,卡里莫夫怀疑他的亲戚马季亚罗夫嘛。你也没有作声!因为不好意思?怕伤和气?胡说!不过是害怕!”

显然,命中注定这一整天是不愉快的。

安娜·斯捷潘诺芙娜走进办公室,维克托看到她一脸愁容,问道:

“安娜·斯捷潘诺芙娜,出了什么事吗?”又在心里想道:“她是不是听说我的一些不愉快的事了?”

“维克托·帕夫洛维奇,”她说,“这样的事落到我头上了,为什么我落得这种下场?”原来,在午休时间人事处把她叫了去,要她写离职申请书。因为院长有指示;要解除没受过高等教育的试验员的职务。

“胡说八道,我真不明白这搞的是什么名堂,”维克托说,“我去叫他们别胡闹,请您放心。”

使安娜·斯捷潘诺芙娜感到特别难受的是杜宾科夫的话,他说,领导对她本人没有任何意见。

“维克托·帕夫洛维奇,这是怎么回事?”她说。“我妨碍您工作了,对不起,请您原谅我吧。”

维克托披上大衣,就穿过院子,朝人事处所在的二层楼走去。“好啊,好啊,”他在心里说,“好啊,好啊。”他再也没有多想。但是这“好啊,好啊”却包含着很多意思。

杜宾科夫一面和维克托打招呼,一面说:

“我正要找您呢。”

“为安娜·斯捷潘诺芙娜的事吗?”

“不是,那不必要。是因为有某些情况,研究所的主要工作人员需要填这样一份履历表。”

维克托看了看很多张表格纸订成的履历表,说:

“哎呀!这要花一个星期的工夫。”

“维克托·帕夫洛维奇,瞧您说的。不过,在填写否定项目的时候,不要划斜线,要写:没有,不是,未参加,等等。”

“我有一个意见,”维克托说,“应该取消解除我们的一级试验员安娜·斯捷潘诺芙娜·洛沙科娃职务的荒唐命令。”

杜宾科夫说:

“洛沙科娃吗?维克托·帕夫洛维奇,我怎么能取消院领导的命令啊?”

“鬼知道这算怎么一回事儿!她拯救了研究所,在炸弹底下保护了所里的财产。可是现在凭着形式上的理由解除她的职务。”

“没有形式上的理由,我们不会解除任何人的职务,”杜宾科夫很神气地说。

“安娜·斯捷潘诺芙娜不仅是一个极好的人,她还是我们实验室里最出色的工作人员之一。”

“如果她的确是无法代替的,那您就去找找科甫琴科同志,”杜宾科夫说,“正好,你们实验室里还有两个问题,您要征求他的同意。”

他把用别针别在一起的两张纸递给维克托。“这是关于选聘人员担任研究员职务的。”他朝一张纸看了看,慢慢念了念:“兰杰斯曼·艾米里·平胡索维奇。”

“哦,这是我写的嘛。”维克托认出杜宾科夫手里的纸,就说。

“这是科甫琴科同志的批示:不符合要求。”

“怎么不符合要求?”维克托问。“我知道他是符合要求的,科甫琴科怎么知道他不符合我的要求?”

“所以您要去和科甫琴科同志谈谈。”杜宾科夫说。他看了看另一张纸,说:“这是我们留在喀山的工作人员的申请书,也需要您去说说理由。”

“哦,怎么啦?”

“科甫琴科同志批的是:目前不宜调动,因为喀山大学的工作十分需要他们,这个问题放到学年结束时再研究。”

他说话声音不高,很温和,好像希望用亲切的声音软化这使维克托不愉快的消息,但是他的眼睛里却没有亲切的神气,只有不怀好意的好奇。

“谢谢您,杜宾科夫同志。”维克托说。

维克托又来到院子里,又一遍一遍地在心里说:“好呀,好呀。”他不需要领导的支持,不需要朋友的情谊,不需要和妻子心灵相通,他可以单独作战。他回到主楼,登上二层。

科甫琴科身穿黑色西装和乌克兰式绣花衬衣,紧跟着向他报告维克托来见的女秘书走出办公室,说:

“请,请,维克托·帕夫洛维奇,请进寒舍坐坐。”

维克托走进摆满了红色安乐椅和大沙发的“寒舍”。科甫琴科请维克托坐在沙发上,自己也挨着坐了下来。他一面听维克托说话,一面微微笑着,他的亲切神情很有点儿像杜宾科夫的亲切神情。而且,在加甫罗诺夫发言评论维克托的论文的时候,他好像也是这样微笑的。

“有什么办法?”科甫琴科把两手一摊,很伤心地说。“这不完全是我们自作主张啊。她曾经在炸弹底下吗?现在这已经不算功劳了。如果祖国有命令的话,每一个苏联人都会到炸弹底下去。”

后来科甫琴科沉思了一下,说:

“还有一种办法,虽然会有人找碴儿。可以把洛沙科娃调任制剂员。科技人员供应卡还给她留着。这我可以办到。”

“不行,这对她是一种侮辱。”维克托说。

科甫琴科问道:

“维克托·帕夫洛维奇,您是希望,苏维埃国家施行一种法律,在您的实验室里施行另一种法律吗?”

“恰恰相反,我正是希望在我的实验室里也施行苏维埃的法律。按照苏维埃法律,不能解除洛沙科娃的职务。”

维克托又问:

“科甫琴科同志,如果要谈法律的话,那您为什么不批准很有才华的小伙子兰杰斯曼进我的实验室?”

科甫琴科咬了咬嘴唇。

“您可知道,维克托·帕夫洛维奇,也许,按照您的要求,他能工作得很好,不过还有一些情况,是研究所的领导应该考虑的。”

“很好,”维克托说,又重复一遍,“很好。”

他又小声问:

“是履历问题吗?亲属在国外?”

科甫琴科不作回答,只把两手一摊。

“科甫琴科同志,如果这种愉快的谈话还能继续下去的话,”维克托说,“请问,为什么您不让我的同事安娜·纳乌莫芙娜·魏斯帕比尔从喀山回来?顺便说一句,她是副博士。我的实验室和国家有什么矛盾?”

科甫琴科带着受难者的脸色说:

“维克托·帕夫洛维奇,您怎么审问起我来了?我对干部负有责任呀,您要理解这一点。”

“很好,很好,”维克托觉得已经到了一点不客气地谈一谈的时候,就说,“那好吧,可敬的同志,我不能继续工作了。研究科学不是为杜宾科夫,也不是为了您。我在这儿也是为了工作,不是为了给人事处创造我无法知道的好处。我要给希沙科夫写报告,让他派杜宾科夫来主持核物理实验室好了。”

科甫琴科说:

“维克托·帕夫洛维奇,说实在的,不要激动嘛。”

“不,我就是不能再工作了。”

“维克托·帕夫洛维奇,您不知道,领导上,尤其是我,有多么看重您的工作。”

“至于你们看重我还是不看重我,我可是一点不放在眼里。”维克托说过这话,在科甫琴科脸上看到的不是生气的表情,而是快活与满意的表情。

“维克托·帕夫洛维奇,”科甫琴科说,“我们无论如何不能让您离开研究所。”

他皱起眉头,又说:

“而且也完全不是因为无人可以代替。难道您以为就没有人可以代替维克托·帕夫洛维奇·施特鲁姆吗?”

最后又用十分亲切的语调问道:

“如果您没有兰杰斯曼和魏斯帕比尔就不能从事科学研究的话,难道全苏联都没有人能代替您吗?”

他看着维克托,维克托感觉到,科甫琴科就要把一些话说出来了,那些话就像不见形迹的雾气,一直缭绕在他们中间,时时触及眼睛、手、脑子。维克托垂下头,这位做出了不起的科学发现的人,这位又傲慢又骄矜、又清高又尖刻的教授、博士和著名学者,顿时消失不见了。这个驼背、窄肩、鬈发、鹰钩鼻子的男子眯缝起眼睛,好像等着挨耳光似的,望着穿乌克兰绣花衬衫的人,等待着。科甫琴科轻轻地说:

“维克托·帕夫洛维奇,不要激动,不要激动,说实在的,不要激动。嗯,您怎么啦,真的,因为这样一点儿微不足道的事,吵闹起来啦。”

五十四

夜里,等妻子和女儿睡了,维克托就开始填履历表。履历表上几乎所有的问题都和战前一样。正因为这是一些老问题,所以他觉得这些问题提得很奇怪,因而使他重新惴惴不安起来。

国家操心的不是维克托在研究中使用的数学器械是否够用,正在实验室安装的设备是否能承担复杂的试验,中子辐射的防护设备是否完善,索科洛夫和维克托的关系及其在科研上的配合好不好,是否有足够的初级研究人员进行不厌其烦的计算,他们是否理解在很多方面全靠他们的耐心、长期的紧张和聚精会神。

这是最重要的调查表,是表中之王。它要了解柳德米拉的父亲的情况、她的母亲的情况,要了解维克托的爷爷和奶奶的情况,要了解他的爷爷和奶奶过去生活在哪里,死在哪里,葬在哪里。维克托的父亲巴维尔·约瑟弗维奇在一九一〇年因为什么去柏林?国家的担心是严肃认真的。维克托把履历表浏览了一遍之后,也传染上了疑心病,对自己家世的可靠和真实性产生了怀疑。

1:姓,名,父称……他是谁,这个在深夜里填履历表的是什么人,是施特鲁姆·维克托·帕夫洛维奇吗?父亲和母亲好像是在国外结婚的,在维克托满两岁的时候,他们又离婚了,他仿佛记得,在父亲的证件中,父亲的名字是宾胡斯,而不是巴维尔。为什么我的父称是帕夫洛维奇?我是什么人?我清楚自己的来历吗?万一我本来是姓哥尔曼,也许是姓萨盖塔奇内呢?也许是法国姓杰弗尔什,也就是俄罗斯的杜布罗夫斯基呢?

他满脑子疑虑,接着又开始填写第二项。

2:出生时间……年……月……日……写明新历与旧历。他约莫生于十二月的一天,可是他怎么知道的呢,他能肯定自己恰恰是生于这一天吗?为了推卸责任,是不是写明“听别人说的”?

3:性别……维克托满怀信心地写上“男”。可是他在心里说:“哼,我算什么男人呀,真正的男子汉见到契贝任被撤职,不会不说话的。”

4:出生地(旧的行政区划:省、县、乡、庄;新的行政区划:州、地区、区、村)……维克托写上:哈尔科夫。妈妈对他说过,他出生在巴赫穆特,可是他出生两个月以后,妈妈迁到哈尔科夫,在哈尔科夫领到他的出生证。怎么办,要不要加以说明?

5:民族……这是第五项。这样简单的、在战前毫无意义的问题,现在几乎成了特别重要的问题了。

维克托握紧笔,用清晰的粗体字写上:犹太族。他还不知道,对于几十万人来说,填写这第五项:加尔梅克族、巴尔卡尔族、车臣族、克里木鞑靼族、犹太族……很快将意味着什么?

他不知道,围绕这第五项发生的阴森可怕的事情会越来越多;他不知道,恐怖、厄运、绝望、没有前途、流血将从邻近的第六项“社会出身”迁徙转移到这一项;他不知道,几年之后,很多人将怀着命中不幸的心情填写这第五项,就像过去几十年中哥萨克军官、贵族和工厂主的子女、神甫的儿子填写邻近的第六项那样。

不过这时他已经感觉和预感到围绕着这第五项的强力线越来越密集。昨天晚上兰杰斯曼打电话给他,他告诉兰杰斯曼,安排工作的事还一点没有头绪。

“我估计就是这样嘛。”兰杰斯曼用恼恨的、责备维克托的口气说。

“是您的履历有问题吗?”维克托问道。

兰杰斯曼对着话筒哼了一声,说:“是我的姓有问题”。[23]

娜佳在晚上喝茶的时候说:

“爸爸,你可知道,玛伊卡的爸爸说,明年国际关系学院再也不招收犹太学生了。”

“好吧,”维克托心里说,“犹太族就犹太族,不能不写。”

6:社会出身……这是一株大树的树干,其树根深深扎进地里,树枝宽宽地铺展开来,下面是许许多多阔大的履历树叶:父亲和母亲的社会出身、父亲的父母的社会出身……妻子的社会出身、妻子的父母的社会出身……如果是离过婚的,还有前妻的社会出身、她的父母在革命前的职业。

伟大的革命是社会革命,是穷人的革命。维克托总觉得,在第六项中反映出穷人在受富人统治的几千年中产生的应有的不信任,是很自然的。他写上:小市民出身。小市民!他算什么样的小市民?!也许是战争启迪了他,他忽然怀疑起来:苏联正当地査询社会出身问题与德国人怀着血腥的目的查询民族属性问题,二者之间是否真正有什么本质的区别?他想起了在喀山晚间的一些谈话,想起马季亚罗夫说的契诃夫怎样看待人的一些话。

他想道:“我以为看重社会特征是有道理的,是应该的。而德国人认为看重民族特征是绝对有道理的。我知道了,毫无疑问,杀犹太人,仅仅因为他们是犹太人,这十分可怕。因为他们是人,他们每一个都是人,有好人、坏人、聪明人、蠢人、笨人、快活人、善良人、反应灵敏的人、吝啬鬼。可是希特勒说:都是一样,反正都是犹太人!我坚决反对!不过我们也有这样一种准则:反正不是贵族,反正是富农出身,是商人出身。至于他们是好人、坏人、有才华的人、善良人、愚蠢人、快活人,有什么相干?要知道,我们的履历表不是商人、神甫、贵族的履历表。是他们的孩子、孙子的履历表。怎么,他们的血统就是贵族血统,就像犹太血统一样吗?怎么,他们生来就是商人,就是神甫吗?这是胡说八道。女英雄索菲亚·佩罗夫斯卡娅是将军的女儿,不是普通的将军,是省长。把她赶走吧!可是当年抓住卡拉科佐夫的警察走狗科米萨罗夫如果填写第六项,也会写‘小市民’。还可以招收他上大学呢。斯大林说过:‘儿子不能为父亲负责。’不过斯大林又说:‘苹果与苹果总是相差不远。’好吧,小市民出身就小市民出身吧。”

7:社会成分……是职员吗?职员就是会计、文书等。他这个职员用数学阐明了原子核的衰变过程,职员马尔科尔想借助新的试验设备证实他这个职员在理论上的推断。

“很对嘛,”他在心里说,“就是职员。”他耸了耸肩膀,站起来,在房里走了一会儿,动了动手掌,好像要把什么人推开。然后他又坐下来,回答表上的问题。

……

29:本人或近亲是否受过审判或审查,是否被捕过,是否受过法律或者行政处分,何时,何地,受处分原因?如果处分已被撤消,说明何时撤消……

对维克托的妻子也提出同样的问题。他心中掠过一阵凉气。这可是不容争辩,不是开玩笑的。他的头脑中闪出一个一个名字……我相信他根本没有罪……是一个不适应现实的人……她是因为不告发丈夫被捕的,好像判了八年,我说不准,我没有和她通信,好像是在捷姆尼科夫,我是偶然听说的,在街上碰到过她的女儿……我记不清了,他好像是在一九三七年初被捕的,是的,被剥夺通信权利十年……

妻子的哥哥原来是党员,我过去很少和他见面;不论我,不论妻子,都不和他通信;岳母好像去看过他,是的,是的,那是在战前很久;他的第二个妻子因为不揭发他,也被送往劳改营,她已经在战争期间死了;他的儿子参加了斯大林格勒保卫战,是志愿参加的……我的妻子和第一个丈夫离婚了,她和第一个丈夫生的儿子,也就是我的继子,在保卫斯大林格勒的战斗中牺牲了……她的第一个丈夫被捕了,离婚之后,她就一点不知道他的情况了……至于为什么被捕,我可说不准,只是模模糊糊听说,好像是托洛茨基分子,不过我不相信,我对这种事丝毫不感兴趣……

维克托顿时充满无限的负罪感,觉得自己不清白。他想起一个悔过的党员在大会上说的话:“同志们,我不是我们的人。”

他忽然想反抗。我不是服服帖帖、百依百顺之辈!上面有人不喜欢我,不喜欢就不喜欢好了!我是孤独的,妻子也不关心我了,不关心就不关心好了!我不能栽诬不幸的人、清白无辜死去的人。

同志们,想到这种种事情,实在惭愧!很多人是无罪的,还有老婆、孩子,他们何罪之有?应该向这些人悔罪,请求他们饶恕。你们是不是想证实我不合格,使人对我不信任,因为我和无辜被害的人有亲戚关系?如果我有错误的话,那我的错误,就是在他们倒霉的时候帮助他们太少了。

可另外一条完全不同的思路却在同一个人的脑子里同时并行着。

我没有和他们保持联系。我没有和阶级敌人通过信,没有收到过从劳改营里来的信件,我没有给他们物质支援,过去和他们见面很少,很偶然……

30:有无亲属在国外(何地,何时出国,出国原因),是否同他们保持联系?

这新的问题增强了他的苦恼。

同志们,难道你们不了解,在沙皇俄国的条件下,侨居国外是免不了的吗?很多穷人侨居国外,爱自由的人侨居国外,列宁也在伦敦、苏黎世、巴黎居住过。为什么你们看到我的姑姑和叔叔以及他们的子女在纽约、巴黎、布宜诺斯艾利斯就眨眼睛呢?……不记得是哪一位朋友说俏皮话:“姑妈在纽约呀……以前我以为,饥饿不是姑妈,却原来,姑妈就是饥饿。”[24]

不过,实在也可观,他在国外的亲属的名单竟比他的论文篇目单短不了多少。如果再加上被镇压的亲戚名单呢?……

好啦,这么看,这个人完啦。进垃圾堆去吧!异己分子!不过这不对头,不对!科学用得着他,而不是加甫罗诺夫和杜宾科夫;他可以为自己的国家献出生命。履历很光彩而善于欺骗和出卖的人还少吗?不是有很多人在履历表上写的是“父亲:流氓”、“父亲:地主”,而在战斗中献出了生命,参加了游击队,走向断头台吗?

这是怎么一回事儿?他知道:这是统计方法!是可能性!在非劳动出身的人中间遇到敌人,比在无产者出身的人中间遇到敌人的可能性大。不过要知道,德国法西斯也是根据可能性大小在消灭一些国家的人民和民族。这种原则是很不人道的。既不人道,又不讲理。对待人只能用人道的办法。维克托一定要设计出另外一种履历表,好使实验室能够招纳人才,那将是人道主义的履历表。

他觉得,和他一起工作的人是俄罗斯人还是犹太人、乌克兰人、亚美尼亚人,都无所谓,其祖父是工人还是老板、富农,都无所谓;他对待共同工作的同志的态度,不是看这位同志的兄弟是否被保安机关逮捕;这位同志的姐妹住在科斯特罗马还是日内瓦,他觉得都无所谓。

他要问的是:您从什么时候开始研究理论物理,您怎样看待爱因斯坦对普朗克老头的批评,您是光喜欢数学推论,还是也喜欢进行试验,您怎样看待海森堡的观点,您是否相信有可能列出统一的磁场方程式?最主要、最主要的,是能力、热情、才气。

如果共同工作的同志愿意回答的话,他还会问,喜欢不喜欢散步,喜欢不喜欢喝酒,是否喜欢听交响乐,是否喜欢塞顿—汤普森为孩子们写的书,托尔斯泰与陀思妥耶夫斯基哪一个更伟大,是否喜欢种花、钓鱼,是否喜欢毕加索,契诃夫的哪一篇小说最好?

他感兴趣的,是将和他一起工作的同志喜欢沉默寡言还是喜欢聊天,是否善良,是否风趣,是不是爱忘事,是不是爱发火,是不是爱面子,会不会和俊俏的薇拉·波诺马列娃干什么风流事儿。

有关这方面的事,马季亚罗夫说得非常好,正因为说得太好了,所以大家都觉得,莫非他是奸细。

天啊,我的天啊……维克托拿起笔,写道:

“艾斯菲莉·谢苗诺芙娜·塔舍夫斯卡娅,姨母,从一九〇九年侨居布宜诺斯艾利斯,音乐教师。”

五十五

维克托走进希沙科夫的办公室,有意地控制着自己,不说一句尖刻的话。他明白:因为他和他的论文在这位当官的院士头脑里处在最差、最末尾的位子上而生气和感到委屈,是很愚蠢的。但是维克托一看到希沙科夫的脸,就感到忍不住要发火了。

“希沙科夫同志,”他说,“俗话说,强扭的瓜不甜,不过,您从来没有关心过设备安装。”

希沙科夫很和气地说:

“一定在最近上你们那儿去看看。”

这位所长恩意隆隆,保证光临,好让维克托感到幸福。

希沙科夫又说:

“总的来说,我觉得,领导上对你们各方面的需要,还是相当关心的。”

“特别是人事处。”

希沙科夫非常和气地问:

“人事处有什么地方给您造成不便?您可是第一个说这种话的实验室领导呀。”

“希沙科夫同志,我想把魏斯帕比尔从喀山调回来,她在核摄影方面是独一无二的专家,却调不回来。我坚决反对解除洛沙科娃的职务。她是一个极好的工作人员,一个极好的人。我实在无法想象,怎么能解除洛沙科娃的职务。这是不合情理的。还有,我要求正式批准选聘的副博士兰杰斯曼的学位。他是一个有才华的小伙子。您还是对我们的实验室重视不够。要不然就不需要说这些话来浪费我的时间了。”

“说这些话也浪费我的时间。”希沙科夫说。

维克托很高兴,因为希沙科夫不再用和善的口气跟他说话了,如果还用和善的口气,他是不好发火的。于是他说:

“令人很不愉快的是,这些问题基本上都是围绕着姓犹太姓的人产生的。”

“原来是这样。”希沙科夫说。他从和平转向进攻。“维克托·帕夫洛维奇,研究所担负着重要的任务。我们是在多么困难的时期担负这样的任务,这是毋须对您说的。我认为,您的实验室在目前不能充分促成这些任务的完成。还有,围绕着您的论文,嚷嚷得太厉害了,您的论文毫无疑问是很有意思的,但也毫无疑问是有争议的。”

他继续咄咄逼人地说:

“这不光是我的看法。很多同志认为,这种嚷嚷会引起科学工作人员思想混乱。昨天有关方同我详细地谈过这个问题。有这样的意见:您应该重新考虑您的论断,您的论断与唯物主义的物质观相矛盾,您应当自己出面谈谈这个问题。有些人出于令我不解的用心,希望在我们应当全力以赴地完成战争提出的任务的时候,把有争论的理论宣布为科学的总方向。这是极其严重的。您却来对什么洛沙科娃的事表示怎样怎样的不满。对不起,我从来不知道洛沙科娃是犹太姓。”

维克托听着希沙科夫的话,不知如何是好了。从来没有谁当面表示反对他的论文。现在他是第一次从这位院士,从他所在的研究所的领导人嘴里听到。

他已经不怕什么后果,一股脑儿把他所想的、因此也就不该说的,全说了出来。

他说,物理学的存在,不是为了证明哲学的正确性。他说,数学推断的逻辑性,胜过恩格斯和列宁理论的逻辑性,党中央科学处的巴季因可以使列宁的观点适用于数学和物理学,而不能使数学和物理学适用于列宁的观点。他说,狭窄的实用主义对科学是有害的,不论这实用主义来自什么人,“就算是来自上帝也罢”;只有伟大的理论能产生伟大的实践。他相信,许多重大的技术问题,而且不只是技术问题,在二十世纪还要依靠核反应理论来解决。如果希沙科夫没有说出名字的那些同志们认为有必要让他发言的话,他很乐意按照这样的精神说一说。

“至于姓犹太姓的一些人的问题,希沙科夫同志,如果您真是俄罗斯知识分子的话,就不应该用开玩笑作回答,”他说,“如果您不答应我的上述要求的话,我只有立即离开研究所。我无法在这儿工作。”

他换了一口气,看了看希沙科夫,想了想,又说:

“在这种情况下,我很难工作下去。我不光是一个物理学家,我还是一个人。我无颜面对等待我帮助、等待我说公平话的人。”

他在说“在这种情况下,我很难工作下去”的时候,就没有勇气再说一遍立即离开的话了。维克托从希沙科夫脸上看出来,他已经发现了这种和缓的说法。

也许正因为这样,希沙科夫强硬起来:

“咱们没有必要用最后通牒式的语言继续谈下去了。我当然不能不考虑您的愿望。”

在整个一天里,维克托一直怀着一种又难受又高兴的奇怪感情。实验室里的仪器和即将安装好的新设备似乎一直就是他的生活、头脑和身体的一部分。他怎么能离开它们单独生存呢?

想起他对所长说的一番离经叛道的话,就觉得害怕。同时他又觉得自己很刚强。他的软弱同时也是他的刚强。不过他怎么能想到,在他取得科学上巨大成就的日子里,在回到莫斯科以后,他会去说这样一番话?

谁也不会知道他和希沙科夫的冲突,但是他觉得,今天同事们对他特别亲热。安娜·斯捷潘诺芙娜抓住他的手,握了握。

“维克托·帕夫洛维奇,我不想对您表示感谢,但我知道,您就是您。”她说。

他一声不响地站在她面前,很激动,而且几乎很幸福。

“妈妈,妈妈,”他忽然在心里说,“你看,你看。”

他在回家的路上打定主意,什么也不对妻子说。可是他还是改不了什么都对妻子说说的习惯。所以在外间里,一面脱大衣,一面就说:

“听我说,柳德米拉,我要离开研究所啦。”

柳德米拉又慌乱,又伤心,但是马上对他说出令他很不愉快的话:

“你那神气,就好像你是罗蒙诺索夫或者门捷列夫似的。你离开了,自会由索科洛夫或者马尔科夫接替你。”

她抬起头来,暂时停止了针线活儿。

“让你的兰杰斯曼上前线去吧。要不然真要让一些有成见的人形成一种看法:犹太人就想把犹太人安排在国防部门的研究所。”

“好啦,好啦,够啦,”他说,“你可记得涅克拉索夫的话:‘不幸的人想的是进光荣的殿堂,结果进的是病房。’我认为我是对得起我吃的粮食的,可是他们却要我检讨错误,检讨异端邪说。哼,真难以设想:检讨错误!这真是岂有此理!明明大家一致推荐我做奖金备选人,大学生们天天请我做报吿。这都是巴季因搞的!不过,哪儿是巴季因?是有人不喜欢我!”

柳德米拉走到他跟前,给他理了理领带,抻了抻上衣下摆,问道:

“你脸色很苍白,大概没吃饭吧?”

“我不想吃。”

“你先就着奶油吃点儿面包,我去把饭热一热。”

然后她往杯子里倒了几滴心脏病药水,说:

“喝吧,我不喜欢你这种模样,让我试试你的脉搏。”

他们朝厨房走去。维克托一面吃面包,一面朝娜佳挂在煤气表旁边的小镜子里看着。

“多么奇怪,难以理解,”他说,“我在喀山何曾想到,我会填这样复杂的履历表,会听今天听到的这种话。好厉害呀!国家与人……有时把人抬得很高,有时毫不费劲儿就把人扔进深渊。”

“维克托,我要和你谈谈娜佳了,”柳德米拉说,“她几乎每天都是过了宵禁时间才回家。”

“前两天你已经对我说过这事儿了。”维克托说。

“我知道我说过了。昨天晚上,我无意中走到窗前,一拉窗帘,却看到娜佳和一名军人走在一起,他们在牛奶铺旁边站下来,接起吻来。”

“噢呀呀。”维克托说着,惊讶得连嚼面包都停止了。

娜佳和军人接吻了。维克托一声不响地呆坐了一会儿,后来就笑起来。也许只有这一条惊人的新闻能使他摆脱沉重的想法,冲淡他的不安心情。有一刹那,他们的目光碰到一起,柳德米拉也不由自主地笑了。此时此刻在他们中间出现了充分的理解,这种理解不需要言语和思考,一生中只能在很少的时间里出现。

所以,柳德米拉听到维克托说的似乎前言不搭后语的话,也就不觉得意外了。他说的是:

“可爱,可爱,不过你说说,我和希沙科夫吵得对吗?”

这思路是很简单的,但要了解就不那么简单了。这里面包括他想到过去的生活,想到托里亚和他的妈妈的遭遇,想到现在在打仗;想到一个人不论得到多大的名和利,等到老了,总是要死的,总有年轻人来接替他,还想到,也许最重要的是一生过得清白。

维克托又向妻子问道:

“你说对吗,应该吗?”

柳德米拉摇了摇头,表示不赞成。几十年融洽、和谐的生活也会产生差异。

“你要知道,柳德米拉,”维克托心平气和地说,“一些实际上很正直的人,往往不会为人处事,爱发脾气,说粗话,不注意方式方法,容易得罪人,在工作上和在家里争吵,都认为是他们不对。可是那些不正直的、爱欺压人的人,却很会待人接物,办事有条有理,沉着镇静,又懂策略,倒往往显得是正派人。”

娜佳在十点多钟回来了。柳德米拉听到钥匙开门的声音,就对丈夫说:

“你和她谈谈吧。”

“你谈比较合适,我不谈吧。”维克托说。

不过等娜佳披散着头发、鼻子红红的走进餐室里,他却说:

“你这是和什么人在大门口接吻?”

娜佳忽然回头看了看,就好像想跑掉。她半张开嘴,望着爸爸。过了一小会儿,她耸耸肩膀,很不在乎地说:

“哦……安得留沙·洛莫夫,他现在在尉官学校。”

“你怎么,打算嫁给他吗?”维克托问道。他听到娜佳那种自信的语调,感到吃惊。他回头看了看妻子,看她是不是看见了娜佳。娜佳像成年人一样眯起眼睛,说出很气愤的话。

“嫁给他吗?”她反问一句。

这话本是维克托问女儿的,可是他一听到又感到十分吃惊。

“可能,要嫁给他!”

过一会儿,她又说:

“也许不会,我还没有最后决定。”

一直没有作声的柳德米拉问道:

“娜佳,你为什么撒谎,又说玛伊卡爸爸送你,又说复习功课?我可是从来没有对自己的妈妈说过谎。”

维克托想起来,追求柳德米拉的时候,有一次她来赴约,说:

“我把托里亚丢给妈妈了,我骗她,说我上图书馆。”

娜佳忽然又恢复了自己的孩子本性,用哭腔和懊恼的腔调叫道:

“在我背后当密探,好吗?你妈妈也在你背后当密探来吗?”

维克托气愤地大声呵斥道:

“混账,你敢顶撞妈妈!”

她带着苦恼而忍耐的神情看着他。

“那好哇,娜佳小姐,就是说,您还没有决定,是嫁给那位年轻上校还是给他做情妇?”

“是的,还没有决定;第二,他不是上校。”娜佳回答说。

难道穿军大衣的小伙子吻的是他的女儿的嘴唇?难道可以和小娜佳,和一个又可笑又聪明的小傻丫头谈恋爱,凝视她的小狗一样的眼睛?但是这是平常而又平常的事。柳德米拉没有作声,她知道,娜佳现在就要生气,不再回答了。她知道,等到只剩下她们两个人,她就要抚摩女儿的头,娜佳就要抽搭起来,不知为什么抽搭,柳德米拉就十分心疼地可怜起她来,也不知为什么要可怜她,因为归根究底,对于一个姑娘来说,和小伙子接吻并不是多么可怕的事情。娜佳也就会把洛莫夫的事一五一十地说给她听,她就会一面抚摩着女儿的头发,回想自己最初接吻的情形,就要想念托里亚,因为生活中不论发生什么事,她都要和托里亚联系起来。托里亚不在了。

这种处在战争深渊边缘上的姑娘的爱情,多么可悲啊。托里亚,托里亚……

可是维克托却怀着做父亲的忧虑心情,还在嚷嚷着。

“那个浑蛋在哪一部分?”他问。“我去找他们的首长谈谈,让他知道,怎么能和不懂事的孩子谈情说爱。”

娜佳不作声。维克托被她的傲慢镇住,不由得也不作声了。过了一会儿,他问:

“你干吗要看着我,就好像高等动物看着一条虫?”

真有些奇怪,娜佳的目光使他想起今天和希沙科夫的谈话。镇定而自信的希沙科夫仗恃着国家和科学院的权力,傲气十足地看着他。在希沙科夫炯炯的目光之下,维克托本能地感觉到所有自己的反抗、最后通牒、发脾气都是徒然的。国家制度的威力像巨石一般耸立着,希沙科夫带着毫不在乎的镇定神气看着维克托在嚷嚷,料定他挪动不了巨石。

而且也很奇怪的是,这会儿站在他面前的小姑娘也意识到,他激动和生气,想做不可能的事,想制止生活的进程,是毫无意思的。

夜里,维克托想到,如果离开研究所,他的日子就很不好过。别人会说他离开研究所带有政治性质,说他已成为不良的反动思想情绪的源泉;而且现在是战争时期,研究所又受到斯大林的特别关注。再说,还有那份可怕的履历表……

还有和希沙科夫那一场很不理智的谈话。还有在喀山说的那些话。还有马季亚罗夫……他忽然觉得非常可怕,很想给希沙科夫写一封和解的信,把今天的一切事情一笔勾销。

五十六

下午,柳德米拉从供应商店回来,看到信箱里有一封信。爬上楼梯后她的心就跳得厉害,这下跳得更厉害了。她手里拿着信,走到托里亚的房间门口,开了门,房间里空荡荡:他今天也没有回来。

柳德米拉看到是她从小就熟悉的妈妈的笔迹,便把信浏览了一遍。她看到叶尼娅的名字、薇拉的名字、斯皮里多诺夫的名字,信里却没有儿子的名字。希望又退到僻静的角落里,但希望没有屈服。

妈妈几乎没有谈到自己生活的情形,只是提到,喀山的房东太太在柳德米拉走后表现出很多令人不快的地方。谢廖沙、薇拉和斯皮里多诺夫还是没有音信。妈妈很担心叶尼娅,看样子,她的生活中发生了很重大的事。叶尼娅在给妈妈的信中暗示有很不愉快的事,暗示她不得不上莫斯科去。

柳德米拉不会忧愁。她只会悲伤。托里亚,托里亚,托里亚。

斯皮里多诺夫成了鳏夫……薇拉成了没有母亲的孤女;谢廖沙活着吗,是不是受了重伤躺在什么地方的军医院里?他的父亲不是被枪毙,便是死在劳改营里了,母亲也死于流放中……妈妈的房子被烧毁了,现在是一个人生活,见不到儿子,也不知道孙儿的下落……

妈妈只字不提她在喀山的生活,没有提到她的身体,也没有提到房间里是否暖和,暖气设备是否改善了。

柳德米拉知道妈妈为什么对这些事缄口不言,是怕她知道了难过。

柳德米拉的房子好像一下子空了,变得冷冰冰的。就好像可怕的无形炸弹落在房子里,把所有的东西都炸坏了,热气跑掉了,只剩下一片瓦砾。

这一天她对维克托想了很多。他们的关系已经坏了。维克托常常对她发火,对她很冷淡,而且特别可悲的是,她对这一切也冷漠了。她太了解他了。从旁人看来,他很像是一个富于理想的和高尚的人。她对人从来没有那种诗意的、热情洋溢的态度,可是玛利亚却把维克托看成具有自我牺牲精神的英雄,一个高尚的人、英明的人。玛利亚喜欢音乐,有时听到弹钢琴,激动得脸都发了白,维克托有时也应她的请求弹弹钢琴。她的天性显然很需要有一个崇拜的对象,于是她为自己塑造了这样一个崇高的形象,为自己臆造出一个实际上不存在的维克托。如果玛利亚天天注意观察维克托的话,她会很快失望的。柳德米拉知道,推动维克托的行动的只是个人主义,他谁也不爱。就是现在,她想到他和希沙科夫的冲突,在为丈夫担心害怕的同时,也感到像往常那样气愤:他为了个人痛快,为了显示自己,为了扮演保护弱者的英雄,连自己的科学、家里人的安宁都可以牺牲。

不过昨天他在为娜佳担心的时候,就忘记了自己的个人主义。可是,维克托能不能忘记自己的一切不愉快的事,为托里亚操操心呢?昨天她估计错了。娜佳没有真正坦率地和她谈谈。这是怎么回事儿?是孩子气,是偶然的,还是她命定的?

娜佳对她说了说一些同伴,她就是在这些同伴的圈子里和那个洛莫夫认识的。她十分详细地说了说一些小伙子,说他们念旧诗,他们议论新艺术和旧艺术,他们对一些事抱的是蔑视和嘲笑的态度,柳德米拉觉得,对那些事是既不能蔑视,也不能嘲笑的。

娜佳很乐意回答柳德米拉的问题,而且看样子说的也都是实话:

“不,我们不喝酒,只喝过一回,那是送一个男孩子上前方。”

“有时谈谈政治。当然啦,不像报纸上那样。不过谈得很少,大概只有一两次。”

但是柳德米拉一问起洛莫夫,娜佳就很生气地回答:

“不,他不写诗。”

“我怎么会知道他的父亲、母亲是什么人,我当然从来也没有看到他们,这有什么奇怪的?他从来不提爸爸,大概他觉得,他是在食品店做生意的。”

这会怎样呢,这是娜佳命中注定的,还是过一个月就会把一切忘得无影无踪?

她在做饭、洗衣服的时候,都在想着妈妈,想着薇拉、叶尼娅、谢廖沙。她给玛利亚打了一个电话,但是没有人接电话,又往波斯托耶夫家里打了一个电话,保姆回答说,女主人出去买东西去了,又往房管所打了一个电话,想找一个修理工来修水龙头,房管所的人回答说,修理工没有来上班。

她坐下来写信。似乎她要写很长的一封信,检讨她不能为妈妈创造必要的生活条件,所以妈妈宁愿一个人住在喀山。从战前起,柳德米拉的亲戚们就不来探望和过夜了。现在就连最亲近的人也不到她在莫斯科的这套大房子里来了。信她也没有写成,只是撕了四张纸。

这一天快下班的时候,维克托打来电话,说他一时不能回来,晚上有些技术人员要来,是他从军工厂请来的。

“有什么新闻吗?”柳德米拉问道。

“噢,在这方面的新闻吗?”他说。“没有,没有什么新闻。”

晚上,柳德米拉又把妈妈的信看了一遍,走到窗前。

月色皎洁,大街上空空荡荡。她又看到娜佳挽着那个军人的胳膊,他们顺着马路朝家里走着。后来娜佳跑起来,穿军大衣的小伙子却站在空荡荡的街心里,望着,望着。柳德米拉这时在心里好像把一切似乎不能结合的东西结合到一起。这里面有她对维克托的爱、她为他分担的焦虑、她对他的愤恨。还有没有吻过姑娘的香唇就离开了人世的托里亚,还有站在马路上的尉官,还有,瞧,薇拉正喜气洋洋地走上自己斯大林格勒住宅的楼梯呢,还有无家可归的妈妈……

她心中充满活着的感觉,活着曾经是她唯一的欢乐和唯一可怕的痛苦。

五十七

维克托在研究所大门口碰到希沙科夫。希沙科夫正从汽车下来。

希沙科夫掀了掀帽子打招呼,没有表示要站下来和维克托说说话儿。

“我要倒霉了。”维克托在心里说。

斯维琴在吃午饭的时候,虽然坐在旁边的桌上,却不看他,也不和他说话。胖子古列维奇在走出食堂的时候和维克托说话,今天口气特别亲热,握住他的手握了很久,但是等所长接待室的门开了一道缝儿,古列维奇便突然和他分手,很快地顺着走廊走去。

在实验室里,正在和维克托商谈如何准备仪器进行核粒子摄影的马尔科夫从记录本上抬起头来,说:

“维克托·帕夫洛维奇,有人告诉我,党委会上很不客气地谈到您。科甫琴科给您罗织罪名,说:‘施特鲁姆不愿意在我们这个集体里工作。’”

“他说就说吧。”维克托说。他觉得自己的眼皮跳了起来。在和马尔科夫谈核粒子摄影的时候,维克托产生了一种感觉:似乎主持实验室工作的已经不是他,而是马尔科夫了。马尔科夫说话已经用的是十分从容的当家人口气,诺兹德林两次走到他面前,向他请示有关仪器安装的问题。但是马尔科夫忽然露出有苦衷和恳求的脸色,他小声对维克托说:

“维克托·帕夫洛维奇,如果您谈起这次党委会,千万不要说是我说的,要不然我就倒霉了:泄露党的秘密。”

“当然,您放心。”维克托说。

马尔科夫说:

“一切都会解决的。”

“唉,”维克托说,“没有我也行啊。不论花费多少心血,都是白费劲儿!”

“我觉得,您说得不对,”马尔科夫说,“我昨天和科奇库罗夫谈过,您该知道,他是一个讲求实际的人。他对我说:‘在施特鲁姆的论文中,数学多于物理,不过,说也奇怪,这使我开了窍,我自己也不知道为什么。’”

维克托明白马尔科夫暗示的是什么:年轻的科奇库罗夫很热心地在研究慢中子作用于重原子核的有关问题,他强调,这些研究将有很大的实用意义。

“科奇库罗夫这样的人一点也不起作用,”维克托说,“起作用的是巴季因之流。可是巴季因认为我应当检讨,承认我把物理学家们引向学究式抽象概念的泥坑。”

显然,实验室里的人都已经知道维克托和领导人的冲突和昨天的党委会议。安娜·斯捷潘诺芙娜用难受的目光看着维克托。

维克托希望和索科洛夫谈谈,但是索科洛夫早晨就上科学院去了,后来打来电话,说有事要耽搁,不一定到研究所来了。

萨沃斯季扬诺夫的情绪却特别好,不住地在说俏皮话。

“维克托·帕夫洛维奇,”他说,“可敬的古列维奇真是一位又闪光又突出的学者。”他在说这话的时候用手摸了摸头和肚子,暗示古列维奇秃头和大肚子。

傍晚,维克托在步行回家的路上,无意中在卡卢加街上碰到玛利亚。她首先唤他。她穿着维克托以前没有见过的一件大衣,所以他一下子没有认出她来。

“太好了,”他说,“您怎么到卡卢加街上来啦?”

她看着他,沉默了一小会儿。后来她摇了摇头,说:

“这不是偶然的,我想见见您,所以我到卡卢加街上来了。”

他很不好意思,轻轻地把两手一摊。他的心慌乱了一小会儿,他以为,她要向他报告很可怕的事情,警告他有危险。

“维克托·帕夫洛维奇,”她说,“我想和您谈谈。我丈夫把情况全对我说了。”

“噢,把我的了不起的成就全说了。”维克托说。他们并排朝前走去,不过走着的似乎是两个互不相识的人。她不说话,他感到气氛很沉重。他侧眼看了看她,说:

“柳德米拉为这事儿骂我呢。您大概也想生我的气了。”

“不,我不生气,”她说,“我知道,是什么迫使您这样做的。”

他很快地看了她一眼。她说:

“您想着您的妈妈。”

他点了点头。然后她说:

“我丈夫不愿意告诉您……他听说,行政领导和党组织结成一伙儿反对您,他听到巴季因说:‘这不是一般的歇斯底里。这是政治上反苏的歇斯底里。’”

“我这算什么歇斯底里?”维克托说。“我就感觉到,你丈夫不愿意把他知道的情况告诉我。”

“是的,他不愿意。我也替他难受。”

“他害怕吗?”

“是的,他害怕。此外,他认为,您原则上是不对的。”

她小声说:

“他是一个好人,他受的折腾太多了。”

“是啊,是啊,”维克托说,“这也叫人痛心:如此高大而勇敢的科学家,如此胆小的心灵。”

“他受的折腾太多了。”她又说了一遍。

“不过,”维克托说,“不应该是您,应该是他把这一切告诉我。”

他挽住她的胳膊。

“玛利亚,”他说,“您告诉我,马季亚罗夫在那儿怎么样?我怎么也弄不清,究竟是怎么一回事儿。”

他现在一想到在喀山说的那些话,就感到提心吊胆,常常想起一些个别的字句,想起卡里莫夫不怀好意的警告,同时也想起马季亚罗夫的猜疑。他觉得,悬在他头顶上的莫斯科阴云不可避免地要和喀山的闲谈联系起来。

“我也不清楚是怎么一回事儿,”她说,“我们寄给马季亚罗夫的挂号信,退回来了。他是换了地址呢,还是离开了?还是出了顶坏的事?”

“是啊,是啊,是啊。”维克托嘟哝说。一时间他不知说什么才好。

玛利亚显然以为索科洛夫对维克托说过那封寄出去又退回来的信。可是维克托根本不知道那封信,显然索科洛夫没有对他说。维克托问她,究竟是怎么一回事儿,指的是马季亚罗夫和索科洛夫的争吵。

“咱们上逍遥公园去。”他说。

“不过咱们走的不是那个方向。”

“卡卢加街这边也有一个门。”他说。

他想更详细地向她问问马季亚罗夫的情况,问问他对卡里莫夫怀疑的一些问题和卡里莫夫所怀疑的问题。在空旷的逍遥公园里没有人打搅他们。玛利亚会马上了解这次谈话的重要性。他觉得,他可以放心地、随便地和她谈谈他所担心的一切问题,她有什么话都会对他说的。

昨天开始化冻了。在逍遥公园的山坡上,有些地方的雪已经化了,露出潮湿的烂树叶,但是一些小沟里的雪还很厚。头顶上是布满薄云的灰色的天空。

“这样的黄昏多么好啊。”维克托一面说,一面吸着潮湿而寒冷的空气。

“是的,很好,一个人也没有,就¥½像在郊外。”

他们在泥泞的小路上走着。遇到水洼儿,他就搀着玛利亚的手,帮她跨过去。

他们一声不响地走了很久,他不想开口说话了,既不想谈战争,也不想谈研究所里的事情,也不想谈马季亚罗夫和他的担心、他的预感和疑虑,他想一声不响地和这个娇小的、走路不敏捷却又轻盈的女人走走,想享受一下不知为什么忽然来临的无限轻松与安宁感。

她也什么也不说,微微低着头,走着。他们走到河岸上,河里依然是黑沉沉的冰。

“太好了。”维克托说。

“是的,太好啦。”她说。

岸边的沥青小路是干的,他们走得快了,就好像两个走远路的行人。他们遇到一位受伤的尉官和一位穿滑雪衫的矮个子、宽肩膀姑娘。他们互相搂抱着走着,不时地接吻。他们来到维克托和玛利亚跟前,又接了一个吻,回头看了看,笑了起来。

“哦,也许娜佳和她的尉官常常这样在这里走来走去。”维克托想道。

玛利亚回头看了看那对青年男女,说:

“多么糟糕。”

她笑了笑,又说:

“柳德米拉对我说过娜佳的事。”

“是呀,是呀,”维克托说,“这真是太出奇了。”

过了一会儿,他说:

“我决定给机电研究所所长打个电话,自我推荐。如果他们不接受,那我就上新西伯利亚或者克拉斯诺亚尔斯克去。”

“有什么办法呀,”她说,“看样子,就得这样。不这样不行。”

“多么糟糕呀。”他说。

他很想对她说说,他对研究、对研究所的爱有多么强烈,他看着很快就要试用的设备,又高兴又伤心,他觉得,他会在夜里上研究所去,隔着窗子看的。他想,也许玛利亚会感到他的话有自我显示的意味,所以就没有说。

他们走到战利品展览馆跟前。放慢脚步,观看漆成灰色的德国坦克、大炮、迫击炮和翅膀带有黑色卐字的飞机。

“就是看着这些不响也不动的东西,都觉得害怕。”玛利亚说。

“没什么,”维克托说,“应当想想,在将来的战争中这些东西会变得像火枪和长矛一样不管用,也就不害怕了。”

他们快要走到公园大门口,维克托说:

“咱们这次溜达到头了,逍遥公园这样小,真遗憾。您不累吧?”

“不累,不累,”她说,“我已经习惯了,步行走路太多了。”

不知是她没有明白他的话的用意,还是装作没有明白。

“您知道,”他说,“不知为什么我和您见面总要靠您和柳德米拉见面或者我和您丈夫见面。”

“是的,是的,”她说,“不这样又怎样呢?”

他们走出公园。城市的闹声包围了他们,破坏了静静地散步时美好的心境。他们走上离他们相遇的地方不远的一个广场。她像个小姑娘望着大人一样,从下面朝上望着他,说:

“您现在可能对自己的研究、对实验室、对仪器感到特别热爱。不过您不可能有别的做法,别人可能,您不可能。我把很坏的情况对您说了,不过我以为,知道真实情况总要好些。”

“谢谢您,玛利亚,”维克托握着她的手,说,“我感谢的不光是这一点。”

他觉得她的手指头在他的手里哆嗦了几下。

“真奇怪,”她说,“咱们分手差不多都是在咱们会面的地方。”

他用开玩笑的口气说:

“难怪古人说:始终如一。”

她皱起眉头,显然是在思索他的话,后来笑起来,说:

“我不懂。”

维克托望着她的背影:是一个不高的、瘦小的女子,像这样的女子,迎面相遇的男子是从来不会回头看的。

五十八

达林斯基过去很少像这次来加尔梅克草原上出差一样,一连几星期过这种苦闷的日子。他给方面军领导人打了一个电报,说在安然无事的左翼边区再待下去没有必要,说他的任务已经完成了。但是方面军领导却表现出达林斯基无法理解的一股固执劲儿,就是不把他召回。

最轻松的是工作时间,最难捱的是休息时间。

周围都是松散、干燥、窸窣作响的沙子。当然这里也有生物:蝎虎和乌龟在沙里沙沙地爬着,尾巴在沙上划出一道道印子,有的地方生长着脆弱的、和沙一样颜色的刺草,老鹰在空中盘旋着,寻找动物的尸体和扔掉的食物,蜘蛛用老长的腿奔跑着。

自然条件的贫乏,十一月的无雪沙漠的寒冷与单调,似乎把人掏空了,不仅人的生活,就连人的思想也贫乏、单调和苦闷了。

达林斯基渐渐屈服于这种沉闷的沙漠的单调。他一向对吃东西很淡漠,可是在这里他老是想着吃饭。第一道菜是用大麦粉和渍番茄做的酸羹,第二道菜是大麦米饭,他一见到这样的饭就头痛。他坐在幽暗的板棚里,面对着洒满一摊摊菜汤的木板桌子,看着人们端着浅浅的洋铁钵子喝汤,就感到难受,想快点儿离开食堂,别听羹匙的叮当声,别闻令人恶心的气味。但是一走出来,食堂又恢复了吸引力,他又想着食堂,数算着到明天吃午饭还有多少时间。

夜里小屋很冷,达林斯基睡不好:脊背、耳朵、脚、手指头都冻得难受,脸颊冻得发木。他睡觉总是不脱衣服,脚上裹两副裹脚布,头用毛巾包起来。

起初他感到奇怪,他在这儿接触到的人似乎想的不是战争,他们的头脑里塞满了吃的问题、抽烟问题、洗衣服问题。但是没过多久,达林斯基在和营长、连长们谈大炮怎样过冬、谈锭子油、谈弹药供应问题的时候,就发现自己头脑里也充满了生活方面的各种各样操心的事、希望和苦恼。

方面军司令部好像远在天外,他只能幻想小一点儿的:到埃利斯塔附近的集团军司令部去住一两天。他想上集团军司令部,不是盼望和蓝眼睛的阿拉·谢尔盖耶芙娜会面,而是思念着洗洗澡,洗洗衣服,吃一碗菜汤白面条。

现在他觉得在鲍瓦那儿过夜都是愉快的了,住在鲍瓦的小屋里实在不坏。而且和鲍瓦谈的不是洗衣服,也不是菜汤。

特别使他受不了的是虱子。

他很长时间不明白为什么身上常常发痒,有时正谈着公事,他忽然拼命在腋下或大腿上抓起痒来,却还不明白谈话对方的会心的笑。他一天一天地痒得越来越厉害。锁骨旁边和腋下发痒已经成了习惯。他以为是害皮疹,认为害皮疹是因为皮肤太干燥了,是尘土和沙子刺激的。有时痒得难受,他在路上走着,忽然站下来,又搔大腿,又搔肚子,又搔屁股。夜里身上痒得特别厉害。达林斯基一醒过来就拼命拿手指甲挠胸前的皮肤,挠上很久。有一次他仰面躺着,把腿跷起挠腿,又一面呻吟着挠腿肚子。越热皮肤越痒,他发现了这一点。一到被窝里浑身就痒得受不了。有时在夜里他到寒冷的空气里,就不怎么痒了。他想上医务所去,要一点治皮癣的药膏。

有一天早晨,他扯了扯衬衣领口,看到领子缝儿里有一些懒洋洋、肥嘟嘟的虱子。虱子非常多。达林斯基又害怕又不好意思地回头看了看睡在他旁边的大尉,大尉已经醒来,坐在床上,脸上带着发狠的表情在敞开的长衬裤上挤虱子。嘴里还不出声地嘟哝着,显然是在进行战斗统计。

达林斯基脱下衬衣,也干起同样的事。这儿的早晨静悄悄,雾蒙蒙。听不见枪炮声,也没有飞机隆隆声,大概正因为这样,在两位军官手指甲下面阵亡的虱子的咯吧声特别清脆。大尉瞥了达林斯基一眼,说:

“嗬,好家伙,像狗熊!不,应该说,像母猪!”

达林斯基一面在衬衣领子上搜索着,说:

“难道不发药粉吗?”

“发是发,”大尉说,“可是有什么用?需要洗澡,可是喝的水都不够。食堂里为了节省水,锅碗几乎都不洗。哪儿有水洗澡?”

“有没有灭虱汽锅?”

“算了吧。只是把衣服熏一熏,熏得虱子红一阵子。唉,我们驻扎在奔萨做后备队的时候,那日子才快活呢!我都没有上过食堂。女房东给我做吃的,而且不是老太婆,是水灵灵的娘们儿。每星期洗两次澡,天天有啤酒喝。”

“怎么办呀?”达林斯基问道。“这儿离奔萨还远。”

大尉一本正经地看了看他,用信任的口气说:

“中校同志,有一个好办法。用鼻烟!把砖碾碎了,和鼻烟掺和在一起。撒到衬衣上。虱子就要打喷嚏,难受得团团转,撞到砖上把头撞碎。”

他是一本正经的,所以达林斯基一下子没有明白他是在进行口头创作。几天之后,达林斯基便听到十来个这种题材的故事。口头创作是很丰富的。

现在他的脑子日日夜夜思索着许多问题:吃饭、洗衣服、换衣服、药粉,用瓶子装开水把虱子烫死,把虱子冻死,把虱子烧死。他连女人也不想了,他想起了他在劳改营里听刑事犯人说的俗语:“有劲儿活,就没劲儿想老婆。”

五十九

整整一天达林斯基都是在炮兵营阵地上度过的。一天中,没听到一声炮响,没有一架飞机在空中出现。营长是一个年轻的哈萨克人。他用纯正的俄语说:

“我想,明年可以在这儿种瓜了。您来吃瓜好啦。”

这位营长觉得在这儿并不坏,他一天到晚露着白牙说笑,用弯弯的短腿在很深的沙子里轻快地来来回回走着,亲热地看着站在油毡小屋旁边的上了套的骆驼。

可是达林斯基看到年轻哈萨克人的快活劲儿,很生气。他希望孤独,所以到傍晚时候,他朝第一连阵地走去,虽然下午他已经去过了。

月亮升上来,老大老大的,黑色多于红色。月亮在黑色而透明的天空里慢慢往上爬升,因为使劲,它的脸涨得越来越红。在带怒气的月光中,夜晚的沙漠、长筒子大炮、反坦克枪和火箭炮显得十分特别,十分惊慌,十分小心。大路上有一队骆驼拉的大车,车上装的是弹药箱和干草。一切无法连接的东西似乎连接起来了:牵引拖拉机,载有部队报纸印刷设备的汽车,无线电台细细的天线,长长的骆驼脖子,还有骆驼从容不迫的波浪式步子,就好像骆驼浑身没有一根硬骨头,全是用橡胶浇成的。

骆驼走过去了,寒冷的空气中留下一股农村的干草气息,当年伊戈尔公爵的大军作战的空旷田野上空,也出现过这样黑色多于红色的老大的月亮。当年波斯人进军希腊,罗马军队进入德意志森林,首席执政官的部队夜晚到达金字塔脚下的时候,天空悬挂的也是这个月亮。

当人们想到过去的时候,总是通过稀稀的筛子筛选出一件件历史大事,把士兵的痛苦、磨难和不幸全部筛掉。在头脑里只剩下空洞的故事,得胜的军队怎样部署,失败的军队怎样部署,参加战斗的有多少战车、石弩、骆驼,或者多少坦克、大炮、飞机。头脑里留下的印象,是英明而幸运的统帅怎样牵制中心,突击侧翼,山冈后面的伏军怎样突然冲出来决定了战斗的结局。再就是很平常的故事:得胜的统帅班师回朝后,被怀疑有意推翻君主,结果因为拯救祖国而献出头颅,或者幸免一死,被流放。

这儿真是艺术家创作的一幅激战之后的图画:一轮朦胧的老大的月亮悬挂在战场上空,身穿锁子甲的英雄们张开手臂睡着,旁边是打坏的战车或者坦克,有些胜利的英雄们抱着冲锋枪,坐在摇摇晃晃的帆布帐篷里,有的头戴古罗马的铜鹰头盔,有的头戴近卫军皮帽。

达林斯基无精打采地坐在炮兵连阵地上的一个弹药箱子上,听两名盖了大衣躺在大炮旁边的战士说话。连长和指导员上营部去了,从方面军司令部来的这位中校似乎也睡熟了。战士们是从通信员嘴里了解他的身份的。两个战士悠然自得地抽着自己卷的烟卷儿,吐着烟圈儿。

这显然是两个好朋友,他们都有真正的朋友才会有的感情,他们相信,一个人生活中发生的每一桩微不足道的小事,对于另一个人往往是很重要的,是值得关心的。

“怎么啦?”其中一个似乎用嘲笑和漠不关心的口气问。

“怎么啦,怎么啦,难道你不知道他的情形?他的脚疼,不能穿这种鞋。”

“那又怎么啦?”

“可是他只能穿鞋子呀,又不能光着脚。”

“噢,就是说,没有发给他靴子。”他的口气中再也没有嘲笑和漠不关心的意味了,他显得对这件事十分关心。然后他们谈起家里的事。

“你猜我老婆写些什么?这也没有,那也没有,不是儿子生病,就是女儿生病,老娘们儿,就是这样。”

“可是我老婆写得更干脆:你们在前方有什么难的,你们有给养,可是我们在这儿过这种战时困难日子,简直活不下去了。”

“都是女人见识,”一个说,“她们躲在大后方,不了解前方是什么样子。她们光看到你的给养。”

“一点儿不错,”另一个说,“她们有时买不到煤油,就以为这是天大的事了。”

“是的,她们有时站站队,似乎比在这沙漠上拿燃烧瓶打坦克都困难。”

他竟说起坦克和燃烧瓶来,其实他和他的朋友都知道,德国人的坦克从来没上这儿来过。在生活中是男人更艰苦还是女人更艰苦这个永远谈不完的话题,也发生在战时这夜晚的沙漠上。

不过还没有得出结论,其中一个就很不果断地说:

“不过,我老婆是有病,她的脊椎骨有毛病,抬一下重东西,就要躺几个星期。”

接着,似乎又换了话题,他们谈起这周围是一块多么可恨的缺水的地方。那个离达林斯基比较近些的战士说:

“她这样写,也没有不好的意思,只是因为不了解。”

另一名战士补充了一下,否认自己有意说军人妻子们的坏话,同时又不否认:

“是的。我这是说气话。”

然后他们又抽了一会儿烟,沉默了一会儿,又说起保险刀片多么不保险,说起连长的新制服,又说起不论多么艰难困苦,还是想活下去。

“你瞧,这夜晚多么好,你要知道,我在上中学的时候,看到这样一幅画:当空一轮明月,战场上到处是战死的英雄。”

“这有什么相同之处?”另一名战士笑道。“那是英雄,咱们算什么,和麻雀一样,咱们干的是蠢事。”

六 十

达林斯基右方响起爆炸声,打破夜的寂静。

“一〇三毫米。”老练的耳朵判断说。脑子里闪过一些念头,那是在敌人的炮弹爆炸时常常出现的:“是不是偶然的?唯一的?是试射?会不会采取交叉射击?是不是进行炮轰?是不是坦克来了?”

一切久经战阵的人都在倾听,脑子里都出现了和达林斯基大致相同的念头。

一切久经战阵的人都能从上百种声音中分辨出一种真正使人担心的声音。一个老练的战士,不论他正在干什么,不论是手里正拿着调羹,或者正在擦枪,在写信,在用手指头抠鼻子,在看报,或者完全无思无虑(一个当兵的在空闲时候有时也会这样),会立刻转过头去,竖起专注而灵敏的耳朵。

这一次马上得到了答案。右边接二连三传来爆炸声,接着左边也传来爆炸声,周围轰隆隆,卡啦啦,硝烟弥漫,一切都震动起来。

这是炮轰!

透过硝烟、灰土和沙子可以看到爆炸的火光,在爆炸的火光中可以看到硝烟。

人们在奔跑,在卧倒。

沙漠上一片凄惨的叫声。炮弹开始在骆驼旁边爆炸,骆驼把大车弄翻,拖着扯断的套绳奔跑着。达林斯基不顾炮弹纷纷在爆炸,站起身来,注视着可怕的景象。

他的脑子里清清楚楚地闪过一个念头:他在这儿看到的是祖国的末日景象。他心中充满了不祥的感觉。这沙漠中疯狂奔跑的骆驼的可怖的叫声,这俄罗斯人的惊骇的喊声,这纷纷奔跑躲避的人们!俄罗斯完了!被赶到靠近亚洲的寒冷的沙漠上的俄罗斯,就要完了,就要死在昏沉而静谧的月光下,亲切而悦耳的俄罗斯语言已经和狂奔的、被德国炮弹炸伤的骆驼的恐怖与绝望的惨叫声合成了一片。

在这痛苦的时刻,他心中出现的不是愤怒,不是仇恨,而是对世上所有的弱者和穷人的兄弟情感;他在草原上遇到的那个加尔梅克人的黑糊糊的苍老的脸,此时此刻不知为什么浮现出来,而且他觉得格外亲切,似乎早就熟识了。

“有什么办法呢,这是注定了的。”他在心里说。他也明白了,如果失败了,他也没有必要活在世上了。他环视了一下躲在掩壕里的士兵们,挺直了身子,准备在这场凄惨的战斗中担负起这支炮兵连的指挥任务,他叫道:“喂,电话员,过来!到我这儿来!”

可是爆炸声忽然停息了。

就在这天夜里,遵照斯大林的指示,三方面军的司令员瓦图京、罗科索夫斯基和叶廖缅科向所属部队发布了进攻的命令,正是这次进攻在一百个小时中解决了斯大林格勒战役的命运和保卢斯的三十三万大军的命运,成为整个战争进程的转折点。

集团军司令部有一通电报在等待着达林斯基:要他去诺维科夫上校的坦克军里去,负责向方面军司令部报告坦克军的战斗行动。

六十一

在十月革命节过后不久,德国空军又对斯大林格勒发电站进行了密集轰炸。十八架轰炸机向发电站投下大批重型炸弹。

一片瓦砾的发电站笼罩着一团团的硝烟,德国空军的毁灭性力量使发电站的工作完全停止了。

在这次轰炸之后,斯皮里多诺夫的手剧烈地哆嗦起来。他端起茶杯喝茶,常常把茶泼洒出来,有时觉得哆嗦的手指头端不住茶杯,只好把茶杯放回桌子上。只有在喝过酒之后,手指头才停止哆嗦。

领导者开始放工人走了,于是工人们便搭过河的船只渡过伏尔加河和图马克河,进入草原,去阿赫图巴中游地区和列宁斯克。

发电站领导人曾经向莫斯科询问过,要求允许撤离,因为车间已经炸毁,他们留在前线已失去意义。莫斯科方面迟迟不作回答,斯皮里多诺夫非常着急。在轰炸之后,党中央马上通知召见党委书记尼古拉耶夫,尼古拉耶夫便乘飞机上莫斯科去了。

斯皮里多诺夫和卡梅绍夫在发电站的瓦砾堆中走来走去,互相劝说着:他们在这儿无事可做,应该离开。可是莫斯科一直没有回话。

斯皮里多诺夫很为薇拉担心。她渡过伏尔加到左岸以后,感到身体很不好,不能上列宁斯克去了。要乘载货汽车在炸坏的路上走一百公里,汽车在冻得像石头一样的土块丛中走,颠得很厉害,一个快到分娩时候的孕妇是受不了的。

几位熟识的工人把她搀到岸边一条驳船上,这条船已经冻在冰上,变成了宿舍。

在发电站第二次被轰炸之后不久,薇拉请快艇上的一位技师给爸爸送来一封信。她叫爸爸放心:在舱里给她让出一块地方,是一个很舒服的角落,还有布幔遮着。在疏散的人中间有别克托夫门诊所的一名护士和一位年老的助产士;离驳船四公里有一所野战医院,如有什么复杂情况,随时可以把医生请来。驳船上有开水炉子,有炉灶,做饭大家一齐动手,粮食由州党委供应。

虽然薇拉要爸爸放心,可是信上的每一句话都引起他的担心。也许,只有一点使他得到安慰,就是薇拉写的:自从打仗以来,这条驳船一次也没有遭到轰炸。如果他能到左岸去,他一定能弄到一部小汽车或者救护车,至少把薇拉送到阿赫图巴中游地方去。

可是莫斯科还是没有回话,没有叫站长和总工程师撤离,虽然现在被炸毁的发电站只需要一小队军事化的保卫人员就够了。工人和技术人员们不乐意没有事在发电站闲待着,一得到站长允许,马上就朝渡口走去。

只有安德列耶夫老头子不愿意到站长这儿来拿盖有圆图章的正式证明信。在轰炸之后,斯皮里多诺夫就劝安德列耶夫上列宁斯克去,他的儿媳妇和孙子就住在那儿,可是安德列耶夫说:

“不去,我要留在这儿。”

他觉得,他在斯大林格勒的河岸边,可以和过去的生活保持联系。也许,再过一段时间他就可以回到拖拉机厂工人村去了。他可以在毁于炮火的房屋中间走走,到他老伴侍弄的小园子里去,把倒下的小树扶起来,支起来,看看埋起来的东西是否还在,然后在歪倒的栅栏旁边的石头上坐一坐。

“瞧,瓦尔瓦拉,缝纫机还在,而且还没有生锈呢,栅栏旁边的苹果树全完啦,是炮弹炸坏的,在地窖桶里的酸白菜只是上面开始发霉。”

斯皮里多诺夫本来想和克雷莫夫谈谈自己的事情,但是十月革命节以后克雷莫夫再也没有上发电站来。

斯皮里多诺夫和卡梅绍夫决定等到十一月十七日,到那时就走,因为在发电站的确无事可干。德军却还在不时地炮轰发电站。在密集轰炸之后十分焦急的卡梅绍夫说:

“斯皮里多诺夫同志,他们既然不停地在轰,可见他们的侦察队一点儿也不顶用。他们的空军随时都可能再来轰炸。要知道德国人执拗得像老牛一样,会照准了一块空地方一个劲儿地猛轰。”

十一月十八日,斯皮里多诺夫和保卫人员告过别,吻了吻安德列耶夫老头子,最后扫视了一遍发电站的瓦砾堆,便离开了斯大林格勒发电站。他一直没有等到莫斯科方面的正式准许。

斯大林格勒战役期间他在发电站干了很多事情,干得很认真,很艰苦。他害怕打仗,很不习惯战争环境,一想到空袭就胆怯,在轰炸时吓得直发呆,然而他还在工作,因此他的工作就尤其艰苦,尤其可贵。

他提着箱子,背着包袱,一面走,一面回头望着,向站在炸毁的大门口的安德列耶夫挥着手,望着已经没有了玻璃的工程技术大楼,望着涡轮车间的凄凉的断墙,望着依然在燃烧的储油室上空的轻烟。

他离开发电站的时候,发电站已经不需要他了,他是在苏军开始进攻的前一天离开的。

但就是他没有捱过去的这一天,却在很多人的眼睛里把他的勤恳、艰苦的工作一笔勾销;有些人本来准备把他称作英雄的,现在却管他叫胆小鬼和逃兵了。

他心中很久都保留着十分痛苦的感情,常常想起,他是怎样一面走,一面回头看,一面挥手,而孤单的老头子怎样站在电站大门口望着他。

六十二

薇拉生了一个儿子。

她躺在驳船舱里,在一张用粗糙的木板钉成的床上。几个女人为了让她暖和,把不少破旧衣服堆到他身上,和她躺在一起的是裹在小被子里的婴儿。要是有人进来,掀开帷幔,她便看到许多人,男人和女人,从上面床铺上垂挂下来的破烂儿。她听到乱哄哄的说话声、孩子的哭叫声和闹腾声。她的头脑里模模糊糊的,烟气腾腾的空气也模模糊糊的。

舱里很闷,同时又很冷,板壁上有的地方结了霜花。人们夜里睡觉不脱毡靴和棉衣。妇女们整天裹着头巾和破被子,不住地呵冻僵的手指头。

小小的窗户几乎挨到冰面,光线勉强可以透进来,所以大白天在舱里都是幽暗的。到晚上就点起油灯。人们的脸被烟子熏得黑糊糊的。舷梯旁的舱门一打开,一团团的热气就冲进舱来,很像爆炸的炮弹的硝烟。

头发蓬乱的老妇人挠着白发和灰发,老头子们坐在地上端着杯子在喝开水,裹着头巾的孩子在各色各样的枕头、包袱、箱子上爬着玩儿。薇拉因为有孩子躺在胸前,觉得她的想法变了,她对一切人的态度变了,身体也变了。

她想到自己的好朋友季娜·麦尔尼科娃,想到照料过她的老奶奶谢尔盖耶芙娜,想到春天,想到妈妈,想到破了的衬衣,想到棉被,想到谢廖沙和托里亚,想到肥皂,想到德国人的飞机,想到斯大林格勒发电站的掩蔽所,想到自己的头发很久没有洗,而她所想到的一切,都充满了对她所生的孩子的感情,都和孩子有关系,其意义的大小都是由和孩子的关系而定。

她看着自己的手、脚、胸膛、手指头。这已经不是那双打排球、写文章、翻书的手。这已经不是那双在学校楼梯上跑上跑下、在暖和的河水里蹦来蹦去、被荨麻扎得痒痒的腿了,也不是街上行人回头看她时看到的那双腿了。

她想着孩子,同时也想着维克托罗夫。飞机场在伏尔加左岸,维克托罗夫就在附近,伏尔加河再也不能把他们分开了。马上就会有飞行员们到舱里来,她就问:“你们认识维克托罗夫上尉吗?”飞行员们会说:“我们认识。”“请你们告诉他,他的儿子和妻子在这儿。”

有些妇女到帷幔后面来看她,摇摇头,又笑,又叹气,有的俯身向着婴儿,哭了起来。

她们为自己哭,为婴儿笑,要懂得她们的心情,是不需要什么话的。

如果有人向薇拉问什么话,那么问话也无非是产妇怎样才能喂好婴儿:乳房是不是有奶水,有没有乳腺炎,潮湿空气是不是使她感到气闷。

产后第三天,父亲来到她身边。他已经不像斯大林格勒发电站的站长:提着箱子,背着包袱,胡子拉碴的,竖着大衣领子,系着领带,鼻子和两腮被冷风吹得通红。

父亲来到她的床前,她看到父亲那打颤的脸最初一会儿不是对着她,而是对着躺在她旁边的小东西。

他背过身去。她从他的肩膀和脊背看出来,他是在哭。她明白,他哭的是妈妈再也不会知道这个外孙,不能像他刚才那样看看外孙了。

过了一会儿,他对自己流泪又生气,又感到不好意思,因为几十个人看见了,他用冻哑了的声音说:

“好啊,因为你,我做外公啦。”

他俯下身去,吻了吻薇拉的额头,又用冰冷的脏手抚摩了几下她的肩膀。然后他又说:

“十月革命节那天,克雷莫夫上发电站来过。他还不知道你妈妈已经不在了。他一个劲儿问叶尼娅的情况。”

一个胡子拉碴的老头子穿一件女式棉袄,露着一团一团的烂棉花,他吃力地喘着气说:

“斯皮里多诺夫同志,现在又是颁发库图佐夫勋章,又是颁发列宁勋章和什么英雄勋章,为的是多杀一些人。我们的人和他们的人杀了多少啦!倒是真应该颁发这么大的勋章,两公斤重的,给您的女儿,因为她在这样艰苦的条件下带来了新生命。”

这是在薇拉生过孩子之后谈起她的第一个人。

斯皮里多诺夫决定留在驳船上,等到薇拉身子硬朗了,和她一起上列宁斯克去。他要上古比雪夫去接受新的任务,上列宁斯克是顺路。他看到驳船上的伙食太差,应当马上为女儿和外孙想想办法,所以等身上暖和过来之后,便前去找州党委的指挥所,州党委指挥所就在附近,在森林中的什么地方。他指望到那儿通过朋友弄一些猪油和糖来。

六十三

这一天在舱里特别难受。伏尔加上空笼罩着乌云。肮脏的冰上到处是垃圾和黑糊糊的泔水,没有孩子在上面玩,妇女们也不在冰窟窿里洗衣服,下游来的冷风撕扯着冻在冰上的破布,又从舱门的缝儿钻进舱里,使整个驳船到处是呼啸声和咯吱声。

人们呆呆地坐着,裹着头巾、棉衣、棉被。最喜欢唠叨的娘们儿也不说话了,倾听着风的吼声、木板的咯吱声。

天色渐渐黑了。这黑暗似乎来自人们难以忍受的痛苦,来自可怕的寒冷、饥饿、肮脏,来自没完没了的战争的折磨。

薇拉躺着,把棉袄一直拉到下巴底下,每一阵风钻进舱里,她都感觉到寒气在面颊上拂过。

此时此刻,她对一切都很悲观:父亲也不能把她送走了,战争永远不会结束,到春天德国人就会侵入乌拉尔,侵入西伯利亚,他们的飞机会永远在天空尖叫,永远有炸弹爆炸声。

她第一次怀疑维克托罗夫离她很近。战场是很多的。也许,不论战场,不论后方,都已经找不到他了。

她掀开小被子的一角,凝视着孩子的脸。他为什么哭呀?也许是她的苦恼传给了他,就像她把温暖和奶水给了他一样。

这一天,严冬的酷寒、凛冽的冷风、遍布辽阔平原与大河上的大规模战争让人们心情沉重。

难道一个人能长期忍受这样饥寒交迫的可怕日子?

为薇拉接生的老奶奶谢尔盖耶芙娜走到她床前,说:

“我看你今天的样子很不好,还不如第一天。”

“没什么,”薇拉说,“爸爸明天就要回来,会给我带吃的东西来。”

尽管谢尔盖耶芙娜听说要给产妇带猪油和糖来,感到很高兴,可她还是气愤地、很不客气地说:

“你们这些当官的人家,总有好东西吃,到处有好吃的东西等着你们。可是我们吃的东西只有一样—冻土豆。”

“安静点儿!”有一个人叫道。“大家安静点儿!”

船舱的另一头响起一个不很清楚的声音。

忽然,那声音变得响亮起来,压倒其他一切声音。

那是一个人就着油灯的亮光在读报:

“最新消息……我军在斯大林格勒市区发起强大攻势……近日来,驻守在斯大林格勒附近要冲地带的我军向德国法西斯军队发起猛攻。进攻从两个方向开始:从斯大林格勒西北部和南部……”

人们一声不响地站着,在哭。一条无形的奇怪的线连接着他们和那些小伙子,那些小伙子此时此刻正迎着寒风在雪地上前进,有的躺在雪地里,浑身是血,用模糊的目光向人世告别。

老头子和妇女们在哭,工人们在哭,孩子们带着不是孩子应有的表情和大人站在一起听人读报。

“我军攻克顿河东岸的卡拉奇市、克里沃穆兹金车站、阿布加萨罗沃市及其车站……”

薇拉也和大家一起流眼泪。她也觉得有一条线连接着那些在黑沉沉的冬夜里前进、倒下去又爬起来、又倒下去却再也爬不起来的人和在这舱里听着进攻消息的受尽苦难的人们。

为了她,为了她的儿子,为了两手浸在冰水里冻裂了口子的妇女们,为了老年人,为了裹着妈妈的破头巾的孩子们,那些人在迎着死亡往前冲。

于是她十分高兴地哭着想,等她的丈夫上她这儿来,妇女、老年人和工人们会一齐把他围住,管他叫“好孩子”!

那人还在念战报:

“我军的进攻仍在继续。”

六十四

值班参谋向空军第八集团军司令汇报了各团一天来的作战情况。将军把放在面前的报表浏览了一遍,对值班参谋说:

“萨卡布卢卡很不走运,昨天他的政委被击落了,今天又有两名飞行员被击落。”

“司令员同志,我往他们团部打过电话,”值班参谋说,“明天安葬别尔曼同志。军委委员说要去参加葬礼,要讲话。”

“我们的委员就喜欢讲话。”司令员笑了笑。

“司令员同志,两名飞行员情况是这样:中尉科罗尔是在第三十八近卫师防地上空被击落的,小队长维克托罗夫上尉是在德军机场上空被敌机打得着了火,还没有飞到前线,就在高空坠落,恰好落在中间地带。步兵看到,几次想到他跟前去,都被德国人打了回来。”

“是啊,常常有这种情况。”司令员说着,用铅笔搔了搔鼻子。“您现在办一件事:和方面军司令部联系一下,提醒他们,萨哈罗夫曾经答应给我们换一辆吉普,要不然很快就没有车子用了。”

死去的飞行员在积雪覆盖的小丘上躺了一夜。寒风凛冽,星光灿烂。黎明时小丘变成粉红色,飞行员躺在粉红色的小丘上。后来吹起贴地的搅雪风,尸体渐渐被雪埋住。

[1]费密(1901—1954),著名丹麦物理学家。

[2]斯宾格勒(Oswald Spengler,1880—1936)),德国历史学家、历史哲学家,著有《西方的没落》。

[3]胡戈·斯廷内斯(Hugo Stinnes,1870—1924)是德国工业家领袖。克虏伯(Krupp)家族是德国大军火制造商世家。

[4]见《马太福音》第七章。

[5]见《马太福音》第二章第十八节。

[6]德拉戈米罗夫(1830—1905),沙俄时代步兵上将,军事理论家。

[7]苏沃洛夫(1730—1800),俄国伟大的军事家、军事理论家、战略家、统帅,俄国史上的常胜将军,俄国军事学术的奠基人之一,著有军事学名著《制胜的科学》。

[8]库图佐夫(1745—1813),俄国元帅、大军事家。1812年拿破仑一世发动对俄战争时被任命为总司令,取得了卫国战争的胜利。

[9]博赫丹·赫梅利尼茨基(1595—1657),乌克兰民族起义领袖,率领乌克兰哥萨克起义反抗波兰的统治。波兰重新统治乌克兰后,赫梅利尼茨基请求俄国出兵联合抗击波兰,并于1654年签订乌克兰同俄国合并的条约,此后直到1991年乌克兰一直是俄罗斯的一部分。

[10]此处是音译。本意是“浑蛋”。

[11]即维克托·帕夫洛维奇。

[12]据说恺撒在元老院遇刺身亡之前说的最后一句话。当时有数十人刺杀恺撒,其中布鲁特斯是恺撒的好友挚交,也是事件的主要谋划者。

[13]在果戈理的小说《伊凡·伊凡诺维奇和伊凡·尼基福罗维奇吵架的故事》中,因为骂了一声“公鹅”,两个好朋友打了一辈子官司。

[14]尼古拉·泽林斯基(1861—1953年),苏联杰出化学家、科学院院士,在催化反应、有机合成等研究领域做出了重要贡献,对石油化学催化转化的研究具有特殊意义。

[15]原文为法语。

[16]帝国保安总局局长。

[17]奥地利萨尔茨堡以南的疗养地。希特勒常在位于此地的别墅举行会议。

[18]此处模仿屠格涅夫《猎人日记》中一篇的开头。

[19]托谢耶夫的原意是“瘦子”。

[20]原文为德语。

[21]指列夫·托尔斯泰。

[22]即同温同压下,相同体积的任何气体含有相同的分子数。由意大利科学家阿伏伽德罗提出。

[23]兰杰斯曼是犹太人的姓。

[24]“饥饿不是姑妈”是谚语,大意是:饥饿是无情的。
