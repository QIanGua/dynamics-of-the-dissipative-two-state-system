\section{ 新版序}

1961年2月14日上午11点40分,克格勃(苏联国家安全委员会)派人闯入瓦西里·格罗斯曼的住宅,搜查一份书稿。结果他们不只带走了那本书的打字稿,还没收了和它相关的草稿和笔记,甚至就连打出这本书的打字机与碳纸都不放过,行动规格形同逮捕一个活人,只不过他们这次要逮捕的是一本书。这本书的名字叫做《生活与命运》,后人管它叫“二十世纪的《战争与和平》”。

格罗斯曼很清楚自己写了些什么,当初他投稿给杂志社的时候难道没料想到会有这样的结局吗?这是后来一些学者争论的细节问题,我们先且别管,还是回到1961年情人节那场“逮捕”事件的现场,看看格罗斯曼事后的反应。他直接写了一封信给苏联最高领导赫鲁晓夫抗议:“有什么理由让我人身自由,却逮捕了这部我为之呈献生命的书?”

当局似乎很在乎这位作者,历经斯大林、赫鲁晓夫、勃列日涅夫三朝而不倒的苏共意识形态大总管,人称“灰衣主教”的苏斯洛夫(Mikhail Suslov)亲自接见了他。以外表斯文谦逊、彬彬有礼而著称,但又深沉冷峻的苏斯洛夫这样子对格罗斯曼说:“我没有读过你这本小说,但我读了对它的评论和报告。……你为什么要把你的书加入到敌人对准我们的核武器当中?又何必让它引起大家关于苏维埃体制到底还有没有必要的讨论呢?……我可以直接告诉你,这本书在两三百年内都不可能有出版的机会。”

一部前苏联禁书,这个身份多少就能决定一本小说的命运了。在上世纪的六十到八十年代,这个身份或许可以让一本书在所谓的“自由世界”受到许多关注,读者通常会期待能在里头读到铁幕背后冷酷悲惨的真相,同时间接确认了自己的幸运与幸福(好在我没活在那一边)。只不过禁书太多,能从“社会主义阵营”这边侥幸逃到另一边去的书也不少,其中只有几个例子可以赢得大名,获得最高声誉。例如《日瓦格医生》与《古拉格群岛》,它们都得到了诺贝尔文学奖(尽管帕斯捷尔纳克最后被迫拒绝领奖)。

问题是这样的背景也会反过来限制这类小说的生命。冷战结束,它们在很多读者眼中似乎就只剩下了历史见证的价值,别无其他。所以今天提起《古拉格群岛》和索尔仁尼琴,很多人都会露出一丝倦怠的神情,觉得那是本过时的书与一个过时的人。《日瓦格医生》更是可悲,因为后来的文档证明,它在西方的流行原来与美国中情局有些关系,被他们利用,当做冷战意识形态争战的兵器,于是无奈沾染上一层政治污迹。

至于苏联这边就更不必提了,禁书自然是没人看得见的书(审查官员例外,他们大概是那个体制内读书最多见识最广的人)。苏联解体前后,虽然它们也曾火热过一阵,但很快就又被打回冷宫,因为“向钱看”的新一代实在没有太大兴趣去务虚地回顾历史,翻看那些昨天以前还没听过的书。所以曾经遭禁的文学,便和它们命运的对立面─那些得到最高当局赞赏,赢了“斯大林奖”的作品,奇诡地共同进入历史,都没有人要看了。事后,无论是在俄罗斯,西方,还是中国,苏联文学仿佛都成了一个几乎不存在的物事。尤其对俄罗斯以外的一般文学读者而言,俄语文学好像只到二十世纪初为止。少数诗人之外,整个苏联似乎没剩下几个值得重读的作者。以中国的历史背景来看,这种情况特别奇怪,因为俄语曾是我们的主要外语之一,沙俄和苏联文学更曾是社会上的主要读物;可今天,它却只是一排排被置放在书架顶层的蒙尘典籍,“小时代”的大时代遗物。

所以《生活与命运》理应过时。一本前苏联禁书,书名土气(更像是十九世纪的产品),翻译成中文近一千页,全书有名有姓的角色超过一百六十人;更要命的,格罗斯曼的文风竟带着一股扑面而来的“社会现实主义”气息。这本书,甚至连它出版的时机都不太对。1980年瑞士首现俄文原版,读者自然寥寥。1985年英译本面世,当年索尔仁尼琴在西方已经红到发黑,名声渐走下坡,大家很容易以为它只不过是《古拉格群岛》的小弟,所以只有一小圈子的人看过这本其实和《古拉格群岛》非常不同的大书。而大部分写书评的,在报刊做文化版的,甚至连瓦西里·格罗斯曼这个名字都没听过。这也难怪,此时已故的他,毕竟不是个有海外公众知名度的异见分子,没有活着流亡、被人宣传的机会。相反地,他在公众面前大概还算是个“体制内作家”呢,曾经入围“斯大林奖”决选名单,二战期间为《红星报》写的战地报道更是风靡全国,得到官方肯定。这类作家,英语世界又怎么会对他感兴趣呢?身为苏联“作协”成员,格罗斯曼那被压抑的后半生是沉默的,《生活与命运》的遭禁亦是同样沉默,国内没有人知晓,国外没有人声张,一切安静。比较奇特的是,和英文版同年面世的法文本,居然一度成为畅销书,我猜那是法国独特环境所致,他们那时大概还会稍稍关心苏联究竟是个极权体制还是共产主义天堂这种老问题。

我在文字和电子媒体介绍书介绍了二十多年,很少遇到像《生活与命运》这样的作品,觉得推荐它是自己不能回避的道德义务。七八年前读到英文本之后就四处向人宣说,想它有机会在中文世界现身。终于到了去年,北京“理想国”愿意承担,重出这部不合时宜的巨著。“重出”,是因为编辑发现它原来早就有过中译,而且还有三种版本,全在上世纪八十年代末九十年代初,只是我孤陋无知而已。比如他们用做底本的这个版本,俄语文学翻译名家力冈先生手笔(另一个被人遗忘的名字,《日瓦格医生》与《静静的顿河》的译者),原来的译名是《风雨人生》。力冈先生的译者序言成于1989年6月10日,最后一段话是非常直白的吁求:“亲爱的读者,读读这部作品吧!它使人清醒,使人觉悟,使人知道自己是一个人!使人知道怎样做一个人!”如此八十年代的笔致,写在八十年代的终点,这本书合该要在新时期的中国被人忘记。生不逢时,往往是许多好书被埋没的原因。《生活与命运》的三种中文译本全出在上世纪八十年代末到九十年代初那两三年。当时,苏联解体已成事实,连带垮掉的还有几十年来的苏联文学;而中国这里,则一面是笼罩了整片大地的低气压,另一面是正在冒头的人欲春芽,自然没有多少人想去碰这一千页的大书,直觉它是苏联版的伤痕文学,会看得叫人呵欠连连。

但是最近十年,它的命运却忽然逆转,一下子又复活过来了,西方每一个评论家都拿它和《战争与和平》相比,并且纷纷奉上一本小说所能得到的最高赞誉,比如说“我用三个礼拜读完,再用三个礼拜复原,在那段日子里我几乎难以呼吸”(琳达·格兰特语)。

第一个拿它和《战争与和平》相比的,并非“别有用心”的西方人(这说法来自豆瓣网上的一则短评,那则评论的作者很不屑西方世界对它的赞誉,认为其背后“别有用心”),而是1988年俄文原版终于能在祖国出版之后的苏联评论界。当时就有人立刻宣告:“那漫长的等待终于结束了!”等待什么?当然就是等待另一本《战争与和平》。就像托翁为拿破仑入侵俄罗斯的战争写出了一部不朽巨著一样,更加惨烈悲壮的“卫国战争”当然也得配上同样伟大的作品。这几乎是他们自二战一结束之后就马上开始了的漫长期待,整个苏联文坛都在寻找接得下这份重担的候选者,好几代苏联作家也都努力地想要满足那份期望,于是一本大书接着一本大书地上市。只不过,它们似乎都还和《战争与和平》有点不小的距离。

《生活与命运》堪比《战争与和平》,最表面的理由在于外形。都是写一场抵抗入侵的战争,都是人物众多、支线庞杂的大书,都以一个家族当做轴线,都是全景式的鸟瞰神目,都在虚构叙述当中夹杂议论沉思。但于我看来,格罗斯曼之所以无愧于前人,是他细致地写出了“战争”与“和平”这两种极端不同的状态,以及连接它们彼此的微妙联系;又在这战争与和平的双重境况当中,几乎让我们看到了苏联社会的全部细节。从斯大林、赫鲁晓夫这等史上留名的大人物(其中甚至还有一段关于希特勒的难忘描绘),一直到大草原上的牧民与农夫;从前线红军在漫天炮火当中的日夜生活,一直到后方官僚体系的具体运作;这个帝国的每一条神经线乃至它最最末梢的毛细血管,全都被格罗斯曼一根根挑选出来耐心检视。

当然,那是战争,就算离战火最远的地方(例如西伯利亚深处的集中营),也很难不受战事影响。所以“战争”与“和平”这两种状态的比对,只不过是个方便说法;可是,我又分明看到了格罗斯曼刻意分别塑造这两种状态的用心。在他笔下,相对安全平静的后方有时候竟比斯大林格勒战线上的最前锋还危险。因为后方的人或许有床可睡,但睡不安稳;或许有饭可吃,但食不下咽。因为他们要担心自己说过的每一句话,生怕犯错;他们要留意权力的走向,以免一不小心走上“邪路”。战壕里的士兵则不然,由于不晓得今晚是否人生在世的最后一夜,反而因此坦荡,想说什么就说什么,便连人际关系也都简单了许多,回复到它最该有的本然面目,喜怒哀乐尽皆自然无碍。夸张点讲,在格罗斯曼笔下,战场上的人居然活得更加像人。

没错,战争“矫正”了很多事情。一个军人的履历表变了,评价他的标准不再是他家有没有出过托洛茨基主义者,父母是不是孟什维克分子;而是他开枪开得够不够准,面对敌军轰炸的时候又够不够冷静。身经百战的老将被人从集中营里放了出来,因为会不会带兵在这时刻要比他在政治上的关系要紧;一个见过大场面的老兵可以放胆批评集体农场的失败,因为同袍现在只在乎他对敌方下一枚袭来炮弹路线的判断。

后方,那片相对平静的大地却还是处在苏联式的“正常”当中。例如主角之一的维克托,他和一群物理学家同事偶尔会在夜话之中趁着酒意胡说,指点江山,开开斯大林的玩笑(斯大林同志太伟大了,他比牛顿更早发现地心引力的作用),批评当局的文艺政策(什么叫做“社会主义现实主义”?它就是党和国家的魔镜,每当党和国家问它世界上谁最正确最伟大,它就会说:你,你,你)。但散伙之后,在回家的路上,刚刚还在一起笑闹的A会别具深意地提醒维克托:为什么B能那么大胆说话?你不觉得奇怪吗?当年大清洗的时候他也被捕,但没几个月就放了回来,那时可没有人回得了呀。再过几天,反过来又轮到B对他发出警告:你得留意A,有人说他和上头的关系非比寻常……

当时维克托研究的是至关重大的核分裂问题(其原型可能是“氢弹之父”萨哈罗夫),他的成果一开始备受赞赏,同事们对他既热情又友好,觉得他是个天才。可是自从上头派来了一个新领导,情况马上就两样了。新领导批评他这个犹太人过度夸大同裔爱因斯坦的成就(别忘记斯大林的政策也是反犹的),指责他在政治上不够合群,甚至使他逐步陷入险境。于是共事多年的朋友渐渐翻脸,在路上碰见会假装不熟,在他缺席的会议上替他检讨鸡毛蒜皮般的过错。就算他那曾被大家夸誉的研究成果,也不知怎的突然显得漏洞百出,无关痛痒。维克托自此孤立,变得更加激愤,勇气也跟着大了不少,随时预备慷慨就义,为他所相信的真理献身。

然而,某天下午,正当他在家准备被逮捕的时候,电话响了。“您好,施特鲁姆同志。”这声音太耳熟了,就是那把大家常常能在电台广播上听见的声音,维克托呆了一呆,心想莫非是有人恶作剧。不会吧?谁敢开这样的玩笑?于是维克托·施特鲁姆严肃地回答:“您好,斯大林同志。”他一边说一边惊讶,“不大相信这是他在电话里说这种不可思议的话”。几分钟过后,斯大林在电话另一端留下了一句神谕般的告别语:“再见,施特鲁姆同志,祝您研究顺利。”

既得神谕,世界遂因此美丽。“维克托原以为,那些拼命整他的人见到他会不好意思的,但是在他来研究所的那一天,他们却高高兴兴地和他打招呼,对直地看着他的眼睛,那目光充满了诚意和友情。特别使人惊异的是,这些人的确很真诚,他们现在的确对维克托一片好意。”他又变回了那个天才物理学家,一切以往很复杂很麻烦的事情现在办起来都很容易了(格罗斯曼不忘评述,说这也是“官僚主义”的特点,平常可以让最简单的小事寸步难行;但在需要集中精力办大事的时候,却又能飞快完成最困难的任务)。他有了专用汽车,他每一句冷笑话都变得那么好笑。就连他的太太上街买东西,前几个星期装作不认识她的妇女也都忽然变得热情温暖。

更甚的是,他还发现大家原来都有很“人情味”的一面,党委书记原来喜欢在黎明时分钓鱼,有同事收养了一个有病的西班牙孩子,另一个同事则以在这冷寒之地种植仙人掌为乐。他心想:“啊,这些人实在不是多么坏。每个人都有人情味儿。”是斯大林的一通电话,使他看见了每个人最可爱最私密的那一面;是那通电话使大家愿意在他面前展演人性。维克托现在是所有人的好朋友了。

不久之后,英国报刊批评苏联当局冤屈几个医生,指控他们毒杀大作家高尔基。不愤西方媒体抹黑,苏联科学界动员各个单位“自发”联署抗议,维克托所在的这个研究所也不例外,他的领导极力邀请他带头在一份声明上头签名。可是在维克托看来,那份声明分明就是错的,它诽谤了一个正直的人,一个曾经对自己家庭有恩的好医生。他觉得英国人批评得没错,苏联确实构陷了一个他自己认识的声誉卓著的医学教授。违心害人,这真是维克托无论如何都做不到的事。才几个星期之前,他连以死明志的心都有,这时应该更不必担心。可一碰到领导和同事们的殷恳目光,“他感触到伟大国家的亲切气息,他没有力量投身寒冷的黑渊……今天他没有,实在没有力量。使他就范的不是恐惧,而是另外一种消磨力量的温顺感情”。出于人性对人际温情的真实需要,而非从天而降的特权与待遇,他开始内心交战,试图说服自己:反正几个被告自己也在法庭上认了罪,我现在加入指控他们又有什么不对呢?反正我也改变不了什么。道理一想通,维克托便掏出了自来水笔,在这份声明签下自己的名字。

今日局外幸运儿,常常不能理解政治高压底下的生活,不明白一个人为什么妥协,为什么要出卖别人,又为什么会出卖自己。于是我们总是如此简易地断定,那是出于恐惧,不够勇气,又或者图谋利益,舍不得悬在头上的萝卜。格罗斯曼却在读者面前展开了复杂的道德处境,让我们发现是非抉择的艰难。维克托昧着良心签署那份害人声明,便不是为了刚刚到手的特权与地位,也不是因为害怕自己会受到惩罚。他的动机,其实只不过是至简单的人性需要罢了;那就是他人的温暖认同,一种被友侪围绕的感觉。

同样的需要,到了战场上头,却能变化出荒谬可笑,但又分外残酷的戏剧,例如一个苏联士兵被炸弹的威力埋进战壕,侥幸不死,并于黑暗中触及另一具温暖的身体,于是本能地紧紧握住对方的手。两个陌生人便借此慰藉那不可言喻的惊恐,都直觉对方一定是生死与共的同袍。过了一会儿,地面上稍稍平静,他们奋力拨开顶上瓦砾,让光线照进坑洞,这个红军战士才发现自己的错误。刚刚和自己那么亲密的伙伴,竟然是个死敌德军。怎么办?立刻翻脸动手?不,他俩尴尬无言,很有默契、很安静地各自爬出洞口,一边四处张望环境,一边提心吊胆朝着己方阵营遁走。亲身经历过战场诸种奇诡的格罗斯曼解释:他们不怕对方在背后开枪,只怕自己的战友看见之前的情景,一报上去这可能就是通敌叛国的死罪了。

没错,这两个正在交战的国家是相似的,至少在令自己人恐惧这一点上。

透过一位审问犯人的纳粹党官之口,格罗斯曼对苏联这场伟大的卫国战争做出了一个最大不讳的宏观判断。原来正邪如此分明的战事,骨子里居然是两个极权体制之间的斗争。那个很懂得心战技巧的纳粹,不断逼着被俘的资深苏共党员承认,他俩其实是镜面的两端:两边都有伟大的领袖,两边都宣称自己拥占了至高无上的真理,两边都把无数人的牺牲当成实现真理的代价,两边都为此培养出了一大批最忠诚最具党性的信徒—例如坐在审讯桌两端的这两个人。

若是如此,这场仗又还有什么意义呢?天地不仁,以万物为刍狗;然刍狗般的士兵却不能接受自己的生命无谓,他们必须相信自己站在正义的那一边,相信自己的死亡背后别有高远的价值。所以,经历过不自由生活的军人会认为自己正在为即将到来的自由而战,只要打败眼前的德军,不只国土和民族会得到保存,甚至就连苏联也都可能会变成一个更加美好的国度。既然这是一场关乎自由及解放的战争,所以在作战交火的这一刻,他们就得亲身践行自由。所以在描写战场的章节里,格罗斯曼时时将视角沉降到沙土飞扬的地面,在一阵阵爆炸声响之间,在一串串从头上掠过的子弹丛中,使读者看见一个个士兵如何在最接近死亡的那一刹那裸呈出人的根本。

尤其是书中那有名的“6—1号楼”,红军留在斯大林格勒德占区中的最后一个据点,就好比淞沪会战当中的四行仓库,一小队战士勇敢地守住了这个残破的建筑,拼死抵挡德军火网包围。这一段故事大可谱成一曲最典型的壮烈史诗。然而格罗斯曼毕竟是格罗斯曼,他的重点不是脸谱化的英雄,而是一组各有偏好各有性格的活人。例如原本从事建筑工程的工兵队队长,他的任务从过去的修盖房子变成了拆毁敌阵当中的建筑,于是“很需要思考思考这种不寻常的转变”。步兵指挥官战前则在音乐学院学声乐,“有时他在夜里悄悄走到德国人盘踞的楼房跟前唱起来,有时唱《春天的气息,不要把我惊醒》,有时唱一段连斯基咏叹调”。这组人会在开枪和躲子弹的空当咒骂食物的贫乏,争论选择女子的关键(“我认为姑娘的胸脯是最要紧的”),乃至于“外星世界有没有苏维埃政权”等各式各样的古怪话题。说着说着,他们还会讲出一些后方“和平”世界连想都不敢想的话:“不能把人当绵羊来领导。列宁那样聪明,就连他也不懂得这一点。所以要革命,为的就是不要任何人领导人。”这座楼是前线中的前线,每一个人都不知道自己还能不能看到第二天的日出,所以它反而也是全书最自由、最有生命力的世界。难怪苏军战线指挥部特地派来的政委(他们担心这个阵地的政治思想会走偏,所以命令一个政委冒着弹雨偷偷潜进指导),能在这里头发现危险的气息。曾在那座楼里和这些不正常的正常人并肩作战过的幸存士兵,则会事后慨叹:如果不认识这些人,生活还“能算是生活吗”?

不要以为格罗斯曼的战争与和平就是美化战争,挖苦和平。不,没有几个作家会比他更了解战争的无情。色彩这么丰富的“6—1号楼”竟然转眼就在地平线上消失了,没有临终遗言,也没有英雄面向镜头的最后笑容,十来二十个鲜明人物就此消失在几行不到的文字里头。这是格罗斯曼杀死他大部分角色的办法,说走就走。为什么会是这个样子?那可全是行进中的漂亮生命呀?且再引一次琳达·格兰特(Linda Grant)的评语:“那是因为生命本来如此。”又或者木心先生更漂亮的一句名言:“我所见的生命,都只是行过,无所谓完成。”

和平也好,战争也好,在《生活与命运》里头皆是人类生存的严苛背景。斯大林与希特勒治下的和平扭曲了人性,两个体制之间的战争却变态地解放了人性,这岂不荒谬?是的,格罗斯曼的二十世纪就是这样荒谬,托尔斯泰式的“正能量”几乎没有一点存在的机会。

世界如此冷酷。一个私底下对国家政策有很多怨言的宣传人员,会在报纸评论上头指出,集体化政策之所以出现饥饿状况,是因为部分富农故意藏起粮食把自己饿死,好恶毒地抹黑国家。一个才瞎了双眼没多久的伤兵,退到后方医院,他在公共汽车站前请人帮忙登车,那些平时可能很懂得爱国爱军的平民百姓,却在车来的时刻自顾自地推挤拥上,不只不理会他,而且还把他撞倒在地上。他“用鸟叫般的声音叫喊起来。他的帽子歪到了一边,无可奈何地摇晃着棍子,他那一双瞎眼,大概也清楚地看见了自己的窘境”。盲人拿棍子敲打着空中,站在那里又哭又叫。一个瞎子,就这样被大家留在这片雪地。而伤兵医院里边,一个母亲终于找到了儿子,她对着尸体小声说话,怕他着凉还替他盖好被子。所有人都对她的平静感到惊讶,却不知道这“就好像老猫找到已死的小猫,又高兴,又拿舌头舔”。一个热心善良的德国老太太在俄国住了一辈子,这时却被当做敌方间谍带走,向当局诬陷她的其实就是她的邻居,可能是为了趁机霸占她的屋子。她的邻居不只不替她说话,而且还有意无意地用开水烫伤老奶奶留下来的猫,不久之后它也死了。一个一心向上的领导最喜欢关怀工人和农民的伙食,老在他们面前严词批评工厂厂长和地方干部,指责他们不真心为民服务。他的言语通俗“接地气”,甚至偶尔带点粗话,老百姓没有不喜欢的。可是一回到办公室,他却只谈数字和指标,要求下属削减群众的生活开支,提高工厂与农场的生产力。经过无数这样的细节之后,我还用得着说集中营里的惨况吗?就提一点好了,几个纳粹高官视察刚刚落成的毒气室,顺便在那四堵白墙之间举办晚宴。桌布上是浪漫的烛火与盛着红酒的玻璃杯,他们对着美食举杯祝贺最后方案的成功,似乎后来死在里头的几百万人真是破坏世界卫生的害虫。这是一个令人喘不过气的世界,在苏式社会现实主义背景下练笔长成的格罗斯曼,冷冷地一字字刻写,犹如照相。

不过,就像潘多拉的盒子似的,格罗斯曼总能灵视般地在密不透风的铁箱内看见一点多余。好比他战时笔记里的这一段:“当你坐下来想要写些关于战争的东西的时候,很奇怪,你总是会发现纸上的空间不够。你写了坦克部队,写到了炮兵。但忽然间,又会记起一群蜜蜂如何在焚烧中的村庄上空飞舞。”这多出来的一点点,不只为他的直白书写抹上一股超自然的诗意,有时候还会替这个世界留下一点最后的希望。

《生活与命运》里头最令大多数读者感动的一幕,当是医生索菲亚主动放弃了最后的求生窗口,好陪着萍水相逢的小男孩达维德走进毒气室,让这个天性喜欢动物的孩子不要孤单死去(他看见被杀的黄牛会哭,怀中总有一个养着蚕宝宝的火柴盒)。另一个同样脍炙人口的段落,是一名刚刚在地上看见儿子尸体的俄国太太,本来悲愤莫名,但在看着一个德军战俘走过的时候,却忘了报复,反而把手里的面包塞给那名瘦弱青年,就连她自己也搞不清楚自己这么做的原因。格罗斯曼管这类异常的善行做“人性的种子”;没有来由的、不起眼的种子。他说:“人类的历史不是善极力要战胜恶的搏斗,人类的历史是巨大的恶极力要辗碎人性的种子的搏斗。”

书里这点点星火,一丝丝人性种子的芽苗,我忍不住坏心眼地怀疑它们其实是不是格罗斯曼的幻想。一个温柔的人不忍,于是文字成全。就像我曾在多年前介绍过的短篇《狗》,格罗斯曼为第一个被人类射上太空的实验狗“莱卡”写下了比现实美好得多的结局,让它回到地面,摇着尾巴回到饲养它的科学家身边,亲吻那双喂过它、摸过它,又把它送出大气层的手。这似乎是格罗斯曼的风格,常把想象力用在最悲伤的事情上头,在想象中陪伴孤独承受苦的生命,陪伴他,安慰他。这不是出于煽情,只是为了不忍。就像他在母亲死于德军手上的多年之后,写了一封寄给母亲的遗书,在那里面,他不停想象母亲最后时刻的情景,似乎自己就在她的身边。他甚至想到了妈妈生前见到的最后一个人,是否就是那个将会把她杀死的士兵。

我的这种猜测,来自我对格罗斯曼这个人的一丁点理解。1961年冬天,他死前两三年,《生活与命运》已被当局收走,完全看不到出版希望;在那个体制之内,他的文学生命也已走到尽头,此时的他拖着病躯来到亚美尼亚旅游。一天,不知是胃癌影响,还是酒精作用,他在朋友的车上忽然腹绞,可生性害羞的他不好意思张扬,眼看就要上吐下泻,尊严尽丧。好在朋友半途停车加油,他趁机奔去厕所。事后,他在笔记里回忆:“我记得莫斯科的作家都不喜欢我,认为我是个失败者,是个可怜虫。他们说得对,我完全同意。不过,就这件事看来,我倒觉得自己还是很幸运的。”他的身子开始破损,他倾其一生的巨著被捕,他的朋友所余无几;他不知道以后人家会拿他和托尔斯泰相比,他不知道俄罗斯政府会在2013年公开交还前苏联带走的文稿,更不可能知道这本书会被俄罗斯电视台改编成收视率极高的电视剧。但他竟然还是觉得自己幸运,就只是因为他来得及上厕所。

梁文道

2015 年7 月于北京
