\section{ 导读}


“为长眠者发声”:

瓦西里·格罗斯曼的生平与作品

[英] 罗伯特·钱德勒  著

李广平  译

1905年12月12日,瓦西里·谢苗诺维奇·格罗斯曼出生于乌克兰的别尔基切夫市(Berdichev),当时那是欧洲最大的犹太人聚居地之一。他父母都是犹太人,起初给儿子起名叫约瑟夫(Iosif),但是这个名字一看就是犹太名,于是就改为俄语里对应的名字,叫瓦西里(Vasily);他们家境殷实,早已融入当地社会。瓦西里年幼的时候,父母好像就已经离异了,他由母亲抚养长大,还有一位有钱的舅舅出钱帮助他们。1910年到1912年,瓦西里和母亲住在瑞士,很可能是在日内瓦。他母亲名叫叶卡捷琳娜·萨韦列夫娜(Yekaterina Savelievna),后来做了法语教师,所以瓦西里一辈子法语都非常出色。1914年到1919年,他在基辅上中学,1924年到1929年,在莫斯科国立大学上学,化学专业。[1]入学不久他就意识到,文学才是自己真正的宿命。但是他对自然科学从未失去兴趣;《生活与命运》的中心人物维克托·施特鲁姆是一个核物理学家,这并非没有缘由,而施特鲁姆在很多方面都是作者的自画像。大学毕业后,格罗斯曼搬到了顿巴斯(Donbass),那是个工业区,他先是在一个矿区当安全检查员,后来又在一所医学院校当化学老师。1932年他得以回到莫斯科,1934年发表了短篇小说《在别尔基切夫市》,得到马克西姆·高尔基、米哈伊尔·布尔加科夫、艾萨克·巴别尔[2]等不同作家的赞誉。那一年,他还出版了一部长篇小说《格留考夫》[3],写的是顿巴斯矿工的生活。1937年,他加入了声望极高的苏联作家协会,长篇小说《斯捷潘·柯尔丘根》(1937年至1940年发表)获得斯大林奖提名。

文学批评界常把格罗斯曼的人生分为两部分。例如,茨维坦·托多罗夫(Tzvetan Todorov)就认为,“功成名就的苏联大作家,彻底脱胎换骨的仅格罗斯曼一人,至少他是洗心革面最显著的。身为奴隶的他死了,一个自由人诞生了”。[4]这话说得可谓掷地有声。但是,若把他前后绝对区分开来,说他三四十年代是一个“从命”的作家,五十年代摇身一变,成了一个“持不同政见者”,还写出了《生活与命运》和《一切都在流动》,那可就错了。《格留考夫》今天读起来也许会显得沉闷,但是在当年,一定具有惊人的力量。1932年,高尔基对初稿颇有微词,说是“自然主义”。其实,“自然主义”是个苏联的暗语,凡是写出来的东西太真实,暴露了苏联的现实,有碍观瞻,统统说是“自然主义”。高尔基报告的结尾建议作者反躬自问:“我为什么要写作?我要证实的是什么真理?我想要哪种真理胜出?[5]即使是在那时,看到高尔基对真理的犬儒态度,格罗斯曼想必一定是厌恶的。然而不容否认,高尔基的直觉很有两下子;格罗斯曼对真理的爱将来会带来什么遭遇,似乎他已经觉察出来了。几年以后,格罗斯曼写了一个短篇小说《四天》,里面引用了一句格言:“最真就是最美。”1961年,《生活与命运》的手稿被抄没以后,格罗斯曼居然给赫鲁晓夫写信,说:“我书里写的是我过去信仰的,并且现在继续信仰的东西。我只写自己的想法,自己的感受,自己的痛苦。”[6]

格罗斯曼身上的某些东西—对真理的爱,或许还有批判的智慧—不仅令高尔基警惕,也引起了斯大林的警觉。用今天的眼光来看,《斯捷潘·柯尔丘根》也像《格留考夫》一样,已经够正统的了,但斯大林还是把它从斯大林奖金提名作品的名单上划掉了。他一锤定音,说这本小说写年轻的革命者,实际上是“同情孟什维克”。[7]其实,格罗斯曼既不是孟什维克,也不是殉道者;不过,在大恐怖时期,他显露出了相当大的勇气。1938年,他第二任妻子奥尔加·米哈伊洛芙娜(Olga Mikhailovna)被捕了。格罗斯曼立刻收养了奥尔加与前夫鲍里斯·古贝尔(Boris Guber)所生的两个儿子,古贝尔此前一年已被逮捕。如果不是格罗斯曼动作快,这两个孩子说不定会被抓起来,关到拘押“人民敌人”子弟的劳改营里去。接着,格罗斯曼给内务人民委员会[8]的秘密警察头子叶佐夫(Yezhov)写信,说奥尔加·米哈伊洛芙娜现在是他的妻子,不是古贝尔的妻子,因此,她前夫的事不应该拿她是问,他们已经完全断绝关系了。那年晚些时候,奥尔加·米哈伊洛芙娜被释放了。[9]格罗斯曼的朋友利普金评论说:“这一切看起来好像再正常不过,可在当时,胆敢给国家首席刽子手写这样的信,一定是非常勇敢的人。”[10]格罗斯曼好几篇描写逮捕和检举的短篇小说,就是在这个时候动笔写的,可是一直到二十世纪六十年代才首次得以出版。

格罗斯曼的不同政见是逐渐发展而成的,并不是经过哪个单个的事件,一下子就成为异见人士了。像大多数人一样,他也有行为前后不一的情况。整个战争期间,他好像既不怕德国人,也不怕苏联秘密警察。但是,1952年,斯大林的反犹运动压力越来越大。官方登出一封公开信,说犹太医生要谋害斯大林的性命,呼吁以最严厉的手段惩办这些医生。格罗斯曼同意在信上签了名。[11]

在那个节骨眼上,格罗斯曼居然示弱,这似乎很奇怪。有可能是他一时的失常:当时他刚刚和诗人兼编辑亚历山大·特瓦尔多夫斯基(Aleksandr Tvardovsky)有过争执,头脑不怎么清楚,就在这个时候,上头让他签字。[12]然而,《生活与命运》几乎是一部百科全书,把极权社会错综复杂的人生百态和盘托出,也从未有人比格罗斯曼更好地明确写出个人要想抵抗极权压力的艰难:

但是有一种看不见的力量把他压住。他感觉到它的威慑的重量,它强迫他按它的意图去想,强迫他按照它的意思写。它就在他身体内部,强迫他的心收缩,溶解他的决心……

只有不曾亲身体验过这种力量的人,见到有人屈服于这种力量,才会感到惊讶。亲身体验过这种力量的人,感到惊讶的倒是另一点:敢于发一下火,哪怕是迸出一句怨言,或者很快地做一个表示抗议的手势。[13]

格罗斯曼并不想掩盖自己的失策。他最自责的是1941年德国入侵之后,没有把母亲从别尔基切夫接出来。但是他也怪罪妻子,因为她和母亲关系不好。战争前夕,格罗斯曼曾向妻子提出接母亲来莫斯科,住在他们家里,妻子奥尔加·米哈伊洛芙娜却说地方太小,不方便。[14]1941年9月,他母亲,叶卡捷琳娜·萨韦列夫娜,被德国人杀害了。同时被害的还有生活在别尔基切夫的30000名犹太人的大多数。

格罗斯曼死后,在他的文件里发现了一个信封。里面有两封信,是他在1950年和1961年写给他死去的母亲的,一封是母亲九周年忌日那天写的,另一封是母亲二十周年忌日那天写的,除了信还发现了两张照片。格罗斯曼在第一封信里写道:“我总在想,你是怎样死的,是怎样走到被害的地方,我想了几十次,也可能想了几百次,杀害你的那个人长得什么样,那人是最后一个见过你的人。我知道,当时你心里一直都在想着我。”[15]有一张照片是母亲和瓦西里的合影,照片上的他还是一个小孩儿;另一张照片是格罗斯曼从一个德国党卫军军官的尸体身上取下来的,照片上是一个大坑,坑里有几百具裸体的女尸,有成年妇女,也有小姑娘。母亲的死令格罗斯曼极度内疚,他和妻子相互指责,这一切都反映在《生活与命运》里了。书中的人物安娜·谢苗诺芙娜(Anna Semyonovna)就是格罗斯曼母亲的形象,她给儿子写了一封信,好不容易才把信托人偷偷带出了犹太人隔离区。在所有为东欧犹太人发出的悲叹之声中,我不知道有哪个比这一封信更令人动容。[16]

格罗斯曼也许把战争当做了赎罪的机会。他不顾视力不好,健康欠佳,报名参军想当一名普通士兵。结果,他被分配到苏联红军的报纸《红星报》当战地记者,很快便赢得各方好评,其坚韧勇敢给几乎所有人留下了深刻的印象。他报道了所有的主要战役,从莫斯科保卫战到攻克柏林。普通士兵和高级将领都爱看他的文章。成群的前线士兵聚集在一起,而其中一人从唯一一份《红星报》大声朗读报纸的内容;作家维克多·涅克拉索夫曾在斯大林格勒参加战斗,他记得“登载着格罗斯曼和爱伦堡文章的报纸被读了又读,直到报纸已经变得破破烂烂”。[17]

没有哪个记者像格罗斯曼那样报道“无情战争的真情实况”(格罗斯曼语)。他的记事本上很多大段的文字,要是被秘密警察看见了,很可能会治他死罪。有些话对军队高官们的形象非常不利,有的报道居然不顾禁忌,把开小差、勾结德国人等通敌行为都记录了下来。

他的笔记本里记满了出乎意料的事情,很多都在《生活与命运》里再现了出来。早期的笔记有这么一条:“前线的气味通常是停尸房和铁匠铺那两种气味兼而有之。”格罗斯曼到斯大林格勒没几天就发回了报道:“落日余晖照在广场上,有一种阴森怪诞的美:浅粉色的天空透过成千上万空洞的窗口和屋顶朝外看着。一幅巨大的宣传画用俗气的颜料写着:‘光辉大道。’”[18]

格罗斯曼采访从来不记笔记,或许是怕吓着被采访的人。他喜欢凭借过人的记忆写稿。他能让各行各业的人,不论男女,都信任他:狙击手、将军、战斗机飞行员、苏军惩戒营里受惩罚的士兵、农民、德国战俘,以及在德国占领区冒着被治罪的危险继续授课的学校教师。《红星报》总编辑奥滕伯格(Ortenberg)写道:“斯大林格勒前线的记者都很惊讶,格罗斯曼居然让师长打开了话匣子,这个沉默寡言的西伯利亚人和他一谈就是六个小时……格罗斯曼问什么,他都毫无保留地奉告。在这战事危急的关头,师长还这么有问必答。”[19]奥滕伯格还写过这样的话:“我们没催过他,因为都知道他是怎么干活的。不管条件多么差,不论是在一灯如豆的破棚子里,还是在野地里,不论是躺在床上,还是在满屋子人的农舍里,他都能写下去,但写得很慢,他始终全神贯注,投入了全部精力。”[20]

1943年,斯大林格勒的德军投降后,苏军先头部队解放了乌克兰。格罗斯曼当时随军报道。他听说在巴比谷(Babi Yar)有十万人惨遭屠杀,其中大部分是犹太人。过了不久,在别尔基切夫,他得知了母亲遇害的详情。《旗帜报》(Znamya)刊出了他的一篇小说《老教师》,讲的是有一个城市,跟别尔基切夫差不多,但没提城市的名,城里有好几百名犹太人遭到屠杀,故事主要讲的是屠杀前发生的事。他还写了一篇文章《没有犹太人的乌克兰》,是对死者的长篇祷告。这篇文章被《红星报》退稿,但是“犹太人反法西斯委员会”的报纸用意第绪语(主要是犹太人的语言,近似德语,也掺杂着希伯来语和斯拉夫语—译者注)刊登了出来。[21]这两篇文章是世界上最先揭露犹太人大屠杀的报道。[22]格罗斯曼还写了一篇生动而冷静的文章《特雷布林卡地狱》(1944年下半年发表),这是世界上第一篇揭露纳粹死亡集中营的文章,其他报道,不论何种语言,都在它后面。这篇文章在纽伦堡审判时再次刊出,还被用作证词。

有关犹太人大屠杀的作品,迄今已经出版很多,可是即便今天,大屠杀惨烈的程度,世人还是难以想象。说到犹太人种族灭绝(Shoah),乌克兰历次屠杀是开始,波兰各死亡集中营是高潮。格罗斯曼是调查纳粹灭犹的第一人。纳粹党卫军竭力销毁波兰特雷布林卡(Treblinka)灭绝营的痕迹,妄图毁灭罪证。格罗斯曼采访了当地农民和四十位幸存者,设法重现了这个灭绝营的内部结构和诱杀伎俩。他深入透彻地写到纳粹的骗术,写到“党卫军研究死亡的神经科医生”如何“再一次蒙骗了人们的思想,故意散播一丝希望……他们一字一顿地大声说:‘妇女儿童要把鞋脱掉,袜子要放进鞋里,要整洁……进浴室的时候必须带上身份证件、钱、毛巾和肥皂。我再说一遍……”[23]英国诗人、哲学家柯勒律治(Coleridge)曾经给“想象力”下过这样的定义:“让灵魂摆脱客观事实的禁锢而获得自由,这种摆脱的能力就叫想象力。”显然,格罗斯曼天生就有这个能力,并且达到了最为高超的水平。

但是,苏联官方的宣传口径是这样的:在希特勒统治下,各族人民的苦难都是一样的。如果有人说,犹太人所受的苦难最为深重,苏联官方就用一个标准答案来反驳:“死人都一样,不要做区分。”

一旦承认了绝大多数死者是犹太人,就没法否认苏联的其他民族是种族灭绝的帮凶了;再说,斯大林本人就是反犹的。1943年到1946年,格罗斯曼和爱伦堡都在为“犹太人反法西斯委员会”撰写《黑书》(The Black Book)。这是一部纪实作品,记述了在苏联和波兰的土地上发生的屠杀犹太人的事件。但是《黑书》从来就没有出版过。[24]不管怎么妥协让步,这样的书,苏联是不会允许出版的。

长篇小说《人民是不朽的》也像《斯捷潘·柯尔丘根》一样获得了斯大林奖提名,可是,尽管评选委员会一致推选,斯大林还是将它否决了。格罗斯曼的下一本小说《为了正义的事业》,刚开始的时候获得好评,可是后来却遭到批判。这可能有两个原因:第一,格罗斯曼是犹太人;第二,当时正是斯大林统治如日中天的时候,战争的实际情况一点儿都不许写,战争第一年的惨败更不许写了。“犹太人反法西斯委员会”其他领导成员都已经被捕的被捕,被杀的被杀,新一波大清洗马上就要开始。1953年3月斯大林去世,若非如此,格罗斯曼几乎肯定也会被捕。

接下来的几年格罗斯曼获得了公众意义上的成功。他被授予了声望极高的“红旗劳动勋章”,《为了正义的事业》也再版了。这个时候,格罗斯曼正在写他那两部杰作:《生活与命运》和《一切都在流动》。这两部作品都是直到1980年代后期才在俄罗斯出版问世。[25]《为了正义的事业》政治上没有《生活与命运》那么异端。作者本来想把《生活与命运》作为《为了正义的事业》的续篇来写。《生活与命运》里的人物,很多也都是《为了正义的事业》里的人物,但是最好把《生活与命运》作为一部独立的小说来看。这本书很重要,不仅是文学巨著,也是史学鸿篇。斯大林统治下的俄国,没有比这本书更为全面的描写。其他持不同政见作家—沙拉莫夫、索尔仁尼琴、曼德尔施塔姆夫人,他们的感召力来自他们都是体制外的人;而格罗斯曼的感召力,至少部分地来自他对苏联社会各个层面都了如指掌。《生活与命运》是一整个时代的写照。在《生活与命运》中,格罗斯曼实现了很多苏联作家竭尽全力却没有取得的成就。书中每个人物,不管如何生动地呈现,都代表了某一群人或某个阶层,其命运是那个阶层的命运的缩影:施特鲁姆代表的是犹太知识分子;戈特马诺夫代表犬儒的斯大林主义官员;1930年代成千上万老布尔什维克被逮捕,阿巴尔丘克和克雷莫夫是其中的两个;1941年苏军一败涂地,当局迫不得已,一度改弦易辙,先不看党员的出身,而看他能不能打仗(至少有几年是这样),诺维科夫就是这样一位可敬的军官,苏联实行这个政策后,他的能力才得到承认。这部小说,不论是文体,还是结构,都没有什么标新立异的地方。但格罗斯曼书中的道德拷问步步紧逼,他把苏联共产主义等同于纳粹主义,这可是异端邪说。若不是因为他这个论调,《生活与命运》几乎就奇怪地符合了当局的要求:要求作家写出真正的、苏维埃史诗般恢弘的作品。然而他却说苏联共产主义和纳粹国家社会主义是互为镜像,那个时候,即使是在西方,能听懂这话的人也没有几个。这个政权最引以为豪的就是打败了纳粹,没有什么比这个异端邪说更触目惊心了。

格罗斯曼有两个知己密友,一个是谢苗·利普金(Semyon Lipkin),一个是叶卡捷琳娜·扎波罗茨卡亚(Yekaterina Zabolotskaya)。1960年10月,格罗斯曼不顾这两个朋友的劝告,把《生活与命运》的手稿交给了《旗帜报》的编辑。当时正是赫鲁晓夫“解冻”时期,格罗斯曼胸有成竹,认为这本小说能够出版。1961年2月的一天,三个克格勃(KGB)军官来到他家,抄没了他的手稿和相关资料,连复写纸和打印色带都没收了。当局不逮捕人而“逮捕”书,苏联历史上只有两次,这回是其中一次。[26]除了《古拉格群岛》,还没有哪本书被认为这么危险。[27]当局叫他在一个保证书上签字,保证不把克格勃这次登门造访的事和别人讲。他拒绝签字。但克格勃的其他要求,他好像照办了。他把这几个克格勃军官领到他表弟家,让他们把其他两份手稿也抄去了。但是,格罗斯曼另外还备了两份手稿,克格勃竟然没发现:一份留给了谢苗·利普金保存,一份留给了廖丽亚·多米尼吉娜(Lyolya Dominikina)保存。廖丽亚是他学生时代的朋友,和文学界没有任何联系。

很多人都认为格罗斯曼过于天真了,居然心存幻想,以为苏联当局会允许他出版《生活与命运》。利普金和扎波罗茨卡亚就持这种观点。根据他们的说法,格罗斯曼之所以同意把这本小说多备一份手稿,全因他们的坚持。[28]但是,诗人科尔涅伊·楚科夫斯基(Kornei Chukovsky)在1960年12月27日那天的日记里这样写道:“格罗斯曼接到赫鲁晓夫秘书打来的电话,说这本小说太好了,正是目前所需要的,说他要把自己的读后感告诉赫鲁晓夫。”这是传闻,不知是真是假。即便没来电话,楚科夫斯基对此事的重视,这就很不一般。[29]

我个人并不觉得格罗斯曼天真。不论是人的心理活动,还是苏联政权的内部运作,显然他都是非常熟悉的。1956年赫鲁晓夫公开谴责斯大林。从那时起,政治形势一直在迅速演变。今天回过头来评说当时的政治形势,事后聪明,肯定不费吹灰之力。艺术批评家伊格尔·格隆斯托克(Igor Golomstock)跟我讲过,当时很多有头脑的人期望值都很高,他们深刻批判苏联政权,但他们都像格罗斯曼一样,一辈子都是在苏联体制内度过的。利普金说得很明白,格罗斯曼知道自己有被捕的可能;我的看法是这样的:格罗斯曼当时有可能只是厌倦了搪塞支吾,当局今天要求这样,明天要求那样,他厌倦了,不想再跟着它的指挥棒转了。他没料到,这回和往常不一样,没逮捕他本人,却把他的小说逮捕了。他把这本书的手稿在廖丽亚·多米尼吉娜那儿也存了一份。[30]不过,为慎重起见,他连利普金都没告诉,以防万一。

格罗斯曼不断要求出版他的小说。隔了一阵子,赫鲁晓夫和勃列日涅夫当政年代主管意识形态的一把手苏斯洛夫召见了他。苏斯洛夫把早就对格罗斯曼说过的话又重复了一遍:这本小说,两三百年内都休想出版。正如讽刺作家弗拉基米尔·沃伊诺维奇(Vladimir Voinovich)曾指出的,比苏斯洛夫的傲慢更令人惊奇的,是他居然很识货,一眼就看出这本小说持久的重要性。[31]

格罗斯曼担心这本小说会就此付之东流,心情非常抑郁。用谢苗·利普金的话说:“格罗斯曼在我们眼前一天天老下去。他那卷曲的头发变了样,白发比以前更多了,有点儿谢顶。哮喘病……又犯了,走起路来趿趿拉拉。”[32]用格罗斯曼自己的话说:“他们在一个黑暗的角落,掐死了我。”[33]但是,格罗斯曼并没有就此歇笔。他写了一篇生动的亚美尼亚游记《愿你和平》,紧接着又完成了《一切都在流动》,这本书批判苏联社会,笔锋比《生活与命运》还要犀利。它一半是小说,一半是沉思,书中有对苏联劳改营的简要研究,关于1930年代大恐怖/大饥荒令人动容的描写,对列宁慷慨激昂的抨击,还有对俄罗斯“奴隶的灵魂”的深刻反思(至今还令俄罗斯民族主义者激愤不已)。可是这个时候格罗斯曼已经罹患胃癌。1964年9月14日晚间,别尔基切夫犹太人大屠杀二十三周年纪念日前夕,格罗斯曼与世长辞了。[34]

* * *

在结构上,《生活与命运》和《战争与和平》差不多:聚焦一个家庭,家族成员各有各的故事,这些故事合在一起,全国的大千世界就一览无余了。亚历山德拉·弗拉基米罗芙娜·沙波什尼科娃是一位精神思想扎根于革命前知识分子民粹主义传统的老太太。她的子女以及子女的家人是这本小说的中心人物。书中有两个次要情节,一个在俄国的劳改营,一个在物理研究所。亚历山德拉·弗拉基米罗芙娜的大女儿叫柳德米拉,这两个情节围绕她的前夫和现任丈夫来写。亚历山德拉·弗拉基米罗芙娜的小女儿叫叶尼娅。书中还有两个次要情节,一个写她的前夫克雷莫夫,一个写她现在的未婚夫诺维科夫。克雷莫夫被逮捕,关进了莫斯科卢比扬卡监狱;斯大林格勒保卫战的时候,诺维科夫指挥坦克集团军,立下汗马功劳,后来鸟尽弓藏,也与当局发生冲突。沙波什尼科夫一家人,亲戚朋友不少,他们又都生出不少故事:有在斯大林格勒发电厂工作的,有在前线当兵的,有在德国集中营里组织暴动的,也有被牲口车运到毒气室处死的。

格罗斯曼曾经写道,斯大林格勒街垒战期间,他只能读一本书,就是《战争与和平》。[35]《生活与命运》这个书名和《战争与和平》相似。他之所以选这个书名,似乎是要挑战读者,把这两本小说比较一番。《生活与命运》经得起这样的比较。托尔斯泰再现了奥斯特利茨战役,格罗斯曼再现了斯大林格勒保卫战,生动的手笔至少不亚于托翁。遭到长时间轰炸是个什么滋味,战时应该有什么“居家”小常识,格罗斯曼也写得非常逼真,例如,书里写到,必须要有一个坚固的地下掩体,这是性命攸关的大事。有一段描写崔可夫将军的地下掩体被摧毁了,结果军官们一个接一个把自己手下的人从掩体里撵了出去,像这样出人意料的有趣段落比比皆是。

书中还描写了斯大林保卫战期间大家不分官阶、一律平等的战友之情,然后笔锋一转,写党的官僚们觉得这种精神比德国人还要凶险,于是要将这种精神根绝。书中描写俄国胜利后斯大林格勒一片悲伤的场景,读来同样感人:战争中全世界都看着斯大林格勒,这座城市当时是“世界名城”,“它的灵魂就是自由”。可是,战役结束以后,它便沦为众多被战火焚毁的城市中的一座了。[36]

也和托尔斯泰一样,格罗斯曼书中采用了与很多人的观点不同的视角:既有普通士兵对身边形势的直接感受,也有史学家、哲学家高远的展望。格罗斯曼全局性的思考比托尔斯泰更有看头,也更多样化;有些想法简练隽永。克雷莫夫在被捕前夕终于明白,无辜战友被捕时自己没有站出来说话,不光是因为害怕:正是“革命的目的以道德的名义摆脱了道德”。[37]克雷莫夫被捕后,他的思想迸发出诗的力量:“从革命的活的机体上把皮撕下来,新时期想用革命的皮来打扮自己,而把无产阶级革命的带血的肌肉和热腾腾的心肝抛进垃圾堆里,因为新时期不需要这些。需要的只是革命的皮,所以把这张皮从活人身上剥下来。披上革命的皮的人便说起革命的话,做起革命的动作,但是脑子、肺、肝、眼睛却是另外一种人的。”[38]

格罗斯曼的反思的力量,并非来自形象的描写,而是来自严谨的逻辑,经过深思熟虑之后慢慢道来。全书从头到尾贯穿着一个非同寻常的观点:极权国家运作的机理和现代物理学一样,都着眼于概率,不关心因果关系;看的是巨大的总量,而非单个的人或粒子。有时候,他把逻辑寓于诗情之中;在斯大林格勒,斯大林从希特勒手里一把夺过反犹主义这把剑,这个夺剑的形象是个画龙点睛的收尾之笔,点明了纳粹主义和斯大林主义本质上是一回事。

格罗斯曼在一篇假借书中人物伊康尼科夫谈论“愚蠢的善举”的文章中最为直截地表达了他的观念。伊康尼科夫以前是托尔斯泰的信徒,不久前亲眼看见20000名犹太人惨遭屠杀。[39]每当听到诸如创造世界新秩序这话,我们最好回想一下这篇文章里的某些想法:

哪里有善的曙光升起—这种善是永恒的,并且永远不会被恶所战胜,当然那种恶本身也是永恒的,也永远胜不过善—哪里就会流血,就会有大批儿童和老人死于非命。不但是人,就连上帝也无法消除现实的恶。[40]

看样子,只有个人才能保住这颗种子令它存活,只有未被国家意识形态征用的语言才能讲到这颗种子。德国人命令伊康尼科夫去修建毒气室,他拒不从命,此举实际上是将他自己置于死地。在此之前,他找到一位意大利神父,用一种令人难忘的混杂着意大利语、法语、德语的大杂烩语言问了一个深奥的问题:“Que dois-je faire, mio padre, nous travaillons dans una Vernichtungslager.”(“咱们在建毒气工厂了。神甫,我该怎么办?”)[41]有人说,格罗斯曼的文笔有点儿笨重,典型的苏式风格;更确切的说法,应该是格罗斯曼能写出各种各样诗一般的语言,有伊康尼科夫笨拙、破碎的语言,也有克雷莫夫自我谴责时那种雄辩的语言,但是他不太相信为诗而诗,所以,只有在平常语言不足以表情达意的时候,他才写诗意的语言。

或许只在一个方面,格罗斯曼不如托尔斯泰:他没有托尔斯泰那样高超的再现鲜活而完整的生命的能力。托尔斯泰刻画的年轻的娜塔莎·罗斯托娃那种形象,《生活与命运》里面是找不到的。但是,格罗斯曼描写的是欧洲史上最黑暗的时代之一,所以尽管最后一章歌颂明媚的春光,写到耀眼的阳光照在冰雪上,别廖兹金(Byerozkin)和他的妻子“从亮光中穿过,就好像从密密的树丛中穿过”,但这部小说的整体色调是阴郁的,大多数的陪衬情节都以主要人物的死亡作结,有时候死去的还不止一人。不过,格罗斯曼并不是没有爱、没有信仰、没有希望。在他的信念里甚至含有一种坚强的、清醒的乐观精神,他坚信,即使身陷苏联或纳粹的集中营,也并非不可能坚守道义,仁慈待人。格罗斯曼能够细腻地理解人的过错、人的疑虑、人的表里不一,理解道义选择是痛苦的、复杂的,这种理解给予他的作品非凡的价值。

这种对于道德的微妙的理解,是让我们将格罗斯曼与另一位作家—契诃夫—联系起来的诸多特质之一,尽管二人在写作篇幅上大不相同。《生活与命运》有很多章节,单个拿出来与契诃夫的短篇小说惊人地相似。阿巴尔丘克和一个朋友争论不休,不料几小时后这个朋友被一个罪犯杀害。阿巴尔丘克把罪犯的名字告诉了劳改营当局,这样做相当于自寻死路。他觉得做一个堂堂君子是立身之本,告发凶手更让他自觉义薄云天。底气一足,对死去朋友的怒气更大了,想好好教训教训他。读者一方面赞赏阿巴尔丘克的勇敢,一方面厌恶他的自命正直。

书中关于斯大林格勒年轻士兵克里莫夫那一章也颇有契诃夫式的讽刺意味。克里莫夫遇到德军轰炸,迫不得已在一个弹坑里躲了几个小时。以为身边躺着的是一个俄国同志,他突然感到一种他不应有的对于人类温暖的需求。这个杀人有术的侦察员于是握住了那人的手。没想到那人是个德国兵,碰巧也在这个弹坑里躲轰炸。等到轰炸结束,这两个士兵才意识到彼此都弄错人了;他俩默默地爬出了弹坑,各自都害怕被上级看见,说自己通敌……在关于红军驾驶员谢苗诺夫的一章里,格罗斯曼提出了相似的问题,但是说得更含蓄。谢苗诺夫被德国人俘虏,在奄奄一息快要死了的时候,德国人把他给放了。这时候,一个乌克兰农家老太赫里斯佳·丘尼娅克把他接进自己的茅舍,给他喂饭,护理他。[42]过了一个多月,谢苗诺夫恢复了体力,一个邻居来串门,谈着谈着就谈起了农业集体化。他简直不敢相信自己的耳朵,他的救命恩人,“这个舒适的农舍的女主人”[43]曾几何时几乎快要饿死了,当时命悬一线,就像他自己刚住进来的时候一样。而赫里斯佳那天晚上睡觉前,觉得要在胸前画个十字才安心;字里行间看得出,如果她早知道谢苗诺夫是赞成农业集体化的,并且是从莫斯科来的,恐怕不一定会救他的命。仅仅十二年前,正是那些莫斯科来的苏共党员、共青团员导致她全家人活活饿死的惨剧。她对人善良,似乎和她的认识水平无关;甚至可能正是因为她的缺乏认识。

正好像《生活与命运》可以作为一系列微型画像来看,在格罗斯曼看来,契诃夫的短篇小说,合在一起,也可以作为一部史诗般宏大的作品来读。格罗斯曼塑造的一个人物向契诃夫表达了敬意,他的一番话道出了格罗斯曼自己的希望和观点:

契诃夫使我们认识了整个的俄罗斯,俄罗斯的各个阶级、阶层、各种年龄的人……但是不仅如此。他使我们认识了这平平常常的许多人,明白吗,俄国的平常人!……契诃夫说:让上帝到一边去吧,让所谓伟大的先进思想到一边去吧,首先是人,我们要善良,要关心人,不管什么人,僧侣、庄稼汉、百万巨富的工厂主、萨哈林的苦役犯、饭店的跑堂;首先要尊重人,怜惜人,热爱人,不这样绝对不行。[44]

我们或许可以把《生活与命运》称为契诃夫式的人性史诗。像任何一部伟大的史诗作品一样,这本书偶尔也超出了史诗的框架。在驶向灭绝营的火车上,一个已届中年、没有孩子的医生索菲亚·奥西波芙娜·列文顿 “收养”了小男孩达维德。格罗斯曼不光把自己的生日—12月12日—给了这个孩子,还把自己很多童年的回忆也给了他。当一个德国军官下令内科医生和外科医生走出队列时,索菲亚没站出来,她不肯扔下达维德不管,不肯扔下她有生以来第一次有了认同感的犹太人们,而宁可放弃自己的生命。一大群人被赶进了毒气室,索菲亚和达维德也在这群人里。达维德是先死的,索菲亚感到孩子的身体在她怀里渐渐沉下去。这一章是这样结尾的:

这孩子的身体小得像鸟儿一样,比她先走了一步。

“我做妈妈了。”她想道。

这是她最后一个念头。

可是她的心还活着:心在紧缩,疼痛,在怜惜你们,活着的和死去的人们。索菲亚感到一阵恶心,就把达维德,已经成了尸体的孩子紧紧搂在怀里,她也成了死人,成了尸体。[45]

索菲亚·奥西波芙娜在弥留的时刻第一次感到了母爱的力量。她终于当上了妈妈—可是,她给孩子带来了生命还是带去了死亡?我们不能说:达维德已经死了。达维德/瓦西里还活着—索菲亚一定也还活着,因为她的心不仅怜悯已经死去和正在死去的人们,不仅怜悯她同时代的人,而且也怜悯“你们大家”,也就是说,怜悯我们这些读者。或许她给瓦西里·格罗斯曼,也给一些读者,带来了更充实、更深刻的生命,虽然这生命痛苦照旧。

格罗斯曼曾经给爱伦堡写过一封信谈《黑书》。正像信里所说,他深感为死者说话,“为长眠者发声”[46],是他的道义责任。 但同样重要的,是他感到死者在支撑着他;他相信死者的力量能够帮助他履行为生者尽力的职责。维克托·施特鲁姆的故事,结尾处有一种谨慎的乐观,从中可以清楚看到格罗斯曼这种责任感。施特鲁姆明知那些人是无辜的,可是不昧着良心构陷他们,自己那几个新到手的特权就没了,于是一反常态地在官方的诽谤信上签上了名。施特鲁姆希望他死去的母亲下次会帮助他,让他有所长进;他在小说里的最后一句话是这样说的:“好吧,咱们就试试吧……也许,我还有足够的力量。妈妈,妈妈,这是你的力量。”[47]

格罗斯曼母亲的二十周年忌日那天,他给母亲写了一封信。他的情感在信中表达得更加明白:“亲爱的妈妈,我就是你,只要我活着,你也就活着。我死以后,你还会继续活在这本书里。我把这本书题献给你,书的命运是和你的命运紧紧连在一起的。”[48]他感到母亲就在这本书里活着,这似乎让他觉得《生活与命运》这本书本身就是一个活体生命。[49]他给赫鲁晓夫写了一封信,以一句挑战的话作结:“我花费毕生心血写成的书正在坐牢,那么,我自己的人身自由、我现在的职位都是毫无意义的,都是虚假的。这本书,我写了就不会抛弃,过去不抛弃,现在也不抛弃……请你把自由还给我的书。”[50]

* * *

约翰·加勒德(John Garrard)和他的夫人卡罗尔(Carol Garrard)合写了一本优秀的格罗斯曼传记《别尔基切夫的灵骨》。约翰·加勒德来信说,格罗斯曼有“两个未愈合的伤口”:

第一个伤口是沉默的文化。苏联犹太人的死亡,当地老百姓做了帮凶。在前苏联的领土上,大家至今还保持沉默,绝口不提这件事。有一位美国和平卫队的志愿者被分配到别尔基切夫工作,上个月她给我来信说,她正在寻找犹太人大屠杀的准确地点。她请乌克兰朋友帮忙寻找(她会说乌克兰语),大家却茫然看着她,都矢口否认,说没发生过这样的屠杀,也没有这样的尸坑。第二个伤口与斯大林格勒战役有关。通往著名的“斯大林格勒陵墓”的花岗岩墙上刻着一排大字:“一个德国兵问道:‘他们又向我们进攻了,他们能是普通人么?’”在陵墓的大厅内,一个苏联红军战士的回答用烫金大字刻在了墙上:“是的,我们确实都是普通人,活下来的没有几个,但是为了神圣的俄罗斯母亲,我们都履行了爱国者的责任。”

这些话是从格罗斯曼一篇文章上摘录下来的,该文题目是《在主传动线上》,最初刊登在《红星报》上,后来《真理报》也转载了。但是,这个纪念馆的设计师们并没有注明这两句话的作者是格罗斯曼。纪念馆的导游人员至今仍然在说,他们不知道这个语录的作者是谁。[51]

纪念馆修建期间,格罗斯曼在默默无闻中死去。纪念馆1959年奠基,1967年完工;《生活与命运》1961年被“逮捕”,格罗斯曼于1964年逝世。苏联当局对待格罗斯曼的方式,似乎是将他劈作两半,两个“格罗斯曼”互不相干:一个是持不同政见的犹太人,他的作品必须保持沉默;另一个则代表了“苏联人民的声音”,他的话可以用巨大的字体刻在墙上,只要不提他的名字就好。直到今天,斯大林格勒陵墓始终没有注明作者就是格罗斯曼。格罗斯曼天上有知,对此可能只会耸耸肩;他“为长眠者发声”,如果言者谆谆听者藐藐,才会更令他失望不安。

2006年6月

2010年11月修订

[1]西方见证犹太大屠杀最为知名的作家普里莫·莱维(Primo Levi)终生从事的也是工业化学师的工作。与格罗斯曼一样,莱维也是精确描写和分析的大师。

[2]见谢苗·利普金的《瓦西里·格罗斯曼的斯大林格勒》(Stalingrad Vasiliya Grossmana,阿迪斯出版社,1986),第10页。巴别尔:“用新的眼光发现了我们的犹太首都。”布尔加科夫:“有价值的东西还是能够出版的!”

[3]这个书名取自德文Glück auf,短语的字面意思是“上来,好运!”,原是矿工从井下回到地面上的时候,地面上的人打招呼用语。后延伸为“祝你好运”。

[4]茨维坦·托多罗夫(Tzvetan Todorov),《希望与回忆》(Hope and Memory,伦敦:大西洋出版社,2005),第50页。

[5]谢苗·利普金,《战车》(Kvadriga,莫斯科:Knizhny sad出版社),第516页。

[6]利普金,《战车》,第577页。

[7]1903年,俄国社会民主党第二次全国代表大会期间,该党分裂为两派:布尔什维克派和孟什维克派。1917年布尔什维克政变后,孟什维克大多被捕或流亡。

[8]苏联安全部门多次改名。按时间顺序,最重要的名称和缩写为:契卡(Cheka),国家政治保安总局(OGPU),内务人民委员会(NKVD),国家安全委员会(KGB,即:克格勃)。

[9]关于这个故事更全面的记述,包括格罗斯曼给叶佐夫写的措辞巧妙的信之全文,见约翰·加勒德(John Garrard)和卡罗尔·加勒德(Carol Garrard)合著的格罗斯曼传记《别尔基切夫的灵骨:瓦西里·格罗斯曼的生活与命运》(The Bones of Berdichev: The Life and Fate of Vasily Grossman,自由出版社,1996),第122—125页和第347—348页。

[10]利普金,《战车》,第518页。托多罗夫责备格罗斯曼没有设法为鲍里斯·古贝尔辩护是毫无道理的,哪怕是暗示性地责备也不对,因为格罗斯曼一旦辩护不仅自己会被捕,连奥尔加·米哈伊洛芙娜也得坐牢。

[11]爱伦堡也是战地记者,也是格罗斯曼的竞争者。爱伦堡常常被认为没有原则,但他这次不仅拒绝签署这封信,还给斯大林写信,解释他为什么拒绝签字。《生活与命运》里的施特鲁姆对索科洛夫的感情很矛盾,暗示着格罗斯曼对爱伦堡也有类似的矛盾情感。见乔纳森·布伦特(Jonathan Brent)与弗拉基米尔·瑙莫夫(Vladimir P. Naumov)合著的《斯大林的最后罪恶:阴谋迫害犹太医生,1948—1953》,第300—306页。感谢艾丽丝·纳西莫夫斯基(Alice Nakhimovsky)为我指出这一点(私人通讯)。

[12]关于这一事件更详尽的记述,见瓦西里·格罗斯曼《大路》(The Road,伦敦:麦克尔霍斯出版社,2010),第75—78页。

[13]《生活与命运》,第687页。

[14]利普金,《战车》,第572页。

[15]瓦西里·格罗斯曼,《大路》,第291页。

[16]《最后一封信》(La Dernière Lettre), 根据这封信写成的剧本,剧中人只有一位女士,2000年由弗里德里克·怀斯曼(Frederick Wiseman)在巴黎搬上舞台,后来又改编成电影。2003年怀斯曼在纽约上演了该剧,英文剧名Last Letter。2005年,格罗斯曼百年诞辰之际,莫斯科上演了俄文版。

[17]弗兰克·埃利斯(Frank Ellis),《瓦西里·格罗斯曼:一个俄国异端分子的起源与演变》(Vasily Grossman: The Genesis and Evolution of a Russian Heretic,牛津/普罗维登斯:伯格出版社,1994),第48页。

[18]瓦西里·格罗斯曼,《参战的作家:瓦西里·格罗斯曼随苏联红军报道:1941—1945》,安东尼·比弗(Anthony Beevor)和卢巴·维诺格拉多娃(Luba Vinogradova)编(伦敦:哈维尔·塞柯出版社,2005),第126页。《光辉大道》是1940年的一部苏联电影名,亚历山德罗夫(Aleksandrov)执导。

[19]格罗斯曼,《参战的作家》,第xiv页。

[20]同上,第62页。

[21]瓦西里·格罗斯曼,《大路》,第68—70页。

[22]《老教师》,首刊于《旗帜报》(1943年,第7期,第8期);《没有犹太人的乌克兰》,首刊于《统一》(Eynikayt,1943年11月25日,12月2日)。

[23]瓦西里·格罗斯曼,《大路》,第144页。

[24]完整的俄文版(至今尚未在俄罗斯出版)分别于1980年在以色列出版,1993年在立陶宛出版。见西蒙·玛吉斯(Simon Markish), 《一位俄国作家的犹太命运》(A Russian Writer’s Jewish Fate),《评论》(Commentary,1986年4月),第42页。

[25]后者早期不完整的版本,由托马斯·惠特尼(Thomas Whitney)译成英文出版,译本差强人意,译名《永远流淌》(Forever Flowing)。格罗斯曼把最后的定本交给了叶卡捷琳娜·扎波罗茨卡亚保存,是一个打字本,中间有手写的插入语。她转赠给了加勒德夫妇,加勒德夫妇又转赠给哈佛大学萨哈罗夫档案馆,现在研究人员可以自由阅读。

[26]1926年5月,苏联国家政治保安总局(OGPU)搜查布尔加科夫的住所,抄走了《狗心》手稿两份,但两年后又还了回来。格罗斯曼总是说,《生活与命运》是被“逮捕”的。其他俄国人说起这件事往往也用“逮捕”这个词。

[27]相比之下,帕斯捷尔纳克曾经把《日瓦戈医生》的手稿拿给朋友们和编辑们看,甚至通过苏联邮政局邮寄。他的罪过不在于写这本小说,而在于拿到国外去出版。

[28]加勒德夫妇,《别尔基切夫的灵骨》,第263—265页。

[29]科尔内·楚科夫斯基(Kornei Chukovsky),《日记:1901—1969》(耶鲁大学出版社,2005),第451页。

[30]这个手稿是在利普金和扎波罗茨卡亚提醒他之前还是之后做备份的,并不清楚。

[31]见《书报审查索引》(Index on Censorship)第5卷(1985),第9—10页。此文根据沃伊诺维奇在1984年“法兰克福书展”上的演讲编译而成。沃伊诺维奇在这次讲话中说,是他在1970年把《生活与命运》偷运到西方的。后来发现这两卷缩微胶卷是在安德烈·萨哈罗夫(Andrey Sakharov)和叶连娜·邦纳(Yelena Bonner)的帮助下制作的。

[32]利普金,《战车》,第582页。

[33]同上,第575页。

[34]9月14日也是格罗斯曼和奥尔加·米哈伊洛芙娜的结婚纪念日。这个日子一定会使格罗斯曼痛苦地想起,由于妻子反感自己的母亲,最后导致母亲悲惨地死去。他的女儿叶卡捷琳娜·科罗特卡娃(Yekaterina Korotkava)告诉我,格罗斯曼死于肺癌,并非外界一直以为的胃癌。

[35]加勒德夫妇,《别尔基切夫的灵骨》,第239页。

[36]《生活与命运》,第818页。

[37]同上,第538页。

[38]同上,第864页。

[39]这是别尔基切夫死难犹太人最初的估计数字。

[40]《生活与命运》,第411页。

[41]同上,第303页。

[42]赫里斯佳·丘尼娅克确有其人,关于格罗斯曼与她的谈话,参见格罗斯曼《参战的作家》,第253页。给格罗斯曼留下深刻印象的人,他往往会写进作品加以纪念。

[43]格罗斯曼诗词片段。

[44]《生活与命运》,第278—279页。

[45]《生活与命运》,第565页。“可是她的心……”这一段的开头改译过。哈丽雅特·穆拉夫(Harriet Murav)的文章《答复大屠杀:博格尔森,格罗斯曼和尼斯特》婉转指出,我这段原先的译文欠佳。感谢她提醒。

[46]加勒德夫妇,《别尔基切夫的灵骨》,第206页。

[47]《生活与命运》,第863页。

[48]瓦西里·格罗斯曼,《大路》,第293页。

[49]参见艾丽丝·纳吉莫夫斯基(Alice Nakhimovsky):“在格罗斯曼自己的作品里,在别人写他的俄语文献中,都屡屡提到这本书是一个活体生命。”(《俄国犹太人的文学与身份》,约翰·霍普金斯大学出版社,1992,第115页)。

[50]费奥多·古贝尔(Fyodor Guber),《记忆与信件》(Pamyat’ I pis’ma,莫斯科:Probel出版社,2007),第102页。

[51]见《欧洲百科全书:1914—2004》中约翰·加勒德写的关于格罗斯曼的文章(斯克里伯纳出版社,2006)。





主要人物表

亚历山德拉·弗拉基米罗芙娜·沙波什尼科娃—老革命家沙波什尼科夫的妻子。有一个儿子,三个女儿。

德米特里(“米佳”)—弗拉基米罗芙娜的儿子,1937年被捕,死于古拉格。

谢廖沙—米佳的儿子,参加斯大林格勒前线战斗。

柳德米拉—弗拉基米罗芙娜的大女儿。

阿巴尔丘克—柳德米拉的前夫,老布尔什维克,被关在古拉格。

阿纳托里(“托里亚”)—柳德米拉与阿巴尔丘克的儿子,参加苏德前线战斗。

维克托·帕夫洛维奇·施特鲁姆—柳德米拉的现任丈夫,苏联国家科学院的物理学家。

娜佳—柳德米拉和维克托的女儿。

玛露霞—弗拉基米罗芙娜的二女儿,斯大林格勒大撤退时死于伏尔加河沉船事故。

斯捷潘·费多罗维奇·斯皮里多诺夫—玛露霞的丈夫,斯大林格勒发电厂的厂长。

薇拉—玛露霞和斯皮里多诺夫的女儿。

维克托罗夫—薇拉的情人,苏军战斗机飞行员。

叶夫根尼娅(“叶尼娅”)—弗拉基米罗芙娜的小女儿。

尼古拉·格里高力耶维奇·克雷莫夫—叶尼娅的前夫,老布尔什维克,红军政委。

诺维科夫—叶尼娅的情人,坦克军军长。


