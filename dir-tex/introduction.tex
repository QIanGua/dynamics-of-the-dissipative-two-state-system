
在物理和化学领域,对于状态被限制在二维希尔伯特空间的量子系统问题是很普遍的。在最简单的例子中,系统只处理一个有两个值的自由度──例如自旋$1/2$原
子核的自旋投影,中性$K$介子的奇异性或者光子的极化。本质上讲我们把这些系统成为二态系统。一个更常见的情形是问题中的系统有连续的自由度$q$,例如几
何坐标,在图一中,它与有两个分立的最小值的势能函数$V(q)$关联。假如势垒高度$V_0$大于$\hbar\omega_0$,这里$\omega_0$与经典分立的小振动频
率$\omega_{+}$和$\omega_{-}$同阶($\omega_{+}$和$\omega_{-}$被假设振幅有相同的量级但不必相同)。除非势能函数$V(q)$的形状很病态,否则很好独立
的量子运动将是经典的,与基态分立的第一激发态(独立的势垒),对左手(右手)将很好近似为$\hbar\omega_{+}(\hbar\omega_{-})$,例如$\hbar\omega_{o}$
(见图一)。此外,如果在两个势垒的基态间偏置(去 谐)$\varepsilon$小于$\omega_0$,那么对于$k_BT<<\hbar\omega_0$(但是$k_BT/\epsilon$是任意的),
我们宣称系统将会被有效限制在二维两个基态展开的希尔伯特空间中。当然尽管我们必须考虑势垒之间隧穿的可能性,对于这个过程典型的矩阵元是
$\hbar\delta_0$,相比$\hbar\omega_0$它是指数级的小,所以隧穿不会这基态的二维希尔伯特空间与系统激发态混合。这种情况的例子包含是有些类型的化学反,
有些晶体的探测运动,二能级系统被认为在很多无定形材料中可以找到。问题中的连续自由度$q$不必是几何的,一个情形是陷在一个 rf SQUID 环中磁流的运动,
额外流偏置接近半个流量子,它在Leggett(1984a)中有详细讨论。(事实上这个例子最初是由目前工作启迪)。简便起见,我们把这种类型,由系统的拓展坐标
描述,视为“截断的”两态系统。

当然,那也存在广为认知相似类型的问题,“基态”希尔伯特空间维度$N$有大于 二的问题。除了明显粒子自旋大于 $1/2$ 的例子之外,有许多“N 态系统问题”
是由于原始复杂问题涉及一个或多个拓展坐标截断产生的。例如,,一个氘-氚 标记的甲基群(CHDT) 在特定种类占据三重简并对称性坐标轴的有机固体的旋转
对应着希尔伯特空间维度为 $3$ 的特定情况。(对于普通的 $CH_{3}$ 群质子 的不可区分性导致了复杂性;例如见 Hewson 1982.)固体中的探测,即使限制
在单胞中也有可能有 $4,6$ 或者 $8$ 种平衡位置。最终或许这类最好的例子 是电子在周期性固体的紧束缚近似:这种情况下的维度是原胞在晶体中的数量。
尽管我们相信我们论文中的很多结果很有可能被拓展到更高维度的情形(参考 Schmid, 1983; Bulgadaev, 1984; Fisher and Zwerger, 1985a;
Guinea et al., 1985a; Weiss and Grabert, 1985),我们将不会在这里具体 考虑它们但应该会将我们注意力聚焦于两态系统。


如果此刻我们将两态系统总体视为脱离它的环境,那么在二维希尔伯特空间中它 的运动可以被简单的哈密顿描述:
\begin{equation}
  \label{eq:1}
  H=- \frac{1}{2}\hbar\Delta_{0}\sigma_{x}+ \frac{1}{2}\varepsilon\sigma_{z}.
\end{equation}
这里 $\sigma_{i}\left(i=1,2,3\right)$ 是泡利矩阵,基底被选为使 $\sigma_{z}$ 的特征值 $+1\left(-1\right)$ 对映系统处局域在右(左)势垒。
量 $\varepsilon$ 是失谐参数,即缺少隧穿时两个势阱局域态的的基态能量之 差 [注意对于 $\omega_{+}\neq\omega_{-}$ 这不仅是势能 $V(q)$ 在最小值处
之差;见 Sec. II]。 我们已经把零点能选为两个两个基态能的平均值。量 $ \frac{1}{2}\hbar\Delta_{0} $ 是两个势垒间的隧穿矩阵元,在 WKB 近似下与
$\omega_{0}$ 相比是指数级的小。

很明显哈密顿量 Eq.(\ref{eq:1}) 完全与自旋 $1/2$ 的磁场中粒子的哈密顿量
$H=-\varepsilon\hat{\sigma}_{z}+\hbar\Delta_{0}\hat{x}$ 等价,并且通过近似
旋转坐标轴它可以被简 单地对角化,本征值为
$\pm(\varepsilon^{2}+(\hbar\Delta_{0})^{2})^{1/2}$ 。但是在大多数实际
感兴趣的情况中这样做是不方便的,因为这个量是直接易受实验测量影响的,通
常是“坐标” $q$,即两态系统中近似的 $\sigma_{z}$。很明显$\sigma_{z}$的一
般动力学只对速率 $\hbar\Delta_{0}/\varepsilon$ 敏感。如果这速率很小,
那么$\hat{H}$的本征态近似为 $\sigma_{z}$的本征态,例如,它们对映系统近
似局域在一个或另一个势阱中的态。如果另一方面
$\hbar\Delta_{0}/\varepsilon$ 很大的话,那么本征态近似在坐标 $q$ 上非
局域;特别来讲,对于 $\varepsilon=0$ 它们对应于著名的偶宇称和奇宇称的
组合
\begin{equation}
  \label{eq:2}
  \psi_{E}= \frac{1}{\sqrt{2}}(\psi_{R}+\psi_{L}),\psi_{0}=\frac{1}{\sqrt{2}}(\psi_{R}-\psi_{L}),
\end{equation}
这里 $\psi_{R}$ 和 $\psi_{L}$ 近似(在$\Delta_{0}/\omega_{0}$阶上)对
应于在右侧或左侧势阱中分立的基态。能级劈裂 $E_{0}-E_{E}$ 是
$\hbar\Delta_{0}$. 在这种情况下,$\sigma_{z}$ 的动力学显示出特殊的振荡
效果:事实上如果 $P(t)$ 是 $P_{R}-P_{L}$,这里 $P_{R}(P_{L})$ 是在右
(左)侧势阱中找到系统的概率,那么如果 $P(0)=1$,那么我们自然发现行为
\begin{equation}
  \label{eq:3}
  P(t)=\cos(\Delta_{0}t),
\end{equation}
这行为惊人地显示了处在左侧或右侧势阱振幅的相位干涉的结果,它没有经典的
相似对应。或许这行为可被观察到的最好的例子是中性 $K-\text{介子}$ 束的
“奇异数振荡”(例如见 Perkins,1972),一个情形二维希尔伯特空间对应非几何自
由度(称为奇异数)。形如 Eq.(\ref{eq:3}) 振荡发生在两个空间分离
的势阱之间的最著名情形是 $NH_{3}$ (氨)分子的“共振反转” (例如见
Feynman et al.,1965);但是如果振荡的情形直接被观察到是罕见的那就没有
什么价值,但是可以从光谱数据中推测出来它的发生。

这项工作的一个主要动力是希望存在 (\ref{eq:3}) 形式的振荡可以在一个体系中被直接观察到,此体系两态问题的 $\Psi_{L}$ 和 $\Psi_{R}$, 不仅对应一个拓展坐标
的不同值(例如文献 SQUID中的流;见 Legget,1984a),而且对应于一些合理边界微观上的区分。这类实验如果可行的话,将会给这些冲突在微观层次提供相当的线索,在量子
力学框架与 “常识” 之间(见 Legget 和 Garg,1985)。

在许多实际兴趣方面,包括与之上相关的目标,问题中两态,$\Psi_{L}$ 和 $\Psi_{R}$, 与一些对称操作联系在一起,并且哈密顿量与这些操作对易,至少在一阶近似上。因此
对于这些情形,$\varepsilon=0$。此种情况的一些例子列在表I中,相关的对称性以及实际中可能会打破它们的效应。这里应该强调一下我们只关心与“c-数”(和时间上的常量)相关
的效应;在下面可以看到,在大多数实际兴趣情形中由于与量子环境联系导致的涨落对称性破缺更为重要。事实上,在很多实际兴趣情形 c-数 对称性破缺效应完全可以忽略。基于
这个原因,本文把大多终点放在 $\varepsilon=0$ 的情况,其中我们应该可以见到其展现出的丰富多样行为。

