
在物理和化学领域,对于状态被限制在二维希尔伯特空间的量子系统问题是很普遍的。在最简单的例子中,系统只处理一个有两个值的自由度──例如自旋$1/2$原子核的自旋投影,中性$K$介子
的奇异性或者光子的极化。本质上讲我们把这些系统成为二态系统。一个更常见的情形是问题中的系统有连续的自由度$q$,例如几何坐标,在图一中,它与有两个分立的最小值的势能函数
$V(q)$关联。假如势垒高度$V_0$大于$\hbar\omega_0$,这里$\omega_0$与经典分立的小振动频率$\omega_{+}$和$\omega_{-}$同阶($\omega_{+}$和$\omega_{-}$被假设振幅
有相同的量级但不必相同)。除非势能函数$V(q)$的形状很病态,否则很好独立的量子运动将是半经典的,与基态分立的第一激发态(独立的势垒),对左手(右手)将很好近似为
$\hbar\omega_{+}(\hbar\omega_{-})$,例如$\hbar\omega_{o}$(见图一)。此外,如果在两个势垒的基态间偏置(去谐)$\varepsilon$小于$\omega_0$,那么对于
$k_BT<<\hbar\omega_0$(但是$k_BT/\epsilon$是任意的),我们宣称系统将会被有效限制在二维两个基态展开的希尔伯特空间中。当然尽管我们必须考虑势垒之间隧穿的可能性,对于
这个过程典型的矩阵元是$\hbar\delta_0$,相比$\hbar\omega_0$它是指数级的小,所以隧穿不会这基态德尔二维希尔伯特空间与系统激发态混合。这种情况的例子包含是有些类型的化学
反应,有些晶体的探测运动,二能级系统被认为在很多无定形材料中可以找到。问题中的连续自由度$q$不必是几何的,一个情形是陷在一个 rf SQUID 环中磁流的运动,额外流偏置接近半个
流量子,它在 Leggett(1984a)中有详细讨论。(事实上这个例子最初是由目前工作启迪)。简便起见,我们把这种类型,由系统的拓展坐标描述,视为“截断的”两态系统。

