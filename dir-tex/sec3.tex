
\section{ 第三部}


一

在斯大林格勒进攻战开始之前几天,克雷莫夫来到第六十四集团军的地下指挥所。军委委员阿勃拉莫夫的副官坐在写字台前就着鸡汤吃饼子。副官放下调羹,叹了一口气,从这口气可以听出来,鸡汤滋味太美了。克雷莫夫的眼睛都湿了,他忽然极其强烈地希望就着白菜汤吃一块饼子。

在布幔后面,副官禀报过以后,就没有声音了。过了一会儿,克雷莫夫听到他已经熟悉的嗄哑的声音,不过这一次那声音不高,克雷莫夫听不清说的是什么。

副官走出来,说:

“军委委员不能接见您。”

克雷莫夫惊讶地说:

“我没有要求接见。是阿勃拉莫夫同志叫我来的。”

副官看着鸡汤,没有作声。

“这么说,是改变主意了?我真不明白。”克雷莫夫说。

克雷莫夫出了地下指挥所,顺着一条干沟朝伏尔加岸边走去,军队报纸的编辑部在那儿。

他走着,因为这次莫名其妙的召唤,因为自己见到别人吃饼子就眼馋,心里十分懊恼,一面倾听着库波罗斯山沟那边传来的零乱的、懒洋洋的炮声。

有一位头戴军帽、身穿军大衣的姑娘朝作战科走去。克雷莫夫朝她打量了一眼,在心里说:“真漂亮!”

他的心又因为习惯的惆怅感紧紧收缩起来,他想起叶尼娅。他又同样习惯地吆喝自己:“追上她,追上去!”又回想起在哥萨克小镇上那一夜,想起那个年轻的哥萨克女子。

后来他想起斯皮里多诺夫:“是一个很好的人,不过他当然不是斯宾诺莎。”

这些念头、懒洋洋的炮声、对阿勃拉莫夫的恼火、秋日的天空,在他的脑海里清清楚楚地回旋了很久。有一名军大衣上戴有绿色大尉领章的司令部工作人员,从指挥所赶来,把他喊住。

克雷莫夫大惑不解地朝他看了看。

“上这儿来,这儿来,请吧。”大尉用手指着一座小屋的门,低声说。

克雷莫夫经过一道岗哨,朝门口走去。他们走进屋里。屋里有一张办公桌,在板墙上用图钉钉着斯大林肖像。克雷莫夫以为大尉找他有事,大概要说:“对不起,营政委同志,您能不能把我们的报告带到左岸,交给托谢耶夫同志?”但是大尉没有这样说。他说的是:

“把您的武器和身份证交出来。”

于是克雷莫夫十分慌乱地说了已经毫无意义的话:

“您有什么权力这样对待我?您想看我的身份证,先把您的身份证给我看看。”

后来,等他相信了这毫无来由、毫无道理但又毫无疑问的事,他就说了类似的情况下成千上万的人在他之说过的话:

“这真荒唐,我简直一点儿也不懂,莫名其妙。”

不过,这已经不是自由的人说的话了。

二

“你别装糊涂。你说,你在被围困期间干了些什么?”

他在伏尔加河左岸,在方面军司令部特别科受到审讯。油漆地板、窗台上的花盆、墙上的挂钟似乎都散发着小地方的宁静气氛。右岸显然有飞机在轰炸;从斯大林格勒方面传来的轰隆声和玻璃颤动声显得似乎又熟悉又亲切。

和自命不凡、嘴唇灰白的侦讯员一起坐在吃饭的桌子旁边的是一个粗野的中校,不知为什么他还没有发作。

可是你瞧,这个肩膀在石灰炉壁上蹭着石灰印子的中校走了过来,走到这个坐在凳子上、当年指导过东方殖民国家工人运动的人,这个身穿军服、佩带政委金星的人,这个生来善良和蔼的人跟前,照他的脸上狠狠打了一拳。

克雷莫夫用手摸了摸嘴巴和鼻子,朝自己的手上看了看,看到手上又是血又是唾液。然后他动了动嘴巴。舌头发僵,嘴唇也麻木了。他看了看刚刚擦洗过的油漆地板,便把血吞咽下去。

深夜,他痛恨起特别科的人。但是起初他既不觉得恨,又不觉得疼。一拳打在脸上,把他的精神打垮了,除了麻木和发僵以外,什么感觉也没有。

克雷莫夫回头看了看哨兵,觉得很不好意思。红军士兵看到一个共产党员挨打!打的是共产党员克雷莫夫,是当着小伙子的面打的,克雷莫夫所参加的伟大革命就是为了这些小伙子。

那个中校看了看表。已经是科长级食堂开晚饭的时间。克雷莫夫被押着在又是灰土又是雪粒的院子里走着,朝着原木搭成的囚室走去。这时候,从斯大林格勒方面传来的空袭的轰隆声特别清楚。在麻木过去之后,他的第一个念头是,德国人的炸弹可以把这小小的囚室炸毁……这个念头又简单又丑恶。

在原木作墙的闷人的囚室里,他感到又绝望,又愤怒,再也控制不住自己。当年是他用嗄哑的嗓门儿叫喊着,向飞机奔去,迎接自己的好朋友季米特洛夫同志;是他抬过蔡特金同志的棺材;现在也是他像个小偷一样看着,特别科人员是不是要打他。是他从重围中把许多人带出来,他们都称他“政委同志”。现在是一个拿枪的农村小伙子用厌恶的目光看着他,看着他这个在审讯中被另外一个共产党员打得满脸是血的共产党员……

他还不能理解“失去自由”这句话的全部意义。但他已经成为另外一种生物,他的一切都应当改变,因为他已经失去自由。

他的眼前发黑……他要去找谢尔巴科夫,去找党中央,他还可以去找莫洛托夫,不把这个坏蛋中校枪毙,决不罢休。你们打电话吧!就打电话给克拉辛吧。要知道,斯大林都听说过我,知道我的名字。斯大林同志有一次问日丹诺夫同志:“这是哪一个克雷莫夫,是在共产国际工作过的那个克雷莫夫吗?”

可是克雷莫夫马上就觉得脚下是深深的泥潭,他就要陷进又黑、又黏、又稠的无底泥潭中……有一种不可抗拒的、比德国的装甲部队更厉害的力量向他扑来。他失去了自由。

叶尼娅!叶尼娅!你看见我吗?叶尼娅!瞧瞧我吧,我遭殃了!我太孤单了,没有人理睬我了,你也不睬我了。

一个坏蛋打了他。他神志模糊,气得手指头都打哆嗦,真想朝特别科的坏蛋扑过去。他过去对宪兵、对孟什维克、对他审讯过的党卫军军官都没有这样痛恨过。

在打他的人身上,克雷莫夫人看到的不是敌人,而是他自己,克雷莫夫,也就是当年那个看到共产党宣言上那句激动人心的“全世界无产者,联合起来!”兴奋得流泪的孩子。这种相近的感觉才真正可怕。

三

天色渐渐黑了。有时这狭小囚室的难闻空气中充满斯大林格勒激战的隆隆声。也许是德国人在攻打着保卫正义事业的巴秋克和罗季姆采夫的部队。

过道里偶尔有走动声。大囚室的门不时打开。那里住的是逃兵、叛徒、趁火打劫的人、强奸犯。他们常常要求上厕所,看守的士兵在开门之前,总要和他们争吵老半天。

把克雷莫夫从斯大林格勒的河边押来的时候,让他在大囚室里待了一阵子。谁也没有注意这位袖子上还带有红星的政委。他们关心的只是有没有带纸,好让他们卷烟卷儿。这些人所想的只是吃,抽烟,满足身体需要。

是谁,是谁控告他?多么痛心啊,知道自己无罪,同时却又觉得犯了弥天大罪,吓得浑身发冷。罗季姆采夫的管道,“6—1”号楼的瓦砾,白俄罗斯的沼地,沃龙涅日的冬天,斯大林格勒的渡河—一切幸福的、愉快的事都已成为过眼云烟。

他现在真想上外面去走走,抬起头看看天空。去看看报纸。刮刮胡子。给弟弟写封信。他想喝杯茶。他还要归还他借来的一本书。看看表。洗洗澡。到箱子里去拿一块手帕。可是他什么也不能了。他失去了自由。

过了一会儿,克雷莫夫被押出大囚室,来到过道里,警备队长骂看守的士兵说:

“我对你说得很清楚,你他妈的为什么把他塞到大房间里?哼,你糊里糊涂,想上前线是不是?”

等警备队长一走开,看守的士兵对克雷莫夫发牢骚说:

“经常是这样。单人囚室总不得空闲!他自己说过,要把该枪毙的关在单人囚室里。如果我把您关进去,该把他关到哪儿去?”

一会儿克雷莫夫就看到几名士兵从单人囚室里押出一名判处枪决的犯人。犯人那一头淡黄色的头发贴在凹进去的狭窄的后脑上。他可能有二十岁,至多二十五岁。

克雷莫夫被带进空出来的单人囚室。他在幽暗中依稀看到小桌子上有一只饭盒,还摸到旁边有一只用面包瓤捏成的小兔子。看样子,这是犯人刚刚捏成的:面包还是软和的,只有兔子的两只耳朵有点儿硬了。

渐渐静下来……克雷莫夫半张着嘴,坐在铺上,睡也睡不着:需要考虑的事情太多了。但是被打昏了的头不能思考,鬓角疼得厉害。头脑里一阵阵长浪,在旋转,奔腾,震荡,想镇定也镇定不了,想什么都想不成。

夜里过道里又有嚷嚷声。值班的士兵在呼唤领班的班长。靴子的踢跶声。克雷莫夫听出警备队长在说话:

“把他妈的那个营政委带出来,让他在警卫室里坐一会儿。”

又补充说:

“重大事故就是重大事故,上级早晚会知道的。”

单人囚室的门开了,一名士兵喊道:

“出来!”

克雷莫夫走了出来。过道里站着一个光着脚、只穿着衬裤的人。

克雷莫夫这一生见过很多可怕的东西,但是他一看到这张脸就觉得,比这张脸更可怕的东西他从来没有见过。这张脸很小,带有肮脏的黄斑。一张脸在可怜地哭着,那皱纹、哆哆嗦嗦的腮和嘴唇都在哭,只有眼睛没哭。不过最好别看那双可怕的眼睛,那眼睛的神情也是极其可怕的。

“走吧,走吧。”士兵催促克雷莫夫说。

到了警卫室里,这名士兵对克雷莫夫说了说发生的重大事故。

“警备队长说要送我上前线,实际上在这儿还不如上前线,在这儿人的神经快要错乱了……把一名故意自伤的弟兄拉出去枪毙。他开枪透过一个大面包打伤了自己的右胳膊。把他枪毙了,用土埋上,可是夜里他又活了过来,又回到我们这儿。”

他对克雷莫夫说话,尽可能既不称“您”,也不称“你”。

“他们搞得太马虎了,简直叫人看着可怕。就是宰牲口也不该这样马虎。可是他们干什么都马马虎虎的。土地是冻的,他们只把荒草扒几下,胡乱撒几把土,转身就走。当然啦,他是能爬出来!如果好好儿地把他埋上,他永远也爬不出来。”

克雷莫夫是常常回答问题,扭转人的思想,为人讲解的,现在却大惑不解地向这名士兵问道:

“不过,他怎么又回来了?”

看守的士兵笑了笑。

“还有呢,带他去枪毙的班长说,既然重新为他办手续,就应该发给他口粮,可是总务科长很凶,发起脾气:既然已经枪毙了,还发什么口粮?依我看,这话也对。是班长太马虎,怎么能叫总务科负责任?”

克雷莫夫忽然问道:

“您在战前是干什么的?”

“战前我在国营农场养蜂。”

“清楚了。”克雷莫夫这样说,因为周围和他的头脑里的一切都糊里糊涂,很不清楚。

黎明时候,又把克雷莫夫押回单人囚室。用面包瓤子捏的小兔子依然在饭盒旁边。不过这会儿小兔儿已经硬了,不软和了。大囚室里传出恳求的声音:

“看守,行行好,带我去解解手吧!”

这时候,草原上升起棕红色的太阳。好像是一个上了冻又沾满泥土的甜菜疙瘩爬到了天上。

不久就把克雷莫夫押上一辆吨半汽车,负责押送的一名和善的中尉就和克雷莫夫坐在一起。司务长把克雷莫夫的提箱交给他。吨半汽车就咯吱咯吱地在冻实的阿赫图巴河边的泥块上蹦跳着,朝列宁斯克的飞机场开去。

他呼吸着潮湿的冷气,他满怀信心和希望—可怕的噩梦似乎已经结束了。

四

克雷莫夫走出小汽车,把灰色的卢比扬卡峡谷打量了一遍。因为长时间的飞机马达声,因为眼前不停地闪过一片片收割完毕和尚未收割的田野、一条条小河、一片片树林,因为心中交替地闪过失望、信心、灰心,这会儿头脑里在轰轰作响。

门开了。他进入窒息人的官气和疯狂的官场严密统治的世界,进入一种生活,这种生活在战争之外,与战争无关,又在战争之上。

在一个闷人的空房间里,在探照灯似的明亮的灯光下,叫他脱光了衣服。在一个若有所思、穿白大褂的人摸他的身体的时候,他打着哆嗦想道,战争的沉雷和钢铁都没有打乱这不知羞耻的手指头一丝不苟的动作。

他想起一名死去的红军战士,在防毒面具里留下进攻前写好的字条儿:“我是为幸福的苏联生活死的,家里还有老婆和五个孩子。”被烧死的坦克手,浑身黑糊糊的,一缕缕头发粘在年轻的头上;成千上万人民的军队,穿过森林和沼地,开炮,打机关枪……

那手指头还在摸着,又镇定,又平静,可是政委克雷莫夫还在炮火下呼喊过:“怎么,格涅拉洛夫同志,您不想保卫苏维埃祖国!”

“转过身去,弯下腰,两脚分开。”

然后,他穿起衣服照相,敞着领口照,板着面孔照,带着表情照,从正面照,从侧面照。然后,他在心里狠狠地骂着娘,在一张纸上盖了手印儿。然后一名忙忙碌碌的工作人员把他裤子上的纽扣剪下来,又拿走他的腰带。

然后他乘着灯光明亮的电梯上去,顺着铺了地毯的长长的、空荡荡的走廊朝前走去,经过一个个带圆孔的门。外科诊所病房。癌外科诊疗室。空气是暖和的,是带有官气的,被电灯照得通亮。这是诊断社会病的X光研究所……

“究竟是谁把我关进来的?”

在这窒闷、不通风的空气中很难思考什么。梦、清醒、过去、未来全都搅在一起。他失去了自我感觉……我是不是有过妈妈?也许,我从来没有妈妈。叶尼娅也是可有可无的了。松树顶上的星星,抢渡顿河,德国人的绿色照明弹,“全世界无产者,联合起来”,每一个门里面都有人,我要死得像个共产党员,莫斯托夫斯科伊这会儿在哪儿,头轰轰直响,难道是格列科夫朝我开枪,卷发的格里高力·叶甫谢耶维奇,共产国际主席,在这走廊上走,多么难闻、多么闷人的空气,多么讨厌的探照灯光……格列科夫朝我开枪,特别科的坏家伙打我一拳,德国人朝我开枪,不知明天我会怎样,我向你们发誓,我什么罪也没有,要说有罪,只有瞎编,好样的老头子在十月革命节在斯皮里多诺夫那儿唱起歌儿,肃反委员会,肃反委员会,肃反委员会,捷尔任斯基当年是这座房子的当家人,亨利·亚戈达,还有明仁斯基,后来就是小个子、绿眼睛的彼得堡无产者叶若夫,现在是又和蔼又精明的贝利亚,当然,当然,我们见过面,我们唱过“起来,饥寒交迫的奴隶……”我什么罪也没有,要说有罪,只有瞎编,难道要把我枪毙?……

在笔直的走廊里走,而生活是乱糟糟的,又是小道,又是山沟、沼地、小河、草原灰土、未收割的庄稼,挤着走,绕着走,当命运笔直的时候,就直着走,走廊,走廊,走廊里有很多门。

克雷莫夫从容不迫地走着,不快也不慢,好像押着他的士兵不在他后面,在他前面。

他一来到卢比扬卡监狱,就产生了一种不同的感觉。

“点的轨迹。”他在按指印儿的时候,这样想道。他不明白自己为什么这样想,虽然正是这个念头表达了他的新的感觉。

所以产生新的感觉,是因为他失去了自己的本来面目。如果他要喝水,会让他喝个够,如果他心脏病发作,突然跌倒在地,也会有医生给他打针抢救。可是他已经不是克雷莫夫,他感觉到这一点,虽然他还不理解这一点。他已经不是原来那个克雷莫夫同志,不能像原来那样穿衣,吃饭,买票看电影,思考,睡觉,总是感觉自己就是自己。克雷莫夫同志本来和所有的人都不同,心灵不同,思想不同,革命前的党龄不同,刊登在《共产国际》杂志上的文章与众不同,各种各样的习惯与众不同,气派与众不同,和共青团员或区委书记、工人、老党员、老朋友、求助者谈话的语调也不同。如今他的身体像人的身体,行动和思维像人的行动和思维,但是克雷莫夫同志作为人的实质、他的尊严、他的自由全消失了。

把他押进一间囚室。囚室长方形,光溜溜的镶木地板,有四张床,铺得平平展展,被子连褶都没有,他顿时感觉出来:三个人用人的好奇的目光看着这第四个人。

他们是人,至于他们是好人还是坏人,他不知道,他们对他敌视还是漠视,他不知道,但是他们对他的好态度、坏态度、冷漠态度,都是人对人的态度。

他坐到给他指定的床上,那三个人坐在床上,膝头放着打开的书本,都一声不响地看着他。他似乎已经失去的美好、可贵的感觉又回来了。

有一个人大块头,宽额头,凸凸的脸,低低的肥厚的额头上面是密密的鬈发,白了的和没有白的,像贝多芬那样蓬乱。

另一个是老头子,两手像纸一样白,光秃的头顶和脸部显得骨骨棱棱的,就好像雕在金属上的浅浮雕,似乎他的血管里流的是雪,不是血。

还有一个和克雷莫夫坐在一张床上,模样很和蔼,因为刚刚摘下眼镜,鼻梁上还带着红红的印子,这人又可怜,又善良。他用手指了指头,微微笑了笑,摇了摇头,克雷莫夫便懂了:看守的士兵在小孔里看着呢,不能说话。

头发蓬乱的人第一个开口说话。

“好吧,”他慵懒然而很和善地说,“我就代表大家欢迎部队来的人。敬爱的同志,您是从哪儿来的?”

克雷莫夫很不好意思地笑了笑,说:

“从斯大林格勒。”

“噢,看到英勇保卫战的参加者,真是高兴。欢迎光临寒舍。”

“您抽烟吗?”白脸老头子很快地问道。

“我抽烟。”克雷莫夫回答说。

老头子点了点头,就低下头看书。

这时和克雷莫夫坐在一起的近视的人说:

“是这样的:我没有给同志们创造方便,我说我不抽烟,就不发给我。”

他问道:

“您离开斯大林格勒很久了吗?”

“今天早晨还在那里。”

“哦……哦……”那个大个子说。“乘飞机来的吗?”

“是的。”克雷莫夫回答说。

“您说说,斯大林格勒怎么样?我们没有订到报纸。”

“您想吃饭,是吗?”和善而近视的人问道。“我们已经吃过晚饭了。”

“我不想吃。”克雷莫夫说。“德国人拿不下斯大林格勒。现在这已经很清楚了。”

“我一直相信这一点。”大个子说。

老头子砰的一声把书合上,向克雷莫夫问道:

“看样子,您是共产党员吧?”

“是的,是党员。”

“小声,小声,只能用小声说话。”和善而近视的人说。

“说到党员身份也要用小声。”大个子说。

克雷莫夫觉得他的面孔很熟悉,他忽然想起这个人:这是莫斯科有名的报幕员。当年克雷莫夫带妻子上圆柱大厅参加音乐会,看到他在舞台上。现在却在这儿见面了。

这时候门开了,看守的士兵往里面看了看,问:

“谁是‘卡’,跟我走!”

大个子回答说:

“我是卡,卡茨涅林鲍肯。”

他站起来,用手指头梳了梳乱蓬蓬的头发,便不慌不忙地朝门口走去。

“这是提审他。”近视的邻床犯人说。

“为什么说‘卡’?”

“这是规矩。前天看守来喊他,就说‘谁是卡茨涅林鲍肯?就叫卡’。真好笑。真怪。”

“是啊,我们都笑了。”老头子说。

“你这个老会计,因为什么也到这儿来啦?”克雷莫夫在心里说。“我也要叫‘克’了。”

犯人们开始睡了,可是强烈的光依然亮着。克雷莫夫觉得有人在小孔里注视着他卷裹脚布,往上提长衬裤,挠胸膛。这是一种专用的灯光,不是为囚室里的人照亮,而是为了能看清他们的活动。如果在黑暗中观察他们更方便的话,就让他们待在黑暗中了。

老会计脸朝墙躺着。克雷莫夫和邻床的近视的人在小声说话,谁也不看谁,而且用手捂着嘴,免得看守的士兵看到他们的嘴巴在动。他们不时地看看旁边空着的床。不知为什么他们在为受审的报幕员担心。近视的人说:

“我们在牢房里都变成兔子了。就像童话里说的,神仙用手一指,人就变成兔子。”

他说起同囚室的人。

老头子也许是社会革命党,也许是社会民主党,也许是孟什维克,他的姓是德列林格。克雷莫夫过去在什么地方听说过这个人。德列林格在监狱、政治隔离室、劳改营里过了二十多年,接近当年莫罗佐夫、诺沃鲁斯基、弗罗连科、菲格纳在施吕瑟尔堡要塞度过的年限。现在把他押回莫斯科,是因为他又作案:他在劳改营里想就农业问题对被划为富农的犯人作报告。

报幕员和德列林格有同样漫长的狱龄。二十多年之前,他开始在肃反委员会捷尔任斯基手下工作,后来又在亚戈达领导的国家政治保安局,在叶若夫领导的内务部,在贝利亚领导的国家安全部工作。他有时在中央机关工作,有时主持大规模的劳改营建设。

克雷莫夫原来也错看了和自己说话的这位鲍戈列耶夫。这位难友原来是一位艺术理论家,古董鉴赏专家,有时还写诗,不过他的诗从来没有发表过,因为不符合时代要求。

鲍戈列耶夫又小声说:

“可是现在,您要知道,什么都完了,完了,我也变成了兔子。”

多么荒唐,多么可怕呀,世界上什么都没有了,只有抢渡布格河、第聂伯河,只有在皮里亚京被围困,只有奥夫鲁奇沼地、马马耶夫冈、“6—1”号楼,只有政治汇报、弹药消耗、政工人员负伤、夜间突击、在战斗中和行军时的政治工作、试射、坦克袭击、火箭炮、总参谋部、重机枪……

在同一世界、同一时间里什么都没有了,只有夜间的审讯、起床号、点名、被押着上厕所、发香烟、搜查、对质、侦讯员、特别会议的决定。

但是这种情形、那种情形都有。

但是为什么他似乎觉得狱友失去自由、住在内部监狱的囚室里是很自然的、不可避免的?而他,克雷莫夫,住在这囚室里、睡在这床铺上就是荒唐的、毫无道理的、不可思议的?

克雷莫夫急不可待地要谈谈自己。他忍不住说:

“我老婆离开我了,没有人给我送东西。”

大个子肃反工作人员“卡”的床铺直到早晨都是空的。

五

战前,克雷莫夫有时从卢比扬卡经过,就猜想这昼夜有人活动的房子里在干些什么。被捕的人在这内部监狱里蹲八个月、一年、一年半:在进行侦讯。然后被捕者的家属就收到劳改营里的来信,于是常常出现一些地名:科米、萨列哈尔德、诺里尔斯克、科特拉斯、马加丹、沃尔库塔、科雷马、库兹涅茨克、克拉斯诺亚尔斯克、卡拉达、纳加耶夫海湾……

但是成千上万的人进入内部监狱之后,就永远没有消息了。检察机关通知家属,说这些人被判剥夺通信权十年。但是在劳改营里根本没有判这种刑的犯人。剥夺通信权十年显然指的是枪决。

有人从劳改营里来信,写道,身体很好,很暖和,如果有可能的话,请寄一些大葱和大蒜去。有人给家属解释说,大葱和大蒜是治坏血病的。至于在侦讯监狱里度过的时间,从来没有人在信里提到。

在一九三七年夏季的夜晚,从卢比扬卡和共青团街经过,是特别可怕的。

闷热的夜晚,一条条街道空荡荡。一座座敞着窗户的楼房黑沉沉的,里面挤满了人,却又像是空旷无人。这种宁静使人毫无宁静感。在遮着白窗帘的明亮的窗户里人影幢幢,在大门口,汽车车门不时地砰砰响着,车灯忽明忽灭。似乎偌大一座城市被卢比扬卡明亮而呆滞的目光封锁住了。脑子里出现了一个一个的熟人。和他们的距离不能以空间来度量,这是用另外的尺度测定的一种距离。天上人间没有一种力量能够越过这一深渊,这深渊等于死的深渊。不过,不是在土里,不是在棺材里,而是在这儿,人还活着,在呼吸,在思考,在哭,没有死。

汽车送来一批又一批被捕的人,成百、成千、成万的人在内部监狱里,在布特尔监狱、列福尔托夫监狱里消失了。

一批批新的工作人员进入区委、人民委员会、军事部门、检察机关、公司、医院、工厂管委会、基层工会、工厂工会、土地管理处、细菌实验室、模范剧院院部、飞机设计院、设计巨型化学与金属产品的研究所,代替被捕的人。

有时候,来接替人民敌人、恐怖分子、破坏分子的人转眼间就成了敌人、异己分子,也被逮捕了。有时又一批接替的人也是敌人,也被逮捕。

有一位列宁格勒的同志悄悄地对克雷莫夫说过,他曾经和列宁格勒同一个区党委的三位书记住在一个囚室里。每一个新上任的书记都揭发过自己的前任,说他是敌人和恐怖分子。在囚室里他们睡在一起,谁也不恨谁。

当年叶尼娅的哥哥米佳·沙波什尼科夫进过这座楼房。腋下夹着一个白色的小包袱,是妻子给他收拾的,有毛巾、肥皂、两套衬衣、牙刷、袜子、三块手帕。他走进这楼房的时候,在脑子里还记着党证上的五位数字、自己在巴黎商务代办处的办公桌、国际车厢,还记着在国际车厢里和妻子明确关系的情景、喝矿泉水和懒洋洋地翻看《金驴记》的情景。

当然,米佳没有任何罪行。可还是把米佳关进来了。克雷莫夫倒是没有被关过。当年柳德米拉的第一个丈夫阿巴尔丘克就在这条灯光明亮、从自由通向不自由的走廊里走过。阿巴尔丘克在前去受审的时候,急不可待地想解开莫名其妙的疑团……可是过了五个月、七个月、八个月,阿巴尔丘克写道:“使我第一次产生杀害斯大林同志的念头的,是德国军事间谍机关的一个头头儿,当初是一位地下工作的领导人使我和他认识的……我们谈话是在五一游行之后,在亚乌斯克林荫道上,我答应再过五天给他最后的回答,我们约定了下一次接头的时间、地点……”

在这里面进行的工作是令人吃惊的。实在令人吃惊。要知道,当年高尔察克手下一名军官朝阿巴尔丘克开枪的时候,他连眼睛也不眨一眨。

当然,是他们强迫他写假供词栽诬自己。阿巴尔丘克当然是真正的共产党员,是坚强的、列宁主义的老战士,他什么罪也没有。可是把他逮捕了,他写了供词……克雷莫夫没有被关过,没有被捕过,没有被迫写什么供词。

有关这类事的情况,克雷莫夫听说过。有些情况是有的人悄悄对他说的,说过之后还要叮嘱:

“不过你要记住,这事你如果说了,哪怕对一个人,对老婆、对妈妈说了,我就完了。”

有些情况是另外一些人透露的。有的人喝多了酒,听到别人自以为是的愚蠢说法,很不服气,无意中说出几句不留心的话,接着就不作声了,到第二天好像顺便说说似的,打着呵欠说:

“哦,我昨天好像胡说了一些什么话,不记得吧?好,不记得更好。”

有些情况是朋友们的妻子上劳改营里去看过丈夫之后对他说的。

不过这一切都是传闻,都是瞎说。克雷莫夫从来就没有遇到这类事。

可是,你瞧。现在把他关进来了。无法设想的、荒唐的、没有道理的事就出现了。当年关押孟什维克、社会革命党人、白党分子、神甫、富农代言人的时候,他连一分钟也没有考虑过,这些人失去自由,等待判决,心里是什么滋味。他没有想过他们的妻子、母亲、孩子。

当然,当爆炸的炮弹越来越近,伤害的不是敌人,而是自己人的时候,他已经不那么心安理得了,因为关的不是敌人,而是苏联人,是党员。当然,在把他特别亲近的一些人、他认为是列宁式的布尔什维克的一些同辈人关进来的时候,他是受到震动的,夜里睡不着觉,思考过,斯大林是否有权剥夺人的自由,折磨他们,枪毙他们。他想到他们遭受的苦难,想到他们的妻子和母亲的苦难。因为他们不是富农,不是白党分子,他们是人,是列宁主义的布尔什维克。

不过他还是安慰自己:不管怎样,他克雷莫夫还没有被关过、被流放过嘛,他还没有写过什么供词,没有被迫招认过什么罪状。可是,你瞧。现在把他克雷莫夫,把列宁主义的布尔什维克关进来了。现在再也无法自我安慰,无法解释,无法说明了。这是事实。

他已经见识了一些情况。牙齿、耳朵、鼻子、光身子的腹股沟都成了搜查的对象。然后是提着剪掉了扣子的裤子和衬裤,又可怜又可笑地在走廊里走,近视的人的眼镜也被没收,他们整天惶惶不安地眯着眼睛,揉搓着眼睛。人进了囚室,便成了实验室里的老鼠,就会产生新的反应,说话声音小小的,上床,起床,大小便,睡觉,做梦,时时刻刻都在观察之下。原来这里的一切是这样残酷,这样荒唐,这样不人道,这样骇人听闻。他第一次明白,在卢比扬卡干的事情这样可怕。要知道,这是在折磨他这个布尔什维克、这个列宁主义者,折磨克雷莫夫同志呀。

六

一天天过去。没有提审克雷莫夫。

他已经知道什么时间吃饭,吃些什么,知道放风的时间和洗澡的时间,知道监狱烟草的烟气、点名的时间,知道图书室里大概有一些什么样的书,认识了一些看守的面孔,常常惶惶不安地等待着同囚室的人被提审归来。被提审次数最多的是卡茨涅林鲍肯。提审鲍戈列耶夫总是在白天。

没有自由的生活!这是疾病。失去自由就等于失去健康。电灯亮着,水龙头里有水,钵子里有菜汤,但是灯光、水、面包都是不同的:是专门供应给你的。有时为了侦讯的需要,可以使犯人一时见不到灯光,吃不到饭,睡不成觉。因为他们得到这一切,不是为了他们本身,这是对待他们的一种工作方法。

瘦得皮包骨的老头子被提审过一次,他回来以后,很神气地说:

“我三个小时不开口,侦讯官先生终于弄清楚了,我的姓确实是德列林格。”

鲍戈列耶夫总是非常和蔼可亲,和同囚室的人说话总是用十分尊敬的口气,常常询问狱友的健康和睡眠情形。有一天,他对克雷莫夫念起诗来,后来他忽然停住,说:

“对不起,您好像不感兴趣呀。”

克雷莫夫笑了笑,说:

“说实在的,我一窍不通。不过我过去看过黑格尔的书,我倒是懂。”

鲍戈列耶夫非常害怕提审。他一听到值班的看守来传他去受审,就惊惶失措。每次受审回来,似乎都瘦了,小了,老了。

他说起对他的审讯,都是前言不搭后语,绕来绕去,而且眯着眼睛。无法理解他的罪名是什么:也许是说他有意谋害斯大林,也许是说他不喜欢用社会主义现实主义精神创作的作品。

有一次大个子肃反工作人员对鲍戈列耶夫说:

“您可以帮助他们制造一条罪状。我劝您这样编造:‘我对一切新事物怀有刻骨的仇恨,凡是获得斯大林奖金的艺术作品,我都不满意。’这样也不过判十年徒刑。尽量不要揭发自己的朋友,揭发朋友并不能保护自己,相反,他们倒是会说您参加什么组织,就会把您关进保密劳改营。”

“您怎么啦,”鲍戈列耶夫说,“他们什么都知道。我能怎么办?”

他常常就他喜欢的话题小声发表议论:我们都是童话中的人物。不论是威风凛凛的师首长、伞兵,不论是马蒂斯、皮萨列夫的高徒,不论是党员、地质学家、肃反工作人员、五年计划的建设者、驾驶员、巨型钢铁产品的制造者,都是童话中的人物。我们本来神气活现,信心十足,可是一跨进这奇异的楼房的大门,魔杖一挥,我们就变成小不点儿,变成小猪崽子、小松鼠。现在我们算什么?不过是小虫儿,不过是蚂蚁蛋儿。

他的见解独到、奇特,显然也很深刻,不过在日常生活方面气量却很狭小,常常担心发给他的东西比别人的少,比别人的坏,担心缩短了放风时间,担心有人在放风时间吃他的东西。

生活中充满各种各样的事件,但生活是空虚的,是虚假的。囚室里的人生存在干涸的河槽里。侦讯员在侦查这河槽、石头、裂缝、高高低低的堤岸。但是当初冲成这河槽的水已经没有了。德列林格很少和人说话,如果说话,大半是和鲍戈列耶夫,显然因为他不是党员。不过他在和鲍戈列耶夫说话的时候,常常发火。

“您是一个怪人,”有一次他说,“第一,您对您瞧不起的人又恭敬又亲热,第二,您天天问我身体怎样,其实我是死是活对于您完全是一样。”

鲍戈列耶夫抬起头看着囚室的天花板,把两手一摊,说:“您听着。”于是拖长声调念道:

“你的甲壳是什么做的,可是龟甲?”

我这样问,得到这样的回答:

“这是我积累的恐惧做成的,

世界上再没有什么比这更结实!”

“这是您写的诗吗?”德列林格问道。

鲍戈列耶夫又把两手一摊,没有回答。

“老头子很害怕,积累了不少恐惧。”卡茨涅林鲍肯说。

吃过早饭以后,德列林格给鲍戈列耶夫看了看一本书的封面,问道:

“您喜欢吗?”

“说实在的,不喜欢。”鲍戈列耶夫说。

德列林格点了点头。

“我也不赞赏这部作品。盖奥尔吉·瓦连季诺维奇说:‘高尔基塑造的母亲形象是圣像,工人阶级不需要圣像。’”

“一代一代的人都在读《母亲》,”克雷莫夫说,“……怎么是圣像?”

德列林格用幼儿园保育员的语调说:

“所有希望奴役工人阶级的人,都需要圣像。比如,在你们共产党的神龛里就有列宁的圣像,也有圣斯大林的圣像。涅克拉索夫不需要圣像。”

似乎不光是他的头顶、额头、手、鼻子是用白骨头旋成的,他的话也当当响,好像是骨头做成的。

“噢呀,真是一个坏家伙。”克雷莫夫在心里说。

鲍戈列耶夫生起气来。克雷莫夫从来没看到这个和蔼可亲、善于隐忍的人这样生气。鲍戈列耶夫说:

“您在对诗的认识方面,只知道有涅克拉索夫,却不知后来又出了布洛克,出了曼德尔施塔姆,出了赫列布尼科夫。”

“曼德尔施塔姆我不了解,”德列林格说,“可是赫列布尼科夫不过是颓废、堕落。”

“去您的吧!”鲍戈列耶夫第一次十分激烈地大声说。“我讨厌透了您那普列汉诺夫的老一套说教。在咱们这房间里,你们是不同派别的马克思主义者,但是有一点是相同的:对诗歌一窍不通,根本不懂得诗是怎么一回事儿。”

说来很奇怪。克雷莫夫一想到,在看守人员的眼里,不论值夜班的、值日班的人员眼里,他这个布尔什维克、这位政委竟和坏老头子德列林格没有任何不同,他就特别不痛快。

所以现在,他这个一向反对象征派、颓废派、一生喜欢涅克拉索夫的人,宁愿在争论中支持鲍戈列耶夫了。

如果皮包骨的老头子说起叶若夫的坏话,他也会信心十足地代为辩护的,会说枪毙布哈林是正确的,妻子不揭发丈夫被流放也是正确的。可怕的判决、可怕的审讯都是正确的。

可是皮包骨的老头子没有说。

这时候一名看守走进来,带德列林格去厕所。

卡茨涅林鲍肯对克雷莫夫说:

“我和他两个人在这房间里过了五天。他一句话也不说。我对他说,两个犹太人,都上了年纪,在卢比扬卡附近的村子里一块儿过了好几个晚上,一句话也不说,实在好笑。不行!他就是不说话!为什么不睬人?他为什么不愿意和我说话?是有血海深仇还是夜里在拉克鲍伊麦拉赫杀了神甫?他要怎样?真是一个老小孩儿。”

“是敌人。”克雷莫夫说。

显然大个子肃反工作人员对德列林格非常感兴趣。

“您要知道,他的罪行很重!”他说。“不可思议!他已经在劳改营里待了很多年,前面还有棺材等着他,可是他毫不在乎。我真羡慕他!来提审他,喊:谁是‘德’?他像树桩一样,就是不作声。直到喊他的姓,他才答应。领导人来到囚室里,打死他,他也不站起来。”

等到德列林格上厕所回来,克雷莫夫对卡茨涅林鲍肯说:

“在历史法庭面前,一切都算不了什么。你我虽然在这里面,还是要痛恨共产主义的敌人。”

德列林格带着好笑和好奇的神气看了看克雷莫夫。

“什么历史法庭,”他没有对着任何人,只是说,“这是历史性的迫害!”

卡茨涅林鲍肯羡慕德列林格的刚强也是枉然。他的刚强已经不是人的刚强。是一种盲目的、非人的狂热用自己的化学热在燃烧空虚而冷漠的心。

俄罗斯的轰轰烈烈的战争、和战争有关的一切大事都很少触动他,他不问前方的战事,也不问斯大林格勒的情形。他不知道新兴的城市,也不知道大力发展的工业。他过的已经不是人的生活,而是在独自下一局没完没了的、抽象的狱中棋。

克雷莫夫倒是对卡茨涅林鲍肯很感兴趣。克雷莫夫感觉出来、看出来,卡茨涅林鲍肯很聪明。他说笑,打诨,瞎扯,但他的眼睛却是深沉的、懒懒的、疲惫的。见过世面、厌倦了人生而不怕死的人的眼睛往往是这样的。

有一次谈起在北冰洋沿岸建筑铁路,他对克雷莫夫说:

“这计划是非常美好的。”

接着又说:

“不过,要实现这一计划,得付出上万人的生命。”

“是有些可怕。”克雷莫夫说。

卡茨涅林鲍肯耸了耸肩膀,说:

“您要是看看劳改队怎样去上工就好啦。全都像死人一般沉默着。头顶上是绿的和蓝的北极光,四周围都是冰雪,黑沉沉的北冰洋在怒吼。在这儿也可以看到强大的力量。”

他劝克雷莫夫说:

“应该帮助侦讯员,他是新干部,很难完成任务……如果帮助他,给他指示,那也是帮助自己,免得一次一次的提审。结果反正一样:专门会议会作出早就作出的决定。”

克雷莫夫正要和他争论,他又说:

“个人清白—是中世纪残余,是神话。托尔斯泰说,世界上没有有罪的人。我们肃反工作人员却得出最严密的结论:世界上没有无罪的人,没有不能判罪的人。逮捕证写的是谁,谁就有罪。在逮捕证上写谁都可以。每个人都可以上逮捕证。给别人写逮捕证写了一辈子的人也可以,摩尔人已经把事情干完,摩尔人可以走了[1]嘛。”

他认识克雷莫夫的很多朋友,有些是在一九三七年经他审讯时认识的。他说起经他审讯的人,既不痛恨,也不抱愧,使人觉得有些奇怪,他说:“这人很有意思,”“真是怪人,”“这人挺讨人喜欢。”

他常常提到法朗士,提到《阿巴纳斯随想录》,喜欢引用巴别尔笔下别尼亚·克里克的话。他说起大剧院的歌舞演员,都亲切地叫他们的名字和父称。他搜集了不少珍本古书。他说了说他在被捕前不久搜集到的一部拉季谢夫文选有多么珍贵。

“要是能把我搜集到的书交给列宁图书馆,那就好了,”他说,“要不然那些浑蛋会让那些书散失了,因为他们不懂书的价值。”

他的妻子是芭蕾舞演员。他担心拉季谢夫文集的命运,显然胜过担心妻子的命运。克雷莫夫说到这个想法,他回答说:

“我的安格琳娜是一个聪明女子,她不会倒霉的。”

似乎他什么都明白,但是什么感情也没有。一些很普通的概念,如离别、磨难、自由、爱情、女人的忠贞、痛苦,他都无法理解。他说起他在肃反委员会工作的头几年,他的声音中出现了兴奋的意味。

“那时候多好呀,那些人多棒呀。”他说。

至于克雷莫夫一生的所作所为,他认为那属于宣传范畴。

他说过斯大林:

“敬佩斯大林,胜过敬佩列宁。他是我真正爱戴的唯一的一个人。”

但是,这个当年参与制定处治反对派首领方案、在贝利亚手下主持北极圈大规模劳改营建设的人,如今在自己原来工作的楼房里,夜间提着剪掉了扣子的裤子前去受审,为什么竟这样心平气和,处之泰然?而孟什维克德列林格用沉默对他表示不满,他却那样不安,那样难受?

有时克雷莫夫自己也怀疑起来。为什么他在给斯大林写信的时候,那样愤怒、冲动,浑身打颤,浑身冒汗。摩尔人已经把事干完,摩尔人可以走了。这事就出在一九三七年,好几万党员,都是像他这样的,甚至比他更好。摩尔人可以走了。为什么他现在对“汇报”这个词儿这样反感?仅仅是因为他坐了牢,正是由于什么人的汇报。过去他常常听取排里政治时事宣传员的政治汇报。那是很平常的事。很平常的汇报。红军士兵里亚鲍什坦贴身戴着十字架,说共产党员是不懂天理的人;里亚鲍什坦进了惩戒连,活了多久呢?红军士兵高尔杰耶夫说他不相信苏联武装力量的强大,认为希特勒一定会胜利;高尔杰耶夫进了惩戒排,活了多久呢?红军士兵马尔凯维奇说:“所有共产党员都是贼,等时候一到,我们用刺刀把他们戳死,人民就自由了。”军事法庭判处马尔凯维奇死刑。都是他汇报的。他还向方面军政治部汇报过格列科夫,如果不是德国的炸弹把格列科夫炸死的话,会当着很多军官的面把他枪毙的。那些被送进惩戒营、被法庭判了刑、在特别科被审讯的人,又是什么感觉呢?

可是在战前,他多次参与办理这一类的案件,心安理得地看待一些朋友的话:

“我在党委说过我和彼得的谈话。”

“他在党的会议上如实地交代了伊万来信的内容。”

“一传讯,他作为一个共产党员,当然应该把一切都说出来,他交代了同志们的思想情况,也交代了瓦洛佳多次来信的内容。”

是的,是的,这些情况都有过。

唉,这又管什么用……所有这些解释,不论是书面的还是口头的,都不能帮助任何人走出监狱。其真正用意只有一点:为的是自己不陷入泥坑,自己摆脱。

克雷莫夫没有很好地维护自己的朋友,实在没有,虽然他不喜欢干这类事情,怕这类事情,千方百计地逃避。他为什么冲动,为什么打颤呢?他希望怎样呢?是希望卢比扬卡的值班看守知道他的孤独?希望侦讯人员同情他被心爱的女子扔掉,在分析案情时要考虑到他夜夜在呼唤她,在咬自己的手,考虑到他母亲还唤他的小名?

夜里克雷莫夫醒来,睁开眼睛,看见德列林格在卡茨涅林鲍肯床前。明亮的电灯光照在老囚犯的背上。鲍戈列耶夫也醒了,用被子盖着腿,坐在床上。

德列林格冲到门口,用皮包骨的拳头擂起门来,用骨头般的声音叫喊起来:

“喂,值班的,快叫医生,犯人心脏病发作啦!”

“别叫,住嘴!”值班看守跑到小孔跟前,喝道。

“怎么能不叫,人要死啦!”克雷莫夫大声叫道。他也从床上跳起来,跑到门口,和德列林格一起用拳头擂起门来。他看到鲍戈列耶夫又在床上躺下来,用被子蒙住头,显然是怕参与这夜晚的特别事件。

一会儿门就开了,走进来好几个人。

卡茨涅林鲍肯昏迷了,他身躯高大,老半天才把他弄到担架上。

早晨,德列林格突然向克雷莫夫问道:

“请问,您这位共产党的政委在前方是不是常常遇到不满的表现?”

克雷莫夫问:

“什么样的不满,对什么不满?”

“我指的是对布尔什维克的集体化政策、对战争的总的领导不满,总之,是指政治上的不满的表现。”

“从来没有。类似的思想表现连影子也没有遇到过。”克雷莫夫说。

“噢,噢,当然,我也是这样想。”德列林格说,并且满意地点了点头。

七

在斯大林格勒城下包围德国人的主张,被认为是十分英明的。

在保卢斯军队两翼秘密集结大量兵力,是袭用原始时代就诞生的原理:当光脚、歪额头、大颌骨的原始人要包围进入洞穴的森林野兽的时候,就是悄悄地在灌木丛中爬的。有什么惊异的呢,是惊异木棒和远程大炮的不同,还是惊异古老武器和新式武器的原理几千年来没有变化?

不过,了解了人类活动的螺旋在不断地向更广和更高的方向增加其螺旋线的同时,却有一个不变的轴,既不必感到失望,也不必感到惊异。

虽然成为斯大林格勒战役关键的包围原理不是新的,斯大林格勒大反攻的组织者们正确地选定了运用这一古老原理的地区,毫无疑问是有功绩的。他们还正确地选定了进行这一战役的时机,很好地训练了军队,巧妙地集结了军队;使三方面军(即西南方面军、顿河方面军、斯大林格勒方面军)很好地配合,也是组织者的功绩;在没有自然条件作掩护的草原地带秘密集结兵力也是很不容易的。南面的部队和北面的部队要从德国人的左肩和右肩擦过,在卡拉奇会合,包围敌人,打碎保卢斯部队的骨头,摘取其心和肺。要花费很多力气制订战役的细节,侦察敌军的火器、兵力、后方、交通线。

不过,最高统帅斯大林、朱可夫、华西列夫斯基元帅、沃罗诺夫、叶廖缅科、罗科索夫斯基和总参的许多有才能的军官参与的这次战役的筹划,其基础仍然是原始人早已运用于战斗实践的两翼包围敌人的原理。

天才的定义只适用于实现了新的思想的人,而且新思想是指核心,不是皮壳;是轴,不是绕轴转的螺旋圈儿。从马其顿王亚历山大时代起,所有战略与战术的拟定,都和这一类的神奇行动毫无共同之处。人的意识震慑于大规模的军事行动,就常常把规模之大和统帅的思想成就之大混为一谈。

战争的历史表明,统帅们在突破防线的战斗中,在追击、迂回、包围战中,运用的并不是新的原理。他们运用的是尼安德特人时代就知道的原理,可以说,这些原理就连那些包围牲口的狼和抵御狼的牲口都知道。

―个能干而认真负责的厂长,一定会保证原料和燃料的及时供应,使各车间保持联系,使工厂生产所需要的几十种大大小小的条件得到满足。

可是,如果历史学家说,是厂长的活动创造了冶金学、电工学和金属的伦琴射线原理,研究工厂史的人的意识就会不赞成:发明伦琴射线的是伦琴,不是我们的厂长……炼铁炉在我们的厂长以前就有了。

真正伟大的科学发明可以使人变得比大自然更聪明。大自然借助这些发明、通过这些发明认识自己。伽利略、牛顿、爱因斯坦在认识空间、时间、物质和力方面所做的事,就属于这样的人类伟大事件。人类通过这些发明,创造了超过自然存在的深度和高度,因此促进了自然界的自我认识并促使自然界更加丰富。

有些已经自然形成的、可以看到、可以感触到的已经存在的原理,只是由人说出来,这是低一级的,是二级发明。鸟飞、鱼游、风滚草和圆石的滚动、风吹得树木摇摇晃晃并且摆动枝叶、海参的喷射运动—这一切都是这种或那种可以感触到的、明显的原理的表现。人类从现象中得出原理,应用于人类环境中,并且根据需要和可能性不断地加以发展。

飞机、涡轮机、喷气式发动机、火箭在生活中是有巨大意义的,人类制造出这些东西应归功于人类的才能,不过并不是天才。

运用人类发现和总结出来的、而不是自然显示的原理做出的发明,属于二级发明,比如在无线电、电视、雷达方面得到运用和发展的电磁场理论原理。释放原子能也属于这样的二级发明。建成第一个核反应堆的费密不应当希求得到人类天才的称号,虽然他的发明已成为世界历史新纪元的开端。

人类借助新的条件,不断地改进人类活动环境中已经存在的东西,比如,在飞行器上安装新的发动机,把轮船上的蒸汽发动机换成电力发动机,又把电力发动机换成原子能发动机,这在发明中属于更低级,属于第三级了。

今天的战争艺术是新的技术条件与旧的原理相配合,人类在这方面的活动,正是属于第三级。否定领导作战的将军的活动在军事上的意义,是不对的。不过,把将军称为天才也是不对的。这样看待一位有才能的指挥生产的工程师,是荒谬的;这样看待一位将军,不仅是荒谬的,而且是有害的,是危险的。

八

两个大锤,每一个都是由几百万吨钢铁和活人血肉铸成的,一南一北,等待着信号。

首先发起进攻的是部署在斯大林格勒西北方的部队。一九四二年十一月十九日上午七时三十分,西南方面军和顿河方面军全线发起了长达八十分钟的强大炮击。炮兵徐进弹幕射击,猛攻罗马尼亚第三集团军盘踞的阵地。

八时五十分,步兵与坦克发起进攻。苏军士气空前高涨。第七十六师在该师管乐队演奏的进行曲乐声中发起冲锋。

下午,敌人防御配系的战术纵深被突破。战斗在广大的地带展开了。

罗马尼亚第四军被击溃了。罗马尼亚第一骑兵师被分割,它与克莱尼亚地区第三集团军其余部队的联系已被切断。

第五坦克集团军从谢拉菲莫维奇西南三十公里的高地上发起进攻,突破罗马尼亚第二军的阵地,很快地向南推进,快到中午的时候,已经占领了佩列拉佐夫以北的高地。苏军的坦克军和骑兵军转向东南方推进,傍晚时候就到达古森卡和卡尔梅科沃,深入罗马尼亚第三集团军后方六十公里。

一昼夜之后,十一月二十日拂晓,集结在斯大林格勒南方加尔梅克草原上的部队发起进攻。

九

诺维科夫在拂晓前很久就醒来了。诺维科夫是那样兴奋,以至于自己感觉不出兴奋了。

“军长同志,您喝茶吗?”维尔什科夫认真又亲热地问道。

“好,”诺维科夫说,“你告诉炊事员,叫他煎几个鸡蛋。”

“上校同志,煎什么样儿的?”

诺维科夫一时没有说话,思索了一会儿,维尔什科夫以为军长在考虑问题,没有听到他的问话。

“煎荷包蛋。”诺维科夫说过,看了看表。“你去看看格特马诺夫起来没有,过半个钟头咱们就要动身了。”

他觉得他没有想,过一个半小时就开始炮火准备,没有想天空就要被几百架强击机和轰炸机闹得轰轰叫起来,没有想工兵就要爬着去剪铁丝网和清除地雷,步兵就要拖着机枪朝着他在炮队镜里观察过多次的雾蒙蒙的山冈奔去。他似乎没有感觉到此时此刻他和别洛夫、马卡罗夫、卡尔波夫的关系。他似乎没有想,昨天在斯大林格勒西北方,苏军坦克进入炮兵和步兵突破的德军防线之后,不停地朝卡拉奇方向推进,再过几个小时,他的坦克就要从南面开去,与北面来的坦克会合,以便包围保卢斯的军队。

他没有想方面军司令部,没有想,明天斯大林也许会在自己的命令中提到诺维科夫的名字。他没有想叶尼娅,没有回忆他在布列斯特跑向机场、天空升起德寇发动的战争的第一道火光的那一天黎明。

但是,他没有想的一切,都在他心中。

他想的是,穿软底的新靴子呢,还是穿皮靴,可不能把烟盒忘了。他想:哼,狗崽子,又给我冷茶。他在吃煎鸡蛋,还掰下一块面包,仔细地揩煎锅上的油。

维尔什科夫报告说:

“您给我的任务完成啦。”

马上又用谴责的语调和信任的口气说:

“我问卫兵:‘他在家吗?’卫兵回答说:‘他能上哪儿去,在跟娘们儿睡觉呢。’”

卫兵说的是比“娘们儿”更难听的词儿,但是维尔什科夫认为,和军长说话不能用这样的词儿。

诺维科夫没有作声,用手指头在扫桌上的面包渣子。

一会儿,格特马诺夫走了进来。

“喝茶吗?”诺维科夫问道。

格特马诺夫用断断续续的声音说:

“该动身了,诺维科夫同志,茶喝过了,该去打德国佬了。”

“嘿,好家伙。”维尔什科夫在心里说。

诺维科夫走进军部的屋子,和涅乌多布诺夫谈了谈联络问题和转发命令问题,又看了看地图。

黑沉沉的夜色,似乎一片寂静,诺维科夫不由得想起在顿巴斯的童年。那时的黎明就是这样,似乎一切都在沉睡,可是过几分钟,空中就会充满汽笛声,人们就会朝矿井和工厂大门走去。但是在汽笛声响起之前就醒来的小别佳·诺维科夫知道,千百只手已经在黑暗中摸裹脚布、靴子,许多妇女已经光着脚在地上走,锅碗瓢盆已经在叮当响了。

“维尔什科夫,”诺维科夫说,“把我的坦克开到观察所,今天我要用。”

“是,”维尔什科夫说,“我把所有的东西装上去,您的东西,政委的东西。”

“别忘了带上可可。”格特马诺夫说。

涅乌多布诺夫披着军大衣走到台阶上。

“刚才托尔布欣中将打电话问,军长是不是上观察所了。”

诺维科夫点了点头,捅了捅司机的肩膀:

“走吧,哈里托诺夫。”

汽车出了小镇,离开最后一户人家,转了一个弯,又转了一个弯,就朝正西开去,擦过一片片白雪和枯草丛。汽车经过一片洼地,第一旅的坦克就集结在这里。诺维科夫忽然对司机说:

“停下!”

他跳下车来,朝着在晨曦中显得黑黝黝的坦克走去。他走着,不和任何人说话,注视着一个个人的脸。他想起前几天在乡村广场上看到的未剪过头的新兵小伙子们。确实,他们是孩子,可是世界上的一切,都是为了要他们到炮火底下去—总参谋部的计划,方面军司令部的命令,一个小时之后他要向各旅旅长发出的命令,政工人员要对他们说的话,作家们在报纸上发表的文章和诗歌。冲啊,冲啊!在黑沉沉的西方他们将遇到的是这种命运:朝他们射击,砍杀,坦克的履带把他们碾碎。

“要举行婚礼啦!”是的,不过没有甜葡萄酒,没有手风琴。“苦啊!”诺维科夫就要这样叫了,十九岁的新郎官们不会转过头去,会老老实实地吻他们的新娘。

诺维科夫觉得他似乎是在自己的弟弟、侄儿、街坊邻居的孩子们中间走着,几千个无形的农妇、姑娘、老妈妈在看着他。

母亲们否定了战争时期存在着派任何人去死的权力。在战场上也能遇到一些暗中同情母亲们的人。这些人说:“别动,别动,你上哪儿去,听,火力多么猛。让他们在那儿等我的报告吧,你在这儿烧烧开水好啦。”这样的人在电话里向上级报告说:“是,把机枪推出去!”可是,放下话筒,就说:“推到前面没有意思,会把一个好小伙子打死的。”

诺维科夫朝自己的坦克走去。他的脸显得阴沉而僵硬,似乎吸进不少十一月拂晓时候黑沉沉的潮气。当坦克发动起来的时候,格特马诺夫用会意的目光看了看他,说:

“诺维科夫同志,你可知道,正是在今天,我很想对你说说:我真喜欢你,你要明白,我相信你。”

十

一片寂静,没有任何声音,似乎世界上既没有草原,也没有晓雾,也没有伏尔加河,只有寂静。黑云上飞过一阵轻快而明亮的波纹,然后灰色的晓雾又变成深红色,忽然轰隆声震动了天空与大地……

近处的炮声与远处的炮声连成一片。回声把连成一片的声音储存起来,又把复杂交错的声音扩散开去,这声音便充满了辽阔战场的巨大空间。

泥土房屋在打颤,黄土从墙上掉下来,无声无息地落在地上。草原村庄里一户户人家的门自动开了又自动关上,湖上的薄冰裂了缝。

狐狸摇着长满软毛的沉甸甸的尾巴跑起来,兔子也跑,不是躲狐狸,而是跟着狐狸跑;夜间的猛禽和白日的猛禽也许是第一次汇合在一起,挥动沉甸甸的翅膀,飞上天空……有些黄鼠也糊里糊涂地从洞里跑出来,就好像迷迷糊糊、头发蓬乱的汉子从着了火的房子里往外跑。

发射阵地上潮湿的早晨的空气,似乎因为接触到几千门大炮的滚热的炮筒,温度上升了一度。

在前沿观察所,可以清清楚楚地看到苏军炮弹的爆炸,看到黑色和黄色的硝烟在旋转,泥土和肮脏的雪纷纷扬起,看到炮火的白光。

炮声停了。一团团硝烟慢慢化为一缕缕干燥、炽热的长发,与潮湿、寒冷的草原雾混合到一起。

天空马上充满新的声音,轰轰隆隆,又沉重,又响亮。一批批苏联飞机向西飞。飞机的轰隆声、啸声、吼声使灰云蔽日的模糊天空变得清晰可触。装甲强击机和歼击机贴近地面飞行,像低低的云片,而在云片之中和云片之上是用粗嗓门儿吼叫的不易看到的轰炸机。

德军飞机盘旋在布列斯特上空,而伏尔加河畔的草原之上是苏军的天空。

诺维科夫没有想这些事,没有回忆,没有比较。他正在经历的事比回忆、比较、思考更重要。

一切安静下来。等着寂静之后发出冲锋信号的人,准备一见到信号就朝罗马尼亚集团军阵地扑过去的人,一时间都在转瞬的寂静中屏住气息。在无声无息、浑浊的太古海洋一般的寂静中,在这几秒钟里,定好了人类发展曲线的转折点。参加保卫祖国的决战多么好,多么幸福。迎着死亡站起来,不是逃避死亡,而是跑去迎接死亡,多么沉痛,多么可怕。年纪轻轻地死去,多么可悲。希望活,希望活着。但愿保留年轻的生命,保留活得还太少的生命,世界上再没有什么愿望比这更强烈的了。这种愿望不在思想中,它比思想更强烈,它在呼吸中,在鼻孔中,在眼睛里,在肌肉里,在贪婪地吸收氧气的血红蛋白中。这愿望是如此之大,没有什么能与之相比,没有什么能测量其大小。可怕。冲锋前的时刻实在可怕。

格特马诺夫大声地、深深地吸了几口气,看了看诺维科夫,看了看战地电话机,看了看无线电发报机。

格特马诺夫看到诺维科夫的脸,感到十分惊异。这张脸已经不是格特马诺夫几个月来常常看到的那张脸。原来那张脸各种各样的表情他都见过的,不论在愤怒的时候、忧虑的时候、傲慢的时候,不论在高兴的时候、愁眉苦脸的时候。

没有压下去的罗马尼亚炮队一个一个地复活了,从纵深处朝前沿阵地进行急促射击。强大的高射炮也对准地面目标开了火。

“诺维科夫同志,”格特马诺夫激动地说,“到时候啦!别考虑太多!”

不仅是在战争时期,他总认为,为了事业牺牲一些人是很自然的,是无可非议的。但是诺维科夫不肯发命令,他吩咐接通重炮团团长洛帕津的电话,刚才他的大炮轰击过拟定的坦克运动的中心地带。

“你瞧着吧,诺维科夫同志,托尔布欣会骂你的。”

格特马诺夫看了看自己的手表。

诺维科夫不仅对格特马诺夫,对自己也不好意思承认自己的可笑的温情。

“我们会损失很多坦克的,心疼坦克呀,”他说,“几十部漂亮的坦克呀,总共不过几分钟的事,等我们把高射炮和反坦克炮压下去,他们就在我们掌心里了。”

在他面前的草原上一片硝烟。和他一起站在战壕里的人目不转睛地看着他。各坦克旅旅长在等待着他通过无线电发出的命令。他充满了一名上校惯有的战斗激情,很不斯文的功名心在紧张地突突跳动,而且格特马诺夫在催促他,他也怕上级。而且他清楚地知道,他对洛帕津说的话,总参历史科不会有人研究的,不会受到斯大林和朱可夫的称赞,不会使他得到盼望已久的苏沃洛夫勋章。

有一种权力,大于不加考虑就叫人去死的权力,那就是在叫人去死的时候深思熟虑的权力。诺维科夫行使了这一权力。

十 一

斯大林在克里姆林宫等待斯大林格勒方面军司令的报告。

他看了看表;炮火准备刚刚结束,步兵已出动,机动部队准备进入炮兵冲开的突破口。空军的飞机在轰炸后方、道路、机场。

十分钟之前,他和瓦图京通过话:西南方面军坦克部队与骑兵部队的推进超过了预计。

他拿起铅笔,看了看仍然沉默的电话机。他想在地图上标出南路人马开始运动的位置。但是一种迷信的感觉使他放下了铅笔。他清清楚楚感觉到,希特勒此时此刻正在想着他,并且知道他也在想着希特勒。

丘吉尔和罗斯福相信他,但是他明白:他们的信任不是绝对的。他们使他生气的是,他们虽然喜欢和他协商,但是在和他商议之前,他们之间已经商量好了。

他们知道,战争来了,总会过去的,而政治是永远存在的。他们赞赏他的逻辑、他的知识、他的清楚的头脑;他们使他恼火的是,总认为他是亚洲式的统治者,不是欧洲式的领袖。

他忽然想起托洛茨基那带有蔑视意味、微微眯着的、凌厉逼人的、聪明的眼睛,他第一次感到可惜,可惜托洛茨基已经不在人世,要不然让他看看今天多好呀。

他觉得自己是幸福的,身体是强壮的,嘴里没有像铅一样讨厌的味道,心口也不疼。在他来说,生的感觉和强的感觉是一回事。战争开始以后,斯大林就感到浑身不自在。元帅们看到他发火,呆呆地、笔直地站在他面前的时候,他仍然感到苦恼;当几千人在大剧院里站着向他致敬的时候,他还是感到苦恼。他总觉得,周围的人一想起他在一九四一年夏天的张皇失措,就偷偷地嘲笑他。

有一次,当着莫洛托夫的面,他抓住自己的头发,嘟哝说:“怎么办……怎么办呀……”在国防委员会会议上,他变了腔调,大家都垂下了头。他有好几次发出毫无意义的指示,他看出,大家都明白这些指示毫无意义……七月三日,他开始发表广播讲话的时候,心情十分慌乱,喝着治病的矿泉水,电波把他的慌乱心情传送出去……朱可夫在六月末不客气地反驳他,他一时间十分尴尬,说:“您想怎样就怎样吧。”有时他想把重任让给在一九三七年被杀害的雷科夫、加米涅夫、布哈林,让他们领导军队、领导国家吧。

他有时会出现十分可怕的感觉:在战场上取得胜利的不光是他今天的敌人。他想象到,跟在希特勒的坦克后面,在硝烟与灰尘中朝他走来的还有那些似乎被他永远制服了、被他打得永世不能翻身的人。那些人从冻土中爬出来,炸翻他们头上的永久冻土,冲破重重铁丝网。载满复活的人的一列列火车从科雷马开来,从科米共和国开来。许许多多农村妇女、儿童从土里爬出来,脸上带着可怕、悲痛、憔悴不堪的神情,走着,走着,用善良而悲伤的眼睛在找他。他比谁都清楚,审判失败者的不只是历史。

有时他恨死了贝利亚,因为贝利亚显然了解他的心情。所有这一切不好的、软弱的情绪持续了不久,只有几天,这一切只是有时候冲出来。

但是他还是常常有沮丧感,胃灼热搅得他不得安宁,后脑常常疼痛,有时头晕得可怕。他又看了看电话机:叶廖缅科该向他报告坦克推进的情况了。现在到了他显示威力的时候。此时此刻决定着列宁缔造的国家的命运,党的合理的中央集权也是在此刻获得实现的可能性,以便在建设大型工厂,建立原子能发电站和热核装置,制造喷气式飞机和涡轮螺旋桨飞机、宇宙火箭和洲际火箭,建筑摩天大楼、科学宫,开凿新的运河和海,在北极圈里建筑公路和城市中实现中央集权。

此时此刻决定着被希特勒占领的法国、比利时、意大利、斯堪的纳维亚国家和巴尔干国家的命运,将要宣布奥斯威辛、布痕瓦尔德和莫阿比特监牢的瓦解,在准备打开纳粹分子建立的九百处集中营和劳动营的大门。

还决定着即将前往西伯利亚的德军战俘的命运。也决定着在希特勒集中营里的苏军战俘的命运,后来在他们获得释放之后,斯大林决定把他们送往西伯利亚,分享德军战俘的命运。

还决定着米霍埃尔斯及其朋友和演员祖斯金、作家贝格尔森、马尔基什、费费尔、克维特科、努西诺夫的命运,要不然在处决以沃夫西教授为首的一批犹太医生的恶性案件之前他们就被处死了。

还决定着波兰、匈牙利、捷克斯洛伐克和罗马尼亚的命运。决定着苏联农民和工人的命运。决定着苏联思想、文学和科学的自由。

斯大林心情激动。此时此刻,国家未来的强盛和他的意志是一致的。他的伟大、他的天才不在于他本身,不以国家与武装力量的大小为转移。他写的书、他的学术著作、他的学说能够有意义,能够成为千百万人研究和赞颂的对象,只有在国家取得胜利的时候。

给叶廖缅科的电话接通了。

“喂,你那儿怎么样?”斯大林也不问好,径直问道。“坦克出动了吗?”

叶廖缅科听到斯大林带火气的声音,赶紧把香烟熄灭了。

“没有,斯大林同志,托尔布欣的炮火准备还没有结束。步兵已经扫清前沿,坦克还没有进入突破口。”

斯大林清清楚楚地骂了几声娘,就把话筒放下。

叶廖缅科又把香烟点着了,便给五十一集团军司令打电话。

“为什么坦克到现在还没有出动?”他问道。

托尔布欣一只手拿着话筒,另一只手拿着一块大手帕在揩胸膛上的汗。他的制服敞开着,雪白的衬衣敞着的领口里露出胖得打褶的脖根。

他克制着喘气,用肥胖人那种不慌不忙的语调回答(因为肥胖的人不仅理智上明白,而且全身都明白,着急是不行的):

“刚才坦克军军长向我报告说,在预定的运动中心地带还有敌人的炮火没有压下去。他要求再等几分钟,让我军炮火把敌方炮火压下去。”

“不能再等!”叶廖缅科严厉地说。“让坦克立即出动。过三分钟向我报告。”

“是。”托尔布欣说。

叶廖缅科本想把托尔布欣骂一顿,可是却突然问道:

“您怎么喘得这样厉害,病了吗?”

“没有,我身体很好,叶廖缅科同志,我刚才吃过早饭。”

“立即行动吧。”叶廖缅科说过这话,放下话筒,随口说:“吃早饭吃得气都喘不上来啦。”又骂了一句很难听的。

等到坦克军军部指挥所里的电话机嗡嗡响起来的时候,虽然因为重新开始的炮轰听不清话筒里的声音,诺维科夫还是明白了,这是集团军司令要求他立即率领坦克进入突破口。

他听完了托尔布欣的话,心里想:“早就料到啦。”他回答说:

“是,中将同志,马上执行。”

然后他朝着格特马诺夫笑了笑,说:

“再打上四分钟还是需要的。”

过了三分钟,托尔布欣又打来电话,这一次他不喘了。

“上校同志,您在开玩笑吧?为什么我听到还在炮击?立即执行命令!”

诺维科夫吩咐电话员接通炮兵团长洛帕津的电话。他听到洛帕津的声音,但他没有说话,看着秒针在走动,等待打满第二个四分钟。

“嘿,我们的头儿真行!”格特马诺夫出自内心地赞叹说。

又过了一分钟,炮声停息下来的时候,诺维科夫戴起耳机,呼唤打头冲向突破口的坦克旅旅长。

“别洛夫!”他喊道。

“有。军长同志。”

诺维科夫张大了嘴,用醉汉般的发狂的声音叫道:

“别洛夫,出动!”

青色的硝烟搅得晨雾更浓了,马达的吼声震得空气嗡嗡直响,坦克军进入突破口。

十 二

十一月二十日凌晨,加尔梅克草原上的大炮开始轰击,布置在斯大林格勒南面的斯大林格勒方面军突击部队向布置在保卢斯右翼的罗马尼亚第四集团军发起进攻的时候,苏军的进攻目标对于德国“B”集团军群司令部就是显而易见的了。

活动在苏军突击集团左翼的坦克军进入查查湖和巴尔曼查克湖之间的突破口,便朝西北向卡拉奇挺进,前去接应顿河方面军与西南方面军的坦克军与骑兵军。

二十日下午,从谢拉菲莫维奇发起进攻的突击集团到达苏罗维基诺以北,给保卢斯集团军的交通线造成威胁。

保卢斯的第六集团军还没感到有被包围的危险。下午六时,保卢斯向“B”集团军群司令魏克斯男爵上将报告说,计划在夜里继续派出侦察小分队在斯大林格勒进行活动。

晚上保卢斯收到魏克斯的命令:停止在斯大林格勒的一切进攻战斗,抽出大量的步兵、坦克兵团和反坦克武器,按梯队形式集中到左翼后面,准备朝西北方向进行突击。

保卢斯在这天晚上十点钟收到的这一道命令,标志着德军在斯大林格勒进攻的结束。

迅速发展的战局使这一道命令也失去了意义。

二十一日,从克列特和谢拉菲莫维奇发起进攻的苏军突击集团,朝自己原来的方向旋转九十度,汇合之后,向卡拉奇地区及其以北的顿河推进,直扑德军斯大林格勒战线的后方。

这一天,四十辆苏军坦克出现在高高的顿河西岸,离保卢斯集团军指挥部所在的戈卢宾镇只有几公里。另外一群坦克毫不费力地夺取了顿河大桥:守桥部队把苏军坦克部队当成了装备着缴获的坦克、常常通过这座桥的训练部队。苏军坦克进入卡拉奇,意在包围德军的两个斯大林格勒集团军,即保卢斯的第六集团军和戈特的第四坦克集团军。为了从后方保护斯大林格勒的阵地,保卢斯的精锐部队三八四步兵师把战线转向西北,进行防御。

就在这时候,从南面进攻的叶廖缅科的部队击溃了德军第二十九摩托化师,打垮了罗马尼亚第六军,朝卡拉奇至斯大林格勒的铁路线推进。

暮霭中,诺维科夫的坦克逼近了罗马尼亚军队的强化防御工事。

这一次诺维科夫再不怠慢。他没有利用黑沉沉的夜色暗地集中坦克为进攻做准备。依照诺维科夫的命令,所有的机器,不光是坦克,还有自行火炮,装甲运输车,装载摩托化步兵的汽车,一下子开足了灯光。

几百道明亮耀眼的灯光划破黑暗。大批战争机械从黑沉沉的草原上朝前涌去,吼声、炮声、机枪声震耳欲聋,刺目的灯光耀眼欲花,罗马尼亚守军惊慌失措,一片混乱。

在短短的战斗之后,坦克继续向前推进。

二十二日上午,从加尔梅克草原出发的苏军坦克进入布济诺夫镇。黄昏时候,在卡拉奇以东,在保卢斯和戈特的两大集团军的后方,一南一北两支苏军坦克先头部队会师了。到二十三日,步兵集团朝奇尔河和阿克赛河推进,成为突击集团可靠的外侧。

红军最高统帅向各部提出的任务完成了—在一百小时内完成了对德军斯大林格勒集团的包围。

局势的下一步发展会怎样呢?是什么决定了下一步发展?是谁的意志表现了历史的厄运?

二十二日下午六时,保卢斯通过无线电向“B”集团军群司令部报告:

“集团军被包围。整个察里察河谷,从苏维埃镇至卡拉奇的铁路线,该地区的顿河桥,河西岸的高地,在英勇抗击之后,转入苏军之手……弹药情况十分危急。粮食只能供应六天。如不能完成环形防御工事,请求给予行动自由。局势可能迫使放弃斯大林格勒以及战线的北段……”

二十一日夜里,保卢斯还收到希特勒的命令,要把他的军队所占据的地区叫做“斯大林格勒堡垒”。

在这之前的一道命令是:“集团军司令及司令部应进入斯大林格勒。第六集团军应进行环形防御,等待进一步指示。”

保卢斯与各军军长商议过之后,“B”集团军群司令魏克斯男爵打电报给最高统帅:“尽管做出这一决定我感到责任沉重,还是应当向您报告:我认为必须支持保卢斯将军撤出第六集团军的建议……”

经常和魏克斯保持联系的陆军总参谋长蔡茨列尔完全赞同保卢斯和魏克斯必须放弃斯大林格勒地区的意见,认为靠空运供应陷入重围的大量军队是不可思议的。

二十三日夜里两点钟,蔡茨列尔用电话通知魏克斯说,他终于说服希特勒放弃斯大林格勒。他说,关于第六集团军突围的命令,将由希特勒于二十四日上午发出。

二十四日上午十时过后不久,“B”集团军群与第六集团军之间唯一的一条电话线断了。

一分钟一分钟过去,等不到希特勒发出的突围的命令,因为必须迅速行动,魏克斯决定自己担起责任,发出突围命令。

通信兵正准备把魏克斯的电报发出去,这时候通信联络勤务科科长却听到最高统帅部发来的元首给保卢斯将军的电报:

“第六集团军被苏军围困是暂时的。我决定在斯大林格勒北郊、科特卢班、一三七号高地、一三五号高地、马林诺夫镇、斯大林格勒南郊等地集中兵力。你们可以相信我能做到我应做的一切,保证你们的供应和适时突围。我了解英勇的第六集团军及其司令,相信第六集团军能尽其职责。阿道夫·希特勒。”

希特勒的决定现在已反映出第三帝国的厄运,决定了保卢斯的斯大林格勒集团军的命运。希特勒用保卢斯的手,用魏克斯、蔡茨列尔的手,用德军各军军长和各团团长的手,用士兵的手,用一切不愿意执行他的决定而又执行到底的人的手,在德国战争史上写下了新的一页。

十 三

在一百小时的战斗之后,西南方面军、顿河方面军、斯大林格勒方面军的部队会合了。

在冬日的昏暗天空下,在卡拉奇郊外遍布辙痕的雪地上,苏军的先头坦克部队会师了。辽阔的积雪的原野被几百条履带划得支离破碎,被炮弹炸出一个个焦糊的窟窿。笨重的坦克在飞雪中迅速奔驰着,白色的雪团在空中颤动。在坦克急转弯的地方,冻土和雪尘一起飞向空中。

苏军的强击机和歼击机吼叫着贴着地面从伏尔加那边飞来,掩护进入突破口的坦克部队。重炮在东北方轰鸣,硝烟弥漫的昏暗天空闪着一道道模糊的亮光。

两辆T—34型坦克面对面停在一座小小的木头房子旁边。浑身脏污的坦克手们,因为作战胜利,捱过了生死关头,心情十分激动,呼哧呼哧、津津有味地吸着寒冷的空气。在坦克里面吸够了带油烟气的窒闷的空气之后,这寒冷空气就使人觉得特别提神了。

坦克手们把黑色的皮帽推到后脑勺上,走进木屋,从察察湖边来的坦克班长从衣服口袋里掏出一瓶酒……一个穿着棉袄和肥大毡靴的妇女,把在她那只打颤的手里叮当直响的玻璃杯放到桌子上,抽搭着说:

“唉,我还以为活不成了呢,我们的大炮好厉害呀,好厉害呀,我在地窖里待了两夜加一天。”

又有两个宽肩膀、小个子、像两个拼图方块似的坦克手走进房里来。

“瞧,瓦列拉,多好的招待。咱们好像也有什么吃的东西。”从顿河方面军来的坦克班长说。

于是,那个叫瓦列拉的小伙子把手伸进很深的衣服口袋,从口袋里掏出用油糊糊的战报包着的一截熏肠,把熏肠分成几份,还很认真地用棕色的手指头把掰掉出来的白白的肥肉往里塞。

坦克手们把酒喝干了,陶醉在幸福中。一名坦克手嘴里塞满了熏肠,一面笑着,一面说:

“咱们会合啦,就是说,你们的酒、我们的熏肠会合啦。”

大家都很喜欢这个说法,坦克手们笑着,嚼着熏肠,重复着这话,感到十分亲热。

十 四

从南面来的坦克上的班长通过无线电向连长报告了在卡拉奇郊外会师的情形。他还补充了几句话,说西南方面军的弟兄们非常好,说他们还共饮了一瓶酒。

这情形迅速地逐级上报,过了几分钟,旅长卡尔波夫便向军长报告了会师的消息。

诺维科夫感觉到,军部里在他周围出现了友好的、欢欣鼓舞的气氛。

坦克军在进军中几乎没有损失,按时完成了该军担负的任务。

涅乌多布诺夫在发出给方面军司令的报告以后,久久地握住诺维科夫的手。这位参谋长平时充满恼恨和火气的眼睛,变得明亮、温和了。

“您瞧,我们的人在没有内部敌人和破坏者的时候,能创造什么样的奇迹!”他说。

格特马诺夫把诺维科夫抱住,用眼睛扫了扫站在旁边的一些指挥员、司机、传令兵、话务员、译电员,抽搭了两下,为了让大家都能听到,他大声说:

“谢谢你,诺维科夫同志,作为一个俄罗斯人、一个苏联人,要感谢你。我格特马诺夫作为一个共产党员,要感谢你,衷心地向你致敬,向你表示感谢。”

他又一次把诺维科夫抱住,并且吻了吻深受感动的诺维科夫。

“你把一切都准备好了,把人都研究透了,什么都预见到了,所以就收获到大量工作结出的果实。”格特马诺夫说。

“哪儿有什么预见?”诺维科夫说。他听到格特马诺夫的话,又不好意思,又快活得不得了。他拿起一叠战报晃了晃,说:“这就是我的预见。我首先寄希望于马卡罗夫,可是马卡罗夫损失了速度,而且偏离了预定的运动中心,在侧翼参与了不必要的局部战斗,损失了一个半小时。我本来以为,别洛夫会不顾两翼,往前直冲,可是别洛夫在第二天不是撇开防御中心不顾一切地朝西北突进,而是和炮兵、步兵一起打起磨蹭战,甚至转为防御,因为这样胡闹损失了十一个小时。而卡尔波夫倒是第一个冲向卡拉奇,像旋风一样毫无顾忌地前进,毫不理睬两翼发生了什么事,第一个切断了德国人的主要交通线。这就是我对人的研究,这就是我的预见。我原来还以为,卡尔波夫需要拿棍子赶,认为他只会左顾右盼,只会保证自己的两翼。”

格特马诺夫笑着说:

“好啦,好啦,谦虚是美德,这我们是知道的。伟大的斯大林教导我们要谦虚嘛。”

诺维科夫感到很幸福。这一天,他多次想到叶尼娅,老是回头看,似乎就要看到她,大概,他的确太爱她了。

格特马诺夫用说悄悄话的小声说:

“诺维科夫同志,我一辈子也忘不了你是怎样推迟八分钟出击的。集团军司令在催促。方面军司令要求立即率领坦克进入突破口。我还听说,斯大林还打电话问过叶廖缅科,为什么坦克没有出动。你竟让斯大林等待。这不是,我们进入了突破口,确实没有损失一辆坦克,没有牺牲一个人。这件事我永远不会忘记。”

深夜,等诺维科夫开着坦克前往卡拉奇地区之后,格特马诺夫来到参谋长跟前,说:

“将军同志,我写了一封信,说明军长擅自推迟八分钟,才开始这场具有伟大意义的关键性战役、这场决定伟大卫国战争命运的战役。请您看看这封信。”

十 五

当华西列夫斯基通过高频电话向斯大林报告包围了德军斯大林格勒集团的消息时,他的助理波斯克列贝舍夫站在他旁边。斯大林也不看波斯克列贝舍夫,有一阵子他半闭着眼睛坐着,好像要睡觉。波斯克列贝舍夫屏住气息,尽可能不响动。

这不仅是他对活着的敌人胜利的时刻。这是他对过去取得胜利的时刻。一九三〇年农村坟头上的荒草会越来越茂密。北极圈里的冰层和雪丘会平静地保持沉默。

他比世界上任何人都懂得:胜利者是不受审判的。

斯大林忽然希望他的孩子们、他的孙女,也就是不幸的亚可夫的小女儿和他在一起。他可以安安静静、心平气和地抚摩小孙女的头,不去理会小屋门外的世界。文静可爱、病弱的小孙女,童年的回忆,凉爽的小花园,远处小河的流水声。其余的一切对于他都无所谓了。因为他的超级权力不依靠军队的强大和国家的强盛。

他没有睁开眼睛,慢慢地用一种特别柔和的、带着喉音的语调说:

“啊,鸟儿落网了,瞧着吧,别想从网里逃脱,咱们无论如何不能分离了。”

波斯克列贝舍夫看着斯大林那稀稀拉拉的白头发,看着他闭着双眼的麻脸,忽然感到手指头发起冷来。

十 六

在斯大林格勒地区的胜利进攻,消除了苏军防线上的许多缺口。消除的不仅是斯大林格勒与顿河两大方面军范围内的缺口,不仅是在崔可夫集团军与布置在北面的苏军几个师之间的缺口,也不仅是在一些脱离后方的连与排之间和隐藏在房屋中的小分队和战斗小组之间的缺口。孤立感、被半包围和包围的感觉也从人们的意识中消失了,换成了整体、团结一致和兵力十足的感觉。这种个人与众多的军队合为一体的意识,便是所谓致胜的士气。

当然,在陷入重围的德军士兵的头脑和心灵中,出现了完全相反的思想变化。由几十万有思想、有感觉的细胞组成的组织,从德国武装力量的肌体上脱离了。虚无缥缈的无线电波和更加虚无缥缈的关于军队和德国一直保持联系的宣传,证实了保卢斯在斯大林格勒的一些师已经被包围。

托尔斯泰当年提出,完全包围一支军队是不可能的,这一说法一再为托尔斯泰时代的军事经验所证实。

一九四一至一九四五年的战争却证明:可以包围一支集团军,把它牢牢困在原地。一九四一至一九四五年战争期间,被围是苏联和德国许多军队的残酷现实。

托尔斯泰的思想在他那个时代毫无疑问是正确的。但是,许多伟大人物提出的有关政治或战争的思想,大都不具有永久的生命力。

在一九四一至一九四五年的战争中,包围之所以成为现实,是因为军队有极大的机动性,而军队的机动性所依靠的后方的笨重庞大,极不灵活。进行包围的部队可以利用机动性的一切有利条件。被包围的部队完全失去机动性,因为在包围中不可能为现代化的军队组织多层次的、大范围的、工厂式的后方。被围的部队陷入瘫痪。进行包围的部队则可以利用陆上和空中的一切机械。

被围的军队失去机动性,不仅是失去军事机械方面的优势。被围的军队的士兵和军官就好像从现代文明世界掉进过去的世界。被围部队的士兵和军官不仅会重新估计作战部队的力量、战争的前景,还会重新评价国家的政策、党的领袖的魔力、法典、宪法、民族性格、民族的过去和将来。

那些像鹰一样洋洋得意地感到自己的翅膀强劲有力、在被缚住的无可奈何的猎物之上翱翔的人,同样也会重新评价上述一些问题,不过,当然带有相反的特点。

保卢斯的军队在斯大林格勒被包围,决定了战争进程的转折。

斯大林格勒的胜利决定了战争的结局,但是在胜利了的人民和胜利了的国家之间仍然进行着无声的争论。人的命运、人的自由取决于这一争论。

十 七

在东普鲁士和立陶宛边境,在格尔利茨秋天的森林里,下着稀稀拉拉的小雨。一个中等个头儿的人披着灰色斗篷,在高大树木之间的小道上走着。卫兵们一见到希特勒便屏住呼吸,一动也不动,雨滴从他们脸上缓缓流下。

他想呼吸呼吸新鲜空气,独自待一会儿。潮湿的空气使人非常舒服。飘洒着可喜的冷雨。一株株多么可爱、多么沉静的大树。在柔软的落叶上走一走,多么惬意。

野战大本营里的人一整天把他气得不得了……斯大林从来不曾引起他的尊敬。在战前他就觉得斯大林所做的一切又愚蠢又笨拙。斯大林的狡猾和奸诈都像庄稼汉一样简单。他的国家也是不像样子的。丘吉尔有一天总会明白新德国的悲剧性作用:正是德国用自己的身体掩护了欧洲,抵挡了亚洲的斯大林的布尔什维克主义入侵。他想象那些主张从斯大林格勒撤出第六集团军的人—他们倒是特别持重,特别恭敬的。使他生气的是那些轻率地相信他的人—他们总是啰啰唆唆地对他表示自己的忠诚。他一直希望带着蔑视的心情想想斯大林,把斯大林想得一钱不值,他又感觉到,他这种愿望是失去优势的感觉引起的……斯大林不过是一个狠毒的、报复心很重的高加索小铺老板。他今天的胜利根本改变不了什么局面……老浑蛋蔡茨列尔会不会暗暗用嘲笑的目光看他?他一想到戈培尔会向他报告英国首相评论他的军事才能的俏皮话,就十分生气。戈培尔会笑着说:“要承认,他说的话实在够俏皮。”在他那聪明而好看的眼睛里会浮现出隐藏得很深的嫉妒者的得意神情。

第六集团军不愉快的处境使他心慌意乱,失去本色。事情主要的糟糕之处,不在于丢了斯大林格勒,不在于一些师被包围;也不在于斯大林赢了他。

一切他都能扭转。

他一向就有一些很普通的想法和嗜好。但是等他变得伟大和具有无限权力之后,这一切就引起人们的赞赏和敬佩。他代表着德意志民族的精神。但是新德国及其武装力量的威力一旦开始动摇,他的英明就会减弱,他的天才就会消失。

他不羡慕拿破仑。他很不喜欢那些在孤独、贫困、一筹莫展的境况中依然十分伟大的人,不喜欢那些在好的和坏的境况中依然保持其力量的人。

他在林中独自散步的时候,也未能摆脱日常事务,并且在内心深处找到了总参谋部和党的领导机构那些墨守成规的人不可能找到的最高明、最切实的答案。他之所以产生难以忍受的烦恼,是因为他又感到他和大家平等了。

要想成为新德国的缔造者,要想燃起战火和奥斯威辛的炉火,创立盖世太保,做一个平常人是不行的。新德国的缔造者和领袖一定要脱离人类。他的思想、感情及日常生活只能在人类之上,在人类之外。

苏联的坦克使他回到了他原来离开的地方。他的思想、他的答案、他的嫉妒心今天不再是对着上帝,对着世界的命运。苏联的坦克又使他回到人间。

独自一人在林中,起初他是感到轻松的,现在他感到有些可怕了。一个人,没有卫兵,没有随侍的副官,他觉得自己像童话中的小孩走进了黑郁郁、到处是妖魔的密林。

童话中的小孩子就是这样走,小羊羔就是这样在林中迷了路,走着走着,也不知道大灰狼从密林深处偷偷朝它走来。从几十年的黑黑的沉淀层中浮出他童年时候的恐怖,想起小人书上的一幅画:一只小羊羔站在阳光明丽的林中空地上,在黑黑的、潮湿的大树丛中露出狼的红眼睛和白牙齿。

他很想像儿时那样,叫喊一声,他想唤母亲,想把眼睛捂起来,想跑。

不过在林中,在大树丛中藏着的是一个团,他的私人卫队,几千个强壮、受过训练、机动灵活、反应迅速的人。他们的生活目的,是不准外人的气息摇动他头上的一根头发,不准外人的气息触碰到他。不少电话机在轻轻地响着,向各处、各地段通报独自在林中散步的元首的每一行动。

他转过身来,压制着想跑的心情,朝着自己野战大本营的暗绿色房屋走去。

卫兵们看到元首走得很急,以为大本营里有急事等着他去。他们怎么能想到,德国元首在林中暮霭初降时候想起了童话中的狼?

在树丛中,大本营一个个窗户里的灯光亮了。他想到集中营火化炉的火光,心中第一次出现人的恐怖。

十 八

苏军第六十二集团军指挥所和许多掩蔽所里的人都产生一种十分奇怪的感觉:很想摸摸自己的脸,摸摸自己的衣服,动动靴子里的脚趾头。德国人不打炮了。静下来了。

寂静得使人头晕。人们觉得,似乎人都变空了,心麻木了,手和脚动作起来和以前有些不同了。在寂静中吃饭,在寂静中写信,夜里在寂静中醒来,似乎是奇怪的,不可思议的。寂静有自己的声音,很静的声音。寂静产生许多似乎很奇怪的新的声音:刀子的叮当声,翻书的沙沙声,地板的吱咯声,光脚丫儿的吧嗒声,笔尖的哧哧声,手枪保险装置的咔嚓声,掩蔽所墙上挂钟的滴答声。

集团军参谋长克雷洛夫走进集团军司令的掩蔽所,崔可夫坐在床上,对面的小桌后面坐着古洛夫。克雷洛夫本想一进门就说说最新的消息:斯大林格勒方面军已经发起进攻,包围保卢斯的问题再有几个小时就可以解决了。他看了看崔可夫和古洛夫,便一声不响地坐到床上。这样重要的消息克雷洛夫都没有对两位故友说说,可见他在他们脸上看到的不是一般的表情。

三个人都不说话。寂静产生了新的、在斯大林格勒久违的声音。寂静还准备产生新的、在战斗的日子里不必要的想法、激情、焦虑。

但是此时此刻他们还不知道什么新的想法;担忧、功名心、凌辱、嫉妒还没有从斯大林格勒的苦难经历中产生出来。他们还没有想到,他们的名字现在和苏联军事历史的光辉一页永远连在一起了。

这寂静的时刻是他们一生中最好的时刻。此时此刻他们只有人的感情,后来他们谁也不能自我解释,为什么他们此刻感到这样幸福、这样悲伤,充满这样的热爱和温情。

在结束了防御战之后,要不要继续说说斯大林格勒的将军们?要不要说说斯大林格勒防御战的一些领导人的可怜的贪求?

真理只有一种。没有两种真理。没有真理,或者伴随着残缺不全的真理、破碎的真理、砍削过的或者修剪过的真理,是很难生活的。部分的真理,不是真理。在这美好的寂静的夜里,让毫无掩饰的完整的真理占据心灵吧。我们要在这样的夜里把人的善良、人的伟大劳动计算在人的名下。

崔可夫走出掩蔽所,慢慢走到伏尔加河岸脊上,木板台阶在他脚下咯吱咯吱响着。天色已经黑下来。西方和东方都没有声音。工厂的轮廓、城市楼房的断垣残壁、一个个掩蔽所都和静默无声的黑沉的大地、天空、伏尔加河融为一体。

人民的胜利就是这样表现自己的。没有军队的分列式,没有轰鸣的混合乐队,没有烟火和礼炮,而是在潮湿的夜晚,在大地、城市、伏尔加河的安宁和静谧中迎接人民的胜利。

崔可夫十分激动,他那被战争磨硬了的心在胸中怦怦跳动着。他仔细听了听:并非寂静无声。从班内沟和“红十月工厂”那边传来歌声。下面,伏尔加河边有低低的说话声,有吉他的声音。

崔可夫回到掩蔽所。正等着他吃晚饭的古洛夫说:

“瓦西里·伊万诺维奇,真奇怪:这么安静。”

崔可夫在鼻子里“嗯”了一声,没有说话。过了一会儿,等他们在饭桌边坐下来,古洛夫说:

“唉,同志,你听到快活的歌儿都哭了,看样子,你也吃了不少苦呀。”崔可夫惊讶地瞥了他一眼。

十 九

在斯大林格勒的山沟坡上挖的一个土室里,几名红军战士围坐在自制的小桌旁,小桌上还有一盏自制的油灯。

司务长在往各人的杯子里斟酒。大家都注视着,这珍贵的液体小心翼翼地上升到司务长粗硬的指甲在玻璃杯上指着的位置。大家把酒干了,就吃起面包。有一名战士把一口面包吃下去之后,说:

“是啊,德国佬打得我们够呛,不过我们还是打赢了。”

“德国佬这一下子老实了,再也扑腾不起来了。”

“扑腾够了。”

“斯大林格勒大劫难到头了。”

“不过他们还是带来太多灾难。把半个俄罗斯烧掉了。”

他们吃了很久,不慌不忙,在不慌不忙中体会着一个人在长期艰苦的工作之后休息、喝酒、吃饭时的幸福和安宁。

头脑迷迷蒙蒙的,但是这种迷蒙有点儿特别,并不使人糊涂。不论面包的滋味、大葱的咯吱声、放在土室墙脚下的枪支,不论伏尔加河、想家的念头、对强大敌人的胜利,以及抚摩过孩子的头发、搂抱过妻子、掰过面包、卷过烟卷儿,如今又夺得胜利的手,对这一切,他们都清清楚楚地感觉到了。

二 十

疏散出去的莫斯科人在准备复员的时候,最高兴的也许不是很快就要见到莫斯科,而是摆脱了疏散时期的生活。斯维尔德洛夫斯克、鄂木斯克、塔什干、克拉斯诺亚尔斯克等城市的街道和房屋、秋日天空的星星、面包的味道—一切都成了令人厌恶的了。

如果他们看到苏联情报局报道的好消息,就会说:

“好啦,现在咱们很快就要走了。”

如果看到令人忧虑的消息,就会说:

“唉,不会再号召家庭团聚了。”

出现了不少传闻,说有些人没有通行证也回到了莫斯科—他们从长途列车上爬到工程列车上,然后又爬到电气列车上,电气列车上没有军队拦截。

人们都忘记了,一九四一年十月,在莫斯科过日子好像是在受刑讯。那时候人们多么羡慕那些用故城不祥的天空换取鞑靼和乌兹别克安宁生活的莫斯科人……

人们都忘记了,在一九四一年十月的灾难日子里,有些没上去火车的人纷纷丢掉箱子和包裹,徒步朝扎戈尔斯克走去,只要能离开莫斯科就行。现在人们也是宁可丢下东西、工作、安顿好的生活,步行回莫斯科,只要能离开疏散地就行。

一心想离开莫斯科和一心想回莫斯科这两种相反的心情的主要实质,就在于一年来的战争改变了人们的意识,对德国人莫名其妙的恐惧变为对苏联力量优势的信任。

在十一月下旬,苏联情报局报道了对弗拉季高加索(即奥尔忠尼启则)地区德国法西斯军队的攻击,然后又报道了在斯大林格勒地区进攻的胜利。在两个星期中,播音员有九次这样广播:“目前,我军继续反攻……再次沉重打击敌军……我军在斯大林格勒城下摧毁敌军的顽抗,突破顿河东岸敌军新防线……我军继续进攻,已推进一二十公里……近日部署在顿河中游一带我军对德国法西斯军队发起反攻……我军在顿河中游地区继续挺近……我军在北高加索继续出击……我军又在斯大林格勒西南方发动突击……我军在斯大林格勒以南发起进攻……”

在一九四三年除夕,苏联情报局发表战报《六周以来我军在斯大林格勒地区进攻作战总结》,综述了德军在斯大林格勒地区被包围的情况。

人们的意识准备转变,要用全新的观点看待现实中的大事,虽然这种思想转变的准备是秘密进行的,其秘密程度不次于准备斯大林格勒进攻战。在人们的潜意识中进行的这种再结晶,在斯大林格勒进攻战之后,第一次明朗化,第一次表现出来。

现在人们思想的变化和莫斯科会战胜利时的思想变化大不一样,虽然从表面上看来没有什么不同。其区别在于,莫斯科会战的胜利主要是促成了对德国人态度的变化。在一九四一年十二月,对德国军队莫名其妙的恐惧心理消失了。

斯大林格勒和斯大林格勒进攻战促成了军队与老百姓的新的自觉。苏联人、俄罗斯人开始从新的角度认识自己,开始从新的角度看待各种民族的人。俄罗斯的历史开始被理解为俄罗斯的光荣史,而不是俄罗斯农民与工人的苦难史和屈辱史。民族性由形式转变为内容,成为世界观的新的基础。

在莫斯科会战初次取胜的日子里,起作用的仍是战前的老的思维形式、战前的观念。

重新认识战争大事,认识苏联武装力量和国家的力量,是巨大的、长期的、广泛的认识过程的一部分。

这一过程在战前很久就开始了,不过主要不是在人民的意识中,而是在人民的潜意识中。

有三件大事是重新认识现实和人与人关系的重要标石,那就是:农村集体化、工业化、一九三七年。

这些事件和一九一七年的十月革命一样,造成了广大阶层的人民的动荡和变化;这些动荡伴随着对人的肉体的消灭,死亡人数超过了消灭俄国贵族阶级和工商业资产阶级的那个时期。

斯大林领导的这些事件,标志着新的苏维埃国家建设者在经济方面的胜利,标志着社会主义在一个国家的胜利。

这些事件是十月革命的必然的结果。

不过,在集体化、工业化和几乎更换了所有领导干部的时期建立起来的新的结构,并不想放弃旧有的思想公式和概念,虽然这些公式和概念对于新结构已失去真正的内容。新的结构利用的是一些旧的概念和成语,这些概念和成语发源于革命前就形成的社会民主党布尔什维克派。国家民族性仍然是新结构的基础。

战争加速了在战前就暗暗进行着的重新认识现实的过程,加速了民族意识的觉醒,“俄罗斯”这个词重新获得了真实的内容。

起初,在撤退时期,这个词大都和一些否定意义的词联系着:俄罗斯落后、一团糟,俄罗斯闭塞,俄罗斯没有希望……但是,民族意识既然出现了,就期待着战争的节日……

国家也渐渐趋向新的范畴的自觉。

民族意识在民族灾难的日子里表现出来,便是强大的、极好的力量。人民的民族意识在这样的时期之所以可贵,因为这种意识是人性的,而不是民族性的。这是人的尊严,人对自由的向往,人对善良的信赖,只不过表现在民族意识的形式中。

不过,在灾难岁月里激起的民族意识可能发展为多种形式。

毫无疑问,一位人事处长,一心要保护本机关不受世界主义者和资产阶级民族主义者的侵犯,这位处长的民族意识和保卫斯大林格勒的红军战士的民族意识,表现是不同的。

苏联这样一个大国的现实,决定了它将把唤起民族意识与完成国家战后面临的任务联系起来—在树立民族主权思想方面,在各个领域树立苏联和俄罗斯的主权观点方面。

所有这些任务不是在战时和战后突然出现的。战前,在农村的种种事件、建立祖国的重工业、干部大换班,标志着斯大林确立的制度作为社会主义新秩序在这个国家的胜利。在那个时候,这些任务就出现了。

俄国社会民主党的亲切的印记被抹去,被取消了。正是在斯大林格勒战役转折的时候,在斯大林格勒的火焰成为黑暗王国的唯一自由信号的时候,这一重新认识过程开始公开化了。

发展的逻辑导致的结果是,人民战争在斯大林格勒保卫战时期达到最高的热潮的同时,也为斯大林提供了可能性,公开宣扬国家民族主义思想体系。

二十一

在物理研究所前厅里贴出的墙报上,有一篇文章,标题是《永远同人民在一起》。

这篇文章说,在伟大的斯大林领导的正在穿越战争暴风雨的苏联,科学具有巨大意义,党和政府给予科学工作者极大的尊敬和光荣,世界上任何国家都不曾这样,即使在艰苦的战争时期,苏联政府也为科学家正常和有成效的工作创造了一切条件。

文章接着谈到研究所担负的巨大任务,谈到新的建设,谈到扩大旧的实验室,谈到理论与实践的联系,谈到科学研究对于国防工业有何等重要意义。

文章谈到全体科学工作者的爱国主义热潮,说科学工作者决不辜负党和斯大林同志的关怀和信任,不辜负人民对苏联知识分子的光荣的先进队伍,对科学工作者的期望。

文章的最后部分写道,可惜,在健康而友爱的集体中也有一些人缺乏对人民、对党的责任感,有一些人脱离了友好的苏维埃家庭。这些人使自己和集体对立起来,把自己的个人利益摆在党交给科学家的任务之上,拼命夸大自己实有的和臆造的功绩。他们之中有些人有意或无意地成为异己的反苏思想的代表,宣扬敌对的政治思想。这些人一般都要求用客观主义的态度对待外国唯心主义科学家的充满反动精神和蒙昧主义精神的唯心主义观点,夸耀自己同这些科学家的联系,从而侮辱俄罗斯科学家的苏维埃民族自豪感,贬低苏联科学的成就。

这些人有时像英勇的卫士,要维护似乎被践踏的正义,企图在短视、轻信的人和糊涂人中间赚得廉价的声名,实际上他们却在挑拨离间,散播不相信俄罗斯的科学力量、不尊重俄罗斯光荣历史和伟大人物的种子。文章号召消灭一切腐朽的、异己的、敌对的东西,消灭一切不利于完成党和人民在伟大的卫国战争期间交给科学家的任务的因素。文章的结束语是:“沿着马克思主义哲学明灯所照亮的光辉道路,沿着列宁和斯大林的党为我们开辟的道路,向着新的科学高峰,前进!”

虽然文章没有点名,但是实验室里的人都明白,矛头是对着维克托·施特鲁姆的。

萨沃斯季扬诺夫对维克托说了说这篇文章。维克托没有去看文章,这时候他站在即将完成新设备安装的同事们旁边。他抱住诺兹德林的肩膀,说:

“不论怎样,这大家伙会大有作为的。”

诺兹德林忽然骂起娘来,骂的是复数代名词,维克托一时不明白他骂的是什么人。快下班的时候,索科洛夫走到维克托跟前。

“维克托·帕夫洛维奇,我很欣赏您。您一整天都在工作,就好像什么事儿也没有。您的毅力真了不起。”

“如果一个人天生是淡黄头发的,决不会因为墙报上的文章变成黑头发的。”维克托说。

他生索科洛夫的气已成了习惯,正因为他已经习惯了,似乎这种气已经没有了。他已经不责备索科洛夫的不坦率和怯懦。有时他自己对自己说:“他有很多好的地方,不好的地方人人都免不了有。”

“是啊,文章与文章不同,”索科洛夫说。“安娜·斯捷潘诺芙娜看了这篇文章,心脏病都发作了。已经把她从医务所送回家了。”

维克托心想:“究竟写的是多么可怕的事?”不过他没有问索科洛夫。至于文章的内容,谁也没有和他说起。人们不和病人谈他的不治之症,大概就像这样。

傍晚维克托最后一个离开研究所。看大门的老头子阿列克谢·米海洛维奇已经调到存衣室工作,他一面给维克托拿大衣,一面说:

“您瞧,维克托·帕夫洛维奇,真是的,在这世界上好人总不得安宁。”

维克托穿好大衣,又上了楼,在墙报栏前站了下来。他看完了那篇文章,惊慌地四处看了看:一时间他仿佛觉得,他马上就要被逮捕了,可是前厅里空空荡荡,十分安静。

他实实在在地感觉到一具脆弱的人体的重量和庞大的国家的重量的悬殊,他感觉到,仿佛国家用巨大而明亮的眼睛死死地盯着他,仿佛国家就要朝他压下来,他就要咯吱一声,尖叫一声,就此消灭了。

街上人很多。维克托觉得,在他与行人之间有一片无主的土地。

在电车里,一个戴着皮军帽的人用兴奋的语调对自己的同伴说:

“你听到最新消息了吗?”

前面座位上有一个人说:

“斯大林格勒!德国佬完啦!”

一个上了年纪的妇女看着维克托,好像是责备他不说话。

他带着温和的心情想到索科洛夫:人人都有缺点,他也有,我也有。

但是他从来没有彻底真诚地承认自己和别人同样有毛病和缺点,所以他马上就想:“他的观点取决于国家是否喜欢他,他的生活是否顺利。等到春天来临,等到胜利了,他一句批评的话都不会说。我却不是这样:不论国家状况是好是坏,不论国家折磨我还是眷顾我,我对国家的态度不会变化。”

到家后他要对柳德米拉说说这篇文章。看样子,当真要整他了。他要对柳德米拉说:“柳德米拉,你瞧瞧,这就是斯大林奖金!想抓人的时候,常常写这样的文章。”

“我们是同命运的,”他想道,“如果请我去巴黎大学举行学术讲座,她会和我一块儿去;如果送我上科雷马的劳改营,她也会跟我去。”

“是你自己把自己弄到这种可怕的地步。”柳德米拉会说。

而他会反唇相讥:

“我要的不是批评,是体贴和理解。研究所里的批评已经够我受的了。”

给他开门的是娜佳。在幽暗的走廊里,娜佳把他抱住,并且把脸贴到他的胸膛上。

“我浑身又冰冷,又潮湿,让我把大衣脱了。出了什么事吗?”他问道。

“难道你没听到?斯大林格勒呀!巨大的胜利。德国佬被包围了。咱们走,快走。”

她帮他脱了大衣,拉着他的手进了房间。

“这儿来,这儿来,妈妈在托里亚的房里呢。”

她把门开了。柳德米拉坐在托里亚的书桌前。她慢慢朝他转过头来,又得意又伤心地朝他笑了笑。这天晚上,维克托没有把研究所里发生的事告诉柳德米拉。

他们坐在托里亚的书桌前。柳德米拉在一张纸上画包围斯大林格勒德军的示意图,向娜佳说着她对作战计划的理解。夜里,维克托在自己的房间里想:“天啊,写一份检讨书吧,大家在这种情况下不都写吗。”

二十二

墙报上出现那篇文章之后,又过了几天。实验室里的工作照常进行着。维克托有时灰心丧气,有时兴致勃勃,很带劲儿地工作,在实验室里走来走去,还不时用手指头在窗台和金属外壳上轻快地敲出自己喜欢听的声音。

他开玩笑说,看样子,在研究所里蔓延起近视流行病,很多熟人面对面遇到他,都带着若有所思的神气从旁边走过去,连招呼也不打。古列维奇老远看见维克托,也摆出一副若有所思的神气,走到大街的另一边,在一张广告前面站下来。维克托为了看个究竟,回头看了看,这时候恰好古列维奇也回头看,他们的视线相遇了。古列维奇做出一副又惊讶又高兴的姿态,鞠了个躬,这一切都不是多么使人愉快的。

斯维琴见到维克托,打了招呼,还小心地碰了碰脚跟表示敬意,不过在打招呼的时候他脸上的表情却很不自然,就好像他在迎接不友好国家的一位大使。

维克托做了统计:哪些人不理睬他,哪些人对他点头,哪些人和他握手问好。

每天他回到家里,第一件事就是问妻子:

“有没有谁来电话?”

柳德米拉的回答一般都是:

“没有,如果不算玛利亚的话。”

她知道她说过这话后他常常问的问题,就又说:

“马季亚罗夫暂时也没有信来。”

“你瞧,”他说,“过去天天给咱们打电话的,现在不怎么打了;过去不怎么打的,现在根本不打了。”

他觉得,家里人对待他也和以前不一样了。有一次他正在喝茶,娜佳从他身边走过,也不向他问好。他厉声对她喝道:

“为什么连招呼也不打?你觉得我不是活物吗?”

显然他在说这话的时候脸上表情显得非常可怜、非常痛苦,娜佳理解他的心情,所以没有顶撞他,而是急忙说:

“好爸爸,爸爸,原谅我。”

就在这一天,他问她:

“娜佳,你还是常常和你那位大将军见面吗?”

她一声不响地耸了耸肩膀。

“我要警告你,”他说,“不许和他谈政治问题。如果在这方面出问题,就更够我受的了。”

娜佳还是没有粗暴地回答,而是说:

“你放心吧,爸爸。”

早晨,他快到研究所的时候,就开始四下里张望,时而放慢脚步,时而加快脚步。他看到走廊里没有人,便垂下头急匆匆地往前走,如果有什么地方的门开了,他的心就紧缩起来。

他终于走进实验室之后,便气喘吁吁,就好像一个士兵终于跑过炮火控制的阵地,进入自己的战壕。

有一天,萨沃斯季扬诺夫来到维克托的办公室里,说:

“维克托·帕夫洛维奇,我和大家都请求您写一份检讨书,检讨检讨。我请您相信,这能够起作用。您想想看,就在您面前摆着大量的工作,应该说,摆着伟大的工作的时候,就在我们这学科的有生力量都指望着您的时候,忽然就这样一下子翻了车,怎么办呀!您写一份检讨书,承认一下错误吧。”

“我检讨什么?我有什么错误?”维克托说。

“哎,还不就是那么一回事儿,大家都这样做嘛,不论是在文学界,在科学界,还有不少党的领导人,还有您喜欢的音乐家们,肖斯塔科维奇也承认错误,写检讨书,检讨过之后,就没有事了,还在继续工作。”

“不过我究竟检讨什么呢?向谁检讨呢?”

“您写给院部,写给党中央。这实际上不是主要的,写给谁都行!主要的是您检讨了。比如,就写:‘我承认错误,我错了,现在认识到了,保证改正。’就写诸如此类的话,您是知道的,这都是老一套了。不过主要的是,这能管用,总是管用的!”

萨沃斯季扬诺夫那一向在笑的、快活的眼睛现在是严肃的。似乎眼睛的颜色也变了。

“谢谢,谢谢,好同志,”维克托说,“您的友情真使我感动。”

又过了一个钟头,索科洛夫对他说:

“维克托·帕夫洛维奇,下礼拜举行学术委员会扩大会议,我认为,您一定要说一说。”

“说什么呢?”维克托问。

“我觉得,您应该解释解释,说干脆些,就是要检讨错误。”

维克托在办公室里踱起来,忽然在窗前站下来,朝院子里看着,说:

“索科洛夫同志,是不是最好还是写一份检讨书?这样比起当众往自己脸上吐唾沫,总要轻松些。”

“不,我以为,您一定要说一说。昨天我和斯维琴谈过,他向我示意,说上面,”他还含含糊糊地朝上面的门檐上指了指,“希望您在会上说一说,而不是要您写检讨书。”

维克托很快地朝他转过身来:

“我既不在会上检讨,也不写检讨书。”

索科洛夫就像一位精神病医生在和病人谈话那样,用十分耐心的语气说:

“维克托·帕夫洛维奇,您在目前的情况下不说话,就等于有意地自杀,有可能把您的问题弄成政治问题。”

“您可知道,使我特别难受的是什么?”维克托问道。“为什么在大家都高高兴兴的胜利日子里我会遇到这样的事?哪一个狗崽子会说我公开攻击列宁主义原理,说我认为苏维埃政权完了?有人就是喜欢拣软的欺。”

“我听到过这种说法。”索科洛夫说。

“哼,去他妈的吧!”维克托说。“我不检讨!”

可是到了夜里,他一个人却躲在自己的卧室里写起检讨书。他感到羞惭,把检讨书撕碎,却马上又写起在学术委员会会议上的发言稿。他重看了一遍,用手在桌上一擂,又把发言稿撕碎。

“就这样,随它去!”他说出声来。“要怎样就怎样吧。坐牢就坐牢好啦。”

他咂摸着自己的最后决定的滋味,一动不动地坐了一阵子。然后他想出一个主意:他可以写一份检讨书的预备稿,如果他决定检讨的话,就交上去。这样不会损伤什么尊严。谁也不会看到这份检讨书,任何人看不到。

他是一个人,门也关着,周围的人都睡了,窗外静悄悄的,没有警笛声,也没有汽车声音。但是有一种看不见的力量把他压住。他感觉到它的威慑的重量,它强迫他按它的意图去想,强迫他按照它的意思写。它就在他身体内部,强迫他的心收缩,溶解他的决心,干预他对待妻子和女儿的态度,混入他的过去,混入他关于年轻时代的一些想法。他开始感觉自己是愚钝的、无聊的,常常说一些枯燥无味的啰唆话使人感到厌烦的。甚至他的著作好像也失去了光彩,蒙上一层灰土,不再使他充满了光明和欢乐。

只有不曾亲身体验过这种力量的人,见到有人屈服于这种力量,才会感到惊讶。亲身体验过这种力量的人,感到惊讶的倒是另一点:敢于发一下火,哪怕是迸出一句怨言,或者很快地做一个表示抗议的手势。

维克托写检讨书是自己留着的,他要收藏起来,不给任何人看,但是同时他心里也明白,这检讨书说不定会用得着的,还是留着吧。

早晨,他一面喝茶,一面看表:该上研究所去了。他充满可怕的孤独感。似乎今生今世再不会有谁上他家来了。要知道,没有人给他打电话,不仅仅是因为害怕。还因为他又无聊,又乏味,又无能。

“不用说,昨天也没有谁问到我了?”他对柳德米拉说过这话,便朗诵起来:“我一个人在窗前守候,看不到客人,也看不见朋友……”

“我忘了告诉你,契贝任回来了,打来电话,说希望看到你。”

“啊,”维克托说,“啊,这事儿你怎么能不吿诉我呢?”他在桌上敲起胜利的乐曲节拍。

柳德米拉走到窗前。维克托不慌不忙地踱着步子,高高的身躯,微微驼背,不时地挥两下皮包,她知道,这是他想着和契贝任见面,在考虑怎么跟他问好,和他说话呢。

这些天来,她十分心疼丈夫,为他担心,但同时也想着他的缺点,想着他的主要缺点—自私。

刚才他还在朗诵:“我一个人在窗前守候,看不见朋友……”现在他上实验室去了,实验室里有很多人,有工作;到晚上他就要去找契贝任,大概不到十二点不会回来,也不想想,她一整天会孤单单的,会一个人站在窗前,房子里空荡荡的,身边一个人也没有,她也看不到客人,看不到朋友。

柳德米拉上厨房里去洗碗。这天早晨她心里特别难受。玛利亚今天也不会打电话来,今天她要上沙鲍洛夫镇去看姐姐。娜佳的事多么使人不放心呀。她不言不语,当然也不顾禁令,仍然天天晚上出去玩儿。维克托天天操心的是自己的事,也不肯想想娜佳。

门铃响了,大概是木匠来了,昨天她和木匠约好,今天要来修托里亚房间的门。柳德米拉非常高兴:活生生的人来了。她把门开了—在幽暗的走廊里站着一个女子,头戴灰色羔羊皮帽,手里还提着箱子。

“叶尼娅!”柳德米拉叫起来。她的声音那样高,那样伤感,连她自己都很吃惊。她一面吻着妹妹,抚摩着她的肩膀,一面说:“托里亚不在了,不在了,不在了。”

二十三

浴盆里的热水细细地流着,流得很慢,只要把龙头多少一开大,水就变成凉的。浴盆上满水用了很长时间,可是姐妹俩觉得,她们见了面好像还没来得及说两句话。

后来,叶尼娅进去洗澡,柳德米拉不时走到浴室门口,问:

“喂,你在里面怎么样,要不要给你擦擦背?注意煤气炉,不要灭了。”

过了几分钟,柳德米拉用拳头敲了敲门,生气地问道:

“你在里面怎么啦,睡着了吗?”

叶尼娅穿着姐姐的毛茸茸的浴衣走出浴室。

“啊,你真是个女妖。”柳德米拉说。

叶尼娅想起来,那天夜里诺维科夫来到斯大林格勒的时候,索菲亚·奥西波芙娜就曾经管她叫女妖。

饭菜已经摆好了。

“有一种很奇怪的感觉,”叶尼娅说,“坐了两天两夜没有卧铺的火车之后,在浴室里洗个澡,就好像回到了和平康乐的时期,可是在心里……”

“你怎么忽然上莫斯科来啦?出了什么事情吗?”柳德米拉问道。

“等一会儿再说,等一会儿。”

她摆了摆手。

柳德米拉说了说维克托的情况,说了说意想不到的娜佳的可笑浪漫史,说了说一些熟人连电话也不来了,碰到维克托就好像不认识。

叶尼娅也说到斯皮里多诺夫上古比雪夫的情形。他变得又可爱又可怜了。调查小组在调查他的问题,在查清之前,不给他安排新的工作。薇拉带着小孩子住在列宁斯克,斯皮里多诺夫说起小外孙就哭。后来她又对柳德米拉讲了亨利逊老奶奶被流放的事,说沙尔戈罗茨基老头子多么可爱,里蒙诺夫怎样帮助她办好户口手续。

叶尼娅的头脑里还回旋着烟雾、车轮的轧轧声和车厢里的说话声,所以她看着姐姐的脸,感觉柔软的浴衣贴着洗得干干净净的身体,坐在又有钢琴又有地毯的房间里,确实感到奇怪。

在姐妹俩互相说的许多事情中,在今天她们高兴的事和伤心的事、好笑的事和感人的事中,总有一些已经离开人世、但永远和她们分不开的亲人和朋友。不论说到维克托的什么,总有他妈妈的影子站在他后面;说起谢廖沙,马上就会出现他进了劳改营的爸爸和妈妈;还有那个宽肩膀、厚嘴唇的腼腆小伙子的脚步声日日夜夜在柳德米拉身边响着。但是她们并没有说起这几个人。

“索菲亚·奥西波芙娜一点音信也没有,就好像沉到地里去了。”叶尼娅说。

“是姓列文顿那个女人吗?”

“是,是,就是她。”

“我不喜欢她。”柳德米拉说。她又问道:“你还画画吗?”

“在古比雪夫没画。在斯大林格勒画过。”

“你可以夸耀夸耀了,维克托在疏散时还带着你的两幅画呢。”

叶尼娅笑着说:

“这是令人高兴的。”

柳德米拉说:

“你这将军夫人,怎么不说说最要紧的?你满意吗?爱他吗?”

叶尼娅一面掩上胸前的衣襟,一面说:

“是的,是的,我很满意,我很幸福,我爱他,他也爱我……”

又用迅速的目光看着柳德米拉,补充说:

“你可知道,我为什么上莫斯科来?克雷莫夫被捕了,在卢比扬卡监狱里。”

“天啊,这究竟是为什么?他可是百分之百的布尔什维克呀!”

“咱们的米佳呢?你那阿巴尔丘克呢?他恐怕是百分之二百的了。”

柳德米拉沉思起来,说:

“要知道,克雷莫夫真是够狠心的!他在普遍集体化时期就不同情农民。我记得我曾经问他:这究竟是怎么回事儿呀?他回答说:都是富农,死就死吧。他对维克托很有影响。”

叶尼娅带着责备的口气说:

“唉,姐姐,你总是想起人不好的地方,而且直截了当地说出来,偏偏是在不应该说的时候。”

“有什么办法,”柳德米拉说,“我是直性子呀,就像车杠一样。”

“好啦,好啦,不过你不要因为你车杠式的美德感到骄傲。”叶尼娅说。

她又小声说道:

“姐姐,我也被传讯了。”

她从沙发上拿起姐姐的头巾,用头巾把电话机捂住,说:

“据说,可以在电话里窃听。他们还要我签了字,保证随传随到。”

“据我所知,你没有和克雷莫夫办理结婚登记手续呀。”

“是没有登记,可是没登记又怎样呢?他们审讯我,就拿我当妻子。我就对你说说吧。他们送来传票,要我带着身份证出庭。我一个一个地回想,想到大哥,想到大嫂,甚至想到你那阿巴尔丘克,所有被捕的熟人我都想到了,却怎么也没有想到克雷莫夫。是快到五点钟把我传去的。那是一个很普通的机关办公室。墙上挂着斯大林和贝利亚的大肖像。一个年轻人,一副平平常常的嘴脸,带着咄咄逼人的神气看着我,开门见山地问:‘您了解尼古拉·格里高力耶维奇·克雷莫夫的反革命活动吗?’我有好几次觉得,我从那里面出不来了。你要知道,他甚至向我暗示诺维科夫。真是个可怕的坏家伙,好像我和诺维科夫接近,为的是搜集他可能泄露的情报,然后交给克雷莫夫。我心里好像一切都变成了木头。我对他说:‘您要知道,克雷莫夫可是一个忠心耿耿的共产党员,和他在一起就像在区党委会里一样。’他对我说:‘噢,这么说,您认为诺维科夫不是苏联的人吗?’我对他说:‘你们干的事情真奇怪,人家在前方和法西斯作战,您这个年轻人却坐在后方败坏人家的名誉。’我以为他听到这话会打我耳光的,可是他有些发窘,红了红脸。总而言之,克雷莫夫被捕了。罪名有些莫名其妙—又是托洛茨基派,又是和盖世太保有秘密关系。”

“多么可怕呀。”柳德米拉说过这话,就在心里想,本来托里亚也可能被包围,可能被怀疑干这种事呀。

“可以想见,维克托听到这消息会怎样,”她说,“他现在神经紧张得可怕,总觉得会有人来抓他。他天天在回想他在什么地方,和什么人说过什么话。特别是常常想到那倒霉的喀山。”

叶尼娅目不转睛地对着姐姐看了一阵子,终于说:

“要不要对你说说,最可怕的是什么?那个侦讯官问我:‘既然您的丈夫对您说过托洛茨基称赞他的文章精彩,您怎么不知道您的丈夫是托洛茨基派?’后来我在回家的路上想起来,确实克雷莫夫对我说过:‘只有你一个人知道这话。’到了夜里,我猛然想起来:诺维科夫秋天上古比雪夫来的时候,我对他说过这话。我觉得,我简直要发疯了,我觉得太可怕了……”

“你倒霉。你就应该遇到这类的事儿。”

“为什么我就应该?”叶尼娅问道。“你也可能会有这种事儿嘛。”

“噢,不是。你丢了一个,又找一个。却要对这一个说那一个的事。”

“不过,你也和托里亚的父亲分手了呀。恐怕你也对维克托说了不少。”

“不,你说的不对,”柳德米拉用肯定无疑的语气说,“这是根本不同的两码事。”

“那又为什么?”叶尼娅问道。她看着姐姐,忽然感到很恼火。“你要知道,你说的话实在太蠢。”

柳德米拉很平静地说:

“我不知道,也许很蠢。”

叶尼娅问道:

“你没有钟吗?我要去库兹涅茨桥24号。”

她已经压不住火气,说:

“柳德米拉,你的性格很乖僻。难怪你住着四居室的一套房间,妈妈却宁愿在喀山孤单单一个人过日子。”

叶尼娅说过这两句无情的话,便懊悔说得太尖刻了,为了让姐姐能感觉到她们之间相互信任的关系还是胜过偶然的争执,就说:

“我希望相信诺维科夫。不过总是,总是……为什么这话让保安人员知道了呢?是怎么知道的呢?这可怕的一层迷雾怎么来的呢?”

她很希望妈妈在她身边。她会把头放在妈妈的肩上,说:“妈妈,我太累了。”

柳德米拉说:

“你可知道,怎么会有这样的事?你那位将军也许会把你们说的话对什么人说说,那人就记下来了。”

“是啊,是啊,”叶尼娅说,“真奇怪,这样简单的问题我竟没有想到。”

来到柳德米拉又清静又安宁的家里,她更清楚地感觉出自己内心的慌乱了……

她离开克雷莫夫时没有感觉到、没有想到的,在分离之后暗暗使他痛苦、使她不安的—尚未断绝的对他的柔情,为他担忧的心情,和他处惯了的感觉—近几个星期以来增强了,又冒出来了。

她在工作时想到他,在电车上想到他,站队买东西时也想到他。几乎每天夜里她都要梦见他,在梦里呻吟,喊叫,惊醒。

梦总是噩梦,总是梦见大火,梦见打仗,梦见克雷莫夫面临危险,而且总是无法使他脱离危险。

早晨,她在匆匆忙忙地穿衣服,洗脸,担心上班迟到的时候,她也在想着他。

她觉得她已经不爱他了。但是,难道会这样时时刻刻想着一个自己不爱的人,会因为他不幸的命运感到这样痛苦吗?为什么每次里蒙诺夫和沙尔戈罗茨基嘲笑克雷莫夫喜欢的一些诗人和艺术家,说他们平庸无才的时候,她很想看到他,抚摩他的头发,亲亲他,心疼心疼他呢?

现在她已经不记得他的思想狂热、他对被镇压者的遭遇漠不关心、他在普遍集体化时期说到富农时那股凶狠劲儿。

现在她想起的只是好的地方,只是带有浪漫色彩的事,令人感动的事,使人伤感的事。现在他征服她的力量是他的弱小。他的眼睛是小孩子的眼睛,他的笑是不知所措的笑,他的动作是笨拙的动作。

她仿佛看到他的肩章被撕掉了,胡子已经花白了,仿佛看到他夜里躺在床铺上,看到他在监狱院子里放风时的脊背……大概他在想,她本能地预测到他今天的遭遇,这就是他们分手的原因。他躺在监狱里的床上,想着她……她做了将军夫人……

她不知道:这是怜悯,是爱情,是良心,还是责任心?

诺维科夫给她寄来通行证,并且通过军用专线和空军里的一位朋友说好了,那位朋友答应用飞机把叶尼娅送到方面军司令部。领导也给她三个星期的假,让她上前方去。

她自己一遍又一遍地安慰自己说:“他会了解的,他一定会了解,我不这样不行。”她知道,她这样对待诺维科夫是很可怕的:他天天在等她。

她给他写了一封信,丝毫不隐瞒地把一切都告诉了他。她把信寄出去以后,就想,军事检察机关会看到这封信的。这一切会给诺维科夫带来非同一般的麻烦。

“不要紧,不要紧,他会了解的。”她一再地说。

不过,问题是,诺维科夫了解是会了解,可是等他了解了,就会从此和她分手的。

她是不是爱他,她爱的是否仅仅是他对她的爱?

当她想到难免要和他最后分手的时候,她感到自己就要孤孤单单,顿时觉得十分可怕,十分痛苦,十分恐怖。

是她自己,是自己心甘情愿毁掉自己的幸福,她一想到这,就觉得难以忍受。

但是当她想到,现在她已经什么也不能改变了,他们是不是彻底分手并不取决于她,倒是取决于诺维科夫,这种想法尤其使她难受。

当她对诺维科夫的想念使她觉得无法忍受、异常痛苦的时候,她就开始想象克雷莫夫的处境。想象着传她去对质……你好,我的可怜的人。

诺维科夫却是高大,强壮,肩宽腰粗,大权在握。他不需要她的支持,他自己能行。她管他叫“胸甲骑兵”。她永远也不会忘记他那英俊可爱的脸,她会永远怀念他,怀念她自己毁掉的幸福。随它去吧,随它去吧,她不怜惜自己,她不怕自己痛苦。

但是她知道,诺维科夫并不是多么刚强。有时他脸上会出现无计可施的、几乎胆怯的表情……而且她对自己也并不是那么残酷无情,对自己的痛苦并不是那么毫不在乎。

柳德米拉好像参与了妹妹的思考,问道:

“你和你那位将军怎么办呀?”

“我很怕想这一点。”

“唉,谁也无法理解你的做法。”

“我不能不这样做!”叶尼娅说。

“我不喜欢你这种不实际。离了就是离了。好了就是好了。用不着藕断丝连,拖泥带水。”

“噢,噢,是要我避祸寻福吗?按这条原则做人,我不会。”

“我说的不是这个。我很尊敬克雷莫夫,虽然我并不喜欢他;你那位将军,我还从来没有见过。既然你决定做他的妻子,就要对他有责任心。你却毫无责任心。他担负着重要任务,在打仗,可是妻子却在这时候送东西给被捕的人。你可知道,这会给他带来什么后果?”

“我知道。”

“那你究竟爱不爱他?”

“你行行好,别问吧。”叶尼娅带着哭腔说,并且在心里说:“我究竟爱谁呢?”

“不,你回答我。”

“我不能不这样做,因为人不是为了快活才进卢比扬卡的大门。”

“不应当只考虑自己。”

“我考虑的就不是自己。”

“维克托也会这样考虑的。归根究底都是个人主义。”

“你的逻辑真是不可思议,我从小就觉得你很古怪。你把这叫做个人主义吗?”

“你这样又有什么用呢?你又不能改变判决。”

“比如,有朝一日把你关起来,那时候你就知道亲人能起到什么作用了。”

柳德米拉想改变话题,问道:

“你这漂泊的新娘,告诉我,你有玛露霞的相片吗?”

“只有一张。你记得吗,是在索科利尼基照的?”

她把头放在姐姐的肩上,用诉苦的语气说:

“我太累了。”

“你休息休息,睡一会儿,今天你哪儿也别去,”柳德米拉说,“我把床给你铺好了。”

叶尼娅半闭起眼睛,摇了摇头。

“不,不,不用。我是活得太累了。”

柳德米拉拿来一个大信封,把一摞照片抖落在妹妹的膝盖上。

叶尼娅翻看着照片,叫了起来:

“我的天呀,我的天呀……这一张我记得,是在别墅里照的……小娜佳多好玩儿呀……这是爸爸流放回来以后照的……米佳还是中学生呢,谢廖沙像他像极了,特别是脸的上一部分……这是妈妈抱着玛露霞,那时候我还没出世呢……”

她发现,在这些照片当中没有一张托里亚的相片,不过她没有问,托里亚的相片在哪儿。

“好啦,夫人,”柳德米拉说,“应该伺候你进餐啦。”

“我的胃口很好,”叶尼娅说,“就像小时候那样,生气不影响吃饭。”

“好啊,那就谢天谢地。”柳德米拉说着,吻了吻妹妹。

二十四

叶尼娅在贴满五颜六色的伪装纸条的大剧院附近下了无轨电车,走上库兹涅茨桥,经过美术基金会展览馆,战前这儿曾经展出她熟悉的一些画家的作品,也展览过她的作品,可是她现在从这里走过,甚至都没有想起来。

她有一种奇怪的感觉。她的生活就像茨冈人玩的纸牌。一下子就变出了莫斯科。

她老远就看到卢比扬卡那座牢固的大楼,黑灰色花岗岩石墙。

“你好,尼古拉。”她在心里说。也许克雷莫夫已经感觉出她走近了,十分激动,却不知道为什么激动。

旧的命运成为她的新命运。似乎已经永远成为过去的,又成为她的未来。

宽敞的新接待室带有明亮的朝街玻璃窗,现在关闭着,仍然在老接待室里接待探望者。

她走进肮脏的院子,顺着一面旧墙朝半开着的接待室的门走去。接待室里一切都显得十分平常:桌子上有许多墨水印子,墙边摆着一张张木沙发,带有木板窗台的一个个小窗户,小窗户便是查询处。

似乎那座俯瞰卢比扬卡广场、斯列津巷、福尔卡索夫巷、小卢比扬卡的多层的石头大楼和这个小小的办公室没有什么联系。

接待室里的人很多,都是探望亲人的,多数是妇女,在各个窗口站着队,有的坐在沙发上,有一个老头子戴着厚玻璃眼镜在桌上填写一张表。叶尼娅看着这些老老少少、男男女女的一张张的脸,心想,他们所有的人的眼神、嘴的形态有很多相同之处,她如果在电车上、在大街上碰到这样的人,就会猜到是上库兹涅茨桥24号来的。

她向一名年轻的值班人员打听。这人穿着红军服装,不知为什么却不像红军。他问叶尼娅:“你是第一次来吧?”然后指了指墙上开的小窗户。叶尼娅站进队伍,手里拿着身份证,她的手掌和手指头都紧张得出了汗。站在她前面的一个戴圆帽的妇女小声说:

“如果在内部监狱没有,就要去马特罗斯·济什纳,然后去布特尔斯克,不过那里是在一定的日子按字母顺序接待的,然后上列弗尔托夫军事监狱,然后再到这儿来。我寻找儿子找了一个半月了。您上军事检察院去过吗?”

队伍移动得很快,叶尼娅心想,这不是好事,大概回答都是敷衍了事,很简短。但是,等到一个穿得很讲究的上了年纪的妇女走到窗口,却停顿了很久。大家小声传说着,值班人员亲自问情况去了,因为在电话里说不详细。那个妇女半侧身朝着队伍站着,眯着眼睛,那表情似乎在说,她在这儿也不认为自己和这群可怜的被捕者的亲属是平等的。

不一会儿,队伍又动起来。有一个年轻女子在离开窗口的时候,小声说:

“回答只有一句:不准送东西。”

旁边一个女子对叶尼娅解释说:

“这就是说,侦讯还没有结束。”

“那能不能见面呢?”叶尼娅问道。

“唉,您怎么啦!”那女子说,并且笑了笑叶尼娅的天真。

叶尼娅从来没有想到,人的脊背这样善于表情,这样明显地表达出人的精神状态。快要走到窗口的人们,不知为什么很特别地伸长了脖子,他们的脊背,连同那耸起的肩膀,那绷紧的肩胛骨,好像是在叫,在哭,在抽搭。

等到叶尼娅前面只有六个人了,小窗户啪的一声关上了,说是休息二十分钟。站队的人在沙发上和椅子上坐下来。

这里有母亲,有妻子;有一个上了年纪的男人,是一位工程师,他的妻子是对外文化协会的翻译,现在在监狱里;有一名女中学生,她的妈妈被捕了,她的爸爸在一九三七年就被判处剥夺十年通信自由;有一位瞎眼的老奶奶,是邻居领她来的,她是来打听儿子的消息;有一位外国女子,不大会说俄语,她是一名德国共产党员的妻子,身穿方格的外国大衣,手里提着一个花布提包,眼睛完全像俄罗斯老奶奶的眼睛。

这里有俄罗斯人,有亚美尼亚人,有乌克兰人,有犹太人,还有莫斯科郊区集体农庄的一名女庄员。在桌子上填表的那个老头子是季米里亚泽夫学院的教师,他上中学的孙子被捕了,显然是因为在晚会上说错了话。

在这二十分钟里,叶尼娅听到和了解了很多事情。

今天的值班员很好……在布特尔监狱不收罐头食品,一定要送大葱和大蒜—治坏血病……在这里,上星期三有一个人拿到了证件,在布特尔监狱关了他三年,一次也没有审问过,就放了……从被捕到进劳改营,一般要过一年左右……不能送好东西;在克拉斯诺普列斯宁羁押监狱,把政治犯和刑事犯关在一起,刑事犯见什么东西抢什么东西……不久前这儿来过一个妇女,她的老头子是一个很大的设计师,老头子被捕了,原来他在年轻时和一个女子有过短时间的关系,生了个男孩子,他一直付给她孩子的赡养费,可是从来没有见过那孩子,等那孩子长大成人,在前线上跑到德国人那边去了,所以设计师被判了十年徒刑,因为他是祖国叛徒的父亲……大部分是依据58—10条定罪进来的。反革命宣传罪,主要是因为瞎扯,随便发表议论……就在五一节前被捕了,一般在节日前抓人抓得特别多……这里来过一个妇女,有一个侦讯官往家里给她打电话,她忽然听到丈夫的声音……

说也奇怪,叶尼娅在这内部监狱的接待室里,倒是比在姐姐家洗过澡以后心里镇定些,轻松些。

有的妇女送的东西被收下,脸上露出幸福的神情。

有一个人用压得低低的声音在旁边说:

“他们说到一九三七年被捕的一些人的情况。都是胡乱说的。他们对一个妇女说,‘你丈夫活着,在干活儿呢。’可是她第二次来,还是那个值班的回答她说:‘你丈夫在一九三九年死了。’”

终于小窗户里面的人抬起眼睛看着叶尼娅了。这是一张普普通通的办事人员的脸,也许他昨天还在消防队办公室里工作,明天,如果上级有命令,他又会到授奖科填报表了。

“我想打听一个被捕的人—克雷莫夫·尼古拉·格里高力耶维奇。”叶尼娅说。她觉得,就连不认识她的人都会察觉,她说话的声音变了。

“什么时候被捕的?”值班人员问。

“在十一月里。”她回答说。

值班人员交给她一张查询表,说:

“您填好,交给我,不用再排队。明天来听回话。”

他在给她表的时候,又看了她一眼,这匆匆的一瞥不是普通办事员的目光,而是克格勃人员的精明和搜索的目光了。

她开始填表,手指头哆嗦着,就像刚才坐在这椅子上的那个季米里亚泽夫学院的老头子。

在和被捕人关系一栏内她写的是“夫妻”,而且用粗粗的笔划描了描。

她把填好的表交去以后,坐到沙发上,把身份证放进手提包。她从手提包的这一格又换到那一格,重放了好几次,她明白了,她是不愿意离开这些站队的人。

此时此刻她只希望一点:让克雷莫夫知道她在这里,知道她为了他已经扔掉一切,看他来了。

但愿他能知道她在这儿,在他跟前。

她在街上走着,暮霭渐渐浓了。她这一生一大半是在这座城市里度过的。但是举行画展的日子,看戏、下饭馆、别墅休养、听交响乐的日子离开她太远了,似乎她没有过过那种日子。斯大林格勒,古比雪夫,诺维科夫那好看的、有时她觉得英俊无比的脸已成为过去。剩下的只有库兹涅茨桥24号的接待室,她觉得她好像是在一个陌生城市的陌生街道上走着。

二十五

维克托一面在外间脱套鞋,和老保姆打招呼,一面看着契贝任房间的半开着的门。

老保姆伊凡诺芙娜一面帮维克托脱大衣,一面说:

“进去吧,进去吧,他在等你呢。”

“娜杰日达·菲道罗芙娜在家吗?”维克托问。

“不在家,昨天她带着侄女上别墅去了。维克托·帕夫洛维奇,您不知道战争很快就要结束了吗?”

维克托对她说:

“听说,有人叫朱可夫的司机问问朱可夫,战争什么时候结束。朱可夫坐上汽车,却问起司机:‘你能不能说说,这战争什么时候结束?’”

契贝任出来迎住维克托,说:

“老人家,不要把我的客人抢去。你请你的客人好啦。”

维克托每次到契贝任这儿来,都感到很兴奋。现在虽然他心里十分苦恼,仍然别有一种已经不习惯的轻松感。

往常维克托走进契贝任的书房,打量着一个一个的书架,总要用开玩笑的口吻说说《战争与和平》里的一句话:“噢,在写呢,没有玩。”

现在他也说:

“噢,在写呢,没有玩。”

书架上十分凌乱,很像车里亚宾斯克工厂车间里那种表面上的混乱。

维克托问:

“您的孩子们有信来吗?”

“收到大儿子的来信,小儿子在远东。”

契贝任握住维克托的手,借助默默无言的握手表达了不需要用话说的心情。老保姆伊凡诺芙娜也走到维克托跟前,吻了吻他的肩头。

“维克托·帕夫洛维奇,您有什么新闻吗?”契贝任问道。

“我的消息,也就是大家的消息。斯大林格勒的消息。现在毫无疑问:德国佬要完蛋了。我个人却没有什么好消息,相反,全是坏消息。”

维克托对契贝任说起自己的倒霉事。

“现在朋友们和老婆都劝我检讨。把自己的正确说成错误。”

他一个劲儿地说自己的事,说了很多。一个害重病的病人,总是日日夜夜想着自己的病。

他撇了撇嘴,耸了耸肩膀。

“我常常想起咱们说过的关于发面和浮上表面的脏东西的那番话……在我周围从来没出现过这样多的肮脏东西。而且不知为什么这一切偏偏出在胜利的日子里,这就特别可恼,特别使人难以容忍。”

他看着契贝任的脸,问道:

“依您看,这不是偶然的吧?”

契贝任的脸非常奇怪:很平常,甚至很粗陋,高颧骨,翘鼻子,像一张庄稼汉的脸。尽管如此,却又十分文雅,十分清秀,伦敦的绅士开尔文勋爵都望尘莫及。

契贝任忧郁地回答说:

“等到战争结束了,咱们再说说,什么是偶然的,什么不是偶然的。”

“也许,到那时候猪都会把我吃掉了。明天就要在学术委员会会议上拿我开刀了。就是说,已经在院部和党委会上把我结果了,只是在会议上宣布一下,说这是人民的声音,群众的要求。”

维克托在和契贝任说话的时候,觉得自己很奇怪:他们谈的是维克托生活中的痛苦的事情,不知为什么心里却很轻松。

“我倒是认为,现在是用银盘子,也许是用金盘子捧着你呢。”契贝任说。

“这为什么?我把科学引进了学究式的抽象概念的泥坑,使科学脱离了实际嘛。”契贝任说:

“是啊,是啊。很奇怪!您知道,男人是爱女人的。女人是男人的人生目的,是男人的幸福、希望、欢乐。但是不知为什么男人总要隐瞒着,这种感情不知为什么成了不体面的东西,男人必须说,他和女人睡觉,是因为她给他做饭,补袜子,洗衣服。”

他把两手举在自己的面前,张开手指头。他的手也是很奇怪的:是一双像铁钳一样有力的干活儿的手,同时又很像一双贵族的手。契贝任忽然发起火来:

“可是我不害臊,我需要爱情并不是为了做饭!科学的价值就在于它为人类造福。可是我们科学院的一些家伙却奉命说:科学是实际的女佣,要依照谢尔巴科夫的家规干活儿:‘您有什么吩咐?’只能准许这样!……不对!科学发明本身有其崇高的价值!科学发明可以改善人,其作用超过蒸汽锅炉、涡轮机、航空和从诺亚时代到我们今天的全部冶金工业。改善心灵,心灵!”

“我倒是赞成您的说法,不过恐怕斯大林同志不赞成。”

“没什么,没什么。这就是事情也有另一个方面。今天麦克斯韦的抽象理论到明天会变为军用无线电呼号。爱因斯坦的引力场理论、薛定谔的量子力学和玻尔理论体系明天就成为最强大的实际力量。这是应该可以理解的。这道理极其简单,就连笨鹅都会懂得。”

维克托说:

“不过,您也曾亲身体验到,政治领导者不愿承认今天的理论明天会变为实际。”

“不,倒是有些相反,”契贝任慢慢地说,“我自己不愿意领导研究所,正是因为我知道:今天的理论明天会变为实际。可是很奇怪,非常奇怪,我原来就认为,希沙科夫会因为发现核反应过程受到提拔。而这种事没有您是不行的—说准确点儿,不是我原来认为这样,而是一直认为是这样。”

维克托说:

“我不理解您辞去研究所职务的动机。您的话我不明白。但是我们的领导向研究所提出了曾经使您担心的任务,这是很明白的。领导者往往在一些比较明显的事情上犯错误。比如伟大领袖一直在加强同德国人的友好关系,而且在战争开始前几天还用特快列车给希特勒送橡胶和其他战略原料。而在我们的事业中……伟大的政治家出错儿就更不算什么。而在我的生活中,一切都翻了个儿。我在战前的著作都是接触实际的。比如,我在车里雅宾斯克就常常上工厂去,帮助安装电子仪器。可是在战争时期……”

他带着快活而无可奈何的神气把手一挥。

“我走进了深深的密林。有时不知是害怕,还是觉得不自在。真的……我想建立核子相互作用物理学,可是这样引力、质量、时间就不存在了,而没有实体的空间也要分为两个,只有磁力意义。在我的实验室里有一个很有才能的年轻人,就是萨沃斯季扬诺夫,有一次我和他谈起我的研究。他问我这一点,又问那一点。我对他说:这还不是理论,这是提纲和一些想法。第二空间—这是方程中的指数,不是实有的。对称只是存在于数学方程中,我不知道,基本粒子的对称是否与之相符。数学答案走到了物理学前面,我不知道,基本粒子物理是否愿意挤进我的方程。萨沃斯季扬诺夫听着,听着,然后说:‘我想起大学里的一位同学,他有一次解一道方程式解乱了,就说:这不是科学,这是一群瞎子集合在荨麻地里……’”

契贝任笑起来。

“确实很奇怪,您自己无法认识到自己的数学方程在物理学方面的意义。就像有一只来自奇异国度的猫,首先出现猫的笑容,然后才出现猫本身。”

维克托说:

“可是,我的天呀,我在内心里却相信:人类生活的主轴恰恰就在这儿。我决不改变我的观点,决不后退。我从来不放弃自己的信仰。”

契贝任说:

“我知道,您离开实验室会有什么样的心情,您的数学和物理学的关系眼看着就要在实验室里显现出来。这是很痛苦的,不过我为您感到高兴,正直的心不会磨灭。”

“只要不把我磨灭掉就行啦。”维克托说。

伊凡诺芙娜送进茶来,把桌上的书推开,腾出地方。

“哦,是柠檬呀。”维克托说。

“您是贵客嘛。”伊凡诺芙娜说。

“我啥也算不上。”维克托说。

“喔,喔,”契贝任说,“干吗要这样?”

“真的,明天就要对我开刀了。我感觉到了。到后天我会怎样呢?”

他把茶杯朝自己跟前移了移,用茶匙在小碟子边上敲着自己绝望心情的进行曲,心不在焉地说:“哦,柠檬呀。”他觉得用同样的语调把这话说了两遍,感到不好意思起来。

他们沉默了一阵子。契贝任说:

“我想和您谈谈一些想法。”

“我很愿意听。”维克托心不在焉地说。

“其实,不过是空想……您知道,关于宇宙无限的概念,现在已经成了人人知道的道理。总星系总有一天会成为某一个俭省的人就着喝茶的糖块,而电子或中子则会成为人类可以纵横驰骋的世界。这已经是小学生都知道的了。”

维克托点了点头,在心里说:“的确是空想。今天老头子有点儿不正常。”同时他想象着明天会议上希沙科夫的样子。“不,不,我不去。要是去,就要检讨,或者争论政治问题,那就等于自杀……”他轻轻打了一个呵欠,想道:“这是心力衰竭。人打呵欠都是因为心脏有毛病。”

契贝任说:

“能够限制无限性的,恐怕只有上帝……因为在宇宙界限之外,必须承认有神的力量。不是这样吗?”

“是这样,是这样。”维克托说。又在心里说:“德米特里·佩特罗维奇呀,我可是没有心思谈哲学,人家要抓我坐牢了。必然的事嘛!再说,我在喀山又和那个马季亚罗夫说直话说了不少。也许也就是暗探,也许是逼着他来套别人的话。我一切都很糟糕。”

他看着契贝任,契贝任注视着他那似乎很用心的目光,继续说:

“我以为,限制宇宙无限性的界限是有的,那就是生命。这界限不在爱因斯坦的曲率范围,而是在生命的对立性和死的物质中。我觉得,可以给生命下定义为自由。生命就是自由。生命的基本原则就是自由。自由与受奴役,生命与死的物体—界限就在这里。再就是,我以为,自由一旦出现了,就开始了自己的演化。演化分两种途径进行着。人比起原生动物有更多的自由。生物世界的整个演化过程就是从自由的最小限度到最高限度的运动过程。这就是生命形式演化的实质。最富于自由的形式,便是最高的形式。这是演化的第一分支。”

维克托看着契贝任,沉思起来。契贝任点了点头,似乎是对他的用心倾听表示赞许。

“还有演化的第二条分支,我以为是数量方面的演化分支。今天,如果一个人的重量算五十公斤的话,全人类的重量就有一亿吨了。这比以前,比如说,一千年前,多得太多了。活物的量依靠死物体供应的养料会越来越多。地球会渐渐充满生物。人类住满了沙漠,住满了北极地区,就要开始进入地下,地下城市和场地的地面会越来越深。地上生活的人就要成为优越的了。然后住满一个又一个行星。如果想象到由于时间无限而生命演化不断,那么将来死物质变生命的过程会在银河系范围内进行。物质将由死的变成活的,变成自由。宇宙就活了,世界上的一切都成了活的,也就是都成了自由的。自由、生命就会战胜奴役。”

“是的,是的,”维克托说,并且笑了笑,“可以拿积分为例。”

“实质就是这样,”契贝任说,“我研究过星体演化,可是我懂得,活的黏液留下的小小灰斑都是轻易动不得的。演化的第一分支,从低级到高级,那是了不起的。将会出现具有一切天然特点的人:到处都能去,什么都知道,什么都能做得到。最近一百年内会解决物质变能的问题和创造活物质的问题。在战胜空间和取得极限速度方面也会有相应的发展。在比较遥远的将来,会朝着掌握能的最高形式,即掌握精神能的方向前进。”

维克托忽然不再觉得契贝任说的一切是空谈了。原来,他不赞同契贝任说的话。

“人能够通过仪器的显示使整个总星系的理性生物的精神活动的内容、节奏具体化。光需要几百万年才能穿越的空间,精神能霎时间就能穿越。上帝的特征—无所不在,将成为精神的成就。不过,人能够与上帝并驾齐驱之后,还不会就此停止。人要解决上帝都无法解决的问题。人要建立和整个宇宙、和另外的空间、和另外的时间的高级理性生物的联系,人类的整个历史与另外的时间相比,只是似有若无的短暂的一闪。人还要建立和微观宇宙的生命的有意识联系,微观宇宙生命的演化,在人类看来只是短短的一瞬。那将是完全消灭时间与空间障碍的时代。人类就会看不起上帝了。”

维克托点了点头,说:

“德米特里·佩特罗维奇,开头我听着您的话,心里在想,我哪儿有心思听哲学议论,人家要抓我去坐牢了,还谈什么哲学。可是我一下子就忘记了科甫琴科,也忘记了希沙科夫、贝利亚同志,忘记了明天也许会把我赶出实验室,后天也许就会把我关起来。不过,您要知道,我听着您的话,不是感到高兴,而是感到失望。您把我们说得很了不起,神话中的大力士赫拉克勒斯在我们面前成了可怜的小矮子。可是就在这时候,德国人就像宰疯狗一样在杀犹太老人和孩子,我们也发生过一九三七年的事,发生过普遍集体化的事,把几百万不幸的农民流放,饥饿,人吃人……您要知道,我总觉得从前一切都单纯些,明朗些。经历了种种可怕的不幸与灾难之后,一切都变得复杂了,难以理解了。人会看不起上帝,可是能不能也看不起恶魔,战胜恶魔?您说,生命就是自由。可是在集中营里的人是不是这样想?生命遍布于宇宙之后,会不会用自己强大的力量建立奴役制,其可怕程度超过您说的对死物质的奴役?您还是告诉我,将来的人在善良方面能不能超过耶稣?这是最主要的!请告诉我,如果无所不能、无所不在的人类仍然带有我们今天的刚愎自用和利己主义,包括阶级的、民族的、国家的、个人的利己主义,人类的强大将给世界带来什么?那时的人会不会把全世界变成总星系规模的集中营?就是说,就是说,请告诉我,您是否相信善良、道德、慈悲心的进化?人是不是在这方面也会进化?”

维克托很抱歉地皱了皱眉头。

“对不起,我一定要请您回答这个问题,这个问题也许比咱们谈的数学方程还要抽象。”

“这个问题并不那么抽象,”契贝任说,“因此也反映在我的生活中。我决定不参加原子裂变的研究。人类要过明智的生活,今天的善良和好心肠是不够的,您说的也是这一点。如果人一旦掌握了原子内部能量的力量,会怎么样呢?今天精神能还处在很可怜的水平。不过我相信未来!我相信,日益发展的不只是人的力量,还有仁爱心,还有人的精神。”

他看到维克托脸上的表情,感到惊讶,就沉默下来。

“我想过,想过这一点,”维克托说,“有一次我也觉得十分可怕!我们在这儿担心人类的不完美。可是,比如说,在我的实验室里,还有谁考虑这一点呢?索科洛夫吗?他有很了不起的才能,可是胆子太小,在国家的力量面前低声下气,认为一切权力都是天生的。马尔科夫吗?他完全置身于善、恶、仁爱、道德等问题之外。他有实干的才能。他解决科学问题,就像棋手研究棋局。我对您说过的萨沃斯季扬诺夫吗?他是一个招人喜欢的、很聪明、很出色的物理学家,但他又是一个所谓没有头脑的轻浮小伙子。他把一大堆相识的姑娘的游装照片带到喀山,他讲究穿戴,喜欢喝酒、跳舞。对于他来说,科学就是运动。解决问题,弄清现象,就是创运动纪录。最要紧的是,不能被欺骗和利用!可是,就连我现在也没有想这些问题。在我们的时代,从事科学研究的应当是具有伟大心灵的人,应当是先知和圣者!可是现在研究科学的却是有实干才能的人、象棋专家、运动员。他们不知道自己在创造什么。您怎么样?可是您不过是您。柏林的契贝任就不会拒绝研究中子!那又怎么办?我呢,我又怎么样?我原来觉得一切都很简单,可是现在觉得不是这样,不是这样……您知道,托尔斯泰曾经认为自己的天才作品是无聊的游戏。我们物理学家进行创作不是靠天才,而是使出全身的力气、全部的心血。”

维克托的睫毛不住地眨巴起来。

“我到哪儿去找信心、力量、百折不挠的精神呀?”他很快地说。他的声音中出现了犹太口音。“啊,我能对您说什么呀?您懂得我现在的苦楚,现在他们整我,只是因为我……”

他没有说完,很快地站了起来,茶匙掉到地上。他哆嗦着,两只手都在哆嗦。

“维克托·帕夫洛维奇,请您不要难过,”契贝任说,“还是来谈点儿别的吧。”

“不,不,请原谅。我要走了,我的头有点儿疼,对不起。”他开始告别。

“谢谢,谢谢。”维克托说,也不看契贝任的脸,觉得自己再也控制不住激动的心情。

维克托朝楼下走去,泪水顺着脸颊扑簌簌流着。

二十六

维克托回到家里,家里人都已经睡了。他觉得,他会在桌前一直坐到天亮,把自己的检讨书写了又写,看了又看,再考虑第一百次:明天他去不去研究所。

在长长的回家的路上,他什么也没有想:没有想在楼梯上流泪,没有想因为忽然激动起来中断了他和契贝任的谈话,没有想他的可怕的明天,也没有想揣在上衣旁边口袋里的给妈妈的信。安静的夜晚的街道使他的心情也安静下来,他的头脑空空的,好像一眼可以看透,可以穿过似的,就像夜晚的莫斯科空旷无人的林荫道。他不难过,不因为刚才流泪感到不好意思,不担心自己的命运,不盼望好的结局。

早晨,维克托朝浴室走去,可是浴室的门从里面锁上了。

“是你吗,柳德米拉?”他问道。

他听到叶尼娅的声音,啊呀了一声。

“我的天,叶尼娅,你怎么在这儿呀?”他说。因为太突然,他呆呆地问道:“柳德米拉知道你来了吗?”

叶尼娅走出浴室,他们拥抱起来。

“你气色不大好啊。”维克托说过这话,接着又说:“我这是随便说的。”

她接着就在走廊里对他说了克雷莫夫被捕的事和她来莫斯科的目的。

他很吃惊。但是他听到这个消息之后,觉得叶尼娅此行尤其难得。假如叶尼娅来时喜气洋洋,一心想的是自己的新生活,他就不会觉得她这样可亲可爱了。

他和她说话,向她问这问那,一面不住地看钟。

“这多么荒唐,多么不可思议,”他说,“你倒是想想尼古拉和我谈的许多话,他常常纠正我的思想。可是你瞧!我满脑子异端邪说,却还自由自在,他这个虔诚的共产党员倒被捕了。”

柳德米拉说:

“维克托,你要注意:餐室里的钟慢十分钟。”

他嘟哝了一句,便朝自己房里走去,在经过走廊的时候,又朝挂钟看了两次。

学术委员会会议定于上午十一时开始。他虽然置身于许多习惯了的东西和书籍之中,却以超乎寻常、近似幻觉的敏锐感清清楚楚地感觉到研究所里的紧张和忙碌。十点半了。

大概索科洛夫开始脱工作服了。萨沃斯季扬诺夫小声对马尔科夫说:

“嗯,看样子,咱们的疯子拿定主意不来了。”

古列维奇挠着厚厚的后脑勺,朝窗外看了看:一部小汽车来到研究所大楼门前,希沙科夫头戴呢帽、身披长长的牧师式斗篷走出汽车。随后又有一部小汽车来到,是年轻的巴季因。科甫琴科顺着走廊走来。会议厅里已经有十五六个人,都在看报纸。他们提前来,因为知道今天的人很多,要先占一个好点儿的位子。斯维琴和研究所党委书记拉姆斯科夫带着一副煞有介事的神气站在党委会门口。白发苍苍的老院士普拉索洛夫拿眼睛朝上望着,在走廊里缓缓走着;他在这一类的会议上说话特别鄙俗。初级研究员们成群成堆地走着,闹哄哄的。

维克托看了看表,从抽屉里拿出自己的检讨书,装到口袋里,又看了看表。

他可以去参加学术委员会会议,不检讨,一声不响地坐一坐……不行……既然去了,就不能不说话,既然说话,就得检讨。可是如果不去,就把自己所有的路切断了……

别人会说:“他没有勇气……有意和群众对立……是政治上的挑战……这样一来,问题的性质就变了……”他从口袋里掏出检讨书,并没有看,马上又装进口袋里。这检讨书他反复看过几十遍了:“我认识到,我对党的领导表示不信任,这种行为不符合苏联人的行动准则,所以……还有,我在研究中没有意识到自己偏离了苏联科学的光辉道路,不自觉地对抗……”

他老是想再看看检讨书,可是他把检讨书一拿到手里,就觉得每一个字他都熟悉得不得了……共产党员克雷莫夫进了卢比扬卡监狱。他维克托又喜欢怀疑,又怕斯大林的残酷,还议论过自由,议论过官僚作风,再加上现在被看做政治问题的事,早就应该被送到科雷马去了……

最近几天他越来越害怕,似乎他就要被捕了。要知道,一般都不是开除公职就完事儿的。先是批判,然后开除,然后抓起来。

他又看了看表。这时大厅里应该已经坐满了人。大家都朝门口看着,小声说着:“维克托·施特鲁姆还没来呢……”有人说:“快到中午了,维克托还没来呢。”希沙科夫坐到主席位子上,把皮包放到桌上。科甫琴科旁边还站着一名女秘书,女秘书是拿着紧急文件来请他签字的。

维克托想到会场上几十个人焦急而不耐烦地等待着,也急得不得了。大概,在卢比扬卡监狱里,在负责他的专案的人的房子里,有些人也在等着:他怎么还没来呀?他仿佛看到中央委员会也有一个面色阴沉的人:怎么他还不来呀?他仿佛看到许多熟人都在对家里人说:“真是疯子。”柳德米拉在心里责备他:托里亚献出生命保卫国家,可是维克托竟在战争时期和国家争执起来。

过去每当他想起他和柳德米拉的亲戚中有那么多被镇压、被流放的人的时候,他总是自我安慰地想:“如果他们问我,我会说:我的亲戚不都是这样的人,还有克雷莫夫呢,他也是我的近亲,是有名的共产党员,老布尔什维克,地下工作者。”

可是现在你瞧克雷莫夫!如果那里面开始审问他,他就会想起维克托的许多牢骚怪话。不过,克雷莫夫跟他也不是那么亲近了,因为叶尼娅已经和他分手了。而且,他和他也没有说过多么危险的话,因为在战前维克托还没有什么特别尖锐的意见。啊,要是问起马季亚罗夫呢?

几十、几百种拉力、压力、推力、撞力合成一种合力,似乎要把他的肋骨折断,把他的头盖骨击碎。

什托克曼博士的话“孤独的人是刚强的”是不对的……孤独算什么刚强:他偷偷地朝四下里打量着,带着自嘲和无可奈何的表情匆匆忙忙地结起领带,把检讨书放到新礼服的口袋里,穿起崭新的黄皮鞋。

就在他穿好衣服站在桌边的时候,柳德米拉走进门来,她一声不响地吻了吻他,就出去了。

不,他不宣读自己的检讨书!他要说说心里的实话:同志们,朋友们,我听到你们的话十分难过,我十分难过地在想,在艰苦奋战取得斯大林格勒战役转折的大喜的日子里,我怎么会这样孤立,怎么会听到自己的同志、兄弟和朋友们的愤怒的谴责……我向你们发誓:我不吝惜全部心血、全部力量……是的,是的,是的,他现在知道要说些什么……快点儿,快点儿,他还来得及……同志们……斯大林同志,我有过错误,到了深渊的边沿,才看清自己的错误。他要说的是他内心深处的话!同志们,我的儿子就牺牲在斯大林格勒城下……他朝门口走去。

就在这最后一分钟里,他最后拿定了主意,剩下的只是快点儿赶到研究所,把大衣脱在存衣室里,走进会议厅,听着几十个人激动的低语声,打量着一张张熟悉的脸,说:“同志们,我请求发言,我要说说这些天来我所想的和我感觉到的……”

但也正是在这几分钟里,他动作缓慢地脱掉上衣,搭在椅背上解下领带,卷了卷,放到桌子边上,坐下来,开始解鞋带儿。

他顿时充满轻松感与清白感。他坐着,很平静地沉思起来。他不信上帝,但是不知为什么此时此刻他觉得仿佛上帝在看着他。他这一生从来没有体验过这样幸福同时又这样安宁的心情。再没有什么力量能够夺去他的正确性了。

他想起妈妈。也许,当他不由自主地改变主意的时候,妈妈在他跟前。因为在这之前一分钟,他还真想去做违心的检讨呢。当他下决心做出最后决定的时候,没想到上帝,也没想到妈妈。但是上帝和妈妈是和他在一起的,尽管他没有想到。

“我心里坦然,我很幸福。”他想。

他又想象起会议的情形,想象着很多人的脸,仿佛听到发言者的声音。

“我心里多么痛快,多么舒畅呀。”他又想道。

他好像从来没有这样认真思索过自己的一生,这样认真想过亲近的人,从来没有这样认真来了解自己和自己的命运。柳德米拉和叶尼娅走进他的房里。柳德米拉看见他脱了外衣,只穿着袜子,敞着衬衣领口,不禁像个老奶奶似的啊呀叫了一声。

“我的天,你没有走呀!那现在会怎么样?”

“我不知道。”他说。

“不过,也许还不迟吧?”她说。然后看了看他,又说:“我不知道,我不知道!你是成年人啊。可是,你在决定这样的问题的时候,应当考虑的不光是自己的原则。”

他没有作声,后来叹了一口气。

叶尼娅说:“姐姐!”

“噢,好吧,好吧,”柳德米拉说,“听天由命吧。”

“是的,柳德米拉,”维克托说,“所以咱们还要慢慢走着瞧呀。”

他用手捂住脖子,笑着说:

“对不起,叶尼娅,我没系领带。”

他看着柳德米拉和叶尼娅,觉得他现在才真正懂得,生活在人世上是多么不容易、多么不可轻视的事,和亲人的关系有多么重要。他明白了,生活会照常进行下去,他又可以发火,可以为琐碎事操心,可以生妻子和女儿的气了。

“好啦,我的事谈够了,”他说,“叶尼娅,咱们来下下棋,你可记得,那次你一连赢了我两局?”

他们把棋摆好,维克托是白棋,第一步走的是王侧小卒。

叶尼娅说:

“尼古拉用白棋往往都是先走王棋旁边的卒子—啊,今天上库兹涅茨桥,不知道会给我什么回话呀?”

柳德米拉弯下身,把便鞋推到维克托脚底下。他也不看,想把脚插进鞋里,柳德米拉带着抱怨的意味叹了一口气,便跪到地上,把便鞋给他穿到脚上。他吻了吻她的头,漫不经心地说:

“谢谢,柳德米拉,谢谢。”

叶尼娅还没有走第一步,就摇了摇头。

“哼,我真不懂。托洛茨基问题是老问题了。一定是出了什么事儿,可是什么事儿呢?”

柳德米拉一面摆正白棋,一面说:

“昨天夜里我几乎一夜没有睡。那样忠实、思想水平那样高的共产党员呀。”

“昨天夜里,你可算睡得很好,”叶尼娅说,“我醒了好几次,你都是在打呼噜。”

柳德米拉生气了:

“胡说,我简直都没有合眼。”

像是在回答那个让她自己不安的问题,她对丈夫说:

“没关系,只要不逮捕,就没关系。如果什么都不给你,我不怕,咱们可以卖东西,可以上别墅去,我到市场上去卖草莓。我还可以到中学里去教化学。”

“别墅不会再让住了。”叶尼娅说。

“难道你们不明白,尼古拉什么罪也没有?”维克托说。“不是那种人。”

他们面对棋盘坐着,看着棋子,看着只走了一步的唯一的一个小卒,说着话儿。

“叶尼娅,好妹妹,”维克托说,“你是凭良心行事。要知道,这是一个人最可贵的东西。我不知道生活会带给你什么,但我相信,你现在所作所为对得起良心。我们最大的不幸,就是我们所作所为不凭良心。我们说的,不是我们所想的。感觉是一样,做的却是另一样。你该记得,托尔斯泰说到死刑,说过:‘我不能沉默!’可是在一九三七年处死成千上万无辜的人的时候,我们却沉默。沉默还算好的呢!还有不少人闹闹哄哄大加赞扬呢。在普遍集体化的可怖时期,我们也沉默。我以为,我们还谈不上社会主义,社会主义不仅仅是在于重工业。社会主义首先要有凭良心的权利。剥夺人的凭良心的权利,是非常可怕的。如果一个人能够凭良心行事,会感到十分幸福的。我替你高兴。你是凭良心行事的。”

“维克托,你不要像佛陀一样说教了,不要把糊涂人弄得更糊涂,”柳德米拉说,“良心有什么用?断送自己的幸福,让一个好人痛苦,这又对克雷莫夫有什么好处?我不相信,等到把他放出来,他会有什么幸福。在他们分手的时候,他是好好儿的嘛。她的良心是对得起他的。”

叶尼娅拿起王棋,在空中转悠了几下,看了看贴在棋子底下的呢子,又放回原处。

“姐姐,”她说,“还能有什么幸福。我想的不是幸福。”

维克托看了看表。他觉得钟表的表盘很平静,长短针似乎带着睡意,十分安宁。

“这会儿他们在那儿讨论得正带劲儿呢。在拼命地批判我呢,不过我既不气,又不恼。”

“要是我,就打那些不要脸的家伙的嘴巴,”柳德米拉说,“一会儿管你叫科学的希望,一会儿照你吐唾沫。叶尼娅,你什么时候上库兹涅茨桥?”

“四点钟。”

“我给你做午饭,吃了再去。”

“今天咱们午饭吃什么?”维克托说。又笑着补充说:“两位女同胞,你们可知道,我对你们有什么要求?”

“知道,知道。你是想干你的事情。”柳德米拉说着,站了起来。

“要是别人,在这样的日子,早气得发疯了。”叶尼娅说。

“这是我的软弱,不是刚强,”维克托说,“昨天契贝任和我谈了很多科学上的问题。可是我另有看法,另有一种观点。就像托尔斯泰那样:他怀疑,感到苦恼,不知道文学对人是否有用,不知道他写的书对人是否有用。”

“哼,你要知道,”柳德米拉说,“你想在物理方面写出《战争与和平》,还早着呢。”

维克托感到十分尴尬。

“是的,是的,柳德米拉,你说得很对,我是胡乱说说。”他嘟哝说,并且不由自主地用责备的目光看了看妻子:天哪,就是在这样的时候,还要着重指出我说的每一句错话呀。

他又剩了一个人。他看起昨天他做的记录,同时在想今天的事情。

为什么柳德米拉和叶尼娅离开他的房间,他就舒畅了?有她们在场,他产生了一种感觉,感觉到自己是虚伪的。他提议下棋,他表示希望干事情,其中都有虚伪性。显然,柳德米拉管他叫佛陀,正是感觉出这一点。而且他在赞美良心的时候,也感到他的声音有虚伪、不自然的意味。他怕别人怀疑他是自我欣赏,就尽可能说一些很平常的话,但是这样故意表示平常,就像在讲道台上布道一样,也有其虚伪性。

有一种模模糊糊的不安使他放不下心来,他不明确:他缺少什么。

他几次站起来,走到门口,倾听柳德米拉和叶尼娅说话的声音。

他不想知道他们在会议上说些什么,不想知道谁的发言特别激烈和凶狠,不想知道他们做了什么样的决议。他要给希沙科夫写一封短短的信,说他病了,最近几天不能上研究所去。以后就不需要这样解释了。能做到的,他总是想尽可能做到。其实,已经没用了。为什么近来他这样怕逮捕?他没干什么坏事呀。他只是随口乱说。而且,其实没说什么了不起的坏话。他们是知道的。但是心里还是惶惶不定,他忍不住朝门口看了看。也许,他是想吃饭?大概,今后不能享受按级别供应了。也不能进高级食堂了。外室里响起轻轻的门铃声,维克托急忙跑出去,朝着厨房高声说:

“柳德米拉,我去开门。”

他把门开了。在幽暗的外室里,玛利亚的一双惶惶不安的眼睛看着他。

“啊,就是的,”她小声说,“我就知道您不会去。”

维克托帮她脱大衣,他的手感觉到传到大衣领子上的她的脖子和后脑勺的温暖,这时他忽然领悟到:他刚才就是在等她的,因为预感到她要来,所以他倾听,并且一再地朝门看。

他明白这一点,因为他一看到她,马上就感到轻松和很自然的喜悦。每次他在傍晚带着沉重的心情从研究所回来,惶惶不安地打量着行人,注视着电车和公共汽车窗外一张张女人的脸,他就是希望遇到她。每当他回到家里,问柳德米拉:“有谁来过吗?”他就是想知道她是不是来过。早就是这样了……她来了,他们说话,开玩笑;她走了,他似乎就把她忘了。当他和索科洛夫说话的时候,柳德米拉说她问候他的时候,她都会出现在他的头脑中。似乎除了他看到她的时候和说她是多么可爱的女子的时候,她都不存在。有时,为了逗引柳德米拉生气,他还说她的好朋友没有读过普希金和屠格涅夫的作品。

他和她在逍遥公园散过步。他看着她,觉得很愉快;他很喜欢她能很快地明白他的话,一听就懂,从来不会理解错;她听他说话时那种孩子般的倾注神情,使他很感动。后来,他们分手,他就不想她了。后来他走在大街上,又想起她来,后来又忘了。

现在他感觉到,她本来一直和他在一起,只是他觉得好像她不在罢了。在他没有想着她的时候,她也和他在一起。他看不见她,他没有想起她,可是她依然和他在一起。他无意去想她,就感觉她不在;却不知,即使在不想她的时候,也总是因为她不在而心神不宁。可是这一天,当他对自己、对和他一起生活而又各有各的生活的人了解得特别深刻的时候,他凝视着她的脸,明白了自己对她的感情。他看着她,感到高兴:那种经常使人惆怅的她不在的感觉一下子消失了。他因为有她和他在一起,感到轻松起来,他不再下意识地感觉她不在了。他近来总是感到自己孤单。他在和女儿、和朋友、和契贝任、和妻子说话的时候,都觉得自己孤单。可是他只要一看见玛利亚,孤单就消失了。

而且这一发现并没有使他吃惊,这是很自然的、无可争辩的。可是在一个月前,两个月前,在喀山的时候,他怎么不明白这简单又无可争辩的事呢?

所以很自然,当他今天特别强烈地感觉到她不在的时候,他的感情就要从深处涌到表面上来,让他意识到它的存在。

因为无论如何对她是无法隐瞒的,所以就在外室里,他带着一副愁容望着她说:

“我一直以为,我像狼一样饿了吧,就一个劲儿地朝门口看,是不是马上来叫我吃饭。谁知我是在等待:玛利亚是不是来了!”

她什么也没有说,就好像没有听见,便走了进来。

她和初次见面的叶尼娅一起坐在沙发上,维克托把目光从叶尼娅脸上移到玛利亚脸上,又移到柳德米拉脸上。两姐妹多么美呀!这一天柳德米拉的脸特别好看。有损她的美的阴沉表情不见了。她的一双明亮的大眼睛露出温柔而惆怅的神气。叶尼娅撩了撩头发,显然是感觉出玛利亚在看她。玛利亚说:

“对不起,不过我没想到一个女子有您这样美,我从来没看到像您这样的容貌。”

她说过这话,脸红了一下。

“玛利亚,你再看看她的手,”手指头柳德米拉说,“还有脖子,还有头发。”

“还有鼻子眼儿,鼻子眼儿。”维克托说。

“怎么,你们拿我当一匹卡巴尔达马呀?”叶尼娅说。“我可不爱听这些。”

“马儿不喜欢这马料。”维克托说。虽然这话的意思不太明确,还是引起了笑声。

“维克托,你是想吃饭了吧?”柳德米拉说。

“是的,是的,不,不。”维克托说。他看到玛利亚的脸又红了。就是说,她听见他在外室里说的话了。

她坐在那里,像只麻雀,灰灰的,瘦瘦的,凸出的不高的额头上面是梳得整整齐齐的、像人民教师一样的头发,穿着肘部补过的针织上衣,维克托却觉得她说的每一句话都充满智慧、善意和文雅意味,每一个动作都显得很优雅、很温柔。

她没有说起学术委员会的会议。她问到娜佳的事,她向柳德米拉借托马斯·曼的《魔山》,向叶尼娅询问薇拉和她的小孩子,还问弗拉基米罗芙娜从喀山的来信说些什么。

维克托没有一下子就明白,玛利亚找到的是唯一正确的谈话方法。她似乎在强调,没有什么力量能够使人不能继续做人,最强大的国家也不能闯进父子、兄弟姐妹的圈子,在这不愉快的日子里,她就这样来赞美和她坐在一起的人,因为国家未能闯进他们的圈子,他们就有权不谈外部强加给他们的一切,而是谈内部实有的情形。

她的估计是对的。在她们谈论娜佳和薇拉的小孩子的时候,他一声不响地坐着,感觉他心中点燃起来的火光又平和又温暖,既不摇晃,又不会熄灭。

他感觉到,玛利亚的魅力征服了叶尼娅。柳德米拉上厨房里去了,玛利亚也去帮她忙活。

“多么可爱的人呀。”维克托若有所思地说。

叶尼娅用讥笑的口气唤他道:

“维季卡,听见没有,维季卡?”

他听到这意外的称呼,愣住了。已经有二十多年,没有人唤他的小名了。

“这位太太像猫一样爱上你了。”叶尼娅说。

“简直是胡扯。”他说。“而且为什么说是太太?她最不像太太了。柳德米拉没有一个女性朋友,可是她和玛利亚实在要好。”

“你和她怎么样?”叶尼娅用讥笑的口气问。

“我是说真的。”维克托说。

她看到他生气了,就微微笑着,看着他。

“叶尼娅,你懂吗?你别胡扯。”他说。

这时候娜佳来了。她站在外室里,急急忙忙地问道:

“爸爸去作检讨了吗?”

她走进房里。维克托把她抱住,亲了亲。叶尼娅眼里闪着泪花,打量着外甥女。

“呀,她身上连一滴我们斯拉夫人的血都没有,” 她说,“纯粹是个犹太姑娘。”

“是爸爸的基因呀,”娜佳说。

“娜佳,你是我的宝贝儿,”叶尼娅说,“外婆就喜欢谢廖沙,我就喜欢你。”

“没关系,爸爸,我们能养活你。”娜佳说。

“这我们是谁?”维克托问道。“是你和你那位中尉吗?你放学回来,洗洗手去吧。”

“妈妈和谁在那儿说话?”

“和玛利亚阿姨。”

“你喜欢玛利亚阿姨吗?”叶尼娅问道。

“依我看,她是世界上最好的人,”娜佳说,“我假如是个男人,一定会娶她。”

“她很善良,是天使吗?”叶尼娅用讥笑的口吻问道。

“怎么,小姨,您不喜欢她吗?”

“我不喜欢圣女,在她们的圣洁中往往隐藏着歇斯底里,”叶尼娅说,“我认为她们还不如明目张胆的坏蛋。”

“歇斯底里?”维克托问。

“维克托,我发誓,这是一般说说,我不是说她。”

娜佳上厨房里去了,叶尼娅又对维克托说:

“我在斯大林格勒的时候,薇拉有一位中尉。现在娜佳也来了一位中尉。来了,又会消失的。他们是多么容易牺牲呀。维克托,这有多悲惨呀。”

“叶尼娅,好妹妹,”维克托问道,“你当真不喜欢玛利亚吗?”

“我不知道,不知道,”叶尼娅急忙说,“有的女人有这样的性格,好像是一种顺从的、善于自我牺牲的性格。这种女人不会说:‘我和男人睡觉,因为我喜欢这样。’而是说:‘这是我的义务,我可怜他,所以牺牲自己。’这些女人睡觉,和好,分手,都是因为她们自己愿意,但她们说的完全是另一样:‘这是需要的,是义务,出自良心,我离开了,我做了牺牲。’可是她什么都没有牺牲,她所做的是她愿意的,而且最可恶的是,这些女人还当真相信自己有牺牲精神。我顶讨厌这样的女人!你知道这是为什么?我常常觉得,我自己就好像属于这一类。”

吃过午饭之后,玛利亚对叶尼娅说:

“叶尼娅,如果您愿意,我可以和您一块儿去。在这方面我有很痛苦的经验。再说,两个人在一起总要轻松些。”

叶尼娅有些发窘,就回答说:

“不,不,多谢了,这种事就需要单独去做。在这方面的痛苦,无法和任何人分担。”

柳德米拉侧眼看了看妹妹,好像是要向她说明她和玛利亚之间的私房话,说道:

“玛利亚觉得你不喜欢她,心里很不是滋味。”

叶尼娅什么话也没有说。

“是的,是的,”玛利亚说,“我感觉出来了。不过请您原谅我说出这话。这都是傻话。您哪有心思想到我。柳德米拉不应该说。现在这么一来,就好像我一定要您改变印象。我不过随便说说。没有什么用意。”

叶尼娅连自己也意想不到的十分真诚地说:

“您怎么啦,您很可爱,您说到哪儿去啦。我是心情很乱,请您原谅吧。您真的很好。”

然后,她很快地站起来,说:

“哦,就像妈妈常说的,我的孩子们:‘我该走了!’”

二十七

大街上行人很多。

“您不急着回家吧?”维克托问。“是不是咱们再上逍遥公园去?”

“您怎么啦,现在已经到了下班时间了,我要在丈夫回家前赶回去。”

他以为她会请他上家里去听索科洛夫说说学术委员会会议情形的。可是她没有作声,他便感到怀疑,是不是索科洛夫怕和他见面。她急着回家,使他很不高兴,不过这完全是自然的嘛。他们路过一个街心公园,离这里不远便是通向顿斯科伊修道院的大街了。她忽然站住,说:

“咱们坐一小会儿,然后我上电车。”

他们一声不响地坐着,但是他感觉出她的激动。她微微偏着头,看着维克托的眼睛。

他们还是没有作声。她的嘴紧紧闭着,但是他似乎听到了她的声音。一切都很清楚,都很明白了,就好像他们彼此都说过了。而且说话又能说什么呢?

他明白,现在出现了非同一般的严重局面,他的生活会出现新的烙印,他会有痛苦的内心慌乱。他不希望给别人造成痛苦,最好永远没有谁知道他们的爱情,也许他们彼此也不会说起。可是也许……不过,现在发生的事,他们的痛苦和愉快,他们是无法互相隐瞒的,这就会带来不可避免的重大变化。现在发生的一切取决于他们,同时好像这已经发生的事是命中注定了的,他们已经无法违抗了。他们之间发生的一切都是事实,自然而然的事实,并非取决于他们,就像白天的亮光不取决于人一样,同时这一事实却不可避免地产生虚假、伪装,产生对待最亲近的人的残酷心肠。要避免这种虚伪和残酷,就取决于他们,只要躲开自然而明亮的光就行。

有一点他是十分清楚的:在这样的时刻,他心里永远不能平静。他将来不论怎样,心里是永远不会平静的。不论他把对他身旁女子的感情隐藏起来,还是让感情冲出来成为他的新的命运,他都不会平静。不论把对她的爱化为长期的思念,还是和她亲近而引起良心上的痛苦,他都不能平静。

可是她还在一个劲儿地看着他,流露着无比幸福而又无比绝望的神情。瞧,他在冲突中没有弯腰,靠很大的狠劲儿坚持住了,可是在这儿,在这长椅子上,他多么软弱,多么无助。

“维克托·帕夫洛维奇,”她说,“我该走了,我丈夫等着我呢。”她握住他的手,说:“咱们今后别再会面了,我已经向丈夫保证不再和您见面。”

他感到心里十分慌乱,就像心脏病人要死的时候那样,由不得人的心跳就要停止了,整个世界开始摇晃,开始翻倒,大地和天空就要消失了。

“玛利亚,这为什么?”他问道。

“我丈夫要我保证今后不再和您见面,我就向他做了保证。这当然很不好,可是他现在心情是这样,他有病,我很担心他的生命。”

“玛利亚。”他说。

在她的声音中,在她的脸上,有一股不可动摇的力量,就像最近和他发生冲突的那股力量。

“玛利亚。”他又说。

“我的天,您也明白,您也看出来,我不隐瞒,为什么要全说出来。我不能,不能呀。我丈夫够苦了。您一切都知道。您要记住,柳德米拉也够苦的了。这是不可能的。”

“是的,是的,我们没有这样的权利。”他一再地说。

他的帽子掉到地上,大概有些人在看着他们。

“是的,是的,我们没有这样的权利。”他又说了一遍。

他吻了吻她的手。当他把她冰凉纤细的手指握在手里的时候,他觉得,使她决定不和他见面的不可动摇的力量,是和软弱、顺从、老实无用联系着的……

她站起来,走了,连头也不回。他却坐着,在想,他这是第一次正视自己的幸福、自己的生活的光明,可是这一切离开他,远去了。他觉得,刚才他吻过手的这个女子,本来可以代替他的一切的,代替他一生所想的、所希望的一切:科学,荣誉,名望。

二十八

学术委员会会议之后,第二天,萨沃斯季扬诺夫给维克托打来电话,问他身体怎么样,问柳德米拉身体好不好。

维克托问起会议的情形,萨沃斯季扬诺夫回答说:

“维克托·帕夫洛维奇,不想使您不痛快,事实上,比我原来预料的更卑劣。”

维克托想:“难道索科洛夫发言了吗?”他又问道:

“做出什么决议吗?”

“很厉害的决议:认为根本不必请院部研究今后的问题……”

“懂了。”维克托说。虽然他早就相信会做出这样的决议,但还是因为意外有些慌乱。“我什么罪也没有,”他想道,“不过还是会叫我坐牢的。那里面知道克雷莫夫没有罪,可是把他关起来了。”

“有人表示反对吗?”维克托问。电话线送来了萨沃斯季扬诺夫没有说出口的难为情。

“没有,维克托·帕夫洛维奇,似乎是一致通过,”萨沃斯季扬诺夫说,“您没有来,对您是很不利的。”

萨沃斯季扬诺夫的声音不太清楚,显然他是在公用电话亭里打电话。

这一天,安娜·斯捷潘诺芙娜也给他打来电话,她已经被解除职务,不上研究所去了,所以不知道学术委员会会议的事。她说,她要上穆罗姆的姐姐家去住两个月,并且请维克托去作客,那股亲切情谊很使维克托感动。

“谢谢,谢谢,”维克托说,“如果上穆罗姆的话,那就不是去玩儿,而是到师范学校去教物理了。”

“天啊,维克托·帕夫洛维奇,”她说,“您怎么会这样呀,我真难受,这都是因为我呀。我哪儿值得呀。”

看样子,她把他说的关于师范学校的话当作对自己的责备。她的声音也不太清楚,显然她也不是在家里打电话,也是用公用电话。

“难道索科洛夫发言了吗?”维克托自言自语地一遍又一遍问。

很晚的时候,契贝任打来电话。这一天,维克托就像害重病的病人一样,只是在别人谈起他的病的时候,他才有劲头儿。显然,契贝任感觉出这一点。

“难道索科洛夫发言了吗?他发言了吗?”维克托问过柳德米拉。但是她当然也和他一样,不知道索科洛夫是否在会上发过言。

在他和与他接近的一些人之间出现了一层迷雾。

萨沃斯季扬诺夫显然是害怕说出维克托想知道的事,不愿意成为他的情报员。他大概在想:“维克托遇到研究所的人,会说:‘我已经全知道了,萨沃斯季扬诺夫已经详详细细地把一切都向我报告了。’”

安娜·斯捷潘诺芙娜是很亲热的,不过在这种情形下她应该上维克托家里来,不应该只是打个电话。

维克托以为,契贝任也应该提出和他一起到天体物理研究所工作,哪怕谈谈这个问题也好。

“他们使我不痛快,我也使他们不痛快,还不如不打电话呢。”他想道。

但更使他不痛快的,是那些根本不给他打电话的人。

一整天他都在等古列维奇、马尔科夫、皮敏诺夫的电话。

后来他又生起安装设备的技师和电工们的气。

“这些狗崽子,”他想道,“他们是工人,有什么可怕的?”

想到索科洛夫,实在无法容忍。是他不准玛利亚给他维克托打电话!谁都可以原谅,不论老熟人、老同事,甚至亲戚,都可以原谅。就是不能原谅这个朋友!一想到索科洛夫,他就十分恼怒,气得不得了,气得连气也喘不上来。同时,他想到自己对朋友不忠,便不知不觉为自己对朋友不忠寻找起辩护的理由。

他由于冲动,给希沙科夫写了一封完全不必要的信,要求把研究所领导的决定告诉他,并且说,因为有病,近日内不能上研究所去工作。

第二天一整天都没有听到电话机铃声。

“好吧,反正是要坐牢的。”维克托想道。他想到这一点并不觉得痛苦,似乎倒是可以得到安慰。就好比生病的人,一想到“好吧,生病就生病吧,反正人总是要死的”,就能得到安慰。他对柳德米拉说:

“唯一能给咱们带来消息的人,就是叶尼娅了。虽然消息都是来自内部监狱接待室。”

“现在我相信,”柳德米拉说,“索科洛夫一定在会上发过言。要不然无法解释,为什么玛利亚不来电话。她知道他发了言,不好意思打电话。不过,到白天等他去上班了,我可以给她打电话。”

“无论如何不要打!”维克托大声说。“你听着,柳德米拉,无论如何不要打!”

“我干吗要管你和索科洛夫关系如何?”柳德米拉说。“我和玛利亚有我们的关系。”

他无法给柳德米拉解释,为什么她不能给玛利亚打电话。他一想到柳德米拉不了解底细,无意中成为他和玛利亚联系的桥梁,便觉得惭愧。

“柳德米拉,现在咱们和人们的联系只能是单方面的。如果一个人坐了牢,他的妻子只有在人家叫她去的时候,才能去。她自己没有权利说:我想上你们家去。丈夫低下了,妻子也就低下了。咱们进入了新的一个时期。咱们再也不能给任何人写信,只能回信。咱们现在也不能给任何人打电话,只能在人家给咱们来电话的时候,拿起话筒。咱们见了熟人,也不能首先打招呼,也许,人家不愿意和咱们打招呼。如果人家和我打招呼,我也不能首先开口说话。也许人家认为可以和我点点头,但是不愿意和我说话。让人家先说,我就回答人家的话。咱们已经进入碰也不能碰的贱民阶层。”

他沉默了一会儿,又说道:

“不过,我们这些不能碰的人也算幸运,常规之中也有例外。也有一两个人—我说的不是自家人,如你妈妈、叶尼娅—不能碰的人对他们是可以充分信任的。不必等待他们发出允许的信号,就可以给他们打电话,写信。比如契贝任!”

“你说得很对,维克托,完全正确。”柳德米拉说。她的话使他吃了一惊。不论在哪一方面,她已经很久没有承认他正确了。“我也有这样的朋友,就是玛利亚!”

“柳德米拉!”他说。“柳德米拉!你可知道,玛利亚已经向索科洛夫做出保证,不再和咱们见面了?这么着,你就去吧,给她打电话吧!喂,打呀,打呀!”

他摘下话筒,递给柳德米拉。

这时候他的感情的小小的一角浮起希望,希望柳德米拉真的打打电话……哪怕是柳德米拉能听到玛利亚的声音也好呀。

但是柳德米拉说道:“啊呀,原来是这样呀。”就把话筒放下了。

“怎么叶尼娅还不回来呀?”维克托说。“患难使我们更加亲密。我觉得她从来没有像现在这样可爱。”

等到娜佳回来,维克托对她说:

“娜佳,有些话我和你妈妈说过了,妈妈会对你详细说说的。在我已经变成可怕的东西的时候,你不能上波斯托耶夫家、古列维奇家和其他一些人家去。所有这些人首先会想到你是我的女儿,我的女儿,我的女儿。你是什么人,明白吗?是我家的一员。我坚决要求你……”

他事先料定她会说什么,料定她会反驳,会生气的。娜佳举起一只手,打断他的话。

“是的,我看到你没有去参加那些造孽的人的会,就全明白了。”

他一时不知如何是好,看着女儿,后来用好笑的口吻说:

“我希望这些事不影响你的中尉。”

“当然不会影响。”

“怎么?”

“不影响就是不影响,你会明白的。”

维克托看了看妻子,看了看女儿,朝她们伸过手去,握了握手,便走出了房间。在他的这一动作中,包含着那样多的慌乱、歉疚、软弱、感谢、挚爱,以至于母女俩挨在一起站了很久,没有说一句话,也没有互相看一眼。

二十九

自从战争开始以来,达林斯基第一次走进攻的道路,他在追赶向西挺进的坦克部队。在雪地里,田野上,道路两旁,到处是烧毁和打坏的德军坦克、大炮、圆头的意大利载重汽车,到处是德国人和罗马尼亚人的尸体。

死亡与严寒为观看者保留着敌军覆灭的场面。混乱、惊慌、痛苦—这一切都印在雪上,凝冻在雪里,在冰雪中保留着机器和人在大路上仓皇奔逃的最后挣扎和绝望情景。

甚至炮弹爆炸的烈火与硝烟,烟气腾腾的篝火,也印在雪上,成为一个个乌黄色斑点、一片片黄色和褐色冰凌。

苏联部队向西挺进,一群群俘虏向东移动。

罗马尼亚人穿的是绿色军大衣,戴的是高高的羊皮帽。他们显然不像德国人那样怕冷,达林斯基看到他们,不觉得这是打垮的军队的士兵,觉得这是一大群一大群疲惫无力的、饥饿的农民,戴着演戏用的皮帽。大家都在嘲笑罗马尼亚人,但是对他们却没有仇恨,而是用一种怜悯和鄙视的目光看待他们。后来他看到,大家对意大利人更没有什么仇恨。

使人仇恨的是匈牙利人、芬兰人,尤其是德国人。

德国俘虏的样子是最糟的。

他们的头上和肩膀都裹了破棉被。他们的腿从靴子以上都裹了破布片和麻袋片,用铁丝和绳子捆着。

不少人的耳朵、鼻子、脸上都有冻成疮的黑斑。腰上挂的饭盒叮当响着,像是戴着镣铐。

达林斯基看着一具具顾不得羞臊露出瘪下去的肚子和生殖器的尸体,看着一张张被草原冷风吹得通红的押队战士的脸。看着雪野上被打得歪七扭八的德军坦克和汽车,看着冻僵的死人,看着被押着向东走去的人们,产生了一种复杂而奇怪的感情。

这是报应。

他想起一些故事,说德国人怎样讥笑俄罗斯农舍的寒碜,带着厌恶而惊讶的表情打量小孩子的摇篮、炉灶、瓦盆、木桶、墙上的画、黏土捏的花公鸡,打量那些看到德国坦克就逃走的孩子们出生和成长的可亲可爱的天地。

汽车司机用好奇的口吻说:

“您瞧,中校同志!”

四个德国士兵用军大衣抬着一个士兵。从他们的脸和绷紧的脖子可以看出来,他们不要多久也会倒下去的。他们摇来晃去地走着。他们裹的破布脱落到脚上,雪粒子击打着他们失神的眼睛,冻僵的手指头死死抓住军大衣的边儿。

“德国佬完蛋啦。”司机说。

“这可不是我们请他们来的。”达林斯基阴沉地说。

可是过了一会儿,一种幸福感一下子向他袭来:在茫茫的雪雾中,在没有开垦的草原上,一队队苏军坦克向西开去,是T—34型坦克,又凶猛,又快,又坚固……

一个个坦克手头戴黑色盔形帽,身穿黑色小皮袄,从舱口里探出半个身子,朝外张望着。他们在辽阔无垠的草原上,在茫茫雪雾中奔驰,身后留下一团团模模糊糊的雪的浪花—幸福和自豪的感觉使他激动得喘不过气来……

炼成了钢铁的又威风又沉痛的俄罗斯向西奔去。

在进一个村子的时候出现了阻塞。达林斯基下了汽车,从排成两排的汽车和盖了帆布的火箭炮旁走过去……一群俘虏正跨过这条道路朝大路上去。从小汽车上走下来一位上校,头戴银灰色羊羔皮帽。能戴这种帽子的,要么是集团军司令,要么和前方军需官十分要好。上校看着俘虏。押队士兵朝俘虏们吆喝着,挥舞着自动步枪。

“快点儿,快点儿,快走!”

有一道无形的墙把俘虏和汽车司机、红军战士隔开,有一种比草原酷寒更厉害的酷冷使眼睛不能对着眼睛。

“长尾巴的,小心点儿,小心点儿。”有一个笑着的声音说。

有一个德国兵爬着过大路。露出一团团棉花的破棉被拖在他身后。他急急忙忙地爬着,不停地倒动着胳膊和腿,连头也不抬,好像在闻脚印子。他朝着上校爬来,站在旁边的司机说:

“上校同志,他会咬您的,真的,他专门瞄着您。”

上校朝旁边跨了两步,等德国兵爬到他跟前,他用靴子一踢。这不太用劲儿的一踢,足可压倒俘虏兵那麻雀一般的力气。俘虏兵的胳膊和腿都伸开了。

他从下面朝踢他的人看了看:在他的眼睛里,就像要死的羊的眼睛里那样,没有责难的神情,甚至也没有痛苦,只有温顺。

“还爬呢,哼,还想侵略呢。”上校一面说,一面在雪上擦着靴底。

在观看的人群里掠过一阵轻轻的笑声。

达林斯基感觉他的头脑一阵迷糊,感觉到已经不是他自己,而是他又认识又不认识的另一个人,一个什么也不含糊的人在支配着自己的行动。

“上校同志,俄罗斯人不打倒下的人。”他说。

“依您看,我是什么人,不是俄罗斯人吗?”上校问。

“您是恶棍。”达林斯基说。他看到上校朝他走来,就抢在上校发火和威吓之前,高声说:“我姓达林斯基!达林斯基中校,斯大林格勒方面军司令部作战科监察员。我对您说的话,我愿意在方面军司令面前,面对军事法庭再说一说。”

上校恨恨地对他说:“好吧,达林斯基中校,您等着瞧吧。”便朝一旁走去。

几名俘虏把躺在地上的俘虏拖到一边。很奇怪,不论达林斯基把脸转向哪一边,他的眼睛总是和挤成一堆的俘虏们的眼睛碰到一起。好像他有什么东西吸引着他们。

他慢慢朝汽车走去,听到有一个讥笑的声音说:

“德国佬有了卫士啦。”

不久达林斯基又上了车往前走,迎面又有一群群穿灰衣的德国俘虏和穿绿衣的罗马尼亚俘虏走来,常常影响汽车开动。

司机侧眼看着达林斯基抽烟时抖动的手指,说:

“我一点也不可怜他们。我可以把他们一个一个都枪毙。”

“好啦,好啦,”达林斯基说,“你要枪毙他们,最好是在一九四一年,在你像我一样,被他们打得头也不回地逃跑的时候。”

一路上他再也没有说话。不过那个俘虏的事并没有使他一心向善。他该有的善心好像已经消耗完了。

当初他上亚什库时走过的加尔梅克草原和今天走的道路多么不同呀。

难道那是他站在沙漠的雾中,站在巨大的月亮底下,望着溃逃的红军,望着一匹匹骆驼一伸一曲的脖子,思虑着俄罗斯土地那最后的边沿上所有亲爱的软弱可怜的人们?

三 十

坦克军军部驻扎在村子边上。达林斯基的汽车来到军部的房子门前。天色已经黑下来。显然,军部来到村里才不久:有些红军士兵正在从汽车上往下卸箱子、褥垫,电话兵在架电话线。

一名站岗的士兵很不情愿地走进过道,唤了一声副官。一名副官很不情愿地走出门来,和所有的副官一样,不是看着来人的脸,而是看着肩章,说:

“中校同志,军长刚刚从旅里回来,在休息呢。您等会儿再来吧。”

“您去报告军长,达林斯基中校来了。懂吗?”来人很傲慢地说。

副官叹了一口气,朝房里走去。过了一分钟,他走出来,高声说:

“中校同志,请进!”

达林斯基上了台阶,诺维科夫出来迎接他。他们高兴地笑着,互相打量了一小会儿。

“终于见面了。”诺维科夫说。

这是一次十分愉快的重逢。

两个聪明的脑袋又像过去一样,俯在地图上面了。

“我现在前进的速度,就跟当初逃跑时一样,”诺维科夫说,“不过在这一地段,超过了逃跑时的速度。”

“这是冬天,冬天,”达林斯基说,“到夏天又会怎样呢?”

“我看没有问题。”

“我也这样看。”

让达林斯基看地图,诺维科夫觉得是一种愉快的享受。他思路敏捷,关注那些似乎只有诺维科夫能够察觉的细节,他提出的问题都是诺维科夫觉得应该考虑的……

诺维科夫放低声音,就像吐露隐秘私情似的说:

“对于进攻中坦克运动地带的侦察、各种目标指示手段的协同运用、基准点示图、相互配合的神圣性—这一切都是必须的。但是在坦克进攻地带,各兵种的战斗行动还是要听命于一个上帝,那就是坦克,我们的乖孩子T—34型坦克!”

达林斯基见过的不仅仅是斯大林格勒方面军南翼活动的地图。诺维科夫从他嘴里了解到高加索战役的一些详情细节,了解到截听到的希特勒和保卢斯交谈的内容,了解到自己还不知道的弗列捷尔皮科将军的炮兵军群的运动详情。

“这已经是乌克兰了,窗外就可以看到。”诺维科夫说。

他指着地图说:

“不过我好像比别人离得近些。祖国就支持我这个军。”

后来,他推开地图,说:

“好啦,咱们别再谈战略战术了。”

“您个人的事还是没有什么进展吗?”达林斯基问道。

“大有进展!”

“怎么,结婚了吗?”

“我现在就天天在等着,她就要来啦。”

“哎呀,你这自由的哥萨克完啦,”达林斯基说,“我衷心恭喜您。可是我还没有头绪呢。”

“哦,贝科夫怎么样?”诺维科夫忽然问道。

“贝科夫嘛,没什么。现在跟着瓦图京[2],老样子。”

“真够刚强,什么都不在乎。”

“应该说,像砥柱一样。”

诺维科夫说:

“好啦,见他的鬼去吧。”

他朝着旁边的屋子喊道:

“喂,维尔什科夫,看样子,你是下定了决心叫我们饿死了。你把政委叫来,我们一块儿吃饭。”

但是用不着去叫政委了,他自己来了,站在门口,用很不痛快的声调说:

“诺维科夫同志,不知怎么搞的,好像罗金冲到前面去了。瞧着吧,他会赶在咱们前头踏上乌克兰土地。”

又对达林斯基说:

“中校同志,现在就是这种时候。现在我们害怕友邻部队,胜过害怕敌军。您大概不是友邻部队的吧?不是,显然不是,您是老战友。”

“我看出来,你是真操心乌克兰问题。”诺维科夫说。

格特马诺夫把罐头朝自己面前拉了拉,故意用吓唬的口吻说:

“好哇,诺维科夫同志,不过你要注意,你的叶尼娅就要来了,我只能让你们在乌克兰土地上登记。就让中校同志做证婚人。”

他举起酒杯,用酒杯指点着诺维科夫,说:

“中校同志,咱们来为他那颗俄国心干杯。”

达林斯基动情地说:

“您说的话好极了。”

诺维科夫记得达林斯基一向对政工人员是十分反感的,就说:

“是啊,中校同志,咱们很久没见面了。”

格特马诺夫打量了一下桌上,说:

“真是没东西招待客人,只有罐头。炊事员往往还没有生起炉子,可是指挥所又得换地方了。日日夜夜在运动。您要是在发动进攻之前上我们这儿就好了。现在停一个钟头,跑一个昼夜。拼命往前跑。”

“哪怕再弄一把叉子来也好呀。”诺维科夫对副官说。

“是您不叫人把汽车上的家什卸下来呀。”副官回答说。

格特马诺夫说起他在收复的领土上经过时见到的情形。

“俄罗斯人和加尔梅克人截然不同,”他说,“有很多加尔梅克人在为德国人唱赞歌。要知道,苏维埃政权什么好处没有给他们呀?!要知道,本来是一块到处是破破烂烂的流浪汉、梅毒到处流行、到处是文盲的地方。可是你瞧,不论把狼喂得多么饱,狼还是贪恋草原。”

他对诺维科夫说:

“你该记得,关于巴桑戈夫的事,我曾经提醒过的。我这个党员的感觉果然没有错。不过你不要介意,我这不是责备你。你以为,我这一生犯的错误少吗?你要知道,民族特征是一个很大的问题,这会有决定性的意义,战争的实践已经把这一点显示出来。你可知道,布尔什维克的主要老师是谁?是实践。”

“我赞成您对加尔梅克人的看法,”达林斯基说,“我不久前就在加尔梅克草原上住过,许多地方我都到过。”

他为什么说这话?他在加尔梅克走过不少地方,对加尔梅克人从来没有不好的感觉,倒是对他们的生活和习惯十分感兴趣。但是,这位军政委似乎有一股磁石般的吸引力。达林斯基随时都想赞同他的意见。

诺维科夫微微笑着看了看他,他倒是很了解政委的精神吸引力,很了解这种力量怎样吸引人对他唯唯称是。

格特马诺夫忽然很坦诚地对达林斯基说:

“我知道,您过去也曾经受到不公正的待遇。不过您不要怪布尔什维克党。党也是希望为人民做好事。”

达林斯基一向认为部队中的政工人员和政委都是一团糟的,这时急忙说:

“您怎么啦,这一点难道我还不了解?!”

“是啊,是啊,”格特马诺夫说,“我们有些地方做得很不对头,但是人民会原谅我们的。会原谅的!我们的同志都是好同志,本质是不坏的。不是吗?”

诺维科夫温和地打量了一下坐在一起的人,说:

“我们的军政委好吗?”

“很好。”达林斯基肯定说。

“就是,就是。”格特马诺夫说。

三个人一齐笑起来。

格特马诺夫似乎猜到诺维科夫和达林斯基的心思,看了看表,说:

“我要去休息了,要不然白天黑夜都在运动,哪怕今天睡上一夜也好。十个昼夜没脱靴子了,就像茨冈人一样。参谋长恐怕还在睡着吧?”

“他哪儿是睡觉,”诺维科夫说,“一来到就去察看新的情况了,因为明天早晨咱们又要转移基地。”

等到只剩下诺维科夫和达林斯基,达林斯基说:

“有些事情我总是理解不透。比如,不久前我在里海附近的沙漠上,心情就特别沉重,好像眼看着就要完了。可是结果怎么样?我们能够组织起这样大的力量!非常强大的力量呀!一切都不在话下。”

诺维科夫说:

“可是我却越来越清楚、越来越多地懂得了,什么叫俄罗斯人!俄罗斯人是勇猛的,好比强悍的狼!”

“是强大的力量!”达林斯基说。“主要的是:俄罗斯人在布尔什维克领导下走在了人类最前面,其余的事都是微不足道的。”

“您听我说,”诺维科夫说,“要不要我再谈谈您的工作调动问题?您能不能到我们军里担任副参谋长?咱们一块儿打打仗,行吗?”

“怎么不行?谢谢。那我给谁当副手?”

“给涅乌多布诺夫将军。这是规矩嘛:中校给将军当副手。”

“涅乌多布诺夫?战前他是在国外的吧?是在意大利吧?”

“不错。就是他。他不是苏沃洛夫,不过,总的说,还是可以共事的。”

达林斯基没有作声。诺维科夫朝他看了看。

“怎么样,事情就这样办吧?”诺维科夫问道。

达林斯基用手指头掀起嘴唇,又撑了撑腮帮子。

“您看见吗,有两个坑?”他问道。“这是一九三七年涅乌多布诺夫审问我的时候打掉了我的两颗牙。”

他们互相看了看,沉默了一会儿,又互相看了看。达林斯基说:

“他这个人当然还是精明能干的。”

“当然,当然,他总不是加尔梅克人,是俄罗斯人嘛。”诺维科夫冷笑说。忽然他高声说:“咱们来干杯,不过喝酒可要真的像俄罗斯男子汉!”

达林斯基生平第一次喝这样多的酒。不过,如果不是桌上的两个空酒瓶,旁边的人谁也不会发觉两个人喝得很猛,很带劲儿,除非注意到他们已经互相称呼起“你”。

诺维科夫不知已经是第几次斟满两杯,说:

“来,不要歇气。”

不会喝酒的达林斯基这一次连气也没有歇。他们谈起撤退,谈起战争一开始的那些日子。他们回忆到布柳赫尔和图哈切夫斯基。他们谈到朱可夫。达林斯基还说了说侦讯官在审讯中想从他嘴里得到什么。诺维科夫说到他怎样在进攻开始之前推迟几分钟出动坦克。但是他没有说在判断几位旅长的行动方面犯了错误。

他们谈起德国人,诺维科夫说,一九四一年的夏天好像锤炼了他,使他的心肠永远变硬了,可是等到押送第一批俘虏,他却下令让俘虏吃好一点儿,吩咐用汽车把冻坏和受伤的俘虏送往后方。

达林斯基说:

“刚才我和你们的政委一起骂加尔梅克人。骂得对!可惜你们的涅乌多布诺夫不在这儿。我该和他谈谈,真该和他谈谈。”

“哼,不是有很多奥廖尔人和库尔斯克人跟德国人勾结吗?”诺维科夫说。“比如做了叛徒的弗拉索夫将军,也不是加尔梅克人。我说的那个巴桑戈夫,是一位很好的军人。涅乌多布诺夫是肃反工作人员,政委对我说过他的情况。他不是军人。我们俄罗斯人会打赢的,会打到柏林,我知道,德国人再也挡不住我们了。”

达林斯基说:

“像涅乌多布诺夫,叶若夫,确实是很大的问题,不过俄罗斯现在只有一个,那就是苏维埃俄罗斯。我知道,哪怕把我所有的牙都打掉,我对俄罗斯的爱不会动摇。我至死都要爱俄罗斯。但是要我做这家伙的副手,我不干,你怎么,同志,不是开玩笑吧?”

诺维科夫又一次把两个杯子斟满,说:

“来,咱们喝。”

然后他说:

“我知道,还会有各种各样的事。我也会变得更糟。”

他忽然换了话题,说:

“唉,我们的事真是可怕。有时一个坦克手被打掉了脑袋,人已经死了,可是还踩着油门,坦克还在前进。一个劲儿地前进,前进!”

达林斯基说:

“我刚才和你们的政委一起骂加尔梅克人,可是我现在却一个劲儿地想着一个加尔梅克老汉。涅乌多布诺夫有多大岁数啦?上他那儿去看你们的新位置,就要跟他见面吗?”

诺维科夫慢慢地用不大听使唤的舌头说:

“我很有福气。再没有更福气的啦。”

于是他从口袋里掏出相片,递给达林斯基。达林斯基一声不响地看了很久,说:

“太美了,真没有说的。”

“美吗?”诺维科夫说。“美倒是算不了什么,像我这样爱她,倒不是因为美。”

维尔什科夫来到门口,站下来,用询问的目光看着军长。

“走开。”诺维科夫慢慢地说。

“喂,你干吗对他这样,他是想问问咱们要不要什么。”达林斯基说。

“算啦,算啦,我还会更糟,会成为下贱的人,我行,用不着教训我。你是中校,和我说话为什么称‘你’?按照军事条令应该这样吗?”

“啊,原来是这样!”达林斯基说。

“算啦,开玩笑你都不懂。”诺维科夫说。心想,幸亏叶尼娅看不见他的醉态。

“愚蠢的玩笑我是不懂。”达林斯基说。

他们表白自己的态度表白了很久,直到诺维科夫提议到新位置去用通条把涅乌多布诺夫打一顿,才算了事。当然他们哪儿也没有去,而是又喝了不少。

三十一

弗拉基米罗芙娜在一天里收到三封信:两封是两个女儿写来的,一封是外孙女薇拉写来的。

她还没有把信打开,只是从笔迹认出是谁的来信之后,就知道信里没有令人愉快的消息。多年的经验告诉她,孩子们大都不喜欢给做母亲的写信报告高兴的事。

三方面来信都请她去:柳德米拉请她上莫斯科,叶尼娅请她上古比雪夫,薇拉请她上列宁斯克。这些邀请向弗拉基米罗芙娜证实了,两个女儿和外孙女的日子都不好过。

薇拉在信里写到父亲,说党内和工作中的一些不愉快的事把他折腾得筋疲力尽。他曾经奉人民委员部的命令去古比雪夫,几天前才从古比雪夫回到列宁斯克。薇拉在信中说,父亲从古比雪夫回来,憔悴不堪,他在发电站坚持战时工作期间都不像这样憔悴。他的问题在古比雪夫一直没有解决,命令他回来,参加恢复发电站的工作,但是告诉他,还不知是否能让他留在发电站人民委员部系统。

薇拉准备和父亲一起从列宁斯克上斯大林格勒去,现在德国人已经不打炮了。市中心还没有收复。去过市内的人说,原来弗拉基米罗芙娜住的房子,只剩了骨架,房顶已经塌了。父亲在发电站住的房子还是完好的,只是石灰剥落了,窗玻璃没有了。父亲和薇拉带小孩子还可以住这所房子。

薇拉写到儿子。弗拉基米罗芙娜看着信都觉得奇怪,小丫头、小外孙女薇拉竟像个大人一样,用一个妇人,甚至是婆婆妈妈的口气写起自己的小孩子的胃病、皮疹、睡觉不安宁、新陈代谢失调。这一切薇拉应该说给丈夫、妈妈听,可是现在她却写信告诉外婆。她没有丈夫,也没有妈妈了。

薇拉提到安德列耶夫,提到他的儿媳妇娜塔莉亚,提到小姨叶尼娅,说父亲在古比雪夫曾经见到她。她没有说自己的事,好像外婆对她的事不感兴趣。

她在最后一页的空白处写道:

“外婆,发电站的房子很大,够咱们住的。我恳求您:来吧。”

薇拉在信里没有写出的,竟用这种突然呼叫的方式表现出来。

柳德米拉的信很短。她写道:

“我看不出我活着有什么意思。托里亚不在了,维克托和娜佳不需要我,他们没有我也能活下去。”

柳德米拉从来没有给妈妈写过这样的信。弗拉基米罗芙娜明白,女儿和丈夫的关系真的出现了裂痕。柳德米拉请妈妈上莫斯科,这样写道:

“维克托一直很不愉快,可是他一向对您比对我更乐意说心里话。”

再往下是这样的话:

“娜佳现在心思深了,有什么事都不和我说了。现在这成了我们家的风气……”

叶尼娅的信却使人一点也摸不清头脑,信里都是一些含糊话,暗示有很大的麻烦和不幸。她请妈妈上古比雪夫去,同时又写着,她有急事要上莫斯科去一趟。叶尼娅还在信里对妈妈说起里蒙诺夫,说他说了不少称赞妈妈的话。她说,妈妈如果见到他,会感到高兴的,他是一个很聪明、很风趣的人,但是在信里又说,里蒙诺夫上撒马尔罕去了。简直叫人不懂:弗拉基米罗芙娜上古比雪夫,怎么会见到他?

只有一点是明白的,所以弗拉基米罗芙娜一看完这封信,就在心里说:“我的孩子是很不幸的。”

三封信使弗拉基米罗芙娜十分激动。三封信都问到她的健康,问她的房间里是不是暖和。这种关怀使她很感动,虽然她明白,年轻人没有考虑她是不是需要她们。她们是需要她的。不过,也许不是这样。为什么她不向女儿求助,为什么女儿向她求助呢?要知道,她现在孤孤单单,又老,又无家可归,儿子和一个女儿死了,谢廖沙又没有音信。她干工作越来越吃力了,心口经常作疼,头经常发晕。她甚至向厂里的技术领导人要求过,要求从车间调到实验室,她一天到晚在机器中间走来走去取检验样品,实在吃不消。下了班她要站队买东西,回到家里还要生炉子,做饭。而生活又是这样艰难,这样困苦!站队还算不了什么。更糟的是空空的店铺门前没有人站队。更糟的是,她回到家里,不做饭,也不生炉子,就饿着肚子睡到又潮湿又冷的被窝里。

周围的人日子过得都很艰难。从列宁格勒疏散出来的一位女医生,对她说过怎样带着两个小孩子在离乌法一百公里的村子里度过了一个冬天。她住在原来被划为富农的人的空房子里,窗玻璃没有了,房顶拆掉了。她天天要到六公里之外去上班,要经过树林,有时在黎明时候在树丛里会看到绿莹莹的狼眼睛。村子里的人都很穷,庄员都不愿意干活儿,说不论怎么干,反正粮食都要被弄走,因为农庄里欠的公粮总是缴不清。邻居的男人上了前线,老婆带着六个孩子在家里过吃不饱的日子,六个孩子只有一双破毡靴。女医生还对弗拉基米罗芙娜说,她买了一只母山羊,夜里有时趟着很深的雪到很远的田野里去偷荞麦,从雪底下往外扒没有收净的发霉的干草。她说,她的两个孩子因为在乡下听了不少粗野的骂人的话,也学会了骂娘,所以喀山小学的一位女教师对她说:“我第一次见到一年级学生像个醉汉一样骂娘,还是列宁格勒来的孩子呢。”

现在弗拉基米罗芙娜住在维克托原来住的小房间里。宽敞的堂屋里住的是二房东夫妇,也就是本来的租户,他们在维克托一家离开之前原是住在偏房里的。二房东夫妇是很不安生的人,常常因为家庭琐事争吵。

弗拉基米罗芙娜很生他们的气,不是因为他们吵闹得不安宁,而是因为他们向她这个遭难的苦老婆子要的房租太高,这么一个小房间,每月房租二百卢布,占她的工资的三分之一还多些。她觉得,这些人的心肠是用胶合板和白铁做成的。他们想的只是吃的和用的东西。从早到晚谈的都是素油、腌肉、土豆、在旧货市场上买的和卖的东西。夜里他们嘁嘁喳喳地说话。二房东太太对丈夫说,住在这房子里的一个做工长的邻居,从农村弄来一口袋白白的瓜子和半口袋玉米,又说今天集市上卖的蜂蜜很便宜。

二房东太太尼娜很漂亮,高高的个子,苗条的身段,灰色的眼睛。结婚之前她在工厂工作,参加过业余文艺活动,演过歌剧,也演过话剧。二房东谢苗·伊凡诺维奇在军事工厂工作,是一名锻工。年轻时候他在驱逐舰上工作过,是太平洋舰队中量级拳击冠军。现在这对夫妇当年的英姿似乎成了不可思议的了—谢苗·伊凡诺维奇早晨在上班之前就喂鹅,给小猪煮食儿,下班回来就在厨房里忙活,淘米,修鞋子,磨刀,洗瓶子,说说工厂里的司机怎样从远地的农庄里弄来面粉、鸡蛋、羊肉……尼娜就和他抢着说自己的无数病症,还说她怎样经常去找名医,说她怎样拿毛巾换豆角,说邻居一个妇女向一个疏散出来的女子买了一件马皮上衣和五个小碟子,说怎样炼猪油和混合油。

他们是不坏的人,但是他们从来没有和弗拉基米罗芙娜谈起过战争,没有谈过斯大林格勒,没有谈过苏联情报局的战报。

他们又怜悯又瞧不起弗拉基米罗芙娜,因为女儿走后,没有了科学院的定量供应,她就经常处于半饥饿状态。她没有糖,没有油,喝的是白开水,菜汤是公共食堂的,有一回连小猪都不肯喝这种汤。她没有钱买木柴。她也没有东西卖。她的穷困使二房东夫妇感到不快。有一天晚上,弗拉基米罗芙娜听到尼娜对丈夫说:“昨天我只好给老婆子一张烙饼,当着她的面吃东西,她饿着肚子坐在那儿看着,实在叫人不舒服。”

夜里弗拉基米罗芙娜睡不好。为什么谢廖沙没有音信?她睡的是柳德米拉原来睡的铁床,似乎女儿夜间的预感和思绪都传给了她。

人多么容易死。活下来的人多么痛苦。她想着薇拉。薇拉的丈夫也许死了,也许是把她忘了,薇拉的父亲很苦恼,件件事情都不顺心……但就连死亡和痛苦都没有消除柳德米拉和维克托之间的隔阂,让他们亲密起来。

晚上,她给叶尼娅写了一封信:“我的好孩子……”可是到了夜里,她为叶尼娅难过起来:真是一个可怜的丫头,她现在日子过得多么不安宁,今后会怎么样呀。

维克托的妈妈,索菲亚·列文顿,谢廖沙……契诃夫是怎么写的:“米修斯,你在哪儿呀?”[3]

“到十月革命节要把鹅杀了。”谢苗·伊凡诺维奇说。

“我拿土豆喂鹅,为的是把鹅杀了吗?”尼娜说。“你听我说,等老婆子走了,我想把地板漆一漆,要不然地板要烂了。”

他们总是谈这样那样的东西,他们生活的天地里充满了东西。在这个天地里没有人的感情,只有木板、铅丹、米、钞票。他们是勤劳而诚实的人,所有的邻居都说,尼娜和谢苗·伊凡诺维奇从来没有拿过别人的一文钱。但是他们既不关心一九二一年伏尔加地区的饥饿,也不关心医院里的伤兵、瞎眼的残疾人、大街上无家可归的孩子。

他们和弗拉基米罗芙娜截然不同。他们对人、对共同事业、对别人的痛苦的冷漠是自然而然的。可是她却常常想着别人,为别人操心,常常因为一些跟自己、跟家里人无关的事情十分愤怒,或者非常高兴……普遍集体化时期的事、一九三七年的事、因为丈夫而进劳改营的一些妇女的遭遇、进入收容所和保育院的失去父母的孩子们的遭遇、德国人杀害俘虏、军事上的挫折和失利,这一切都使她十分痛苦,使她不得安宁,就像她自己家里遭遇了不幸。

她这一点,不是她读过的好书教她的,也不是生活、朋友、丈夫教她的,也不是来自她出身的民意党人家庭的传统。她就是这样,不可能是另一种样子。她没有钱,到发工资还有六天。她没有东西吃。她的全部财产可以用一块手帕包起来。但是她在喀山,一次也没有想过在斯大林格勒的住宅里被烧掉的东西,没有想过家具、钢琴、茶具、丢掉的羹匙和叉子。她甚至也没有心疼被烧掉的书。

而且,她竟远离思念着她的亲人,跟志趣迥异的人住在一座房子里,这也有点儿奇怪。

在收到亲人来信之后的第三天,卡里莫夫来找弗拉基米罗芙娜。

她见他来了,十分高兴,请他一块儿喝用野蔷薇煮的开水。

“您收到莫斯科来信很久了吗?”卡里莫夫问道。

“才三天。”

“是这样,”卡里莫夫说,并且笑了笑,“我是想问问,从莫斯科来一封信走多久?”

“您看看信封上的邮戳。”弗拉基米罗芙娜说。

卡里莫夫仔细看了看信封,忧虑地说:

“走了九天。”

他沉思起来,似乎信走得慢对他有一种特别的意义。

“据说,这是因为检查,”弗拉基米罗芙娜说,“天天信很多,无法及时检查。”

他用好看的黑眼睛朝她的脸上看了看。

“这么说,他们在那儿一切顺利,没有什么不愉快的事吗?”

“您的气色很不好,”弗拉基米罗芙娜说,“您一副病容。”

他就像否认别人的责难似的,急忙说:

“您说的不对!恰恰相反!”

他们谈起前方的战事。

“连孩子们都明白,现在战争出现了决定性的转折。”卡里莫夫说。

“是呀,是呀,”弗拉基米罗芙娜笑了笑,“现在连小孩子都明白了,可是去年夏天所有的圣人都认为,德国人一定会胜利。”

卡里莫夫忽然问道:

“您一个人过日子,大概很困难吧?我看到,您是自己生炉子。”

她沉思起来,皱起眉头,就好像卡里莫夫问的问题很复杂,一下子回答不上来。

“您是来问我生炉子是不是困难的吗?”

他摇了几下头,后来沉默了很久,一面看着放在桌上的两只手。

“最近把我传了去,询问我们在这儿聚会和谈话的情形。”

她说:

“那您干吗不说?干吗要说什么炉子?”

卡里莫夫注视着她的眼神,说:

“当然,我不能否认,我们谈过战争,谈过政治。如果说四个成年人仅仅谈电影,那是可笑的。当然,我说,我们不论谈什么,我们说的都是苏联爱国主义者该说的话。我们都认为,人民在党和斯大林同志领导下一定会取得胜利。总的来说,问的问题还不是带有敌意的。但是过了几天,我担心起来,简直睡不着觉。我仿佛觉得,维克托出了什么事情。而且,马季亚罗夫又出了一件奇怪的事。他上古比雪夫的师范学院去,有十天了。这儿的学生等着他上课,可是不见他回来,系主任往古比雪夫发了电报,可是没有回音。我夜里躺在床上,脑子里直翻腾。”

弗拉基米罗芙娜没有作声。

他小声说:

“真不得了,几个人在茶余酒后说说话儿,就要怀疑,就要传讯。”

她没有作声。他用询问的目光看了看她,恳求她说话,因为他已经把一切都对她说了。可是她没有说话,于是卡里莫夫觉得,她没有说话是要让他明白:他没有把话全说出来。

“事情就是这样。”他说。

弗拉基米罗芙娜没有作声。

“哦,我忘了,还有呢,”他说,“他,也就是那个同志,还问:‘你们谈过言论自由的问题吗?’是的,谈过这方面的问题。哦,还有,后来忽然问我,是不是认识柳德米拉的妹妹和她的丈夫,好像是姓克雷莫夫的。我从来没有见过他们,维克托也从来没有对我说起过她。我就是这样回答的。后来又问:维克托是否和我个人谈过犹太人的地位问题?我问:‘为什么偏偏和我谈?’他们回答说:‘您要知道,您是鞑靼人,他是犹太人。’”

等到卡里莫夫已经告过别,穿着大衣、戴着帽子站在门口,用手指头敲着当初柳德米拉从里面抽出报告儿子受重伤的那封信的信箱。

弗拉基米罗芙娜说:

“不过,很奇怪,这跟叶尼娅有什么关系?”

当然,不用说,不论卡里莫夫,不论她,都无法回答:为什么喀山的内务人民委员部工作人员,要问住在古比雪夫的叶尼娅以及在前方的她原来的丈夫?

很多人都相信弗拉基米罗芙娜,她经常听到一些类似的事情和自我表白,很容易觉察到说话的人有话没有说完。她也不想给维克托发出警告,她知道,这没有任何用处,只能使他更加提心吊胆。她也不想猜测,是哪一个参与闲谈的人把话说出去或者告密的;想猜出这样的人是很难的,有时到末了这种事恰恰是最不受怀疑的人干的。内务部门的案子有时是在无意中酿成的,比如,因为信里一句含含糊糊的话,一句笑话,因为不小心在厨房里当着邻居的面说的一句话;这样形成的案件不算稀罕。可是,为什么侦讯员忽然向卡里莫夫问起叶尼娅和克雷莫夫?

她又是很久不能睡着。她很想吃东西。从厨房里飘来油饼香味,好像是用素油在烙土豆饼,还有洋铁盘子的叮当声,谢苗·伊凡诺维奇安静的说话的声音。天啊,她多么想吃啊!今天中午食堂里的菜汤简直是泔水汤,她没有喝完,现在觉得十分可惜。吃的念头截断别的念头,把别的念头搅乱了。

第二天早晨她来到工厂,在门口岗棚里遇到厂长的秘书,是一个上了年纪、面孔像男子似的不和善的女人。

“沙波什尼科娃同志,中午休息时候,请到我这儿来一下。”女秘书说。

弗拉基米罗芙娜很惊奇:难道厂长这样快就答应了她的请求?她在工厂的院子里走着,心中忽然出现了一个想法,随即就把这个想法说出口来:

“在喀山住够了,我回家去,上斯大林格勒去。”

三十二

战地宪兵队队长哈尔布传唤连长列纳尔德,让他到德军第六集团军司令部来。

列纳尔德迟迟未到。保卢斯新发了一道命令,严禁小汽车使用汽油。所有的汽油都归集团军参谋长施密特将军掌握。这样一来,即便死十次,都别想得到将军批的五公升汽油。现在不仅没有汽油供应士兵的打火机,也没有汽油供应军官的小汽车了。

列纳尔德只好等待司令部往城里送机要信件的汽车,一直等到晚上。

小汽车在结了冰的柏油路上奔驰着。在前沿阵地的掩蔽所和掩体之上,在无风而寒冷的空气中,飘荡着半透明的淡淡的烟气。在大路上,一群群伤兵头上裹着手帕和毛巾,朝城里走着,还有司令部从城里调往工厂去的士兵,头上也裹着毛巾,腿上还裹着破布。

司机把汽车停在路边躺着的一匹死马跟前,检查起马达来。列纳尔德看着几个胡子拉碴、面带忧虑之色的人用斧子在砍冻肉。有一个士兵爬到露出来的马的肋骨上,就像一个木匠在没有盖好的屋顶的椽子上干木匠活儿。旁边的瓦砾堆里生着一堆火,用三角架支着一口黑锅,周围站着的士兵有的戴钢盔,有的戴军帽,有的裹着棉被,有的裹着围巾,背着冲锋枪,腰上挂着手榴弹。炊事兵用刺刀不停地把从水里往上冒的一块块马肉往下按。掩蔽所顶上有一名士兵不慌不忙地在啃一块马骨头上的肉,那块马骨头很像一张特大型号的口琴。

忽然夕阳把大路和一座空荡荡的楼房照得通亮。楼房的一个个被烧空了的眼眶充满了冰冷的血,被战争的硝烟弄脏又被炮弹炸翻起来的积雪泛出金黄色,死马的黑红色腹腔也亮堂了,大路上的卷地风雪像铜蒺藜似的盘旋起来。

晚霞具有一种特性,可以揭示事物的本质,可以使视觉变为画面,变为历史,变为感情,变为命运。一片片泥污和烟熏的痕迹在即将离去的夕阳中像成百上千的人在说话,人会看到逝去的幸福、无法挽回的损失、痛心的失误,也会看到希望的永恒的美。

这是穴居时代的场面。威风一时的勇士们,民族的精英,大日耳曼的建造者们,被抛出了胜利的道路。

列纳尔德看着裹了破布的人们,凭自己的锐敏感觉理解了:理想正如这西下的夕阳,就要消失了。

如果精力极其旺盛的希特勒、掌握着最先进理论的强盛而有作为的民族,能够把这些望着煮马肉的锅上冒出灰烟的人们,带到冰封的伏尔加河的静静的岸边,来到这瓦砾场上,来到这肮脏的雪地上,来到这夕阳染红了的窗子前面,能够使他们这样乖乖地顺从,可见生命的深处有一股多么愚蠢,多么迟钝的力量……

三十三

保卢斯的司令部设在被烧毁的百货公司大楼的地下室。长官们按照既定的次序一个个来到自己的办公室,值班参谋向他们报告有关文件的内容,报告战局变化、敌军的行动。

电话机不停地发出叮铃声,打字机嗒嗒响着,司令部第二科科长申诺克低沉的笑声从胶合板的门后面传出来。来去匆匆的副官们的皮鞋依然在石板地上咯吱咯吱响着,装甲部队司令戴着单眼镜来到自己的办公室之后,走廊里依然有法国香水的气味,似乎与潮气、香烟气味、皮鞋油气味混合,又似乎没有混合。身穿皮领军大衣的集团军司令从地下办公室的狭窄通道上走过的时候,说话声和打字机声音依然会一下子停下来,几十双眼睛依然会注视着他那沉思的长着鹰钩鼻子的脸。保卢斯的日程依然像原来那样安排,依然将原来那样多的时间用于饭后抽烟,同集团军参谋长施密特将军交谈。无线电话务士官依然常常带着粗俗的傲慢神情,不顾正常的日程安排,不理睬亚当斯上校垂下的眼睛,带着希特勒的标明“亲手交接”的电报,径直走向保卢斯。

当然,表面上一切都没有变化,但实际上自从被包围的那一天起,司令部里的人的生活中发生了许多变化。

他们喝的咖啡的颜色有了变化,变化还表现在向战线西面架设的电话线,表现在新的弹药消耗标准,表现在每天都发生的“容克”运输机穿越空中封锁时着火和坠毁的可怕场面。还出现了一个新的名字—曼施坦因,这个名字在官兵们耳朵里压倒了其他的名字。

列举这些变化是没有必要的,毋须本书描述,这些变化也是显而易见的。很明显,以前吃得饱饱的人,现在常常感到饿了;很明显,以前挨饿和吃不饱的人的脸色变了,变成了土色。当然,德军司令部里的人也发生了内在的变化:高傲的、目空一切的人不再那么神气活现,好吹牛的不再吹牛,原来十分乐观的人骂起了元首,并且开始怀疑他的政策的正确性。

但是,在那些迷恋于民族国家的无人性精神,被其束缚的德国人的头脑和心灵中,还开始了特别的变化。这些变化不仅触及人类生活的土壤,而且触及土壤的下层,正因为这样,人们还没有明白,没有觉察到。

这种变化过程很难感觉出来,就像很难感觉出时间在移动一样。在饥饿的痛苦中,在夜晚的恐怖中,在大难临头的感觉中,慢慢地开始了人性自由的解放过程,也就是人变为人、生命战胜非生命的过程。

十二月的白昼越来越短,十七个小时的寒冷的夜晚越来越长。包围圈越来越紧,苏军大炮和机枪的火力越来越猛……啊,俄罗斯草原上的寒冷是多么严酷无情,就连习惯了寒冷、穿着皮袄和毡靴的俄罗斯人都感到难以忍受。

头顶上是寒冷而严酷的天空,天空流露着一股无情的肃杀气氛,一串串冷冰冰的星星像锡制的树挂似的,出现在冻得一动不动的天上。死去的和注定要死的人怎么会懂得,这是几千万德国人过了十年惨无人道的生活之后,开始过人的生活的最初时刻!

三十四

列纳尔德来到第六集团军司令部门前,在苍茫的暮霭中看到一名灰脸的岗哨孤单地站在傍晚时候的灰墙边,他的心就剧烈地跳动起来。等他来到司令部的地下室走廊里,他看到的一切,使他又留恋,又悲伤。

他看到一扇扇门上用哥特字体写的牌子:“第二科”、“副官处”、“科赫将军”、“德拉乌里克少校”。他听到打字机的嗒嗒声,他听到说话声,体验到一种感觉,感觉到与他熟悉、亲近的作战伙伴、党内的同事、党卫军战友们紧密相连的父子兄弟般的感情—他看到他们在夕照中—他们的命要完了。

他来到哈尔布的办公室门口,还不知道要谈的是什么,不知道这位党卫军少校是不是想和他谈自己的感受。

正如在和平时期在十分熟悉的党内工作的同事中常见的,他们并不看重军衔的高低,在彼此相处中保持着同志间的随便态度。他们见了面,一般都会一边闲聊,一边谈着工作。

列纳尔德善于用几句话说明复杂事情的实质,他的话有时会在一级级报告文稿中作长途旅行,一直到达柏林的最高层办公室。

列纳尔德走进哈尔布的办公室,简直认不得他了。列纳尔德凝视着他那胖胖的、并没有消痩的脸,一下子弄不清楚:难道仅仅是哈尔布那聪明的黑眼睛的神情发生了变化?

墙上挂着斯大林格勒地区的地图,一个炽热的、无情的红圈子围住了第六集团军。

“列纳尔德,咱们在岛上了,”哈尔布说,“围绕咱们这个岛的不是水,而是下等人的仇恨。”

他们说起俄罗斯的寒冷、俄罗斯的毡靴、俄罗斯的油脂,说俄罗斯的酒害人,本是取暖的,结果越喝越冷。哈尔布问,在前沿阵地上官兵关系有什么变化。

“如果想一想的话,”列纳尔德说,“我看不出一个上校的想法和士兵们的议论有什么不同。总的说,都是一种调调儿,没有什么乐观的。”

“各个营里在唱这种调调儿,司令部里也在唱这种调调儿。”哈尔布说。为了加强效果,又慢慢地说:“而这一合唱的领唱人便是我们的上将。”

“唱是唱,但是和以往一样,还没有人倒戈。”

哈尔布说:

“我有一点疑问,这和根本问题有关系。希特勒要第六集团军坚持,保卢斯、魏克斯、蔡茨列尔却表示要拯救官兵的性命,提出要投降。我得到命令,要我秘密地征求意见,斯大林格勒被包围的部队是不是有可能在一定程度上脱离指挥。俄罗斯人把这叫做自由行动。”

他把“自由行动”这个词儿说得很准确、清楚、漫不经心。

列纳尔德懂得问题的严重性,沉默了一阵子。然后他说:

“我想先说说个别情况。”于是他谈起巴赫:“在巴赫的连里,有一个面貌不清的士兵。这个士兵原来是年轻人取笑的对像,可是现在,从被包围的时候起,大家都跟他亲近起来,一齐看着他……我开始考虑他们这个连,考虑这个连的连长。在胜利的时候,这个巴赫是全心全意拥护党的政策的。可是现在我猜想,他的头脑里在发生变化,他在看风向了。所以我就问自己:为什么他连里的士兵和不久前他们天天取笑、又像疯子、又像小丑的一个人亲近起来?这个人在这危难时期会干出什么呢?他会把士兵们带到哪儿去呢?他们的连长又会怎样呢?”

他接着说:

“回答这一切是很难的。但是有个问题我可以回答:士兵们不会造反。”

哈尔布说:

“现在可以特别清楚地看出党的英明了。我们不仅毫不动摇地清除了人民身体上受传染的部分,也清除了表面上健康、但在困难环境中有可能腐烂的部分。各城市、部队、农村、教堂里的自由主义分子和思想敌人都已清除干净。牢骚、怪话、匿名信不管有多少,都没什么事。哪怕敌人不是在伏尔加河上包围我们,而是在柏林把我们包围,也不会有人造反!这一切我们都要感谢希特勒。还应感谢上帝,是上帝在这样的时期给我们派了这个人来。”

他听了听头顶上滚动着的低沉而缓慢的隆隆声。在很深的地下室里,无法听清,是德军的大炮在发射,还是苏联空军的炸弹在爆炸。

哈尔布等到轰隆声渐渐平息下来之后,说:

“您享受普通军官待遇,实在不应该。我把您列入一份名单,在这份名单中都是最受看重的党内朋友和保安人员,师部里会按时把机要通信文件送给您。”

“谢谢,”列纳尔德说,“不过我不希望这样,我只享受别人也享受到的待遇。”

哈尔布把两手一摊。

“曼施坦因怎么样?听说,给他供应了新的装备。”

“我不相信曼施坦因,”哈尔布说,“这方面我赞同集团军司令的看法。”

因为多少年来他说的一切都属于高度机密范围,所以很习惯地用小声说:

“我有一份名单,都是一些重要的党内朋友和保安工作人员,在必要撤离时保证在飞机上有他们的位子。这份名单上也有您。假如我不在,由奥斯津上校代理。”

他看出列纳尔德眼睛里有疑问神情,就解释说:

“可能,我要飞往德国。事情高度机密,所以既不能靠文件,也不能靠电报。”

他眨了眨眼睛,说:

“在起飞之前我要好好地喝一顿,不是因为高兴,而是因为害怕,苏联人打掉很多飞机了。”

列纳尔德说:

“哈尔布同志,我不坐飞机。我劝大家战斗到底,如果我把大家抛下,感到有愧。”

哈尔布微微欠了欠身子,说:

“我没有权利劝您不要这样。”

列纳尔德有意冲淡过分严肃的气氛,就说:

“如果可能的话,请帮助我从司令部回到团里去。因为我没有汽车。”

哈尔布说:

“无能为力!我是第一次完全无能为力!汽油在老狗施密特手里。我一点也弄不到。懂吗?我是第一次!”在他的脸上出现了朴实的、不是他自己本来的—也许正是本来的—表情,正是这种表情使列纳尔德一见面没有认出他来。

三十五

傍晚时候,天气稍微暖和了一些,下了一场雪,把战争的硝烟痕迹和泥污掩盖起来。巴赫在黑暗中巡视着前沿工事。枪响处闪烁着微弱的白光,圣诞节火花一样,白雪被信号弹映照得时而发红,时而泛出闪烁不定的柔和的绿光。

在这一阵阵的闪光中,一条条石头山岭,一个个洞穴,像冻住的波浪似的一道道断墙,新走出的许许多多羊肠小道—有去吃饭走出的、上厕所走出的、搬运弹药走出的、往后方送伤员走出的、掩埋死者走出的—这一切都显得很异常、很特别。同时一切又显得十分熟悉、平常。

巴赫来到一处地方,这地方受到苏军火力控制,一部分苏军就隐藏在一座三层楼的断墙内,现在那里面却响起手风琴声和悠扬的歌声。

墙上的豁口便是苏军前沿的观察点,可以看到一座座工厂的厂房和冰封的伏尔加河。

巴赫唤了一声哨兵,但是没听清岗哨的答话,因为这时有一颗炸弹突然爆炸,冻土块打鼓似的纷纷撞击着楼房的断墙;这是关了马达低空滑翔的苏军小飞机投下的小型炸弹。

“一只瘸腿的俄罗斯老鸹。”一名哨兵说着,指了指黑沉沉的冬日天空。

巴赫蹲下来,胳膊肘撑在一块熟悉的凸出的石头上,四下里打量了一阵子。高高的墙上晃动着淡淡的、红红的影子,这说明苏军士兵在生炉子,烟囱红了,射出暗淡的亮光。看样子,在苏军的掩蔽所里,士兵们在大吃大嚼,在热热闹闹地喝热咖啡。

在右面,在苏军战壕与德军战壕接近的地方,可以听到钢铁撞击冻土的缓慢而低沉的声音。

苏军躲在地下,缓慢然而不断地把自己的战壕向德军推移。像这样在石头般的冻土中推进,其中就有一股笨拙而强大的劲头儿。似乎是土地本身在移动。

下午,一名中士向巴赫报告说,从苏军战壕扔过来一颗手榴弹。手榴弹炸坏了连队锅灶的烟囱,把很多脏东西撒进战壕里。

快到黄昏时候,一名身穿白色小皮袄、头戴新皮帽的苏军士兵从战壕里探出身子,骂起娘来,并且威胁似的挥舞着拳头。

德国人没有开枪,他们本能地明白,这事儿是士兵自发的行动。

那名苏军士兵叫喊起来:

“喂,狗崽子们,想喝俄国酒吗?”

这时从战壕里爬出一名蓝灰色眼睛的德国兵,为了不让军官们听见,用不很大的声音喊道:

“喂,俄国人,不要照头上开枪。还要回家看妈妈呢。你把枪拿去,把皮帽子给我。”

苏军战壕里回答了一句话,而且是很简短的一句。虽然是一句俄语,可是德国人懂了,而且很生气。一颗手榴弹飞来,飞过了战壕,在交通壕里爆炸了。但是已经没有人对这感兴趣了。

中士艾捷纳乌克也把这一情况向巴赫报告了,巴赫说:

“喊就让他们喊吧。没有人跑过去嘛。”

可是这时候,这名满嘴生甜菜气味的中士报告说,士兵别津科费尔不知用什么方式和敌军交换了物品,他的口袋里有方块糖和苏军士兵的面包。他还拿了一名弟兄的刮脸刀代为交换,答应给他换一块炼油和两盒压缩饼干,说定要一百五十克炼油作为代替交换的佣金。

“还有什么好说的,”巴赫说,“马上把他给我叫来。”

可是,原来上午别津科费尔在执行上级的任务时就英勇牺牲了。

“那您想叫我怎么样?”巴赫说。“反正德国人和俄国人早就在做生意了。”

可是中士艾捷纳乌克无意开玩笑。他一九四〇年五月在法国受的伤还没有完全愈合,两个月前就被飞机送到斯大林格勒,离开了德国南部他所服务的警察营。他天天挨饿挨冻,又是虱子咬,又是担惊害怕,一点幽默感都没有了。

那边,一座座隐隐约约、在黑暗中很难看清的白色石头楼房,那是巴赫初到斯大林格勒生活过的地方。满天繁星的九月的天空,浑浊的伏尔加河水,大火之后通红的墙壁,再过去便是俄罗斯东南部的草原,那是亚洲沙漠的边界。

城市西郊的房屋沉没在黑暗中,大雪覆盖的瓦砾呈现在眼前—那就是他的生活……他为什么在医院里给妈妈写那封信?大概妈妈把那封信给古别尔特看了!他为什么要和列纳尔德交谈?

人为什么要有记忆?为什么真想一死了事,什么都不再想起?他在被包围之前不应当对人生那样认真,应当采取疯狂的醉态,应当干他在长期的困难年月里没有干过的事情。

他没有杀害过孩子,一生没有逮捕过什么人。但是他拆毁了很不牢实的保护心灵纯洁、拦阻周围黑暗的堤坝。集中营和犹太人的血朝他涌来,把他漂起,把他冲走,他与黑暗之间的界限已经没有了,他已经成为这黑暗的一部分。

他这是怎么一回事?是不足道的事,是偶然的事,还是他的心灵必然的发展?

三十六

连队的掩蔽所里很暖和。有的坐着,有的躺着,朝低矮的天花板跷着腿,有几个人睡着,用军大衣蒙着头,露着黄黄的光脚板。一名特别瘦的士兵扯着领口,用世界上所有的士兵观察自己的衬衣缝和衬裤缝都会用的仔细而又凶狠的目光打量着衣缝,说:

“你们可记得去年九月咱们住过的那个地下室?”

另一个躺着的士兵说:

“我见到你们,已经是在这儿了。”

有几个人回答说:

“可以说,那个地下室真好……那儿还有床,就像是很讲究的房间……”

“也有人在莫斯科郊外就灰心丧气了。我们却一直打到伏尔加河边。”

有一名士兵在用刺刀劈一块木板,这时他打开炉门,往火里添小木片儿。炉火照亮了他胡子拉碴的大脸,那张脸由灰灰的石头颜色变成红红的古铜色。他说:

“哼,你要知道,用不着得意,咱们是从莫斯科郊外的泥坑来到更臭的泥坑。”

放背包的黑暗的角落里响起一个快活的声音:

“现在倒是很清楚,没有更好的办法过圣诞节啦:吃马肉。”

一谈起吃,大家都活跃起来。大家争论起煮马肉怎样去掉马肉的汗臭味儿。有的说要撇掉滚汤上面的黑沫,有的说不能用大火煮,有的说要把马屁股上的肉去掉,还不能把冻肉放到冷水里,要一下子放进滚水里。

“侦察兵日子过得顶快活,”一名年轻士兵说,“他们可以搞到俄罗斯人的东西,又拿这些东西在地下室里养活自己的俄罗斯娘们儿。可是有的傻瓜还觉得奇怪,不知道为什么有些年轻漂亮的娘们儿就喜欢侦察兵。”

“我现在已经不想那种事儿了,”在生炉子的士兵说,“不知道是情绪问题,还是伙食问题。倒是希望在临死前看看孩子。哪怕看一眼也好……”

“军官们可是在想!我在住着老百姓的一个地下室里见到过连长。他在那儿就像自家人,一家人。”

“你到那个地下室里去干什么的?”

“我吗,我是送衣服去洗。”

“我曾经在集中营里当过看守。常常看到俘虏们捡土豆皮吃,还为烂白菜叶子打架。我那时候想,哼,这简直不是人。谁知我们现在也成了猪。”

堆放背包的黑暗处有一个声音像唱歌一样地说:

“从抢母鸡开的头!”

门突然开了,随着一团团潮湿的热气,出现了浑厚而响亮的声音。

“起立!立正!”

在雾气中闪过巴赫的脸,接着响起陌生的皮靴声,于是掩蔽所里的人看到了师长浅蓝色的军大衣,眯着的近视眼,戴着金戒指、用绒布擦着眼镜的苍老的白手。

他用他那不太用劲就能在练兵场上既让团长们听见又让站在左翼的普通士兵们听见的声音说:

“你们好。稍息。”

士兵们很不整齐地向他问好。将军坐到一个木箱子上,炉火黄黄的光在他胸前的黑色铁十字上掠过。

“平安夜到了,我向你们祝贺。”老将军说。

陪他来的几名士兵把一个箱子抬到炉子旁边,用刺刀把箱盖撬开,从里面拿出一株株用玻璃纸包着的巴掌大小的圣诞枞树。每一株枞树上都装饰着金线、珠子、小小的水果糖。

将军看着士兵们把玻璃纸包解开,招手把上尉叫到跟前,对他小声说了几句话,于是巴赫大声说:

“中将要我告诉你们,圣诞礼物是用飞机从德国送来的,飞行员在斯大林格勒上空受了致命伤,在皮托姆尼卡降落。等到把他从驾驶舱里抬出来,他已经死了。”

三十七

大家用手掌托着小小的枞树。小枞树到了暖和的空气里,挂起许多小小的露珠儿,顿时使地下室里充满枞针气味,驱走了那种难闻的停尸间和铁匠铺的气味—前沿阵地的气味。

坐在炉前的老将军的白头上似乎散发出圣诞节的气味。

巴赫敏感的心感觉出此时此刻的可悲与美妙。这些曾经瞧不起苏军重炮火力的人,这些凶狠、粗暴、挨够了饥饿和虱子咬、苦于弹药不足的人,不用说话一下子就明白了:他们需要的不是绷带、不是面包、不是弹药,而是这些装饰着无用的玩意儿的枞树枝儿,这些孤儿院的小小糖果。

士兵们把坐在箱子上的老将军围住。是他在夏天带领摩托化师的先头部队来到伏尔加河边。他一生时时处处都在做演员。他不仅在队列前演戏,在和司令谈话时演戏,就是在家里,和妻子在一起,在公园里散步的时候,和儿媳妇、和孙子在一起的时候,他都在演戏。夜里他一个人睡在被窝里,他的将军裤放在旁边安乐椅上的时候,他也在演戏。当然,他在士兵们面前也要演戏,当他问起他们的母亲,当他皱起眉头,当他听到士兵们的风流事儿说起粗俗的笑话,当他问到士兵们的伙食而且故作关心地舀起汤尝尝的时候,当他在尚未埋上的士兵坟前垂下严肃的头的时候,当他在新兵队列前发表格外语重心长的、慈父般的讲话的时候,他都是演戏。这种表演不仅在外部,而且发自内心,溶化在思想中、在心中。他不知道他在表演,要把他和他的表演分开是不可能的,就好比无法把盐从盐水中滤出来。他带着他的表演来到连队掩蔽所,他敞开大衣,坐在炉旁的箱子上,都是表演。他镇定而忧伤地看了看士兵们,并且向他们祝贺,也是表演。老将军从来不觉得自己在表演,一旦明白了自己在表演,就表演不成了,就从他身上脱落了,就好比冻结的盐从冷冻的水中分离了出来,剩下淡水,剩下了老年人对挨饿、受罪的人的怜悯心。坐在束手无策的不幸者中间的是一个束手无策、软弱无力的老人。

一名士兵轻轻地唱起一支歌儿:

枞树呀,枞树,

你的针叶多么绿……

有几个人跟着唱起来。针叶的气味使人心醉,儿歌的声音好像圣者的喇叭声:

枞树呀,枞树……

一股股被忘却、被抛弃的感情从海底、从冷冻的深处漂浮出来,早已不再想起的一些念头挣脱出来……

这些念头既不使人愉快,又不使人轻松。但是它们的力量是人的力量,也就是世界上最大的力量。

大口径的苏军炮弹一个接一个沉雷般地爆炸。俄国佬有些生气,显然是猜到被包围的人在过圣诞节。谁也没有注意顶上掉下来的碎土,没有注意炉子里冒出一阵红红的火星。

急促的铁鼓声撞击着大地,大地吼叫着—是俄国佬打起了他们心爱的火箭炮。接着重机枪又嗒嗒响了起来。

老将军坐着,垂着头—这是长期生活劳累了的人常有的姿势。舞台上的灯光熄了,卸了妆的人来到灰色的白日亮光下。现在各种不同的人都一样了。不论是率领摩托化部队进行过闪电式突击的传奇式的将军,微不足道的士官,还是被怀疑有反对国家的不良思想的士兵施密特,全都一样了。巴赫心想,列纳尔德此时此刻是不会受什么影响的,他已经不可能有什么变化,他的德国的、国家的观念不可能变为人的观念了。

他转过头朝门口看了看,却看到列纳尔德来了。

三十八

连里最出色的士兵什通普弗,常常使新兵又怕又敬佩的,现在变了。他那长着一双明亮的眼睛的大脸消瘦了。军服和大衣变成了保护身体、抵挡俄罗斯寒风的皱皱巴巴的旧衣服。他不再说俏皮话,他说的笑话也不使人觉得好笑。

他比别人饿得更难受,因为他的块头大,需要量也大。

因为他天天饿得难受,所以早晨一起来就出去找东西吃。他在瓦砾堆中翻来翻去地寻找,向人讨东西吃,捡面包渣子吃,上厨房里值班。巴赫总是看到他那留神而紧张的脸色。他不仅在空闲时间,而且在作战时间也在想吃的东西,找吃的东西。

巴赫有一次朝居民的地下室走去的时候,看到一名饥饿的士兵宽宽的脊背和宽宽的肩膀。这名士兵在一块空地上翻来翻去地寻找着,这地方在被包围之前是厨房和本团供应科的仓库。他在地上捡白菜叶子,寻找和橡子一样大的冻土豆,当时因为太小没有下锅的。从石头墙后面走出一个高高的老婆子,穿着破烂的男军大衣,腰里扎着绳子,脚上穿的是穿坏了的男式足球鞋。她迎着士兵走来,凝神注视着地面,用一个粗铁丝做成的钩子在雪地上扒拉着。

他们都没有抬头,从雪地上碰到一起的影子互相看到了。

大块头德国兵抬起眼睛看着高大的老婆子,带着信赖的神气在她面前拿着一片烂了不少窟窿的云母色的白菜叶子,慢慢地,因此显得很庄重地说:

“您好,老太太。”

老婆子慢慢撩开溜到额头上的头发,用善良而聪明的黑眼睛看了一眼,很庄重地慢慢回答说:

“你好,先生。”

这是两个伟大民族的代表最高水平的会见。除了巴赫,谁也没看到这次会见,士兵和老婆子也很快忘记了这次会见。

天气暖和一些了,大片大片的雪花落到地上,落到红红的碎砖上,落到坟前十字架的横木上,落到被打坏的坦克上面,落进未掩埋的死者的耳朵眼儿里。

暖和的雪雾呈现出青灰色。大雪把空中填塞得满满的,把风挡住,把枪炮声淹没,把大地与天空连接混合成一个模模糊糊、轻轻颤动的、柔和的、灰色的整体。

雪花一片一片地落在巴赫的肩膀上,似乎是一片一片的寂静落在安静的伏尔加河上,落向死寂的城市,落向一匹匹马的骨架;到处都在下雪,不仅是在大地上,而且在星星上,整个寰宇到处都是雪。死者的尸体、武器、带脓血的破布、碎砖碎石、炸得弯弯扭扭的钢铁,全都被埋到雪底下。

这不是雪,这是时间,柔软而洁白的时间,落向人类争夺城市的战场,一层一层地往上铺,于是今天渐渐变成过去,而且在慢慢闪动的毛茸茸的雪中没有未来。

三十九

巴赫躺在印花布幔后面的一张床上,在地下室的一个很小的隔间里。一个睡着了的女人的头枕在他的肩上。她的脸因为太瘦,很像一张孩子脸,同时又像一张衰老的脸。巴赫看着她那细细的脖子和肮脏的灰色衬衣里露出来的白白的胸脯。他为了不把女子弄醒,轻轻地、慢慢地把她的松开的辫子拉到嘴唇上。头发有一股香气,有一股生气,带有弹性,而且热热乎乎的,好像有血在头发里流着。

女子睁开了眼睛。

这个讲求实际的女人有时无忧无虑,又可爱又滑头,又能忍耐又有心计,又驯顺又爱发脾气。有时她似乎很傻,很消沉,常常愁眉苦脸。有时她唱唱歌儿,她唱的俄语歌儿有时带有德国歌曲的调儿。

他没有问过她在战前是干什么的。他想来找她,就来找她。他不想和她睡觉的时候,就想不起她来,不操心她是不是能吃饱,苏联狙击手是不是把她打死了。有一次他从口袋里掏出他偶然得到的一块干饼,给了她,她十分高兴,可是后来她把这块干饼给了和她住在一起的一个老婆子。这使他非常感动。不过,他每次来找她,差不多总是忘记带点儿什么吃的东西。

她的名字很奇怪,叫季娜,不像欧洲人的名字。

季娜显然在战前并不认识那个和她住在一起的老婆子。是一个令人讨厌的老婆子,又爱说奉承话,心眼儿又坏,虚伪得不得了,酒瘾也大得不得了。这会儿她正在很有节奏地拿一根原始的木杵在木臼里捣着,在舂烧糊而且洒过煤油的黑黑的小麦。

在被包围以后,士兵们就开始常常到一些地下室里去找老百姓。以前士兵们从来不理会老百姓,现在有很多事情要到那些地下室里去办:不用肥皂而用草木灰洗衣服,把一些废渣做成吃的东西,缝补衣服。地下室里的人主要是一些老婆子。但是士兵们不光是去找老婆子。

巴赫以为,谁也不知道他上这个地下室里来。但是有一次,他正坐在季娜的床上,握着她的手,却听见布幔外面有人说德语,有一个似乎很熟悉的声音说:

“别上这布幔里面去,上尉先生在里面。”

这会儿他们在一块儿躺着,没有说话。他的一生—朋友、书籍、他和玛利亚的恋爱、他的童年、他出生的城市里的一切、他上的中学和大学、轰轰隆隆地远征俄罗斯,这一切都已失去意义……这一切成为一条道路,通向这张用烧糊的木板拼成的板床……他一想到他可能失去这个女子,就觉得十分害怕。他找到了她,他上她这儿来了,在德国、在欧洲发生的一切,都是为了他能遇到她……以前他不懂得这一点,他常常把她忘了,他觉得她可爱,正因为他和她的关系丝毫没有什么认真的成分。现在除了她,在这世界上什么都没有了,一切都沉没在雪里……只有这张很美的脸、这微微向上翻的鼻孔、奇怪的眼睛和这使人着魔的、孩子般的可怜而又慵懒的神情。她在十月间在战地医院里找到了他,步行去看他,可是他不愿意见她,没有出来和她见面。

她看到他没有喝醉。他跪下来,吻起她的手,又吻起她的脚,然后抬起头来,把额头和脸颊贴到她的膝盖上,他很快、很急切地说起话来,可是她不懂他的话,他也知道她不懂他的话,因为她只懂保卫斯大林格勒的士兵说的那种可怕的话。

他知道,这场战争使他遇到这个女子,现在这场战争就要使她和他分手,使他们永远分开。他跪着,搂住她的腿,看着她的眼睛,她听着他说得很快的话,很想明白、很想猜出他说的是什么,他是怎么一回事儿。

她从来没见过德国人的脸上有这样的表情,她原来以为,只有俄罗斯人才会有这样痛苦、这样恳求、这样可爱、这样失魂丧魄的眼神。

他在对她说,他在这地下室里,吻着她的脚,第一次不是从别人的话里,而是凭自己的心灵懂得了爱情。他觉得她比他过去的一切都可贵,比母亲、比德国、比他今后将和玛利亚过的生活更可贵……他爱上了她。国家筑起的高墙、民族仇恨、重炮的弹幕射击都算不了什么,都抵不过爱情的力量……

他感谢命运,是命运让他在死亡的前夕懂得了这一点。

她不懂他的话,只懂得德国人常说的一些要东西和骂人的话。

但是她猜到他是怎么一回事儿,她看出他的慌乱神情。这个德国军官的饥饿而轻浮的恋人带着宽容而爱怜的心情看出他的软弱。她明白,命运就要使他们分手了,她比他要平静些。这会儿她看着他的绝望神情,感觉到她和这个人的关系正在变为感情,这感情的强烈与深厚使她十分吃惊。这是她在他的声音中听出来,在他的狂吻中感觉出来,在他的眼睛里看出来的。

她带着沉思的神情抚摩着巴赫的头发。在她的机灵的头脑里却出现了一种担心的想法:这股模模糊糊的力量可别把她抓住,把她捆起来,把她害死……她的心紧张地跳着,跳着,她不想听那狡猾的、使她觉得有危险、使她害怕的声音了。

四 十

叶尼娅认识了一些新朋友,都是在监狱接待室排队的人。他们常常问她:

“您怎么样,有什么消息吗?”

她已经有了经验,所以不光是听别人劝吿,自己也说说:

“您不要担心。也许,他在医院里呢。在医院里挺好,都想离开牢房上医院里去呢。”

她已经打听到克雷莫夫就在内部监狱里。他们不肯收她送的东西,不过她没有灰心丧气,因为在库兹涅茨桥常常是这样,一次不收,两次不收,到后来他们突然会自己提出来:

“把东西交给我吧。”

她上克雷莫夫原来的房子里去过,女邻居对她说,两个多月前有两名军人和房屋管理员来过,把房门打开,拿走了很多文件和书,把门封起来,就走了。叶尼娅看着带有绳子状小尾巴的火漆印,站在旁边的女邻居说:

“不过,您行行好,千万不要说是我说的。”

等她把叶尼娅送到门口,又鼓了鼓勇气,小声说:

“他可真是一个好人呀,他是自愿上前方的。”

她在莫斯科没有给诺维科夫写信。她的心里很乱!又是怜惜,又是爱,又是后悔,又为前方的胜利高兴,又为诺维科夫担心,觉得对不起他,怕永远失掉他,又因为无可奈何感到痛苦……不久之前她还在古比雪夫,准备到前方去找诺维科夫,她觉得她和他的关系是理所当然的,是无法拆散的,就像命中注定了的。但叶尼娅怕的是,永远和诺维科夫联系在一起,就将永远和克雷莫夫分开。诺维科夫的一切有时使她觉得很陌生。她觉得他所操心的事、指望的事、他的朋友圈子全是陌生的。她觉得为他招待客人,接待朋友,和将军夫人、上校夫人们交往,是不可思议的。

她想起诺维科夫对契诃夫的《主教》和《没意思的故事》都不感兴趣。他倒是更喜欢德莱塞和福伊希特万格那些带有倾向性的小说。可是现在,当她明白她和诺维科夫的分手已成定局,她再也不会回到他身边的时候,她却觉得她在爱着他,常常想起他是怎样百依百顺,不论她说什么,他都连忙表示赞同。叶尼娅感到很痛苦:难道他的手永远不再抚摩她的肩膀,她再也看不到他的脸了吗?

她从来没遇到过刚强、决绝与人性、胆怯这样奇怪地结合到一起。她是那样爱他,他一点也没有那种残酷的狂热,他有一种特别的、通情达理和朴素的男子汉的善良。她一想到她和亲人的关系中出现了阴暗的、不纯洁的成分,马上就觉得惶惶不安。保安机关怎么知道克雷莫夫对她说的话呢?……她和克雷莫夫的关系是不可轻视的,她和他过的一段生活无法一笔勾销。

她要跟克雷莫夫一起走。就算他不原谅她,她该当永远受他的责备,但是他是需要她的,他在监狱里一直想着她。

诺维科夫和她分离会感到痛苦,但是他能撑得住。可是她却不明白,究竟怎样她心里才能平静。要是知道他已经不再爱她,已经安下心来,已经原谅了她,她心里就平静了吗?还是相反,知道他还爱她,还十分苦恼,还不原谅她,她心里就平静吗?而且对她自己来说,究竟怎样更好呢?是知道他们已永远分手,还是在内心深处相信他们还会在一起?

她给亲人造成多大的痛苦呀。难道这一切她不是为了别人幸福,而是因为自己古怪,是为了自己吗?真是个精神变态的疯子!

晚上,当维克托、柳德米拉、娜佳坐下来吃饭的时候,叶尼娅看着姐姐,忽然问道:

“你可知道,我是什么人?”

“你吗?”柳德米拉惊讶地问。

“是的,是的,我。”叶尼娅说。并且自己声明说:“我是一条小狗,女性的。”

“是小母狗吗?”娜佳快活地说。

“是的,是的,就是的。”叶尼娅回答说。

忽然大家一齐哈哈大笑起来,虽然知道叶尼娅没有心思笑。

“你们听我说,”叶尼娅说,“在古比雪夫有一回里蒙诺夫到我那儿来,对我说过婚外情是怎么一回事儿。他说,这是一种精神上的维生素缺乏症。比如说,丈夫和妻子在一起过长久了,他就会发生精神饥饿,就像老牛缺乏盐,或者像极地工作人员几年见不到蔬菜。妻子成了一个为所欲为的、专横、强硬的人,于是丈夫就开始盼望有―个亲切、温柔、百依百顺、羞涩的女子。”

“你那个里蒙诺夫是浑蛋。”柳德米拉说。

“要是一个人缺乏A、B、C、D这几种维生素,又会怎样呢?”娜佳问道。

后来,等大家都已经准备睡觉的时候,维克托说:

“叶尼娅,我们常常讥笑知识分子像哈姆雷特一样充满矛盾,讥笑知识分子多疑,不坚定。我在年轻时也很鄙视这些特点。可是现在我的看法不同了:有些人之所以能有伟大的发明,能写出伟大的作品,就因为他们不坚定和怀疑,他们做的事情不比那些宁折不弯的人少。如果有必要,他们也会赴汤蹈火,也会到枪林弹雨之下,一点也不比那些刚强的、宁折不弯的人差。”

叶尼娅说:

“谢谢,维克托,你这是说的小母狗吗?”

“就是。”维克托说。他很想对叶尼娅说一些开心的话。

“叶尼娅,我又看了看你的画,”他说,“我喜欢的是,画里有感情,要不然就会像那些左派画家一样,画里只有勇敢和革新,而没有灵魂了。”

“哦,还感情呢,”柳德米拉说,“绿色的男子,蓝色的房子。完全脱离了实际。”

“你可知道,”叶尼娅说,“马蒂斯说:‘我用绿颜色的时候,并不意味着我要画青草;我用蓝颜色的时候,并不意味着我要画天空。’颜色表现的是画家的内心感情。”

尽管维克托一心想对叶尼娅说说开心的话,可是他还是忍不住用取笑的口吻插话说:

“可是埃克尔曼却说:‘如果歌德像上帝一样创造世界,他还是把草创造成绿的,把天空创造成蓝的。’这话我听说过很多遍了,可是我对我用来创造世界的物质另有一种态度……是的,所以我知道,既没有颜色,又没有颜料,只有原子和原子之间的空间。”

但是这一类的谈话是不多的,大部分谈的是战争、检察机关……

这是很难过的日子。叶尼娅准备回古比雪夫。她的假期快完了。

她很怕向领导解释。因为她是擅自上莫斯科来的,接连好几天她天天上监狱去,而且向检察机关和内务人民委员部写了申诉书。

她一生害怕官场,害怕写呈文,每次在换身份证之前她都睡不好觉,提心吊胆。可是近来似乎命运强迫她只能和公安局、检察机关打交道,只能和户口簿、身份证、传票、申诉书打交道。

姐姐家里有一种很不自然的安静气氛。

维克托不去上班了,经常一个人坐在自己的房间里。柳德米拉从配给商店回来,总是心情很坏,很难过,说一些熟人的家属不和她打招呼了。

叶尼娅看出来,维克托的神经十分紧张。他一听到电话铃声就哆嗦,急忙抓起话筒。在吃午饭或吃晚饭的时候常常突然打断别人的话,说:“别作声,别作声,我好像听到有人按门铃。”他便去开门,回来时很不自然地笑着。姐妹俩心里明白,为什么他总是紧张地等待着门铃响—他是怕逮捕。

“迫害恐惧症就是这样害起来的,”柳德米拉说,“在一九三七年精神病医院里住满了这样的人。”

叶尼娅看到维克托天天这样提心吊胆,所以他对她的态度就特别使她感动。有一次他说:

“叶尼娅,你记住,你住在我家,为被捕的人操心,不管人家怎么想,我一点也不在乎。你明白吗?这就是你的家!”

晚上,叶尼娅很喜欢和娜佳谈谈。

“你太聪明了,”叶尼娅对娜佳说,“你不像一个小姑娘,倒是像以前的苦役政治犯秘密团体的一名成员。”

“不是以前,而是未来的,”维克托说,“你大概常常和你那位中尉谈政治了。”

“谈又怎样?”娜佳说。

“顶好还是光接接吻。”叶尼娅说。

“我也是这样说,”维克托说,“这样总要安全些。”

娜佳确实老是想谈谈一些尖锐的问题。有时她忽然问起布哈林,有时问,列宁是不是真的很看重托洛茨基,列宁在生前最后几个月是不是很不愿意见斯大林,是不是列宁有一份遗嘱被斯大林隐藏起来,不让人民知道。当叶尼娅单独和她在一起的时候,并没有向她问起洛莫夫中尉的事。

但是,从娜佳谈政治、谈战争、谈曼德尔施塔姆和阿赫玛托娃的诗、谈自己和同伴们的聚会和谈话,叶尼娅了解了洛莫夫以及娜佳和他的关系,比柳德米拉了解的还多。

洛莫夫显然是一个很尖刻的小伙子,性格孤僻,对一切公认的、有定论的事抱嘲笑态度。他显然自己在写诗,所以娜佳受他的影响,嘲讽和蔑视别德内依和特瓦尔多夫斯基,对肖洛霍夫和奥斯特洛夫斯基不感兴趣。显然,有时娜佳耸着肩膀说的就是他的话:“革命者要么是愚蠢,要么是欺骗人。不能为虚构的未来的幸福,牺牲整个一代人的生命嘛……”

有一次娜佳对叶尼娅说:

“小姨,你可知道,老一代的人一定需要信仰一点儿什么:克雷莫夫信仰列宁和共产主义,爸爸信仰自由,外婆相信人民和干活儿的人,可是我们新一代认为这都是愚蠢的。总的说,信仰就是愚蠢。应当过没有信仰的生活。”

叶尼娅突然问道:

“这是你的中尉的哲学吗?”

娜佳的回答使她吃了一惊:

“再过三个星期,他就上前线了。从生到死—这就是他的全部哲学。”

叶尼娅和娜佳谈着谈着,不觉想起了斯大林格勒。薇拉就是这样和她谈心,薇拉就是这样谈起恋爱。可是薇拉那种单纯而分明的感情和娜佳的怅惘多么不同啊。叶尼娅那时候的生活和她今天的情形多么不同啊。那时候关于战争的一些想法和今天在胜利的日子里的一些想法多么不同啊。可是,战局变化了,娜佳说的“从生到死”并没有变化。至于一个人以前是不是喜欢弹着吉他唱歌,是不是志愿参加过伟大的建设,相信共产主义的远景,是不是读过阿年斯基的诗,不相信虚幻的后代的幸福,对于战争都无关紧要。

有一天,娜佳拿出一首手抄的劳改营歌曲给叶尼娅看。

歌里说到寒冷的船舱,说到大洋上怒吼的风涛,说到“犯人们在轮船上颠簸,紧紧拥抱,好像亲兄弟”,说到迷雾中出现了马加丹—“科雷马地区首府”。

刚来莫斯科的时候,娜佳一谈起这一类的话题,维克托就很生气,不叫她说下去。

可是在这些日子里,他有很多变化。现在他常常按捺不住,就当着娜佳的面说,看到那些歌功颂德的祝贺信,简直恶心,什么“伟大的导师,体育工作者的好朋友,英明的父亲,雄才巨擘,光辉的天才”,还有那些话,又是谦虚的,又是关心群众的,又是慈祥的,又是体察民情的。造成一种印象,似乎斯大林在耕地,炼钢,在托儿所用羹匙喂小孩子,拿机枪作战,而工人、士兵、学生和学者们只要向他祈祷就行了,并且,假如没有斯大林,整个伟大的民族就会像可怜的牲口一样死掉。

有一天维克托数了数,斯大林的名字在这一天的《真理报》上被提到八十六次,第二天他看到一篇社论中就有十八次提到斯大林的名字。

他抱怨非法的逮捕,抱怨没有自由,抱怨任何一个没有什么文化而有党证的领导人都认为自己有权指挥科学家和作家们,有权评价他们的高低,教导他们。

他产生了一种新的心情。对于国家发怒的歼灭性力量,他越来越害怕,越来越感到孤独、可怜,像小鸡一样软弱无力,感到大祸临头,因而有时产生一种绝望,一种生死由命、听之任之的心情。

早晨,维克托跑到柳德米拉的房间里,柳德米拉看到他脸上那种兴奋和欢喜的表情,简直不知如何是好,因为在他脸上出现这种表情太不平常了。

“柳德米拉,叶尼娅,咱们又踏上乌克兰的土地了,刚才广播的!”

下午,叶尼娅从库兹涅茨桥回来,维克托看了看她的脸,就像早晨柳德米拉问他那样向她问道:“怎么啦?”

“把东西收下了,把东西收下了!”叶尼娅连说了两遍。

就连柳德米拉也明白,转交的东西和叶尼娅附上的信对于克雷莫夫将意味着什么。

“死者要复活了。”她说。接着又说:“恐怕,你还是爱他的,我没见过你这样的眼神。”

“你要知道,我大概是疯了,”叶尼娅小声对姐姐说,“要知道我这样高兴,一方面是因为克雷莫夫能够收到我的东西,另一方面因为今天我明白了:诺维科夫不可能,绝对不可能干卑鄙的事情。你懂吗?”

柳德米拉十分生气,说:

“你不是疯了,你比疯了还坏。”

“维克托,我求求你,给我们弹一支曲子吧。”叶尼娅恳求说。

在这一段时间里,他从来没有弹过钢琴。但是现在他不推却,拿来乐谱,给叶尼娅看了看,问:

“就这一支,好吗?”

柳德米拉和娜佳一向不喜欢听音乐,便上厨房里去了,维克托就弹起来。叶尼娅听着。他弹了很久。弹完一曲,他没有说话,也没有看叶尼娅,后来又弹起另一支乐曲。有时候她觉得,维克托在哭泣,可是她看不到他的脸。门忽然一下子开了,娜佳叫道:

“快打开收音机,有命令!”

钢琴声停了,响起钢铁般洪亮的声音,此刻正是播音员列维坦在播音:“我军发动强攻,收复了这座城市和重要的铁路枢纽站……”然后列举了在战斗中表现特别出色的一些将军和部队,列举的第一个名字是集团军司令托尔布欣。列维坦那兴奋的声音忽然说:“还有诺维科夫上校统率的坦克军……”

叶尼娅轻轻地“啊”了一声,后来,等到播音员用深沉而动情的声音说“为祖国独立和自由而牺牲的英雄永垂不朽”,她已经哭了起来。

四十一

叶尼娅走了,维克托家里只剩下一片忧伤气氛。

维克托常常一连几个钟头坐在书桌旁,一连几天不出家门。他很害怕,似乎到街上他就会遇到特别使人不快的、敌视他的人,会看到他们那杀气腾腾的眼睛。

电话铃完全哑了,如果两三天中有一次电话铃响,柳德米拉就说:

“这是找娜佳的。”

确实不错,是打给娜佳的。

维克托不是一下子就明白他的事情的严重性的。最初几天他甚至感到很轻松,因为他可以安安静静地坐在家里,置身于他心爱的书中,看不到那些不怀好意的、阴沉的眼睛。

但是家里的安静很快就使他难受起来,这种安静不仅使他苦恼,而且使他惶惶不安。实验室里怎么样了?研究进行得怎样?马尔科夫在干什么?他一想到实验室里正需要他,他却坐在家里,就觉得十分着急。但是,反过来想,想到实验室里没有他照样很好地在干着,他也十分难受。

柳德米拉在街上遇到疏散中的女友斯托伊尼科娃,是在科学院机关工作的。她对柳德米拉详细地说了说学术委员会会议的情形,因为她自始至终担任会议记录。

最主要的是,索科洛夫没有发言!他没有发言,尽管希沙科夫对他说:“索科洛夫同志,我们想听听您的意见。您和施特鲁姆在一起工作多年。”他回答说,夜里他的心脏病发作过,说话很困难。

但是很奇怪,维克托听到这个消息并没有丝毫感到高兴。

代表实验室发言的是马尔科夫。他说话比别人有分寸,不说是政治问题,主要是说维克托的脾气不好,甚至还提到他的才气。

“他不能不发言,他是党员嘛,不发言不行,”维克托说,“不能怪他。”

但是大多数发言都是很可怕的。科甫琴科似乎把维克托说成是骗子和坏蛋。他说:“这个施特鲁姆不来开会,太不像话了,我们要换一种方式和他说话,看样子,他就希望这样。”

白发苍苍的普拉索洛夫,就是曾经把维克托的著作与列别杰夫的著作相提并论的那位,说:“某些人围绕着施特鲁姆的可疑的空论,发动了一场无耻的叫嚣。”

物理学博士古列维奇的发言也很恶劣。他说,他曾经过高估计维克托的著作,是犯了很大的错误,并且暗示说维克托有民族偏执性,说,在政治上糊涂的人在科学上必然也糊涂。

斯维琴把维克托称作“可敬的”,并且援引了维克托说过的话,即:物理学是统一的,不分美国物理学、德国物理学、苏联物理学。

“是有这么一回事儿,”维克托说,“不过在会上引用私人之间说的话,就等于告密。”

使维克托吃惊的是,皮敏诺夫也在会上发了言,虽然他已经和研究所没有关系,没有人迫使他发言。他检讨说,他过高地估价了维克托的著作,而没有看到著作的缺陷。这实在是令人吃惊的。因为皮敏诺夫说过,维克托的著作挑起他祈祷的心情,说他能够有助于这一著作的出现,感到无限幸福。

希沙科夫说的不多。研究所党委书记拉姆斯科夫提出决议方案。决议是很严厉的,要求院部清除腐烂部分,保护健康的集体。特别令人气愤的是,决议中只字不提维克托·施特鲁姆的科学成就。

“总归索科洛夫的表现还是十分正派的。可是究竟为什么玛利亚不和咱们来往了呢,难道他这样害怕吗?”柳德米拉说。

维克托什么也没有说。

真奇怪!他没有生任何人的气,虽然他没有耶稣那样宽恕一切的度量。他没有生希沙科夫的气,也没有生皮敏诺夫的气。他也不恼恨斯维琴、古列维奇、科甫琴科。只有一个人使他十分生气,使他气得难受,气得发胀,他一想到他,就浑身发热,连气也喘不过来。似乎一切反对维克托的残酷无情、不公正的事都是来自索科洛夫。索科洛夫怎么能不准玛利亚上维克托家里来!多么胆怯,多么无情,多么卑鄙,多么下贱!

但是他却不敢对自己承认,他所以这样懊恼,不仅是认为索科洛夫对不起他,也因为他暗暗感觉到自己也对不起索科洛夫。

现在柳德米拉常常谈起生活方面的事。

多余的住房面积、房管所要的工资证明、食品供应卡、划定供应的新食品店、新的季度的限额供应卡、过期的身份证和换身份证时必须出具的机关证明—这一切都是柳德米拉日日夜夜操心的事。还有,到哪儿去弄钱来过日子?

以前维克托常常很带劲儿地开玩笑:“我要研究研究家庭的理论问题,成立一个家庭实验室。”但是现在没有什么好笑的了。他这个科学院通讯院士拿到的津贴勉强可以偿付住房、别墅租金和水电煤气费。况且,他充满了孤独感。

可是,总得过日子。

到高等学校去教书,他也不行了。一个在政治上有污点的人不能再接触青年人了。

上哪儿去呢?他因为在科学界有相当的地位,也无法去做卑微的工作。任何一个干部见到一个科学博士要干技术编辑或中学物理教员,都会“啊嘿”一声,不给办手续。

当他一想到自己的研究完了,想到自己的穷困,想到受人支配、受人欺凌,觉得特别难受的时候,就在心里想:“还不如快点儿坐监狱呢。”可是那样柳德米拉和娜佳就没有人管了。她们还要过日子。还说什么上别墅采草莓来卖呀!人家就要把别墅收回了。因为到五月里就要办理续租手续了。别墅不是科学院的,而是政府部门的。他因为马虎没有及时交租金,本想把拖欠的租金和上半年的预付金一把交齐。一个月之前这点儿钱在他算不了什么的,现在这数目就使他觉得可怕了。

上哪儿去弄钱?娜佳还需要一件大衣呢。

去借债?可是,没有还债的指望,不能借债。

变卖东西?可是,在战争时期谁又买瓷器,买钢琴?而且也舍不得,柳德米拉很喜欢她收藏的瓷器之类,就连现在,托里亚牺牲之后,她有时还欣赏欣赏这些东西。

他常常想,还不如上兵役局去,放弃科学院的免征权,去要求当一名士兵,上前线去。

他一想到这里,心里就平静下来。

可是接着又出现了焦虑和痛苦的想法。柳德米拉和娜佳怎么过呢?去教书?把房子交出去?他马上就想到房管所和民警。夜间搜捕,罚款,记录。房屋管理员、地段民警督察、区房产科监察、人事处女秘书,对于一个老百姓来说,这些人有多么厉害,多么威风,多么了不起。一个失去依靠的人,会感到连坐在票证科的小姑娘都是一种强大的、不可动摇的力量。

维克托在整个一天里都觉得恐惧,无能为力,绝望。但是他的心情不是始终一样的,不是毫无变化的。一天中不同的时间有不同的恐惧,不同的苦恼。早晨起来,刚刚出了暖和的被窝,当窗外还是寒冷而朦胧的晨曦的时候,他就像一个孩子遇到巨大的力量袭来,感到有一种无可奈何的心情,很想钻回被窝里,蜷起身子,皱紧眉头,一动不动。

上午,他思念他的研究工作,特别想上研究所去。这时他觉得自己成了没有人要的人,成了无用、无能的人。

似乎国家一发怒,不仅能够剥夺他的自由、他的安宁,而且能够剥夺他的智慧、他的才华、他的自信心,把他变成一个又呆、又笨、又灰沉的人。

快到吃午饭的时候,他有了精神,高兴起来。可是一吃过午饭就苦恼起来,愚钝,沉闷,什么也不想。

等到暮色渐浓,恐怖也随之渐强。他现在很怕黑暗,就像石器时代的野人进入了黑沉沉的密林。恐怖越来越剧烈,越来越厉害……维克托思前想后,往事今朝一齐涌来。残酷无情、不肯饶人的死神在窗外黑暗中等待着。外面就会响起汽车声,马上就会响起门铃声,房子里马上就会响起皮靴声。无处躲藏。突然,又来了一种发狠又痛快的冷漠心情,一切都无所谓了!

维克托对柳德米拉说:

“沙皇时代那些叛乱的贵族倒是快活。失宠之后就坐上马车,离开京城,到奔萨的领地上去!在那儿可以打猎,可以在农村寻欢作乐,有邻居,有花园,写写回忆录。可是,你们这些自由主义的知识分子试试看:两个星期的审查和鉴定往密封的档案袋里一装,想打扫院子都没有人要你。”

“维克托,”柳德米拉说,“咱们能过得去!我可以缝衣服,在家里给人家做活儿,可以绣手帕,还可以去做试验员。可以养活你。”

他吻了吻她的手。她不明白,为什么他的脸上出现了负疚和痛苦的表情,他的眼睛里出现了诉苦和祈求的神情……维克托在房间里踱着,小声唱着古老的情歌:

……他孤单单,无人相伴……

娜佳听说爸爸想当志愿兵上前线,说:

“我有一个女同学叫托尼娅·科干,她爸爸当了志愿兵。他是古希腊学科的专家,进了奔萨的一个预备团,分派他在那儿打扫厕所。有一天连长来上厕所,他因为近视把脏东西扫到连长身上,连长照他的耳朵打了一拳,把鼓膜都打破了。”

“那有什么,”维克托说,“我不把脏东西扫到连长身上就是了。”

现在维克托跟娜佳说话,就和跟大人说话一样了。他对女儿似乎从来没有像现在这样好过。近来她一放了学就马上回家,这使他很感动,他认为这是她不希望让他担心。和爸爸说话的时候,她那一向带有讥笑神气的眼睛里出现了新的神气—严肃而温柔的神气。

有一天晚上,他穿起大衣,朝研究所走去。他很想朝自己的实验室的窗户里看看:里面的电灯是不是亮着,是不是有人在上夜班,也许,马尔科夫已经完成设备安装了吧?但是他没有走到研究所,怕碰见熟人,便拐进一条巷子,拐弯朝家里走。巷子里很黑,空荡荡的。他忽然感到十分幸福。雪花,夜晚的天空,寒冷的新鲜空气,脚步声,黑郁郁的枝丛,木头小房窗户里透过伪装窗帘射出来的细细的一缕灯光—这一切都十分美好。他呼吸着夜晚的空气,他在安静的小巷里走着,谁也看不到他。他还活着,他还是自由的。他还要什么,幻想什么呢?他来到家门口,幸福感就消失了。

起初几天,他紧张地等着玛利亚到来。一天天过去,玛利亚没有给他来过电话。他的研究,他的名声,他的安宁,他的自信心,一切都被剥夺了。难道也把他最后的庇护所—爱情,夺走了吗?

有时他灰心绝望,用手抓住自己的头发,好像他看不见她就没法活下去。有时他嘟哝说:“这有什么,这有什么,这有什么。”有时他自己对自己说:“现在谁还喜欢我呀?”

可是在他绝望的深处还有一个小小的光明点—就是他和玛利亚保持着心灵的纯洁。他们很痛苦,但是没有给别人造成痛苦。但是他明白,他的一切想法,哲学上的想法,平静的想法,恼恨的想法,都不能回答他心中出现的问题。

他生玛利亚的气,他嘲笑自己,他悲伤地听天由命,他想着对柳德米拉的责任,想着如何对得起良心—这一切都只不过是为了战胜他的绝望。每当他想起她的眼睛、她的声音,他就苦恼得不得了。难道他再也看不到她了?

当他感到分手不可避免,感到失落得难以忍受的时候,他就不顾内心的羞愧,对柳德米拉说:

“你知道,我一直在担心马季亚罗夫,不知道他会不会出什么事儿,不知道是不是有他的消息。你打电话问问玛利亚,好吗?”

最奇怪的也许是他还在继续进行研究。他研究是在研究,可是苦恼、不安、痛苦并没有停息。研究不能帮助他战胜苦恼和恐惧,研究没有成为他的精神良药,他并非希望通过研究忘却难受的念头,忘却心灵的绝望。研究比药物的力量更强大。他还在研究,因为他不能不研究。

四十二

柳德米拉对维克托说,她遇到房管员,他请维克托上房管所去一趟。

他们就猜因为什么要叫他去。因为住房面积超标?换身份证?兵役局要检查?也许,有人报告了叶尼娅没有登记就在这里住过?

“你当时就该问一下,”维克托说,“那样咱们就用不着在这里费脑筋了。”

“是的,当时应该问,”柳德米拉也说,“可是我慌了,因为他说,叫你丈夫上午来吧,反正他现在不上班了。”

“啊,天呀,他们已经全知道了。”

“管院子的,开电梯的,邻居家的保姆,都在看着嘛。有什么奇怪的?”

“是的,是的。你可记得,战前来过一个年轻人,带着红红的小本子,要你向他报告,有谁上邻居家来过?”

“我怎么不记得,”柳德米拉说,“我不客气地大声骂了他一句,他只在门口说了一句‘我以为你很有觉悟呢’,就走了。”

这件事柳德米拉说过很多遍。他平时听她说的时候,总要插话,为的是让她说简单些,可是现在他一再要求她说说详细情形,再不催她。

“你听我说,”柳德米拉说,“也许,是因为我在市场上卖了两块桌布?”

“我认为不是。如果是那样,就不会单单叫我去,也应该叫你去。”

“也许,是要你签什么字?”柳德米拉犹犹豫豫地说。

他的心绪异常阴沉。他一直想着他和希沙科夫、和科甫琴科谈的话,他说的话太危险了。他想起在大学里的时候,那时候他说话太随便了。他和米佳争论过,和克雷莫夫争论过,虽然有时他也赞成克雷莫夫的观点。可是他这一生从来没有敌视过党,敌视过苏维埃政权。忽然他想起他在某地、某时说过的一些特别尖锐的话,不觉浑身都凉了。可是克雷莫夫这个坚定的、坚持思想原则的共产党员,这个狂热的信徒,从来不怀疑什么的,却被逮捕了。他和马季亚罗夫、和卡里莫夫说过那么多离经叛道的话,又会怎样呢?多么奇怪呀!

通常一到傍晚,黑暗渐渐来临的时候,他就战战兢兢地想到可能要逮捕他,而且恐惧感越来越强,越来越厉害,越来越使他受不了。但是等到他觉得完蛋已成了定局,他就一下子快活起来,轻松起来!哼,去他的吧!

一想到他的研究成果得到的不公正待遇,似乎他就要发疯了。但是当他一想到他又笨又蠢,想到他的研究不过是对现实世界的粗野、无味的嘲弄,思想不再是思想,而成为一种活着的感觉时,他就愉快起来。

现在他甚至根本不再考虑检讨自己的错误。他是渺小可怜的,是无知的,检讨也不会有什么改变。谁也不要他。不论检讨不检讨,愤怒的国家都把他看得一文不值。

在这段时间里,柳德米拉变化得很厉害。她已经不在电话里对房管员说:“请您马上给我派一个修理工来。”不再到楼梯上去检査:“这是谁又把垃圾倒在洞口外面?”她穿衣服有点儿不正常,摸到什么穿什么。有时到配给商店去买素油,毫无必要地穿起名贵的皮大衣;有时扎起灰色的旧头巾,穿起战前就想送给电梯女工的大衣。

维克托看着柳德米拉,心里想着他们两个再过十年、十五年,会是什么样子。

“你可记得,在契诃夫的《主教》里,母亲放牛,对一些妇女说,她的儿子当年做过主教,可是很少有人信她的话?”

“我读过已经很久了,那还是在小时候,不记得了。”柳德米拉说。

“那你要再读一读。”维克托很生气地说。

他一直因为柳德米拉不喜欢契诃夫而生她的气,他怀疑,契诃夫有很多小说她没有读过。

可是很奇怪,很奇怪!他越是不行,越是没有办法,越是接近于精神上的全熵状态,他在房管员眼里,在票证科小姑娘、户籍员、办事员、试验员、科学家、朋友们的眼里,甚至在亲人们的眼里,甚至也许在契贝任的眼里,也许在妻子的眼里,越是不值钱,可是在玛利亚眼里却越是可贵,越是可亲。他们没有见面,他却知道,却感觉出这一点。他每遇到新的打击,新的凌辱,他都要在心里问她:“玛利亚,你看见我了吗?”

他就这样和妻子坐在一起,和她说着话儿,想的却是她不知道的心思。电话铃响起来。现在电话铃声只能引起他们的惊慌,就好比在夜里收到报告祸事的电报。

“哦,我知道,他们说过要给我打电话,谈谈做临时工的事。”柳德米拉说。

她拿起话筒,眉毛扬了起来,她说:

“他就来。”

“找你。”她对维克托说。

维克托用眼睛问:“是谁?”

柳德米拉用手捂住话筒,说:

“是一个不熟悉的声音,我想不起来啦。”

维克托接过话筒。

“请吧,我听着呢。”他说,一面看着柳德米拉问询的眼睛,在小桌上摸到铅笔,在一小片纸上写了几个歪歪斜斜的字母。柳德米拉没有注意他在做什么,慢慢画了一个十字,然后又给维克托画了一个十字。他们没有说话。

他仿佛听到:“……现在苏联各广播电台联播……”

这声音极像一九四一年七月三日向人民、军队和全世界说“同志们,兄弟们,朋友们……”的声音,现在这声音只对这握着电话筒的一个人说:

“您好,施特鲁姆同志。”

此时此刻,得意、软弱、害怕被什么流氓捉弄的心情、写好的检讨书、履历表、卢比扬卡广场的楼房……这一切一切念头,念头的片断、感情的片断全都混合到一起,搅成了一团。

出现了一种极其明朗的命运已定的感觉,同时又夹杂着一种失去分外可亲、分外动人的极好的东西的悲伤心情。

“您好,斯大林同志。”维克托说。

他感到吃惊,不大相信这是他在电话里说这种不可思议的话。

“您好,斯大林同志。”

总共在电话里谈了两三分钟。

“我认为,您的研究方向是很有意义的。”斯大林说。

他的声音很缓慢,带有喉音,带有用声音强调的表现力,似乎是有意这样,这声音非常像维克托在收音机里听到的那种声音。维克托有时候为了好玩儿,在自己家里模仿这种声音。在代表大会上听过斯大林的讲话或者被召见过的人也常常这样模仿他的声音。

难道是有人作弄他?

“我对自己的研究是有信心的。”维克托说。

斯大林沉默了一会儿,大概是在考虑维克托的话。

“在这战争时期,您是不是感觉缺乏外文资料,仪器设备是否齐全?”斯大林问道。

维克托用自己也意想不到的真挚口吻说:

“非常感谢,斯大林同志,研究工作条件完全正常,很好。”

柳德米拉在旁边站着,好像斯大林能看见她,她在听说话。

维克托朝她摆了摆手,意思是:“坐下,怎么不害臊……”可是斯大林又沉默了,在考虑维克托的话,后来说:

“再见,施特鲁姆同志,祝您研究顺利。”

“再见,斯大林同志。”

维克托放下话筒。他们面对面坐着,还像几分钟之前说起柳德米拉在市场上卖掉两块桌布时那样。

“祝您研究顺利。”维克托忽然用很重的格鲁吉亚口音说。

屋里的餐柜、钢琴、椅子依然没有变化,两只没有洗的碟子依然像刚才谈房管员时那样,摆在桌子上。这样没有变化,真不可思议,使人无法理解。因为一切都变了,一切都翻了个儿,他们的命运完全不同了。

“他对你说的是什么?”

“没什么特别的,他是问,是不是因为缺乏外文资料影响我的研究。”

维克托尽量装出平静和无动于衷的神气说。

他因为自己一时竟有这样强烈的幸福感,觉得很难为情。

“柳德米拉,柳德米拉,”他说,“你想想看,我没有检讨,没有低头,也没有给他写过信。他是自己,自己打电话的!”

真是不可思议!这件事的威力无比巨大。难道是他曾经日夜焦灼不安,睡不着觉,填履历表时发呆发愣,抓住自己的头发,思索在学术会议上对他的批判,回想自己的过错,在心里检讨、求饶,等待逮捕,想着自己的穷困,提心吊胆地想着如何跟身份证管理员和票证科的小姑娘打交道?

“我的天啊,天啊,”柳德米拉说,“托里亚再也不会知道这种事儿了。”

她走到托里亚的房间门口,把门开了。

维克托拿起话筒,又把话筒放下。

“万一是有人开玩笑呢?”他说着,走到窗前。

从窗子里可以看到空荡荡的大街,有一个穿棉袄的女人走过去。

他又走到电话机跟前,弯起手指头在话筒上敲了敲。

“刚才我的声音怎么样?”他问。

“你说得很慢。你要知道,我自己也不明白,为什么我一下子就站了起来。”

“是斯大林嘛!”

“也许,真是开玩笑呢?”

“瞧你说的,谁敢开玩笑?开这种玩笑起码要判十年徒刑。”

不过一个钟头之前,他还在房间里踱来踱去,哼唱戈列尼谢夫–库图佐夫的情歌“他孤单单,无人陪伴”呢。

斯大林打的电话呀!在莫斯科一年当中也只有一次或两次传说着:斯大林给电影导演多夫任科打电话了,斯大林给作家爱伦堡打电话了。

不需要他下命令:给某人奖金,给某人住房,为某人造研究所。他太伟大了,用不着说这些小事。这一切自会有他底下的人操办。他们可以从他的眼神,从他的声调中猜测他的心意。他只要亲切地对一个人笑一笑,这个人的命运就变了—这个人就会从黑暗中、从默默无闻的状态中一下子来到荣华富贵的倾盆大雨之下。就会有许多有权有势的人向这个幸运儿顶礼膜拜,就因为斯大林对他笑过,或者在电话里对他说过笑话。

人们会到处传说这些交谈的详情细节,斯大林说的每一句话都使人们吃惊。话越是平常,就越是使人吃惊。似乎斯大林不可能说家常话。很多人在传说,他有一次打电话给一位有名的雕塑家,开玩笑说:

“你好,老酒鬼。”

还有一次他向另一个名人,一个老好人问到被捕的朋友,那个名人慌了,回答得含糊不清,斯大林说:

“您没有把自己的朋友保护好。”

还在传说,他有一次往一家青年报的编辑部打电话,副主编接电话,说:

“我是布别金。”

斯大林问:

“布别金是什么人?”

布别金回答说:

“要查一查。”他说着,就把话筒扔下。

斯大林又叫接通了电话,说:

“布别金同志,我是斯大林,请您说说,您是什么人?”

据说,布别金在这之后,在医院里住了两个星期,害的是神经震荡。

他一句话可以使成千上万的人头落地。元帅、人民委员、党中央委员、州委书记—这些人昨天还指挥着千军万马驰骋战场,还领导着边区、自治州、巨大的工厂,今天由于斯大林一句发怒的话就会变得不值一文,变成劳改营的尘土,就会手拿饭盒,在劳改营的厨房外等候领取一勺稀稀的菜汤。

还在传说,有一天夜里,斯大林和贝利亚去看不久前从卢比扬卡监狱放出来的一位格鲁吉亚的老布尔什维克,在他那儿一直坐到天亮。住在这座院子里的人夜里不敢出来上厕所,早晨也不去上班。据说,给来客开门的是担任居民小组长的一名产科女医生,她穿着睡衣出来,手上还抱着小哈巴狗,她很生气:夜已经很深了,还有人来按门铃。后来她说:“我把门开了,看见一张相片,相片活动起来,冲着我来了。”据说,斯大林来到走廊里,对着电话机旁边贴的一张纸看了很久,那是居民们画道道儿记录打电话次数的,为的是按次数付款。

这些事情使人感到惊异和好笑,正因为一些话和一些情形很平常,至于斯大林竟会在几家合住的房子的走廊里走,更是不可思议的!

要知道,凭他一句话就可以出现大规模的建筑,一队队的伐木工人就会开进原始森林,成千上万的人群就去开凿运河,建造城市,在极夜地区和永久冻土地带开辟道路。他本身就代表着伟大的国家。阳光是斯大林宪法的阳光。斯大林的党……斯大林的五年计划……斯大林的建设……斯大林的战略……斯大林的空军……伟大的国家就表现在他的性格、他的气派中。

维克托一遍又一遍地重说着:

“祝您研究顺利……您的研究方向很有意义……”

现在很清楚:斯大林知道,国外已经开始关注深入研究核反应的物理学。

维克托早就察觉,围绕着核反应的一些问题出现了一种奇怪的紧张氛围,他在英美一些物理学家的文章的字里行间,在一些不大合乎思维逻辑的半吞半吐的话里,感觉出这种紧张氛围。他发现,有些经常在物理学杂志上发表论文的研究者的名字现在不见了,有些研究重核分裂的人好像失踪了,也没有人引用他们的著作。他觉得,问题范围一接近铀原子核的衰变问题,就格外紧张,不再说了。

契贝任、索科洛夫、马尔科夫不止一次谈起这方面的问题。不久之前契贝任还说到一些人眼光短浅,看不到和中子作用于重核的实用远景。契贝任本人倒是不想在这一领域进行研究……

在充满士兵的皮靴声、炮火与硝烟、坦克履带声的空气中,出现了新的、无声的紧张氛围,所以这个世界上最有力的手拿起电话筒,这位理论物理学家便听到了他那缓慢的声音:“祝您研究顺利。”

于是一道新的淡淡的阴影,无声无息、隐隐约约地落到燃遍战火的大地上,落到白发苍苍的老人和孩子们的头上。人们还没有感觉到、还不知道这一道阴影,还没有觉察出注定要出现的力量已经诞生。

从几十位物理学家的书桌,从写满希腊字母的一张张纸,从书橱和实验室,到将来成为震撼世界的强大力量,成为国力强大的标志,还有很长的一段道路。

道路已经开头,无声的阴影也越来越浓,渐渐变成黑暗,准备把偌大的莫斯科和纽约笼罩住。

维克托本来以为他的研究成果已经永远锁进他家里的书桌的抽屉了,可是现在有了出头之日。他的研究成果即将离开监狱,进入实验室,成为教授们讲课和作报告的话题。他没有想到科学真会取得可喜的胜利,自己会取得胜利,现在他又可以推动科学,可以培养学生,可以在杂志和书本上存在了,又可以操心他的想法是否和计算、摄影实际结果相符了。可是在这一天,他却不是为这一切感到高兴。

使他兴奋的是另一种原因,那就是他的虚荣心对迫害他的人取得了胜利。不久前他似乎还不恼恨他们。就是现在他也不想报复他们,让他们倒霉,但是他一想起他们干的一切坏事、欺人的事、残忍的事、怯懦的事,心灵和理智上就感到幸运。他们对待他越是粗暴,越是卑鄙,他现在想起来越是感到痛快。

娜佳放学回来,柳德米拉喊道:

“娜佳,斯大林给你爸爸打电话了!”

维克托看到女儿穿着脱掉一半的大衣、拖着围巾跑进屋里的那种激动的样子,就更明显地想象到有些人在今天或明天听说这件事时那种惊慌的神情。

他们坐下来吃午饭。维克托突然把羹匙放下,说:

“我简直一点儿也不想吃。”

柳德米拉说:

“恨你的人、害你的人这一下子完啦。我可以想象出来,在研究所里,甚至在整个科学院,将会出现什么样的情形。”

“是啊,是啊,是啊。”维克托说。

“妈妈,在限额商店里,那些太太们又要跟你打招呼,又要对你笑了。”娜佳说。

“是啊,是啊。”柳德米拉说着,笑了笑。维克托一向瞧不起阿谀奉承的人,可是现在一想到希沙科夫会做出一副奉承的笑容,就非常高兴。

很奇怪,不可理解!他感到高兴和胜利的同时,总有一股惆怅从心的深处往外冒,总有一种怜惜,怜惜此时此刻似乎正在离他而去的一种最珍贵的东西。似乎他有错,对不起什么人,但是究竟有什么过错,对不起谁,他却不清楚。

他喝着他很喜欢的土豆荞麦粥,想起了小时候在基辅,春天的夜里出来,星星在开花的栗树枝间闪着泪眼的情景。那时候他觉得世界是美好的,前途是广阔的,充满美妙的光和善意。今天,在他的命运已经决定的时候,他似乎在和自己对于美好的科学的爱告别—纯洁的爱、孩子般的爱、几乎是宗教式的爱,在和几个星期之前的那种心情告别—克制住巨大的恐惧,没有自我欺骗时体验到的感情。

他只能对一个人说说这些,但是那人现在不在他身边。

还有奇怪的。他有一种很急切的心情,希望所有的人快点儿都知道发生的事情。希望研究所、大学课堂、党中央委员会、科学院院部、房管所、别墅区管理处、各大学教研室、各个科学协会都知道这件事。可是,索科洛夫是不是知道,维克托觉得无所谓。不是在理智上,而是在心深处暗暗不希望玛利亚知道这个消息。他猜想,当他被排挤、倒霉的时候,她更爱他,他觉得是这样。

他对女儿和妻子说起战前她们就知道的一件事:斯大林一天夜里来到地铁车站,他微微有些酒意,挨着一个年轻女子坐下来,问她:“我能帮您什么忙吗?”那女子说:“我想去看看克里姆林宫。”斯大林在回答之前,想了想,说:“这一点也许我能办得到。”

娜佳说:

“你瞧,爸爸,你今天真了不起,妈妈居然让你把这个故事说完,没有打断你。要知道,这故事她已经听过一百一十次了。”

于是他们又一次,也就是第一百一十一次讥笑起那个天真的女子。

柳德米拉问:

“维克托,遇到这种情形,是不是应该喝点儿酒?”

她拿来一盒水果糖,原是为娜佳过生日准备的。

“吃吧,”柳德米拉说,“不过,娜佳,不要一吃起来就和狼一样。”

“爸爸,吃吧,”娜佳说,“咱们为什么要笑地铁里那个女人?你怎么不向斯大林问问米佳舅舅和克雷莫夫的事?”

“瞧你说的,这怎么可能呢?”他说。

“依我看,可能。要是外婆,马上就会说的,我相信她会说。”

“可能,”维克托说,“可能。”

“哎,别瞎扯了。”柳德米拉说。

“怎么瞎扯?这是问舅舅的事。”娜佳说。

“维克托,”柳德米拉说,“应该给希沙科夫打个电话。”

“你显然对这件事的意义估计不足。用不着给任何人打电话。”

“你还是给希沙科夫打个电话吧。”柳德米拉执拗地说。

“等斯大林对你说‘祝你成功’,你给希沙科夫打电话好啦。”

这一天维克托产生了一种很奇怪的新的感觉。大家把斯大林神化,他过去一直感到很气愤。报纸从第一版到最后一版到处都是他的名字。又是肖像,又是半身雕像,又是全身塑像,又是歌剧,又是长诗,又是颂歌……

他被称作父亲、天才……

使维克托气愤的,是他的名字遮没了列宁的名字,竟把他的军事才能说得比列宁的治国才能还高。在阿列克赛·托尔斯泰的一个剧本里,列宁很勤快地划着了火柴,让斯大林点着烟斗抽烟。在一位画家笔下,斯大林昂首阔步地走在斯莫尔尼宫的台阶上,列宁急急匆匆、毕恭毕敬地跟在他后面。如果在画着列宁和斯大林跟人民在一起,那么,只有一些老头子、老妇人和小孩子亲切地看着列宁,而倾注着斯大林的却是一些武装巨人—腰缠机枪子弹带的工人、水兵。历史学家写到苏维埃国家的危难时期,不论是喀琅施塔得叛乱时期,保卫察里津时期,还是波兰入侵时期,都要歪曲事实,说列宁经常向斯大林请教。党的历史学家们给予斯大林参加过的巴库罢工和他曾经主编过的《斗争报》的地位,超过了俄国的全部革命运动。

“《斗争报》,《斗争报》,”维克托常常很生气地说,“当年有热里雅鲍夫,有普列汉诺夫,有克鲁泡特金,有十二月党人,可是现在只剩了《斗争报》,《斗争报》……”

千余年来俄罗斯一直是君主专制和专制独裁国家,是沙皇和宠臣们的国家。但是在千余年的俄罗斯历史中谁也不曾有过斯大林这样大的权力。可是今天维克托不气愤,不害怕了。斯大林的权力越大,颂歌和定音鼓越响,这尊活神像脚下的神香烟云越浓,维克托的幸福感越强烈。

天色渐渐黑下来,可是他不害怕了。

斯大林和他说话了呀!是斯大林对他说:“祝您研究顺利。”

等到天色完全黑下来,他来到大街上。

在这黑沉沉的晚上,他不再感到绝望和大祸临头了。他心里是宁静的。他知道,在签发逮捕证的地方已经知道了一切。他想到克雷莫夫、米佳、阿巴尔丘克、马季亚罗夫,想到切特韦里科夫,就感到奇怪。他们的命运没有成为他的命运。他怀着感伤和不可理解的心情想着他们。

维克托为他的胜利高兴,那是他的精神力量、他的头脑取得的胜利。他也不管,为什么今天的幸福和被批判那天似乎感觉到母亲跟他在一起时那种幸福有所不同。现在马季亚罗夫是不是会被捕,克雷莫夫是不是会供出他来,对他都无所谓了。他生平第一次不为自己说的一些离经叛道的笑话和不小心的话担惊受怕。

到很晚的时候,柳德米拉已经睡了,电话铃响了起来。

“您好。”一个很轻的声音说。维克托一听就激动起来,似乎更超过白天的激动。

“您好。”他说。

“我不能听不到您的声音。您对我说点儿什么吧。”她说。

“玛莎,玛申卡。”他说过这话,就不作声了。

“维克托,我亲爱的,”她说,“我不能对我丈夫撒谎。我对他说了,我爱您。我向他发誓永远不再见您。”

早晨,柳德米拉走进他的房里,抚摩了抚摩他的头发,吻了吻他的额头。

“我在梦里仿佛听到,昨天夜里你跟什么人通电话。”

“没有,你是做梦了。”他镇静地看着她的眼睛,回答说。

“记住,今天你要上房管所去一趟。”

四十三

看惯了军装的人,一看到侦讯员的西装上衣,觉得很奇怪。侦讯员的脸倒是一张很平常的脸,像这种黄白色的脸,在办公室里的少校和政工人员中是很常见的。

回答开头几个问题很容易,甚至轻松愉快,似乎其他一切也会十分清楚,就像姓、名和父称一样简单明了。

从犯人的回答似乎可以感觉出一种迫切地想帮助侦讯员的心情。侦讯员好像对他一点也不了解嘛。他们之间的办公桌并没有把他们分开。他们都交过党费,看过《恰巴耶夫》,听过党中央的指示,在五一节前都被派到工厂企业去做过报告。

例行公事的问题很多,犯人渐渐镇静下来。很快就会问起实质性问题的,他就要说说他是怎样带着人突围的。

终于弄清了,坐在桌前这个敞着军服上衣领口、被剪掉了纽扣、胡子拉碴的人有名字、父称、姓,出生于秋天,俄罗斯族,参加过两次世界大战和一次国内战争,没有参加过匪帮,没有犯罪前科,参加联共(布)二十五年,曾被选为共产国际代表大会代表,还当过世界工会太平洋地区会议的代表,没有得过勋章和荣誉武器……

想到当年被包围,想到跟他一起转战在白俄罗斯沼地上和乌克兰土地上的许多人,克雷莫夫感到心慌意乱。

他们之中是谁被捕了呢,是谁在审讯中经受不住,丧失了良心?可是一个突如其来的涉及另一段很早时期的问题使克雷莫夫大吃一惊:“您说说,您什么时候和弗里茨·加肯认识的?”他沉默了半天,然后说:

“如果我没有记错的话,那是在全苏工会中央理事会,在托姆斯基的办公室里,如果我没记错的话,那是在一九二七年春天。”

侦讯员点了点头,好像他很清楚早年这些情况。

然后他深深吸了一口气,打开标有“档案”字样的公文夹,不慌不忙地把白色小丝带解了开来,翻起一页页写满了字的纸。克雷莫夫模模糊糊看到用各种颜色的墨水写的字,看到打字机打的字,行距有稀的,有密的,还有用红铅笔、蓝铅笔和普通铅笔写的标注,有的笔道很粗,有的是仔细贴上去的。

侦讯员慢慢翻着材料,就像一个好学生满有把握地翻着书本,早就知道他已经把课程学透了。

他偶尔看看克雷莫夫。这时候他像一位画家,看看他的画是否与模特儿相像:外貌,性格,心灵的窗户—眼睛……

他的目光变得多么阴沉。他那很平常的脸—这样的脸一九三七年以后克雷莫夫在区党委、州党委、区公安局、图书馆和出版社常常见到—忽然变得很不平常了。克雷莫夫觉得,他整个的人是由一些拼图方块组成的,但这些拼图方块没有合成一个整体,没有成为一个人。一块方块是眼,另一块是慢腾腾的手,还有一块是问问题的嘴巴。方块乱了位置,失去比例,嘴巴大得出了格,眼睛移到嘴巴底下,长到蹙紧的额头上,额头则移到应该长下巴的地方。

“嗯,嗯,是这样。”侦讯员说。他脸上的一切又像人的样子了。他把公文夹合上,公文夹上的小带子他没有系上。

“就像没有系上的鞋带儿。”裤子和衬裤上的扣子都被剪掉了的被捕者心中想道。

“共产国际。”侦讯员一字一字、郑重其事地说。接着用平常的语调说:“尼古拉·克雷莫夫,共产国际工作人员。”随后又一字一字、郑重其事地说:“第三共产国际。”

他一声不响地沉思了很久。

“啊呀,好厉害的小娘们儿穆丝卡·格林贝格。”侦讯员忽然带着很起劲又狡黠的神气说,就像男子之间说玩笑话儿。克雷莫夫感到很难为情,不知如何是好。脸一下子红了。

有过这事儿!已经很久了,可是一想起来就难为情。那时候他好像已经爱上叶尼娅了。好像那是他下了班去找老朋友,想把钱还他,好像是借了钱买车票的。底下的事他就记得很清楚,不是“好像”了。老朋友康斯坦丁不在家。他本来也不喜欢她。她不住地抽烟,抽得嗓子都哑了,谈起什么,都自以为有两下子,她是哲学研究所的党委副书记,不错,她很美,如大家说的,是一个标致娘们儿。唉,所以他就和康斯坦丁的老婆在沙发上干了那种事,而且后来又和她会过两次……

在一个钟头之前,他还以为,这是从乡下区里提拔上来的一名侦讯员,对他一点儿也不了解。可是过了一阵子,侦讯员却一个劲儿地问起和克雷莫夫一起工作过的外国共产党员,他知道他们的小名和外号,知道他们的妻子和情妇的名字。他的档案材料这样丰富,不是一种好兆头。就算克雷莫夫是一位伟人,每一句话对于历史都有举足轻重的意义,也未必值得把这么多鸡毛蒜皮、乱七八糟的小事收进档案里。

可鸡毛蒜皮的小事是没有的。

不论他到过哪儿,都留下他的脚印,有人跟着他的脚跟走,记下他的生活。他取笑同志的话、读过一本书的感想、在庆贺生日时开玩笑的祝酒词、在电话里说的三分钟的话、开大会时给主席团递的不太客气的条子—这一切都收进了系小带子的公文夹。

他的言语、行动被搜集起来,晒干了,做成了大型标本。这是多么不怀好意的手指头如此勤劳地搜集野草、荨麻、飞廉、滨藜……

伟大的国家竟在研究他和穆丝卡·格林贝格的艳史。一些闲话和琐事与他的信仰编结在一起,他对叶尼娅的爱没有什么意义,有意义的倒是一些不足道的偶然的艳遇,他简直分不清大节和小节了。他说过的一句对斯大林的哲学常识不太客气的话,似乎比他十年日日夜夜为党工作更值得注意。一九三二年他在洛佐夫斯基的办公室里和一位德国同志谈话的时候说,在苏联的工会运动中国家的成分太多,无产阶级的成分太少,这是真的吗,是那位同志告密的。

“您要明白,侦讯员同志。”

“应该称呼公民。”

“是,是,公民。这是捏造,是有成见。我在党内有四分之一世纪。我在一九一七年发动过士兵起义。我在中国工作过四年。我日日夜夜为党工作。许多人都了解我……在卫国战争期间我志愿上前线,在最危难的时刻大家都相信我,跟着我走……我……”

侦讯员问道:

“您怎么,是来这儿领立功奖状的吗?要不要填表领嘉奖证书?”

确实,他不是来领立功奖状的。

侦讯员摇了摇头,说:

“您还怪妻子不给您送东西呢。瞧您这个丈夫!”

这话是他在牢房里对鲍戈列耶夫说的。我的天啊!卡茨涅林鲍肯用开玩笑的口气对他说:

“一位希腊人预言:一切都会过去;我们则可以断言;一切都会密告上去。”

他的一生进入系小带子的公文夹之后,便失去体积、长度、比例……一切一切都成为黏糊糊、乱糟糟的、灰灰的一团,连他自己也不知道什么更值得注意:是在潮湿、闷热的上海的四年超强度工作,斯大林格勒的抢渡,对革命的忠忱,还是因为在“松树”疗养院对一位不太熟悉的文学家说的批评苏联报纸内容贫乏的几句气话?侦讯员又和蔼、又亲切地小声问道:“现在请您对我说说,法西斯分子加肯是怎样吸收您参加谍报和破坏工作的。”

“您不是开玩笑吧?”

“克雷莫夫,别装蒜。您该看到,您走的每一步我们都是很清楚的。”

“正因为这样,所以……”

“克雷莫夫,您老实点儿。您骗不了保安机关。”

“不过,这是捏造!”

“是这样的,克雷莫夫。我们有加肯的供词。他在交代自己的罪行中,谈到他和您的罪恶关系。”

“您哪怕拿出十份加肯的供状,这都是假的!是捏造!如果你们有加肯这样的供状的话,为什么还相信我这个间谍和破坏者,让我做军事政委,带领人作战?你们干什么去了,你们是干什么的?”

“您怎么,是叫您到这儿来教训我们的吗?是请您来领导保安机关工作,是不是?”

“说什么领导,说什么教训!要摆事实,讲道理。我了解加肯。他不可能说他吸收我干什么。不可能!”

“为什么不可能?”

“他是共产党人,是革命战士。”

侦讯员问:

“您一直相信这一点吗?”

“是的,”克雷莫夫回答说,“我一直相信!”

侦讯员一面点头,一面翻档案材料,一面似乎无可奈何地说:

“既然一直相信,那就是另一回事儿了……就是另一回事儿了……”

“您就看看吧。”他用手掌捂住一张纸的一部分,说道。

克雷莫夫粗粗地看着上面写的字,耸了耸肩膀。

“太没出息了。”他很厌恶地说。

“为什么?”

“这人没有勇气挺直身子说,加肯是一名忠诚的共产党人,又不肯昧着良心诬陷他,所以就躲躲闪闪。”

侦讯员把手移了移,让克雷莫夫看了看签名和日期:克雷莫夫,一九三八年二月。

他们都沉默了一会儿。然后侦讯员厉声问道:

“也许,是他们打您,所以您写了这样的证明材料吧?”

“不是,没有打我。”

侦讯员的脸又分裂成好几块拼图方块,那气愤的眼睛流露着厌恶的神情,嘴巴在说:

“还有。您在被包围的时候,有两天离开了自己的队伍。敌人用军用飞机把您接到德军集团军群司令部,您交出了重要情报,又接受了新的指示。”

“痴人说梦。”被剪掉了衣服扣子的人嘟哝说。

可是侦讯员继续进行审问。现在克雷莫夫已经不觉得自己是具有崇高、明确的思想,随时准备为革命上断头台的强者了。

他感到自己是一个软弱、不坚定的人,他说过不该说的话,传播过荒唐的谣言,他竟敢嘲笑苏联人民对待斯大林同志的感情。他不善于识别朋友,在他的朋友当中有很多人被镇压了。他的理论见解十分混乱。他和朋友的妻子私通。他用可耻的两面派态度写了有关加肯的证明材料。

难道坐在这儿的是我吗?难道这一切都是我的事吗?这是一个梦,是夏夜的一个梦。

“在战前您为国外的托洛茨基中央组织提供过有关国际革命运动主要人物思想状况的情报。”

怀疑这样一个可鄙、肮脏的人叛变,不必是疯子,也不必是坏蛋。克雷莫夫如果在侦讯员的位子上,也不会相信这样一个人。这个人十分了解在一九三七年接替被镇压或被解职、降职的党内工作者的一批新的党干部。这是一些气质和他不同的人。他们读的书不同,读法也不同,他们不是读,而是“仔细研究”。他们看重舒适的物质生活,革命的牺牲精神与他们格格不入,或者说,不是他们性格的基础。他们不懂外语,喜欢自己的俄罗斯本性,说俄语也不按标准音。他们之中有聪明人,但是他们的主要长处和本领似乎不在于思想和理智,而在于办事能力和机警,善于见风使舵。

克雷莫夫明白,不管新干部还是老干部,都在党的一致与共同性中得到统一,分歧不要紧。但是他觉得自己比这批新人优越,觉得他这个列宁主义的布尔什维克比他们好。

他没有注意到,现在他和侦讯员的关系已经不在于他是否愿意和这位新干部亲近,承认这位新干部是党的同志。现在,和侦讯员认同的愿望变成了可怜的希望,希望对方和他亲近,哪怕同意他一生的所作所为不全是坏的、低下的、不忠诚的。

现在,连克雷莫夫也没有觉察到这样的事是怎么发生的:一个充满自信的侦讯员成了一名充满自信的共产党员。

“如果您真的能够诚心悔改的话,哪怕您还对党多少有一点爱护之心的话,那就该承认自己的罪行,帮助帮助党。”

克雷莫夫一下子打掉侵蚀着他的大脑皮层的软弱,叫了起来:

“您别想从我身上得到什么!我决不写假口供。您听见吗?就是用刑,我也不写!”

侦讯员对他说:

“您考虑考虑吧。”

他又翻起档案材料,没有看克雷莫夫。时间一点一点过去。他把克雷莫夫的档案材料推到一边,从桌子抽屉里拿出一张纸。他似乎忘记了克雷莫夫,不慌不忙地写着,皱起眉头思索着。后来他把写好的东西看了一遍,又想了想,从抽屉里拿出一个信封,就在上面写地址。也许,这不是一封公函。后来他又看了一遍地址,在姓氏下面画了两道着重线。后来他往自来水笔里灌了墨水,又把笔头上滴的墨水擦了半天。然后他削起烟灰缸上的铅笔,其中有一支铅笔的铅芯一削就断,但是侦讯员没有生铅笔的气,很耐心地削了又削。后来他在指头上试了试铅笔尖儿。

被捕者确实在考虑。要考虑的事情太多了。

哪儿来的这么多告密者!必须想一想,弄清楚是谁告的。这还用说?是穆丝卡·格林贝格……侦讯员还要问到叶尼娅的……确实很奇怪,为什么还没有问到她,一点也没有提到她……难道有关我的材料是瓦西亚提供的?但是我究竟有什么,有什么好承认的呢?我现在在这儿,不明白的还是不明白,党啊,你这一切为的是什么?斯大林呀,斯大林,因为什么样的罪过,打击这么多善良、刚强的人?可怕的不是侦讯员提出的问题,而是他的沉默、他避而不谈的东西。卡茨涅林鲍肯说的不错。当然,他会问起叶尼娅的,显然她已经被捕了。这一切是怎么来的,怎么开头的呢?我怎么会蹲起监牢?我这一生多么苦恼,有多少窝囊事儿。斯大林同志,饶恕我吧!只要有您一句话就行,斯大林同志!我有错误,我糊涂,我乱说过,我怀疑过,党全知道,全看见了。我为什么,为什么要和那个文学家闲扯呀?不过,还不是一样。可是,突围又有什么问题?这简直荒唐,简直是诬陷,捏造,诽谤。为什么,为什么我当时没有说加肯是我的朋友,我的好兄弟,我不怀疑他是纯洁的。这样加肯那不幸的眼睛就会从他身上移开了……

侦讯员忽然问道:

“喂,怎么样,回想起来了吗?”

克雷莫夫把两手一摊,说:

“我没有什么好回想的。”

电话铃响起来。

“喂,我听着呢。”侦讯员说。他瞟了克雷莫夫一眼,说:“是的,你准备吧,快要到时候了。”

克雷莫夫觉得似乎说的是他。后来侦讯员放下话筒,又拿起话筒。这次的电话很奇怪,好像旁边坐的不是一个人,而是一只两条腿的兽。看样子,侦讯员是在和他老婆聊天。开头谈的是生活上的问题:

“上配给商店去过吗?鹅吗,这很好……为什么凭一号券不卖?谢廖沙的老婆往科里打过电话,说凭一号券买了一条羊腿,请咱们去吃呢。告诉你,我在小卖部买了一些奶渣,不,不是酸的,有八百克……今天煤气怎么样?你不要把西装忘了。”

后来他又说起来:

“喂,怎么样?别太烦恼,要多加注意。做梦啦?穿什么?还穿短裤?可惜……喂,小心点儿,等我回去,你已经要上学校去了……收拾房间吗,很好,不过要小心,不要拿重东西,你无论如何不能拿重东西。”

在这儿这样随便地叙家常,有点儿不可思议:越是像日常的、平常人的谈话,谈话的就越不像人。猴子模仿人的行动,样子就有点儿可怕……同时克雷莫夫感到自己也不是人,因为当着一个外人的面,是不会说这一类的话的……

“我吻你……你不愿意……好,算啦,算啦……”

当然,如果按照鲍戈列耶夫的理论,克雷莫夫只是安卡拉猫,是青蛙、金翅雀,或者树枝上的一只小虫儿,这样就一点没有什么奇怪的了。

到末了侦讯员问:

“要烤糊了吧?好,快去,快去,再见。”

然后他拿出一本书和一个笔记本,看起书来,还不时地做笔记,也许他是准备小组讨论,也许是准备作报告……

他带着很大的火气说:

“您怎么一个劲儿地跺脚,就好像在做体操?”

“公民,我的两脚发麻。”

但是侦讯员又埋头看起书来。

过了十来分钟,他心不在焉地问:

“喂,怎么样,回想起来了吗?”

“公民,我要上厕所。”

侦讯员叹了一口气,走到门口,轻轻唤了一声。当一只狗在不适宜的时候要求出去游逛的时候,狗主人的脸色往往就是这样。进来一名穿野战军服的士兵。克雷莫夫用老练的目光把他打量了一眼:腰里扎着皮带,白衬领干干净净,军帽戴得端端正正—一切都很像样。只是这名士兵干的不是士兵该干的事情。

克雷莫夫站起来,因为在椅子上坐的时间太久,两条腿都麻木了,一开始迈步直打战。在厕所里,他在士兵的注视下急急忙忙地想着,回来的路上也急急忙忙地想着。有很多事情要想。

等克雷莫夫从厕所里回来,侦讯员不见了,在他的位子上坐的是一个穿军服的年轻人,佩戴着镶了红绦的蓝色大尉肩章。大尉用阴沉的目光看了看被捕者,就好像有不共戴天的仇恨。

“干吗站着?”大尉说。“喂,坐下!把身子坐直,老家伙,干吗弓着背?等我给你两下子,你身子就直起来了。”

“一见面就这样。”克雷莫夫心里想道。他害怕起来,在战场上他都没有这样害怕。

“这一下子要来劲儿了。”他想。

大尉吐了一个烟团儿,在灰色的烟团中响着他的声音:

“这是纸、笔。怎么,要我替你写吗?”

大尉很喜欢侮辱克雷莫夫。也许,这是他的职责?要知道,在前方有时要炮兵对敌军进行扰乱性射击,炮兵就日日夜夜打炮。

“你是怎么坐的?你是上这儿睡觉的吗?”

过了几分钟,他又呼唤被捕人:

“喂,你听着,怎么,我不是对你说话吗,跟你无关吗?”

他走到窗前,拉起厚厚的窗帘,把电灯熄了,一道阴沉的晨曦射进克雷莫夫的眼睛。克雷莫夫自从来到卢比扬卡,这是第一次看见白天的光。

“一夜过去了。”克雷莫夫想道。他一生是否有过更坏的早晨?难道在几个星期之前是他无思无虑地躺在炸弹坑里,对他厚待的钢铁在头顶上呼啸着,他感到那样幸福和自由?

可是时间错乱了:他进入这个房间是很久以前,斯大林格勒却是刚刚过去的事。

窗子面对着内部监狱的天井,窗外光线灰沉,毫无生气,不像亮光,倒像脏水。一切东西在这晨光下似乎比在电灯光下更阴沉,更带有官气和敌意。

不,不是靴子变小,是两脚麻木了。

在这儿怎么把他过去的生活和工作与一九四一年被包围联系起来?是谁的手指头把不能连接的东西连接到了一起?这是为了什么?谁要这样?为什么?

他想到这些,心里十分难过,以至于有时他忘记了脊背和腰的酸痛,感觉不到他肿胀的两腿已把靴筒塞满了。

加肯、弗里茨……我怎么忘了,一九三八年我也是坐在这样一个房间里,也是这样坐着,不过,不是这样:那时候口袋里有通行证。现在倒是想起了那最卑鄙的心思:一心想讨好所有的人,不论是开发通行证的办事人员,值班守卫,还是穿军服的电梯工。那一位侦讯员说:“克雷莫夫同志,请您帮帮我们的忙吧。”不,最卑鄙的还不是一心想讨好。最卑鄙的是一心想表示忠诚!啊,这一下他倒是回想起来了!在这方面只要忠诚就行了!于是他表示了忠诚,他说出加肯在评价斯巴达克运动方面的错误,说他对台尔曼没有好感,说他想要稿费,说他在艾丽萨怀孕的时候和她离了婚……当然,他也想起了好的……侦讯员记下了他的话:“我和他多年相交,认为他不大可能直接参与反党的破坏活动,不过不能完全排除他有进行两面派活动的可能性……”

啊,是他报告的……在这儿的档案夹里所搜集到的有关他的一切,都是也想表示忠诚的同志们说的。为什么他要表示忠诚?是党员的义务吗?胡说!真正的忠诚只能这样:拿拳头在桌子上狠狠一擂,高声说:“加肯是我的朋友和兄弟,他没有罪!”可是他却搜索枯肠,拼命找毛病,拼命迎合那个侦讯员,因为没有侦讯员的签名,他有通行证也出不了灰色大楼的大门。他还回想起来,当侦讯员说“请等一下,克雷莫夫同志,我在您的通行证上签个字”的时候,他感到多么急切、多么幸福。他帮助他们把加肯打进了监狱。他这个忠诚的人带着签了字的通行证上哪儿去了呢?不是去找朋友的妻子穆丝卡·格林贝格了吗?不过他说的有关加肯的一切,都是事实。但那里面说的有关他的一切,也都是事实呀。他确实对菲佳·叶甫谢耶夫说过,斯大林各方面的缺陷都和哲学上的无知有关系。要说出他遇到过的人,实在可怕:尼古拉·伊凡诺维奇、格里高力·叶甫谢耶维奇、洛莫夫、沙茨金、比亚特尼茨基、洛米纳泽、留京、红头发的什里亚普尼科夫,他还到列夫·鲍里索维奇的“科学院”去过,还有拉舍维奇、扬·加马尔尼克、卢波尔,他还去研究所找过里亚萨诺夫老头子,在西伯利亚有两次住在老朋友艾海家里,还有基辅的斯克雷普尼克、哈尔科夫的斯坦尼斯拉夫·科西奥尔,噢,还有卢特·菲舍尔,哦……幸亏侦讯员没有想起主要的一个,要知道当初列夫·达维多维奇和他的关系是不坏的……

我算是烂透了,还有什么说的。不过,为什么?他们的罪过不比我的大呀!不过我可是没有签字。别急,克雷莫夫啊,克雷莫夫,你会签字的。他们都签字了,你怎么能不签字!大概,最卑鄙的手段留在最后。就这样三天三夜不让人睡觉,然后就开始殴打。是的,反正这一切不大像社会主义。我的党有什么必要把我消灭?要知道,当年搞革命的是我们,而不是马林科夫,不是日丹诺夫,不是谢尔巴科夫。我们对革命的敌人都是毫不留情的。为什么革命对我们毫不留情?也许,革命就是毫不留情。也许,这不是革命,这个大尉算什么革命,这是黑帮,是一伙流氓。

他呆呆地坐在椅子上,时间一点一点过去。

背也疼,腿也疼,疲惫无力,身子想挺直也挺不起来。顶好能躺到床上,动一动光光的脚趾头,跷一跷腿,挠挠小腿肚子。

“别睡觉!”大尉喝道。就像在发布战斗命令。

好像只要克雷莫夫闭一会儿眼睛,苏维埃国家就会垮了,前线就会崩溃。克雷莫夫一辈子也没有听到过这么多骂人的脏话。

朋友们、亲近的助手、秘书、推心置腹的交谈者都在搜集他的一举一动。他越想越害怕:“这是我对伊凡说的,只是对伊凡说过。”“我跟格里沙谈过,我和格里沙从一九二〇年就相识。”“这话我和玛什卡·海尔别尔说过,哎呀,玛什卡呀,玛什卡。”

他忽然想起侦讯员说的,他别想等叶尼娅送东西……这是他不久前在囚室里和鲍戈列耶夫说的。直到现在还有人在填充克雷莫夫标本呢。

下午,给他端来一钵子汤。他的手抖得厉害,只好弯下头去,就着钵子的边儿喝汤,汤匙像敲鼓一样碰得叮当响。

“你喝起来像头猪。”大尉阴沉地说。

后来又是一件大事:克雷莫夫要上厕所。他走在走廊里的时候,已经什么也不想了,可是,他站在便池前的时候又想了,想的是:幸亏把扣子剪掉了,要不然,手这样发抖,裤裆还解不开,也扣不上呢。

时间又是一点一点地过去。戴着大尉肩章的国家胜利了。他的头脑里出现一团浓浓的灰雾。大概,猴子的头脑里就有这样的雾。不再有过去和未来,不再有系着小带子的档案夹。只有一个愿望:把靴子脱下来,挠挠痒,睡一觉。

那个侦讯员又来了。

“您睡好了吗?”大尉向道。

“领导不是睡觉,是休息。”侦讯员故意用教导的口吻说。他说的是很久以前军队里的一句俏皮话。

“是的,”大尉说,“不过部下眼皮有些肿了。”

就像一个工人来接班,总要看看自己的车床,认真地和上一班工人交换一下意见,侦讯员就是这样看了看克雷莫夫,看了看办公桌,说:

“好啦,大尉同志。”

他看了看表,从抽屉里拿出档案夹,解开小带子,翻了翻档案材料,很有兴致、很带劲儿地说:

“好吧,克雷莫夫,咱们继续进行。”

于是他们又进行下去。

侦讯员今天问的是战争。他在这方面也知道很多很多:他知道克雷莫夫担负的任务,知道一些团和集团军的番号,能说出和克雷莫夫一起作战的一些人的名字,知道克雷莫夫在政治部说过的一些话,知道他对将军写的文理不通的便条所提的意见。

克雷莫夫在前方所做的工作、在德军炮火下做的一些报告、在撤退和艰难困苦的日子里对士兵们的鼓舞—所有这一切一下子全不存在了。

他成了胡说八道的可怜虫,成了两面派,瓦解同志们的斗志,把不信任和失望情绪传染给他们。是德国侦察队帮他越过前线以便继续进行间谍和破坏活动,还有什么可怀疑的吗?

在重新开始审问的头几分钟里,睡足了觉的侦讯员那股精神劲头儿也传给了克雷莫夫。

“随您怎样,”他说,“我永远不会承认自己是间谍!”

侦讯员朝窗外看了看:天已经开始黑了,他看不清桌上的材料了。

他开了台灯,把蓝色的窗帘放下来。

凄厉的、野兽般的叫声从门外传来,并且忽然断了,没有声音了。

“好吧,克雷莫夫。”侦讯员说着,又在桌旁坐下来。他问克雷莫夫,是否明白,为什么从来没有提升过他的军衔。他听到的是不太明确的回答。

“所以嘛,克雷莫夫,您在前方一直是一名营级政委,可是您应该是一位集团军甚至方面军的军委委员呀。”

他盯着克雷莫夫,沉默了一会儿,也许,第一次用一个侦讯员的目光看了看,得意地说:

“托洛茨基亲口说过您的文章‘十分精彩’。如果这个坏蛋夺取了政权,您会升上很高的位子,‘十分精彩’—是开玩笑的吗!”

“这就是王牌了,”克雷莫夫心想,“他把王牌打出来了。”

他以为,克雷莫夫会把一切都说出来了,什么时候,在什么地方,不过,这样的问题也可以拿来问问斯大林同志。克雷莫夫同志和托洛茨基主义没有任何关系,他一直反对托洛茨基的意见,一次也没有赞成过。

最要紧的是脱脱靴子,躺下去,跷一跷肿胀的腿,睡一会儿,同时在睡梦中挠挠痒。可是侦讯员很亲切地小声说起来:

“为什么您不愿意帮我们的忙呀?难道问题在于,您在战前没有什么罪行,在被包围时没有恢复关系,没有秘密进行联系?……问题要严重得多,深刻得多。问题在于党的新的方针。您要在新的斗争阶段帮助党。为此必须抛弃过去的一些见解。这样的任务只有布尔什维克能够担当。所以我要和您谈谈。”

“那就好吧,好吧,”克雷莫夫慢慢地、昏昏沉沉地说,“可以设想,我不自觉地成了敌视党的观点的代表。就算我的国际主义和独立自主的社会主义国家观念相矛盾。就算我因为本性,在一九三七年以后和新的方针、新的人物格格不入。我愿意承认,可以承认。不过,至于间谍,破坏……”

“还要这‘不过’干什么?您瞧,您已经走上正路,承认自己敌视党的事业。难道形式有什么意义?如果您承认了最根本的,还要您这个‘不过’干什么?”

“不,我不承认我是间谍。”

“就是说,您根本不想帮助党。一谈到问题,您就溜进树林子里,是这样吗?您是狗屎,真不识抬举!”

克雷莫夫一下子跳起来,扯了一下侦讯员的领带,然后用拳头在桌上一擂,电话机里有什么东西叮当响了一声,又咕咕了两声。他用响亮的嗥叫声叫了起来:

“你这狗崽子,坏蛋,当我领着人在乌克兰,在布良斯克森林作战的时候,你在哪儿呀?冬天我在沃罗涅日作战的时候,你又在哪儿?你这坏蛋,到过斯大林格勒吗?难道我对党一点事情没有做过吗?你这副宪兵嘴脸,你就在这儿,在卢比扬卡保卫苏维埃国家吗?我在斯大林格勒不是保卫我们的事业吗?你在上海的白色恐怖下呆过吗?你这败类,高尔察克匪帮打穿了我的左肩,还是打穿了你的左肩?”

然后,他被打了一顿。但不是像在方面军特别科那样干脆利落地打在脸上,而是打得很讲究,很科学,很有生理学和解剖学的素养。打他的是两个穿着新军装的年轻人,他对他们喊着:

“你们这两个坏蛋,应该把你们送到惩戒连去,把你们编进反坦克枪小组……两个逃兵……”

他们自顾自打着,既不生气,又不发狂。似乎他们打得不够狠、不够猛,但是这种打法很有些可怕,就像很平静地说出的卑鄙话,往往格外可怕。

克雷莫夫的嘴里流出血来,虽然一次也没有打到他的牙齿,这血也不是从鼻子里,不是从牙花子,不是从咬破的舌头里流出来的不像在阿赫图巴那样……这是从肺部深处流出的血。他已经不记得他在哪儿,不记得他是在做什么……他上面又出现了侦讯员的脸。侦讯员指着挂在桌子上方的高尔基画像,问:

“伟大的无产阶级作家马克西姆·高尔基说什么来着?”

接着又像个教师似的用教导的口吻回答说:

“如果敌人不投降,就消灭他!”

然后他看到天花板上的电灯,看到一个佩戴窄小肩章的人。

“好吧,既然医生认为没事儿,”侦讯员说,“那就用不着休息了。”

一会儿,克雷莫夫又坐在桌前,听着明白易懂的教导:

“咱们就这样坐上一个星期,一个月,一年……咱们就来干脆的:就算您没有任何罪行,但我对您说什么,您就全写下来。这样就不会再打您了。明白吗?也许,特别会议会审判您,但是不会打您了—这是很重要的事。您以为,您挨打,我就舒服吗?我们可以让您睡觉。明白吗?”

―个小时一个小时过去,谈话还在进行着。似乎再没有什么能够使克雷莫夫震惊,使他脱离昏昏沉沉的迷糊状态。但是,他听着侦讯员的一番新的说法,还是惊愕得半张开嘴巴,抬起头来。

“所有这些事都是老早的事了,可能已经忘记,”侦讯员指着克雷莫夫的档案材料说,“可是您在斯大林格勒战役期间对祖国的可耻背叛行为,是不会被忘记的。有见证人,也有材料可以证实!您在被德军围困的‘6—1’号楼里进行活动,瓦解战士们的政治觉悟。您鼓动热爱祖国的格列科夫背叛祖国,企图动员他投向敌方,司令部和党派您到这座楼房里去担任作战政委,您辜负了司令部的信任,辜负了党的信任。您进入这座楼房之后,担当了什么角色?竟做了敌人的间谍!”

快到天亮时候,又把克雷莫夫打了一顿。他觉得自己仿佛沉进温暖的黑色牛奶中。又是那个佩戴窄小肩章的人擦着注射器的针头,点了点头。又听见侦讯员说:

“既然医生认为没关系,就没什么。”

他们面对面坐着。克雷莫夫看着对方的疲惫的脸,觉得奇怪的是,痛恨的心情消失了:难道是他曾经抓住这个人的领带,想把这个人勒死?现在克雷莫夫心中又出现了同这个人的亲近感。桌子已经不能把他们分开,坐在一起的是两个同志,两个苦命人。

克雷莫夫忽然想起那个枪毙以后没死、穿着血糊糊的衬衣从夜晚的秋日原野回到方面军特别科的人。

“这也是我的命运,”他想道,“我也无处可去。已经晚啦。”

后来他又要求上厕所,后来昨天的那个大尉又来到,把窗帘拉起,把灯熄了,抽起烟来。

于是克雷莫夫又看到白天的亮光,阴森森的,好像不是来自太阳,来自天上,而是来自内部监狱的灰色砖墙。

四十四

几张床全空着,另外三个人也许搬到别的囚室去了,也许他们都在受审。

他被打得皮开肉绽,失去自制力,带着被遗弃的人生躺在床上,腰部疼得非常厉害,好像他的肾被打坏了。

在人生毁灭的痛苦时刻,克雷莫夫懂得了女人爱情的力量。妻子!只有她珍爱这个被无情的铁脚践踏得血肉模糊的人。他浑身是血,她会给他洗脚,给他梳理蓬乱的头发,她看着他的失神的眼睛。他的心灵被伤害得越厉害,世上的人越是厌恶他、瞧不起他,她就越是觉得他可亲可爱。她跟在汽车后面跑,她在库兹涅茨桥站队,在劳改营铁丝网外面等候,她一心想着给他送几块水果糖、几头大蒜,她在煤油炉上给他烙糖饼,她愿意花费几年的时间,为的是哪怕跟他见半个小时的面……

不是所有睡过觉的女子,都能跟妻子一样。

他因为绝望得像挨刀割一样,就也想唤起另一个人的绝望。

他想好了一封信的开头几句:

“你听到这事会十分高兴的,不是因为我被抓了起来,而是因为你已经离开我了,你可以感谢你那耗子般的本能,使你离开了下沉的船……我是一个人……”

眼前闪过侦讯员桌子上的电话机……一头健壮的公牛打他的腰,打他的腋下……大尉拉起窗帘,把灯熄了……档案材料沙沙响着,他在沙沙声中渐渐入睡……

忽然有一根烧得红红的、弯弯的锥子扎进他的头盖骨,似乎他的脑子发出焦糊味:是叶尼娅·尼古拉耶芙娜告密,出卖了他!

十分精彩!十分精彩!这是有一天早晨在兹纳缅卡,在共和国革命军事委员会主席办公室里对他说的话……那个尖下巴胡、戴着光闪闪的夹鼻眼镜的人看过克雷莫夫的文章,就很亲切地小声说了这话。他记得:那天夜里他对叶尼娅说,党中央把他从共产国际召回,让他在政治出版社主编一本书。“当年也算一个人物呀。”他想道……就是那天夜里他对叶尼娅说,托洛茨基看了他的文章《革命与改良—中国与印度》,说:“十分精彩。”

说这话的时候没有旁人在场,他也没有对任何人转述过,只是对叶尼娅说了说,这就是说,侦讯员是从她嘴里听说的。是她告密的。

他再不觉得已经有七十个小时没有睡觉,他似乎已经睡足了。是强迫她的?反正还不是一样。同志们,米哈伊尔·西多罗维奇,我完了!把我弄死了。不是手枪子弹、不是拳头把我打死的,不是死于不能睡觉。是叶尼娅把我弄死的。我来写供状,什么都承认。有一个条件:你们要说明,是她告密的。

他从床上爬下来,用拳头擂起门来,值班守卫马上就朝小孔里窥视,他朝守卫喊道:

“带我去见侦讯员,我什么都招认。”

值班班长走来,说:

“别吵闹,等什么时候提审,您招认好啦。”

他不能一个人待在这儿。还不如挨打,昏迷过去。既然医生认为没事儿……

他一瘸一拐地走到床边,当他觉得再也经受不住精神上的痛楚,当他觉得头脑就要碎裂,觉得好像有成千上万的碎片往心里、喉咙里、眼睛里直钻的时候,他明白了:叶尼娅不可能告密!于是他咳嗽起来,哆嗦起来:

“原谅我,原谅我吧。我没有福气跟你在一起,这怪我,不怪你。”

自从捷尔任斯基踏进这座楼房里来,这里的人从来没有体会过的美妙感情来到他心中。

他醒了过来。一头贝多芬式乱发的大块头卡茨涅林鲍肯坐在他的对面。克雷莫夫对他笑了笑,他那低低的肥厚的额头皱了起来。克雷莫夫明白,卡茨涅林鲍肯认为他的笑是精神失常的表现。

“我看见了,他们打得您很厉害。”卡茨涅林鲍肯指着克雷莫夫血糊糊的衣服说。

“是的,打得挺厉害,”克雷莫夫歪着嘴回答说,“你们怎么样?”

“我上医院去逛了逛。他们两个都走了:特别会议又判了德列林格十年,就是说,一共是三十年了;鲍戈列耶夫转到别的囚室去了。”

“啊……”克雷莫夫说。

“您说说吧。”

“我在想,”克雷莫夫说,“到了共产主义社会,新的克格勃会秘密搜集人的一切好的行为,搜集每一句好话。那时的谍报人员会在电话里窃听一切和忠诚、正直、善良有关的言论,并且在书信里寻找,从公开的谈话里提炼,把一切好的汇集到卢比扬卡来,归入档案。光搜集好的!这儿将增强人的信心,而不是像现在这样摧毁人的信心。第一块基石是我砌的……我相信,我胜利了,告密、谎言没有把我制服,我相信,我相信……”

卡茨涅林鲍肯漫不经心地听他说着,插话说:

“这话都很对,将来会这样的。不过应该补充的是,编成这种美好的档案之后,会把您弄到这大楼里来,还是要枪毙。”

他用问询的目光看了看克雷莫夫,怎么也无法理解,克雷莫夫那土黄色的脸,那凹下去又肿起来的眼睛,那带着黑色血印子的下巴,为什么在幸福而安详地笑着。

四十五

保卢斯的副官亚当斯上校站在打开的手提箱前面。

保卢斯的勤务兵里特尔蹲着,在地上铺了报纸,把所有内衣放在报纸上,在挑拣着。

夜里,亚当斯和里特尔在元帅的办公室里烧文件,烧掉了保卢斯亲自用的大地图,本来亚当斯认为那是神圣的战争遗物。

保卢斯一夜没有睡。他早晨也没有喝咖啡,冷漠地看着亚当斯在忙活。他不时地站起来,跨过放在地上等待焚烧的一摞摞文件,在房子里走一走。用麻布裱过的一些地图烧得很不痛快,把炉条堵塞起来,里特尔不得不用炉钩一再地清理炉膛。

每一次里特尔打开炉门,元帅都要把手伸到炉口。亚当斯把军大衣披到元帅的肩上。但是元帅不耐烦地动了动肩膀。于是亚当斯又把大衣挂到衣架上。

也许,元帅此时已经看到自己在西伯利亚的俘虏营里:他和士兵们一起站在火堆前烘手,前前后后都是空旷的荒野。

亚当斯对元帅说:

“我叫里特尔往您的提箱里多装一些厚实的内衣。我们小时候想象的最后审判与事实不符:既不会有火,也不会有火炭。”

这天夜里施密特将军来过两次。电话线被切断了,电话机不响了。

自从被包围的那一刻起,保卢斯就明白,他率领的军队不能在伏尔加河上继续作战了。

他看出来,当初保证他夏季攻势胜利的一切条件—战术、心理、气象、技术,都在往不利的方向变化,正数已变为负数。他向希特勒要求:第六集团军应当协同曼施坦因在西南方冲破包围圈,开辟一条通道,把部队带出去,并且做好思想准备,大部分重武器只好丢下。

十二月二十四日叶廖缅科的部队在麦绍夫卡河地区给予曼施坦因部队以重创之后,任何一个步兵营营长都清楚了,在斯大林格勒进行抵抗是不行的。不清楚这一点的只有一个人。他把第六集团军改为方面军前哨,即从白海到捷列克河的方面军。他宣布第六集团军是斯大林格勒的堡垒。可是第六集团军司令部里的人却说,斯大林格勒已经变成战俘集中营。保卢斯又通过加密电报报告说,有一些有利于突围的条件。他等待着可怕的怒火爆发,因为还没有人敢于两次反对最高统帅的意图。他听说过,希特勒曾经扯掉龙德施泰特元帅胸前的骑士十字勋章,在场的布劳希奇吓得心脏病都发作了。和元首是开不得玩笑的。

元月三十一日,保卢斯终于收到了回电:授予他元帅军衔。他又做了一次尝试,想说明自己的正确,得到的是帝国的最高勋章—带有橡树叶的骑士十字勋章。

他渐渐意识到,希特勒已经开始拿他当死人对待了—这等于死后追授元帅军衔,死后追授带橡树叶的骑士十字勋章。他现在只有一样用处:创造英勇抵抗的领导者的悲剧形象。国家宣传机构已经把他率领的几十万人宣扬为圣徒和受难者。这些人还活着,在煮马肉,在捕杀斯大林格勒最后的一些狗,在野地里逮乌鸦,捉虱子,把烂纸卷在纸里当烟抽,可是这时候国家的广播电台却为这些未死的英雄播放雄壮的哀乐。

他们还活着,在呵冻红了的手指头,他们的鼻孔里还流着鼻涕,他们的头脑里还闪着一个一个的念头,想吃,想偷,想装成病人,想投降做俘虏,想上地下室里和苏联娘们儿亲热亲热,可是这时候国家的儿童合唱队和少女合唱队已经在广播里唱:“他们死了,为的是德国的生存。”似乎他们的罪恶而美好的生命能够复活,国家就一定灭亡。

一切正如保卢斯预言的。

他怀着无比难过的心情,感觉自己断言军队会毫无例外地全部完蛋是说对了。他从自己的军队的完蛋中也不由自主地产生一种奇怪的满足,感到自己的高明。

在节节胜利的日子里被压制下去、驱赶出去的一些念头又进入脑际。

凯特尔和约德尔把希特勒称为“神圣的元首”。戈培尔说,希特勒的悲剧就在于,他在战争中不可能遇到与之匹敌的天才统帅。蔡茨列尔则说,希特勒曾要求他把战线拉直,因为弯曲的战线有损他的美感。那么,就像神经错乱、神经衰弱似的不肯进攻莫斯科,又算什么呢?那么,那一次突然变得优柔寡断,下令停止进攻列宁格勒,又算什么呢?他的坚决抵抗的狂热战略的基点是:害怕失去威望。

现在一切都完全明朗了。

但是正是完全明朗才可怕。他可以不服从命令!当然,元首会处死他的。但是他可以救活许多人。他在很多人的眼里看到了责难的神气。他可以,可以挽救军队!他怕希特勒,怕丢掉性命!保安总部驻集团军司令部的最高代表哈尔布前几天在飞往柏林的时候,用含糊的语言对他说,即使在德国这样的民族中,元首也是太伟大了。是的,是的,噢,当然。

全是矫揉造作的腔调,全是虚夸腔调。

亚当斯打开收音机。从噼啪的杂音中出现了音乐声:德国在为斯大林格勒的死者举行安魂祈祷。音乐声中隐藏着一股特别的力量。也许,对于民族,对于未来的许多战役来说,元首创作的神话比起拯救挨冻挨饿挨虱子咬的许多人更为重要。也许,你在阅读条令、安排战斗时间表、观看作战地图的时候,并不了解元首的逻辑。

可是,也许,在希特勒为第六集团军设计的受难光环中,会出现保卢斯及其军队的新生,他们在未来德国的新命运。

在这方面起作用的不是铅笔、计算尺和计算器。起作用的是一位奇怪的军需将军,他有另外的计算标准,有另外的储备。

亚当斯呀,亲爱的亚当斯,忠实的亚当斯,要知道,一个具有极高的精神气质的人总是必然有所怀疑的。只有那些目光短浅、永远觉得自己正确的人才会凌驾于世界之上。气质高尚的人不会凌驾于国家之上,不会做出什么伟大的决定。

“他们来了!”亚当斯叫起来。他吩咐里特尔:“拿开!”于是把打开的提箱推到一边,又抻了抻自己的军服。

胡乱放进提箱里的元帅的袜子后跟上有窟窿,里特尔紧张焦急起来,不是怕性子焦躁的保卢斯穿到破袜子,而是怕不怀好意的苏联人的眼睛看见这袜子上的窟窿。

亚当斯站着,把两手放在椅背上,背着马上就要打开的门,用镇静、关切、爱护的目光看着保卢斯,他觉得,元帅的副官就应该这样。保卢斯多少挺了挺身子,不靠在桌子上,把嘴唇紧紧闭起。就是在此刻元首也希望他演戏,于是他准备演戏。

门就要开了,黑暗的地下室的这个房间就会对大地上活着的人起重要作用。痛苦和焦虑过去了,只剩下惧怕,怕的是,推门的不是也准备演出盛大的话剧的苏军指挥部的代表,而是习惯了轻轻扣自动枪扳机的勇猛的苏军士兵。还有一种担心未来的念头:等演戏一收场,人的生活就要开始了,是什么样的生活呢,上哪儿呢,是上西伯利亚,进莫斯科的监狱,还是进集中营的棚屋?……

四十六

夜里,伏尔加东岸的人看到,斯大林格勒的天空被五彩缤纷的信号弹映照得通明。德军投降了。

就在这天夜里,不少人从伏尔加东岸朝斯大林格勒涌去。因为到处都在传说,留在斯大林格勒的居民最近一个时期饿坏了,所以士兵和军官们以及伏尔加舰队的水兵们纷纷带着面包和罐头来了。有些人还带着酒和手风琴。

但是很奇怪,这些不带武器,在夜里最先来到斯大林格勒的士兵,在把面包交给城市保卫者,又拥抱又接吻的时候,却好像很伤心,既没有笑,也没有唱歌。

一九四三年二月二日早晨,雾气沉沉。伏尔加河面融化的冰凌和冰窟窿冒着腾腾的水气。在炎热的夏日和寒冷的北风天里一样阴沉的荒凉草原上升起了太阳。干干的雪在又平又广阔的原野上飞驰,时而卷成圆柱,旋成雪轮,时而突然失去动力,落了下来。东风的脚掌留下一处处脚印:刺草吱吱作响的茎上围了雪领子,沟坡上留下一道道雪的波纹,露出光秃的泥土,一个个小土包露出秃顶……

站在斯大林格勒的河岸上看去,跨过伏尔加河的人们好像是从草原的雾中冒出来的,好像他们都是严寒和冷风塑成的。

他们来斯大林格勒无事可干,领导没有派他们来,这儿的战事结束了。是他们自己要来。有红军士兵、修路工人、面包师傅、参谋人员、驭手、炮兵、前方被服厂的裁缝、修理车间的电工和机械工。和他们一起过伏尔加河、爬岸坡的有裹着围巾的老头子,有穿军装棉裤的老太婆,有些小男孩和小姑娘还拖着小小的雪橇,上面装着包袱和枕头。

这座城市发生了奇怪的事情。汽车喇叭声响了起来,拖拉机的发动机开始轰鸣,喧闹的人们拉着手风琴的人走在街上,跳舞的人的毡靴踩得积雪越来越结实,士兵们欢叫,大笑。可是城市没有因此活过来,城市好像死了。

几个月之前斯大林格勒就不再过自己的正常生活了:市里的学校、工厂、女装商店、业余剧团、市公安局、托儿所、电影院,一个一个地关闭了。

在烧遍各街区的大火中诞生了一座新的城市—战时的斯大林格勒。战时城市有自己的街道和广场布局,有自己的地下建筑、自己的街道交通规则、自己的商业网、自己的工厂车间、自己的手工业、自己的坟地、酒吧间、音乐厅。

每一个时代都有自己的世界名城。它是时代的灵魂,时代的意志。

第二次世界大战是全人类的重要时代,在这一时代的一定时期内斯大林格勒成为世界性的城市。它成为人类的思想和激情。许多工厂为它加工产品,许多报刊为它报导,许多议会领袖为它发表演说。但是,当成千上万的人从草原上来到斯大林格勒,空旷的街道上到处是人,第一批汽车的马达声响起来的时候,这座战时的世界名城就不再存在了。

这一天的报纸报道德军投降的详细情形。欧洲、美洲、印度的人都知道了,保卢斯元帅是怎样从地下室里走出来,在舒米洛夫将军的第六十四集团军司令部里怎样对德国的将军们进行了初步审讯,保卢斯的参谋长施密特将军穿的是什么样的衣服。

这时候,世界大战的首城已经不存在了。希特勒、罗斯福、丘吉尔的眼睛已经在寻找世界大战的新的集中点。斯大林用手指头敲着桌子,问总参谋长,要把斯大林格勒的部队从现在已成为后方的地区调往新的集结地区,交通工具是否够用。战时的世界名城,尽管还到处是能征惯战的将军和巷战的高手,还到处是武器、作战地图、交通壕,可是已经不再存在了,它开始踏上新的生活轨道,这生活轨道靠今日的雅典和罗马开辟。历史学家、陈列馆解说员、教师和总是感到寂寞的中学生已经不知不觉渐渐成为城市的主人。

一座新的城市渐渐诞生。这是一座劳动和日常生活的城市,有工厂、学校、托儿所、公安局、戏院、监狱。

薄薄的雪掩盖了往火线上输送弹药和面包、搬运机枪、抬送粥桶的小路,也掩盖了狙击手、观测员、截听员进入自己秘密的石头小屋的弯弯曲曲的隐蔽小道。

薄薄的雪掩盖了联络员从连里跑向营里的道路,掩盖了巴秋克师前往班内伊山沟、肉类联合加工厂和水塔的道路……

薄薄的雪掩盖了这座伟大城市的居民去向邻居要黄烟、喝几杯生日酒,上地下澡堂里洗澡,打牌,上邻居家去尝酸白菜的道路;掩盖了他们走亲访友,去找钟表匠、打火机修理人、裁缝、手风琴手、仓库管理员的道路。人们在铺设新的道路。

人们走路不再紧贴着断垣残壁,不再绕来绕去躲着走。

像网一般的战时的大路、小道都盖上了薄薄的雪,在这盖了雪的总长有百万公里的道路上,没有一个新鲜脚印。

一层薄雪上面,很快又盖上一层,雪下的小路模糊不清了,完全消失了……

这座世界名城的老居民有一种说不出的幸福和空虚感。保卫斯大林格勒的人却产生了一种奇怪的苦恼。

城市空了。集团军司令、各步兵师师长、民兵波里亚科夫老头子、士兵格鲁什科夫都感觉到这种空虚。这种感觉是不应该有的,难道可以因为大战胜利、再没有死亡而产生苦闷?

不过事实就是这样。司令员桌上装在黄黄的皮套子里的电话机不响了,机枪护罩上积起了雪领子,炮队镜和射击孔都落满了雪;磨破和起了毛的平面图和地图从图囊转入军用包,又从军用包转入一些排长、连长、营长的手提箱和行李包……―群一群的人在炮火摧毁的房屋中间走来走去,拥抱,呼喊“乌啦”……人们你看看我,我看看你。“小伙子们多么好啊,又勇猛,又单纯,又善良,我们穿的是棉袄,戴的是棉帽,你们穿戴都跟我们一样。我们都干了不少事,想想我们干的是什么事,都觉得可怕。我们把世界上最有分量的东西抬高了,把真理抬到了歪理之上,你倒是试试看……以前那是在童话里说的,现在可不是童话。”

全是乡亲:有的是库波罗斯山谷来的,有的是班内伊山谷来的,有的是从水塔附近来的,有的是“红十月”工厂的,有的是马马耶夫冈来的,和他们在一起的还有市中心的居民,有原来住在察里津河边的,住在码头区的,住在油库附近的坡下的……他们又是主人,又是客人,他们自己向自己祝贺,冷风吹得旧铁皮叮当作响。有时他们向空中放几枪,有时拉响一颗手榴弹。他们见了面就拍肩膀,有时还拥抱,用冰冷的嘴唇接吻,过后又不好意思地、快活地骂两声……他们一齐从地下冒出来,有钳工、旋工、农民、木匠、挖土工人,他们打退了敌人,他们重犁了石头、钢铁、泥土。

世界名城与其他城市的不同,不仅在于人们都感觉到它与全世界的工厂与土地都有联系。

世界名城与众不同,在于它有灵魂。

战时的斯大林格勒就有灵魂。它的灵魂就是自由。

反法西斯战争的首城变成了无声无息、冰冷的瓦砾场,战前苏联这个工业与港口州城不存在了。

十年之后,这儿将有成千上万的囚徒筑起雄伟的大坝,建起世界上一流的国家级大水电站。

四十七

一名德国士官在掩蔽所里醒来,不知道已经投降,因此出了一件事情。他开了一枪,打伤了萨德涅普卢克中士。这事引起苏联人的愤怒。他们正监视着一个个德国兵从仓库里走出来,把枪支丢进叮叮当当响着、越来越大的枪支堆里。

俘虏们走着,尽量不朝两边看,表示他们的眼睛也做了俘虏。只有满脸黑白胡茬的士兵施密特在走出来的时候,微微笑着打量着苏军士兵们,似乎相信会看到一张熟悉的脸。

昨天刚从莫斯科来到斯大林格勒方面军司令部的微微有些酒意的菲里莫诺夫上校,和他手下的一名翻译站在一起,他们在这个受降点负责接受维格列尔将军的师投降。

菲里莫诺夫的军大衣上佩戴着新的金色肩章,带有红色镶边和黑色绦带,在斯大林格勒的营长、连长们那肮脏、烟熏火燎的军装棉袄和皱皱巴巴的暖帽当中,在德国俘虏那同样肮脏、同样经受了烟熏火燎、同样皱皱巴巴的衣帽当中,显得格外突出。

昨天他在军委的食堂里说,在莫斯科的军需总库里保存着很多金线,本来是为沙俄的军队做肩章用的,他的朋友们都认为,弄到用这种优质的旧材料做的肩章是很大的幸运。

在响起枪声,受了轻伤的萨德涅普卢克叫起来的时候,上校大声问道:

“是谁开枪,怎么一回事儿?”

有好几个声音回答说:

“是一个糊涂虫,一个德国人。已经把他结果了……他好像还不知道……”

“怎么不知道?”上校叫道。“这个坏蛋,他觉得我们流的血还少吧?”

他对担任翻译的高个子犹太裔政治指导员说:

“把他们的长官给我找出来。他这个坏蛋头儿,应该为这一枪负责任。”

这时候上校发现了士兵施密特那微微笑着的大脸,便叫起来:

“这坏蛋,又打伤了一个,你高兴,是不是?”

施密特不明白,为什么他非常想表示好意的笑竟引起这位苏联首长的喝叫,等到似乎和这声喝叫毫无联系的手枪声响过,他已经什么也不明白,踉跄一下,便倒在后面跟上来的士兵脚下。他的尸体被拖到一旁,他侧身躺着,认识他的人和不认识他的人一个一个从他身旁走过。后来,等俘虏们走光了,孩子们也不怕死人,爬进空了的仓库和掩蔽所,在木板床上起劲儿蹦跳起来。

菲里莫诺夫上校这时候在查看一名营长的地下室,他赞叹这里面的一切都搞得很牢固、很舒服。一个士兵把一名目光镇静而明亮的年轻德国军官带到他面前,翻译说:

“上校同志,这是中尉列纳尔德,是您吩咐带来的。”

“是哪一个?”上校惊讶地问。因为他觉得这名德国军官的脸很讨人喜欢,又因为他生平第一次干了杀人的事心里很不是滋味,就说:

“您把他带到集中点,不要出什么事儿,您要亲自负责,让他活着走到那儿。”

最后审判日快完了,被枪杀的德国兵脸上的笑容已经不见了。

四十八

方面军政治部第七科军事翻译组组长米海洛夫中校,负责押送被俘的元帅前往第六十四方面军司令部。

保卢斯走出地下室,没有理会苏联的官兵。官兵们都用十分好奇的目光打量着他,估价他那从肩到腰镶着绿皮的元帅军大衣和灰色兔皮帽。他昂首阔步地走过去,也不看斯大林格勒的一片瓦砾,径直走向等待着他的司令部的吉普车。

米海洛夫在战前常常参加外交方面的接待,所以他和保卢斯在一起应付自如,一眼便能分清冷淡的恭敬与不必要的殷勤。

米海洛夫和保卢斯并肩坐着,注视着他的面部表情,等待着元帅先开口说话。这位元帅的表现和他参与预审的其他将军的表现很不一样。

德军第六集团军参谋长用慢条斯理的懒洋洋的声音说,灾难是罗马尼亚人和意大利人造成的。长着鹰钩鼻的济克斯特·冯·阿尔尼姆中将阴沉地晃荡着奖章,补充说:

“不仅是加里波第和他的第八集团军,还有俄罗斯的寒冷,再加上粮食和弹药不足。”

佩戴着骑士铁十字勋章和五次负伤奖章的白发苍苍的坦克军军长施列麦尔打断这场谈话,要求保留他的提箱。于是大家都开口了,不论是温和地笑着的医务部长里纳尔多将军,还是脸上带有刀伤疤的阴沉的坦克师师长柳德维克上校。保卢斯的副官亚当斯上校丢掉了盥洗用品的箱子,特别激动,他张着两只手,摇晃着脑袋,豹皮帽的两只帽耳也摇晃着,就像刚从水里出来的一条良种狗。

他们又成了人,但还是没有怎么变好。身穿整洁的白色小皮袄的汽车司机小声回答米海洛夫吩咐开慢一些的话:

“是,中校同志。”

他想等到战后回家之后,对司机弟兄们说说保卢斯的情形,夸耀一番:

“当年我开着汽车押送保卢斯元帅的时候……”

此外,他还想把汽车开得有点儿与众不同,好让保卢斯想:

“瞧,苏联司机,技术真是一流的。”

在战场上待久了的人,看到苏联人和德国人一个挨一个地混杂在一起,觉得有点儿不可思议。一组组快活的士兵在搜索地下室,爬进自来水管道,把德国人赶到寒冷的地面上。

苏军士兵在空场上、街道上用推拉和吆喝对德军重新进行整编:把不同兵种的士兵排成一列列行军纵队。

德国人看着一只只紧握武器的手,乖乖地走着,尽可能不打趔趄。他们这样乖,不仅是因为他们害怕苏联人的手指头可以轻轻地扣一下扳机。胜利者有一股威风,有一股令人昏迷、令人难受的劲头儿迫使人们服从。

送元帅的汽车向南开去,俘虏队迎着汽车走来。宏亮的扬声器大声叫着:

昨日里我出发远程,姑娘在门口挥头巾相送……

两个人架抬着一名伤病员。被抬的人用苍白的脏手搂着他们的脖子。于是两颗头几乎挨在一起,在他们之间的是一张毫无生气的脸和火辣辣的眼睛。

四名士兵用被子从地下室里抬出一名伤员,一堆堆青黑色的钢铁武器堆在雪地里,就像一个个去了穗的钢铁麦秸垛。

战士们鸣枪致敬—将一名牺牲的红军战士葬入坟墓。

旁边横七竖八地躺着德国人的尸体,是从医疗队的地下室里拖出来的。罗马尼亚士兵戴着贵重的黑白两色皮帽,哈哈笑着,挥着手,嘲笑活着的和死去的德国人。

一队队俘虏从苗圃方向,从察里津、从专家公寓走来。他们走的是一种很特别的步子,那正是失去自由的人和动物走的步子。受轻伤和冻伤的人拄着棍子和烧糊的木板条子。他们走着,走着。似乎所有的人只有一张青灰色的脸,所有的人只有一双眼睛,所有的人只有一副痛苦与烦恼的表情。

真奇怪!在他们当中竟有那么多小个子、大鼻子、低额头,长着可笑的兔子嘴和麻雀般小头的人。竟有那么多黑皮肤的阿利安人,满脸粉刺、脓疱、雀斑。

这是一些不漂亮的弱者,这都是妈妈生的、妈妈疼爱的人。那些大下巴、翘嘴唇、浅色头发、白净脸皮、挺着胸脯的恶徒和民族似乎消失了。

多么奇怪,这一群群由妈妈生养的不漂亮的人和一九四一年秋天德国人用树条和棍子赶往西边集中营的那些俄罗斯妈妈生养的苦难的不幸人群,如同兄弟般相像。在仓库和地下室那边,不时地响起手枪的声音,向冰封的伏尔加河移动的人群就像一个人一样,全都懂得这枪声的意义。

米海洛夫中校看着跟他坐在一起的元帅。司机也在反光镜里看着。米海洛夫看到的是保卢斯的痩长的脸颊,司机看到的是他的额头、眼睛和闭得紧紧的嘴巴。

他们的汽车擦过炮筒朝天的大炮,擦过正面带有十字标的坦克,擦过帆布篷在风中拍打的载重汽车,擦过装甲运输车和自行火炮。

第六集团军的钢铁躯体、它的肌肉都冻进了土里。人群在旁边慢慢移动着。似乎人群也会停住,也会冻住,冻进土里。

米海洛夫、司机和一名押解士兵都在等待着保卢斯,等着他呼唤、转头。但是他却不作声。真不明白他的眼睛在看什么,不明白他的眼睛给他的心灵带来什么。

保卢斯是不是怕他手下的士兵看见他,还是希望他们看见他?

忽然保卢斯向米海洛夫问道:

“请您告诉我,什么叫马合烟?”

米海洛夫听到这个突如其来的问题,还是不明白保卢斯在想些什么。元帅操心的,是希望每天有汤喝,有烟抽,睡得暖和。

四十九

一座二层楼的地下室,原是德国秘密警察战地派出机构的驻地。有一些德军俘虏正从里面往外抬苏联人的尸体。

有些妇女、老头子、小孩子不顾寒冷,站在哨兵旁边,注视着德国人把尸体放到冻实的土地上。

大部分德国人带着木然的神情,他们慢腾腾地走着,无可奈何地呼吸着死尸的气味。

其中只有一个穿军官大衣的年轻人,用肮脏的手帕裹着鼻子和嘴巴,像马抽搐似的不住摇晃着头,就好像有马蝇在咬。他的眼睛流露着痛苦得快要发疯的神情。

俘虏们把担架放在地上,先不忙着把尸体抬下来,而是要站在旁边思索一会儿。因为一些尸体的胳膊和腿被砍下来了,所以要看看哪一条胳膊或腿是哪一具尸体上的,好把胳膊、腿与身子摆放在一起。大部分死者半裸着身子,穿着内衣,有的穿着军裤。有一具尸体完全光着身子,嘴大张着,好像在叫喊,肚皮贴到脊梁上,阴部有红红的毛,两条腿细细的。

很难设想,这些嘴巴和眼窝都成了大窟窿的尸体不久前还是有名有姓、有家的活人,不久前还在说:“亲爱的,好姑娘,吻吻我吧,你看看我,不要把我忘了。”还盼望能喝到一杯酒,还在抽烟。

显然,只有裹着嘴巴的军官能感觉到这一点。

但偏偏是他让站在地下室门口的妇女们特别气愤,她们都很留心地注视着他,而漫不经心地看着其余的战俘,其中有两个人穿的大衣上还带着撕掉了党卫军标志留下的新鲜印子。

“哼,你还恶心呢。”一个领着小孩子的矮个妇女注视着那名军官,嘟哝说。

穿军官大衣的德国人感觉到一位苏联妇女那种缓慢而沉重的目光在他身上的压力。仇恨的感情一旦产生,就要寻找而且一定要找到着力点,就好比凝聚在森林上空雷雨云层里的电力,盲目地寻找轰劈的树木,是不会找不到的。

和穿军官大衣的德国人抬一副担架的是一名小个子士兵,脖子上缠着方格毛巾,腿上裹着麻袋片,用电话线扎着。

一声不响地站在地下室门口的人的目光是很不和善的,所以德国人一进入黑沉沉的地下室就觉得轻松,而且都不急着走出来,宁愿在黑暗里闻臭气,不愿到新鲜空气里去见阳光,每次德国人带着空担架朝地下室里走去,都能听到他们已经熟悉的俄罗斯人的骂声。

俘虏们在向地下室走的时候,并不加快脚步,因为他们本能地感觉到,他们只要一有什么急促的动作,人群就会扑向他们。穿军官大衣的德国人叫了起来,哨兵生气地说:

“你这小子,有什么意见,你怎么,要是那个德国佬倒下去,你替他抬吗?”

德国兵在地下室里议论起来:

“挨骂的暂时还只有这位中尉。”

“你可注意那个娘们儿,一个劲儿地看着他呢。”

在地下室的黑暗处有一个声音说:

“中尉,哪怕这一次您就留在地下室里。要不然他们一收拾您,我们也要遭殃。”

中尉用含含糊糊的声音嘟哝说:

“不,不,不能躲,这是最后的审判。”

他又对自己的搭档说:

“走吧,走吧,走吧。”

这一次从地下室里往外走,中尉和他的搭档走得比一般多少快一点儿,因为抬的尸体轻些。他们抬的是一个未成年的姑娘。尸体已经蜷缩,干瘪,只有那散乱的亮闪闪的头发保持着青春的小麦色的美,披在死掉的鸟儿般可怕的黑褐色小脸周围。人群轻轻地啊呀了一声。

那个矮矮的娘们儿尖声叫起来,叫声就像一把寒光闪闪的刀子,插进寒冷的空中。

“孩子呀!孩子呀!我的孩子呀!”

这一声声对别人的孩子的呼叫震动了人群。这个妇女梳理起死人头上那尚带有烫发痕迹的头发。她注视着那张脸和僵了的歪嘴唇,她同时看到的又是这可怕的容貌,又是活泼、可爱,曾经在襁褓里对着她笑的那张脸儿,只有当妈妈的才会这样。

这个妇女站起身来。她朝那个德国人走去。大家都看到了这一点。她的眼睛看着他,同时在地上寻找没有跟其他砖头冻在一起的砖头,寻找她那有病痛的、因为干重活儿和被冷水、开水、碱水弄伤了的手拿得起来的砖头。

哨兵感觉到不可避免要出事情,但也无法制止这个妇女的行动,因为她比他和他的自动步枪更刚强有力。德国俘虏们的眼睛也都不能离开她,孩子们也都聚精会神地、急切地看着她。

可是这个妇女什么也看不见了,只看到那个裹着嘴巴的德国人的脸。她自己也不明白她是怎么一回事儿,她带着一股力量,这股力量支配着周围的一切,她自己也受这股力量支配着,在自己的棉袄口袋里摸到昨天一名红军战士给她的一块面包,把面包递给那个德国人,说:

“给你,你拿着,吃吧。”

后来她自己也不明白,怎么会有这种事儿,为什么她要这样。她一生中有过许多受气、绝望、懊恼的时刻:她和诬赖她偷油的邻居吵架,被不愿听她家长里短地告状的区苏维埃主席从办公室里赶出来,儿子结婚后把她从正屋里撵出来,怀孕的儿媳妇骂她老娼妇。每到这种时刻,她总是伤心得不得了,连觉也睡不着。后来有一天夜里她躺在床上,想起了这个冬天的早晨,也是又伤心又懊恼,心想:“我过去傻,现在还是傻。”

五 十

诺维科夫的坦克军军部开始收到各旅旅长报来的令人不安的情报。侦察队发现了德方没有参加过战斗的新的坦克部队和炮兵部队,显然敌人是从大后方调来了后备兵力。

这些情报使诺维科夫担心起来:先头部队在推进,不能保障两翼,如果敌人切断了为数不多的几条冬季道路,坦克就得不到步兵的支援,得不到燃料。

诺维科夫和格特马诺夫讨论了这一情况。他认为,必须立即督促落在后面的后勤部队赶上来,并且暂时停止坦克前进。格特马诺夫很希望坦克军为解放整个乌克兰奠定基础。他们决定:诺维科夫下部队去,就地检查情况,格特马诺夫负责督促落在后面的后勤部队赶上来。

诺维科夫在去各旅之前,给方面军副司令打了一个电话,把情况报告了一下。他事先就知道司令会怎样回答,司令当然不会担负责任:既不会下令叫坦克军停下来,也不会主张诺维科夫继续前进。

果然,副司令吩咐火速向方面军侦察科询问敌军情况,同时答应把他和诺维科夫的通话内容报告司令。

在这之后,诺维科夫和友邻部队步兵军军长莫洛科夫进行了联系。莫洛科夫是一个粗暴的、爱发火的人,总是怀疑友邻部队向方面军司令提供对他不利的情报。他们吵过嘴,甚至还骂过娘,虽然不是直接骂个人,骂的是坦克与步兵之间的脱节越来越厉害。

诺维科夫又打电话给左面的友邻部队炮兵师师长。

炮兵师师长说,没有方面军的命令,他不能再向前推进。

诺维科夫明白他的意图:这位炮兵师长不愿意只起辅助作用,只是保证坦克“射门”,他自己也想“射门”。

诺维科夫和炮兵师长通话刚刚结束,参谋长便走了进来。诺维科夫从来没见过涅乌多布诺夫这样性急,这样慌乱。

“上校同志,”他说,“空军集团军参谋长给我打来电话,说他们准备把支援我们的飞机转移到方面军的左翼。”

“这是怎么啦,他们害了神经病,还是怎的?”诺维科夫叫道。

“这事儿很简单嘛,”涅乌多布诺夫说,“有人不希望我们首先进入乌克兰。希望因为这件事得到苏沃洛夫勋章和赫梅利尼茨基勋章的人多得很。没有空军掩护我军就只能停止前进了。”

“我马上给司令打电话。”诺维科夫说。

但是给司令的电话没有打成,因为叶廖缅科上托尔布欣的集团军里去了。诺维科夫又给副司令打电话,副司令不愿意做出任何决定。他只是对诺维科夫为什么没有下部队去表示惊讶。

诺维科夫对副司令说:

“中将同志,我军是方面军各部中西进最远的,不经过协商,就这样撤除对我军的空中掩护,这算怎么一回事儿?”

副司令很恼火地对他说:

“司令部更知道怎样利用空军,参加进攻战的不是你们一个军。”

诺维科夫不客气地说:

“要是坦克受到空中轰击,我怎么对坦克手们说呢?我拿什么掩护他们呢,拿方面军的指示吗?”

副司令这一次没有发火,倒是用和解的口吻说:

“您下部队去吧,我把情况报告给司令。”

诺维科夫刚刚放下话筒,格特马诺夫走了进来。他已经穿起大衣,戴起皮帽。一看到诺维科夫,就带着无可奈何的神气把两手一摊。

“诺维科夫同志,我以为你已经走了呢。”

他婉转而亲切地说:

“后勤部队落后了。可是后勤部队副司令对我说,不能让坦克去和受伤、生病的德国人追着玩儿,浪费紧缺的汽油。”

他带着幽默的神气看了看诺维科夫:

“真的,我们又不是共产囯际的分部,我们是坦克军。”

“这和共产国际有什么关系?”诺维科夫问道。

“您走吧,走吧,上校同志,”涅乌多布诺夫用恳求的口气说,“时间很宝贵。我保证尽一切可能和方面军司令部谈谈。”

自从那天夜里达林斯基说过那番话之后,诺维科夫就一直在注视这位参谋长的脸,注意他的动作、声音。每当涅乌多布诺夫拿起羹匙,拿叉子叉腌黄瓜的时候,拿电话筒的时候,拿红铅笔、拿火柴的时候,他心里都在想:

“难道就是这只手打掉达林斯基的牙?”

但是现在诺维科夫没有看涅乌多布诺夫。诺维科夫从来不曾看到涅乌多布诺夫这样亲热、这样惶惶不安,甚至这样可爱。

涅乌多布诺夫和格特马诺夫愿意把命赔上,也要让坦克军第一个跨进乌克兰的边界,让各旅一停不停地继续向西推进。

他们为此可以进行任何冒险,但是只有一点他们不愿意冒险:如果失败,他们不愿意担负责任。

诺维科夫心中不由得出现一股狂热:他想用无线电向方面军报告,坦克军先头几个排已经率先跨越乌克兰边境。这件事没有什么军事意义,没有给敌军造成特别损失。但是诺维科夫希望这样报告。为了取得军事上的荣誉,为了得到方面军司令的感谢,得到勋章和华西列夫斯基的称赞,为了将在广播中宣布的斯大林的通令,为了得到将军头衔,为了让友邻部队羡慕,他希望这样。类似的感情和思想从来没有支配过他的行动,但是也许正因为这样,这种感情和想法现在一旦出现,就特别强烈。

这种愿望没有任何不好的因素……还是像在斯大林格勒,还是像在一九四一年,寒冷仍是无情的,士兵们依然劳累得筋疲力尽,依然有死亡的威胁。但是战争的气氛已经不同了。诺维科夫不了解这一点,所以很惊异,他第一次这样容易、这样一听就明白格特马诺夫和涅乌多布诺夫的话,没有生气,没有懊恼,这样自然地和他们的想法一致。

他的坦克如果加速推进,确实有可能早几个钟头把几十个乌克兰村庄的侵略者赶出去,他看到老人和孩子们兴奋的脸,会非常高兴,会有乡下老婆婆拿他当亲儿子一样,把他抱住,吻他,他的眼里会涌出泪水。新的热情在同时酝酿着,在战争中渐渐形成了新的精神主导方向,而在一九四一年和斯大林格勒河岸边战斗中曾经为主的方向仍然保留和存在,但不知不觉已渐渐成为次要的了。

第一个明白超前完成战争任务的,是一九四一年七月三日在广播中呼唤“兄弟姐妹们,我的朋友们……”的那个人。

很奇怪,诺维科夫虽然和催他动身的格特马诺夫、涅乌多布诺夫一样着急,却迟迟不肯动身。直到他已经坐上汽车,他才明白了原因:他是在等待叶尼娅。

他已经有三个多星期没有收到叶尼娅的信。他每次下部队回来,都要看看,叶尼娅是不是站在军部的台阶上迎接他。她成了他生活的参与者。

在他和旅长们说话的时候,在方面军司令部给他打电话的时候,在他开着坦克冲向前沿阵地、坦克被德军炮弹炸得像一匹小马似的浑身哆嗦的时候,她都和他在一起。他对格特马诺夫说起童年的事情,似乎是说给她听。他想:“啊,我可不能喝酒,要是喝了,叶尼娅一下子就闻出酒气。”有时他想,她会注意到的。他有时很担心地想:“她要是知道我把少校送交法庭,会说什么呢?”

他有时进入前沿观察所的地下室,在一片烟气、电话员的声音、枪炮声和炸弹爆炸声中,会忽然殷切地想起她……

有时他想起她以前的生活,萌生妒意,便惆怅起来。有时他梦见她,等他醒过来,就再也睡不着了。

有时他觉得,他们的爱情会至死不渝,有时却担心起来,怕今后又是他一个人。

他上汽车的时候,仔细看了看通往伏尔加河的大路。大路上空空荡荡。后来他生起气来:她早就应该来到了。也许,她病了?他又想起来,在一九三九年听说她嫁了人,他怎样准备自杀。他为什么偏偏爱她?要知道,有一些爱过他的女子并不差。也许这是幸福,也许是一种病—对一个人非想不可的毛病。好在他没有跟军部里任何一个姑娘发生关系。等她来了,他没有任何顾虑。不错,在三个星期以前他干过一件罪过的事。要是叶尼娅在路上过夜,住在那座罪过的房子里,那一家的年轻女子和她说起话儿,会把他描述一番,说:“那位上校真是一个可爱的男子。”怎么脑子里想起这些乱七八糟的东西,想起来就没有完……

五十一

第二天快到中午时候,诺维科夫从下面部队驱车返回军部。道路被坦克履带碾得坑洼不平,再加上到处是冻土块,一路上汽车不住地颠簸,他被颠得腰、背、后脑勺都疼,似乎坦克手们的疲惫和许多夜不能睡招致的昏沉都传染给了他。

汽车快到军部了,他仔细看了看站在台阶上的两个人。他看到:是叶尼娅和格特马诺夫站在一起,望着渐渐开近的汽车。顿时像火烧一样,头脑里来了一股狂热的劲儿,他高兴得几乎到了难以承受的程度,连气都喘不上来了,他猛地往前一冲,好等车一停就跳下车去。可是坐在后座上的维尔什科夫却说:

“政委和他的女医生在呼吸新鲜空气呢。真应该往他家里寄一张照片,他家夫人才高兴呢。”

诺维科夫走进军部,接下格特马诺夫递给他的一封信,信翻过来一看,认出是叶尼娅的笔迹,把信装进口袋里。

“好吧,你听着,我说说情况。”他对格特马诺夫说。

“你怎么不看信,不爱她了吗?”

“没关系,等一会儿再看。”

涅乌多布诺夫走了进来。诺维科夫就说:

“问题在于人。打仗的时候人在坦克里睡觉。全累倒了。几位旅长也是这样。卡尔波夫还勉强能撑得住,别洛夫跟我正说着话就睡着了,他一连五个昼夜没睡了。坦克手们走路都睡觉,疲乏得连饭也不想吃了。”

“诺维科夫同志,你怎么样,摸了摸情况吗?”格特马诺夫问道。

“德国佬没有什么行动。在我们这地段不会有什么反突击。他们这儿没有什么兵力,不值一提。是弗列捷尔·皮科和菲克的部队。”

他说着,手指头摸着信封。有一小会儿他把信封放开,可是马上又抓住,就好像信会从口袋里跑掉似的。

“好,明白了,清楚了,”格特马诺夫说,“现在该我对你说说了:我和涅乌多布诺夫同志把这事儿捅到天上了。我和赫鲁晓夫同志说了,他答应不把我们地段的空军撤走。”

“他不管作战呀。”诺维科夫说着,就开始在口袋里拆信封。

“噢,这要看怎么说,”格特马诺夫说,“刚才涅乌多布诺夫同志得到空军司令部的答复,空军继续留在我们这儿。”

“后勤部队也要跟上来了,”涅乌多布诺夫急忙说,“条件算是可以了。主要就看您了,中校同志。”

“把我降为中校了,他是太兴奋了。”诺维科夫心里想道。

“是啊,哥儿们,”格特马诺夫说,“看来,是我们要第一个来解放乌克兰了。我对赫鲁晓夫同志说:坦克手们一个劲儿地缠着军部,希望把坦克军命名为乌克兰军。”

诺维科夫听到格特马诺夫这种假话,十分恼火,就说:

“他们只希望一点:好好睡一觉。要知道,已经有五天五夜没睡了。”

“这么说,诺维科夫同志,就这样定了,咱们继续推进,向前冲吧!”格特马诺夫说。

诺维科夫把信封打开一半,把两个指头伸进去,摸到了信纸,心里一阵紧缩,急切地想看到那熟悉的字迹。

“我想做这样一个决定,”他说,“让大家休息十个小时,哪怕多少恢复一下体力。”

“啊呀,”涅乌多布诺夫说,“咱们这一睡,在这十个小时里把世界上的一切都要错过了。”

“等一等,等一等,咱们来研究研究。”格特马诺夫说。他的脸、耳朵、脖子都有些红了。

“就这样啦,我已经研究过了。”诺维科夫微微笑着说。

格特马诺夫忽然发作起来。

“哼,这些家伙真见鬼……没睡够呢,这是什么时候!”他叫道。“以后再找时间睡觉吧!到那时候再睡觉就他妈的没事了。就为了睡觉让全军停留十个钟头?诺维科夫同志,我反对这种不争气的想法!你不是推迟冲进突破口的时间,就是叫大家睡觉!这已经变成制度性的毛病!我要向方面军军委汇报。你领导的不是托儿所!”

“等一等,等一等,”诺维科夫说,“那一次直到把敌人的炮火压下去,我才带领坦克冲进突破口,你因为这事吻过我呀。你最好把这一点也写进报告里。”

“我因为这事吻过你?”格特马诺夫流露出惊愕的神情说。“你简直是说梦话!”

他突然说:

“我可以直截了当地告诉你,我作为一名共产党员,担心的是,你这个纯正的无产阶级出身的人,一直在受着异己分子的影响。”

“啊,是这样,”诺维科夫用响亮的声音说,“好吧,明白了。”

他站起来,把肩膀挺直了,发狠地说:

“我是军长。我说了算数。格特马诺夫同志,要写我的报告,写中篇,长篇,您就写吧,写给斯大林,我也不含糊。”

他走到旁边一个房间里。

诺维科夫把看过的信放在一旁,吹起了口哨,就像过去小时候那样吹,就像那时候站在邻家的窗前,呼唤小伙伴出来玩耍……也许,他有三十年没吹过口哨了,现在忽然吹了起来……

后来他带着好奇的神情看了看窗外:啊,还亮着呢,夜晚还没有来临。然后他神经质地、高兴地说:

“谢谢,谢谢,一切都应该谢谢。”

后来他仿佛觉得,他就要死了,要倒下去了,但是他没有倒下,而是在房里踱了一会儿。后来他看了看放在桌上的白白的信,觉得这好像是空壳子,是皮壳,毒蛇已经从皮壳里爬了出来,于是他用手在腰上和胸膛上摸了摸。没有摸到毒蛇,已经爬进去,钻进去了,正在像火一样撕咬着心呢。

然后他站到窗口。司机们在朝着去上厕所的电话员姑娘玛露霞笑。军部坦克的一名机修员从井边提来一桶水。一群麻雀在房东家牛棚门口的一堆麦秸里刨来刨去找食儿。叶尼娅对他说过,麻雀是她喜欢的鸟儿……可是他浑身就像火烧一样,就像房子着了火:梁断,顶塌,橱子倒下,家什掉落,书籍、枕头像鸽子一般在烟火中翻筋斗……

“我将终身感谢你的纯洁与高尚,但是我有什么办法,过去的生活比我强大,无法把它消灭,无法忘记……不要责备我吧,不是因为我没有错,而是因为,不论我,不论你,都不知道我的错误在哪儿……原谅我吧,原谅我吧,我在哭,为咱们两个痛哭。”

这算什么?……

她还哭呢!他可是满腔愤怒。真是害人虫!毒蛇!要打她的嘴巴,打她的眼睛,拿手枪把子打断这母狗的鼻梁……可是转瞬间又异常突然地出现了一种无能为力的感觉,任何人、任何力量都不能帮助他,只有叶尼娅能,可是正是她,正是她害了他。于是他转脸朝着她应该从那边来看他的方向,说:

“叶尼娅,你怎么对我这样呀?叶尼娅,你听着,叶尼娅,你看看我,看看我成了什么样子啦。”

他向她伸过手去。

后来他想:为什么要这样呀,他已经毫无希望地等了这么多年了,不过她既然已经决定了,要知道她已经不是小姑娘,如果过了这么多年,后来决定了的话,就应该懂得,已经决定了呀。

过了几秒钟,他又在痛恨中寻求自我解救:“当然,当然,当我是一个代理少校,在荒山野岭上、在尼科利斯克—乌苏里斯克流浪的时候,她是不愿意的,等我做了军长,她愿意了,她是想做将军夫人,女人呀,女人,你们都是一样。”

他马上就看出这种想法的荒谬—不对,不对,要是这样倒好呢。因为她这一去,是回到那个人那儿去,那个人就要进劳改营,就要上科雷马去,她有什么富贵可言呢?……俄罗斯妇女呀,真是涅克拉索夫的诗:她不爱我,倒去爱他……不,不是爱他,是怜悯他,就是怜悯。为什么就不怜悯我?我现在比谁都不如,所有在卢比扬卡监狱里的、在所有劳改营里的、在所有军医院里的缺胳膊少腿的,都比我有福气,要是现在叫我进监狱,我连眉头都不皱一下,要是这样,你选谁呢?选他!他和你是一种气质的,我是另一种气质的,所以她管我叫“陌生人,陌生人”。当然,就算我做了元帅,总归还是粗汉子,矿工,没有文化的人,不懂她的见鬼的画儿……他大声地、恨之入骨地问:

“究竟为什么,为什么呀?”

他从后面的口袋里掏出手枪,在手里掂量了几下。

“我要自杀,不是因为我活不下去,是叫你痛苦一辈子,叫你一辈子……一辈子良心不得安宁。”

后来他把手枪收起来。

“过一个星期她就把我忘了。”

他也应该忘掉,想也不想,连头也不回!

他走到桌前,又看起信来。

“我的可怜的,亲爱的,我的好人!!!”可怕的不是无情,而是这些亲热的、心疼人、可怜人的话。这些话简直使人难受,甚至使人连气都不能喘。他仿佛看到了她的胸脯、肩膀、膝盖。她要去找那个可怜的克雷莫夫。

“我对自己毫无办法。”她在又挤又闷的车厢里,有人问她上哪儿去,她说:“去找丈夫。”她的眼神是亲切、温顺的,像狗眼一样,带有惆怅神气。

他在窗口望着,她是不是来找他了。两个肩膀哆嗦起来,鼻子哼哧起来,他叫起来,一面拼命憋着,压制着直往外冲的号哭。他想起来,他还叫人从方面军军需处给她弄来了巧克力糖、牛轧糖,还对维尔什科夫说过:“你要是动一动,我把你的头揪掉。”他又自言自语地说:“你看,我的亲爱的,我的叶尼娅,我有什么办法呀,你哪怕多少怜悯怜悯我也好。”

他很快地从床底下拖出手提箱,把叶尼娅的来信和照片拿出来,这里面有他多年来一直随身带着的照片,有最近一封信里寄的照片,有第一次给他的一张比身份证照片还小的包在玻璃纸里的照片。他用强劲有力的手指头撕起来。他又把她写的信撕成碎片,他从闪过的字里行间,从纸片上的残句,辨认着他读过几十遍的使他销魂的话,他看着她的脸、嘴巴、眼睛、脖子消失在撕碎的照片堆里。他撕得很急,很快。他越撕越感到轻松,就好像他一下子从身上把她揪了下来,把她踩得死死的,他摆脱了这个魔鬼。

他没有她也活了这么多年嘛。今后还是能活!一年后他从她身旁走过,心连跳都不会跳一下。“我才不稀罕你呢!”他一想到这一点,就感到自己想得很荒谬。心里的东西是揪不掉的,心不是纸做的,人生的一切不是用墨水记在心上的,不能把心撕成碎片,不能把印在脑子里和心中的多年的印象抹掉。

他已经使她成为他的工作、思想、灾难的参与者,成为他的刚强和软弱的见证人……

撕碎的信并没有消失,读过几十遍的话依然留在脑海里,她的眼睛依然从撕碎的照片上望着他。

他打开橱子,倒了满满一杯酒,喝干了,抽了一支烟,又抽起一支,虽然呛得厉害。头嗡嗡响起来,心里燥得难受。他又大声问道:“叶尼娅,亲爱的,心肝儿,你做的什么事呀,你做的什么事呀,你怎么能这样呀?”然后他把碎纸片装进提箱,把酒瓶放进橱子里,心里说,喝了酒,多少轻松些了。

……坦克很快就要进入顿巴斯,他就要:回到家乡,他要到父母的坟地上,让父亲看看有出息的小别佳,让母亲可怜可怜苦命的儿子。等战争结束,他就上哥哥家去,住在哥哥家里,侄女会说:“别佳叔叔,你怎么不说话呀?”

他忽然想起童年时候:他家有一条卷毛狗出去找狗交尾,回到家时被咬得浑身是伤,毛被撕掉许多,被咬掉了一只耳朵,头都肿了,眼睛肿成了一条缝儿,嘴也歪了,站在台阶前,丧气地耷拉着尾巴,爸爸朝狗看了看,很亲切地问:

“怎么,你做伴郎了吧?”

是的,他也做伴郎了……

维尔什科夫走了进来。

“上校同志,您在休息吗?”

“是的,多少休息一下。”

他看了看表,心想:“明天七点以前暂不推进。要用无线电密码通知下去。”

“我再到各旅去一趟。”他对维尔什科夫说。

汽车开得很快,多少分散了一些他的心思。吉普车现在的速度是每小时八十公里,路又很坏,汽车不住地颠簸,摇晃,蹦跳。

司机一再地感到害怕,用诉苦的眼神要求诺维科夫允许减低速度。

他走进马卡罗夫的旅部。短短的几个小时里一切变化有多大呀!马卡罗夫的变化又多大呀,就好像几年没有见面了。马卡罗夫忘记了行军礼,困惑不解地把两手一摊,说:

“上校同志,刚才格特马诺夫转发了方面军司令的命令:撤销休息一夜的命令,继续前进。”

五十二

三个星期之后,诺维科夫的坦克军调为方面军的后备军。这个军需要补充人员,修理机械。在战斗中前进了四百公里,人和机械都疲劳了。

接到调为后备军命令的同时,还接到一道命令,要诺维科夫上校去莫斯科,到总参谋部和高级指挥干部总部去,至于他以后是不是还回到坦克军,不十分清楚。

在他离开期间,暂时由涅乌多布诺夫少将代理军长职务。在这之前好几天,旅级政委格特马诺夫就得到消息,说党中央已决定在近期内把他从部队中调回去,要派他担任顿巴斯已经解放的一个州的州党委书记,党中央认为这一工作具有特别重要的意义。

召唤诺维科夫去莫斯科的命令,在方面军司令部和装甲部队总部引起不少议论。有些人说,这次召他去,没有任何特别的用意,诺维科夫在莫斯科待几天,就会回去继续当他的军长。有些人说,这事和诺维科夫在进军最紧张的时候发出休息十个小时的命令有关系,还和推迟几分钟率军进入突破口有关系。还有些人则认为,他和功劳很大的军政委与参谋长的工作关系没有搞好。

消息灵通的方面军军委秘书说,有人责备诺维科夫有不正当的男女关系。这位军委秘书有一段时间曾经认为,诺维科夫的问题就在于他和军政委的关系不协调。但是事实显然不是这样。这位军委秘书亲眼见过格特马诺夫写给最高层领导的信。格特马诺夫在信中表示反对撤销诺维科夫的军长职务,说诺维科夫是一名出色的指挥员,具有非同一般的军事才能,在政治方面和道德方面也是一个无可指责的人。

不过特别使人惊异的是,诺维科夫在接到召他去莫斯科的命令的那天夜里,在许多个痛苦不堪的不眠之夜之后,第一次安安稳稳地一觉睡到天亮。

五十三

似乎有一列轰轰隆隆的火车载着维克托在奔驰,一个人在火车里是难以设想家里的宁静的。时间变得紧密了,时间里填满了各种各样的事情、各种各样的人、电话铃声。有一天希沙科夫来到维克托家里,恭恭敬敬,盛情殷殷,一再问起身体健康,一再用开玩笑的亲热口吻解释,希望把过去的一切忘记,那一天似乎已经过去有十年之久了。

维克托原以为,那些拼命整他的人见到他会不好意思的,但是在他来研究所的那一天,他们却高高兴兴地和他打招呼,对直地看着他的眼睛,那目光充满了诚意和友情。特别使人惊异的是,这些人的确很真诚,他们现在的确对维克托一片好意。

他现在又听到评价他的著作的许多好话。马林科夫召见了他,带着关切的神情用聪明的黑眼睛注视着他,和他谈了四十分钟。维克托感到吃惊的是,马林科夫很了解他的研究情况,专业词汇运用得相当自如。

在告别时马林科夫说的话也使维克托感到惊异:

“如果我们在某种程度上干扰了您在理论物理方面的研究,我们会感到很难过。我们十分懂得:没有理论,就没有实践。”

他完全没有料到会听到这样的话。

在见过马林科夫的第二天,他看到希沙科夫那种不安的、请求的目光,想起那一次希沙科夫在家里召开会议,不请他施特鲁姆时那种懊恼和受辱的心情,都觉得奇怪。

马尔科夫又是那样和蔼可亲了,萨沃斯季扬诺夫又说起俏皮话讥讽人了。古列维奇来到实验室里,把维克托抱住,说:

“我多么高兴呀,我多么高兴呀,您真是福星本雅明[4]。”

火车还在载着他奔驰。

领导人征求维克托的意见,他是否认为有必要在原有实验室的基础上建立独立的研究机构。他还乘专机去过乌拉尔,陪他前去的是一位副人民委员。为他配备了专用小汽车,柳德米拉上配给商店可以坐小汽车,有时还顺便捎上几个星期之前尽量装做不认识她的那些妇女。

凡是以前似乎很复杂、很麻烦的事,现在办起来非常容易、非常顺手了。

年轻的兰杰斯曼十分感动:科甫琴科往家里给他打电话,杜宾科夫一个钟头的工夫就给他办妥了调入维克托的实验室的手续。

安娜·纳乌莫芙娜从喀山回来,对维克托说,她的调离手续两天的工夫就办妥了,来到莫斯科,科甫琴科还派小汽车到车站去接她。杜宾科夫书面通知安娜·斯捷潘诺芙娜,说决定恢复她的工作,并且说,已经和副所长谈妥,缺勤期间的工资全部补发。

新的工作人员每餐都受到款待。他们开玩笑说:“我们的全部工作可以归结为:从早到晚在内部食堂里转悠和吃。”可是,他们的工作当然不是在这方面。

实验室里安装起来的新设备,在维克托看来已经很不完善了。他想,再过一年,这些设备就会使人感到好笑,就像斯蒂芬森的火车头了。

维克托生活中发生的一切变化,似乎十分自然,同时又完全反常。事实上,维克托的研究确实是很重要、很有意义的,为什么不可以褒扬呢?兰杰斯曼也是一名有才能的科学家,他为什么不能在研究所工作呢?安娜·纳乌莫芙娜也是一名不可多得的人员,为什么让她在喀山闲待着呢?

同时维克托也明白,如果不是斯大林的电话,研究所里的人谁也不会称赞他的出色的研究成果,兰杰斯曼尽管有很高的才能,仍然会没有事干。

不过要知道,斯大林的电话也不是出自偶然,不是随心所欲、异想天开。要知道,斯大林就是国家,国家是不会随心所欲、异想天开的。

维克托以为,许多组织方面的事情,如招收新工作人员,做计划,定购仪器,召集会议,会占用他不少时间。但小汽车跑得很快,会议时间很短,开会也没有人迟到,他的意愿贯彻得很容易,上午最宝贵的时间他都可以用在实验室里。在这最重要的几个小时的工作时间里,他是完全自由的。没有任何人限制他,他可以想他感兴趣的事情。他的科学依然是他的科学。这完全不像果戈理的小说《肖像》中那位画家的情形。

谁也不敢侵犯他在科学方面的兴趣。以前他可是最害怕这一点。“我真正自由了。”他惊讶地想。

维克托不知为什么想起工程师阿尔捷列夫在喀山的议论,说军事工厂的原料、电力、机械都能及时得到供应,不存在拖沓问题。

维克托在心里说:“很明显,这种神话般的作风,这种没有官僚主义的作风,恰恰是官僚作风。为国家主要目的服务的事情,干起来就像开特别快车。官僚主义的力量有两个相反的方面:它既能阻止任何运动,又能加给运动非同寻常的速度,甚至可以飞出地球引力范围之外。”

但是他现在不再常常想起在喀山的小屋里晚间的闲谈了,就是想起来心里也泰然,他觉得马季亚罗夫也不是多么出众、多么聪明的人了。现在他不再老是担心马季亚罗夫的命运,不再老是想到卡里莫夫害怕马季亚罗夫,马季亚罗夫害怕卡里莫夫了。

一切事情不知不觉似乎变成很自然的,合情合理的。维克托过的日子成为常规。维克托渐渐习惯了这种日子。以前过的日子似乎成了例外。维克托对以前那种日子渐渐生疏了。阿尔捷列夫的看法未必对吧?

以前他一走进人事处,看到杜宾科夫看他的目光,就要生气,就要发急。可是杜宾科夫现在却成了一个又热心又和善的人。

他打电话给维克托,常说:

“我是杜宾科夫,想麻烦您。维克托·帕夫洛维奇,我打扰您了吧?”

他本来觉得科甫琴科是一个两面三刀、心狠手辣、见到谁害谁的阴谋家,是奉行秘密的不成文规则、丝毫不顾工作真正实质的官僚。谁知,科甫琴科也有一些完全不同的特点。他每天都要上维克托的实验室里走一走,十分平易近人,很有一副民主作风,常常和安娜·纳乌莫芙娜开开玩笑,见了人都要握手问好,有时和钳工、机械师们聊一聊,说他年轻时候就在车间里做过旋工。

维克托多年来一直不喜欢希沙科夫。有一次他应邀上希沙科夫家吃饭,希沙科夫却原来是一个十分热情好客的人,还是一个美食家,又会说俏皮话和笑话,又有上等白兰地,还是一位版画收藏家。更主要的,原来他还是维克托的理论的崇拜者。

“我胜利了。”维克托在心里说。但是他当然也明白,他取得的不是最高的胜利,跟他有关系的人改变了对他的态度,不再阻碍他,而是帮助起他来,这决不是因为他的聪明、天才或者别的什么本领征服了他们。

不过他总归是高兴的。他胜利了!

几乎每天晚上广播电台都要播送“最新消息”。苏军攻势不断扩展。维克托现在觉得,把自己生活的必然变化同战争的必然进程,同人民、军队、国家的胜利联系在一起,是很简单、很容易的了。

但是他明白,不是那么简单的,不能简单地嘲笑自己一心只想看到“这儿是斯大林,那儿也是斯大林,斯大林万岁”这种简单明了的情形。

本来他认为,行政领导人和党的活动家们就是在自己家里天天谈的也是干部的纯洁问题,天天用红笔批文件,对自己的老婆朗读《联共党史简明教程》,连做梦也要梦到暂行条例和必守法令。

维克托却一下子又看到这些人带有人情味的另一面。

党委书记拉姆斯科夫原来是一个喜欢钓鱼的人,战前他常常和妻子、儿子一起坐小船在乌拉尔的一些河上游玩。

“嘿,维克托·帕夫洛维奇,”他说,“黎明时候上河边去,露水亮晶晶的,河边的沙子凉丝丝的,把钓丝抖搂开来,河水还是郁郁的,毫无声息,等着你垂钓……真是人生莫大的乐事。等战争结束了,我吸收你参加钓鱼协会。”

科甫琴科有一次和维克托谈起儿科疾病。使维克托吃惊的是,他知道许多治疗佝偻病和咽峡炎的方法。原来,他除了有两个亲生的儿子以外,还收养了一个西班牙孩子。西班牙孩子常常生病,他常常自己给孩子治病。

甚至没有什么人情味的斯维琴也对维克托说起他搜集的一些仙人掌,甚至在寒冷的一九四一年冬天都没有冻死。

维克托心想:“啊,这些人实在不是多么坏。每个人都有人情味儿。”

当然维克托在内心深处也明白这些变化是怎么一回事儿,知道实际上什么也没有变化。他不是糊涂虫,他不是犬儒主义者,他会思考。

在这些日子里他想起克雷莫夫说的他的老同志巴格良诺夫的事。巴格良诺夫原是军事检察院的侦讯长,一九三七年被捕,在一九三九年短短的别里耶夫自由化时期从劳改营里放出来,回到莫斯科。

克雷莫夫说了说巴格良诺夫那天夜里怎样从车站径直来,到他家,穿着破衬衣、破裤子,口袋里装着劳改营的释放证。那天夜里他说了不少热爱自由的话,同情所有劳改营里的人,准备今后做一个养蜂人和园林工作者。

但是,他的生活渐渐恢复了原来的样子,他的腔调也渐渐变了。

克雷莫夫笑着说了说巴格良诺夫的思想怎样渐渐地、一步一步地变化。不久,他的军装发还给他了,这个时期他的想法还是符合自由主义观点的,不过他已经不像丹东那样义正词严地揭露残酷的事了。

可是终于他的劳改营释放证换成了莫斯科的居民身份证。马上就可以感觉出他想踏上黑格尔的立场:“一切存在的即是合理的。”后来还了他住房,他说起话来就完全不同了,他说,在劳改营里有不少判刑的人是犯了叛国罪。后来发还了他的勋章。后来恢复了他的党籍和党龄。

恰好在这时候,克雷莫夫在党内遇到不快的事。巴格良诺夫就再也不给他打电话了。有一天克雷莫夫在外面碰到他。他从停在苏联检察院门前的一辆小汽车里走出来,军装领子上添了两个菱形的领章。那天夜里他穿着破烂衣衫、揣着释放证坐在克雷莫夫家里,说许多人无辜被判刑,说使用暴力十分荒唐,这时候才过了八个月。

“那天夜里我听了他的话,还以为他永远不再进检察院的大门了呢。”克雷莫夫冷笑说。

当然,维克托想起这件事,并且对娜佳和柳德米拉说了说,不是无缘无故的。

他对死于一九三七年的人的态度丝毫没有变。他依然害怕斯大林的残酷。

一个维克托成为成功的弃儿还是幸运儿,人们的生活不会变化;死于集体化时期的人、一九三七年被枪毙的人,不会因为某一个维克托得不得勋章和奖章,不会因为马林科夫召见他或者没有把他列入希沙科夫的邀请名单而复活。

这一切维克托十分理解,也牢牢记着。不过在这种理解和记忆中也出现了新东西……他常常对妻子说:

“有多少没出息的人呀!许多人多么怕挺起腰来做正直的人,多么容易屈服,多么容易妥协,多么卑鄙可怜。”

他有一次甚至带着责备的心情想到契贝任:

“他过分热衷于旅游和爬山运动,正是他下意识地害怕生活的复杂性;他离开研究所,则是他有意识地害怕面对我们生活中的主要问题。”

当然,他还是有所变化的,他感觉出这一点,但却不明白,究竟变化的是什么。

五十四

维克托恢复上班之后,没有在实验室里碰到过索科洛夫。在维克托来上班之前两天,索科洛夫害了肺炎。

维克托听说,索科洛夫在害病之前和希沙科夫谈妥了自己的工作问题。索科洛夫被任命为一个新组建的实验室的主任。总之,索科洛夫还是一帆风顺的。

至于索科洛夫为什么要求所领导把他调出维克托的实验室,就连无所不知的马尔科夫也不知道真正的原因。维克托听说索科洛夫要离开,也不觉得难过和惋惜。倒是一想到和他见面,和他一起工作,就觉得沉重。如果见了面,他有什么眼神,索科洛夫看不到呀。当然,他无权像以前那样老是想着朋友的妻子。他无权思恋她。他无权和她秘密约会。

如果有人向他说起类似的事,他会感到十分愤慨。因为这是欺骗妻子!欺骗朋友!可是他还在思念她,盼望和她会面。

柳德米拉已经和玛利亚恢复了来往。她们先在电话里表白了很长时间,后来见了面,又哭又各自检讨,说自己太糊涂,不应该怀疑和不信任朋友。

天啊,生活多么复杂,多么难以理解呀!玛利亚,真诚而纯洁的玛利亚却没有以真情对待柳德米拉,昧了良心!不过她这样做是为了她对他的爱情!

现在维克托很少见到玛利亚了。他所知道的有关她的事,差不多都是柳德米拉对他说的。

他听说,索科洛夫因为在战前发表的著作,被推荐为斯大林奖金备选人。他听说,索科洛夫收到英国年轻的物理学家一封热情洋溢的信。他听说,索科洛夫将在不久就要举行的科学院选举中被选为通讯院士。这一点是玛利亚对柳德米拉说的。他自己有时和玛利亚短时间见面,现在不谈索科洛夫了。

工作上的操心、会议、出差都不能消除他经常的苦闷,他时时盼望和她见面。柳德米拉对他说过好几次:

“我真不懂,索科洛夫为什么对你这样反感。就连玛利亚也对我解释不清楚。”

要解释是很简单的,不过玛利亚当然不能认真地向柳德米拉解释。她对丈夫说了自己对维克托的感情,已经够受的了。

这种表白永远破坏了维克托与索科洛夫的关系。她已经向丈夫保证不再跟维克托相会。玛利亚哪怕对柳德米拉露出一句,他将会很长时间对她的什么情况都不知道,不知道她在哪儿,她怎么样了。要知道,他们过去会面太少了,而且每次会面又是那样短暂!每次会面他们很少说话,只是手挽着手在街上走走,或者一声不响地在街心公园的凳子上坐坐。

在他遭遇挫折和倒霉的时候,她以特别敏锐的感情理解他所遭遇的一切。她能猜出他的思想,能猜出他的行动,甚至好像她事先能够知道他将遇到的一切。他心里越是痛苦,想见到她的愿望就越是强烈,越是迫切。他觉得,他今天的幸福就在于这种完全与充分的理解。似乎,有玛利亚和他在一起,他就很容易战胜自己的一切痛苦。他和她在一起就是幸福的。

在喀山有一天夜里他们说过话儿,在莫斯科他们在逍遥公园溜达过一次,有一次还在卡卢加大街的街心公园的凳子上坐了几分钟—说实在的,不过就是这些。而且这都是在过去。就算加上现在的事:他们通过几次电话,有几次他们在街上遇见,再加上这几次短时间的见面,他都没有对柳德米拉说。

但是他明白,他的过错和她的过错不能用他们暗地里在长凳子上坐的时间来衡量。他的过错不小:他爱她。为什么她在他的生活中占据了这样大的地盘?

他对妻子说的每一句话,都只有一半真实。每一个举动,每一瞥目光,都不由得带上了虚假成分。他有时装做漫不经心地问柳德米拉:

“喂,怎么样,你的好朋友给你来电话了吗?她怎么样?索科洛夫身体好吗?”

他听说索科洛夫一帆风顺,十分高兴。但他高兴不是因为他对索科洛夫一片好心。而是不知为什么他觉得,只要索科洛夫一切顺利,玛利亚就可以不受良心责备了。

从柳德米拉口里打听索科洛夫和玛利亚的情形,是一件很不痛快的事。这对于柳德米拉,对于玛利亚,对于他,都是一种污辱。

但是,他在和柳德米拉谈到托里亚,谈到娜佳,谈到弗拉基米罗芙娜的时候,也是真话中夹杂着假话,到处有虚假。为什么,是什么原因?他对玛利亚的感情,的的确确是他心灵、思想、心意的真实情形。为什么这种真实却产生了这么多的虚假?他知道,他一旦抛开这种感情,就会使柳德米拉,使玛利亚,使自己摆脱虚假。但是,就在他觉得应该抛开他无权享受的爱情的时刻,却有一种不安分的感情,害怕痛苦,搅乱思想,一个劲儿地劝他:“这种虚假并不是那么可怕,对谁都没有什么害处。痛苦比虚假可怕得多呢。”

有时他觉得,他会有力量、有狠心和柳德米拉离婚,拆散索科洛夫的家庭,这时他的感情就推动着他,用完全相反的方式欺骗他的思想:

“要知道,虚假是顶要不得的,还不如和柳德米拉离婚,只要不再对她说假话,也可以不再让玛利亚说假话。虚假比痛苦更可怕!”

他没有觉察,他的思想已经成为他的感情的驯顺的奴仆,感情在牵着思想走,要想走出这转来转去的圈子只有一条出路:忍痛斩断情丝,牺牲自己,而不是牺牲别人。

他对这一切想得越多,越是理不出头绪。他对玛利亚的爱情竟不是他生活中的真情,而造成他生活中的虚假,这怎么能理解,怎么能弄清楚!去年夏天他和标致的尼娜有一段浪漫史,那不是中学生的浪漫史。他和尼娜不仅是在街心公园里散散步。但是,背叛的感觉、家庭不幸的感觉、对不起柳德米拉的感觉,他却是现在才有。

他在这些事情上花费了很多心思、精力和激情,看起来,普朗克创立量子论花费的力气也不会少。有一段时间他认为,他只是因为受挫折和倒霉,才产生了这种爱情……若非如此,他不会有这样的感情……

但是他现在功成名就了,希望看到玛利亚的心情却没有减弱。

她是一种特殊气质的女子,不爱金钱、荣华和权势。她一直希望和他共度灾难、痛苦和穷困……于是他担心起来:现在他一切好转了,她会不会不再理睬他呢?

他明白,玛利亚把索科洛夫奉若神明。就这一点也使他十分难受。

也许,叶尼娅说的话是对的。像这种第二次爱情,是婚后生活多年之后产生的,它确实是精神维生素缺乏的结果。就比如老牛很喜欢舔盐,因为牛一年到头在青草、干草和树叶中找不到盐。这种精神饥饿渐渐增长,就会产生很大的力量。过去是这样,现在也是这样。啊,他可是知道自己的精神饥饿是什么滋味……玛利亚和柳德米拉太不一样了。

他的一些想法是真实的,还是虚假的?维克托没有注意到,一些想法不是出自理智,决定他的行动的不是这些想法的正确与否。他已经不受理智的支配。他看不到玛利亚,就觉得痛苦;一想到可以见到她,就觉得幸福。

有时他想象他们会在一起永不分离,就觉得无限幸福,为什么他想到索科洛夫,不觉得良心有愧?他为什么不觉得羞惭?

是的,有什么羞惭的?不过只是在逍遥公园里走了走,在长凳上坐了坐。

啊,为什么要在长凳上坐呀!他还想和柳德米拉离婚,他还想对自己的朋友说,他爱他的妻子,他想把她夺过来。

他想起他和柳德米拉的生活中一切不好的事情。他想起柳德米拉对他的妈妈怎样不好。他想起柳德米拉不让他从劳改营回来的堂兄在家里过夜。他想起她的冷酷、粗暴、执拗、无情。

他一想起这些不好的地方,就心狠起来。要干冷酷的事,只要心狠就行。不过柳德米拉和他过了一辈子,一直和他同甘苦,共患难。柳德米拉已经白了头发。她受过许多苦。难道她光是不好的吗?要知道,多少年来他一直因为有她而感到自豪,喜欢她的正直和诚实。是的,是的,他是曾经打算干冷酷的事。

早晨,维克托正准备上班的时候,想起不久前叶尼娅来过,就想道:“叶尼娅走了,上古比雪夫去了,这样倒是好。”他想到这里,觉得不好意思起来,就在这时候柳德米拉说:

“在我们家坐牢的人当中,又增加了一个克雷莫夫。好在叶尼娅现在不在莫斯科。”

他本想责备她说这种话,但是忽然想起刚才自己所想的,就没有作声,因为他觉得,如果责备她,他就太虚伪了。

“契贝任给你来过电话。”柳德米拉说。

他看了看表。

“晚上我早点儿回来,再给他打电话吧。另外,可能我又要乘飞机上乌拉尔去。”

“要去很久吗?”

“不。只待两三天。”

他急着要走,今天是很重要的一天。

他的研究很重要,许多事情很重要,都是国家的事情,但他个人的思想似乎被反比例定律支配着,是渺小、卑微、微不足道的。

叶尼娅临走的时候,请求姐姐常到库兹涅茨桥去看看,送给克雷莫夫二百卢布。

“柳德米拉,”他说,“你应该把叶尼娅叫你转交的钱送去了,可能你已经错过了接待日期。”

他说这话,并不是因为他在为克雷莫夫和叶尼娅操心。他说这话,是因为他想到,柳德米拉这样不重视所托,可能会促使叶尼娅很快地再上莫斯科来。叶尼娅再来莫斯科,就要开始写申诉书,写信,打电话,把维克托的家变成在监狱和检察院活动的基地。

维克托明白,这些想法不仅是渺小、卑微的,也是可鄙的。他想到这里,感到不好意思,就连忙说:

“你给叶尼娅写封信,就说你和我都请她上莫斯科来。也许,她现在很需要上莫斯科来,可是没有邀请,她不好意思来。你听见吗,柳德米拉?马上就给她写!”

他说过这话之后,感到轻松了,但是他又知道,他说这番话为的是自我安慰……说来实在奇怪。当他坐在自己的房间里,没人理睬,又怕房管员又怕票证处的姑娘的时候,他的头脑里想的是人生、真理、自由、上帝……那时候谁也不需要他,电话铃一连几个星期都不响,熟人在街上碰见都不和他打招呼。可是现在,当几十个人在等着他,又给他写信,又给他打电话,小汽车的喇叭在窗外轻轻响着的时候,他却再也摆脱不了一些空泛无聊的想法、卑微的烦恼、庸人的担心。不是担心说错了话,就是担心笑得不是地方,总是有一些微乎其微、庸俗无聊的想法伴随着他。

在斯大林打过电话之后,有一段时间他觉得他今后可以完全不必害怕了。可是结果他还是在害怕,只是这害怕不同了,不再是平民的害怕,而是贵族的了—可以坐汽车,可以往克里姆林宫打电话,但害怕还是害怕。

对别人的学术成就抱嫉妒的、运动员式的态度—原来似乎是不可能的,现在变成很自然的事了。他在担心:别人会不会超过他,会不会纠正他的错误?

他不太愿意和契贝任交谈,似乎没有力量进行长久的、花费力气的谈话。他还是把科学对国家的依赖关系想象得太简单。因为他确实是自由的嘛:现在谁也不认为他的理论体系是学究式的毫无意义的东西了。现在谁也不敢扼杀他的理论体系了。国家需要物理学理论。现在这一点希沙科夫明白了,巴季因也明白了。为了让马尔科夫在试验方面,让科契库罗夫在实践方面表现出他们的本事,就需要有理论家做后台。在斯大林打过电话之后,所有的人都一下子明白了这一点。怎么向契贝任解释,是斯大林的电话使他在研究中得到了自由呢?可是他为什么对于柳德米拉的缺点不能容忍了呢?可是他为什么对待希沙科夫这样和善呢?

他现在很喜欢马尔科夫。领导人的私事,一些秘密的和半秘密的情况,一些无伤大雅的手腕和非同儿戏的阴谋诡计,是否被邀参加主席团而引起的喜悦或懊恼,有谁进入某些特别名单或者在名单中没有名字—他对这一切都有了兴趣,他的的确确关心起这些事。

也许,他现在宁愿花一个晚上和马尔科夫闲扯,也不愿像在喀山那样和马季亚罗夫认真探讨。

马尔科夫极善于发现一些人的可笑之处,毫无恶意地同时又十分辛辣地嘲笑一些人的弱点。他具有文学才能,同时又是一流的科学家,也许,他是国内最有才华的物理试验工作者。

维克托已经穿好大衣,柳德米拉说:

“玛利亚昨天来过电话。”

他很快地问:

“什么事?”

显然,他的脸色都变了。

“你怎么啦?”柳德米拉问道。

“没什么,没什么。”他说着,从走廊回到房间里。

“说实在的,我也不明白,究竟有什么不愉快的事。大概是科甫琴科往他们家里打过电话。总而言之,她还和以往一样替你担心,怕你又惹出什么事儿。”

“究竟怎么一回事儿?”他焦急地问道。“我真不明白。”

“我不是说了嘛,我也不明白。看样子,她是觉得在电话里说起来不方便。”

“好吧,那你就再说一遍。”他说着,解开大衣,坐到门口的一张椅子上。

柳德米拉看着他,摇了摇头。他觉得,她的眼睛带着责难和伤心的神情看着他。她好像证实他这种感觉,说:

“瞧,维克托,你说早晨给契贝任打个电话都没有时间,可是一听说玛利亚,就有时间听了……甚至还走了回来。已经不早啦。”

他侧着眼睛朝上看了看她,说:

“是的,我要迟到了。”

他走到妻子跟前,握住她的手亲了亲。她抚摩了几下他的后脑勺,轻轻地理了理他的头发。

“瞧,现在玛利亚多么重要,多么叫人感兴趣,”柳德米拉小声说,又凄然笑了笑,说,“还说她分不清巴尔扎克和福楼拜呢。”

他看了看:她的眼睛湿润了,他觉得她的嘴唇好像也在哆嗦。

他无可奈何地把两手一摊,走到门口又回头看了看。

她脸上的表情使他吃了一惊。他一面下楼一面想,如果他和柳德米拉离了婚,今后再也不见面了,那么,她脸上这种表情,这种无可奈何的、痛苦、感人,为他也为自己羞臊的表情,将永远不会从他的脑海里消失,直到生命的最后一天。他明白,这几分钟里发生了十分重要的事,妻子让他知道,她看出了他对玛利亚·伊凡诺芙娜的爱情,他也证实了这一点……

他还知道一点。他看到玛利亚,就觉得幸福,如果他觉得他再也看不到她了,他就连气也不能喘了。

等维克托的汽车渐渐来到研究所,希沙科夫的小汽车也跟了上来,两部小汽车几乎同时在大门口停下来。

他们并肩在走廊里走着,就像刚才他们的汽车并排行驶一样。希沙科夫挽住维克托的胳膊,问道:

“就是说,您要乘飞机外出吗?”

维克托回答说:

“看样子,要出去一趟。”

“很快咱们就要永远分手了。您现在相当于一位国家元首了。”希沙科夫开玩笑说。

维克托忽然想:

“如果我问他,您爱过别人的妻子吗,他会说什么?”

“维克托·帕夫洛维奇,”希沙科夫说,“您是否得便,在两点左右上我这儿来一下?”

“到两点钟我就没有事了。遵命。”

这一天他工作很不顺利。

在实验厅里,马尔科夫不穿外衣,挽着衬衣袖子,走到维克托跟前,很起劲地说:

“维克托·帕夫洛维奇,如果您有时间,等会儿我上您的办公室去。有一件很有意思的事和你说说。”

“我在两点钟要到希沙科夫那儿去,”维克托说,“您迟一点儿来吧。我也有一点儿事要和您说说。”

“您在两点钟要上希沙科夫那儿去吗?”马尔科夫反问一句,又沉思了一会儿,说:“可能我猜到了,他要找您干什么。”

五十五

希沙科夫一看到维克托,就说:

“我已经想打电话给您,提醒您呢。”

维克托看了看表。

“我觉得,我没有迟到呀。”

希沙科夫站在他面前,又肥又大,穿着讲究的灰色西服,满头银发的大脑袋。但是维克托觉得希沙科夫的眼睛里已经没有冷淡和倨傲的神气了,这是一个读了大仲马和里德的不少小说的小孩子的眼睛。

“亲爱的维克托·帕夫洛维奇,今天我请您来,有一件特别的事,”希沙科夫笑着说,并且拉住维克托的手,把他拉到椅子跟前,“是一件很重大的、不太愉快的事。”

“站着谈吧,天天坐得太多了。”维克托说着,用烦闷的目光打量了一下这位肥大院士的办公室。

“咱们就来谈谈不愉快的事吧。”

“是这样的,”希沙科夫说,“在国外,主要是在英国,发动了一场卑鄙的运动。我们担负着战争的主要重担,可是英国的科学家们并不要求尽快开辟第二战场,却展开了一场极其奇怪的运动,煽动敌视我们国家的情绪。”

他看了看维克托的眼睛,维克托知道那是一种毫无掩饰的、直露的目光,那是有些人要做坏事时的目光。

“是的,是的,是的,”维克托说,“可是,究竟是一场什么样的运动?”

“一场诽谤运动,”希沙科夫说,“他们公布了一份据说是我国被杀害的科学家和作家的名单,报道了因为政治问题被镇压者的离奇数字。他们怀着不可理解的,也可以说是不可告人的用心,想推翻经过侦查和判定的普列特尼奥夫和列夫医生害死马克西姆·高尔基的罪行。这一切都发表在接近政府人士的一家报纸上。”

“是的,是的,是的,”维克托一连说了三遍,“还有什么吗?”

“基本上就是这些。还提到遗传学家切特韦里科夫,组织了一个保护他的委员会。”

“希沙科夫同志,”维克托说,“可是,切特韦里科夫确实被捕了呀。”

希沙科夫耸了耸肩膀。

“维克托·帕夫洛维奇,您知道,我没有过问过保安机关的工作。不过,如果他确实被捕了,那显然是因为他犯了罪。你和我总是没有被捕呀。”

这时候巴季因和科甫琴科走进办公室。维克托明白,希沙科夫是在等他们,显然事先他已经和他们商量过了。他甚至没有对刚进来的两个人解释正在谈的是什么,只是说“请吧,请吧,两位同志,请坐”,就又接着对维克托说:

“维克托·帕夫洛维奇,这些无稽之谈又传到了美国,刊登到《纽约时报》上,这自然引起苏联知识界的愤慨。”

“当然啦,不可能不愤慨。”科甫琴科用十分亲切的目光看着维克托的眼睛,说。

他那栗色眼睛的眼神是那样亲热,以至于维克托很自然地产生的一种想法也说不出口了:“苏联知识分子根本就看不到《纽约时报》,怎么会愤慨呢?”

维克托耸了耸肩膀,嗯了两声,这些动作可以被理解为他赞同希沙科夫和科甫琴科的说法。

“很自然,”希沙科夫说,“在我们知识界出现了一种愿望,对这种卑鄙的诽谤给予应有的回击。我们起草了一份文件。”

“哼,你什么也没有起草,是别人起草的。”维克托在心里说。

希沙科夫又说:

“这份文件是用书信的形式。”

这时巴季因小声说:

“我看过这份文件,写得很好,写的都是应该说的话。签名的人不要多,应该是我国最大的一些科学家,具有全欧洲和全世界名望的。”

维克托一听到希沙科夫开头的几句话,就明白了谈话的目的。他只是不知道希沙科夫究竟要他干什么:在学术委员会会议上发言,写文章,还是参与发表声明?现在他明白了:要他在公开信上签名。

恶心的感觉向他袭来。他像在那一次要他检讨的会议之前那样,又感觉到自己的可怜而卑贱的实质。

有几百万吨岩石就要朝他的头上压下来……普列特尼奥夫教授呀!维克托立即想起《真理报》上报道一个女人歇斯底里地控诉这位老医生进行肮脏活动的文章。

如往常一样,报纸刊登的事就成了事实。显然,读了不少托尔斯泰、契诃夫和柯罗连科的书,使人们养成了对俄罗斯文字几乎奉若神明的态度。但是终于有一天,维克托清清楚楚看出来,报纸在说谎,普列特尼奥夫教授受到了诽谤。

过了不久,普列特尼奥夫和克里姆林医院的著名内科医生列文就被捕,并且供认害死了马克西姆·高尔基。

三个人都望着维克托。他们的目光是亲切、和蔼、充满信心的。他是自己人嘛。希沙科夫已经像兄弟般地承认了他的著作的伟大意义。科甫琴科也把他看得很高。巴季因的眼睛好像在说:“是的,我对您做的事情原来是很反感的。但是我错了。我不懂。党已经纠正了我的错误。”科甫琴科打开红色公文夹,把打字机打好的公开信递给维克托。

“维克托·帕夫洛维奇,”他说,“应该告诉您,英国人和美国人发动的这场运动,是直接为法西斯效劳的。可能这是第五纵队的间谍策动的。”

巴季因插话说:

“干吗还要向维克托·帕夫洛维奇进行宣传?他和咱们都一样,有一颗苏联爱国者的心。”

“当然,”希沙科夫说,“正是这样。”

“谁又能怀疑这一点呢?”科甫琴科说。

“是的,是的,是的。”维克托说。

最奇怪的是,这几个人不久前对他又鄙视又不放心,现在却对他又信任又亲热,这种信任和亲热显然极其自然,而且他虽然一直记着他们对他的残酷,却很自然地接受了他们的友好感情。

就是这种友情和信任束缚着他,剥夺了他的力量。

假如他们大声呵斥他,用脚踢他,打他,也许他会大吼起来,会刚强些的……

斯大林和他通过电话。现在和他坐在一起的几个人都记得这一点。

可是,天啊,他们要他签名的这封信多么可怕呀。这封信关系到多么可怕的事呀。

他实在无法相信普列特尼奥夫教授和列文大夫会杀害伟大的高尔基。他妈妈来莫斯科的时候找列文看过病,柳德米拉更是常常在他那儿治病,他是一个很聪明、很细心、很和善的人。诬陷这样两位医生的人,有多么残忍?

这种诬陷是中世纪黑暗的再现。医生竟成了杀人犯!医生竟害死伟大的作家,害死最后一位俄罗斯文学大师。谁需要这种血腥的诬陷?这是迫害异己,是宗教审判的火堆,就像杀害异教徒,又是烟,又是恶臭,像烧开的焦油。这一切怎么能和列宁,和社会主义建设,和伟大的反法西斯战争相联系呢?

他拿起公开信的第一页。

希沙科夫问他,站着是不是舒服,光线行不行,是不是坐到椅子上?不用,不用,很舒服,谢谢。他看得很慢。把一个一个的字塞进脑子,脑子却不能吸收,就像要把沙子塞进苹果里。他看到:

“你们袒护人类的败类和不肖之徒、玷辱了崇高的医生称号的普列特尼奥夫和列文,是在助长法西斯仇恨人类的思想。”

他又看到:

“苏联人民英勇地在同法西斯进行战斗,是法西斯在用新的形式推行中世纪的迫害异己、民族大洗劫、宗教审判的火刑、刑讯和拷打。”

我的天啊,怎么能不叫人发疯呀。

他又往下看:

“我们的子弟在斯大林格勒流的血,取得了反法西斯战争的转折,你们却有意无意地在袒护第五纵队的间谍……”

是的,是的,是的。

“我们的科学工作者受到人民和政府的无比爱护和关怀,这是世界上任何一个国家都没有的。”

“维克托·帕夫洛维奇,我们在这儿说话,不妨碍您吧?”

“不,不,没关系。”维克托说。他心里想:“有些人很幸运,或者能够开开玩笑把事情敷衍过去,或者这会儿正在别墅里度假,或者在生病,或者……”

科甫琴科说:

“我听说,斯大林同志知道这封信,很赞成我们科学家的这一行动。”

“所以才要维克托·帕夫洛维奇签名呢……”巴季因说。

维克托感到苦恼,感到厌恶,感到自己就要屈服。他感触到伟大国家的亲切气息,他没有力量投身寒冷的黑渊……今天他没有,实在没有力量。使他就范的不是恐惧,而是另外一种消磨力量的温顺感情。

人是多么奇怪、多么令人吃惊的造物呀!他有力量去死,却没有足够的力量拒绝甜饼和冰糖。

如果有一只手抚摩你的头,拍你的肩膀,那手就成了无敌的手,你再也无力把它推开。

胡说,为什么要诬蔑自己?他要甜饼和冰糖干什么?他对生活条件和物质享受一直很平淡。他的见解、他的著作、他一生最珍贵的东西在反法西斯战争时期成为有用的、可贵的。这确实就是幸福!

而且,说实在的,这究竟是怎么一回事儿呢?他们都在预审中承认了呀。他们在法庭上也供认了。他们已经承认害死了伟大的作家,怎么能相信他们无罪?

拒绝在公开信上签名吗?那就是同情杀害高尔基的凶手!不,不可能。怀疑他们招供的真实性吗?就是说,那是强迫的!可是强迫一个正直而善良的知识分子承认自己是雇佣的杀人凶手并因而换得死刑和可耻的名声,只有用拷打的办法。然而,这样的怀疑,即使有一丝一毫,那也是神经错乱。

不过,在这种卑劣的信上签名,那是令人厌恶,令人作呕的。在他的头脑中出现了一些话和对这些话的回答……

“同志们,我有病,我的冠状动脉痉挛。”

“胡说,想借口生病来逃避呢,您脸上的气色挺好嘛。”

“同志们,干吗要我签名,我只是在很小的专家圈子里有些名气,国外很少有人知道我。”

“胡说!(听到这个“胡说”十分快活)都知道您,还不光是知道呢!而且没有您的签名,这信就没有意义,也无法让斯大林同志看,他会问:为什么没有施特鲁姆的签名?”

“同志们,我直截了当对你们说吧,我觉得某些说法不够妥当,会给我们整个科学界造成不好的影响。”

“维克托·帕夫洛维奇,请,请,请您提出具体意见,我们很高兴修改您认为不妥当的说法。”

“同志们,要理解我的意思,比如,你们在这儿写的:人民的敌人巴别尔,人民的敌人、作家皮利尼亚克,人民的敌人瓦维洛夫院士,人民的敌人、演员梅耶霍德……不过我是一个物理学家,数学家,是从事理论研究的,有些人认为我精神失常,因为我研究的领域太抽象。说实在的,我是不够格的,最好还是不提这些人吧,因为这些事我一点儿也不明白。”

“维克托·帕夫洛维奇,您不要客气吧。您十分善于分析政治问题,您的逻辑性很强,您该记得,有多少次您说到政治方面的问题,说得何等深刻呀。”

“啊,天呀!你们要知道,我还有良心呀,我很痛心,我很难过,再说,也不是非我不可,为什么非要我签名不行,我太痛苦了,让我的良心享受一点儿安宁吧。”

可是马上又变得软弱无力,不由自主,出现了喂饱了和受宠的牲畜那种驯顺的感情,怕生活又受到新的摧残,怕又一次担惊受怕。

这是怎么回事儿?又要把自己放到大家的对立面?又要冷清孤单?应该认真对待生活了。他已经得到连想也不敢想的东西。他现在能自由地从事自己的研究,受到无比的关怀与照顾。而且他也没有祈求,没有检讨。他是胜利者!他还要什么呢?斯大林都亲自给他打了电话呀!

“同志们,这事关系重大,我希望多少想一想,最好等明天再决定。”他又在心中说。

他马上又想象到:这样他会一夜不眠,痛苦,焦虑,犹豫不决,突然下决心,又因为下了决心而害怕,又犹豫不决,又下决心。这一切折腾起人来,就像凶恶、无情的疟疾。是他自己要把这种折磨延长若干小时。他已经没有力气了。快点儿,快点儿,快点儿吧。

他掏出自来水笔。

他马上看出来,希沙科夫看到他这个顶不随和的人今天这样随和,都惊愕得发了呆。

整整一天维克托没有进行研究。谁也没打搅他,谁也没给他打电话。是他自己不能进行研究。他不能进行研究,是因为这一天他觉得研究工作枯燥、空洞、毫无意思。

有哪些人在公开信上签了名?契贝任签名吗?约费签过名吗?克雷洛夫是否签过名?曼德尔施塔姆呢?他真想躲到什么人背后去。不过,拒绝签名是不可能的。那就等于自杀。啊,根本不是这么回事儿。也可以拒绝嘛。不,不,都有道理。因为谁也没有威胁他。如果他是因为像畜生一样害怕而签了名,那倒是轻松些。可是他签名不是因为害怕呀。是因为有一种愚昧、令人恶心的驯顺感情。

维克托把安娜·斯捷潘诺夫娜叫到自己的办公室里来,请她明天把新设备上进行的试验的一组胶片洗出来。

她记下来了,却依然坐着没有走。

他用询问的目光看了看她。

“维克托·帕夫洛维奇,”她说,“我以前认为,言语是表达不出心情的,可是现在我想说说:您可明白,您的所作所为对于我和其他一些人有什么样的意义?这对于人们来说,比一切伟大的发明都重要。就因为您活在世界上,一想到这一点,心里就觉得幸福。您可知道,钳工们、清洁工和门卫人员是怎么说您的?都说您是一个正派人。我多次想上您家里去,可是我怕。您要知道,我在最困难的日子里一想到您,心里就觉得轻松,觉得安宁。谢谢您,就因为有您。您是人!”

他什么也没有来得及说,她就很快地走出了办公室。他想跑到街上去,狂跑,狂叫……因为他太痛心,太羞愧。不过,痛心和羞愧还不止这些,这只是开头。快到下班的时候,电话铃响起来。

“您听出来了吗?”

天啊,还问他是不是听出来呢。不仅是耳朵,就连握着话筒、顿时紧张起来的手指头也听出这声音了。这是玛利亚又在他最难受的时刻出现了。

“我是在公用电话亭子里打电话,声音很不清楚,”玛利亚说,“我丈夫身体好些了,我现在时间多一些了。如果可以的话,明天八点钟还上那个街心公园来。”

她忽然说:

“我亲爱的,我的心上人!我真替您担心呀。有人带着一封公开信上我家来,噢,您明白我说的是什么吧?我相信,这是您,是您的刚强帮助我丈夫顶住了,我们一切都还平平安安。可是我马上想到,您这一下子要惹出麻烦来了。您性格那样倔强,有时候别人会碰一个疙瘩,您就会碰得粉身碎骨。”

他挂起话筒,用两手把脸捂住。他已经明白自己处境之可怕:今天不是敌人在残酷地折磨他。是亲近的一些人在折磨他,用的刑具是他们对他的无比信任。

他回到家里,连大衣也没有脱,就给契贝任打电话。柳德米拉站在他面前,他在拨契贝任家的电话号码,他相信,断然相信,他的朋友和老师也会因为喜欢他,使他受到无情的创伤。他急急匆匆,甚至来不及对柳德米拉说说在公开信上签名的事。天啊,柳德米拉的头发白得多么快呀。是的,是的,真不应该,不能再让她伤心了!

“好消息不少,都看到战报,”契贝任说,“不过我没有什么了不起的事。噢,今天我和几位可敬的人士吵了一场。您可听说一封什么公开信了吗?”

维克托舔了舔发燥的嘴唇,说:

“是的,听说一点点儿。”

“好啦,好啦,我明白,这种事儿不好在电话里说,等您回来之后,咱们见了面再说说吧。”契贝任说。

嗯,好吧,好吧,不过,还有娜佳,她马上也要回来了。天啊,天啊,他干的是什么事……

五十六

夜里,维克托睡不着。他心里太痛苦了。这种可怕的苦恼是从哪儿来的?真是沉重的负担,沉重的负担。还胜利者呢!

他在害怕房管所的普通办事员的时候,比现在要刚强些,自由些。今天他甚至都不敢进行争论,不敢表示怀疑。他成为胜利者之后,便失去了心意的自由。他怎么好意思见契贝任呀?也许,他见了契贝任会泰然自若,就像他回到研究所那一天许多快快活活、亲亲热热迎接他的一些人那样?

这一夜他想到的一切,都使他伤心,使他难过,使他不得安宁。他的笑、他的动作表情、他的行动都和他自己格格不入,都和他作对。今天晚上娜佳的眼睛里有一种怜悯和憎恶的神情。

只有经常使他气愤、经常顶撞他的柳德米拉听他说过以后,马上就说:

“维克托,不应该难过。我觉得你最聪明,最实在。既然你已经这样做了,就是说,应该这样。”

为什么他现在愿意承认一切、肯定一切呢?为什么不久前他不能容忍的事现在可以容忍了呢?不论和他谈什么,他都用乐观的态度看待。

军事上的胜利与他个人命运的转折是一致的。他看到军队的强大、国家的强盛、前途的光明。为什么他今天觉得马季亚罗夫的一些说法如此浅薄无味?

在他被抛出研究所,他拒绝检讨的那一天,他心里有多么坦然,多么轻松。在那些日子里,亲人就是他的莫大幸福:柳德米拉、娜佳、契贝任、叶尼娅……啊,见了玛利亚,他对她怎么说呢?他一向那样瞧不起胆小的索科洛夫,瞧不起他的顺从和听话。可是今天呢?他怕去想母亲,他在她面前有愧。他很怕再拿起她最后一封信。他又害怕又苦恼地了解到,他已经无力保卫自己的灵魂,无法使灵魂不受侵蚀。他本身正在滋长一种力量,这种力量渐渐使他成为奴隶。

他干了很卑鄙的事!他看着许多不幸的、血肉模糊的人软弱无力地倒下去,他还要朝他们投石头。

因为揪心的痛苦,因为剧烈的折磨,他的额头上渗出了汗珠。

他有什么理由感到自负?他有什么权利在别人面前夸耀自己的纯洁和勇气?他有什么权利评论别人,不原谅别人的弱点?

渺小的人和高尚的人都有不足之处。他们的区别在于:渺小的人做了好事,就要夸耀一辈子;高尚的人做了好事,一点也不注意,而长期记在心里的是他所做的坏事。

可是他却常常夸耀自己的勇敢和正直,讥笑别人的软弱和怯懦。可是现在他把很多人出卖了。他鄙视自己,他为自己感到羞臊。他的家,他的光明和温暖,都化为灰烬,化为齑粉。

他和契贝任的友谊、对女儿的疼爱、对妻子的感情、对玛利亚的无希望的爱情、他个人的幸福与不幸、他的著作、他的心爱的科学、他对母亲的爱和对她的悼念—一齐从他的心中消失了。

他为什么要犯这样可怕的罪过?世界上的一切与他所失去的东西相比,是微不足道的,不论是从太平洋岸直到黑海岸的辽阔大国,还是科学,与一个小小人物的正直与纯洁相比,都是微不足道的。

他清楚地看到,现在还不晚,他还有力量抬起头来,做自己的母亲的好儿子。

他不想寻求安慰,不想为自己辩护。就让他所做的这件卑鄙下贱的坏事永远成为对他的责难吧。让他终生时时刻刻记着吧。一个人应该不是一心想着去干什么大事,不是要以这样的大事作为骄傲和夸耀的资本。不是,不是,不是!

年复一年,每天,每时每刻都需要进行斗争,保卫自己做人的权利,保持纯洁与善良的权利。在这种斗争中既不需要骄傲,也不需要虚荣,需要的只有搏斗。如果在可怕的时期出现了毫无希望的时刻,一个人就不应该怕死,如果还想做一个人的话,就不应该怕。

“好吧,咱们就试试吧,”他说,“也许,我还有足够的力量。妈妈,妈妈,这是你的力量。”

五十七

卢比扬卡附近村庄里的一个又一个夜晚……

克雷莫夫被审讯之后,躺在床上,呻吟着,想着,和卡茨涅林鲍肯说着话儿。

原来克雷莫夫觉得布哈林和雷科夫的招供、加米涅夫和季诺维耶夫的招供、托洛茨基派、右倾或左倾中央的案件过程、布勃诺夫和穆拉洛夫以及什里亚普尼科夫的遭遇都是不可思议的,现在他觉得都是可以想象的了。从革命的活的机体上把皮撕下来,新时期想用革命的皮来打扮自己,而把无产阶级革命的带血的肌肉和热腾腾的心肝抛进垃圾堆里,因为新时期不需要这些。需要的只是革命的皮,所以把这张皮从活人身上剥下来。披上革命的皮的人便说起革命的话,做起革命的动作,但是脑子、肺、肝、眼睛却是另外一种人的。

斯大林!伟大的斯大林!也许,最有权势的一些人正是最没有主见的人。是时代和环境的奴隶,是当今的驯服而恭顺的奴仆,见到新时期来了,就恭恭敬敬地打开大门。

是的,是的,是的……见了新时期不低头的人,就要进垃圾堆。

现在他知道是怎样摧毁一个人了。搜身,剪掉纽扣,拿走眼镜,这样使一个人产生身体不值钱的感觉。到了侦讯室里,一个人会感到自己参加革命、参加国内战争根本不算什么,自己的知识和自己的工作更是不值一提。就是说,这是第二步:叫你知道不仅是身体不值钱。

而对于那些坚持继续做人的人,就进行百般折磨,一直要把人的体力和精力都弄垮,使人服服帖帖,毫无反抗之力,直到使人既不盼望正义,又不盼望自由,也不盼望安宁,只是盼望早日了结已经使人十分痛恨的人生。

审讯工作几乎总是取胜的过程,就在于肉体的人和精神的人是一致的。精神和肉体是互相沟通的,进攻的一方只要击溃和突破人的肉体防线,就能使机动兵力进入突破口,控制精神,迫使人无条件投降。

他没有力量想这一切,也没有力量不想这一切。究竟是谁出卖他?谁密告他?谁诬陷他?他觉得他现在对这个问题没有多大兴趣了。

他一向自以为得意的,是他能使自己的生活服从理性。可是现在不是这样了。理性说,他和托洛茨基的谈话情形是叶尼娅告密的。可是他现在整个的生活、他和侦讯员周旋、他还能够呼吸、他依然是克雷莫夫同志,其支撑点就是相信叶尼娅不可能干这种事。有一小会儿他竟会对此失去信心,他都感到奇怪。没有什么力量能够使他不相信叶尼娅。尽管他知道,除了叶尼娅,谁也不知道他和托洛茨基的谈话,尽管他知道女人容易变心,女人是软弱的,尽管他知道叶尼娅已经扔掉他,在他一生最艰难的时候离开了他,他还是相信。

他把审讯的经过对卡茨涅林鲍肯说了说,但是只字未提这件事。

卡茨涅林鲍肯现在不开玩笑,也不扮鬼脸了。

确实克雷莫夫没有把他看错。他是很聪明的。但是他说的一切都很可怕、很奇怪。有时候克雷莫夫觉得,把这个老肃反工作人员关进内部监狱,没有什么不应该的。不可能不这样。有时克雷莫夫觉得他是一个疯子。

这是国家保安机关的诗人和歌手。

他有一次用赞赏的口气对克雷莫夫说,上次开党代会上,休息的时候斯大林问叶若夫,为什么他在执行肃反政策上犯了扩大化的错误,张皇失措的叶若夫回答说,他是执行斯大林的直接指示的,斯大林就对着围住他的代表们很忧郁地说:“这也是一名党员说的。”

他还说了说亚戈达遇到的可怕的事……

他还说起肃反部门的一些大人物,他们懂得伏尔泰,知道拉伯雷,敬仰魏尔兰,当年都在这座日夜不眠的大房子里做过领导工作。

他还说过一个在莫斯科干了多年刽子手的一个很可爱、很老实的拉脱维亚老头子,这个老头子在行刑的时候,常常要求把就刑的人的衣服脱下来,交给保育院。他又说了另一个行刑者的事。那个人日日夜夜地喝酒,没有活儿干就十分苦闷,在没有派到他杀人的时候,他就到莫斯科附近的国营农场去杀猪,把猪血装在瓶子里带回来,说是医生叫他喝猪血治贫血病。

他向他描述,在一九三七年每天夜里怎样对判定所谓剥夺通信自由的人执行死刑,夜里莫斯科焚尸炉的烟囱怎样冒浓烟,被动员参加行刑和抬运尸体的共青团员们怎样一个个疯了。

他说了说怎样审讯布哈林,加米涅夫多么倔强。有一天夜里他和克雷莫夫一直谈到天亮。

这天夜里,这名肃反工作人员发展和丰富了他的理论。

卡茨涅林鲍肯对克雷莫夫描述了新经济政策时期的新资产阶级分子弗伦克尔的不寻常遭遇。弗伦克尔在实行新经济政策初期在奥德萨建立了发动机工厂。在二十年代中期他被逮捕并被押送到索洛韦茨基群岛上。他在索洛韦茨基劳改营里的时候,向斯大林提供了一份天才的方案。这个老肃反工作人员在这里用的字眼儿就是“天才的”。

他在这份方案中用大量经济学和技术方面的数据论证了如何利用成千上万的犯人修建道路、堤坝、水电站,开凿运河。

这位被囚禁的新资产阶级分子便成了克格勃的中将,因为当家的十分看重他的想法。

二十世纪忽然闯入简单劳动时期,这种被神圣化的劳动实际是囚犯连队的劳动和旧式的苦役劳动,是锹、镐、斧头和锯子的劳动。

劳改营世界也开始吸收现代文明,也使用电力机车、自动升降机、推土机、电锯、涡轮机、割矿机、大量的汽车和拖拉机。劳改营世界装备了运输和联络飞机、无线电联络和通讯系统、自动车床、现代化的选矿系统。劳改营世界设计、规划、建造矿井、工厂、新的海洋、宏伟的水电站。

劳改营世界发展十分迅速,并存的旧的苦役式劳动显得很可笑,很好玩儿,就像孩子们的拼图方块。

但是,卡茨涅林鲍肯说,劳改营还是跟不上现实的发展,因为现实不断地向劳改营提供人力。有许多学者和专家还是派不上用场,他们和技术与医务没有任何关系……

有一些全世界知名的历史学家、数学家、天文学家、文学评论家、地理学家、世界美术研究专家、研究梵文和古凯尔特语的学者,在劳改营系统都派不上什么用场。劳改营的发展还不够,还不能利用这些人的特长。他们干的是粗活儿,或者在事务工作方面和文教科做一些所谓笨活儿,或者在残废营里闲待着,根本无法运用他们的知识,他们的知识往往是极其渊博的,不仅在苏联,而且在全世界都得到极高的评价。

克雷莫夫听着卡茨涅林鲍肯不停地说,就好像一位学者在介绍自己一生的主要事业。他不仅是歌颂和赞美。他还是个研究者。他进行比较,揭示缺点和矛盾,联系,对照。

在劳改营外面也存在着缺陷,当然,其形式是不那样明显的。在现实生活中有不少人做的不是他们能做的工作,不是发挥其所长,在各个大学、各个编辑部、科学院各研究所都有这类现象。

卡茨涅林鲍肯说,在劳改营里,刑事犯统治着政治犯。刑事犯又霸道,又野蛮,又懒惰,又贪财,动不动就不要命地打架、抢夺,阻碍着劳改营劳动生活和文化生活的发展。他接着说,就是在劳改营铁丝网里面,科学家和著名文化界人士的工作也要由不学无术、无能和见识短浅的人领导。劳改营好像是外面社会的扩大而加强的映像。不过铁丝网内外的现实不是相反的,而是符合对称定律的。

他接着又说起来,不过不是像一位歌手,也不像一位思想家,而是像一位预言家了。

如果勇敢而连续不断地推进劳改营制度的发展,排除阻力和缺陷,这种发展必将导致界线的消灭。劳改营就会同外面的社会融为一体。这种融合,这样消灭了劳改营与外面社会的对立,就是伟大原则的成熟和胜利。劳改营制度虽然有种种缺陷,但也有一个起决定作用的优点。只有在劳改营里,最高原则,也就是理性,能够毫不掩饰地反对个人自由原则。理性可以使劳改营高度发展,高度发展就可以创造条件使其自我消灭,与乡村和城市的生活融为一体。

卡茨涅林鲍肯担任过劳改营设计院的领导。他认为,科学家和工程师们能够在劳改营的条件下解决最复杂的问题。他们能够解决世界科学技术思想方面的任何问题。只要能很好地领导他们,为他们创造较好的物质条件就行。有一种古老的说法,说是没有自由就没有科学,是完全不可信的。

“等到两方面水平接近了,”他说,“我们就可以宣布铁丝网里面和外面的生活相等了,就用不着关押人了,我们就不必再发逮捕证了。我们只建立监狱和政治隔离所,文教处就可以对付任何不合常规的人。到那时候就会出现意想不到的太平局面。”

取消劳改营将是人道主义的胜利。同时所谓个人自由这种乱七八糟的、原始的、穴居时代的原则在这之后也不会占上风,不会猖獗起来。相反,这种原则倒是可以完全消除。

在沉默了很久之后,他又说,也许,几百年之后,这种制度会自行消灭,在这种制度自行消灭过程中,渐渐产生民主和个人自由。

“世界上没有什么东西是永远存在的,”他说,“但是我不希望生活在那样的时代。”

克雷莫夫对他说:

“您的一些想法是极不正常的。据说,一些精神病医生在精神病医院里工作时间久了,自己的精神也会不正常。请原谅我这样说。不过,您在这里面待得太久,不是没有影响的。卡茨涅林鲍肯同志,您把保安机关看成了上帝。确实应该把您撤换下来。”

卡茨涅林鲍肯很和善地点了点头,说:

“是的,我相信上帝。我是一个信神的愚昧的老头子。每一个时代都要依照自己的面貌创造一个上帝。保安机关是明智和强有力的,保安机关统治着二十世纪的人类。过去这样的力量,人类曾经奉若神明的力量,就是地震、雷电、森林大火。现在不光是把我关起来,而且把您也关起来了。也应该把您给撤换了。总有一天会弄清楚,究竟是您说得对,还是我说得对。”

“可是德列林克老头子现在回去了,回劳改营去了。”克雷莫夫说。

他知道这话会引起反应的。果然,卡茨涅林鲍肯说:

“就是这个可恶的老头子搅乱了我的信仰。”

五十八

克雷莫夫听到声音不高的说话声:

“刚才广播说,我军击溃了斯大林格勒的德国集团军群,好像把保卢斯抓住了,说实话,我没有听清楚。”

克雷莫夫叫喊起来,挣扎起来,两脚在地上乱动,想走到穿棉军装和毡靴的人群中去……人群的那种亲切的嚷嚷声淹没了旁边正在进行的不高的谈话声;格列科夫从斯大林格勒的瓦砾堆里摇摇晃晃地朝着克雷莫夫走来。

医生抓住克雷莫夫的手,说:

“应该休息一下……再注射一针樟脑剂,脉搏每跳四下都要停一下。”

克雷莫夫把咸咸的一团东西吞下去,说:

“没什么,继续进行吧,医生认为没有关系嘛,我反正不招认。”

“你会招认的,你会招认的,”侦讯员用工厂老技师那种和善而自信的口吻说,“有许多比您更硬的人都招认了。”

这第二次审讯过了三个昼夜之后结束了。克雷莫夫又回到囚室里。

值班守卫把一个白布包着的小包放到他身边。

“喂,犯人,请在转交单据上签个名。”他说。

克雷莫夫看了看转交物品的清单,清单上的字迹十分熟悉:葱,蒜,糖,白面包干。清单下面写着:“你的叶尼娅。”

天啊,天啊,他哭了……

五十九

一九四三年四月一日,斯皮里多诺夫接到苏联电力委员部的撤换工作的通知;要他交出斯大林格勒发电站的工作,前往乌拉尔,到一座不大的、用泥炭发电的发电站去担任站长。这处分不算重,因为本来也可以送交法庭的。斯皮里多诺夫在家里没有说起电力委员部这道命令,决定再等州党委的决定。四月十四日,州党委因为他在艰难的日子里擅离职守,给予他严重警告处分。这项决定也算很宽容的,因为本来也可以把他开除出党。但是斯皮里多诺夫觉得州党委做出这样的决定是很不应该的,因为州委的同志们都知道,他一直坚持到斯大林格勒保卫战的最后一天,他是在苏军已经开始进攻的那一天上左岸去的,他是为了去看看在船舱里分娩的女儿。在州党委的会议上他本想分辩一下,可是普里亚欣非常严肃,说:

“您可以向中央监察委员会上诉,我估计,什基里亚托夫同志会认为州党委的决定太宽容,太姑息。”

斯皮里多诺夫说:

“我相信,中央监察委员会会取消这种决定。”

但是,因为他听到不少有关什基里亚托夫的事情,他还是有点儿怕提出上诉。

他担心和怀疑的是,普里亚欣的面孔那样严肃,不仅是和斯大林格勒发电站的事有关系。普里亚欣当然记得,斯皮里多诺夫与叶尼娅和克雷莫夫有亲戚关系,他自然不喜欢一个知道他和坐牢的克雷莫夫有多年关系的人。

在这种情况下,即使普里亚欣想帮助斯皮里多诺夫,也不能帮助了。假如他这样做了,对他不友好的人(有权势的人周围总会有不友好的人)马上就会向有关部门反映,说普里亚欣因为同情人民敌人克雷莫夫,竟帮助克雷莫夫的亲戚、怕死的斯皮里多诺夫。

但是,很明显,普里亚欣不帮助斯皮里多诺夫,不仅是因为他不能,而是因为他不愿意。显然,普里亚欣知道,克雷莫夫的岳母已经来到斯大林格勒发电站,正住在斯皮里多诺夫家里。大概普里亚欣也知道,叶尼娅常和母亲通信,不久前还寄来自己给斯大林的申诉书的底稿。

在州党委会议散会之后,斯皮里多诺夫到小卖部去买乳酪和香肠,在这里碰见州保安局局长沃罗宁。沃罗宁带着好笑的神气看了看他,并且用好笑的口吻说:

“斯皮里多诺夫真是一个天生的好当家,刚刚受过严重警告处分,就做起家务事来啦。”

“一家人要吃饭呀,有什么办法,我现在做外公啦。”斯皮里多诺夫说着,笑了笑,是一种苦笑,无可奈何的笑。

沃罗宁也对他笑了笑,说:

“我以为你准备办移交呢。”

斯皮里多诺夫听了这话,心里想:“幸亏把我赶到乌拉尔去,要不然在这儿就完了。薇拉和小孩子怎么办呀?”

他搭吨半载重汽车回斯大林格勒发电站,透过驾驶室的模糊的玻璃望着他就要离开的被战争摧毁的城市。他想着,在战前他的妻子就是走这条如今已是堆满瓦砾的人行道去上班;他想着供电网,想着等到从斯维尔德洛夫斯克运来新电缆,他已经不在斯大林格勒发电站了;想着小外孙因为营养不足,胳膊和胸前出了很多小疙瘩。他想道:“严重警告就严重警告好了,有什么了不起?”他想,不会发给他“保卫斯大林格勒”奖章的,不知为什么一想到奖章他就非常伤心,其伤心的程度竟超过离别这座他长期生活、工作,流着泪安葬了玛露霞的城市。他甚至因为得不到奖章懊恼得大声骂起来,所以司机问他:

“斯皮里多诺夫同志,您这是骂谁?是不是有什么东西忘在州党委啦?”

“是的,我忘记了,”斯皮里多诺夫说,“可是它没有忘记我。”

斯皮里多诺夫家几个房间里又冷又潮湿。代替炸掉的窗玻璃的是胶合板和木板。墙上的石灰有很多地方脱落了。饮用水要用桶提上三层楼。房间里生火的是用铁皮做的小炉子。有一个房间暂时关上不用,厨房也没有用,眼下成了放木柴和土豆的仓房。

斯皮里多诺夫、薇拉和小孩子、在他们回来之后便从喀山赶来的弗拉基米罗芙娜,住在原来做餐室的大房间里。原来薇拉住的紧靠厨房的小房间里住着安德列耶夫老头子。

本来斯皮里多诺夫可以修修天花板,粉粉墙壁,砌两座砖炉,发电站里还有干这种事的一些工人师傅,材料也是有的。

但是不知为什么一向操心家事、果断干练的斯皮里多诺夫不愿意请人做这些事情。

显然,薇拉和弗拉基米罗芙娜也觉得住在战后残破的家宅里更舒服些,因为战前的生活已经毁灭,为什么要让屋子恢复原来的样子,又使人想起一去不再返的生活?

弗拉基米罗芙娜来了之后,又过了几天,安德列耶夫的儿媳妇娜塔莉亚也从列宁斯克来了。她在列宁斯克和已故的婆婆的妹妹吵了一架,又把儿子暂时丢给她,就上斯大林格勒发电站来找公公。

安德列耶夫一看到儿媳,就生起气来,对她说:

“你以前和你婆婆吵,现在又和她的妹妹吵。你怎么能把孩子丢在那儿呀?”

看样子,娜塔莉亚在列宁斯克过的日子十分艰难。她一走进安德列耶夫住的房间,打量了一下天花板、墙壁,就说:“这儿太好了!”虽然这儿一点儿也没有什么好的:天花板上的板条子已经露了出来,角落里还堆着石灰,烟囱已经不成样子。

窗户上堵了一块胶合板,上面嵌了一小块玻璃片,房间里的光线就是透过玻璃片进来的。

从这自制的小窗户望出去,一片凄惨景象:到处是断垣残壁,有红颜色的,也有蓝颜色的,还有破烂的铁皮屋顶。

弗拉基米罗芙娜一来到斯大林格勒,就生起病来。她因为生病,暂时没有上城里去。她很想去看看她那烧毁的房子。

最初几天,她克制着病痛,帮薇拉做事情:生炉子,洗尿片,在炉子的铁皮烟囱上烘尿片,把脱落的石灰搬到楼梯平台上,甚至还尝试过从下面往上提水。

但是她的病情越来越重,在烧得很暖和的房间里她会觉得冷,在很冷的厨房里她的额头会冒出汗来。

她想硬撑过去,不说自己有病。但是有一天早晨,她上厨房里去抱木柴,却一下子昏迷过去,倒在地板上,把头都跌破了。斯皮里多诺夫和薇拉把她搀到床上躺下来。

弗拉基米罗芙娜苏醒过来以后,把薇拉叫到床前,说:

“你要知道,我在喀山在柳德米拉家里过的日子不如在你们家里。我上这儿来,不光是为了你,也为了我自己。我只是怕,我躺在这儿不能动,会把你累坏。”

“外婆,我有你在这儿就很好。”薇拉说。

可是薇拉确实感到十分艰难。水,木柴,牛奶,一切东西都要花很大力气才能弄来。外面的阳光已经有了暖意,可是房间里又冷又潮湿,不得不把炉子烧旺些。

小米佳的胃有毛病,夜里常常哭,妈妈的奶也不够他吃。薇拉一天到晚在房间和厨房里忙活,要不然就是出去买牛奶和面包,洗锅洗碗,从下面往上提水。她的两手泡得红肿,脸也被风吹红了,而且出现了冻斑。因为劳累,因为天天活儿干不完,她心中无时无刻不感到阴雨和沉重。她不梳头,很少洗脸,也不照镜子,生活的重担把她压坏了。她时时刻刻非常想睡觉。到晚上,胳膊、腿、肩膀都酸疼,很想休息。她一躺下,米佳就哭。她就爬起来,走过去喂奶,把尿片换一换,抱起来在房间里走一走。过一个钟头,他又哭起来,她就又爬起来。天蒙蒙亮,他就醒来,再也不睡了,于是她就在朦胧的晨曦中又开始了新的一天—不等睡够,便脑袋昏昏沉沉地上厨房里抱柴,生炉子,烧开水,准备给爸爸和外婆泡茶,开始洗衣服。但奇怪的是,她现在一点也不发脾气了,变得又和善又有耐性。

娜塔莉亚从列宁斯克来了以后,薇拉的日子轻松些了。

娜塔莉亚来了以后,安德列耶夫便上斯大林格勒北部的拖拉机厂工人村去住了几天。也许是他想看看发电站和自己的房子,也许是因为儿媳妇把孩子丢在列宁斯克,生她的气,也许是他不愿意让她吃斯皮里多诺夫家的粮食,所以走的时候把他的供应卡给她留下了。

娜塔莉亚不等休息过来,在来到的那一天就动手帮薇拉的忙。

啊,她干起活儿多么轻快、有劲儿,年轻的手一干起活儿,那沉甸甸的水桶、盛满了水的煮衣锅、满口袋的煤炭全都变轻了。

现在薇拉可以抱着孩子上外面玩一会儿了,可以在石头上坐坐,看看那闪闪发光的春水,看看草原上升起的蜃气。

四周静悄悄的。战场已经移到几百公里之外。似乎德军飞机在空中嗡嗡直叫,炮弹不停地爆炸,生活中充满了火、恐惧和希望的时候,心里倒是轻松些。

薇拉看着小孩子满脸的脓疙瘩,心疼起来。她同时也怜惜起维克托罗夫。上帝,上帝,苦命的万尼亚,生一个儿子竟是这样瘦,这样虚弱,这样爱哭。

然后她踏上到处是垃圾和碎砖的楼梯,上了三楼,干起活儿,她的苦恼便沉没在忙碌中,沉没在浑浊的肥皂水中,沉没在炉子的灰烟里,沉没在墙壁散发的潮气中。

外婆把她叫到床前,抚摩着她的头发,外婆平时那安详又明亮的眼睛里出现了异常悲痛和温柔的神情。薇拉没有跟任何人谈起过维克托罗夫,没有跟爸爸谈,没有跟外婆谈,甚至也没有对五个月的米佳说过。

娜塔莉亚来到以后,房间里的一切都变了样子。她刮掉墙上的霉斑,把发黑的墙角都粉刷了,地板上有些脏东西就像长在上面似的,她都擦洗干净了。她还进行了一次大规模的清扫,本来薇拉准备等天暖和了再干的—她把一层一层楼上的垃圾全部清除了。

下午,她又把长长的黑蟒蛇似的烟囱收拾好了。烟囱本来歪歪扭扭,接缝处不住地往下滴松脂色的脏水,滴得地板上一个一个的小水洼儿。娜塔莉亚在烟囱上涂了石灰,又把烟囱抻直了,用铁丝捆上,在接缝处挂了几个空罐头筒,脏水就往里面滴。

她来的第一天,就和弗拉基米罗芙娜很要好了,虽然她好像是一个爱吵爱闹的泼辣女子,还喜欢说男女之间的粗野话,应该不是弗拉基米罗芙娜喜欢的人。娜塔莉亚很快就认识了许多人,有线路工人,有涡轮房里的工人,有载重汽车的司机。

有一次,娜塔莉亚去站队买东西刚刚回来,弗拉基米罗芙娜对她说:

“娜塔莉亚,有一位同志问你来着,是一位军人。”

“是一个格鲁吉亚人吧?”娜塔莉亚问道。“他要是再来,您把他撵走。大鼻子鬼,想向我求婚呢。”

“这么着急?”弗拉基米罗芙娜惊讶地问。

“您以为他们能沉得住气吗?他要我在战后上格鲁吉亚去呢。我把楼梯擦洗得干干净净,难道是为了跟着他走?”

晚上她对薇拉说:

“咱们上城里去,今天有电影。司机米沙用汽车送咱们去。你带小孩子坐在驾驶室里,我可以在车厢里。”

薇拉摇了摇头。

“你去吧,”弗拉基米罗芙娜说,“我的身体要是好一些,我也跟你们去了。”

“不去,不去,我怎么也不能去。”

娜塔莉亚说:

“还是要好好地过下去呀,要不然咱们都成了鳏夫和寡妇了。”

然后她又带着责备的口气说:

“你天天待在家里,哪儿也不想去,你也没有把爸爸照应好。我昨天洗衣服,他的衬衣和袜子都很破了。”

薇拉抱起孩子,走到厨房里。

“米佳,你说说,你妈妈不是寡妇吗?……”她问。

斯皮里多诺夫这些天十分关心岳母,两次从城里请来医生给她看病,帮薇拉给她拔火罐,有时把水果糖塞到她手里,说:

“您不要给薇拉,我已经给她吃过了,这是留在橱子里专门给您的。”

弗拉基米罗芙娜明白,女婿有很不愉快的事,心里很苦闷。但是每次她问他州党委方面是不是有什么消息,他总是摇摇头,说起别的事情。只有那一天晚上,当他接到通知,说即将处理他的问题的时候,他回到家里,挨着岳母在床坐下来,说:

“我这都怎么搞的呀,假如玛露霞知道我的事情,会发疯的。”

“他们究竟说你有什么错儿?”岳母问。

“全是错。”他说。

这时候娜塔莉亚和薇拉走了进来,谈话就中断了。弗拉基米罗芙娜望着娜塔莉亚,心想,是有这样一种刚健而顽强的美,任何艰难的生活对这种美都无可奈何。娜塔莉亚的一切都很美,不论是脖子,青春的胸脯,还是腿,几乎露到肩膀的匀称的手臂。弗拉基米罗芙娜心想:“真是一位没学过哲学的哲学家。”她常常发现,有一些没有过惯贫苦日子的女子,一遇到艰难的环境就憔悴下来,不再注意自己的容貌,像薇拉就是这样。她很喜欢那些做季节工的姑娘们,那些干重活儿的女工,军事调度员姑娘们,她们住在棚子里,在灰土和泥水中干活儿,却还要烫发,照镜子,往脱了皮的鼻子上搽粉。有些顽强的鸟儿就是在刮风下雨的天气,也要不顾一切地唱自己的歌儿。

斯皮里多诺夫也望着娜塔莉亚,后来突然抓住薇拉的手,把她拉到怀里,搂住她,好像请求原谅似的,吻了吻她。

弗拉基米罗芙娜也好像没头没脑地说:

“有什么了不起的,斯捷潘,咱们死还早着呢!就连我这个老婆子还想把身体养好,在世上多活几年呢。”

他很快地看了看她,笑了。这时娜塔莉亚往脚盆里倒了不少热水,端到床前,跪下来,说:

“弗拉基米罗芙娜,我给你洗洗脚,现在屋里很暖和。”

“你疯啦!傻瓜!快起来!”弗拉基米罗芙娜叫道。

六 十

有一天下午,安德列耶夫从拖拉机厂工人村回来了。

他走进屋里,一看到弗拉基米罗芙娜,他那忧郁的脸笑了—这些天她第一次起了床,脸色还很苍白,还很消瘦,坐在桌旁,戴起了眼镜,正在看书。

他说,他很久都找不到他的房子原来所在的地方,到处是战壕,炸弹坑一个连着一个,到处是碎瓦片和坑洼。

工厂里已经有很多人,每时每刻都有人回来,甚至民警也有了。参加民兵队的人还没有什么消息。大家都在掩埋士兵,埋好了,又不断地发现还有死人,有的是在地下室里,有的是在战壕里。到处是碎钢片,废铁……

弗拉基米罗芙娜问他,他上那儿去是不是很难走,他在哪儿睡的,怎么弄到吃的,炼钢炉破坏得是不是很厉害,工人们有没有东西吃,他是不是见过厂长。

上午,在安德列耶夫回来之前,弗拉基米罗芙娜对薇拉说:

“我平时常常讥笑预感和迷信,可是今天我平生第一次肯定无疑地预感到,安德列耶夫会带来谢廖沙的消息。”

可是,她错了。

安德列耶夫说的事情是很重要的,不管听他说的人是幸福的还是不幸的。工人们对安德列耶夫说:没有东西吃,也不发工资,地下室和土室里又冷又潮湿。厂长变成了另一个人,当初德国佬向斯大林格勒进攻的时候,他在车间里跟工人们亲热得不得了,现在连话也不愿意说了,他的房子已经修好了,还从萨拉托夫弄来了小汽车。

“现在发电站情况也很差,不过没有什么人恼恨站长,很明显,大家不好过,他也不好过。”

“他是很不痛快呀。”弗拉基米罗芙娜说。“老人家,你打算怎么办?”

“我是来告别的,我想回家,虽然家也没有了。我在公共宿舍里找了个地方,在一个地下室里。”

“很对,很对,”弗拉基米罗芙娜说,“不论怎么样,总算是在家里。”

“这是我挖出来的。”他说着,从口袋里掏出一个生了锈的顶针。

“不久我也要进城,上果戈理大街去,看看自己的家,翻翻碎瓦断砖,”弗拉基米罗芙娜说,“真想回家呀。”

“你现在起床是不是早了一点儿,你的脸色还很苍白。”

“我听到你说的一些事,十分难受。真希望在这块神圣的土地上的一切是另一种样子。”

他咳嗽了几声。

“您该记得,斯大林在前年说:兄弟姐妹们……可是现在,打败了德国人,就连厂长的小院子不通报也别想进去,兄弟姐妹们却住在土室里。”

“是啊,是啊,这种状况是不大好。”弗拉基米罗芙娜说。“唉,谢廖沙还是一点音信也没有。”

晚上,斯皮里多诺夫从城里回来。早上他上城里去的时候,没有对任何人说州党委要处理他的问题。

“安德列耶夫回来了吗?”他生硬地操着厂长的口气问道。“谢廖沙没有什么消息吗?”

弗拉基米罗芙娜摇了摇头。

薇拉一下子就看出来,爸爸醉得很厉害。从他开门的猛劲儿,从他那拼命忽闪的难过的眼睛,从他把带回来的东西往桌子上放的那股神气,脱大衣的样子,问问题的口气,都可以看出这一点。

他走到睡在衣服篮子里的米佳跟前,俯下身来。

“你不要朝着他呼酒气。”薇拉说。

“没关系,让他受点儿训练。”斯皮里多诺夫快活地说。

“你快坐下吃饭吧,恐怕你光是喝酒,没有吃东西。外婆今天是第一次起床。”

“噢,这太好啦。”斯皮里多诺夫说着,把羹匙掉在碟子里,往衣服上溅了不少菜汤。

“哎呀,斯捷潘,你今天醉得真厉害,”弗拉基米罗芙娜说,“这是因为什么喜事儿呀?”

他把碟子推开。

“你吃呀。”薇拉说。

“你们听我说,是这样的,”他低声说,“我有一个消息。我的问题已经定了,在党内受到严重警告,部里来的命令是,要我上斯维尔德洛夫斯克州,到一个很小的发电站去,是烧泥炭发电的,农村型的,总而言之,一降到底了,住房可以保证。搬迁费相当于两个月的工资。明天就开始办移交。可以弄到车票。”

弗拉基米罗芙娜和薇拉对看了一眼,然后弗拉基米罗芙娜说:

“可见,喝酒是有充分理由的,没说的。”

“妈妈,你也跟我们去吧,给您单独一个房间,好些的。”斯皮里多诺夫说。

“恐怕到那儿也只能给你们一个房间。”弗拉基米罗芙娜说。

“妈妈,反正有一个房间也要给您住。”他还是生平第一次唤她妈妈。也许是因为醉了,他眼里还噙着泪水。娜塔莉亚走进来,斯皮里多诺夫换了话题,问道:

“工厂的情形怎样,我们的老头子是怎么说的?”

娜塔莉亚说:

“刚才他等您的,现在他睡着了。”

她坐到桌旁,用拳头支着腮,说:

“他刚才说,工人在工厂里炒瓜子吃,这就是他们的主要食品。”

她忽然问道:

“斯捷潘·费多罗维奇,听说您要走,是吗?”

“是这样啊!我也听说了。”他快活地说。

她说:

“工人们都舍不得让您走。”

“有什么舍不得的,新的站长季什卡·巴特罗夫是一个很好的人。我和他在大学里是同学。”

弗拉基米罗芙娜说:

“你们到了那里,谁能给你补袜子补得这样好呀?薇拉又不会。”

“这倒的确是一个问题。”斯皮里多诺夫说。

“这么看,娜塔莉亚还得跟你们一块儿去呢。”弗拉基米罗芙娜说。

“好吧,”娜塔莉亚说,“我去!”

大家都笑起来,但是说过笑话之后,沉默中却出现了难为情和紧张的气氛。

六十一

弗拉基米罗芙娜决定和女婿、薇拉一道走,她到古比雪夫就停下来,准备在叶尼娅那儿住一些时候。

临走之前的一天,弗拉基米罗芙娜向新站长借了一部汽车,要上城里去看看自己那毁掉的房子。

在路上,她问司机:

“这儿是什么?以前这儿是什么?”

“以前什么时候?”司机生气地问道。

在城市废墟中显露出生活的三个层次:战前的生活,战时的生活,今天正在重新寻找自己的和平轨道的生活。有一座房子原来是一家化学干洗店和织补店,几个窗子全用砖堵起来,每个窗子上都留了小洞,在作战时期德国一个近卫师的机枪手从小洞里往外打机枪,现在就在小洞里卖面包,有不少妇女在洞口排着队。

在瓦砾丛里到处是掩蔽所和土室,在里面住过士兵、无线电通讯兵,驻扎过指挥所,在里面写过报告,装填过机枪弹带,上过自动步枪子弹。

可是现在烟囱里冒着和平的炊烟,掩蔽所旁边晒着衣服,孩子们在玩耍。和平生活从战争中生长出来,虽然这生活还是很贫困、穷苦的,几乎还像战时那样艰难。

有一些战俘在清除主要街道上的碎石断砖。在暂作食品商店的一些地下室外面,有不少人带着小桶在排队。罗马尼亚战俘们懒洋洋地在砖石堆里翻来翻去,在清理尸体。看不见红军士兵,只是偶尔见到几个水兵。司机对弗拉基米罗芙娜解释说,伏尔加舰队留在斯大林格勒为的是扫除地雷。在许多地方堆着新运到的木板、木条和水泥。这都是刚运到的建筑材料。有些地方已经把瓦砾堆到一旁,重新开始浇灌柏油马路。

在一处空旷的场地上,有一个妇女拉着一辆两轮的板车,车上装着很多包袱,两个孩子拉着拴在车杠上的绳子在帮她拉车。

大家都一心一意要回家,回斯大林格勒来,可是弗拉基米罗芙娜来了却又要走。

弗拉基米罗芙娜问司机:

“斯皮里多诺夫要离开斯大林格勒发电站,您也舍不得吧?”

“我有什么舍不得的?”司机说。“斯皮里多诺夫叫我开车,新站长也叫我开车。都是一个样。开了派车单,我就开。”

“这儿是什么?”她指着一排厚厚的外墙问,墙上开了大大的窗洞。

“是各种各样的机关。还不如给人住。”

“以前这儿是干什么的?”

“以前保卢斯就住在这儿,就是从这儿把他带走的。”

“在那以前呢?”

“您认不出来吗?这是百货大楼。”

似乎战争把以前的斯大林格勒挤走了。可以清楚地想象到,德国军官怎样从地下室里走出来,德军元帅怎样从熏黑的墙壁旁边走过,哨兵怎样向他敬礼。可是,难道弗拉基米罗芙娜就是在这儿买过大衣料,买过手表送给玛露霞做生日礼物,还带着谢廖沙上这儿来,在二楼体育用品部给他买过冰鞋?

那些去看马拉霍夫岗、凡尔登、鲍罗金诺战场的人,看到小孩子、洗衣服的妇女、拉干草的大车、拿草耙的老头子,大概也像这样感到奇怪……这儿,现在是葡萄园的地方,曾经有一队一队的法国大军开过,一辆辆蒙着帆布的货车经过。那儿,有一座农舍,还有集体农庄的一群瘦弱的牲口,还有许多苹果树的地方,曾经有缪拉特元帅的骑兵经过,库图佐夫曾经在这儿坐在椅子上挥动他那苍老的手发动俄军反攻。在那座冈上,鸡群和羊群在乱石丛中找食儿的地方,纳希莫夫曾经在那儿站过,托尔斯泰所描写的光闪闪的炸弹曾经从那儿飞过,曾经有伤兵在那儿呻吟,英国的子弹曾经在那儿呼啸。

弗拉基米罗芙娜也觉得这些排队的妇女、破烂的房舍、这些卸木板的汉子、晒在绳子上的衣服、带补丁的褥子、像蛇一样的长筒袜子、贴在断墙上的布告都十分奇怪。

她感觉出来,斯皮里多诺夫说到在区委会争论如何分配劳动力、木材、水泥的时候,他觉得今天的生活多么乏味,他觉得斯大林格勒《真理报》一味地报道清理废钢铁、打扫街道、修建澡堂和工人食堂,有多么枯燥。他一说起轰炸,说起大火,说起集团军司令舒米洛夫上斯大林格勒发电站来,说起德国坦克从山冈上开来,说起苏联炮兵用炮火迎击这些坦克,就十分带劲儿。

战争的命运就是在这些街道上决定的。这一战役的结局决定着战后世界的版图,决定着斯大林伟大的程度或者希特勒政权恐怖的程度。在整整九十天里,克里姆林宫和贝希特斯加登都在想着,说着,梦魂萦绕着一个词儿—斯大林格勒。

斯大林格勒势必左右历史哲学,左右未来的社会制度。

世界命运的阴影把当初这座充满普通生活的城市遮住,使人不再看到。斯大林格勒成为未来的象征。

这位老妇人渐渐驶近自己的住宅,不自觉地受到渐渐在斯大林格勒显示出来的力量的影响,她当初是在这儿生活,教育子孙,给女儿们写信,害病,买东西的。

她请司机把车停住,走下汽车。她很吃力地在遍地瓦砾的空荡荡的街道上走着,注视着断垣残壁,似乎相识又不相识地辨认着邻近她的房子的一座座房屋的残骸。

她的房子朝街的一面墙还保留着,她的老花眼从空空的窗洞里看到了自己的住房的墙壁,认出了褪了色的蓝绿两色涂料。但是几个房间里已经没有地板,没有天花板,没有楼梯,她也无法上楼看看了。砖墙上还留着大火的痕迹,许多地方的砖已成为碎块。

她真切又痛心地回忆起自己的一生,回忆起几个女儿、不幸的儿子、孙子谢廖沙,回忆起无法挽回的损失,想到自己孤单单的白头。一个穿着旧大衣、破皮鞋的病弱老婆子,望着一座毁掉的房子。

什么在等待着她呢?她这个七十岁的人是不知道的。“生活还在前面。”她想道。什么在等待着她所爱的一些人呢?她不知道。春日的天空透过她的房子的空空的窗洞,朝她望着。

她的亲人们过得都不算好,生活动荡而又前路难测,充满了担忧、痛苦、错误。柳德米拉怎么样呢?家庭不和睦会造成什么结果?谢廖沙呢?还活着吗?维克托活得多么不容易。薇拉和女婿斯捷潘会怎样呢?斯捷潘能不能重新建立家庭,过上安宁的日子?聪明、善良但也厉害的娜佳今后又会怎样?薇拉呢?会不会被独身、穷困和生活的重担压垮?叶尼娅会怎么样,她是不是跟着克雷莫夫上西伯利亚?她自己会不会进劳改营?会不会像米佳那样死掉?国家会不会饶恕谢廖沙?他的父母都已无辜死在劳改营。

他们的命运为什么都这样艰难,这样令人难以捉摸?那些病死的、牺牲的、被处死的人依然和生者保持着联系。她还记着他们的微笑、他们的笑声、他们说的笑话、他们的忧郁和怅惘的眼睛、他们的希望和失望。

米佳曾经抱着她,说:“没什么,妈妈,顶要紧的是,你不要为我担心,在这劳改营里也有一些好人。”索菲亚·列文顿,一头黑发,上嘴唇上面还有细细的茸毛儿,又年轻,又快活,又有气性,还常常朗诵诗。可怜的安娜·施特鲁姆总是很忧郁,很聪明,喜欢嘲笑人。托里亚吃起碎乳渣通心粉狼吞虎咽,很不斯文。她生气托里亚光知道张嘴吃,一点也不愿意帮妈妈的忙,要是对他说:“你连一杯水也不给妈妈倒……”他就说:“……好的,好的,我来倒,可是为什么娜佳不倒?”还有玛露霞!叶尼娅总是讥笑你那种老师式的说教,你常常教训人,用正统思想教训斯捷潘……你和别廖兹金家的小孩子斯拉瓦,和老奶奶瓦尔瓦拉一起沉到了伏尔加河里。米哈伊尔·西多罗维奇,您给我解释解释吧。天啊,他还能解释什么呀……一切生活得不好的人,总是怀着苦楚、隐隐的悲痛、怀疑的心情盼望着幸福。有些上她这儿来,有些给她写信,她常常有一种很奇怪的心情:她有一个和睦的大家庭,可是在心里却有一种孤独感。

现在她这个老婆子还活着,还一直盼望着好日子,又有信心,又怕有灾祸,又为一些活着的人担心,为死了的难受,也为活着的难受。现在她站在这儿,望着毁掉的房子,欣赏着春日的天空,甚至不觉得自己在欣赏天空。她站着,自己问自己,为什么她所爱的一些人的未来吉凶难卜,为什么他们一生有这么多的失误。她不知道,正是在这种困惑不解中,在这种迷惘、痛苦和混乱中,就有答案,就有理解,就有希望;她也不知道,她已经发自内心地理解了他和他的亲人们生活的意义,尽管不管是她,还是她的亲人,谁也说不出自己是在等待什么;尽管他们都知道,在可怖的时期一个人是否幸福完全由不得自己,世界的命运可以为人造福或招祸,可以使人获得荣誉或者使人沦落,把人变为集中营里的尘土,但世界的命运,历史的浩劫、国家发怒的厄运、胜利的荣光、失败的耻辱,所有这些都不能改变那些可以称为人的人。不论等待着他们的是劳动的荣誉,还是冷落、失望和穷困、集中营和死亡,他们都会像人一样生活,像人一样死去,那些牺牲的人便是能够像人一样死去的人—这就是他们可歌可泣的做人的胜利,战胜了世界上过去和今后不断反复出现的气焰万丈的、非人性的一切。

在这最后的一天,不仅从早晨就喝酒的斯皮里多诺夫醉得晕晕乎乎。弗拉基米罗芙娜和薇拉在即将离开的时候,头脑里也晕晕乎乎的。来过几批工人,问到斯皮里多诺夫。斯皮里多诺夫交代了最后几件事,上区委办手续转组织关系,给几个朋友打电话告别,又上兵役局交还了免役证,在各个车间里转了一会儿,和工人们说说话儿,等到在涡轮房里暂时剩下他一个人的时候,他把脸颊贴到凉丝丝的、不动的飞轮上,疲惫地合上了眼睛。

薇拉忙着收拾东西,在炉子上烘尿片,把牛奶煮熟装到瓶子里,准备在路上给米佳喝,又装了一袋子面包。这一天她要和维克托罗夫,和妈妈永远分别了。他们就要留在这儿,这儿再没有谁想起他们,问起他们了。

她一想到她现在是家里的女主人,是镇定的,安于艰难生活的,心里就得到一点儿安慰。弗拉基米罗芙娜望着外孙女因为一直睡不足觉布满血丝的眼睛,说:

“薇拉,往往就是这样。离开经受了许多苦楚的家,比什么都难受。”

娜塔莉亚去烙饼子,给斯皮里多诺夫一家人带着在路上吃。她一大早就背着木柴和面粉上工人村一个熟识的妇女家里去,那一家有一座俄式炉子,她就在那儿做馅,和面。她在厨房里忙活得满脸通红,显得分外年轻、标致。她不住地照着小镜子,笑着,自己的鼻子和腮上沾了不少面粉,可是等那个熟识的妇女一走出厨房,她就哭了起来,泪珠子扑簌簌往面团上落。

那个熟识的妇女发现她掉眼泪,就问道:

“娜塔莉亚,你怎么哭呀?”

娜塔莉亚回答说:

“我跟他们处惯了。老奶奶挺好,我也舍不得那个薇拉,也舍不得她那没有父亲的小孩子。”

女主人细心听完了她的解释,说:

“娜塔莉亚,你不说老实话,你不是因为老奶奶哭。”

“不,我是因为老奶奶。”娜塔莉亚说。

新站长答应让安德列耶夫走,但是要他再在斯大林格勒发电站待五天。娜塔莉亚说,这五天她要陪公公一起过,然后她就上列宁斯克到儿子那儿去。

“以后会知道,咱们下一步上哪儿去。”她说。

“以后你怎么就会知道?”公公问道。但是她没有回答。

大概就是因为什么也不知道,她才哭。安德列耶夫老头子不喜欢儿媳妇对他表示关怀。她觉得,他可能还记着她和婆婆争吵,对她还有意见,不肯原谅她。

到吃午饭的时候,斯皮里多诺夫回家来了。他说了说在机械车间和工人们告别的情形。

“就是在家里,整个上午来看你的人就像朝圣一样,”弗拉基米罗芙娜说,“五个一批,六个一群,不断地来找你。”

“这么说,都收拾好啦?卡车五点钟准时开到。”他笑了笑。“感谢巴特罗夫,他还是派了车。”

事情都交代了,东西都收拾好了,可是斯皮里多诺夫的醉态和神经质的紧张依然没有消失。他开始重新收拾皮箱,重新整理包裹,似乎他急不可待地要走。不一会儿,安德列耶夫从邮局回来了,斯皮里多诺夫问他:

“怎么样,有没有从莫斯科发来的关于电缆的电报?”

“没有,什么电报也没有。”

“哎呀,这些狗东西们在捣蛋呢,要不然到五月就可以开始送电了。”

安德列耶夫对弗拉基米罗芙娜说:

“您的身体还不行,怎么能走呀?”

“没什么,我能行。再说,有什么办法,这又不是在果戈理大街自己家里。这儿已经有油漆工来过,看过了,要把房子修一修给新站长住呢。”

“真是太不讲情理了,他就是等一两天也好哇。”薇拉说。

“他怎么算是不讲情理?”弗拉基米罗芙娜说。“总要过日子呀。”

斯皮里多诺夫问:

“饭做好了吗,还等什么?”

“等娜塔莉亚烙的饼。”

“啊,要是等烙饼,咱们就要耽误上火车了。”斯皮里多诺夫说。

他不想吃饭,但是他还留了酒准备在告别席上喝,他非常想喝酒。

他一直想到自己的办公室去看看,哪怕在那儿待几分钟也好,但是不大合适,因为巴特罗夫正在召开各车间主任会议。他因为感到苦恼,越来越想喝酒。他不住地摇头:咱们要赶不上车了,赶不上了。

这种怕误车的心情,焦急等待娜塔莉亚的心情,不知为什么使他感到愉快,但是他怎么也不明白,究竟为什么感到愉快;他也没有想起来,战前他准备和妻子上戏院的时候,就是这样不住地看表,焦急地说:“咱们要赶不上了。”

他今天很想听到有关自己的好话,因此心情更坏了。于是他一遍又一遍地说:

“为什么要舍不得我这个逃兵和胆小鬼?还有,恐怕我是毫不要脸,才希望得到参加保卫战的奖章。”

“真的,咱们不等了,吃饭吧。”弗拉基米罗芙娜看到斯皮里多诺夫很不自在,就说。

薇拉把一锅菜汤端了来。斯皮里多诺夫拿来一瓶酒。弗拉基米罗芙娜和薇拉都不想喝酒。

“没关系,咱们都像男子汉一样,痛痛快快喝两杯吧,”斯皮里多诺夫说过这话,接着又说,“也许,咱们还是等一等娜塔莉亚?”

恰好在这时候,娜塔莉亚提着篮子走了进来,把一摞一摞的烙饼放到桌子上。斯皮里多诺夫给安德列耶夫和自己各斟了满满的一杯酒,给娜塔莉亚斟了半杯。

安德列耶夫说:

“去年夏天咱们就是这样在果戈理大街弗拉基米罗芙娜家里吃烙饼。”

“看样子,这些饼子一点也不比去年的饼子差。”弗拉基米罗芙娜说。

“那一次吃饭的有多少人呀!可是现在只有外婆,你们两位,再加上我和爸爸了。”薇拉说。

“咱们已经把斯大林格勒的德国佬打垮了。”安德列耶夫说。

“伟大的胜利!可是人付出了多么高的代价。”弗拉基米罗芙娜说。接着又说:“多喝点儿汤,到路上咱们就只能吃干的,接连几天吃不到热的东西了。”

“是啊,路上是很辛苦的,”安德列耶夫说,“上车也很难,连车站都没有,火车都是从高加索开往巴拉绍夫的,在咱们这儿是过路车,车上人非常多,除了军人,还是军人。不过,也从高加索运来了白面包。”

斯皮里多诺夫说:

“像云彩一样朝咱们涌来了,这云彩是怎么来的?是苏联胜利了。”

他心里想,不久前在斯大林格勒发电站还能听见德军坦克的轰隆声,可是现在已经把他们赶到几百公里外。现在战场已经是在别尔哥罗德、丘古耶夫附近,已经是在库班了。

于是他又说起在心里憋得难受的话:

“好吧,就算我是逃兵,但是,该是谁来处分我?就让斯大林格勒的战士们来处分我吧。我在他们面前有愧。”

薇拉说:

“老人家,那一次在您旁边坐的是莫斯托夫斯科伊。”

可是斯皮里多诺夫打断她的话。今天他心里难受得实在憋不住了。他对女儿说:

“我给州委第一书记打了一个电话,想和他道道别,不管怎么说,在整个保卫战时期,在所有的企业领导人中,我是唯一留在右岸的,可是他的副手巴鲁林不给我接电话,说:‘普里亚欣同志没时间和您说话。正忙着呢。’好吧,他忙着就忙着吧。”

薇拉就好像没听到爸爸的话,又说:

“那一天谢廖沙旁边坐的是托林中尉,现在那位中尉哪儿去啦?……”

她非常希望能有谁说他能上哪儿去,他可能还活得好好儿的,正在打仗呢。

假如能听到这样的话,她今天苦恼的心也许会多少得到宽慰。但是爸爸又打断她的话,说:

“我对他说,你也知道,我今天要走啦。他却对我说,好吧,那您就写信吧,有什么话就在信里说吧。好吧,去他妈的吧。来,再喝一杯。咱们在这儿喝酒是最后一次了。”

他端起酒杯,朝着安德列耶夫:

“老人家,过去有什么不周到之处,请多多担待。”

安德列耶夫说:

“瞧你说的,斯皮里多诺夫同志。这儿的工人阶级都舍不得你。”

斯皮里多诺夫干了一杯,沉默了一会儿,就好像沉进了水里。后来就喝起汤来。饭桌上静下来,只能听到吃烙饼的声音,再就是斯皮里多诺夫用汤匙喝汤的声音。这时候小米佳哭了起来。薇拉连忙站起来,走到孩子跟前,把他抱起来。

“弗拉基米罗芙娜,您吃饼呀。”娜塔莉亚像祈求活命一样,恳切地小声说。

“我一定吃。”弗拉基米罗芙娜说。

斯皮里多诺夫带着得意、醉意和幸福的果断神气说:

“娜塔莉亚,我当着大家的面对您说。您在这儿没什么事可干,还是回列宁斯克把孩子带上,上乌拉尔我们那儿去。咱们在一块儿,在一块儿要好过些。”

他想看看她的眼睛,可是她把头垂得低低的,他只能看到她的额头和好看的黑眉毛。

“老人家,您也上我们那儿去吧。在一块儿要好过些。”

“我还上哪儿去?”安德列耶夫说。“我没有多少劲头儿活了。”

斯皮里多诺夫很快地打量了一下薇拉。薇拉抱着小米佳站在桌旁,在哭。

这一天他第一次看到他就要离开的房屋的墙壁,这时他的揪心的痛苦,因为被撤职,失去荣誉和心爱的工作而勾起思绪,使他快要发疯、气得他不能为保卫战胜利而高兴的处分,他的懊恼和耻辱—这一切顿时全都消失,全都失去意义。

这时和他坐在一起的岳母,他一直热爱又永远失去了的妻子的母亲,吻了吻他的头,说:

“没什么,没什么,我的好孩子,生活还在前头。”

因为从傍晚就生起炉子,整整一夜木屋都很闷热。

一位寄居的女子和昨天刚刚从军医院来她这儿度假的伤员丈夫几乎一夜没有睡。他们说话的声音很小,为的是不吵醒房东老大娘和睡在大箱子上的小姑娘。

老大娘很想睡着,可是睡不着。她生气的是,女房客和丈夫说话的声音很小—这倒是影响了她,她不由得用心听起来,尽可能地把她听到的一些个别的词儿联系起来。

也许,如果他们说话声音大一些,老大娘多少听一会儿,也就睡了。她甚至想敲敲板墙,说:

“你们的声音为什么那样小,怎么,有什么好听的事儿吗?”

老大娘有好几次听出完整的句子,后来声音小得又听不清了。

那名军人说:

“我从军医院里来,就连一块水果糖也没办法带来。不用说在前方了。”

“我呀,”女房客说,“也只能拿素油炒土豆招待你。”

后来他们说话的声音就很小了,一点也听不清了,后来好像女房客哭了。

老大娘听到她说:

“这是我的爱情把你保住了。”

“哼,这坏小子!”老大娘在心里把军人骂了一句。

老大娘迷迷糊糊睡了几分钟,显然是打起鼾来,所以说话的声音大些了。

她醒了过来,仔细听起来,听清楚了:

“皮沃瓦罗夫给我往军医院里来信说,不久前才给了我中校军衔,马上又把我提为上校。集团军司令亲自提名的。要知道,也是他把我提为师长的。还有列宁勋章。这一切都是因为那一次战斗,那一次我被埋住了,和在车间里的各营失去联系,还像鹦鹉一样唱歌儿。我有一种感觉,就好像我是骗子。我觉得真不自在,这种情形你都想象不到。”

后来他们显然发觉老大娘不打鼾了,于是说话的声音又小了。

老大娘是独身一人,她的老头子在战前就死了,独生女儿在斯维尔德洛夫斯克工作,不和她住在一起。在战争期间老大娘这儿没有住过什么人,她不明白,为什么昨天来了一名军人,她心里就这样七上八下的。

她不喜欢女房客。她觉得女房客是一个没有头脑、不能独立生活的女人。女房客每天起身很晚,她的小女孩穿得很破烂,弄到什么就吃什么。她大部分时间沉默不语,坐在桌边,朝窗外望着。可是有时候她来了兴头儿,就干起活儿来,原来她什么事都会做:又会缝衣服,又能擦地板,还做得一手好菜汤,虽然是城里人,却还会挤牛奶。显然,她是心里有些不自在。她的小女孩也有点儿任性。非常喜欢和小甲虫、蟋蟀、蟑螂玩儿,而且不像别的孩子,她还傻里傻气地吻小甲虫,说故事给小甲虫听,然后把小甲虫放掉,自己就哭起来,又呼喊,又叫唤小甲虫的名字。秋天老大娘从树林里给她带回一只小刺猬,小女孩就时刻不离地跟着小刺猬跑,小刺猬上哪儿,她上哪儿。小刺猬一发出哼哼声,她就快活得发了疯。小刺猬要是跑到五斗橱底下,她就挨着五斗橱坐在地板上等着,并且对妈妈说:“轻点儿,小刺猬睡觉啦。”等到小刺猬跑回树林里,她有两天都不想吃饭。

老大娘总觉得,她的女房客会上吊的,所以她很担心:拿小姑娘怎么办呀?她已经这么大年纪了,可不愿意添麻烦。

“我用不着照应什么人。”她说。她确实提心吊胆,想到哪天早晨她一起来,发现女房客上吊了,她该拿小姑娘怎么办呀?

她认为,女房客是被丈夫扔了,丈夫在前方另找了一个年轻的女子,所以她天天在愁思苦想。丈夫很少给她来信,就是来了信,她也不显得愉快。想叫她说说心里话是不可能的,她什么也不说。邻居一些妇女也发现,老大娘的女房客是一个很古怪的女人。

老大娘跟着丈夫吃够了苦。丈夫又喜欢喝酒,又喜欢吵闹。他打起人来也不像一般人,常常用火叉或者棍子打她。他也打女儿。他不喝酒的时候,也不会使人快活:又小气,又喜欢找碴儿挑毛病,像个老娘们儿一样,盆儿碗儿的事都要管管:这又不对,那又不对。说她做饭做得不好吃,买东西也不会买,挤牛奶也挤不好,床铺也铺得不整齐。而且每说一句话都要骂娘。他把她也教会了,她现在稍有不开心,就骂起娘来。连她心爱的母牛也要骂。丈夫死的时候,她一滴眼泪也没有掉过。他一直把她折腾到老。拿他有什么办法呀,他是一个酒鬼。他在女儿面前也不怕丑,叫人想起来都觉得难为情。打起鼾来像打雷一样,特别是在喝醉的时候。她的母牛也那样喜欢跑,简直太喜欢跑了,一有机会就离开牛群到处跑,一个老年人要是天天跟着它跑,只有累死。

老大娘时而倾听隔壁的悄声低语,时而想想自己和丈夫过的不和睦的日子,在恼恨的同时,也怜惜起丈夫。不管怎么说,他干活儿还是很劳累的,工资也很低。如果没有奶牛,他们的日子就很不好过。而且他死也是因为他在矿井里吸的煤灰太多了。这不是,她还没有死,还活着呢。当年他还从叶卡捷林堡给她买了一串项链,现在女儿还戴着呢……

一清早,小姑娘还没有醒,女房客便和丈夫到邻村去买面包,在那儿可以凭军人乘车证买到白面包。

他们手挽着手,一声不响地走着。要在树林中走一公里半,走到湖边,再顺着岸边往前走。

积雪还没有化尽,变成了淡蓝色。雪成为大块的、毛边的结晶体,呈现出湖水般的淡蓝色。在小丘的阳坡上,积雪在融化,化雪水在路边水沟里哗哗响着。雪的亮光、水的亮光、覆盖着薄冰的水洼的亮光照得人眼花缭乱。亮光是那样强烈,从亮光中穿过,就好像从密密的树丛中穿过。亮光又扰人,又碍事,当他们走到一个冻住的水洼上的时候,被踩疼的冰突然在阳光中闪烁起来,就好像亮光在脚下发出碎裂声,裂成许多尖尖的、带刺的碎光片。亮光在路边水沟里流着,在有石头拦路的地方,亮光膨胀起来,飞溅起来,发出丁丁淙淙的声响。春天的太阳离大地非常近了。空气又清冽又温暖。

他觉得,他的嗓子本来冻坏了,喝酒烧坏了,硝烟灰尘呛坏了,骂娘骂脏了的,现在被这亮光和天上的蓝色洗干净了,涮干净了。他们走进树林里,来到林边几棵松树的树荫下。这儿仍然有薄薄的一层雪没有融化。在松树上面,几只松鼠在绿枝上忙活着,下面,在结了一层冰壳的雪地上,有一大片啃过的松球,还有尖牙咬下的许多碎木屑。

树林里十分宁静,亮光被一层一层的松针挡住,所以没有喧嚷,也不叮叮响,只是小心翼翼地罩着大地。

他们依然一声不响地走着,他们又在一起了,就因为这样,周围的一切都变得美好了,春天来了。

他们不约而同地站了下来。两只吃得肥肥的红腹灰雀儿停在枞树枝上。红红的肥胖的胸脯,就像在施了魔法的雪中绽开的两朵花儿。此时此刻的宁静是奇异而美妙的。

在这种宁静中,会想起去年的树叶,想起过去的一场又一场风雨,筑起又抛弃的窠巢,想起童年,想起蚂蚁辛辛苦苦的劳动,想起狐狸的狡诈和鹰的强横,想起世间万物的互相残杀,想起产生于同一心中又跟着这颗心死去的善与恶,想起曾经使兔子的心和树干都发抖的暴风雨和雷电。在幽暗的凉荫里,在雪下,沉睡着逝去的生命—因为爱情而聚会时的欢乐,四月里鸟儿的悄声低语,初见觉得奇怪、后来逐渐习惯了的邻居,都已成为过去。强者和弱者、勇敢的和怯懦的、幸福的和不幸的都已沉睡。就好比在一座不再有人住的空了的房子里,在和死去的、永远离开这座房子的人诀别。

但是在寒冷的树林中比阳光明丽的平原上春意更浓。在这宁静的树林里的悲伤,也比宁静的秋日里的悲伤更沉重。在这无言的静默中,可以听到哀悼死者的号哭和迎接新生的狂欢……

还是黑沉和寒冷的,但是不要多久,大门和栅栏门就要打开,空荡荡的房子里又要热闹起来,又会充满孩子的笑声和哭声,又会响起女人的匆忙而动听的脚步声,满怀信心的男主人就要走进房子里来了。

他们站着,挎着面包篮子,没有说话。

