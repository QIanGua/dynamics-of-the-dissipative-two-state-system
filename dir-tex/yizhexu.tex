\section{ 译者序}


这是苏联赫鲁晓夫时期的一部禁书。斯大林时期禁书很多很多,勃列日涅夫时期也不少,比较开明的赫鲁晓夫时期禁书不多,主要的就是两部,一部是《日瓦戈医生》,另一部便是这部作品。《日瓦戈医生》有幸在国外很快出版,并因而使作者获得诺贝尔奖金。这部作品在作者生前一直未能出版。其遭遇比《日瓦戈医生》更苦、更悲惨。

格罗斯曼是一位铁骨铮铮的伟大作家。正因为如此,他一生坎坷,他的作品的遭遇也是这样;正因为如此,在熟悉苏联文学的我国读者中,还很少有人知道这位伟大作家的名字。

瓦西里·格罗斯曼是苏联的犹太裔作家。1905 年出生于乌克兰。1929 年毕业于莫斯科大学数学物理系[1]。卫国战争之前,著有革命历史题材的长篇小说《斯捷潘·柯尔丘根》。卫国战争开始后,以《红星报》军事记者身份上了前线。在前线深入实际采访的同时,还勇敢地参加作战。1942年写出反映苏联人民英勇奋战的中篇《人民是不朽的》,因而蜚声文坛。 1943年开始创作反映斯大林格勒保卫战的两部曲。1952 年两部曲的第一部《为了正义的事业》问世。受到广大读者的热烈赞誉。诗人巴让说,这是一部富有人性的、思想深刻的、不说恭维话的作品。其中心思想是:建立伟大功绩的主要是人民群众,不是像另外一些作品那样,把一切功绩归于斯大林。正因为此,这部作品一方面受到广大人民的欢迎和赞誉,另一方面,很快就受到官方评论界的严厉批判。1956 年起,格罗斯曼的作品不准再版,格罗斯曼的名字从此在文坛消失。

格罗斯曼以顽强的毅力在极其困难的条件下继续创作斯大林格勒保卫战两部曲的第二部,并于1960 年完成。这便是本书《风雨人生》[2]。

这已经是在苏共二十大之后,文学解冻已经开始。然而第二部的遭遇却更为悲惨。

他把第二部手稿交给《旗帜报》编辑部。有几家报纸已经刊出小说的片断,本书出版的消息和广告都已发出,作家和读者都在欢欣鼓舞地等待着这部作品出版。但是因为《旗帜报》编辑部怕负责任,把这部作品上报。结果,保安机关抄了格罗斯曼的家,把所有的底稿抄走,全部焚毁,彻底消灭。苏斯洛夫说:这样的作品也许过二三百年才能出版!

作者也在1964 年患癌症不幸病逝,未能看到这部凝聚了全部心血的作品问世。

但是,这部作品的一份复写稿侥幸保存了下来。后来被拍成微缩胶卷偷运出国,于1980 年在瑞士出版。嗣后又被译成多种文字,在西方引起很大的轰动。评论家称之为:“这是本世纪真正的《战争与和平》。”

《风雨人生》于1988 年在苏联出版后,引起热烈的反响。苏联评论家写道:“我们的评论家们常常叹息:为什么见不到描写1941至1945 年战争的《战争与和平》?瞧,这就是!”有的作品,曾经红极一时的,随着时代的变迁,渐渐失去色彩;有的作品,曾经被压制、被扼杀的,随着时代的变迁,越来越显示出其生命力。书之所以遭禁,往往是由于书中触及了一些不能触及的问题,或者其中某些观点与当局的观点相抵触。历史上,当统治者走向历史的反面,不能代表人民利益的时候,便划定界限,设置幕障,不准透过幕障观察问题,不准说界外的话。格罗斯曼却透过帷幕、透过迷雾观察事物,说话只顾事实和真理,不顾界限,因而触怒了当时的领导层,因而这部作品成为超级禁书!

格罗斯曼通过作品中人物的言语和思想发表了极其深刻、极其朴素的见解。是的,极其深刻,又极其朴素、极其简单。译者原来以为,深刻总是高深、深奥、复杂的同义语,是朴素、简单的反义词。译过这部作品之后,才懂得了:原来最深刻的道理也就是最朴素、最简单的道理。比如,一个国家与政党是不是进步的,要看是否能提高人民的生活,是否能最大限度地保障人民的自由。这个道理多么朴素、多么简单!

格罗斯曼本来就是一位有胆有识的作家。斯大林去世,苏共二十大以后,苏联知识界思想渐渐得到解放,格罗斯曼,则是走在思想解放运动的最前列。因此写作第二部时的思想深度又与写作第一部时大不相同。第二部中虽然有些人物仍是第一部中的人物,但事实上已经是另一部作品了。

作品以斯大林格勒保卫战为中轴,以沙波什尼科夫一家的活动为主线,描绘出从前线到后方、从战前到战后、从城市到乡村、从高层到基层、从莫斯科到柏林、从希特勒的集中营到斯大林的劳改营……的广阔社会生活画面。正因为作家有敏锐的目光、无所畏惧的胆量和深厚的功力,他所描绘的画卷是真实的。评论者称《风雨人生》是当代的《战争与和平》,就是说,和托尔斯泰的《战争与和平》一样,它为我们提供了一幅真实的当代社会生活画卷。

作者运用的是传统的手法。用真正的现实主义精神和现实主义手法写出的作品具有震撼人心的力量。真正的现实主义是有强大的生命力的。那些粉饰苦难现实的作品不是现实主义的作品。

当人民处在苦难中的时候,特别需要作家的真诚和勇气!

格罗斯曼和广大人民一起经历了集体化时期,经历了1937 年的所谓肃反运动,经历了伟大的卫国战争,眼见广大人民用鲜血换得胜利之后,依然受到不公平的待遇,作家洒着眼泪书写历史事实,探索苦难根源。

我和老友冀刚合作翻译了《日瓦戈医生》,现在我又翻译了《风雨人生》。这是两部最著名的反思作品。但我觉得,这两部作品有很大的不同。帕斯捷尔纳克是真诚的,是有良心的作家,但他写作《日瓦戈医生》,只是一种叹息和悲伤,谈不到反思。格罗斯曼则不仅有真诚和良知,而且更有勇气,更有认识的勇气、面对现实的勇气。他写作《风雨人生》,不仅旨在创作真实的社会生活画卷,而且旨在进行深沉的反思。在所有的反思作品中,《风雨人生》是最应该称作反思作品的。

格罗斯曼的观点并非今日苏共领导的观点。而《风雨人生》今天能够在苏联出版,任凭评论界和广大读者评说、赞誉,这说明今天苏共领导的开明。如果一个政党是真心实意为人民服务的,而不是实际的法西斯独裁者的话,是不应该压制不同意见的。人民的天下,人民可以对任何问题进行随意的探讨,这是理所当然的事。这也许是鉴别人民政府与独裁政府的主要标志之一。

我一生译过不少苏联作品,其中我最喜欢的是两部,一部是《静静的顿河》,另一部便是这部作品了。这部作品并无曲折离奇的故事情节,但处处扣人心弦。

亲爱的读者,读读这部作品吧!它使人清醒,使人觉悟,使人知道自己是一个人,使人知道怎样做一个人!

力 冈

1989 年6 月10日于安徽师大