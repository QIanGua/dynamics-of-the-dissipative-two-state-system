

\section{ 第一部}


一

田野上雾气沉沉。顺着公路伸展开去的高压线上,闪烁着汽车车灯的反光。

没有下过雨,但黎明时的大地是潮湿的,在禁止通行的信号灯亮起的时候,湿漉漉的柏油路面上就会出现晃晃不定的红色的光斑。在很多公里之外就感觉到集中营的气氛:电线、公路和铁路纷纷朝集中营延伸,越来越密集。这是线路纵横交错的地区,一条条线路把大地,把秋日的天空和夜雾划成许许多多矩形和平行四边形。

远方的警报器送来长长的、低沉的鸣声。

公路紧挨着铁路,装载着一袋袋水泥的汽车队,有一阵子几乎和一列长得不见头尾的军用货车并排前进。穿军大衣的司机们没有看在一旁行进的列车,也没看车上一个个灰点儿似的人脸。

雾中出现了集中营的铁篱:一道道铁丝网架在钢筋混凝土桩上。棚屋一座连一座伸展开去,排成一条条又宽又直的街道。从这些棚屋的单调一律,就可看出这座庞大集中营的不人道。

在千百万的俄罗斯农舍中,没有也不可能有两座完全一模一样的。凡是有生命的东西,都各有各的特性。两个人不可能一模一样,两丛蔷薇也不可能一模一样。如果强行消除生命的独立性和各自的特点,生命就会消失。

头发斑白的火车司机装做漫不经心的样子,瞅着从一旁闪过的混凝土桩柱、架着旋转探照灯的高架和钢筋混凝土塔楼,从反光镜里可以看见塔楼上都有士兵守在旋转式机枪旁。司机朝副司机挤了挤眼睛,机车发出警告信号。亮着电灯的扳道房、停在彩条拦路竿后的一长串汽车和牛眼似的红色信号灯一闪而过。

从远处传来迎面开来的列车的汽笛声。司机对副司机说:

“祖凯尔来啦。听这大大咧咧的嗓门儿,能听得出来。他这是卸了载,开着空车上慕尼黑去。”

空载的列车轧轧地开过来,与开往集中营的军车交会。被撕裂的空气发出震耳欲聋的声音,车厢间灰蒙蒙的空隙一闪一闪地晃过。转眼间,被撕成碎片的空间和秋日的曙光又连成一片,有节奏地奔驰着。

副司机掏出口袋里的小镜子,照了照满是油污的脸。司机招招手,借过他的小镜子。

副司机用激动的声音说:

“唉,阿普菲尔师傅,我敢说,如果不是车厢要消毒,咱们回来能赶上吃午饭,不会弄到早晨四点钟才筋疲力尽地赶回来。好像消毒这种事儿就不能在枢纽站搞似的。”

老司机很讨厌没完没了地搞消毒。

“发长信号,”他说,“咱们不要上备用线,要直接开进大卸场。”

二

自从参加共产国际第二次代表大会之后,米哈伊尔·西多罗维奇·莫斯托夫斯科伊第一次认真运用自己的外语本领就是在德国人的集中营了。战前他住在列宁格勒,和外国人交谈的机会不多。现在他不由得想起当年侨居伦敦和瑞士的情景,那时候,因为天天和各国革命家在一起,说话、争论、唱歌用的都是多种欧洲语言。

邻铺的意大利神甫加尔季告诉他,关在集中营里的有五十六个民族的人。

这些在集中营棚屋的数万名居住者,他们的命运,他们的脸色,他们的衣服都是一样的,他们都拖着脚步走路,喝的都是甘蓝和俄罗斯囚犯叫做“鱼眼”的人造西米熬成的菜汤。

对于管辖者来说,集中营里的人的区别仅在于号码和缝在上衣上的布条的颜色:红色的是政治犯,黑色的是怠工者,绿色的是小偷和杀人犯。

集中营里的人因为语言不通,彼此不了解,但共同的命运把他们结合起来。分子物理学家、古文献学家和连自己的名字也不会写的意大利农民、南斯拉夫牧民睡在一起。当年有厨子精心调制菜肴、吃不好还会使女管家惴惴不安的人和天天吃腌鳕鱼的人一起穿着木底鞋去干活儿,还要忧心忡忡地张望着:留络腮胡子的德国佬是不是来了?

集中营里的人各不相同的遭际中有相同之处。追寻往事的梦不论萦系着意大利土路边的小园,萦系着北海边悲怆的涛声,还是博布鲁斯克郊外领导干部住房里橙黄色的灯罩,所有囚犯过去的岁月都是美好的。

一个人在进集中营之前的生活越是艰难,现在越是起劲地说谎。

这种说谎不是为了欺骗,而是为了赞美自由:在集中营外面的人不可能是不幸福的……

这座集中营战前叫做政治犯集中营。

国家社会主义党[1]创造了新型的政治犯—没有犯过罪的罪犯。

许多人被关进集中营,只是因为在同朋友交谈中说了一些不满意法西斯制度的话,或者说了一些涉及政治的笑话。他们既没有散发传单,也没有参加地下政党。他们的罪名,是他们有可能参加这些活动。

在战争时期将俘虏关进政治犯的集中营,也是法西斯的新创造。这里有在德国境内被击落的英国和美国飞行员,还有投靠了德国秘密警察的红军指挥员和政委。他们的任务是提供情报,配合行动,出点子,在各种各样的声明上签名。

集中营里还有怠工者,也就是有意不干兵工厂和军事工程中的活儿的故意旷工者。因为不好好干活儿而把工人关进集中营,也是国家社会主义党的一项发明。

集中营里有些人衣服上缝的是紫布条,那是从法西斯德国出去的德国侨民。这也是法西斯的新发明:只要离开德国,不管在国外如何循规蹈矩,都要成为政治敌人。

衣服上带绿布条的人,也就是小偷与盗贼,在政治犯的集中营里是享有特权的一部分人;警方依靠他们监视政治犯。

利用刑事犯控制政治犯,也是国家社会主义党的新发明。

在集中营里还有一些人遭际特殊,还没有发明适合他们的布条子颜色。但是就连玩蛇的印度人,从德黑兰来德国学绘画的波斯人,以及学物理的中国留学生,国家社会主义党都为他们准备好了铺位、一小锅菜汤和十二小时挖地的活儿。

军用列车日日夜夜朝集中营,朝一座座死亡的营地开来。空中回响着车轮的轧轧声、机车的吼叫声、成千上万衣服上缝着五位数蓝色号码的囚犯出工时杂沓的脚步声。一座座集中营成为新欧洲的一座座城市。这些城市一天天扩大起来,有自己的规划,有自己的街道和广场,有医院、市场、火葬场、运动场。

跟这些集中营城市相比,跟火化炉上空一道道可怖的黑红色火光相比,那些坐落在城郊的一座座老式监狱,显得多么单纯,多么古朴啊。

看样子,为了控制大量的囚犯,似乎也需要有数量庞大,甚至上百万的军队来监督和管理。但事实却不是这样。常常一连几个星期在集中营里见不到穿党卫军制服的人!囚犯们自己担任起集中营城市里的警察队。囚犯们自己维持营里的秩序,自己监督着,只准许烂土豆、冻土豆进他们自己的锅,把大土豆、好土豆挑出来送往军需品供应站。

囚犯们在集中营的医院和化验室里当医生和化验员;当清洁工,打扫集中营的街道;当工程师,为集中营里提供照明用电和暖气,为集中营里的机器制造零件。

充当又凶狠又卖力的集中营警察的是“卡波”[2],在左臂上戴着宽宽的黄臂章,有营头儿、区头儿和室头儿。他们从上到下监管着营里的一切活动,从全营的事情,到每个人夜间在床铺上的言行。这一部分囚犯可以参与营当局的机密大事,甚至可以参与编制分类名单、在特种囚室里收拾囚犯等事。看样子,即使营当局完全撤离,这些囚犯仍然会让铁丝网上保持着高压电流,叫人跑不掉,还继续干活儿。

这些“卡波”卖力地为营当局效劳,但也常常唉声叹气,有时甚至哭起那些被送往火化炉的人……不过,这种二重性并不彻底,他们不会把自己的名字列入分类名单。莫斯托夫斯科伊感到特别可怕的是,国家社会主义党并不是戴着单片眼镜、傲然不可一世、与一般人不同的外来者。国家社会主义党就像自己人一样住在集中营里,和普通人没有什么区别,也像普通人一样开玩笑,他们的玩笑也会逗人笑,他们是平常人,一言一行都和平常人一样,他们通晓囚犯们的语言,十分了解囚犯们的思想和心情。

三

莫斯托夫斯科伊、阿格丽宾娜·彼得罗芙娜、军医索菲亚·列文顿和司机谢苗诺夫在那个八月之夜在斯大林格勒郊外被德军俘虏之后,被带到了一个步兵师师部。

经过审讯之后,德国人把阿格丽宾娜·彼得罗芙娜放了,翻译官并且根据战地宪兵队人员的指示,给她带上一大块豌豆面包和两张三十卢布的红钞票;谢苗诺夫被编入俘虏大队,送往维尔佳契村地区的集中营营部。莫斯托夫斯科伊和索菲亚·奥西波芙娜·列文顿被带到集团军司令部。

莫斯托夫斯科伊在那儿最后一次看到索菲亚·奥西波芙娜:她站在到处是灰土的院心里,帽子没有了,肩章、领章被撕得耷拉下来,那悲怆和愤恨的眼神和脸色,使莫斯托夫斯科伊感到欣慰。

在第三次审讯之后,莫斯托夫斯科伊被徒步押往火车站,车站上有一列运粮的军车正在装车。有十个车厢装运许多姑娘和小伙子去德国做工。在军车开动的时候,莫斯托夫斯科伊听到一片妇女的哭声。他被锁在硬座车厢的小乘务室。押解他的士兵并不粗暴,但是在莫斯托夫斯科伊问他什么话的时候,他的脸上却流露出聋哑的神气。从中可以感觉出,他一心一意地注视着莫斯托夫斯科伊。动物园工作人员用火车运送动物,动物在笼子里沙沙蠕动,有经验的工作人员就是这样一声不响、一心一意地注视着笼子的。等到火车来到波兰总督管辖区的土地上,乘务室里又进来一名乘客—一位波兰主教,是个白头发、高个子的漂亮老头儿,眼睛里露出悲戚的神气,嘴唇像年轻人那样丰满。他马上就对莫斯托夫斯科伊说起希特勒对波兰宗教界的残酷迫害。他说俄语带有很重的波兰口音。莫斯托夫斯科伊不客气地对天主教和教皇骂了一顿之后,他不作声了,而且,莫斯托夫斯科伊再问他什么话,他也只是用波兰话简短地回答一下。过了几个钟头之后,就让他在波兹南下车了。

过了柏林,莫斯托夫斯科伊被带进集中营……这一营区关押的是秘密警察特别感兴趣的囚犯,他来到这里,似乎已经过了很多年。在这种特别营区里,生活条件比劳动营里要好些,但这是实验室里被试验动物的富足生活。有时值班的把一个人叫到门口—原来是一个朋友要以优惠条件进行平等交换,用烟草换食品,这个人便得意洋洋地回到铺位上。有时同样叫另一个人到门口去,这人便中断了谈话,朝门口走去,交谈者就再也等不到他把话说完了。过一两天,就会有“卡波”来吩咐值班的把破衣烂布打扫出去,有人就会用讨好的口气问“卡波”队员凯泽:能不能睡到空出来的床铺上?已经习惯了七扯八拉的闲谈,从囚犯分类到火化尸体,到集中营里的足球队—最好的队是挖地的“沼地兵”,前锋很棒,攻势很猛,波兰队后卫不行。各种各样有关新式武器的传闻、国家社会主义党头头儿钩心斗角的传闻,大家都听腻了。传闻总是又好又不真实,是集中营囚犯的麻醉剂。

四

天快亮时下了一场雪,直到中午也没有化。俄罗斯人感到又欢喜又悲伤。这是俄罗斯在思念他们,将母亲的头巾扔在他们的苍白而痛楚的脚下,染白了棚屋顶,远远看去,一座座棚屋很像家乡的房屋,呈现出一派乡村气象。

但这只闪现了一会儿的欢喜,一与悲伤相遇,立刻就沉没在悲伤中。

值班的原西班牙士兵安得列阿走到莫斯托夫斯科伊跟前,用似通不通的法语说,一个担任文书的朋友看到有关一个俄国老头子的文件,但是那个文书没来得及细看,办公室主任就把文件带走了。

“这文件就是决定我的命运的。”莫斯托夫斯科伊心里想。并且对自己的镇静感到高兴。

“不过没关系,”安得列阿小声说,“还是可以了解到的。”

“向营警备司令了解吗?”加尔季神甫问道。他的大眼睛在昏暗中闪着黑黑的亮光。“还是向治安总部代表利斯本人了解?”

白天的加尔季和夜晚的加尔季差别之大,使莫斯托夫斯科伊感到吃惊。白天谈的是菜汤,谈新来的人,跟同房间的人商量交换食品,回味加了大蒜的辛辣的意大利吃食儿。

被俘的红军知道他爱说的口头语“全体完蛋”,每次在集中营的广场上碰见他,老远就朝他喊:“帕德列老爹,全体完蛋!”并且笑着,就好像给这话打气。他们以为“帕德列”是他的名字,所以喊他帕德列老爹。

有一天晚上,关押在特别营区的一些苏联指挥员和政委跟他开玩笑,问他是不是真的守戒不接近女色。

加尔季听着法语、德语和俄语大杂烩,一笑也不笑。

然后他说起来,莫斯托夫斯科伊就把他的话翻译出来。他说的是,俄国革命者为了自己的信仰可以去服苦役,上断头台。为什么诸位就怀疑,一个人为了宗教信仰可以不接近女人呢?这跟牺牲生命无法相比呀。

“算啦,话不能这样说。”旅政委奥西波夫说。

夜里,等营里的人都睡了,加尔季就变成另外一个人。他跪在床铺上,做起祷告。集中营城市的所有苦难就好像沉没在他那炽热的眼睛里,沉没在那眼睛的柔和而分明的黑光中。他褐色的脖子上筋绷得紧紧的,就像在干活儿,长长的神情恬淡的脸呈现出忧郁而幸福的执着表情。他祷告很长时间,莫斯托夫斯科伊便在这个意大利人又低又快的祷告声中沉沉入睡。莫斯托夫斯科伊常常在睡一两个钟头后醒来,这时候加尔季已经睡了。加尔季睡觉很不安生,就好像要在睡梦里把自己的两种特性,把白天的特性和夜晚的特性合到一起,又打鼾,又咬牙,还有滋有味地咂吧嘴,像打雷一样把胃里的气直往外倒,忽然又拉长声音唱起赞美诗,赞颂上帝和圣母的大慈大悲。

他从来没有责备过这位老苏共党员不信教,倒是常常向他询问苏俄的情况。加尔季一面听莫斯托夫斯科伊叙说,一面不住地点头,好像对于关闭教堂和寺院,对于苏维埃国家没收东正教大量地产这样的事表示赞许。他的一双黑眼睛带着悲伤的神气望着这位老共产党员,于是莫斯托夫斯科伊很生气地用法语问他:

“您听懂了吗?”[3]

加尔季笑起来,平时他谈起辣汁肉丁和番茄沙司,常常这样笑。

“您说的我全懂。我只是不懂,您为什么要说这种事?”[4]

关押在特别营区里的苏联战俘们也是要做工的,所以莫斯托夫斯科伊只有在晚上和夜里才能见到他们,跟他们谈一谈。古泽将军和旅政委奥西波夫不做工。

经常跟莫斯托夫斯科伊聊天的是一个很古怪、令人很难断定其年龄的人—“海象”伊康尼科夫。他睡在全屋最差的地方,也就是睡在门口,又有冷飕飕的过堂风,又有带味儿的大马桶,马桶盖不住地砰砰响。

苏联囚犯管伊康尼科夫叫“老伞兵”,把他看作疯子,对他又怜悯又厌恶。他具有不寻常的耐性,那样的耐性只有疯子和白痴才有。他从来不害伤风感冒,虽然在睡觉的时候连秋雨打湿的衣服也不脱。真正能够用这样响亮、这样清楚的嗓音说话的似乎也只有疯子。

他跟莫斯托夫斯科伊是这样认识的。他走到莫斯托夫斯科伊跟前,一声不响地对着他的脸打量了老半天。

“这位同志,您有什么好事儿要说?”莫斯托夫斯科伊问道。

伊康尼科夫拉长声音说:

“说好事儿?什么是好,什么是坏?”

莫斯托夫斯科伊听到这话,笑了。这话忽然把他带到了童年时代,那时候大哥从神学校回来,常常和父亲争论神学上的事。

“这是老掉牙的问题了,”莫斯托夫斯科伊说,“佛教徒和古时的耶稣教徒早就想过这个问题。马克思主义者为解决这个问题,也花了不少脑筋。”

“解决了吗?”伊康尼科夫问道。那声调让莫斯托夫斯科伊觉得十分好笑。

“现在红军正在解决这个问题,”莫斯托夫斯科伊说,“请恕我直言,您的语调中有一种橄榄油味道,不是牧师的橄榄油,便是托尔斯泰主义者的橄榄油。”

“不可能不是这样,”伊康尼科夫说,“因为我是托尔斯泰主义者。”

“真没想到!”莫斯托夫斯科伊说。他对这个古怪人产生了兴趣。

“您要知道,”伊康尼科夫说,“我相信,布尔什维克在革命以后对教会的打击,对于耶稣教思想是有益的,因为教会在革命前已经进入很可怜的状态。”

莫斯托夫斯科伊很和善地说:

“您可真是一位雄辩家。我终于在老年看到了福音的奇迹。”

“不,”伊康尼科夫愁眉苦脸地说,“在我们看来,你们为了目的不择手段,而你们的手段是残酷的。您不要把我看成什么奇迹,我不是什么雄辩家。”

“那么,”莫斯托夫斯科伊忽然十分恼火地说,“要我怎样为您效劳呢?”

伊康尼科夫像个军人一样,以“立正”姿势站着,说:“请不要笑话我!”他的痛苦的声音显得十分悲戚。“我到您这儿,不是来开玩笑的。去年九月十五日,我看到两万犹太人被杀害,有妇女,有儿童,有老头子。那一天我明白了,如果有上帝的话,是不容许这种事的,这一下我看清楚了,上帝是没有的。在今天的一片黑暗中,我看见你们的力量,是这种力量在同可怕的恶势力斗……”

“那好吧,”莫斯托夫斯科伊说,“咱们来谈谈。”

伊康尼科夫在干挖土的活儿,在营属土地的沼泽地带,那里正在铺设一系列粗大的水泥管道,以便把使洼地变成沼泽的河水和脏水排出去。在这一地带干活儿的人就叫“沼地兵”。分到这儿来的一般都是营方不喜欢的人。

伊康尼科夫的手小小的,手指头细细的,指甲像小孩子的一样。他从工地上回来,常常满身泥浆,浑身湿漉漉的,走到莫斯托夫斯科伊床铺前,问道:“可以在您身边坐一坐吗?”他也不看对方,就坐下来,微微笑着,用手抹抹额头。他的额头有点儿奇异—不怎么大,却饱鼓鼓的,发亮,而且亮得出奇,就好像跟那肮脏的耳朵、暗褐色的脖子和手以及磕断的指甲不是一个人身上的。经历简单的苏联战俘都觉得他是一个难以理解的神秘人物。

伊康尼科夫家的祖先从彼得大帝时代起一代接一代都是神甫。只是最后一代人走了另外的道路—伊康尼科夫和所有的兄弟都遵奉父命进了世俗学校。

伊康尼科夫进了彼得堡工学院,但因为迷上了托尔斯泰主义,到最后一学年便离开学校,去彼尔姆省北方做起人民教师。他在农村待了八年左右,后来移居南方,来到敖德萨,在一艘货轮的机器房里当钳工,到过印度、日本,在悉尼住过。革命以后他回到俄罗斯,参加了农业公社。这是他多年的理想,他相信,农业公社的共产主义劳动,能够创造人间的天国。

在全面实行集体化的时候,他看到一列列军车满载着被没收了土地家产的富农家庭的男女老少。他看到许许多多瘦弱不堪的人倒在雪地里,再也没有起来。他看到一座座“被封闭的”、人口死绝的村庄,村庄里的门和窗都被钉死。他看到一个被捕的农妇,衣服褴褛,脖子上露出筋骨,一双干活儿的手黑糊糊的,押解的人带着恐怖的表情望着她:她因为饿疯了,吃掉了自己的两个孩子。

这时候,他虽然没有离开公社,却宣讲起福音书,祈求上帝拯救死者。结果他被关进监狱,不过很快就弄清,是三十年代的灾难使他的神志错乱了。在监狱的精神病院里强制治疗一年之后,他出了监狱,前往白俄罗斯,住到大哥家里去。大哥是一位生物学教授。他在大哥帮助下,在科技图书馆找到工作。但是一件件可悲的事对他产生了难以磨灭的影响。

等到战争开始,德国人占领了白俄罗斯,伊康尼科夫看到战俘的苦难,看到白俄罗斯城乡成千上万犹太人被杀害。他又陷入发狂状态,恳求相识和不相识的人掩藏犹太人,他自己也想方设法拯救犹太妇女和儿童。不久他就被告发,侥幸躲过了绞索,进了集中营。

这位破衣烂衫的肮脏“伞兵”的头脑里非常混乱,他主张对超阶级的道德进行荒唐可笑的分类。

“哪儿有强权,”他对莫斯托夫斯科伊说,“哪儿就有灾难,就流血。我见过农民遭受的大灾大难,还说实行集体化是为了做好事。我不相信什么好事,我只相信人性的良善。”

“照你的说法,要是将来做好事把希特勒和希姆莱绞死,咱们也要害怕啦。那您就尽管害怕吧。”莫斯托夫斯科伊回答说。

“您要是去问希特勒,”伊康尼科夫说,“他也会说,设立集中营是做好事。”

莫斯托夫斯科伊觉得,在跟伊康尼科夫争论的时候,不论说什么道理,都好比用刀子切海蜇,怎么切也切不开。

“那位生在六世纪的叙利亚基督教徒说的道理,在今天还是适用的,”伊康尼科夫又说,“‘要清算罪过,要饶恕犯罪的人。’”

在这个屋里还有一个俄罗斯老头子,姓切尔涅佐夫。他只有一只眼睛。看守把他那只人造的玻璃眼球打碎了,那个空空的红眼窝在他苍白的脸上显得非常不协调。他在和人谈话的时候,用一只手捂着空洞的眼窝。

他原来是孟什维克,一九二一年从苏联逃出。在巴黎住了二十年,在银行里当会计。他因为号召银行职工反抗德国新经理的措施,被抓进集中营。莫斯托夫斯科伊尽量不跟他接触。

看样子,莫斯托夫斯科伊博得的声望使独眼的孟什维克感到不安。不论是西班牙士兵,还是挪威文具店老板,比利时律师,都喜欢接近这位老布尔什维克,常常向他求教。

有一天,苏联战俘中的头头儿叶尔绍夫少校坐到莫斯托夫斯科伊的铺上。他微微靠在莫斯托夫斯科伊身上,把一只手搭在他肩上,又快又急切地说起话来。

莫斯托夫斯科伊忽然回头看了看,切尔涅佐夫正在远处的床铺上望着他们呢。莫斯托夫斯科伊觉得,他那只好眼睛里的苦闷神情,比起打掉的眼睛留下的红红的空窟窿还要可怕。

“是啊,伙计,你是不大快活的。”莫斯托夫斯科伊心里想。但并没有幸灾乐祸的心情。

大家时时刻刻需要叶尔绍夫,这不是偶然,是有道理的。“叶尔绍夫在哪儿?没看见叶尔绍夫吗?叶尔绍夫同志!叶尔绍夫少校!叶尔绍夫说的……去问叶尔绍夫吧……”别的棚屋里的人也常常来找他,他的床铺周围总有人来来往往。

莫斯托夫斯科伊管叶尔绍夫叫“思想领袖”。十九世纪六十年代和八十年代的一些社会活动家都是思想领袖。还有民粹派,还有风云一时的米海洛夫斯基。在希特勒的集中营里居然也有自己的思想领袖!独眼者的孤独在这营里似乎成了悲哀的象征。

自从莫斯托夫斯科伊蹲沙皇的牢房,已经几十年过去了,而且那时候是另一个时代,是十九世纪。

现在他常常想起当年的情形,那时候因为有些党的领导人不相信他主持实际工作的能力,他非常生气。现在他感到自己是强有力的,每天他都看到,他的话不论古泽将军,旅政委奥西波夫,还是天天愁眉苦脸、忧心忡忡的基里洛夫少校,都是多么看重。

在战前,使他可以自慰的是,他一直不受重用,不用接触那些使他反感、使他愤慨的事。斯大林在党内的独断独行,对反对派的血腥镇压,对党内老干部的不尊重—这些事他都没有接触到。他非常了解、非常敬重的布哈林的被害,使他感到非常沉痛。但是他知道,在任何问题上与党对抗,就会不自觉地站到反对自己所献身的列宁的事业的立场上。有时他觉得苦恼,他怀疑:他不发一言,不站出来反对自己不赞成的事情,也许是他软弱,是他胆小怕事?战前许多事使人不寒而栗!他常常想起已故的卢那察尔斯基,他多想再看到他啊,跟他交谈是那样轻松,不等一句话说完,他们彼此很快就了解了。

现在,在可怖的德国集中营里,他感到自己有信心,有力量。只有一种不舒服的感觉时刻不离开他。他即使在集中营里,也无法恢复年轻时那种鲜明、完整的感情:在自己人当中是自己人,在外人当中是外人。

有一天,一位英国军官问他,在苏联不能发表反马克思主义的观点,这是不是影响他研究哲学。

“这对别人也许有影响。对我这个马克思主义者没什么影响。”莫斯托夫斯科伊回答说。

“我问这个问题,正因为您是一位老马克思主义者。”英国军官说。虽然莫斯托夫斯科伊听到这话心中,皱了皱眉头,他还是恰当地回答了英国人。

这也并非因为像奥西波夫、古泽、叶尔绍夫这样一些跟他十分亲近的人,有时候也使他感到很不痛快。问题在于,他感到自己心中有许多东西变得陌生了。过去在和平时期,他兴高采烈地去赴老朋友的约,聚会结束时却发现这人已变得格格不入。

但是,和今天的时代格格不入的东西就生长在他身上,已成为他自己的一部分,又该怎么办呢?……又不能跟自己决裂,不能避而不见。

他在和伊康尼科夫谈话的时候,有时会发火,很粗暴,还常常嘲笑他,管他叫脓包、孱头、蠢货、窝囊废。尽管常常嘲弄他,有时候很长时间看不到他,却又想他。

这就是在莫斯托夫斯科伊年轻时坐牢的年代和今天之间的主要变化。

在年轻时候,朋友和同志身上的一切都是可亲的,容易理解的。敌人的每一种思想、每一种观点都是格格不入,毫无道理的。

可是现在他常常在异己者思想中发现他在几十年前珍视的东西,而在朋友的思想和言谈中有时会不可理解地出现异己的东西。

“这大概是因为我在世上活得太久了。”莫斯托夫斯科伊心里想。

五

一位美国上校住在特别营区的一个小小的单间里,准许他在傍晚时候自由走出营区,给他吃的是特别伙食。据说,从瑞典方面有人来要求关照他,是罗斯福总统通过瑞典国王提出这一要求的。

有一天,上校把一大块巧克力糖送给生病的苏联少校尼科诺夫。在特别营区里,最使他感兴趣的是苏联战俘。他想和苏联人谈谈德国人的战略,谈谈战争头一年失败的原因。

他常常跟叶尔绍夫交谈,看着这位苏联少校既严肃又愉快的聪明的眼睛,忘记苏联少校不懂英文。他觉得奇怪的是,长相这样聪明的人怎么会不懂他的话,怎么会听不懂有关他们共同关心的问题的谈话。

“难道您一丁点儿也听不懂吗?”他懊恼地问道。

叶尔绍夫用俄语回答说:

“我们可敬的军士什么语言都懂,只是不懂外语。”

不过,借助微笑、眼神、拍肩膀构成的语言,再加上一二十个发音不准的俄语、德语、法语和英语单词,集中营里的苏联人还是常常跟几十种不同语言的民族的人谈谈友谊、合作、互相支持和对家庭、妻子、儿女的思念。

一些变了音的俄语、法语、英语单词,加上十来个在集中营里新出现的德语单词,足以表达简单而复杂的集中营生活中特别重要的东西。

也有一些俄语单词,如伙计、香烟、同志,是很多民族的囚犯共同使用的。有一句俄语“不行啦”是说明快要死的囚犯的状况的,已经成为大家的共同语言,所有五十六个民族的人都在使用。

大日耳曼民族带着学来的一二十个单词闯入居住着伟大俄罗斯人民的城市和乡村,于是成千上万俄罗斯农村妇女、老人和儿童跟成千上万的德国士兵用这些单词打起交道:“羊羔,老总,举起手来,母鸡,鸡蛋,完蛋。”这种交道绝不是什么好交道……

苏联战俘之间也谈不出什么好结果,有些人宁死不愿卖国,另一些人却千方百计要参加苏奸弗拉索夫的伪军。他们谈得越多,争论得越多,彼此的隔阂越大。到后来他们就不说话了,彼此越来越仇恨,越来越鄙视。

这种不言不语,被恐怖、希望和苦难连接在一起的这些混乱的人群,说着同一种语言的人们的互不理解和仇恨,正反映出二十世纪可悲的灾难之一种。

六

在下雪的日子,苏联战俘到晚上一谈起来特别悲伤。就连性格刚强、常来聚会的兹拉托克雷列茨上校和旅政委奥西波夫也愁眉苦脸,很少言语了。大家都苦闷不堪。

炮兵少校基里洛夫坐在莫斯托夫斯科伊的铺上,垂着肩膀,轻轻地摇着头。似乎不光是那黑沉的眼睛,是他整个巨大的身躯充满了苦闷。那些生存无望的癌症患者往往有这样的眼神。就连最亲近的人看到这样的眼睛,在怜惜的同时,也会想:“你顶好快点儿死吧。”脸色发黄、喜欢到处转悠的柯佳科夫指着基里洛夫,小声对奥西波夫说:

“他不是想上吊,就是想去投伪军。”

莫斯托夫斯科伊搓着长满白白的胡茬子的两腮,随口说:

“哥们儿,听我说说。真的,这样很好。难道还不明白吗?列宁缔造的国家的局面一天天叫法西斯受不了。法西斯没有多少选择的余地:要么把我们吃掉,把我们消灭,要么自己完蛋。从法西斯对我们的仇恨,正可以看出列宁事业的正义性。还有一点也是很重要的。你们要明白,法西斯越是恨我们,我们越是应该相信我们是正义的。我们一定能胜利。”

他猛然转过身去对着基里洛夫,说:

“您这是怎么回事儿呀,嗯?您该记得高尔基的事。有一次他在监狱的院子里走来走去,有一个格鲁吉亚人对他喝道:‘你干吗要像挨了打的母鸡?把头抬高点儿!’”

大家笑起来。

“是的,是的,把头抬起来,”莫斯托夫斯科伊说,“你们想想看,这是伟大的苏维埃大国在捍卫共产主义思想!希特勒要较量,就让他试试吧!斯大林格勒坚持着,没有失守。战前有时候觉得,螺丝帽是不是拧得太紧、太狠啦?可是现在真的连瞎子都看清楚了:只要目的正确,一切手段都不为错。”

“是的,我们的螺丝帽拧得太紧了。这话您说得很对。”叶尔绍夫说。

“拧得还不够呀,”古泽将军说,“假如拧得再紧些,希特勒就到不了伏尔加河边了。”

“用不着我们教导斯大林。”奥西波夫说。

“好啦,”莫斯托夫斯科伊说,“要是死在监牢或者水漉漉的矿坑里,就什么也谈不到了。咱们应该想的不是这个。”

“那又该想什么呢?”叶尔绍夫高声问道。

坐在一起的人互相看了看,又朝四下里看了看,没有作声。

“唉,基里洛夫呀,基里洛夫,”叶尔绍夫忽然说,“咱们这位老人家说得很对:法西斯痛恨我们,我们应该高兴。不是我们消灭他们,就是他们消灭我们。明白吗?你想想看,进集中营找到自己人,总归是自己人跟自己人。不过就是这么回事儿。没什么大不了的!我们是刚强的人,还要给德国人一点颜色看看呢。”

七

第六十二集团军司令部有一整天跟各部失去联系。许多部队的无线电接收机被炸毁;到处有电话线被炸断。

在伏尔加岸边轻轻颤动的土地猛烈震动起来的时候,人们望着流动的、碎波粼粼的河水,有时会觉得伏尔加河是不动的。这时候几百门苏联重炮在伏尔加右岸轰击。马马耶夫冈南坡的德军驻地四周飞起一团团泥土。

一团团旋转飞舞的灰土,经过重力编织的奇妙、无形的筛子,进行了筛选,沉重的土块和泥团落到地上,轻的灰尘飞向天空。被震得耳聋和眼睛发红的红军士兵每天都有好几次跟德军坦克和步兵相遇。

司令部和军队失去了联系,就觉得这一天长得叫人受不了。

为了打发这一天,崔可夫、克雷洛夫和古洛夫什么办法都想过:摆出要做事的样子,写信,争论敌军可能推进到什么地方,开玩笑,喝酒,有小菜也喝,没有小菜也喝,沉默,倾听炸弹爆炸声。铁旋风在掩蔽所周围呼啸,把一切敢于在地面上露头的活物扫倒。司令部瘫痪了。

“咱们来捉傻瓜吧。”崔可夫说着,把装满香烟头的老大的烟灰缸推到桌子角上。

就连参谋长克雷洛夫也沉不住气了。他用手指头敲着桌面,说:

“情况没有更糟的啦,像这样待下去,可别叫人家吃掉。”

崔可夫分好了牌,宣布:“红桃主牌。”可是接着就把牌掺和到一起,说:“咱们像兔子一样坐在这儿玩起牌了。不行,不能这样!”

他心事重重地坐着。他的脸显得很可怕,脸上呈现出剧烈的仇恨与痛苦表情。

古洛夫就像在预测自己的命运似的,也若有所思地说:

“是啊,这样过上一天,准会心力衰竭死去。”

过了一阵子,他大笑起来,说:

“在师里上厕所是一件极其困难的、可怕的事。有人告诉我,柳德尼科夫的参谋长一下子跑进掩蔽厕所,喊:‘乌拉,同志们,我……’他一看,他爱上的那位女医生正蹲在里面呢。”

天黑下来,德寇的空袭也停止了。一个被大炮轰鸣声和机枪嗒嗒声吓坏了的人,如果在夜间来到斯大林格勒河岸上,也许会以为,这是不怀好意的命运之神在决战时刻把他带到斯大林格勒来了,然而对于久经战阵的人来说,这时候正好刮刮胡子,洗洗衣服,写写信,参战的钳工、旋工、电焊工、钟表匠则修修打火机,修修闹钟,还用炮弹壳做油灯,从军大衣上撕下布条子做灯芯。

一闪一闪的爆炸的火光照耀着河岸的斜坡、城里的断垣残壁、一个个油桶、一座座工厂的烟囱,在这种短暂的闪光里,河岸与城市显得又阴郁又悲切。

在黑暗中,司令部的电话总机活跃起来了,打字机嗒嗒地响起来,打印出一叠叠战斗情报,小小发动机发出嗡嗡声,电报机轧轧响起来,电话员在话机里互相呼唤着,以便把通往各师、各团、各炮兵连、步兵连指挥所的线路接通。来到司令部的通信兵老气横秋地轻轻咳嗽着,联络官在向值班作战参谋汇报。

集团军炮兵司令波扎尔斯基老汉、渡河敢死队队长特卡琴柯工程兵将军、刚刚穿上草绿色士兵军大衣的西伯利亚师师长古尔捷夫、带领一师人驻扎在马马耶夫冈下的斯大林格勒本地人巴秋克中校都急着要向崔可夫和克雷洛夫汇报。在向集团军军委委员古洛夫作的汇报中,可以听到一些传遍斯大林格勒的名字,如迫击炮手别斯季尔柯、神枪手瓦西里·扎伊采夫和安纳托里·契诃夫、巴甫洛夫中士,还有第一次在斯大林格勒响起来的名字,如绍宁、弗拉索夫、布雷辛,他们在斯大林格勒的第一天就获得英雄的称誉。而在前沿阵地上,纷纷把折成等腰三角形的书信交给邮递员:“飞吧,书信,从西向东……带去我的问候,再把回信带回来……日安,噢,也许该说:晚安……”前沿阵地上在掩埋死者,死者就在掩蔽所和掩体旁边度自己长眠的第一个夜晚,同志们就在旁边写信,刮脸,吃面包,喝茶,在自制的浴槽里洗澡。

八

斯大林格勒守卫者最困难的日子来到了。

在城市混战中,在进攻与反攻中,在争夺科技宫、工厂、银行大楼,在争夺地下室、院子和广场的战斗中,毫无疑问德军都占优势。

德军插进斯大林格勒南部拉普申公园、库波罗斯沟和叶尔山卡一带的楔形攻势在逐渐扩大,德军的机枪手躲在河边,向伏尔加左岸的红镇南部进行扫射。作战参谋每天在地图上改动战线的位置,看着蓝色标志不断地往前爬,苏方红线与蓝色伏尔加河之间的地带一天天在收缩,越来越狭小。

主动权,战争的灵魂,这些天一直在德国人手里。他们一个劲地在前进,不论苏军怎样发狠反击,都阻挡不住他们缓慢然而不停的前进。

德寇的飞机一天到晚在天空吼叫,用重磅炸弹在苦难的大地上打出一个个窟窿。许多人的脑子里都有一个摆脱不掉的可怕想法:明天或者一个星期之后,已经被德军进攻的铁齿咬得七扭八曲的苏军防地,会变成一条细细的线,这条线甚至会断,那又该怎么办呢?

九

深夜,克雷洛夫将军在自己的掩蔽所的床铺上躺了下来。鬓角隐隐作痛,因为接连抽了几十支烟,喉咙里火辣辣的。他用舌头舔了舔发燥的上腭,转过身朝着内壁。睡意朦胧中,往日的情景纷纷来到脑海里:塞瓦斯托波尔和敖德萨的战场,罗马尼亚步兵冲锋时的呐喊声,铺了石板、长满常春藤的敖德萨的院落和塞瓦斯托波尔的英俊的水兵。

他仿佛觉得自己又在指挥所里,彼得罗夫将军[5]的夹鼻眼镜模模糊糊地闪着光;闪光的镜片又变成千万闪光的碎片,又是波涛翻滚的大海,又是德军炮弹炸碎的岩石扬起的灰色尘雾,灰色尘雾在水兵和步兵头顶上飘飘荡荡,飘到萨普山顶上。

他听到海浪无精打采地拍打着潜水艇,听到潜水艇的水兵粗声粗气地叫喊:“跳!”仿佛他跳入浪涛中,但他的脚马上碰到潜水艇的艇身……于是最后看了一眼塞瓦斯托波尔,看了看天上的星星,看了看岸上的大火……

克雷洛夫沉沉入睡。梦里依然是战争的情景。潜水艇从塞瓦斯托波尔开往诺沃罗西斯克……他蜷着麻木了的腿,胸前背后出汗都湿透了,发动机的声音震得两鬓昏昏的。忽然发动机不响了,潜水艇轻轻地沉到海底。气闷得不得了,被一行行虚线似的铆钉划成许多方块的金属顶压在头上……

他听到许多声音在吼叫,听到水的拍溅声,一颗深水炸弹爆炸了,海水冲击过来,把他从床铺上冲下来。克雷洛夫睁开眼睛:四周围都是火,一股股大火经过敞开的掩蔽所门口朝伏尔加河奔去。可以听到人的叫喊声、自动步枪的嗒嗒声。

“拿军大衣,拿军大衣把头蒙起来!”

有一个不相识的红军士兵对克雷洛夫喊道,并且把军大衣递过来。但是克雷洛夫推开红军士兵,高声问:“司令员在哪儿?”他忽然明白了:这是德国人烧着了油桶,着了火的石油正朝伏尔加河涌去。

看样子,要从这奔流的火海中逃生已经不可能了。溢出的石油填满了坑坑洼洼,在交通壕中汹涌奔流。大火轰轰直响,在流淌的石油上噼啪乱飞。泥土和石头一沾到油就冒起烟来。一道道漆黑闪光的石油从被燃烧弹打穿的油库里往外直涌,像是大卷大卷的烟与火被塞进了油罐,现在都伸展开来了。

几亿年前活跃在地球上的生物,那些野蛮可怕的原始怪物,从厚厚的地层中钻了出来,狂吼怒号,它们巨大的脚掌到处奔窜,贪婪地吞食着一切。烈火窜起几百米高,在高空放出一团团可燃的气体,一闪一闪地喷射着火焰。大片的烈火是那样凶猛,气流简直来不及向燃烧的碳氢分子给氧,微微颤动的浓黑烟层把秋夜的星空和燃烧的大地阻隔开来。从下面望着这油烟滚滚的黑色的苍穹,实在可怕。

一道道火柱和烟柱拼命向上窜,有时像是发怒发威的猛兽的姿态,有时又像晃动的白杨和颤抖的山杨。黑红两色在一团团烈火中不停地旋转,就像跳舞时混在一起的、松开辫子的黑发和红发姑娘。

燃烧的石油在水面上平平地流了开去,经河水冲动,咝咝地响着,冒着烟,弯弯曲曲地流动着。

奇怪的是,这时候已经有很多战士知道怎样可以到达岸边。他们叫喊着:“这儿来,这儿来,顺这条小路!”有些人已经有两三次来到被大火包围的掩蔽所前,帮助司令部的人员逃到岸边土台上,有一小堆脱险的人就站在这里,这是涌入伏尔加河的燃烧的石油分岔的地方。

一些穿棉衣的人帮助司令员和司令部的军官们逃到岸边。这些人把他们认为已经死去的克雷洛夫将军从火里抬出来,他们眨巴了几下烧焦的睫毛之后,又穿过密密的红色蔷薇丛朝各指挥部的掩蔽所奔去。

第六十二集团军司令部人员在伏尔加河边小小的土台上一直站到早晨。大家用手护着脸,遮挡着灼热的空气,不时弹着衣服上的火星,望着司令员。司令员披着军大衣,头发从帽子底下露出来,耷拉在额头上。他皱着眉头,阴沉着脸,然而显得很镇定,好像在深思。

古洛夫环顾着站在一起的人,说:

“这么着,咱们没烧死……”他又摸了摸滚烫的军大衣纽扣。

“喂,你这位带锹的弟兄,”工程兵司令特卡琴柯喊道,“赶快在那儿挖一道小沟,要不然那个小土包上的火就要流过来啦!”

他对克雷洛夫说:

“将军同志,全都乱套啦,火像水一样流起来,伏尔加河着了火烧起来。好在没有大风,要不然咱们全烧死啦。”

当微风从河面上吹来,高大的火幕轻轻晃动、倾斜过来的时候,人们纷纷躲避燎人的火舌。有的人走到水边,用水把靴子打湿,水一到滚烫的靴筒上很快就蒸发了。有的人一声不响,拿眼睛盯着地面,有的人一个劲儿地四下里打量着,有的人为了缓和紧张情绪,开起玩笑:“在这儿不用火柴也行了,要抽烟可以向伏尔加借火,也可以向风借火。”也有人不住地抚摩自己身上,摇着头,不时试试皮带金属环的热度。

传来几响爆炸声,这是司令部警卫营掩蔽所的手榴弹爆炸了。然后机枪子弹带里的子弹嗒嗒响了起来。一发德军的迫击炮弹在烟火中呼啸而过,在远处的伏尔加河上爆炸。河岸上有几个远远的人影在黑烟中闪过,看样子,是有人想把指挥所的火引开,转眼间一切又消失在烟与火之中。

克雷洛夫凝神望着四周流动的大火,已经不回想,不比较了……德国人会不会趁大火时候发起进攻呢?德国人不会知道我军司令部现在处在什么状态,昨天的俘虏还不相信我们的司令部在右岸呢……很明显,这是个别行动,就是说,有可能待到早晨没有事儿。只是千万不要起风。

他回头看了看站在一块儿的崔可夫,崔可夫正凝视着呼啸蔓延的大火;他那沾了许多黑烟子的脸好像火烧的,又像红铜铸的。他摘下帽子,拿手捋了捋头发,这一下子就像汗淋淋的乡村铁匠了;火星在他卷曲的头发上直蹦。他仰头看看呼呼响的烟火翻腾的天空,又回头看看伏尔加河,河上缭绕盘旋的烈火中隐隐出现了黑黑的缺口。克雷洛夫不由得想,自己担心的问题,司令员也在紧张地考虑着:德国人会不会在夜间发动大规模进攻?……如果能活到早晨,司令部往哪儿安?……

崔可夫感觉到参谋长的目光,便对他笑了笑,用手在头顶上画了一个大圈子,说:

“太漂亮啦,他妈的,不是吗?”

这场熊熊大火,在伏尔加河彼岸,在斯大林格勒方面军司令部所在的红色花园看得十分清楚,参谋长萨哈罗夫中将一收到有关大火的情报,就报吿了司令员叶廖缅科[6],总指挥请萨哈罗夫亲自前往电话总机和崔可夫通话。萨哈罗夫呼哧呼哧地喘着,急急忙忙顺着小路走去。副官打着手电筒,不时地提醒说:“将军同志,小心点儿!”并且不时用手推开挡在小路上的苹果树枝。远方的火光照耀着一棵棵树干,并且变成红色的斑点落在地上。这些晃晃不定的光斑使人心中惶惶不安。四周一片寂静,只能听到哨兵低沉的喝问声,这种情形使模糊而无声的火光显得特别可怕。

来到总机所在地,女值班员望着呼哧呼哧直喘的萨哈罗夫说,无法和崔可夫联系,电话、电报、无线电话都打不通……

“跟师里联系呢?”萨哈罗夫急忙问道。

“中将同志,刚才跟巴秋克通过电话。”

“要巴秋克,快点儿!”

女值班员战战兢兢望着萨哈罗夫,已经认定这位将军厉害又暴躁的脾气马上就要发作了,忽然高高兴兴地说:

“通了,将军,请吧。”她把话筒递给萨哈罗夫。

跟萨哈罗夫说话的是师参谋长。他像电话员姑娘一样,听到方面军司令部参谋长呼哧呼哧喘粗气,听到他的严厉的声音,胆怯起来。

“你们那儿情况怎么样,请汇报一下。能跟崔可夫通话吗?”

师参谋长汇报了油库起火的情况,汇报了大火扑向集团军司令部的情形,又说,师里无法跟司令员取得联系,还说,看样子,那儿的人没有全部牺牲,因为透过烟与火可以看到有一些人站在岸边,不过,不论从陆路还是在河上驾船都无法接近他们—伏尔加河烧起来了。巴秋克已经带着师部警卫连沿着河岸朝大火奔去,试图把火流引开,帮助站在岸上的人从大火包围中冲出来。萨哈罗夫听完师参谋长的汇报后说道:

“请转告崔可夫,要是他还活着的话,请转告崔可夫……”

他没有说下去。

电话员姑娘对这样长时间的停顿感到惊异,她等待着将军嗄哑的声音再响起来,用胆怯的目光朝萨哈罗夫看了看:将军依然站着,将手帕捂在眼睛上。

这一夜,有四十名司令部的指挥员在倒塌的掩蔽所里葬身火海。

十

在油库的大火之后,克雷莫夫很快就来到斯大林格勒。崔可夫把新的指挥所安在伏尔加堤岸脚下,在巴秋克师所属一个步兵团的防地上。崔可夫来到团长米海洛夫大尉的掩蔽所,看了看这宽敞的、用许多木头撑着的土室,满意地点了点头。这位司令员看着满脸雀斑的红头发大尉悲伤的脸,很快活地对他说:

“大尉同志,你造掩蔽所没有按规格办事,造得有点像元帅府。”

于是,团部便带上那简单的几件家具,迁到下游几十米的地方;红头发的米海洛夫也依样行事,毫不客气地把自己手下的一位营长挤走了。那位营长没有了住处,却没有再去挤自己的连长,因为他们住得已经够拥挤了,只叫人在高地上新挖了一个土室。

克雷莫夫来到第六十二集团军指挥所的时候,这儿的工兵作业正在紧张地进行,挖掘司令部各部门之间的交通壕,挖掘联系政工人员、业务人员和炮兵的大小地道。

克雷莫夫见过自己的司令员两次—他出来察看工程情况。

世界上也许没有任何地方像在斯大林格勒这样认真对待建造住所的事。在斯大林格勒造掩蔽所,既不是为了暖和,也不是为了让后来人佩服。能不能见到下一个天亮,活到下一顿午饭,主要取决于掩蔽所盖板的厚度、交通壕的深度、厕所的远近以及在空中是否能看到掩蔽所。

在谈到一个人的时候,都要谈他的掩蔽所。

“今天巴秋克的迫击炮在马马耶夫冈上干得漂亮……而且,他的掩蔽所也真不错,门是橡木的,特别厚,跟国会大厦的门一样,真是个聪明人……”

有时候,会这样说一个人:

“没说的,昨天夜里他转移了,丢了主要阵地,跟下属各部失掉了联系。他的指挥所在空中能看得见,用防雨布当门,可以说只能挡挡苍蝇。真是个没用的人,我听说,他老婆在战前就不跟他了。”

跟掩蔽所和土室有关的各式各样的传闻,在斯大林格勒多不胜数。有一个故事说,罗季姆采夫的指挥部所在管道里忽然涌进了水,师部人员一齐游上岸去,有人就开玩笑,在地图上标出罗季姆采夫指挥部冲进伏尔加河的地点。有一个故事说的是巴秋克那扇出了名的门如何被打掉的。还有一个故事,说饶鲁杰夫连同他的指挥部怎么给活活埋在拖拉机厂的掩蔽所里。

斯大林格勒的堤岸上密密麻麻排满了掩蔽所,克雷莫夫觉得这就像是一艘巨大的战舰:舰舷的一侧是伏尔加河,另一侧面对着连成一片的敌方火力网。

克雷莫夫接受政治部的委托,来解决罗季姆采夫师步兵团团长与政委之间的纠纷。他在动身来罗季姆采夫师部的时候,准备先向师部的军官们作一个报告,然后就来解决这件纠缠不清的事。集团军政治部一名勤务员把他带到一个宽阔管道的石砌洞口前,罗季姆采夫的师部就在里面。岗哨通报了方面军司令部派出的这位营政委的到来,就有一个低沉的嗓门儿说:

“叫他上这儿来吧,要不然还尝不到这儿的滋味呢。”

克雷莫夫在低低的拱顶下走着,感到指挥所里的人都拿眼睛看着自己,就向胖胖的团政委作了自我介绍。团政委穿着士兵棉军装,坐在罐头箱子上。

“啊,能听听报告太高兴啦,这可是好事儿,”团政委说,“要不然,我们听说,马内尔斯基,还有什么人,来到左岸,可是不打算上斯大林格勒我们这儿来呢。”

“另外,我还接受政治部主任的委托,”克雷莫夫说,“来解决步兵团团长和政委之间的事。”

“我们有过这样的事儿,”师政委回答说,“不过昨天已经解决了:有一颗一吨的炸弹落在步兵团的指挥所上,炸死十八个人,其中有团长,也有政委。”

他用坦然而随便的口气说:

“不知为什么他们一切都相反,就连外貌都截然不同:团长穿着朴素,他是农民的儿子;政委天天戴着手套,手上还戴着戒指。现在两个人躺在一块儿了。”

他是一个善于控制自己与别人的情绪而不受情绪影响的人,这时急忙换了口气,用快活的声音说:

“我们师驻守在科特鲁班山下的时候,有一次我开着自己的汽车送莫斯科来的巴维尔·费多罗维奇·尤金上前线去作报告。这位军委委员对我说:‘要是出什么差错,我砍你的脑袋!’我跟他受够了罪。一有飞机,我们马上就扎到排水沟里。我很小心,不想掉脑袋。不过尤金同志也很小心自己的性命,表现得很主动。”

听他们谈话的一些人微微笑着,克雷莫夫又感觉他的话里有令人不快的怜悯与嘲笑的意味。克雷莫夫平时跟队列指挥员的关系很好,跟参谋人员的关系也完全过得去,而跟自己的同行政工人员相处,往往感到很不痛快,常常不能以诚相见。现在这位师政委就使他很不痛快:才上前方没有几天,就自以为是老战士了,恐怕只是在战争前夕才入党的,也许还不知道恩格斯是什么人呢。

但是,看样子,克雷莫夫也有什么地方使师政委很不痛快。克雷莫夫一直有这种感觉。在副官给他安排住处的时候,请他喝茶的时候,都是这样。几乎每一个军事部门都有自己特殊的、与众不同的对人对事作风。罗季姆采夫师部里的人总是以自己的年轻将军为荣。克雷莫夫做完报告以后,大家就开始向他提问题。坐在罗季姆采夫旁边的师参谋长别尔斯基问道:

“请问,作报告的同志,同盟国究竟什么时候开辟第二战场?”

师政委半躺在紧靠管道石壁的窄窄的床铺上,坐起来用手扒了扒干草,说道:

“别着急。我更感兴趣的倒是我们的指挥部准备怎样行动。”

克雷莫夫很不高兴地瞟了师政委一眼,说:

“既然你们的政委提出这样的问题,那就不应由我来回答,应该由将军来回答了。”

大家一齐看了看罗季姆采夫。罗季姆采夫便说:

“高个子在这儿连腰都伸不直。一句话,这儿是管道。防守是可以的,再没有更大的优越之处了。从这种管道里发动进攻是不可能的。倒是希望发动进攻,可是在管道里无法调集后备兵力。”

这时候电话铃响了,罗季姆采夫抓起话筒。所有的人都朝他看了看。罗季姆采夫放下话筒,朝别尔斯基弯下身去,小声说了几句话。别尔斯基探身去拨电话,但是罗季姆采夫用手按住电话机,说:

“干吗,难道您没听见?”

在炮弹壳制的油灯那晃晃不定、烟气腾腾的灯光照耀着的管道里,在石头拱顶下,能听见很多声音。一阵一阵的机枪声在头顶上咔嗒嗒响,就像大车过桥。不时有手榴弹爆炸声。任何声音在管道里引起的共鸣声都非常响亮。

罗季姆采夫时而把这个参谋人员叫来,时而把那个参谋人员叫来,又把沉不住气的话筒拿到耳朵上。有一小会儿他注意到坐在不远处的克雷莫夫的目光,便亲切地像对自家人一样笑了笑,对他说:

“报告员同志,伏尔加的天气放晴了。”

电话不断地响起来。克雷莫夫听着罗季姆采夫在讲话,大致了解了发生的情况。年轻的副师长鲍里索夫上校走到将军跟前,俯下身对着放在箱子上的斯大林格勒地图,清清楚楚地画了一条垂直的粗粗的蓝线,穿过苏方防区的红色虚线,直到伏尔加河边。鲍里索夫用阴郁的眼睛意味深长地看了看罗季姆采夫。罗季姆采夫看见一个穿斗篷的人从幽暗中朝他走来,猛地站了起来。

看到来人的步子和脸上的表情,马上就明白了他是从哪儿来的。他浑身笼罩着一团肉眼看不见的火气,就好像在他那急急匆匆的动作中,不是斗篷在沙沙地响,而是这人浑身的电在哧啦哧啦地爆炸。

“将军同志,”他用埋怨的口气嚷道,“狗日的把我逼到冲沟里,逼到河边来啦。给我增援!”

“你要不惜任何代价把敌人阻挡住。我没有后备兵力。”罗季姆采夫说。

“是,不惜任何代价。”穿斗篷的人回答说。当他转身朝出口走去的时候,大家都看清楚了,他知道他将要付出什么样的代价。

“就在这一带吗?”克雷莫夫指了指地图上弯弯曲曲的河岸,问道。但是罗季姆采夫没来得及回答他。管道出口处响起手枪射击声,还有手榴弹爆炸的红色火光闪了几下。尖利的指挥官的哨声响起来。参谋长跑到罗季姆采夫跟前,叫道:

“将军同志,敌人朝我们指挥所冲来了!……”

多少有点卖弄自己的镇静语调、用彩色铅笔在地图上镇定地描画战局变化的师长忽然不见了。瓦砾场和荒草沟里的战争跟铬钢、阴极灯和无线电设备息息相关的感觉消失了。这个薄嘴唇的人很带劲地高声喊道:

“喂,全师部注意!检查一下自己的武器,带上手榴弹,跟我来,把敌人打回去!”

从他的声音中,从他又快又狠地在克雷莫夫身上扫过的目光中,流露出又冷酷又厉害的要打仗的狠劲儿。一时间使人觉得,这个人的主要力量不在于他的老练,不在于他的军事知识,而在于他的残酷、剽悍的气质。

几分钟之后,师部的军官、文书、通信员、电话员慌乱笨拙地拥挤着,从师部的管道里涌了出来,跨着轻快的步子跑在前面的是罗季姆采夫,他被一闪一闪的战火照耀着,朝冲沟奔去,爆炸声、枪声、呐喊声、骂声就是从那儿传来的。

等到克雷莫夫气喘吁吁地同前面几个人一起跑到冲沟边,朝下面一看,他的颤动的心里顿时出现了一种又憎恶、又恐怖、又痛恨的感情。沟底晃动着模糊的人影,射击的火花忽明忽灭,时而亮起绿眼睛,时而亮起红眼睛,钢铁的啸声在空中一个劲儿地响着。克雷莫夫看到的仿佛是一个巨大的蛇洞,千百条被惊动的毒蛇在里面咝咝乱叫,闪动着眼睛,在荒草丛里沙沙地、飞快地乱爬。

他带着愤怒、憎恶和临阵的惊惧,开枪射击黑暗中闪动的火光和在沟坡上快速爬动的人影。

在离他几十米的地方,德国人出现在沟沿上。接二连三的手榴弹爆炸声震荡着空气与大地。德军突击队正奋力冲向管道出口。

人影和射击的火光在黑暗中闪动,呐喊声、呻吟声时起时落。好像一口巨大的黑锅在翻滚,克雷莫夫整个身心都掉进这咕嘟嘟直冒泡的滚水中。他已经不能像原来那样思索和感触了。有时他觉得他还能操纵要把他卷进去的旋涡的转动,有时他充满死的预感,仿佛这树胶似的浓浓的黑暗在往他的眼睛和鼻孔里流,已经没有空气可以呼吸,头顶上也没有星空,只有黑暗、冲沟和在荒草中沙沙乱爬的怪物。

已经无法对战况作出判断了,可与此同时他透彻明白地感觉到,自己与那些在沟坡上匍匐爬行的人们休戚相关,感到自己与他们并肩作战。罗季姆采夫就在附近,这也令他感到欣慰。

在三步之外分不清是敌是友的夜战中产生这种奇异的感觉,往往跟另一种很难理解的奇妙感觉联系着,这就是对整个战斗进程的感觉,判断战斗中双方的实力,预测战斗的进程。

十 一

一个在烟火包围中脱离了群体的战士,处于茫然状态中凭直觉对整个战斗局势的判断,往往也比在司令部对着军事地图作出的判断更准确。

在战斗发生转机的时刻,有时会出现惊人的变化,这时候一直在进攻而且似乎已到达目标的士兵张皇四顾,再也看不见跟自己一起开始向目标挺进的战友,而他一直视为单枪匹马、愚蠢孱弱、经不住打的敌人竟成了浩荡的大军,因而是不可战胜的了。这种战斗转折的时刻,参战者能清楚地感觉到,而对于那些企图从表面去预测和理解的人来说却是神秘难测的。在这样的时刻,心理和精神会发生变化:勇猛而聪明的“我们”会变成胆小而脆弱的“我”,一度被看作区区猎物的倒霉的敌人,会变成可怕而强大的“他们”。

一路勇往直前、克敌制胜的战士能理解战斗中的一切情形:这里一枚手榴弹爆炸……那儿机枪在扫射……那个躲在掩体里打枪的人就要逃跑了,他不可能不跑,因为他是一个人,是单个儿的,跟那单个儿的大炮,跟那单个儿的机枪,跟他旁边也在单独作战的士兵不是一起的;可是我—就是我们,我就是这许多展开进攻的强大步兵,我就是这整个支援炮队,我就是所有支援坦克,我就是这照亮整个战场的信号弹。可是忽然之间我成了一个人;原来分散又经不住打的敌人,如今合成一个可怕的整体,步枪火力、机枪火力、炮兵火力都成了整体,再也没有什么力量帮助我战胜这个整体。唯一的办法就是逃跑,就是把头藏起来,把肩膀、额头、下巴缩起来逃命。

在黑夜里遭到突然攻击的人们,起初感到自己弱小、孤立。但他们一旦开始瓦解汹涌扑来的敌人的力量,就会感到自己也成为一个整体,胜利的力量就在这种整体的力量中。

在对这种转变的理解中,往往就包含着使军事有资格被称为艺术的东西。

感到孤单,感到强大,从前者到后者的意识转变,在这中间不仅包含着连队、营队夜战中各种事件的联系,而且表现出军队和民族军事实力的变化。

有一种感觉是参加战斗的人几乎全部丧失的,那就是时间的感觉。一个少女在新年舞会上狂舞了一夜,说不出她在舞会上待的时间是长还是短。

一个囚犯在牢狱里蹲了二十五年,会说:

“我在牢里好像过了一万年,又好像只过了短短的几个星期。”

少女³这一夜遇到许许多多转瞬即逝的事情—某处投来的目光,音乐的片断,微笑,轻轻的触碰—每一次都是那样短促,在感觉中留不下时间的长度。但这些短促的瞬间合在一起,便形成长时间的感觉,给她带来终生的欢乐。

囚犯的情形则相反,他在监狱的二十五年由许许多多长得使人难受的单位时间组成,如早点名到晚点名之间的时间,早饭到中饭之间的时间。但是这些痛苦的时间合在一起,却似乎产生了另一种感觉:因为一月又一月,一年又一年过得十分单调无味,时间因而简化了,缩短了……因此可以同时出现短暂的感觉和漫长的感觉,欢度新年之夜的人和在牢狱里过了几十年的人可以有相似的感觉。在两种情况下,许多事情糅合在一起,都会同时产生短暂与漫长的感觉。

一个人在战斗中体验的漫长与短暂,则是一个更为复杂的变化过程。在战斗中感觉到的变异更甚,个人最初的感觉常常被扭曲、颠倒。在战场上有时候秒变得很长,小时变得很短。漫长的感觉常常来自瞬间—炮弹与炸弹的呼啸,射击与爆炸的火光。

短暂的感觉有时来自长时间的事件—冒着炮火穿过崎岖不平的田野,从一个掩体向另一个掩体匍匐前进。肉搏战则是超出时间范畴的。那时候就连清醒也是模模糊糊,结果,整体与局部叠加,变得颠倒扭曲。

在这里,局部的事态是变化无穷的。

对于战斗时间的感觉变异极大,以至于这种感觉是完全模糊的,感觉漫长的不一定漫长,感觉短暂的也未必如此。

耀眼得令人看不见的强光,漆黑得令人看不见的黑暗,呐喊,爆炸声,自动步枪的嗒嗒声……在时间的感觉被打成碎片的混乱中,克雷莫夫极其清楚地意识到:德国人被打败了,被打退了。他和并肩作战的那些文书、通讯员一样,是靠内心感觉意识到这一点的。

十 二

黑夜过去了。烧焦的荒草丛中躺着一具具死者的尸体。河水在岸边发出悲凉的叹息。看到遍布弹坑的土地,看到烧毁的房屋的残壁,使人心中无限凄怆。

新的一天开始了,战争很大方地准备着—而且大方到极点—为新的一天准备足够的硝烟、瓦砾、钢铁以及肮脏而血腥的绷带。过去的一天天也是这样。除了这弹片炸翻的大地和烈焰腾腾的天空,世界上再也没有什么了。

克雷莫夫坐在箱子上,头靠着管道的石壁,打起盹儿。

他听着参谋人员含糊不清的声音,听见茶碗在响—师政委和参谋长在喝茶,用带着睡意的声音说话。他们说,被俘的德国兵是一名工兵,他们的工兵营是几天之前从马格德堡空运来的。克雷莫夫脑子里闪过小时候在课本里看到的一幅图画:戴尖顶帽的赶驮人赶着两匹大屁股的肥马,两匹马拼命要把粘在一起的两个屁股蛋儿挣开。小时候这幅画在他心里引起的乏味又浮上他的心头。

“这太好啦,”别尔斯基说,“就是说,后备队到啦。”

“是啊,当然很好,”瓦维洛夫附和说,“师部要反攻了。”

这时候克雷莫夫听到罗季姆采夫低沉的声音:

“花儿,花儿,果儿结在工厂里。”

克雷莫夫似乎把所有的精力在夜战中耗尽了。要想看到罗季姆采夫,必须转过头去,但是克雷莫夫没有转头。他想:“汲干了水的井会感到自己是空的,大概就是这样。”他又打起盹儿,低沉的说话声、枪声、爆炸声汇合成一种单调的嗡嗡声。

但又有一种新的感觉进入克雷莫夫的脑际,于是他又觉得自己仿佛躺在一个房间里,百叶窗开着,他凝视着射在壁纸上的晨光的一个斑点。那斑点爬到挂镜的边棱上,像彩虹一样扩散开来。一个小男孩的心颤抖起来,一个两鬓斑白、腰间挂着沉甸甸的手枪的人睁开眼睛,四下里看了看。

一个人身穿旧军装,头戴绿星的军帽,站在管道当中,在拉小提琴。

瓦维洛夫看到克雷莫夫醒来,俯下身子,对他说:

“这是我们的理发员鲁宾契克,拉得好极啦!”

有时候有人说两句开玩笑的粗话,毫不客气地把手风琴打断;有时候有人用压倒小提琴声的高嗓门儿问:“让我说说话,好吗?”便向参谋长汇报起来,小调羹在铁茶缸里叮当响着;有人打起长长的呵欠,“啊哈哈哈哈……”就扒拉起干草。

理发员细心地注意着:自己拉小提琴是不是妨碍军官们做事,准备随时停住不拉。

此刻克雷莫夫想起了白发苍苍、身穿黑色燕尾服的捷克著名小提琴家扬·库贝利克[7],为什么他觉得库贝利克也会拜倒在师部的理发员面前,自叹不如呢?为什么像小河流水一样简单的曲子,那纤细、颤抖的小提琴声,此时此刻似乎比巴赫和莫扎特更能表现出人的心灵的广度和深度?

克雷莫夫又一次感到孤独的痛苦。叶尼娅离开他了……他又一次痛苦地想,叶尼娅的出走是他一生的关键:他还在,但等于死了。她真的走了。

他又一次想,有许多可怕的、残酷无情的事应当对自己说说……不应该再羞怯,不应该再用手套捂着脸……

小提琴声似乎唤醒了他对时间的感觉。

时间好比是一方透明的境地,人在其中出现,活动,又消失得无影无踪……大批的城市在时间中出现又消失。时间把它们带来,又把它们带走。但是他头脑中出现的完全是另外一种特殊的时间概念。这种概念是说:“我的时间……不是我们的时间。”

时间进入人生,进入国土,生长在人生与国家生活中,可是等到时间离开,消失了,人还会在,国家还会在……国家还在,可是国家的时间逝去了……人还在,可是人的时间消失了。时间哪儿去了?人还在,还在呼吸,在思索,在哭泣,而时间,那唯有的、特有的、只跟他有关系的时间走了,逝去了,消失了,他还在。

最艰难的,是做时间的弃儿。不能生活在自己的时间中的弃儿,其命运是最痛苦的。谁是时间的弃儿,一下子就能辨认出来,不论是在干部处,在区党委会,在军队里的政治处,在报社,在大街上……时间喜爱的只是时间产生的那些人—自己的孩子、自己的英雄、自己的劳动者。时间永远、永远不会喜爱已逝的时间的孩子,就好比女人不爱过时的英雄,后娘不会疼爱前妻的孩子一样。

时间就是这样:不断地流逝,可依然生存着。一切都在,只有时间在不断地流逝。时间离去时多么轻盈,多么静悄。昨天你还是那样有信心,那样愉快,那样坚强,你还是时间的儿子。可是今天来了另一个时间,你还不了解它呢。

在战斗中被撕碎的时间,又从理发员鲁宾契克的小提琴里冒出来。小提琴告诉一些人,他们的时间来了,告诉另一些人,他们的时间要逝去了。

“逝去了,逝去了。”克雷莫夫想道。

他看着政委瓦维洛夫那平静而和善的大脸,瓦维洛夫不时地喝两口茶缸里的茶,用劲儿慢慢在就着香肠吃面包,他那一双令人看不透的眼睛转向管道口那个明亮的光斑。

罗季姆采夫瑟瑟缩缩地挺起披着军大衣的肩膀,带着宁静而开朗的面部表情对直地凝望着拉小提琴的人。担任师炮兵总指挥的白发苍苍的麻子上校皱着眉头,看着摆在面前的地图,因为皱眉头脸相显得似乎很凶,只有从他那忧伤而亲切的眼神可以看出来,他没有看地图,他是在听。别尔斯基飞快地写着给集团军司令部的报告;他似乎一心一意地在工作,但是他虽然在写,却歪着头,侧耳朝着小提琴。稍远处坐着不少红军战士,有通信员、电话员、文书,他们那疲惫的脸上和眼睛里露出严肃的表情,那种表情常常可以在嚼面包的农民脸上看到。

克雷莫夫忽然想起一个夏夜……年轻的哥萨克姑娘那一双大大的黑眼睛,她那火辣辣的情话……人生还是美好的!

等到小提琴一曲奏过,听到潺潺的流水声,是水在木板下流过,于是克雷莫夫觉得,他的心就像一口看不见的井,本来干了、空了,这会儿轻悄悄地流进水来。

半个钟头之后,小提琴手已经在为克雷莫夫理发了,并且用那种常常使人发笑的理发师的故意夸张的严重口气问,刮脸是不是把克雷莫夫刮疼了,又用手摸摸:两边腮是不是刮好了?在到处是灰土与钢铁的一片愁惨惨的气氛中,香水与香粉的气味显得分外不协调,分外别扭,分外凄凉。

罗季姆采夫眯起眼睛,把洒了香水和扑了香粉的克雷莫夫打量了一遍,满意地点点头,说:

“不坏,给客人理得很像样子。现在来把我修理修理。”

小提琴手那一双大大的黑眼睛充满幸福的神气。他打量着罗季姆采夫的头,抖了抖白布护巾,说:

“少将同志,两边鬓角是不是多少剪短一点儿?”

十 三

油库大火之后,叶廖缅科大将就准备动身上斯大林格勒来看崔可夫。

这一危险的行动没有任何实际意义。

不过,从人心和人道的角度来说,非常需要这样做。于是叶廖缅科用了三天时间等待渡河。

红色花园里的掩蔽所明亮的四壁显得十分宁静,苹果树枝的阴影在司令员清晨散步的时刻显得异常亲切可爱。

远处的轰隆声、斯大林格勒的火光与树叶的沙沙声、芦苇的诉怨声汇合到一起。这些声音合在一起,使人说不出地难过,因此司令员在清晨散步的时候常常唉声叹气,常常骂娘。

早晨,叶廖缅科把自己要去斯大林格勒的决定告诉了萨哈罗夫,并且要他代理司令事务。

他同送早餐的女服务员开了开玩笑,批准副参谋长飞往萨拉托夫去待两天,接受了一位野战军司令员特鲁法诺夫将军的请求,答应派兵轰炸罗马尼亚人强大的炮兵中心。他说:

“好啦,好啦,我给你远程轰炸机。”

副官们都在猜,为什么司令员心情这样好。是崔可夫那边有好消息?是在高频电话中谈得非常满意?还是收到了家书?

但是这类信息通常是不会不经过副官们的,莫斯科没有和司令员通电话,崔可夫那边来的消息不是令人愉快的。

吃过早饭,这位上将穿起棉军装,便去散步。副官帕尔霍敏柯走在离他十来步远的地方。司令员像往常一样不慌不忙地走着,挠了几下大腿,又朝伏尔加河看了看。

叶廖缅科走到正在挖地槽的一些劳动营士兵跟前。这是一些上了年纪的人,后脑勺都晒成了深褐色。他们的脸上流露出忧愁和不愉快的神情。他们一声不响地干着活儿,并且很生气地望着这个胖胖的、头戴绿色军帽、站在地槽边不干事的人。

叶廖缅科问道:

“同志们,请你们说说,在你们当中谁干活儿最差?”

劳动营的士兵们觉得这个问题来得正好,他们挖土已经挖厌了。大家一齐瞟了瞟其中一个汉子,那汉子把口袋翻过来,把烟末子和面包渣子倒在手心里。

“可以说,是他。”有两个人说,并且望了望其他的人。

“是这样,”叶廖缅科严肃地说,“就是说,是这个人。他是顶不行的啦。”

那名士兵老气横秋地叹了一口气,用郑重而和善的目光从下面朝叶廖缅科望了两眼,看样子,他以为发问的人问这样的话不是为了正经事儿,而是随便问问,为了说说玩儿,为了解闷,所以就没有插嘴。

叶廖缅科又问道:

“在你们当中谁干活儿最好?”

大家指了指一个白了头发的人。那稀稀的头发护不住头,头晒成了深褐色,就好像枯草遮不住阳光,土地被晒焦了。

“就是他,特罗什尼科夫,”有一个人说,“他真卖力。”

“他干活儿干惯啦,不干活儿简直不行。”另外有人说,就好像在替特罗什尼科夫表示谦虚。

叶廖缅科把手伸进裤子口袋里,掏出明晃晃、光闪闪的金表,很吃力地弯下身去,把表递给特罗什尼科夫。特罗什尼科夫莫名其妙地望着叶廖缅科。

“拿着,这是给你的奖励。”叶廖缅科说。他依然望着特罗什尼科夫,说:“帕尔霍敏柯,你发一份奖励通报。”

他继续往前走去,听到背后乱哄哄地响起许多兴奋的声音,挖土的士兵又赞叹又欢笑,祝贺干惯了活儿的特罗什尼科夫的意外收获。

方面军司令等待渡河已经等了两天。这几天跟右岸的联系几乎断了。能够开到崔可夫那边的快艇,在一路上有限的几分钟内就被打穿六七十个洞,开到岸边时已是洒满了鲜血。

叶廖缅科很生气,很恼火。

六十二号渡口的指挥官们听到德军的炮声,害怕的不是炸弹和炮弹,而是怕司令员发火。叶廖缅科觉得,德军迫击炮、大炮、飞机的狂轰滥炸,全怪那些少校们玩忽职守,全怪那些大尉们不灵活。

夜里,叶廖缅科从掩蔽所里走出来,站在离河很近的一个沙包上。红色花园的掩蔽所里,放在方面军司令面前的作战地图,在这里仿佛能听见轰隆轰隆的响声,看到弥漫的硝烟,散发着生与死的气息。

他仿佛看到了他亲手画的前沿阵地的火力线,看到了表示保卢斯[8]的军队冲向伏尔加河的一个个粗大的楔形,看到了他用有色铅笔画的防御中心和火器集中地点。但是,当他看着摊在桌上的地图的时候,他觉得自己有力量改变和推动战线,他能使左岸的重炮吼叫起来。在那里他感到自己是主人,是机械师。

在这里他的感觉完全不同了……斯大林格勒的火光,天空慢慢滚动的隆隆声—这一切惊心动魄,表现出不以司令员的意志为转移的巨大力量和势头。

在隆隆的炮声和爆炸声中,从工厂区传来隐隐约约的长长的呐喊声:啦啦啦啦啦……在斯大林格勒的步兵奋起反击的这种长长的呐喊声中,不光有示威的意味,也有悲伤与忧闷的意味。

“啦啦啦啦啦……”的声音在伏尔加河上扩散开去。这种战斗的“乌啦”声在夜晚寒冷的河面上、在寒冷的秋日星空下回荡着,好像渐渐失去了激昂的劲头儿,渐渐变化着,忽然在其中出现了另外的东西—不是激情,不是豪气,而是心灵的悲伤,那心灵好像在同可爱的一切告别,好像在呼唤自己的亲人醒来,从枕头上抬起头来,最后一次听听父亲、丈夫、儿子、兄弟的声音……士兵的忧伤紧紧压住上将的心。

习惯于督促作战的司令员,忽然被战斗吸引住了。他站在松散的沙上,像一个孤零零的士兵,大片的战火与轰隆声使他惊心动魄,他站着,就像成千上万的士兵站在那边的岸上那样。他觉得,领导人民战争,他的本事是不够的,他驾驭不了这场战争,指挥不了这场战争。也许,正因为有这种感觉,叶廖缅科将军在对战争的理解方面达到了最高的高度。

天快亮的时候,叶廖缅科乘快艇到达右岸。事先得到电话通知的崔可夫来到河边,注视着飞速前进的装甲快艇。

叶廖缅科缓步走下快艇,他那沉甸甸的身子压得搭在岸上的跳板一弯一弯的。他很不灵活地踩着岸边的石子,走到崔可夫跟前。

“崔可夫同志,你好。”叶廖缅科说。

“您好,上将同志。”崔可夫回答说。

“我来看看你们在这儿过得怎样。你似乎在油库大火中没有烧坏嘛。连胡子眉毛都还好好的。甚至还没有瘦呢。可见我们给你吃得还是不坏。”

“白天黑夜都坐在掩蔽所里,怎么能瘦呢?”崔可夫回答说。因为司令员说给他吃得不坏,他听到这话觉得不痛快,就回敬说:

“这算怎么回事儿,我在河岸上接待起客人来啦!”

果然,叶廖缅科听到崔可夫管他叫斯大林格勒的客人,真的生气了。等到崔可夫说“请赏光到寒舍一叙”,叶廖缅科回答说:

“我就在这新鲜空气里待一待挺好。”

这时候,对岸的大炮隆隆地响了起来。

河岸被大火、照明弹和爆炸的火光照耀着,而且显得非常空旷。亮光时弱时强,有时雪亮雪亮的,亮得刺眼。叶廖缅科注视着到处是掩蔽所和通道的堤岸,注视着堆在水边的石头,一堆堆石头从黑暗中露出来,又轻悄而敏捷地钻进黑暗中。

有一个粗大的嗓门儿缓慢而有力地唱着:

让满腔的义愤如波涛翻腾,

这是人民的战争,神圣的战争……

因为在岸边和堤坡上都看不到人,因为周围的一切,不论大地天空,不论伏尔加河,都被火光映照着,就觉得这节拍缓慢的歌儿是战争自己唱的,不是人唱的,是那沉甸甸的歌词从人们身边滚过。

叶廖缅科因为自己被面前的情景吸引住,感到不好意思起来:他真的像是到斯大林格勒的主人这儿做客来了。他很生气,因为看样子崔可夫知道他心里惶惶不安,所以才过河来,知道这位方面军司令在红色花园的干芦苇沙沙声中散步的时候心里有多少烦恼。

叶廖缅科向遭受火难的这一方战场的主人问起后备兵力的调度、步兵与炮兵的配合和德军在工厂区的集结情况。他提问题,崔可夫回答,因为应该回答上级首长的问题。

他们沉默了一会儿。崔可夫很想问:“历来防御都是很了不起的,但是进攻究竟怎样呢?”

可是他没敢问。叶廖缅科会以为斯大林格勒的防守者没有足够的耐心,要求卸肩上的担子。

忽然,叶廖缅科问道:

“你的父亲和母亲好像是在图拉州,住在农村里吧?”

“是住在图拉州,司令员同志。”

“老人家有信给你吗?”

“有信,司令员同志。父亲还在干活儿呢。”

他们对看了一眼,叶廖缅科的眼镜片被火光映红了。看样子,他们就要谈谈有关斯大林格勒的真正实质性问题了,这是他们两个独独需要谈的。可是叶廖缅科说:

“你大概想问我这个方面军司令经常被问到的问题—关于补充生力军和弹药的问题,是不是?”

此时此刻唯一有意义的谈话就这样一直没有开始。站在堤岸上的哨兵不时地朝下面望望。崔可夫听着炮弹的啸声,抬起眼睛,说:

“大概那个战士在想:哪儿来的这两个怪人站在河沿上?”

叶廖缅科嗯了一声,没有多理会。到了该告别的时候了。按照不成文的规矩,一个站在炮火下的首长要走,通常只是在下级一再要求他离开的时候。但是他对危险那样不在乎,就像根本没这回事儿似的,所以这些规矩也跟他无关。他毫不在乎、同时又很敏锐地随着飞过的一颗迫击炮弹的呼啸声转过头来。

“好啦,崔可夫,我该走啦。”

崔可夫注视着开走的快艇,在岸上站了一会儿。他觉得快艇后面拖着的一道白浪像一条白手绢,好像一个女子摇着白手绢向他告别。叶廖缅科站在甲板上,望着对岸。他像波浪似的在从斯大林格勒那边来的模糊的火光中悠荡着,而快艇驶过的河面似乎动也不动,像一片石板。

叶廖缅科烦恼地在甲板上踱来踱去。几十种习惯的念头出现在他的头脑里。许多新的任务摆在方面军司令部的面前。现在主要的是调集装甲部队,准备在左翼进行突击,这是最高统帅部交给他的任务。这事儿他对崔可夫一点也没有提。

崔可夫回到自己的掩蔽所,站在门口的哨兵、外室里的办事人员、应召前来的古里耶夫师的参谋长—所有听到崔可夫沉重的脚步声立即站起来的人都看出来,司令员的心情很坏。原因是不难猜想的。

因为各师兵力的消耗越来越大。因为在不断的进攻与反攻中,德军的楔形攻势不住地吞食斯大林格勒的土地。因为两个满员的步兵师最近刚从德国后方开到,集结在拖拉机工厂地区,正虎视眈眈地等待行动。

是的,崔可夫没有对方面军司令说出自己所有的烦恼、忧虑和担心的事。

但是不论崔可夫,不论叶廖缅科,当时都不知道这次会面不能令人满意的原因在哪里。主要是他们会面中有公事以外的东西,这东西当时他们两个人都不能说出口来。

十 四

十月的早晨,别廖兹金少校醒来,想了想妻子和女儿,又想了想大口径的机枪,听到他到斯大林格勒一个月来已经习惯了的轰隆声,便把士兵格鲁什科夫唤来,叫他打洗脸水。

“这水是凉的,照您以往的命令。”格鲁什科夫微笑着说。他想起别廖兹金每天早晨洗脸时的快活表情。

“老婆和女儿在乌拉尔,恐怕已经下雪了,”别廖兹金说,“她们也不给我来信,唉……”

“少校同志,会来信的。”格鲁什科夫说。

趁别廖兹金洗脸、穿衣的时候,格鲁什科夫向他汇报了这天早晨发生的一些事。

“一挺大口径机枪朝食堂扫射,把管理员打死了;二营副参谋长一出门,肩膀就被弹片打伤;工兵营弟兄们捞了不少被炸弹震昏的鲈鱼,有五公斤,我去看过;他们把鱼送给了营政委莫夫绍维奇大尉。政委同志来过,对我说,等您醒了,打个电话给他。”

“知道了。”别廖兹金说。他喝了一杯茶,吃了点牛腿肉冻,打了个电话给政委和参谋长,说要到各营里去看看,穿起军装,便朝门外走去。

格鲁什科夫把毛巾抖了抖,挂到钉子上,摸摸腰上的手榴弹,拍了拍衣兜,看烟荷包在不在,摘下挂在角落里的自动步枪,便跟着团长往外走。

别廖兹金从昏暗的掩蔽所里走出来,一遇到明亮的光线不由得眯起眼睛。一个月来已经很熟悉的情景又呈现在他的面前:一摊摊翻起的黄土,褐色的斜坡上到处是油污的帆布,帆布遮盖着一个个士兵的土室,土灶的烟囱里冒着一缕缕炊烟。上方是一座座掀去了房顶的黑黑的工厂厂房。

左边,离伏尔加河比较近的地方,是“红十月工厂”的高耸的烟囱,还有一些货车车厢拥挤在歪倒的机车旁边,就像一群发了呆的羊围着被打死的头羊。再远处是像宽花边似的已无人烟的城市废墟,秋日的天空化为无数个蔚蓝色的斑点,从一个个残破的玻璃窗口映照出来。工厂的厂房之间烟气腾腾,火光闪闪,明亮的空中一会儿响起长长的嗖嗖声,一会儿响起干巴巴的嗒嗒声,就好像工厂仍在照常开工生产。

别廖兹金细心地看了看本团三百米长的防地。防地从工人村的房屋中间穿过。他心里有种感觉,使他能够在乱糟糟的废墟和街道中分辨出来,红军战士在哪座房子里烧饭,德军士兵在哪座房子里吃腌肉,喝烧酒。

别廖兹金弯下头,骂了一句,一颗迫击炮弹在空中呼啸而过。

在对面的冲沟斜坡上,一股硝烟遮住一个掩蔽所的门口,霎时间响起剧烈的爆炸声。邻师的联络部长从掩蔽所里出来看了看。他没穿制服上衣,只穿着背带裤。他刚刚走了一步,又响起啸声,便赶紧退回去,把门关上。一颗迫击炮弹在十来米远处炸开来。巴秋克站在冲沟拐角处堤坡上一个掩蔽所的门口,看着眼前的情景。

等到联络部长又想往前走,巴秋克啊呀了一声,喊道:“炮弹!”德国人就像听到他的命令似的,又打了一发炮弹。

巴秋克发现了别廖兹金,高声喊道:

“你好,邻居!”

这样在荒凉的小路上走过,实际上是可怕的、送命的事。德国人睡足了觉,吃饱了早饭,特别有兴趣监视小路,见到什么人都打,决不心疼子弹。别廖兹金来到一个转弯处,在一堆废铁旁边站了一会儿,他看出前面一截路有危险,便说:

“格鲁什科夫,来,你头一个跑过去。”

“您怎么啦,这怎么行啊?他们的狙击手在那儿。”格鲁什科夫说。

头一个跑过危险地带,一向被认为是首长的特权。德国人往往来不及打第一个跑过的人。

别廖兹金看了看周围德国人盘踞的房子,对格鲁什科夫挤了挤眼睛,便朝前跑去。

等他跑到可以遮挡德军视线的土包跟前,背后“啾”、“啪”清清楚楚响了两声,这是德国人打了一颗爆炸子弹。

别廖兹金躲在土包下边,抽起烟来。格鲁什科夫大步快跑起来。一梭子子弹扫在他的脚下,好像一群麻雀从地上飞了起来。格鲁什科夫朝旁边一跳,踉跄一下,跌倒在地上,又跳起来,跑到别廖兹金跟前。

“差点儿叫他们扫倒。”他说。喘了几口气之后,又解释说:“我想瞅准这个时候:德国佬没打到您,一定会懊恼得抽起烟来。可是,看样子,这是一个不抽烟的家伙。”

格鲁什科夫摸了摸缝得马马虎虎的棉制服前襟,又骂了几声德国佬。

他们走近营指挥部的时候,别廖兹金问道:

“格鲁什科夫同志,什么地方伤着了吗?”

“打到我的鞋后跟,把后跟打掉啦,该杀的德国佬。”格鲁什科夫说。

营指挥部设在工厂食品店的地下室里,潮湿的空气中还有酸白菜和苹果的气味。

桌上点着两盏用炮弹壳做的高高的油灯。门口还钉着一块牌子:“买卖双方,以礼相待。”

地下室里驻着两个营指挥部,一个步兵营营部,一个工兵营营部。两位营长,鲍丘法罗夫和莫夫绍维奇,都坐在桌旁吃早饭。别廖兹金推开门的时候,听见鲍丘法罗夫很带劲儿的声音:

“我不喜欢掺水的酒,依我的口味,根本不用掺水。”

两位营长站起来,挺得笔直。参谋长把一小瓶伏特加藏在一堆手榴弹里,炊事员用身子把刚才莫夫绍维奇跟他谈过的鲈鱼挡住。鲍丘法罗夫的传令兵蹲在那儿,遵照自己的首长的吩咐正准备把唱片《中国情歌》放到留声机转盘上,也飞快地站了起来,只来得及拿下唱片,转盘依然在嗡嗡地空转。在该死的留声机转得格外起劲儿的时候,传令兵一面按照战士守则两眼向前直视着,一面用眼角捕捉鲍丘法罗夫凶狠的目光。

两位营长和一起吃早饭的其他人都深知首长们的偏见:首长们认为,营里的人要么作战,要么用望远镜观察敌人,要么对着地图考虑问题。可是人总不能二十四小时都打枪,不能二十四小时都跟上级和下级打电话,也要吃饭呀。

别廖兹金朝旁边瞟了瞟嗡嗡响的留声机,笑了笑。

“好啦。”他说。接着又吩咐:“请坐,同志们,吃你们的饭吧。”

这话可能是反话,不是他的真意。于是在鲍丘法罗夫的脸上出现了羞愧和认错的表情,因为莫夫绍维奇率领的是独立工兵营,不是直属部下,所以他的脸上只有羞愧,而没有认错的表情。他们各自的下属脸上的表情大致也可以这样分类。

别廖兹金又用极不愉快的腔调继续说:

“莫夫绍维奇同志,你们的五公斤鲈鱼在哪儿?这事儿全师都传遍了。”

莫夫绍维奇依然带着那种羞愧的表情说:

“炊事员,把鱼拿出来看看。”

炊事员在这儿是唯一在履行自己分内职责的,他直率地说:

“按大尉同志吩咐,已经照欧洲人的做法给鱼填馅,放了辣椒、桂叶,可是没有白面包,也弄不到洋姜。”

“好,知道啦,”别廖兹金说,“填馅的鱼我在一位叫萨拉·阿罗诺芙娜的女人家里吃过。说实话,我不怎么喜欢。”

地下室里的人一下子全明白了,团长压根儿就没想追究此事。

好像别廖兹金知道,鲍丘法罗夫夜里打退了德国人,天快亮的时候他被埋在土里,放《中国情歌》唱片的那名传令兵一面翻土,一面喊:“大尉同志,别泄气,一定能把您救出来……”

他好像也知道,莫夫绍维奇经常带着工兵在受坦克威胁的街道上爬,用黄土和碎砖把成棋盘状排列的反坦克地雷伪装起来……

他们的青春又高高兴兴地迎来一个早晨,又可以举起铜缸子,说:“来,祝你健康,干一杯!”又可以吃腌白菜,抽烟了……

本来嘛,什么事儿也没有。地下室的主人们只是在上级首长面前站了一小会儿,随后就请他一块儿吃起来,他们就快快活活地看着团长吃腌白菜。

别廖兹金常常拿斯大林格勒的战役跟往年的战争相比。他过去打过不少仗。他明白,他能经受得住这样的紧张状态,只是因为他心中平静镇定。战士们也正是因为这样,才能在这种似乎只能使人疯狂、使人恐怖或者使人疲惫的日子里喝菜汤,修鞋子,谈老婆,议论好的和不好的首长,做调羹……他看到,没有这种发自内心的镇定,不论在作战中多么剽悍勇猛,都不能长期经受这种紧张状态。别廖兹金觉得胆怯和怕死倒是一时的毛病,有点儿像伤风感冒,是可以治好的。

什么是勇敢,什么是胆怯,他实在说不清。战争开始的时候,有一次上级批评别廖兹金胆小,因为他自作主张带着一团人从德军火力包围中撤了出来。来斯大林格勒之前不久,他命令一位营长把人带到高地的另一面斜坡上,为的是不白白地挨德军迫击炮的打。师长却用责备的口气说:

“这是怎么回事,别廖兹金同志,原来我听说您是个勇敢而镇定的人呀。”

别廖兹金没有作声,叹了一口气。也许,这些人把他看错了。

鲍丘法罗夫有一头火红的头发,碧蓝碧蓝的眼睛。他好不容易克制着他那忽而发笑忽而又生气的习惯。莫夫绍维奇瘦瘦的,长长的雀斑脸,黑黑的头发里有几缕白发,用嗄哑的嗓门儿回答别廖兹金的问题。他掏出笔记本,画起他提出的受坦克威胁地段新的布雷方案示意图。

“把这图撕下来给我,让我好记住。”别廖兹金说。他俯到桌子上小声说:

“师长给我打过电话。集团军侦察队得到情报:德国人正在把兵力调出城区,集中兵力对付我们。坦克很多。明白吗?”

别廖兹金留心听了听附近的爆炸声,震得地下室墙壁直打颤。他笑着说:

“你们这儿还平静。在我那条冲沟里这段时间一定有三四个人从司令部里来过啦,各种各样的工作组不断地来。”

这时又一声爆炸,震得房子直摇晃,好几片石灰从天花板上落了下来。

“不错,是很平静,谁也没怎样干扰我们。”鲍丘法罗夫说。

“好就好在没人干扰。”别廖兹金说。

他很坦率地小声说着,真正忘记了他也是首长。他所以忘记,因为他做惯了下属,不习惯做首长。

“你们看,首长是怎么干的?为什么你不进攻?为什么没有占领高地?为什么有损失?为什么没有损失?为什么不汇报?为什么你睡觉?为什么……”

别廖兹金站起身来。

“咱们走,鲍丘法罗夫同志,我想看看你们的防地。”

工人村的这条街上一片凄凉景象。糊着各色花纸的房屋内墙触目皆是,花坛和菜园到处被坦克碾轧过,还有天知道为什么深秋还在开花的几株孤零零的大丽菊,都显得无限凄凉。

别廖兹金忽然对鲍丘法罗夫说:

“唉,鲍丘法罗夫同志,我老婆没有信来。我在路上碰到过她,可是现在又没有信了,我只知道她带着女儿上乌拉尔去了。”

“少校同志,会来信的。”鲍丘法罗夫说。

一座二层楼的半地下室里,在用砖头堵起来的窗户脚下,躺着一些伤员,等着到夜里往后方送。地上放着一桶水、一个茶缸,迎着门在两个窗户之间的墙上贴着一张小画《少校求婚》。

“这是后方,”鲍丘法罗夫说,“前沿阵地还在前面。”

“咱们也要上前沿去。”别廖兹金说。

他们穿过前厅,进入一个塌了天花板的房间,立刻有一种好像从工厂办公室进入了车间的感觉。空气中充满了火药令人不安的辛辣气味,子弹壳在脚下咯吱咯吱响。奶油色的摇篮里还堆着反坦克地雷。

“那座破屋昨天夜里被德国佬夺去了,”鲍丘法罗夫走到窗户跟前说道,“真可惜,那屋子挺不错,窗户朝西南,可以把我整个左翼控制在火力底下。”

在用砖堵起来、只留了窄窄的小孔的窗户旁边有一挺重机枪,机枪手没戴帽子,头上缠着肮脏的绷带,正在上弹带,一号射手露着白牙,正在吃香肠,准备过半分钟再扫射。

走过来一位中尉连长。他的军服上衣口袋里插着一枝白色的翠菊花。

“好样儿的。”别廖兹金笑着说。

“少校同志,能见到您,太好啦,”中尉说,“我昨天夜里对您说的,果然不错,他们又朝‘6—1’号房子进攻了。是九点正开始的。”他看了看表。

“团长在这儿,你向他汇报。”

“对不起,我没认出来。”中尉连忙行了一个军礼。

六天以前,敌人在该团的防区中切断了几座楼房之间的联系,并且开始按照德国人的作风认真地把这几座房子逐个蚕食。苏军枪炮的火光在一片瓦砾中熄灭,防守士兵的生命也随之熄灭。但是一座工厂楼房的地下室很深,苏联守军依然在这里抵抗。结实的墙壁没有被炮火摧毁,虽然有许多地方被炮弹打穿,被迫击炮打得坑坑点点。德国人想从空中把这座楼房摧毁,三次派鱼雷飞机来向这座楼投掷破坏力很大的鱼雷。

这座大楼各个角落都被炸毁了,但是地下室在一片瓦砾中安然无恙,守军清扫了震落的碎片,安好机枪、小炮,又开始反击。而且这座房子的位置很好,德国人还没有找到隐蔽的进攻通道。

向别廖兹金汇报的连长说:

“夜里我们曾经试着朝他们那儿去,没有成功,死了一个,两个负伤回来了。”

“卧倒!”这时观察哨的士兵厉声喊道。几个人就地卧倒。连长话还没有说完,就把两臂一挥,就像要跳水一样,扑通一声倒在地上。

啸声越来越尖利,突然变成震天动地、惊心动魄的轰隆声,爆炸发出又臭又令人窒息的气味。一根黑黑的粗木头咚的一声倒在地上,又蹦了两下,滚到别廖兹金的脚下。别廖兹金觉得炸下来的一小段木头差点儿砸在他的腿上。

他忽然看到,那是一颗没爆炸的炮弹。这一刹那间紧张情绪到了极点。

但是炮弹没有爆炸,而且那吞没天地、遮断过去、斩断未来的黑黑的阴影消失了。

连长站了起来。

“这条毒蛇。”不知是谁松了一口气,说。

另外一个人笑起来,说:

“我还以为这一下全完啦,把头都蒙上啦。”

别廖兹金擦了擦额头上忽然冒出来的汗,捡起地上的白翠菊花儿,抖了抖上面的砖瓦灰,别到中尉的上衣口袋上,说:

“算我送给你的……”

他又对鲍丘法罗夫说:

“为什么你们这儿还算平静,因为没有首长来。首长总是想向你要点儿什么:你有好炊事员,我就要你的炊事员。你有好手艺的理发员或者裁缝,我也要。什么便宜都要捞!你挖了好的掩蔽所,要让给我。你的酸白菜好吃,也要送给我。”

他忽然向中尉问道:

“为什么那俩人没到被围的弟兄们那边就回来了?”

“团长同志,他们负伤了。”

“明白了。”

“您是幸运的。”等他们从房子里走出来,穿过菜园的时候,鲍丘法罗夫说。菜园里,黄黄的土豆茎叶丛中,是第二连的战壕和一个个土室。

“谁知道我幸运还是不幸,”别廖兹金说着,跳进战壕,“在战场上嘛……”不过他说这话的口气就像在说:“在疗养院里嘛。”

“土地最能适应战争,”鲍丘法罗夫说,“土地已经习惯了。”

他又接起团长刚才的话头,说:

“别说炊事员,有时候首长连女人都要要去呢。”

整个战壕里闹腾起来,响起惊惶的呼唤声、噼噼啪啪的步枪声、短短的自动步枪扫射声和机枪扫射声。

“连长牺牲了,指导员索什金在指挥,”鲍丘法罗夫说,“这是他的掩蔽所。”

“明白了,明白了。”别廖兹金说着,朝掩蔽所半开着的门里面望了望。

在机枪旁边,红脸、黑眉毛的指导员索什金赶上他们,用特别高大的嗓门儿一个字一个字地报告说,连队现在向德国人开火,是想使他们不能集中力量向“6—1”号楼房进攻。

别廖兹金拿过他的望远镜,观察着一道道短短的射击火线和迫击炮喷出的火舌。

“瞧,三楼第二个窗户,好像有一个狙击手躲在那儿。”

他刚刚说过这话,他所指的那个窗户里闪起一阵火光,一颗子弹嗖的一声,打在战壕壁上,不偏不倚正在别廖兹金的头和索什金的头中间。

“您很幸运。”鲍丘法罗夫说。

“谁知道我幸运还是不幸。”别廖兹金回答说。

他们顺着战壕来看这个连发明的土法装置:反坦克枪用机枪脚架固定在大车轮子上。

“这是我们连的高射炮。”一个满脸灰尘和胡茬、眼神惶惶不安的中士说。

“坦克在一百米处,在那座绿顶小屋旁边!”别廖兹金用训练时的声调喊道。

中士很快地转了转车轮,反坦克枪长长的枪筒转向地面。

“德尔金那儿有一名战士,”别廖兹金说,“反坦克枪上装了狙击枪瞄准器,一天打坏三挺机枪。”

中士耸了耸肩膀。

“德尔金挺舒服,在车间里待着呢。”

他们又顺着战壕往前走,别廖兹金接着在巡视一开始就谈起的话头,说:

“我安排给她们寄了包裹,挺好的东西。可是,您瞧,老婆没有信来。老是不见回信。我甚至不知道,东西是不是寄到啦。也许,是不是病了?在疏散的时候少不了生灾害病。”

鲍丘法罗夫忽然想起,很久以前,常常有去莫斯科干活儿的木匠回到村子里,给父母、妻子和儿女带回不少礼物。他们觉得农村家庭生活的和睦和温暖比莫斯科的繁华、热闹和夜晚的华灯更有吸引力。

过了半个钟头,他们回到营指挥所,但是别廖兹金没有进地下室,就在院子里同鲍丘法罗夫告别。

“你们要尽一切可能支援‘6—1’号楼,”他说,“你们不要再派人上他们那儿去了,到夜里我们团里派人去。”

稍停,他又说:

“还有……我不喜欢你们那样对待伤员。你们指挥所里有沙发床,可是伤员却睡在地上。还有,你们也不去弄新鲜面包,大家都在吃干面包。这是第二。还有,你们的连指导员索什金醉得那样厉害。这是第三。还有……”

鲍丘法罗夫听着,感到吃惊:团长在防地上走了一下,怎么就全发现啦……还发现一名副排长穿着德国人的裤子……第一连连长手上戴着四只手表。

别廖兹金提醒说:

“德军会进攻的。明白吗?”

他朝工厂走去,已经钉上鞋后跟、缝好棉衣上绽线处的格鲁什科夫问道:

“咱们回去吗?”

别廖兹金没有回答他,只对鲍丘法罗夫说:

“打个电话给团政委,就说我上工厂第三车间,到德尔金那儿去了。”挤了挤眼睛,又说:

“给我送点儿腌白菜来,要好的。好歹我也是首长嘛。”

十 五

托里亚没有信来……每天早晨,柳德米拉·尼古拉耶芙娜·沙波什尼科娃送母亲和丈夫去上班,又送娜佳去上学。母亲第一个出门;她是有名的喀山肥皂厂化验室的化验员。亚历山德拉·弗拉基米罗芙娜从女婿的房间门口经过的时候,往往要说说她从厂里工人嘴里听来的那句笑话:“六点上班的是主人,九点上班的是职工。”

她出门之后,是娜佳走,说准确一点儿,她不是走,而是飞跑,因为没法子叫她按时起床,她都是在最后一分钟跳起来,抓起袜子、裙子、书、练习本,一面吃早点,一面咕嘟嘟地灌茶,然后一面下楼梯,一面围围巾,穿大衣。

等到娜佳走了,维克托·帕夫洛维奇·施特鲁姆坐下来吃早饭的时候,壶里的茶已经凉了,只有重新把茶烧一烧。

娜佳一说“顶好快点离开这个偏僻的鬼地方”,亚历山德拉·弗拉基米罗芙娜就要生气。娜佳不知道,杰尔查文[9]当年在喀山住过,阿克萨科夫[10]、托尔斯泰、列宁、济宁[11]、罗巴切夫斯基[12]都在这里住过,高尔基当年还在喀山的面包店干过活儿。

“怎么这样老化,这样麻木!”弗拉基米罗芙娜说。一个老奶奶这样责备一个十几岁的少女,听起来简直觉得奇怪。

柳德米拉看出来,母亲一如过去,乐于跟人打交道,对新的工作很感兴趣。她在心里赞赏母亲这种精神力量的同时,又有另外一种感觉:在这种苦难的时候,怎么还会对脂肪的氢化作用、对喀山的街市风光和博物馆感兴趣?

有一天,维克托对妻子说起弗拉基米罗芙娜的心是年轻的,柳德米拉憋不住,回答说:

“妈妈这不是年轻,是老年人的自我中心。”

“外婆不是自我中心,她是民粹派。”娜佳说。接着又补充说:“民粹派都是好人,但不是非常聪明的人。”

娜佳发表意见都用绝对的口气,而且,大概因为总感到时间不够,常用简短的形式。如说“胡扯”只说“扯”。她经常注意苏联情报局的战报,熟悉军事动态,爱谈政治。娜佳暑假期间去了一趟集体农庄,回来之后对妈妈大谈集体农庄劳动生产率不高的原因。

她在学校的分数一向不给妈妈看,只有一次很慌乱地说:

“妈妈,我的操行得了四分。可能因为有一次数学老师叫我离开教室,我一面往外走,一面扯着嗓门儿喊‘古德—呗!’引起了哄堂大笑。”

娜佳像许多殷实家庭的孩子一样,战前根本不知道操心柴米油盐的事,自从疏散到后方,却经常谈起口粮,谈凭票供应商店的好和坏。她还知道素油比牛油好,知道每一种荞麦粉的优缺点,知道吃块糖比吃砂糖划得来。

“你听我说,”她对妈妈说,“我想好了,从今天起,你给我喝的茶里加蜂蜜,不要再往里加炼乳。我看这样对我更好,对你还是一样。”

有时娜佳愁眉苦脸,用嘲笑轻蔑的态度对待长辈,说话粗鲁。有一天,她当着妈妈的面对爸爸说:“你是个糊涂虫!”而且口气那样凶狠,弄得爸爸不知如何是好。有时妈妈看到她一面看书一面哭。她认为自己是个落后的、不走运的人,命定要过艰难、不幸的日子。

“谁也不愿意和我交朋友,我太蠢,没有人喜欢我,”有一天她在饭桌上说,“没有人会娶我。等我上完了医药专科班,就上农村去。”

“在偏僻的农村里可没有药房。”弗拉基米罗芙娜说。

“关于嫁人的问题,你的估计过分悲观啦,”爸爸说,“近来你出挑得越来越好看啦。”

“算啦。”娜佳说着,狠狠地看了爸爸一眼。

夜里,妈妈常常看到,娜佳纤细光洁的手臂从被窝里伸出来,手里拿着诗集。有一天,娜佳用提包从科学院供应商店领回两公斤奶油和一袋大米,说:

“很多人,包括我在内,都是一些卑鄙下贱之徒,才用这种办法弄吃的。爸爸拿学问换黄油,也是没出息。就好像病人、没文化的人和没力气的孩子都应该过吃不饱的日子,因为他们不懂物理,或者不能超额百分之二百完成生产计划……只有上等人才能吃奶油。”

吃晚饭的时候,她又用挑畔的口气说:

“妈妈,给我两份蜂蜜和奶油,因为我早晨起晚了没吃到。”

娜佳有很多地方像爸爸。柳德米拉发现,最容易使丈夫生气的,正是女儿跟爸爸相像的一些地方。

有一天,娜佳简直像是模仿爸爸的口气,说起波斯托耶夫:

“骗子,饭桶,滑头!”

爸爸生气地说:

“你这个没出校门的中学生,怎么敢这样说一个院士?”

但是柳德米拉还记得,维克托上大学的时候,说到很多有名的院士,就说:“小人,饭桶,官迷,软骨头!”

柳德米拉明白,娜佳不会过得多么痛快,她的性格太古怪、孤僻,太不合群了。

娜佳走后,便是维克托喝茶,吃早点。他斜着眼睛看着书,嚼也不嚼就往下吞,脸上露出愚笨、惊愕的神情。他用手指头去摸茶杯,眼睛也不离开书本,说:

“要是行的话,给我倒一杯热点儿的。”

她熟悉他的一切动作:有时挠头,有时撅嘴,有时歪着脸剔牙,这时她便说:

“天啊,维克托,你什么时候去把牙齿治一治?”

她知道,他挠头、撅嘴,是在考虑自己的论文,完全不是因为头皮或者鼻子发痒。她知道,如果她说“维克托,你根本听不见我对你说的是什么”,他仍然会侧眼看着书,说:“我全能听见,还可以重复一遍:维克托,你什么时候把牙齿治一治?”然后又露出惊愕的神情,吞东西,像神经病人一样愁眉苦脸,这一切将意味着,他在评审一位熟悉的物理学家的论文的时候,有些地方他赞成,有些地方他不赞成。然后他会一动不动地坐上很久,然后开始频频地点头,不知为什么带着一副温顺的神情,像老年人那样的苦闷神情—害脑肿瘤的病人的脸上和眼睛里常常有这样的表情。柳德米拉又猜道:维克托是在想母亲。

当维克托在喝茶,思考自己的论文,唉声叹气,流露出苦闷神情的时候,柳德米拉望着她吻过的那双眼睛,她梳理过的那一头鬈发,那曾吻过她的嘴唇,那眉毛、睫毛,那一双手,她修剪过指甲的细细的手指头,嘴里说着:

“唉,你这个邋遢鬼!”

她知道他的一切,知道他临睡前爱在床上读儿童书刊,熟悉他去刷牙时脸上的表情,记得他穿着礼服,做有关中子辐射的报告时响亮而微颤的声音。她知道他喜欢乌克兰甜菜芸豆汤,知道他爱在梦中轻声呻吟,不住地翻身。她知道他的皮鞋后跟坏得多快,衬衫袖子脏得多快。她知道他爱睡两个枕头,知道他在穿过城市广场时提心吊胆。她知道他的皮肤气味,知道他袜子上的窟窿是什么样子。她知道他在饿了等着吃饭的时候爱哼哼小曲儿,知道他脚拇指上的指甲的形状,知道他两岁时母亲唤他的小名。她熟悉他沙沙的脚步声,知道他上高年级预备班时跟他打架的孩子们的名字。她知道他爱嘲笑人,爱逗弄托里亚、娜佳和同志们。就连现在,心情几乎总是十分沉重的时候,他逗她说,她的好朋友玛利亚·伊凡诺芙娜·索科洛娃读书太少,有一次在谈话时把巴尔扎克说成福楼拜。

他很擅长逗柳德米拉,她一听就要生气。现在她果然恼火了,言辞反驳,替女友辩护:

“你总是笑话跟我要好的人。玛利亚有自己的爱好,她不需要读很多书,她常常能感觉出书上说的事。”

“那当然,当然,”他说,“她相信《马克斯和莫里茨》是法朗士写的。”[13]

她知道他的音乐爱好,知道他的政治观点。她有一次看到他哭。她看到过他发疯似的撕自己身上的衬衣,一条腿被长衬裤绊住,只用一条腿蹦到她面前,举起拳头,做出要打人的样子。她看惯了他耿直无所畏惧的性格,熟悉他在灵感上来时的样子。她见过他朗诵诗歌,也见过他喝泻药。

她感到,丈夫现在对她有气,虽然他们的关系表面上一如往常。但是,已经有了变化,变化只有一点:他不再同她谈自己的论文了。他跟她谈朋友们的来信,谈食品与日用工业品定量供应。他有时也谈起研究所和实验室的事,谈工作计划的讨论情况,说说同事们的情形:萨沃斯季扬诺夫喝了一夜酒,一到研究所就呼呼大睡;试验员在墙根下煮土豆;马尔科夫准备进行一系列新的试验。

但是,他的论文,他的心事,以往只跟柳德米拉一个人谈的心事,现在缄口不言了。

他曾经对柳德米拉说,他把自己未考虑成熟的一些设想的笔记念给几个最要好的朋友听,第二天他就有一种不愉快的感觉,觉得写那篇论文没有意思了,很怕再去碰。

他只对一个人可以倾吐自己的疑虑,念片断的笔记,说出大胆而过于自信的设想,事后不会感到任何不快。这个人就是柳德米拉。

现在,他跟她也不再谈了。

现在,他在苦闷的时候,就指责柳德米拉,从中寻求解脱。他经常一个劲儿地想着母亲。想着以前从来不曾想过、如今法西斯使他不能不想的问题:想到自己的犹太血统,想到母亲是犹太人。

他在心里责怪柳德米拉,怪她对待他的母亲太冷淡。有―天他对她说:

“假使你跟母亲的关系能处得好,她会跟咱们一起住在莫斯科的。”

可她在心中数了数维克托对待托里亚粗暴的、不对头的地方。不用说,这类的事是不少的。

她一想起来心里就恼火,他对待她前夫的儿子那样不公道,把托里亚看得那样坏,那样不肯原谅他的缺点。可是娜佳又暴躁、又懒、又邋遢、又不愿意帮妈妈料理家务,他都可以原谅。

她想起维克托的母亲,她的境遇是很糟的。但是,维克托怎么能要求柳德米拉对安娜·谢苗诺芙娜好呢?要知道安娜·谢苗诺芙娜对待托里亚也不好。她每次来信,每次到莫斯科,都让柳德米拉觉得受不了。总是娜佳,娜佳,娜佳……娜佳的眼睛像维克托……娜佳兴趣广泛,娜佳机灵,娜佳喜欢动脑筋。安娜·谢苗诺芙娜疼爱儿子与溺爱孙女融为一体。可托里亚就连拿叉子的姿势也跟维克托不一样。

而且,很奇怪,近来她比过去更多地想起自己的第一个丈夫,也就是托里亚的父亲。她很想找到他的亲人,找到他的大姐,他们见到托里亚的眼睛,一定会十分高兴,阿巴尔丘克的姐姐一定会认出托里亚的眼睛、他弯弯的大指头、宽宽的鼻子是弟弟的眼睛、手和鼻子。

正如她不愿想起维克托对待托里亚的种种好处一样,她原谅了阿巴尔丘克一切坏的方面,就连他把她和吃奶的孩子扔掉,不准托里亚姓他的姓阿巴尔丘克,她也原谅。

上午柳德米拉一个人在家里。她盼望有这样的时刻,家里人常常打搅她的思绪。世界上的一切事情,战争,姐妹们的命运,丈夫的论文,娜佳的性格,母亲的健康,她对伤兵的怜惜,对在德国俘虏营中牺牲者的悼念—这一切都产生于她对儿子的思念,归根结底都是由于她为儿子担心。

她觉得,母亲、丈夫和女儿的感情是用另一种矿石熔炼成的。她感到,他们对托里亚的挂念和爱都不深。对她来说,整个世界就是托里亚;对他们来说,托里亚只是世界的一部分。

一天天过去,一个星期一个星期过去,托里亚没有信来。

每天电台广播苏联情报局的战报,每天报纸都满载战争消息。苏联军队不断撤退。战报和报纸上经常提到炮兵。托里亚就在炮兵部队。托里亚没有信来。

她觉得,只有一个人真正了解她的痛苦,就是索科洛夫的妻子玛利亚。

柳德米拉不喜欢同教授夫人们交往,她一听到她们谈丈夫的学术成就,谈服装,谈家里的保姆,心里就有气。但是,因为腼腆的玛利亚那温和的性格跟她的性格相反,因为玛利亚对待托里亚的态度使她很感动,所以她很喜欢玛利亚。

她跟玛利亚谈起托里亚比跟丈夫和母亲谈起来更随便,而且每次谈过之后心里都会轻松些,安宁些。尽管玛利亚几乎每天都要上她家来,然而她总是感到奇怪,为什么她的好朋友这么久没来,她不时地朝窗外望着,盼着玛利亚那瘦瘦的身影和好看的脸蛋快点儿出现。

托里亚还是没有信来。

十 六

弗拉基米罗芙娜、柳德米拉和娜佳都坐在厨房里。娜佳不时把练习本上的纸撕下来,揉一揉,丢进炉子里,奄奄一息的红红的火苗就会旺一会儿,炉子里满满一大堆维持不久的火苗。弗拉基米罗芙娜侧眼看着女儿,说:

“我昨天上一个化验员家里去,天啊,她家又穷,住得又挤,又没有东西吃,咱们家就像皇上过的日子了;她家来了一些街坊,闲谈起来,谈起在战前顶喜欢什么:有的说喜欢小牛肉,有的说喜欢腌黄瓜肉汤。那个化验员姑娘却说,她顶喜欢解除警报的信号。”

柳德米拉没有作声,娜佳却说:

“外婆,咱们家在这儿已经有好多好多朋友啦。”

“可是你一个也没有。”

“没有倒也好。”柳德米拉说。“维克托现在常常上索科洛夫家去。那儿常常聚集各种各样乱七八糟的人。我真不明白,维克托和索科洛夫跟这些人会一连扯上几个钟头……拿黄烟熏喉咙怎么也熏不厌。怎么一点不心疼玛利亚·伊凡诺芙娜,她还需要休息呢,可是有他们在那儿,她既不能躺一躺,又不能坐一坐,而且挨够了烟熏。”

“我很喜欢那个鞑靼人卡里莫夫。”弗拉基米罗芙娜说。

“那是一个讨厌的家伙。”

“妈妈跟我一样,她谁也不喜欢,”娜佳说,“就喜欢玛利亚阿姨。”

“你们都是怪人,”弗拉基米罗芙娜说,“你们有你们在莫斯科的生活环境,这种环境你们带到这儿来啦。在火车上,在俱乐部和戏院里,找不到你们圈子里的人—不是一个圈子,而是圈子套圈子,你们的朋友都是和你们在一个地方盖有别墅的一些人,这是我在叶尼娅那儿就观察到的……你们可以根据非常微小的特点判断是不是自己圈子里的人:‘哼,她真浅薄,连布洛克的诗都不懂;他真落后,连毕加索的画都不喜欢……哼,她居然送给他玻璃花瓶,太不雅致了……’不过维克托是民主派,他瞧不起一切陈腐的玩意儿。”

“瞎扯,”柳德米拉说,“这跟别墅有什么相干!那些粗俗的小市民,有别墅还是没别墅,跟他们没什么可交往的,讨厌。”

弗拉基米罗芙娜发现,女儿越来越容易向她发火了。

柳德米拉对丈夫提意见,教导娜佳,批评她的过错,也原谅她的过错,溺爱她,又不承认溺爱她。柳德米拉觉得母亲对她这些做法始终持保留态度。母亲没挑明自己的态度,但这种态度是存在的。有时维克托跟岳母交换一下眼色,他的眼睛里便流露出好笑和会意的神情,就好像他事先就跟岳母谈过柳德米拉性格的古怪了。他们谈没谈过,都没什么意义,问题在于家庭中出现了一种新的东西,这种东西本身的存在,改变了以往的家庭关系。

维克托有一天对柳德米拉说,如果他处在她的位子上,就让母亲当家做主,让她觉得自己是主人,不是客人。

柳德米拉觉得丈夫的话不是真心实意的,她甚至以为,他是想特别显示他对岳母的真心实意和与众不同的态度,好让柳德米拉很自然地想到她对婆婆的冷淡。

她有时因为丈夫爱孩子,特别因为他爱娜佳,产生嫉妒心。如果坦白对他说出这一点,那是好笑的,也是不好意思的。但现在不是嫉妒。怎么能承认,哪怕对自己承认,母亲无家可归,来到她家里栖身,惹她生气,使她感到是负担呢?而且这种气愤是很奇怪的,这种气愤和爱、和孝心一同存在,因为如有必要,她可以把最后一件衣服脱给母亲,跟母亲分食最后一块面包。

弗拉基米罗芙娜有时忽然感觉到,她很想无缘无故地哭上一场。有时她想死,想晚上不回家,在同事家的地板上过夜,有时忽然想收拾收拾,上斯大林格勒去,去找谢瘳沙、薇拉和斯捷潘·费多罗维奇。

弗拉基米罗芙娜在大多数情况下都赞成女婿的意见和做法,柳德米拉却几乎总是不赞成。娜佳发现这一点,就对爸爸说:

“妈妈欺负你,你找外婆说说去。”

这会儿弗拉基米罗芙娜就说:

“你们俩过得像猫头鹰一样阴沉惨淡。但维克托是个正常的人。”

“这都是空话,”柳德米拉皱着眉头说,“等到了回莫斯科的日子,您和维克托就快活了。”

弗拉基米罗芙娜忽然说:

“你可知道,我的好女儿,等到能够回莫斯科的那一天,我就不跟你们走了,我要留在这儿,我到莫斯科你们家里住着不舒服。你明白吗?我要劝叶尼娅搬到这儿来,或者我上古比雪夫,住到她那儿去。”

这在母女关系中是非常难堪的时刻。积压在母亲心中的不痛快,在她拒绝去莫斯科的话中一下子全表露了出来。柳德米拉心中的不痛快,这一下子也清楚了。但是柳德米拉委屈起来,就好像她一点也没有对不起母亲的地方。

弗拉基米罗芙娜望着柳德米拉痛苦的表情,也觉得内疚。夜里她想谢廖沙想得最多,有时想起他怎样发火,怎样争吵,有时想象着他穿起军装的样子,他的眼睛大概更大了,因为他可能消瘦了,两个腮瘪了下去。她对谢廖沙有一种特别的感情,因为他是她那个不幸的儿子留下的孩子。儿子也许是她在世界上最最钟爱的人……她有时对柳德米拉说:

“你别为托里亚那么难过吧,你要知道,我为托里亚担心也不次于你。”

在这番话里面有虚假的,与她对女儿的爱不相称的成分—她并不怎样为托里亚担心。就是这会儿,两个人都坦率到极点,却又害怕自己的直率,不承认自己的直率。

“《真诚可贵,互爱更重要》—这是奥斯特洛夫斯基又一部剧作。”娜佳说。

弗拉基米罗芙娜很不痛快,甚至带着一种恐惧的心情看了看这个十年级中学生:她自己还没有理解到的,这个中学生却理解到了。

没多久,维克托回来了。他用自己的钥匙开了门,一下子就来到厨房。

“可喜的意外,”娜佳说,“还以为你要在索科洛夫家里待到很晚呢。”

“啊,都在家里,都在炉子跟前,我很高兴,太妙啦,太妙啦。”他说着,把手伸向炉火。

“把鼻子揩一揩,”柳德米拉说,“有什么妙的,我真不懂!”

娜佳扑哧一笑,学着妈妈的语调说:

“喂,把鼻子揩一揩,你没听见吗?”

“娜佳,娜佳。”柳德米拉用警告的口气说。她不跟任何人分享教训丈夫的权利。维克托说:

“是的,是的。风太冷啦。”

他朝房间里走去,从开着的门里可以看到,他在书桌旁坐了下来。

“爸爸又在书的封面上写字了。”娜佳说。

“这不是你管的事。”柳德米拉说。又向母亲解释起来:“他为什么这样高兴?是因为我们都在家吗?他的心理是:如果有谁不在家,他会担心的。现在他还有问题要考虑,没有担心的事来分他的心了,所以他高兴。”

“轻点儿,要不然咱们当真要妨碍他了。”弗拉基米罗芙娜说。

“恰恰相反,”娜佳说,“要是大声说话,他根本就不注意,要是轻声细语,他就会走过来问:‘你们这是说什么悄悄话儿?’”

“娜佳,你说你爸爸,就像一位导游解说动物的习性。”柳德米拉说。

她们同时大笑起来,并且互相看了一眼。

“妈妈,您怎么能这样冤枉我?”柳德米拉说。

弗拉基米罗芙娜一声不响地抚摩了几下她的头。

然后他们就在厨房里吃饭。维克托觉得,这天晚上厨房里的温暖具有一种特别美妙的气氛。

他的生活基调一如既往进行着。近来他一直想把实验室中的一些彼此矛盾的试验结果弄明白。他坐在饭桌旁,有一神奇怪而幸福的急切感,他的手指头因为想去拿铅笔而急得哆嗦起来。

“今天的荞麦饭真好。”他用调羹敲着空碟子说。

“这是有所指吧?”柳德米拉问道。

他把碟嫀推到妻子跟前,问道:

“柳德米拉,想必你记得蒲劳脱的假说[14]吧?”

柳德米拉莫名其妙地拿起调羹。

“那是关于元素起源的。”亚历山德拉·弗拉基米罗芙娜说。

“噢,我记得,”柳德米拉说,“一切元素来源于氢气。不过,这跟荞麦饭有什么关系?”

“荞麦饭?”维克托反问道。“蒲劳脱的情形是这样的:他说出相当准确的假说,是因为当时在测定原子量方面存在着很大的错误。如果当时能够像杜马和斯塔斯[15]那样准确地测定了原子量,他就不会假设许多元素的原子量是氢的若干倍了。他之所以说对了,是因为他的错误。”

“可是,这究竟跟荞麦饭有什么关系呀?”娜佳问道。

“荞麦饭?”维克托惊异地问道。等他想起来,便说:“跟荞麦饭没什么关系……要弄清荞麦饭很难,要研究清楚,需要一百年。”

“这是你今天的报告的题目吗?”弗拉基米罗芙娜问道。

“不是,是随便说说,不是做什么报告,没什么用意。”

他捕捉到妻子的目光,感觉出来:她是明白的,明白他又一心一意想他的论文了。

“怎么样?”维克托问道。“玛利亚·伊凡诺芙娜来过吗?也许对你讲过巴尔扎克的作品《包法利夫人》吧?”

“去你的吧!”柳德米拉说。

夜里,柳德米拉一直等着丈夫跟她谈他的学术论文。但是他没有谈,她也什么都没有问。

十 七

维克托觉得十九世纪中期物理学家的想法太天真,亥姆霍兹[16]的观点太天真,他把物理学的任务归结为研究仅仅由于距离不同而产生的吸力和推力。

力场是物质的灵魂!能源波与物质微粒的联系与统一……光粒度……是光滴簇射还是闪电式波?

量子理论提出以新的定律(即概率定律)代替有关物理个体的一些定律;这是一些特殊统计学的定律,这种统计学抛弃个体概念,只承认总体。维克托觉得十九世纪的物理学家很像是一些染了胡子、身穿硬领硬袖口服装、聚集在台球桌周围的人。这些好深思的男子手拿尺子和怀表,皱着浓浓的眉毛,在计算速度与加速度,测量活跃在绿绒世界空间中的有弹性的小球的质量。

但是,用金属棒测量好的空间、用精密的怀表测定的时间忽然开始变异、拉长和收缩。空间与时间的稳定,不是科学的可靠基础,而是禁锢科学的牢狱。严厉审判的时刻来临了,几千年来的真理被宣判为迷误。真理就像在蚕茧里一样,在由来已久的偏见、谬误和失误中沉睡了许多世纪。

世界已是非欧几里得时代,世界的几何特点已经是用质量及其速度来表示了。世界一旦被爱因斯坦从绝对时间与空间的桎梏中解放出来,科学就以空前的高速度发展起来。

两股潮流:一股潮流是探索宇宙,另一股潮流是深入探索原子核的奥秘,这两股潮流各自朝前奔驰,而彼此又不失去联系,虽然一股潮流在秒差距世界中奔跑,另一股则以毫微米为计算单位。物理学家对原子核的研究越深入,越能明白星体发光的规律。在遥远星系的光谱中观察到红移现象,才产生了宇宙在无垠的空间渐渐扩散的概念。但是,只要认定空间是有限的、透镜状的,而且被速度和质量所扭曲,就可以设想是银河系之外的空间本身在扩张。

维克托毫不怀疑:世界上没有人比科学家幸福……有时候,比如早晨上班的路上,在晚上散步时,或者今天夜里这样思考自己的论文的时候,他充满了幸福、宁静、欣喜的感觉。

使银河系充满微弱的星光的力量,是在氢变为氦的过程中释放出来的……

战前两年,两个年轻的德国人用中子分裂了重原子核,苏联物理学家在自己的研究中用另外的办法得到了相似的结果,忽然体会到十万年前穴居的人类第一次生起火堆时的心情……

不用说,在二十世纪,物理学决定着主要方向……就像在一九四二年,斯大林格勒已成为世界大战各条战线中的主攻方向。

但是,在维克托·施特鲁姆身后,紧紧跟随着他的是怀疑、煎熬和不信。

十 八

维佳[17],我相信我的信能到你手里,虽然我在战线这边,在围了铁蒺藜的犹太人隔离区里。你的回信我是永远收不到的,我要死了。我希望你能知道我最后一些日子的情形,带着这种希望我会更轻松地离开人世。

维佳呀,真正了解人是很难的……七月七日,德国人进了城。在市公园里,无线电在广播最新的消息,我给病人看完病以后从门诊部出来,站下来听一听,女播音员在用乌克兰语播送一篇评论战事的文章。我听到远处的枪声,接着就有一些人从公园里跑过去,我便朝家里走去,感到惊讶不解,为什么我没有听到空袭警报笛声。我忽然看到一辆坦克,并且有人喊:‘德国佬打进来啦!’

我说:“别制造慌乱!”前一天我还去找过市苏维埃秘书,问他什么时候撤离,他生气地说:“这事儿还早得很,我们连名册还没造呢。”总而言之,是德国人来了。整个夜里,邻居们互相串来串去,最安静的是我和小孩子们。我打定主意:大家怎样,我就怎样。起初我很害怕,知道我再也见不到你了,多么想再看你一眼,吻吻你那额头和眼睛,可是后来我想,你在安全的地方,这是幸运。天快亮的时候,我睡着了。等我醒来,感到非常苦恼。我在自己的屋里,在自己的被窝里,可是感到自己犹如身在异国,孤孤单单,举目无亲。在苏维埃政权年代里我忘记了自己是犹太人,这天早晨,又使我想了起来。德国人站在汽车上到处大喊大叫:“打倒犹太佬!”

接着,有些邻居也叫我想起这一点。门房的老婆站在我窗前对一位女邻居说:“谢天谢地,这一下犹太佬完啦。”这是怎么回事儿呀?她的儿子娶的还是犹太女人,这个老奶奶常常去看儿子,还对我夸过她的孙子呢。

还有一个女邻居,是个寡妇,有一个六岁的女儿阿列娜,一双很美的蓝眼睛,过去我在给你的信里提到过的;她来到我这里,对我说:‘安娜·谢苗诺芙娜,请您把东西搬出去,今天晚上我搬到你屋里来。’‘好,我搬到你屋里去。’‘不,您搬到厨房后面那个小贮藏室里去。’

我没有同意。那个小贮藏室既没有窗户,又没有炉子。

我上门诊所去了。等我回来,一看:我的房门被砸开了,东西被扔到小贮藏室里。女邻居对我说:“我把沙发床留在我这儿了,反正您的新房间放不下。”

很奇怪,她还是职业学校毕业的,她去世的丈夫是一位会计,是一个很好、很老实的人。她说:“您是黑人口了。”那口气好像是在说:这对她是有利的。可是她的阿列娜整个晚上都坐在我这儿,我给她讲故事。这是我的新居,她不肯回去睡觉,是妈妈把她抱走的。后来,我们的诊所又开了,我和另一位犹太医生被解职了。我要求付给我本月的工资,可是新的所长对我说:“您在苏维埃政权下干的,让斯大林付给您工资吧,您可以写信到莫斯科向他要去。”护士玛露霞搂住我,小声哭起来:“天啊,您怎么办呀,你们怎么办呀。”特卡乔夫大夫也握了握我的手。我不知道,是幸灾乐祸,还是怜悯一个要死的浑身癞皮的老猫,那目光使人受不了。没想到我会有这一天。

有很多人使我吃惊。不光是没有知识、没有文化、得罪过的人。就像一位退休的七十五岁的老教师,过去常常问起你,要我转达他的问候,说你是“我们的光荣”。可是在这些可恨的日子里,他一见到我就转过脸去,连招呼也不打了。后来有人告诉我,他在警备司令部召开的大会上说:“空气清洁了,没有大蒜气味了。”他干吗要这样,这些话有损他的声誉。在那次大会上,有多少人在诽谤犹太人啊……不过,维佳,你自然会想到,不是所有的人都去参加那次大会。很多人没有去。你要知道,在我的印象中,从沙皇时代起,反犹太主义是跟“米哈伊尔天使长同盟”的克瓦斯爱国主义联系着的。在这儿我看到,那些叫喊把犹太人赶出俄罗斯的人,在德国人面前低声下气,奴颜婢膝,随时准备以三十个德国银币的代价把俄罗斯卖掉。郊区有些坏人来抢房子,抢衣服被褥;当年霍乱暴动时有些人杀死医生,大概就是这样的。有一些没骨气的人,对一切坏事都唯唯称是,生怕有人怀疑他们反对当局。

朋友们不断跑来报告消息,他们的眼睛像疯子的眼睛,人好像在迷迷糊糊的说胡话的状态中。出现了一句很奇怪的常用语:“转藏东西。”似乎藏在邻居家要保险些。我觉得转藏东西就像做游戏。

很快就贴出勒令犹太人搬迁的通告。只准许带十五公斤的东西。墙上到处张贴着黄色的通告:“一九四一年七月十五日下午六时以前,所有居民必须迁往老城区。”不搬迁者,格杀勿论。

于是,维佳,我也准备搬迁了。我带了一个枕头、几件衣服、你送给我的一个碗、一把调羹、一把小刀、两个碟子。一个人不也够了吗?我又带了几样医疗器械。带了你的信和一些照片,有去世的妈妈和达维德舅舅的照片,还有你和爸爸睡在一起的那张照片,带了普希金选集、都德的《磨坊书简》、莫泊桑的《一生》、一本小字典,还带了一本契诃夫的小说集,里面有《没意思的故事》和《黑衣教士》这两篇,这样,我的篮子就装满了。在这屋顶下,我给你写过多少信,夜晚在这里哭过多少回呀,现在我可以对你说说我的孤单了。

我向房子告别,向小园告别,在树下坐了几分钟,又向邻居告别。有些人实在奇怪。两个女邻居就当着我的面争论起谁要我的椅子,谁要我的书桌,等我跟她们告别,两个人都哭了起来。我恳求巴桑柯家的人,如果战后你来打听我的情况,请他们对你说详细一点儿,他们也答应了。最使我感动的是看家狗托比克,最后一个晚上它跟我特别亲热。

以后你要是来了,好好喂喂它,感谢它对我这样一个老婆子的亲热情谊。

等我收拾好了,就想:我怎么能把网篮提到老城呢?这时候,我的病人舒金来了。他平时愁眉苦脸,我之前觉得他是一个硬心肠的人。他帮我提东西,给了我三百卢布,并说每星期要给我送一次面包。他在印刷厂工作,因为眼病没有让他上前线。战前他在我那里看过病。以前如果有人要我说说哪些人心肠好,富有同情心,我会说出几十个名字,可是说不到他。你要知道,维佳,他来过以后,我才又感到自己是一个人,就是说,拿我当人待的不光是看院子的狗呢。

他对我说,市印刷厂里正在印通令:禁止犹太人在人行道上走;犹太人必须在胸前佩戴六角星黄色标记;犹太人不得乘车乘船,不能到澡堂洗澡,不能上医院、电影院,不准买黄油、鸡蛋、牛奶、水果、白面包、肉、除土豆以外的所有蔬菜;在市场上买东西只准许在傍晚六点以后,即在农民渐渐离开市场的时候。老城区围上铁蒺藜,不准外出,只能在监押下进行强制性劳动。如发现犹太人藏在俄罗斯人家里,罪同窝藏游击队,对窝藏者处以死刑。

舒金的丈人是农村的一位老汉,他从附近一个丘得诺夫镇上来。他亲眼看见,当地所有的犹太人都带着包袱和提包被赶进了树林,枪声和凄惨的叫喊声在树林里响了一整天。一个犹太人也没有回来。住在舒金丈人家里的德国人夜里很晚才回来,都喝得醉醺醺的,接着又喝到天亮。又喝又唱,还当着老头子的面分那些胸针、戒指、手镯。我不知道,这是偶然的一次暴行,还是也在等待着我们的厄运的前兆。

孩子呀,我前往中世纪犹太隔离区的一路上,多么伤心啊。我在城市里走着,这是我工作了二十年的地方。我们先是走在空荡荡的蜡烛街上。但是等我们来到尼科尔街上,就看到几百个人前往那被诅咒的隔离区。因为许许多多白包袱、白枕头,一条街都变白了。生病的便由人搀着。马尔古里斯大夫瘫痪的老父亲由两个人抬着。一个年轻人抱着老母亲,妻子和几个孩子背着包袱跟在后面。食品杂货店经理戈尔顿是个胖子,走得气喘吁吁,穿着皮领大衣,脸上的汗直往下流。有一个年轻人使我吃惊:他没有带东西,头抬得高高的,面前拿着打开的一本书,脸上是一副傲视一切和镇定的神气。但是跟他一起有多少吓疯了的人啊。

我们在马路上走着,许多人站在人行道上看。

有一阵子我跟马尔古里斯一家人走在一起,听到一些妇女同情的叹息声。有些人在笑穿皮大衣的戈尔顿,虽然他的样子很可怕,并不可笑。我看到许多熟悉的脸。有些人轻轻向我点头,跟我告别,有些人转过脸去。我觉得,在人群中没有完全平静的眼睛,有好奇的,有幸灾乐祸的,但是有几次我也看到哭红的眼睛。

我定神一看,看出面前有两种人。一种是穿皮衣戴皮帽的犹太男人和裹了毛头巾的女人。另一种是站在人行道上穿夏装的人。女人穿着淡颜色女衫,男人不穿外衣,有些人穿着绣花的乌克兰衬衫。我觉得,似乎太阳也不再为走在马路上的犹太人发光了,似乎他们走在寒冷的十二月的夜里。

在隔离区入口处我同送我的舒金告别,他给我指了指铁丝网边一块地方,说以后给我送东西就在那儿会面。

你可知道,维佳,我进了铁丝网,是什么样的感觉?原以为,我会十分害怕的。其实不然,在这种牲口圈里我心里倒是轻快些。决不是因为我有什么奴性。不是。决不是。周围都是跟我相同命运的人,在隔离区里我不需要像马一样在马路上走,没有恶意的目光,熟识的人用正眼看我,而不是躲避我。在这牲口圈里,大家都带着法西斯强加给我们的标记,因此在这里这种标记并不多么刺我的心。在这儿我感到自己不是任人宰割的牲口,而是落难的人。因此我轻快些。

我跟我的同事、内科大夫施佩林一同住在一套两居室的土坯房里。施佩林有两个成年的女儿和一个十二三岁的儿子。我有时看着这孩子痩瘦的小脸和忧伤的大眼睛,看了很久。他叫尤拉,可是有两次我喊他维佳,他给我纠正:“我是尤拉,不是维佳。”

人的性格多么不同啊!施佩林在五十八岁的年纪依然充满了精力。他弄到褥垫、煤油、一大车劈柴。夜里又弄来一袋面粉、半袋豆角。他不论弄到什么,都十分高兴,就像一个新婚的男子。昨天他又挂起壁毯。他一再地说:“没什么,没什么,咱们能挨过去。要紧的是准备些吃的和烧的。”

他对我说,应当在隔离区办学校。他甚至提出要我教尤拉法语,每节课报酬一碟子菜汤。我答应了。

施佩林的胖老婆凡妮·鲍莉索芙娜常常叹气:“全完啦,咱们完啦。”可是一面这样,一面监视着大女儿柳芭,防备她抓一把豆角或者掰一块面包送给别人。柳芭是一个善良而可爱的姑娘。妈妈喜欢的小女儿阿莉娅却坏到了顶点:又厉害,又多疑,又小气;常常骂父亲,骂姐姐。战争前夕她从莫斯科到这儿来探亲,就待在这儿没有走。

我的天,这周围多么穷啊!要是有人说犹太人有钱,说犹太人总是攒着钱准备过灾难的日子,那就请他上我们旧城区来看看吧!灾难的日子来了,再没有比这更大的灾难了。要知道,在老城里不光是带着十五公斤东西搬来的人,这儿还有长久的住户,有老匠人,有工人,有护士。他们住得多拥挤呀!吃得多么坏呀!更叫人难以想象的是一座座矮矮的、破破烂烂的土坯房!

维坚卡[18],我在这儿看到很多坏人—这些人又贪婪,又狡猾,甚至时时刻刻准备出卖一切投靠敌人。这儿有一个很可怕的人,名叫艾普什津,是从波兰一个小城来到我们这里的。他戴着袖章,常常跟德国人一起进行搜查,参加审讯,和乌克兰警察一起喝酒,他们派他到各家要酒,要钱,要东西。我见过他两次。这人高高的个儿,非常漂亮,穿着讲究的奶油色西装,就连缝在胸前的黄色六角星,也显得像黄黄的菊花。

不过,我还想对你说说别的事。我以往从来没感到自己是犹太人,我从小就生活在俄罗斯朋友的圈子里,我最喜欢的诗人是普希金和涅克拉索夫,在地方自治局派任医生的全俄代表大会上,我同观剧的代表一起为斯坦尼斯拉夫斯基主演的《万尼亚舅舅》流下眼泪。当年,维坚卡,当我还是一个十四岁女孩子的时候,我们家要动身迁往南美洲。我对爸爸说:“我决不离开俄罗斯,要不然我就投河。”所以我就没有走。

在这灾难的日子里,我心中充满了对犹太民族的母爱。以前我从不曾有过这种爱。好孩子,我觉得这种爱就像我对你的爱。我常常上病人家里去,小小的屋子里往往挤着几十个人:有半瞎的老人,有吃奶的孩子,有孕妇。我习惯在人的眼睛里寻找症候,青光眼症候,白内障症候。现在我不能那样看人的眼睛了—在眼睛里我看到的只是心灵的反映。维坚卡呀,都是美好的心灵!这是悲哀而善良,苦难而乐观,屈从于强权压制而又超越了强权的心灵。维佳,这是多么刚强的心灵!

你要知道,有些老头子、老奶奶多么关心地向我问到你呀。有些人多么热心地安慰我,虽然我从来没有对他们诉过苦,虽然他们的境遇比我更惨。

有时我觉得,不是我去给人治病,而是好心的人民这个医生在医治我的心灵。为了酬谢我的治疗,他们送给我一块面包、几个葱头或者一把豆角,这是多么令人感动。

维坚卡,你要知道,这决不是出诊费!有一次,一个老工人攥住我的手,一面往我的小包里塞几个土豆,一面说:‘唉,唉,大夫,请您原谅。’我的眼里涌出了泪水。这里面有一种纯洁、善良、可亲的东西,我还不能用言语表达出来。

我不想安慰你,说我现在过得很好;我的心并没有痛得撕裂成碎片,你可能会感到吃惊。但是你不要太难受,不要以为我挨饿,这段时间我还从来没有挨过饿。还有,我也不感觉自己是孤独的。

这儿的人究竟怎样呢?好也好得使我吃惊,坏也坏得使我吃惊。人与人大不相同,虽然都经历着同样的命运。电闪雷鸣的时候,大多数人都想方设法尽量躲避大雨,但是你要知道,这并不意味着所有人都一样。而且躲雨的方法也各有不同。

施佩林大夫相信,对犹太人的迫害是暂时的,是战争时期的事。像他这样的人是不少的。我看到,一些人越是乐观,器量越小,越是自私。如果在吃饭时候有人来了,阿莉娅和她妈妈都要赶紧把吃的东西藏起来。

施佩林对我态度很好,尤其因为我吃得很少,我带回来的东西总是吃不了。但是我决定离开他们,跟他们在一起很不舒服。我要另找安身的地方。一个人越是悲伤,越不指望活下去,就越是大方、善良,心肠越好。

那些命定要死的穷人、白铁匠、裁缝们,比起那些千方百计积攒吃食儿的人,要高尚得多,慷慨得多,也聪明得多。那些年纪轻轻的女教员、古怪的老教师和象棋高手施皮尔贝格、文静本分的图书馆女管理员、比小孩子还无用然而一直幻想制造土手榴弹把隔离区武装起来的工程师莱维奇,他们都是些多么古怪、多么不实际、多么可爱、多么悲伤、多么善良的人啊。

在这儿我看出来,希望几乎永远跟理智没有什么联系,希望不是出自理智,我觉得,希望出自本能。

维佳,人总是满怀希望地活着,就好像今后还要活很多很多年。无法知道这是愚蠢还是聪明,不过情形就是这样。我也服从这一规律。这里也有两个妇女从镇上来,也对我说了我的朋友舒金对我说的事。附近的德国人见到犹太人就杀,也不怜惜老弱妇孺。德国人和警察常常乘汽车来,抓几十名男子去挖土沟,过两三天,德国人把犹太人赶到土沟边,开枪屠杀,一个不留。城市周围的村镇到处出现这种掩埋犹太人的丘坟。

隔壁住着一个从波兰来的姑娘。她说,在波兰经常杀人,犹太人被杀得一个不留,只是在华沙、罗兹和拉多姆的几个隔离区里还有一些犹太人。我把这一切好好想了想,完全明白了:把我们集中在这里,不是为了像保护比亚沃维扎密林区的欧洲野牛一样把我们保护起来,而是为了便于宰杀。根据计划,再过一两个星期就轮到我们了。可是,你要知道,我虽然知道是这样,还是继续为病人看眼睛,并且说:“如果按时用药水洗眼睛,过两三个星期就会好的。”我还在观察着一个老头子的眼睛,过半年到一年就可以为他摘除白内障了。

我还在教尤拉法语,为他的发音不准伤脑筋。

在这里,德国兵常常撞进来抢东西,哨兵为了寻开心,常常在铁丝网外面开枪向孩子们射击,越来越多的人断言,我们的厄运随时会来到。

谁知,至今人们还活着。甚至不久前我们这儿还举行过婚礼。听到几十种传闻。有时,来一位邻居,高兴得喘着粗气说,我军转入反攻啦,德国佬跑啦。有时会飞来消息,说苏联政府和丘吉尔向希特勒提出了最后通牒,希特勒下令不要杀犹太人。有时又有消息说,要用犹太人交换德国战俘。

实在说,哪儿也没有像隔离区里这样多的期望。世界上有各种各样的事情,所有的事情,事情的主旨、起因总是一样的:都是为了解救犹太人。多么富有想象力的期望呀!

这些期望的来源都是一个,即求生的本能,这种本能不顾一切地否认那些一定要我们死绝的可怕的兆头。就像我,望着眼前的一切,就不相信:难道我们都是判了死刑在等死的人吗?理发匠、鞋匠、裁缝、医生、修炉匠,都在干活儿嘛。甚至还开设了小小的产科医院,说确切一点儿,是接生小屋。人们还在洗衣服,晒衣服,做饭,孩子们从九月六日起又上学了,做妈妈的又向老师打听孩子的分数了。

施皮尔贝格老头儿把几本书送去装订。施佩林家的阿莉娅每天早晨做早操,临睡前都要卷头发,跟爸爸争吵,向爸爸要两块夏装衣料。

我从早到晚都很忙,又看病,又教课,缝补衣服,洗衣服,准备过冬,往夹大衣里填棉花絮。我听着一件件犹太人遭殃的事:我熟识的一位法律顾问的妻子,因为给孩子买了一个鸭蛋,被打得失去知觉;药剂师西罗达的小孩子想从铁丝网下面钻出去,捡滚出去的皮球,哨兵开枪打穿了他的肩膀。然后是一个又一个的传闻。

终于传闻不再是传闻了。今天德国人赶着八十名年轻男子去干活儿,据说是挖土豆。于是有些人非常高兴,以为可以带几个土豆给家里人吃了。但我知道挖的是什么样的土豆。维佳,隔离区的夜晚是很特别的时间。孩子,你该记得,我常常教你对我说实话,儿子总是应该对妈妈说实话的。但是,妈妈也应该对儿子说实话。维佳,别以为你妈妈是刚强的人。我是软弱的人。我怕疼,一坐到牙科的椅子上就打哆嗦。小时候怕打雷,怕黑。老来我怕生病,怕孤独,怕我病了不能工作,成为你的负担,是你让我有这种感觉。我怕打仗。维佳,现在每天夜里我都很害怕,怕得心里直发冷。死神在等待着我。我很想向你呼救。

过去你是孩子的时候,常常跑到我跟前要我保护。现在,在我脆弱无力的时刻,多么想把头藏到你的膝盖上,让你这个又聪明又有力的儿子掩护我,保护我。维佳,我不是意志刚强的人,我很软弱。常常想到自杀。但我不知道,是软弱,是刚强,还是渺茫的期望,使我没有死。

不过,不说了。我一睡着了就做梦。常常梦见去世的妈妈,跟妈妈说话。昨夜我梦见萨沙·沙波什尼科夫,梦见当年跟他一起住在巴黎的情景。但是我一次也没有梦见你,虽然我时时想着你,特别是在恐怖不安的时候。这会儿我醒来,忽然看到这顶棚,想起德国人在我们的国土上,我变成了麻风病人,就觉得我并没有醒,而是睡着了,在做梦。

可是过了几分钟,就听见阿莉娅和柳芭争论该谁去挑水,听见有人在说,昨天夜里德国人在附近一条街上把一个老汉的头打穿了。

一个熟识的师范学校女学生来找我,要我去给人看病。原来,她掩护着一位肩膀受伤、又烧伤了一只眼睛的中尉。这个可爱的、痛苦不堪的小伙子说的是口音很重的伏尔加土话。昨天夜里他钻进铁丝网,在隔离区里找到了藏身之地。他的眼睛伤得不重,经过我治疗,就不会化脓了。他讲打仗,讲我们的军队撤退,使我难过起来。他想休息几天之后,就穿过前线到那边去。有好几个小伙子要跟他一块儿去,其中一个就是我的学生尤拉。啊,维克托,我要是能跟他们一块儿走该多好呀!我能为这个小伙子出一点力,实在高兴,觉得就好像我自己也参加了反法西斯战争。

一些人给他送来土豆、面包、豆角,有一个老奶奶还给他打了一双毛线袜。

今天一整天都处于十分紧张的状态中。昨天晚上阿莉娅通过她的俄罗斯女友弄到一个在医院死去的俄罗斯年轻姑娘的身份证。到夜里阿莉娅就要走了。今天,一个熟识的农民从铁丝网外面路过,我们听他说,被派去挖土豆的犹太人挖的是一些很深的坑,在离城四俄里的地方,靠近飞机场,就在去罗曼诺夫镇的路上。维克托,你记住这个地方,将来你可以在那儿找到合葬的坟墓,妈妈就在那里面。

就连施佩林也全明白了。他一整天脸色煞白煞白的,嘴唇不住地哆嗦着,慌乱地问我:“有技术的人是不是有希望活下来?”确实有人说,在有些镇上,一些好的裁缝、鞋匠、医生没有被杀害。

到晚上施佩林还是找来一个砌炉子的老头子,在墙上打了一个隐蔽的洞,收藏粮食和盐。晚上我和尤拉一起读《磨坊书简》。你该记得,咱们一起读我最喜欢的那篇《老人们》,那时候咱们互相看看,大笑起来,两个人都笑出了眼泪。然后我给尤拉指定后天要上的功课。需要这样。但是,我看着他那悲戚的脸,看着他抄写语法章节的手指头,我的心情多么沉重啊。

这样的孩子有多少呀。聪明的眼睛,黑黑的鬈发,在他们当中,应该有未来的学者、物理学家、医学教授、音乐家,也许还有诗人。

我看着他们每天早晨去上学,那种严肃的样子,完全不像孩子,瞪得大大的眼睛里流露着悲哀的神气。有时候他们也玩起来,打打架,哈哈大笑一阵子,然而并不因此就感到快活些,倒是更觉得可怕。

大家都说,孩子是我们的未来,但是这些孩子又怎样呢?他们再也不能成为音乐家、鞋匠和裁缝了。昨天夜里,我心里非常明晰,可以想象得到,这个由长髯飘飘、心事重重的老大爷和唠唠叨叨、做得一手好甜饼的老大娘构成的熙熙攘攘的世界,一切婚嫁习俗、民谚俚语、节日欢笑,很快就会消失得无影无踪。战争过后生活又会沸腾起来,可是我们不会再出现了,我们消失了,就像当年的阿兹特克人一样。

向我们报告挖坟消息的那个农民还告诉我们,昨天夜里他老婆哭着说:“他们又做裁缝又掌鞋,又制皮子又修钟表,又开药铺卖药……把他们全杀了,以后怎么办呀。”

我还清楚地想象到,将来有人从废墟旁路过,可能会说:“你该记得,这儿住过犹太人,住过修炉匠鲍鲁赫;礼拜六晚上他的老婆子常常坐在长凳子上,孩子们就在她的身边玩儿。”另一个人会说:“在那棵老梨树下面常常有一位女医生,我忘记她姓什么了,她给我治过眼睛,她干完活儿以后,总是搬一张藤椅,坐在那儿看书。”会是这样的,维佳。

就好像一阵可怕的气息从脸上吹过,大家都感到死期近了。

维坚卡,我想告诉你……不,不是这个,不是这个。

维坚卡,我这封信就要写完了,就要拿到铁丝网跟前,交给我的朋友。要给这封信收尾可是不容易的,因为这是我和你最后一次谈话,等我送出这封信以后,就要准备永远离开你,你再也无法知道我死前的情形了。这是我最后的告别。在永远分离之前,在告别的时候,我该对你说点什么呢?在这些日子里,正如在一生中一样,你是我的慰藉。每天夜里我都想起你,想起你小时候的衣服、你最初读的一些小书,想起你的第一封信、你上学的第一天,我一个劲儿地在回想,从你生下来的日子到最后一次收到你的信息,六月三十日的那封电报。我一合上眼睛,就觉得似乎你在保护着我,拦挡着即将来临的灾难。等我一想起周围发生的情况,又觉得庆幸,因为你不在我身边,免于劫难。

维佳,我总是孤身一人。在失眠的夜晚我常常难过得哭起来。可是这一点谁也不知道。一想到我还能对你说说我的一生,就感到快慰。我要说说,为什么我和你爸爸离婚,为什么很多年来我一个人生活。我还常常想,等维佳知道了他的妈妈犯过错误,做过不理智的事,曾经争风吃醋,曾经跟所有的年轻人一样,会感到吃惊的。但是等不到跟你好好说一说,就要孤单单地了结此生了,这是我的命运。有时我觉得,我不应该离你这样远,我太爱你了,我以为,我这样爱你,就应该跟你在一起安享晚年。有时我又觉得,我不应该跟你生活在一起,我太爱你了。

好啦,最后……祝你永远幸福,跟你所爱的人、你周围的人、比妈妈更亲近的人在一起,永远幸福!永别了!街上传来妇女们的哭声、警察的喝骂声,可是我看着这一页页的书信,就觉得我被保护了,这苦难深重的可怕世界奈何不了我了。我怎么能结束这封信啊?孩子,哪能甘心到此结束?哪儿有人类语言,能够表达我对你的爱?吻你,吻你的眼睛,你的额头、头发。你要记住,在幸福的日子里,在痛苦的时候,都有母爱伴随着你,任何人不能把母爱杀死。我的好维佳……这就是妈妈给你最后一封信的最后一句话。活下去,活下去,永远活下去……

十 九

战争爆发前维克托从来没有想到他和母亲都是犹太人。不论在小时候还是上大学时期,母亲都没有跟他说起这一点。他在莫斯科大学的那几年里,没有一位同学、一位教授、一位班级领导跟他提过这种事儿。

战前不论在研究所还是在科学院里,从来没听到有人谈这种事儿。他从来也没有想到要跟娜佳谈谈这种事儿—对她说一说,她的母亲是俄罗斯人,父亲是犹太人。

爱因斯坦和普朗克[19]时代竟成了希特勒时代。秘密警察和科学昌盛同时出现。十九世纪,质朴物理学的世纪,与二十世纪相比,多么人道!二十世纪杀死了它的母亲。法西斯主义的原理和现代物理学的原理有可怕的相似之处。

法西斯主义根本没有个性的概念,没有“人”的概念,把一切看作大规模的总体。现代物理学谈的是物理个体的这种或那种总和中出现一些现象的最大与最小可能性。难道法西斯在其可怖的秘密机构中奉行的不也是量子政治和政治概率论吗?

法西斯主张消灭居民中一些阶层,消灭一些民族和种族,其根据是在这些阶层和民族中,人们进行公然和隐蔽的反抗的概率大于其他阶层和民族。只讲概率和整体。

不过,当然不能这样!毫无疑问,法西斯之所以一定会灭亡,正因为它将原子和砂石的规律应用于人类。

法西斯和人类不能共存。法西斯要是胜利了,人类将不再存在,只剩下一些实质已经改变的人形皮囊的动物。等到富于理性和良知的人类胜利了,法西斯就会灭亡,被压迫者又会重新成为人。

这不等于承认契贝任关于发面桶的说法吗?今年夏天他还和契贝任争论,反对这种说法。他觉得,那一次同契贝任谈话已经过了很长很长时间,从那个莫斯科的夏日黄昏到今天,似乎已经有几十年过去了。

似乎那不是维克托·施特鲁姆,而是另一个人走在当时的喇叭广场上,激动地倾听,信心十足地热烈地争论。

母亲……玛露霞……托里亚……

有时候,他觉得科学是欺骗,使他看不见现实生活的疯狂与残酷。

也许,科学成为可怕的时代的同伴,成为其盟友,不是偶然的。他感到多么孤独啊。没有人跟他谈谈自己这些想法。契贝任离得很远。波斯托耶夫会感到这一切很奇怪,没意思。

索科洛夫倾向于神秘主义,对于暴虐者的残酷与凌辱表现出一种奇怪的宗教式的顺从情绪。在他的实验室工作的是两位卓越的科学家,一位是实验物理学家马尔科夫,一位是又放荡又聪明的萨沃斯季扬诺夫。但是如果维克托跟他们谈这些事,他们会认为他是疯子。

他从抽屉里拿出母亲的信,又读起来。

“维佳 ,我相信我的信能到你手里,虽然我在战线这边,在围了铁蒺藜的犹太人隔离区里……孩子,哪能甘心到此结束呀?……”

仿佛一把冰冷的尖刀戳进他的咽喉……

二 十

柳德米拉从信箱里抽出一封军邮信。

她大步走进房间,把信封对着亮光,从老大的信封上撕去一条边儿。

有一刹那她觉得,从信封里抖搂出来的将是托里亚的相片:小小的,脖子还擎不住头,光着屁股躺着,两条小腿像狗熊一样盘着,撅着小嘴。

不知怎的,她似乎不是在看信,而是在专心吸取那一行行文字,那是文化不高的写信人特有的工整字体。吸着吸着,她明白了:他活着,活着!

她弄清楚了,托里亚的胸部和腰侧受了重伤,流了很多血,身体十分虚弱,自己不能写信,四个星期以来一直在发烧……可是,幸福的泪水遮住了她的眼睛,一会儿之前她还是多么绝望啊!

她走到楼梯上,看过了信的前面几行,便放心地朝柴棚子里走去。她在寒冷而幽暗的柴棚里看完了信的中间和结尾部分,这才想到,这信是临死前跟她告别。

柳德米拉把劈柴往麻袋里塞。虽然她过去常常就诊的莫斯科加加林胡同门诊所的医生嘱咐她不能举三公斤以上的东西,而且只能做缓慢而从容的动作,这一次她却像个农妇一样,哼哧一声,把满满一麻袋湿劈柴扛到肩上,一口气上了二楼。她把麻袋往地上一放,桌上的碗盏叮叮当当乱晃起来。

柳德米拉穿起大衣,裹上头巾,来到街上。

行人从她身边走过,又回过头来看她。

她穿过大街,一辆电车发出尖利的铃声,电车司机朝她扬了扬拳头。

如果向右一拐,就可以顺着一条胡同到母亲工作的工厂去。

如果托里亚死了,他的父亲也不会知道,到哪一个集中营里找他去呀,也许,他早就死了……柳德米拉朝维克托的研究所走去。走到索科洛夫家门前,顺步走进院子,敲了敲窗子,窗帘依然没有拉开—玛利亚不在家。

“维克托·帕夫洛维奇刚刚回自己房间了。”有一个人对她说。她也道了谢。虽然她没弄明白是谁跟她说话,是熟识的人还是不熟识的人,是男人还是女人。于是她顺着试验大厅朝前走去,大厅里像往常一样,似乎很少有人在干事情。总觉得这儿的男人或者在聊天,或者抽着烟在看书,女人总是忙活着:用烧瓶煮茶,用溶剂洗指甲,或者织毛衣。

她看到一些小东西,几十样小东西,还看到试验员卷烟用的纸。

来到维克托的工作室里,几个人大声跟她打招呼,索科洛夫快步朝她走来,几乎是跑到她跟前,摇晃着一个老大的白信封,说:

“咱们有希望啦,这是回迁的计划和安排,要咱们带着所有的东西、仪器设备和家小回莫斯科去。不坏吧?虽然日期还没有定下来。不过总是有这回事儿!”

她觉得他那喜洋洋的脸和眼睛是可憎的。难道玛利亚会这样欢欢喜喜跑到她跟前吗?不会,不会。玛利亚一下子就会明白的,看到她的脸就完全能看出来。

要是知道她在这里会看到这么多喜洋洋的脸,她肯定不会来找维克托的。维克托也是高兴的,到晚上他会把高兴带回家里去,娜佳会感到幸福的,他们就要离开可憎的喀山了。

这种欢喜是青春的鲜血换来的。所有的人,不论多少人,能抵得上这青春的鲜血吗?

她带着责难的神情抬眼望着丈夫。

他那一双会意的、充满不安神气的眼睛也望着她的一双阴沉的眼睛。

等到剩下他们两个人,他告诉她,刚才她一进来,他就知道出事了。

他看完了信,一遍又一遍地说:

“没法子呀,天啊,没法子。”

维克托穿起大衣,他们便朝门口走去。

“我今天不来了。”他对索科洛夫说。索科洛夫正跟新派来的人事处长杜宾科夫站在一起。杜宾科夫高高的个子,圆圆的脑袋,肥大而讲究的上衣裹在宽阔的肩膀上依然显得紧巴巴的。

维克托把柳德米拉的手放开一小会儿,小声对杜宾科夫说:

“我们想着手编迁回莫斯科的表单,但今天不行了,以后我再告诉您。”

“维克托·帕夫洛维奇,不用操心,”杜宾科夫低声说,“目前还不必着急。这是将来的计划,一切草拟工作由我来干。”

索科洛夫招了招手,点了点头,维克托便知道索科洛夫已经猜到他又遇到难过的事儿了。

冷风在大街上飞驰着,卷起一股股灰尘,忽而像绳子一样滴溜溜绕圈儿,忽而一下子撒开去,就好像扔掉不能吃的发黑的粮食。冷风飕飕,树枝像敲骨头一样嘎嘎直响,电车轨泛着寒冷的青光,一派凛冽肃杀景象。

柳德米拉转过脸来。冻僵的、消瘦的脸因为痛苦显得年轻了。她朝着丈夫,用祈求的目光望着他。

他们过去养过一只猫,初次生崽就难产死了。这猫在濒死之时,慢慢爬到维克托跟前,呜咽着,瞪大发亮的眼睛望着他。可是,在这无边无涯、空荡荡的天空下,在这无情的、灰尘滚滚的大地上,又能向谁恳求、向谁祈祷呢?

“这是我工作过的军医院。”她随口说。

“柳德米拉,”他忽然说,“你上军医院去一下,可以弄清楚这封军邮信是从哪儿来的。以前怎么没有想到呀!”

他看着柳德米拉上了台阶,跟值班人员交谈起来。

维克托走到角落里,后来又回到军医院门口。行人匆匆走过,大都带着网兜和玻璃罐,玻璃罐里盛着灰色的菜汤,菜汤里游荡着灰色的通心粉和土豆。

“维克托。”妻子喊他。

他从她的声音听出来,她已经镇定下来了。

“是这样的,”她说,“这是从萨拉托夫来的。不久前一位副主任医生上那儿去过。他把那儿的街道和门牌号写给我了。”

马上出现了许多事情和问题:什么时候轮船开到,怎样能买到船票,要带一些吃的用的,要借钱,要弄一封证明信。

柳德米拉·尼古拉耶夫娜走的时候既没带用的,也没带吃的,甚至没带什么钱,也没有票,是趁上船时又挤又乱,挤上去的。

她带走的只是在黑暗的秋日黄昏同母亲、丈夫、娜佳分别时的印象。黑黑的波浪在舷边喧响,下游来的风吹打着,呼啸着,掀起一阵阵水珠和飞沫。

二十一

乌克兰敌占区一个州的州党委书记杰敏季·特里福诺维奇·格特马诺夫被任命为坦克军的政委,这个坦克军是在乌拉尔组建的。格特马诺夫在赴任之前,乘飞机飞往乌法,他的家小疏散在那里。同志们和乌法的工作人员都十分关怀他的家小:生活和居住条件都不坏。格特马诺夫的妻子加林娜·捷连季耶芙娜在战前因为新陈代谢不好,特别肥胖,在疏散期间还是没有瘦下来,甚至又多少胖了一些。两个女儿和一个还没有上学的儿子显得非常健康。

格特马诺夫在乌法过了五天。临走前亲友们来送别:有他的小舅子尼古拉·捷连季耶维奇,乌克兰人民委员会办公室副主任;有他的老同志、基辅人马舒克,保安机关干部;有他的连襟萨盖塔克,乌克兰中央宣传部的负责干部。

萨盖塔克来时已经十点多钟,这时候孩子们已经睡了,大家说话的声音很小。格特马诺夫说:

“同志们,咱们要不要喝点儿莫斯科酒?”

格特马诺夫身上的部件都是很大的:斑白蓬松的大脑袋,额头十分宽阔,鼻子又肥又厚,手大,指头粗,肩膀宽厚,脖子粗壮。但是他作为各个粗大部件的组合体,个头儿却不大。而且奇怪的是,在他那张大脸上,特别吸引人和令人难忘的是那一双小小的眼睛:窄窄的,勉勉强强从肥厚的眼皮底下露出来。眼睛的颜色不很分明,很难断定主要是灰色还是蓝色。但是那眼睛极其敏锐、灵活,有很强的洞察力。

加林娜·捷连季耶芙娜轻快地站起她那沉重的身子,从房间里走了出来,于是男子们静了下来;不论在农舍里还是在城里的聚会,即将上酒的时候常常是这样的。一会儿加林娜就端着托盘回来了。她那一双肥胖的手居然能在短短的几分钟里打开那么多的罐头,弄来那么多餐具,使人感到奇怪。

马舒克打量了一下挂着乌克兰花布壁毯的墙壁,看了看宽大的沙发床、一瓶瓶好酒和罐头,说:

“加林娜·捷连季耶芙娜,我还记得你们家这张沙发床,你们能把这床运出来,真有两下子,可见你们有一定的组织才能。”

“你别忘了,”格特马诺夫说,“疏散的时候,我不在家。全是她一个人!”

“诸位,总不能把这沙发床留给德国人,”加林娜·捷连季耶芙娜说,“杰敏季已经完全习惯了这张沙发,从州委会一回到家,就在这上面看材料。”

“哪儿是看材料?是睡觉!”萨盖塔克说。她又到厨房里去了,马舒克故弄玄虚地小声对格特马诺夫说:

“噢嘿,我可以想象,咱们的杰敏季·特里福诺维奇将认识一位女医生,一位军医。”

“是的,会把你照顾得好好的。”萨盖塔克说。

格特马诺夫把手一摆,说:

“算啦,你们怎么搞的,我是个病人。”

“当然不是,”马舒克说,“是谁在基斯洛沃斯克夜里三点钟才回房?”

几位客人哈哈大笑起来,格特马诺夫随便然而使劲地盯了内弟一眼。

加林娜走进来,环视了一下正在笑的男子们,说:

“我刚一出门,你们就不知想什么鬼花样欺负起我的可怜的杰敏季来啦。”

格特马诺夫就往酒杯里斟酒,大家都聚精会神地吃起小菜。

格特马诺夫望了望挂在墙上的斯大林像,举起酒杯说:

“来吧,同志们,为咱们的父亲干第一杯,咱们祝他永远健康!”

他说这话是用同志式的、有点儿随便的语调。语调所以这样随便,是因为斯大林的伟大是众所周知的,但是围坐在桌旁的几个人为他祝酒,首先是因为爱戴他这样一个朴实、谦逊和关心下属的人。画像上的斯大林眯缝着眼睛,打量着满桌的酒菜和加林娜那丰满的胸脯,似乎在说:“好,同志们,我把烟斗点着,坐到你们跟前来。”

“一点不错,愿我们的父亲永远健康!”女主人的弟弟尼古拉·捷连季耶维奇说。“我们没有斯大林怎么行?”

他把酒杯端到嘴边,转头看了看萨盖塔克,看他是不是说点儿什么。但是萨盖塔克看了看画像(好像在说:“父亲呀,还有什么好说的?你什么都知道嘛。”),就把酒喝干了。大家都把杯干了。

杰敏季·特里福诺维奇·格特马诺夫是沃罗涅日州的里夫内那个地方的人,但是他多年在乌克兰做党的工作,长期跟乌克兰同志共事。和加林娜结婚之后,他在基辅的关系更巩固了,因为她有许多亲戚在乌克兰的党政机关中担任要职。

格特马诺夫一生的经历说起来相当简单。他没有参加过国内战争,宪兵没有追捕过他,沙皇的法庭从不曾把他发配到西伯利亚。他在会议和党代会上作报告通常都是念发言稿。他念得很好,通顺流畅,而且富于表情,虽然稿子不是他自己写的。当然,念发言稿很容易,因为都是用大号铅字印的,间距很大,而且斯大林的名字都是用特制的红色铅字印出的。他当初是一个精明能干、循规蹈距的小伙子,本想进工学院,但是却被调到保安机关工作,并且很快就成为区委书记的贴身警卫员。后来他受到赏识,被送到党校学习,然后分配到党的机关工作,先是在区委组织部,后来又到中央委员会的人事局。过了一年,他就成为领导干部处的指导员。一九三七年以后,他很快就做了州党委书记,就是说,成了一州之主。

他说一句话,就可以决定大学教研室主任、工程师、银行经理、工会主席、农民集体经济、剧院演出的命运。

党的信任!格特马诺夫很懂得这几个字的伟大意义。党是信任他的!他这一生尽管没有成就伟大的著作、显赫的发明、辉煌的胜仗,但他付出了巨大的、目标明确、坚持不懈的劳动,而且是如履薄冰、常常不能安眠的劳动。这种劳动的最重要和最高意义就在于,劳动是根据党的需要,是为了党的利益。对于这种劳动的最重要和最高的奖赏只有一种,那就是党的信任。

在任何情况下,不论是处理幼儿园孩子们的问题,改组大学里的生物学教研室,还是处理生产塑料品的车间占用图书馆地盘的问题,他的决定都必须符合党性精神和党的利益。领导者对一件事、一本书、一部电影的态度都必须符合党性精神,因此,不论多么困难,在党的利益与个人喜好出现矛盾的时候,他都要毫不动摇地抛弃他做惯了的事情,抛弃他十分喜欢的书。但是格特马诺夫知道,还有更高水平的党性,其实质就是:这个人根本就没有与党性精神相矛盾的爱好与志趣;对于一个党的领导者来说,一切可爱的东西与可贵的东西之所以可爱可贵,就因为它代表党性精神。

有时格特马诺夫为了符合党性精神而作出的牺牲,是很残忍、很严酷的。一旦事关党性,就应该不讲个人感情,不动恻隐之心;长辈恩师,乡里乡亲,都不必顾及。在这种情况下,不必因为一些词儿,如“背信弃义”、“不够朋友”、“害人”、“出卖”等等而感到不安。但是,党性精神一旦到了炉火纯青的程度,就不需要牺牲了。因为一切个人感情,如爱情、友谊、同乡情谊,只要与党性精神相背,就很自然地不再存在。

党所信任的人做的劳动是默默无闻的。但这种劳动是巨大的,需要毫无保留、毫不吝啬地花费心思和精力。党的领导者不需要有科学家的才能,也不需要有作家的天陚。领导者的权力高于科学家的才能和作家的天赋。成百上千具有研究才能、歌唱才能、写作才能的人都要如饥似渴地听取格特马诺夫的指示和决定,虽然格特马诺夫不仅不会唱歌,不会弹琴,不会演戏,而且也不能鉴赏和深刻理解学术著作和诗歌、音乐、绘画作品。他的话所以具有决定性的力量,就在于党委托他代表党在文化艺术方面的利益。

一个人民的代言人和思想家,也未必拥有一个州党委书记这样多的权力。

格特马诺夫认为,“党的信任”这一概念的最深刻的实质就表现在斯大林的意见、感情和态度中。党的路线的实质,也在于斯大林对于自己的战友,对于人民委员和元帅们是否信任。

几位来客谈的主要是格特马诺夫即将担任的新的军事职务。他们知道,格特马诺夫有希望得到更重要的任命。在党内有他这样地位的人,一旦转到军事岗位,大都会成为集团军军委委员,有的甚至会成为方面军军委委员。

格特马诺夫被任命为军政委后,曾经感到不安和懊丧,还通过担任中央组织部委员的一个朋友打听,上面是不是有对他不满意的地方。结果,没有任何值得担心的事。

于是格特马诺夫为了自我安慰,开始从好的方面设想这一任命:是坦克部队决定战争的命运,坦克部队都是在主攻方向进攻。派往坦克军的不是随便什么人;宁可把有的人派往不太重要的地段,到无足轻重的集团军里去任军委委员,也不能派到坦克军里去。这说明了党对他的信任。不过他还是有些不安:要是穿上军装,对着镜子说:“集团军军委委员、旅级政工干部格特马诺夫。”那他是会挺高兴的。

不知为什么,坦克军那位上校军长最使他恼火。他还从来没见过这位诺维科夫上校,但是他所知道和打听到的有关诺维科夫的一切,他都不喜欢。

同桌共饮的几位亲戚很理解他的心情,谈他的新任命,谈的都是使他高兴的方面。萨盖塔克说,坦克军极有可能被派往斯大林格勒,斯大林格勒的方面军司令叶廖缅科将军,内战时期还在骑兵第一集团军的时候,斯大林同志就认识他了,斯大林同志常常通过高频电话同他谈话,每次他去莫斯科,斯大林同志都要接见他。不久前这位司令员到过莫斯科郊外斯大林同志的别墅,斯大林同志跟他谈了有两个钟头。在斯大林同志这样信任的人麾下作战,真是好极了。

后来又说,尼基塔·谢尔盖耶维奇[20]同志常常提到格特马诺夫在乌克兰的工作,如果格特马诺夫到赫鲁晓夫同志担任军委委员的方面军去,那就更好啦。

“斯大林同志派赫鲁晓夫同志上斯大林格勒前线来,不是随便派的,这是举足轻重的战线,不派他又派谁呀?”马舒克说。

加林娜慷慨激昂地说:

“怎么,斯大林同志派我家杰敏季到坦克军里去,就是随便派的吗?”

“算了吧,”格特马诺夫很直率地说,“我到军里去,就好比把州委第一书记调为区委书记。没什么可高兴的。”

“不是的,不是的,”萨盖塔克很严肃地说,“这一任命表现了党的信任。这区委,不是一般的农村的区委,而是马格尼托戈尔斯克区委,第聂伯罗捷尔任斯基区委。军不是一般的军,是坦克军!”

马舒克说,格特马诺夫将去担任政委的那个坦克军的军长,是不久前才任命的,以前没指挥过大部队。这是不久前到乌法来的一位前线特工处的工作人员告诉他的。

“他还对我说了一些话呢。”马舒克说。但他却不接着说下去,只是说:“不过,还用得着对您说吗,杰敏季·特里福诺维奇,您是非常了解他的,也许比他自己更了解呢。”

格特马诺夫把敏锐、精明、本来就细小的眼睛眯得更细了,肉嘟嘟的鼻孔翕动了两下,说:

“就算更了解吧。”

马舒克脸上闪过几乎觉察不出的冷笑,但桌上的人都发觉了。说来奇怪,虽然马舒克是格特马诺夫家的近亲和自家人,而且在亲戚圈子里是个谦逊、喜欢说笑的人,可是格特马诺夫夫妇听着他那柔和而委婉的声音,望着他那黑黑的、神情悠闲的眼睛和苍白的长脸,总感到有点儿紧张。格特马诺夫自己也感觉到这一点,却不觉得奇怪,他明白,马舒克是有来头的,有时连格特马诺夫都不知道的事情,马舒克却知道。

“这人怎么样?”萨盖塔克问道。

格特马诺夫用居高临下的语气回答说:

“噢,是这样的,是战争时期露头角的人,战前没什么突出的表现。”

“担任过重要职务吗?”马舒克笑着问道。

“算啦,什么重要职务,”格特马诺夫把手一挥,“不过,这人是有本事的,据说是一名很好的坦克手。军参谋长是涅乌多布诺夫将军。我跟他在十八次党代表大会上见过面。是一个精明强干的人。”

马舒克说:

“是伊拉里翁·英诺肯季耶维奇·涅乌多布诺夫吗?那不用说,先前我在他那儿工作过,后来命运把我们分开了。战前我还跟他在拉夫连季·帕甫洛维奇[21]的会客室里见过一面。”

“分开是分开了,”萨盖塔克笑着说,“你要辩证地对待,要看到同一性和统一性,而不是对立性。”

马舒克说:

“战争时期一切事情都很奇怪:一名上校干起军长,涅乌多布诺夫将军却成了他的下属!”

“没有作战经验,只好屈就了。”格特马诺夫说。

马舒克还是不服,说:

“笑话,涅乌多布诺夫吗,单是他的威望就够啦!他是革命前的老党员,有丰富的军事工作和国务工作的经验!有一个时期大家都推测他将担任部委委员呢。”

其余的客人也都支持他的意见。

他们对格特马诺夫的同情,这会儿用为涅乌多布诺夫抱不平的方式来表示,是非常合适的。

“是啊,战争把一切都搞乱了套啦,还是快点儿结束吧。”女主人的弟弟说。

格特马诺夫把张开手指的手掌朝萨盖塔克伸了伸,说:

“您认识莫斯科那个克雷莫夫吧?他在基辅,在中央演讲团做过国际形势报告。”

“是在战争开始前不久来的吗?那个过激分子?当年在共产国际工作过的那个人?”

“是的,就是他。我那位军长就准备跟克雷莫夫原来的妻子结婚。”

大家听到这个消息,不知为什么都感到非常好笑,虽然谁也不认识克雷莫夫原来的妻子,也不认识准备跟她结婚的军长。

马舒克说:

“噢,怪不得都说老兄神通广大。连结婚的事都知道啦。”

“可以说,精细人有精细人的本事。”尼古拉·捷连季耶维奇随口说。

“那当然……最高统帅部是不会赏识马大哈的。”

“是啊,咱们的格特马诺夫可不是马大哈。”萨盖塔克随口说。

马舒克就好像一下子来到自己的办公室里,用谈日常事务的严肃语气说:

“这个克雷莫夫过去也到过基辅,我还记得他,是个政治面貌不清的人。很久以前就跟右翼分子和托洛茨基分子有牵连。恐怕还没有完全搞清楚……”

他说得直接而又坦率,就好像针织厂厂长谈自己的工作或者技术学校教师讲课时那样。不过,大家都知道,他这种直爽只是表象,其实他比谁都知道什么事情能说,什么事情不能说。格特马诺夫是一个常会以自己的大胆、干脆和坦诚的言谈惊倒四座的人,可他很清楚,在兴高采烈看似随性的表象下面,隐藏着没有说出的深层的东西。

通常比别的客人更忙碌、更操心、更严肃的萨盖塔克,不希望轻松气氛遭到破坏,就用快活的语调对格特马诺夫说:

“因为他不怎么可靠,就连老婆都不跟他了。”

“如果因为那样,倒是好呢,”格特马诺夫说,“我听说,我们那位军长要娶的完全是一个乖僻的女人。”

“算啦,你真是瞎操心,”加林娜说,“最要紧的是,夫妻要有爱情。”

“爱情当然是重要的,这是大家都知道,都不会忘记的,”格特马诺夫说,“不过,此外还有些东西,可惜有些苏联人忘记了。”

“这话对,”马舒克说,“不论什么,咱们都不应该忘记。”

“正因为忘记了,于是感到惊讶不解,为什么党中央不批准,为什么这样,为什么不这样。自己不珍视党的信任。”

忽然加林娜惊讶不解地拉长声音说:

“听你们谈话都感到奇怪,就好像根本没有战争,你们关心的只是那位军长要娶的是什么人,他的未来妻子原来的丈夫是谁。杰敏季,你这是准备去跟谁打仗?”

她用嘲笑的目光朝男子们看了看,她那美丽的棕色眼睛都有点儿像丈夫的小眼睛了—大概是那股锐利的神气有点儿像。萨盖塔克用忧伤的口吻说:

“怎么会忘记战争啊……从每一座农舍到克里姆林宫,到处都有我们的兄弟和孩子奔赴战场。战争,是伟大的战争,是保家卫国的战争。”

“斯大林同志的儿子瓦西里是战斗机飞行员,还有米高扬同志的儿子也在空军里作战;我听说,贝利亚同志的儿子也在前线,只是不知道在哪一兵种。伏龙芝的儿子是一名中尉,好像是在步兵里……还有,伊巴露丽的儿子牺牲在斯大林格勒城下。”

“斯大林同志有两个儿子在前方,”女主人的弟弟说,“另一个儿子叫雅可夫,是炮兵指挥员。确切地说,他是第一个儿子,瓦西里是小儿子,雅可夫是大儿子。小伙子很不幸,被俘了。”

他忽然觉得他触及了许多年长的同志认为犯禁的东西,就不再说了。

尼古拉·捷连季耶维奇想打破沉默局面,用直率和无所顾忌的口吻说:

“顺便说说,德国人还散发彻头彻尾伪造事实的传单呢,说斯大林的儿子雅可夫主动向他们提交了口供。”

但是他周围的气氛更沉闷了。他谈的事情,不论开玩笑还是当真,都不应该提及,是应该回避的。谁要是听到有关斯大林跟妻子的关系的传闻表示气愤,那么,这位好心好意的谣言驳斥者所犯的罪过,决不比谣言传播者小,因为谈这类事情就是不容许的。

格特马诺夫忽然转过脸朝着妻子,说:

“这种事儿我是不操心的,因为情况由斯大林同志掌握着,而且掌握得牢牢的,就让德国人瞎折腾好啦。”

尼古拉·捷连季耶维奇用负罪的目光接住格特马诺夫的目光。

不过,自然,这不是一些好斗的人坐到桌上来了;他们聚会,也不会因为偶然出现的尴尬局面而闹出大乱子。

萨盖塔克用和善而友好的语调说了两句,在格特马诺夫面前帮尼古拉·捷连季耶维奇打圆场:

“这话是对的,不过我们总是担心,不希望在自己的地段上出什么纰漏。”

“还有,不希望胡说八道。”格特马诺夫补充说。

他几乎直截了当地责备起来,而不是缄默不语,这说明他原谅了尼古拉·捷连季耶维奇,于是萨盖塔克和马舒克都点了点头,表示赞同。

尼古拉·捷连季耶维奇知道,这件微不足道的错事很快会被忘记的,但不会忘得十分彻底。将来一旦谈起干部情况,谈起提拔,谈起特别重要的任命,在提到尼古拉·捷连季耶维奇的名字时,格特马诺夫、马舒克、萨盖塔克都会点头的,点头是点头,但在审干人员一再查问时,会微微笑一笑,说也许,多少有点儿轻率。”并且用小指头尖儿表示这一点点儿。

大家心里都明白,有关雅可夫的事不会都是德国人胡编乱造的。但正因为如此,决不能涉及这个话题。

萨盖塔克特别清楚这方面的情形。他在报社工作多年,先是掌管新闻报导科,随后掌管农业科,后来又干了两年某加盟共和国报纸的总编。他认为,他的报纸的主要任务是教育读者,而不是不加分析地发布关于各种各样、常常带有偶然性的事件的乱七八糟的消息。如果总编萨盖塔克认为应当避开某一事件,认为不应当报道严重的歉收、思想不縀¯的作品、内容不健康的影片、牲畜瘟疫、地震、战列舰沉没,认为不应当看到一下子夺走成千上万人生命的海洋巨浪的力量,不应当看到煤矿的大火,那么,这些事件对于他来说就没有任何意义了,他觉得,这些事件就不应耗费读者、记者和作家的精力。有时他需要用特别的方式解释现实中这样或那样的事件,这种解释往往异常大胆、异常奇特,跟平常的观念大相径庭。他觉得,他这位总编的力量、经验、本事就在于他能够使读者接受必要的、可以达到教育目的的观点。

在大规模推行集体化时期,曾经出现极端的冒进现象。在斯大林的文章《胜利冲昏头脑》发表之前,萨盖塔克曾写文章说,在大规模开展集体化时期发生饥饿现象,是由于富农蓄意埋藏粮食,不吃粮食,因而浑身浮肿,整村整村的富农连同小孩、老头子、老奶奶蓄意死亡,是给国家抹黑。

并且接着刊登一批材料,报道集体农庄托儿所里的孩子天天喝鸡汤,吃甜饼和米粉肉饼。可是孩子们还是瘦了,浮肿了。

战争开始了,这是俄罗斯立国千余年来最残酷、最可怕的一次战争。在战争的头几个星期和头几个月里,在经受特别残酷考验的时期,战争毁灭性的火焰照亮了种种事件的真实、可悲的进程,战争决定着一切的命运,甚至党的命运。这一灾难性的时期过去了。于是剧作家考涅楚克立即就在自己的剧本《前线》中解释说,战争的失败是由于愚蠢的将军们不能执行最高统帅部的指示,最高统帅部是永远不会错的。

这天晚上,注定了不是尼古拉·捷连季耶维奇一个人经历不愉快的时刻。马舒克在翻看一本皮封面的大纪念册,在一页页硬纸上贴着不少照片。他忽然带着紧张的表情扬起眉毛,大家不由得探过身来看。这是格特马诺夫战前在自己的州委办公室里拍的照片,他坐在宽阔的办公桌边,穿着半军服式样的制服上衣,他的上方悬挂着斯大林肖像,肖像非常大,只有州委书记办公室里才能有这样大的领袖像。肖像上的斯大林的脸被红蓝铅笔涂得乱七八糟,下巴上添了深蓝色的小胡子,两个耳朵上还挂着淡蓝色的耳环。

“这孩子真胡闹!”格特马诺夫惊叫起来,像女人一样把两手一拍。

加林娜·捷连季耶芙娜十分慌乱,环视着客人们,一再地说:

“要知道,你们要知道,昨天这孩子在临睡前还说:‘我爱斯大林伯伯,跟爱我爸爸一样。’”

“这是小孩子淘气。”萨盖塔克说。

“不,这不是淘气,这是故意捣蛋。”格特马诺夫叹口气说。他用询问的目光看了看马舒克。他们两个人此刻都想起同一件事:他们的一位同乡的侄子,是个工学院的学生,在学校用汽枪射击斯大林肖像。

他们知道,那个愣头愣脑的学生是瞎胡闹,没有什么政治用心。那位同乡是农机站站长,是个好人,他请求格特马诺夫挽救他的侄子。格特马诺夫在开过州党委常委会议以后,跟马舒克谈起此事。马舒克说:

“杰敏季·特里福诺维奇,我们又不是小孩子,他是有心还是无心,这没有什么意义……可是如果我把这件事情了结了,也许明天就有人上报到莫斯科,告到贝利亚同志那儿去,说马舒克纵容姑息枪击伟大领袖斯大林肖像的分子。今天我在这办公室里,明天我就成了集中营里的灰土。您愿意承担责任吗?也会有人说您:今天射击肖像,明年射击的就不是肖像了,可是为什么格特马诺夫要同情这个小伙子,他为什么赞成这样的行动呀?怎么样?您敢承担吗?”

过了一两个月,格特马诺夫问马舒克:

“那个射击肖像的学生怎么样啦?”

马舒克用平静的目光望着他,回答说:

“不值得问啦,原来是个坏蛋,富农的孽子,他在法庭上承认啦。”

于是现在格特马诺夫用询问的目光望着马舒克,又说了一遍:

“不,这不是淘气。”

“算啦,”马舒克说,“这孩子才五岁,还是应该考虑年龄的。”

萨盖塔克说话的口气十分恳切,大家都感觉出他话里的热诚:

“说实在的,我没办法对孩子们讲原则性……应该是应该,可是于心不忍。我望着孩子们,就希望他们都好好儿的……”

大家都用赞同的目光看了看萨盖塔克。他是一个很不幸的父亲。他的大儿子维塔利在上九年级的时候,就过起花天酒地的日子,有一次因为在饭店里参加流氓活动被警察拘留,父亲只好打电话给内务部副人民委员,了结这件丑事。参加那次流氓活动的有将军和院士等名人的儿子,还有一位作家的女儿和农业部人民委员的女儿。战争时期,萨盖塔克的儿子想以志愿兵身份参军,于是父亲安排他进了两年制的炮兵学校。维塔利因为不守纪律被学校开除,并且有可能随着增补连队被送往前方。

现在维塔利在迫击炮学校学习已经有一个月了,什么事也没有发生,父亲和母亲都很高兴,并且觉得有希望了,但他们总还是有些担心。

萨盖塔克的二儿子叫伊戈尔,两岁的时候害了小儿麻痹症,就变成了残疾人。一双又干又细的腿不能走路,只有靠拐杖活动。伊戈尔不能到学校去上学,老师们就到家里来教他,他学习很用心,很勤奋。

萨盖塔克夫妇为了伊戈尔的残疾,不仅在乌克兰,而且在莫斯科,在列宁格勒,在托木斯克求遍了神经科名医。凡是国外有关的新药,萨盖塔克都通过商务代办或驻外使馆弄了来。他知道,他可能因为过分溺爱孩子受到责备。但他同时也知道,他的罪过并不是死罪。因为他看到一些州的领导干部都有很强的父子感情,也就认为新派人都是特别钟爱自己孩子的了。他知道,他为伊戈尔用飞机从敖德萨请来巫婆,通过快传邮路把远东一个老神汉的草药弄到基辅来,也都不算什么。

“我们的领袖们都是一些特殊人物,”萨盖塔克说,“我就不说斯大林同志了,他没有什么可说的,就连他的亲密战友们也都是这样……他们在这个向题上也总是把党摆在父子感情之上。”

“是的,他们都明白:不是对每个人都提出这样的要求。”格特马诺夫说,并且不指名地说了一位党中央书记严肃对待自己犯错误的儿子的事。谈话气氛忽然一变,大家亲切而随便地谈起儿女们。似乎他们的精神力量的强弱,他们能不能幸福欢乐,都取决于儿女们的脸蛋儿红与不红,儿女们是否从学校里带回好分数,是否能顺利地升级。

加林娜谈起自己的女儿:

“斯维特兰娜在四岁以前身体很不好,老是肠炎,肠炎,折腾得很瘦弱。只有一种偏方能治:吃研碎的新鲜苹果。”

格特马诺夫说:

“今天她在去上学之前对我说:‘班上同学管我和卓娅叫将军女儿。’卓娅却不在乎,笑着说:‘有什么了不起的,将军女儿是很大的光荣!我们班上的元帅女儿才真神气!’”

“你们瞧,”萨盖塔克快活地说,“他们还不满足呢。伊戈尔前几天对我说……第三书记,没什么了不起。有什么好神气的?”

米柯拉本来也可以谈谈自己的孩子的许多好笑和愉快的事,但是他知道,在萨盖塔克谈儿子的机灵和格特马诺夫谈女儿的机灵的时候,他就不应该谈自己孩子的机灵了。

马舒克若有所思地说:

“过去在农村里我们的爹跟孩子们是很随便的。”

“他们总归也是喜欢孩子的。”女主人的弟弟说。

“喜欢当然喜欢,不过也常常打孩子,我挨打挨得厉害,”格特马诺夫说,“我还记得一九一五年我去世的父亲出发去打仗时的情形。他很不简单,干到士官,得过两枚乔治勋章。妈妈为他收拾行装,把包脚布和绒衣装到背包里,又装上煮熟的鸡蛋、面包,我和妹妹躺在床上,看着父亲在黎明时候最后一次在饭桌边坐了一阵子。他给过道里的水缸里挑满了水,劈了不少木柴。妈妈后来常常提起这些事。”

他看了看手表,说:

“噢呀……”

“就是说,明天要走啦?”萨盖塔克说着,站起身来。

“七点钟的飞机。”

“从民航机场走吗?”马舒克问道。

格特马诺夫点了点头。

“这样好些,”尼古拉·捷连季耶维奇说着,也站起身来,“要不然到军用机场有十五公里呢。”

“既然去当兵,这都算不了什么。”格特马诺夫说。

他们开始告别,又嚷嚷起来,笑起来,还互相拥抱了一阵子,等到客人们穿起大衣,戴上皮帽,来到走廊里,格特马诺夫说:

“当兵的人什么都能习惯,当兵的人可以用烟暖和身子,用锥子刮脸。可是跟孩子们分离,就是当兵的也不能习惯。”

从他的声音,从他脸上的表情,从要走的客人们望着他的那种神情可以看出来,这已经不是说笑话了。

二十二

夜里,格特马诺夫穿了军装,坐在写字台边写信。妻子穿着睡衣坐在他旁边,注视着他的手的移动。他把信折叠起来,说:

“这是给区卫生局长的,如果你需要专门治疗,需要出外就诊,可以找他。具体手续由弟弟给你办,局长只是开介绍信。”

“领取限额物品委托书你写了吗?”妻子问道。

“这用不着,”他说,“你可以打电话找州委办公室主任,最好找普济琴柯本人,他会给办的。”

他把写好的一叠信、委托书、便条检查了一遍,说:

“好,该写的好像都写了。”

他们沉默了一会儿。

“亲爱的,我真为你担心呀,”妻子说,“你这是去打仗。”

他站起来,随口说:

“你自己多保重,把孩子们照应好。白兰地放到提箱里了吗?”

她说:

“放进去啦,放进去啦。你可记得,两年前也像这样,你天不亮就给我写了不少委托信,然后飞到基斯洛沃斯克去了?”

“现在基斯洛沃斯克被德国人占了。”他说。

格特马诺夫在房里踱了一会儿,听了听,说:

“孩子们睡了吗?”

“当然,都睡了。”加林娜说。

他们朝孩子们的房间走去。奇怪的是,这两具又胖又重的身躯在幽暗中挪动起来一点声息也没有。沉睡的孩子们的头在雪白的枕头上显得格外黑。格特马诺夫细心地倾听孩子们的呼吸声。

他用手按住胸口,免得剧烈的心跳声惊醒孩子们。在这幽暗之中,他感到有一股强大而剧烈的感情,犹如利剑穿心,挂念孩子们的将来,按捺不住感伤、焦虑和怜惜。他非常想抱起儿子,抱起两个女儿,吻吻他们睡眼惺忪的脸蛋儿。他感到他的柔情是不能自制的,对儿女的怜爱是压抑不住的,这时候他心慌意乱,站在那里,尴尬,迷惘,浑身无力。

想到他即将担任的新职务,他并不害怕,也不担心。他常常改变工作,很容易找到正确的路线,正确路线也就是总路线。他知道,他在坦克军里也可以奉行这条路线。

可是,在这里,怎么能把钢铁的严厉、坚定,跟毫无规律可循的儿女情统一起来呢?

他回头看了看妻子。她站在那儿,像乡下人那样用手托着腮。她的脸在幽暗中好像瘦了,变年轻了。他们婚后第一次到海滨去,住在海边的“乌克兰疗养院”,那时候她就是这个样子。

小轿车喇叭在窗外轻轻地响了一声,这是州党委的汽车来了。格特马诺夫又转身朝着孩子们,摊开两条手臂,这一动作表示:虽然感情炽烈,但也无可奈何了。

在走廊里,他说过嘱告的话,吻别妻子之后,穿起短皮袄,戴好皮帽,站在那里,等着司机把皮箱拎出来。

“好啦。”他说着,忽然从头上摘下皮帽,走到妻子跟前,把她抱住。在这又一次、最后一次吻别中,就在外面潮湿的冷空气从半开的大门冲进来,同家里的热气混合的时候,就在毛烘烘的熟皮袄毛皮挨到香喷喷的绸睡衣的时候,他们都感觉到,他们那似乎成为一体的生活忽然分开了。他们的心碎了。

二十三

叶尼娅·沙波什尼科娃来到古比雪夫,住在一个德国老太婆家里。德国老太婆燕妮·亨利霍芙娜·亨利逊很久以前在沙波什尼科夫家做过保姆。

叶尼娅从斯大林格勒来到安静的小屋里,跟一个老太婆住在一起,觉得很稀奇;老太婆也一直流露着惊讶不解的表情,没想到一个扎两条小辫儿的小姑娘会变成一个成年的女子。

亨利逊老太婆住的是一间昏暗的小屋,这是过去一个大商人家里的女仆住的房间。现在每个房间里都住着一家人,每个房间都用屏风、布幔、毡毯、沙发靠背分成几个小小的房间,在里面睡觉、吃饭、会客,护士在里面为瘫痪的老头子打针。

一到傍晚时候,厨房里就嗡嗡地响起许多人的声音。

叶尼娅很喜欢这熏黑了屋顶的厨房,很喜欢煤油炉那黑红色的火焰。

一件件衣服晾在绳子上,身穿长衫、棉袄、制服的邻居们在绳子中间穿来穿去,菜刀、柴刀闪闪放光。妇女们弯身在木盆或脸盆里洗衣服,呼出一团团热气。巨大的炉灶从来没有生过火,瓷砖砌的炉壁又冷又白,就像在上个地质年代就熄灭了的火山那覆盖着积雪的山坡。

这座住宅里住着一位上了前线的格鲁吉亚工人的家小,住着一位妇科医生、一位保密工厂的工程师、一位担任配给商店出纳员的单身老妈妈,还有一位在前方牺牲的理发员的遗孀,还有邮政总局的警卫长,在最大的房间里,也就是过去的会客室里,住的是一家门诊所的主任。

这座住宅十分宽大,就像一座城市一样;这里面甚至有自己的疯子,是一个安安静静的疯老头,眼睛像一只小狗的那样温柔善良。

大家住得很拥挤,但是互不往来,而且不太和睦,有时吵几句,有时相安无事,有时互相隐瞒自己的家事,有时又很大方地用大嗓门儿把自家生活中所有的事说给邻居听。

叶尼娅想要描绘这所房子,不是景物,也不是其中一户户邻居,而是这些人在她心中挑起的情感。

这种情感是复杂的,极难表现,就连高明的艺术家也无能为力。人民和国家的强大军事威力,与这黑黑的厨房的穷困、卑琐、飞短流长混在一起;威力无比的钢铁武器,与厨房里的一只只小铁锅、一堆堆土豆皮混在一起,于是便产生了这种情感。

表现这种情感,常常弄得线条不成线条,轮廓不成轮廓,结果变成支离破碎的形象和光点的拼凑,从这种拼凑中看不出任何意义。

亨利逊老奶奶是一个腼腆、和蔼、热心的人。她穿着白领的黑长袍;虽然总是忍饥挨饿,但她的两颊总是红红的。

她在脑海里还清清楚楚地记得一年级学生柳德米拉淘气的事情,记得小玛露霞说的一些可笑的话,还记得两岁的米佳常常戴着围兜跑到餐室里张着小手,喊:“吃唤(饭),吃唤(饭)!”

现在亨利逊老奶奶在一位牙科女医生家里做佣工,照料女医生有病的妈妈,不包住宿。女医生被市卫生局派到区里去了,要五六天才能回来,于是亨利逊晚上在她家里睡,好照应那个不久前中风之后行动不便的老妈妈。

亨利逊老奶奶完全没有财产观念。她常常对叶尼娅说对不起,请她允许自己打开通风小窗,好让她的三色老花猫进出活动。她的主要兴趣和操心事都和老猫有关系,就怕邻居欺负她的猫。

担任车间主任和工程师的邻居德拉金,常常带着不友好的嘲笑神气望着她那皱皱巴巴的脸,望着她像姑娘一样又细又直的身躯,望着她系在黑带子上的夹鼻眼镜。这个平民出身的人感到气愤的是,亨利逊老奶奶依然那样留恋过去,并且常常带着傻笑讲她在革命前怎样带着孩子们乘轿式马车在外面玩儿,怎样陪着太太上威尼斯,上巴黎,上维也纳去。她带大的许多“小家伙”成了邓尼金部下、弗兰格尔部下[22],都被红军打死了,但是老太婆念念不忘的只是当年小家伙们害猩红热、白喉、结肠炎的情形。

叶尼娅对德拉金说:

“比她更厚道、更老实的人我还没有遇到过呢。您要相信,在这座宅子里,没有比她心眼更好的人了。”

德拉金带着男子汉那种放肆的、毫无顾忌的神气直盯着叶尼娅的眼睛,回答说:

“唱赞美歌吧,燕子,唱吧。沙波什尼科娃同志,为了一块居住的地方,您就卖身投靠德国人啦。”

看样子,亨利逊老奶奶不喜欢健康的孩子。她照应过一个身体十分虚弱的孩子,是一位犹太裔厂长的孩子,她对叶尼娅说得最多的就是这个孩子,还保存着他的练习本、他画的画,每次说到这个安静的小男孩的死,她都要哭一场。

她在沙波什尼科夫家做保姆,是多年以前的事了,但是她还记得所有的小孩子的名字和外号,而且一听说玛露霞已经死了,就哭了起来;她一直在用歪歪扭扭的字体给远在喀山的亚历山德拉·弗拉基米罗芙娜写信,但是这封信怎么也写不完。

她对叶尼娅说,革命前她带的孩子,吃早饭常常是一碗很稠的肉汤和一片鹿肉。她常常拿自己的口粮喂猫,管猫叫“我的可爱的银宝贝”。老猫也非常依恋她,尽管是一个阴森而粗暴的畜生,可是一看到老奶奶,立刻就变得快活又温驯。

德拉金常常问她对希特勒是什么态度:

“怎么样,您大概很欢迎他吧?”

但是留了个心眼的老奶奶说自己是反法西斯的,并且管希特勒叫吃人魔王。

她是一个很无用的人,不会洗衣服,不会煮饭,要是到商店里去买火柴,售货员必然会在匆忙中把她一个月的糖票或肉票从供应卡上剪去。

现在的孩子完全不像她称作“和平时期”的那时候她带过的孩子。一切都变了,就连玩儿也不一样了。“和平时期”的女孩子们玩的是抛圈儿游戏,用一根根系了带的漆棍儿抛掷橡皮扯铃,玩没什么弹性的彩色皮球,皮球装在白色网兜里。今天的女孩子们打排球,游泳,冬天穿着滑雪裤打冰球,又叫又嚷,吹着口哨。

现在的孩子比亨利逊老奶奶更懂得赡养费、流产,更知道用欺骗的方法得来供应卡,知道那些为别人的妻子从前方带回奶油和罐头的上尉和中校。

叶尼娅很喜欢这位德国老奶奶回忆她的童年时代,回忆她的父亲和哥哥米佳。老奶奶对米佳记得特别清楚,他害过百日咳和白喉,她照料过他。

有一天亨利逊老奶奶说:

“我还记得我一九一七年的最后一家东家。老爷是财政部次长,他在餐室里走来走去,说:‘全完啦,庄园烧掉啦,工厂停工,通货膨胀,金库被抢光。’他们家就像现在你们家一样,一家人都跑散了。老爷、太太、小姐上了瑞士,我带大的孩子去投科尔尼洛夫[23]将军当了志愿军。太太哭着说:‘我们天天在告别,完啦。’”

叶尼娅凄然笑了笑,没有作声。

有一天傍晚,来了一名地段警察,交给亨利逊一张传票。这位德国老奶奶戴上绣了小白花的女帽,嘱托过叶尼娅代她喂猫,就上警察局去了,说是从警察局出来还要去照料牙科医生的妈妈,过一天才能回来。等到叶尼娅下班回来,看到屋子里空空荡荡的,邻居们告诉她,亨利逊老奶奶被警察局抓起来了。

叶尼娅去打听她的情况。警察局里的人告诉她,老奶奶将跟随运送德国人的军用列车上北方去。

过了一天,一名警察和房屋管理员来拿走了被査封的一只篓子,里面装满了破布、发黄的相片和发黄的信件。

叶尼娅找有关部门打听,怎样可以把毛围巾送给老奶奶。有一个人在小窗户里向叶尼娅问道:

“您是什么人,是德国人吗?”

“不是,我是俄罗斯人。”

“那您回家吧。不要乱问。”

“我问的是怎样送毛围巾。”

“您明白不明白?”那人在小窗户里用那样一种低声问道,叶尼娅一听那口气就怕了。

这一天晚上,她听到一些邻居在厨房里说话。他们说的是她。

有一个声音说:

“她的做法总归是不大漂亮。”

另一个声音说:

“可是依我看,她很聪明。先是一只脚插进来,然后向有关部门汇报老太婆的事,把老太婆扫地出门,现在她是房间的主人了。”

有一个男人声音说:

“算什么房间,一点点儿小屋。”

还有一个声音说:

“是呀,这种女人是不会吃亏的,跟这样的女人在一起,是不会不吃亏的。”

猫的命运是很凄惨的。它无精打采、死气沉沉地坐在厨房里,这时候一些人在争论,把它弄到哪里去。

“让这只德国猫见鬼去吧。”女人们说。

德拉金忽然声明,他要参与喂猫。但是猫离了亨利逊老奶奶之后,没有活多久。有一个女邻居,不知是有意还是无意,用开水烫伤了它。猫不久就死了。

二十四

叶尼娅很喜欢她在古比雪夫的独身生活。

也许,她从来没有像现在这样自由过。尽管生活艰苦,可是心里有种轻松自在。有很长时间,她没有报上户口,没领到供应卡,每天凭饭票在食堂吃一顿饭。从早晨她就想着什么时候到食堂里去领一碟子菜汤。

在这个时期她很少想到诺维科夫。她想克雷莫夫想得多些,几乎老是在想,但是这种想念的内部光强度不大。

想念诺维科夫的心情常常出现又消失,并不使她感到苦恼。

但是有一次在大街上,她老远看到一个穿军大衣的高个子军人,有一瞬间,她以为那是诺维科夫。她顿时激动得喘不上气来,两腿也软了,浑身出现了一种幸福的感觉,高兴得不知如何是好。过了一分钟,她明白自己看错了,马上也就忘记了自己的激动。

到夜里她忽然醒来,心想:“为什么他不写信呀?他知道我的地址嘛。”

她一个人生活,身旁既没有克雷莫夫,也没有诺维科夫,没有亲人。她觉得,这种自由的单独生活就是幸福。不过,这只是她觉得罢了。

这时候在古比雪夫有许多莫斯科的人民委员部、机关和莫斯科报社的编辑部。这是从莫斯科迁来的临时首都,有外交使团,有大剧院的芭蕾舞,有著名的作家,有莫斯科的报幕员,有外国记者。

这成千上万的莫斯科人拥挤在一个个狭小的房间里,有的住旅馆客房、有的住公共宿舍,各自干着原来的事情:各部门的负责人、各个局和各个总局的首长、人民委员,领导着属下人员,掌管着国民经济。特任大使和全权大使们乘坐豪华的汽车,拜会苏联对外政策的领导人。乌兰诺娃,列梅舍夫和米哈伊洛夫照常演出芭蕾舞和歌剧,令观众心醉入迷;美联社代表沙皮罗先生在记者招待会上向苏联情报局局长洛佐夫斯基发难;作家们在为本国和外国的报纸与电台写文章;记者们在军医院里搜集材料,写战地通讯。

但是,莫斯科人的生活在这里变得完全不同了。大不列颠王国特任全权大使的夫人克里普斯太太,每天凭饭票在旅馆食堂里吃饭,没有吃完的面包和糖块用报纸包起来,带回自己的房间;世界各个报纸和通讯社的记者们常常上市场去,在伤兵们中间挤来挤去,买本地的土烟丝自己卷烟卷,津津有味地评论烟草的味道;倒换着两只脚,站在澡堂前排长队;以慷慨闻名的作家们,在讨论世界大事和文学问题的时候,喝着土制烧酒,拿定额的面包当下酒菜。

一个个大机关挤在古比雪夫的一层层狭小的楼上;苏联各大报的领导人在家用的桌子上接见来访者,下班后孩子们就在这桌子上做功课,妇女们就在上面做针线活儿。

庞大的国家机构过起流浪生活,就出现了有趣的事情。

叶尼娅因为报户口,遇到很多麻烦。她开始在设计院工作,院长里津中校是个高高的男子,说话声音低低的、轻轻的,从接收这个没办好户口手续的工作人员的第一天起,就因为怕负责任而发愁。里津叫她上公安局去,同时给她开发了录用证书。

公安局派出所的工作人员收下叶尼娅的身份证和录用证书,叫她三天以后来听回话。

叶尼娅在约定的那一天来到昏喑的走廊里,坐在走廊里等候接待的人脸上都带着一种特别的表情,这种表情只有来公安局办理身份证和户口手续的人才会有。她走到小窗口跟前。一只涂着暗红色指甲油的女人的手把身份证递给她,一个平静的声音说:

“不予办理。”

她站进长队,等待跟户籍股股长谈一谈。站队的人在小声说着话儿,打量着在走廊里走过的一个个抹了口红、穿着棉制服和皮靴的公安局的姑娘们。有一个身穿夹大衣、头戴军帽、军装领子从围巾里面露出来的人,踏着咯吱咯吱直响的皮靴,不慌不忙地走过去,用小小的钥匙开了门上的锁,不知是英国锁还是法国锁—这人便是户籍股长格里申。接待开始了。叶尼娅发现,轮到被接待的人并没有久等之后终于轮到的欣喜,而是一面朝门里走,一面四处打量着,就好像准备在最后一分钟跑掉似的。

叶尼娅在等候接待的时间里,听了不少报不上户口的事。有些女儿在母亲家里,瘫痪的姑娘在哥哥家里,都报不上户口。有的妇女来这里看护伤残军人,也没办到户口。

叶尼娅走进格里申的办公室。他一声不响地向她指了指椅子,看了看她的材料,说:

“您这个不能办理。还有什么要说的?”

“格里申同志,”她一开口,声音就哆嗦起来,“您要知道,这段时间我一直领不到供应卡呀。”

他用一眨不眨的眼睛看着她,他那张年轻的宽大的脸流露出一种若有所思的淡漠神情。

“格里申同志,”她说,她的声音又哆嗦起来,“您设身处地想想看,怎么能这样?古比雪夫就有一条以沙波什尼科夫命名的街。那是我的父亲。他是萨乌拉的革命运动发起人之一,可你们却不准他的女儿报户口。”

他用平静的眼睛望着她;他听着她说的话。

“需要有军调令,”他说,“没有军调令我不能办。”

“我就是在军事机关工作呀。”叶尼娅说。

“从您的证件看不出是在军事机关。”

“在军事机关就行吗?”

他不情愿地回答说:

“行。”

第二天早晨,叶尼娅来到办公室,对里津说,公安局不给办户口手续。他把手一摊,用低低的细嗓门儿说:

“哎呀,真胡闹,难道他们不懂,您一开始工作,就成了我们不能缺少的工作人员,您负责的是国防性质的工作?”

“就是啊,”叶尼娅说,“他说,需要有一张证明,证明咱们的机关隶属于国防人民委员会。请您开一张证明,今天晚上我再带着证明上公安局去。”

过了一阵子,里津找到叶尼娅,用抱歉的口吻说:

“需要由公安机关发来查询公函。没有查询公函我无法开发这一类的证明。”

傍晚她又来到公安局,等着被接待。她一面痛恨自己那种讨好的微笑,一面请求格里申发函向里津查询。

“任何查询公函我都不会发。”格里申说。

里津听说格里申不肯发函,叹了一口气,沉思一会儿,说:

“就这样吧,您去要求他,哪怕打个电话向我查询也行。”

第二天傍晚叶尼娅要去见一位莫斯科来的文学家,他父亲的旧识里蒙诺夫。于是她一下班就赶到公安局去,向排队的人要求允许她进去见户籍股长,“只要一小会儿”,只提一个问题。人们耸耸肩膀,把脸转了过去,她懊恼地说:

“好吧,等就等吧,谁是最后一个?”

这一天,公安局留给叶尼娅的印象特别沉重。有一个两腿浮肿的女人在户籍股长的办公室里发起火来,高声喊:“我求求你们!我求求你们!”一个断胳膊的人在格里申的办公室里骂起娘来。接着有一个人也大吵起来,喊:“我就是不走!”不过他很快就走了。在吵闹的时候却听不到格里申的声音,他一直没有提高嗓门儿,就好像他不在,人们自己在吵,在自己吓唬自己。

她排队等了有一个半钟头。她又一面痛恨自己讨好的笑脸,痛恨自己忙不迭地说“谢谢!”(人家不过略略点头让坐),一面恳求格里申给她的领导打电话,并说,里津起初是犹豫的,说没有注明日期和盖有公章的函调,恐怕不能开具证明信,后来他好不容易同意了,他可以写证明信,但必须标明是“回答某月某日您的口头查询”。

叶尼娅把事先写好的一张纸条放到格里申面前,她在纸条上用又大又清楚的字体写明电话号码、里津的名字和父称、军衔、职务,又用小字在括号里写明,午休时间从什么时候开始,到什么时候为止。但是格里申对放在他面前的纸条连看也不看,就说:

“我不进行任何形式的查询。”

“那为什么?”她问。

“不必要。”

“里津中校说,如果连口头查询也没有,他无权开发证明。”

“他既然无权,就不开好啦。”

“可是我怎么办呀?”

“我怎么知道?”

叶尼娅见他那样平心静气,真没了主意,假如他发脾气,说她无理纠缠,她倒是轻松些。可是他半侧着身子坐在那儿,连眼皮也不动一动,一点也不着急。

许多男子在跟她交谈的时候,都会发现她很美,她也总会感觉到这一点。但是格里申看着她的那种神情,就好像看着眼睛里流泪水的老奶奶或者残废人。她一进他的办公室,就不再是人,不再是年轻女子,只是一名求告者了。

她感到自己的弱小,感到他手握强大的权柄,茫然失措了。她在大街上走着,匆匆忙忙,因为已经比约定会见里蒙诺夫的时间晚了一个多钟头,不过,匆忙归匆忙,她已经不因为这次会见感到兴奋了。她似乎还闻到公安局走廊里的气味,似乎还看到一张张等候接待的人的脸,看到暗淡的灯光照耀着的斯大林肖像,还有旁边的格里申。格里申又镇静,又坦然,掌握着钢铁般的国家大权。

里蒙诺夫高高胖胖的,老大的头,秃顶周围有一圈像年轻人一样的鬈发,他高高兴兴地迎住她。

“我正怕您不来呢。”他说着,就帮叶尼娅脱大衣。他开始向她询问亚历山德拉·弗拉基米罗芙娜的情况:“从大学时代起,我就认为您的妈妈是英勇刚强的俄罗斯妇女的典型。我在作品中经常写到她。不是写她个人,而是写她这样一种类型。”

他放低了声音,又回头朝门外看了看,问道:

“听到米佳的什么消息吗?”

然后他谈起绘画,两个人开始一起骂列宾。里蒙诺夫在电炉上煎起鸡蛋,并且说,他是国内做鸡蛋饼的能手,就连“民族饭店”的厨师都向他学习过呢。

“怎么样?”他一面请叶尼娅吃鸡蛋饼,一面很不放心地问道。又叹了一口气,说:“对不起,我就喜欢吃。”

公安局的所见所闻给她的压力多么大啊!她来到里蒙诺夫这温暖的、摆满了书籍杂志的房间里,不久又来了两个上了年纪的、通晓艺术又幽默风趣的人,可是她的一颗打着寒颤的心还一直感觉到格里申的存在。

但是自由而机智的谈话的力量也是强大的,于是叶尼娅一时间也就忘记了格里申,忘记了排队的人们一张张苦恼的脸。似乎除了谈鲁布廖夫,谈毕加索,谈阿赫玛托娃和帕斯捷尔纳克的诗和布尔加科夫的戏剧,人生再没有什么事了。

她来到大街上,马上就忘记了方才高雅的谈话。

格里申,格里申……在这座宅子里,谁也没有同她谈过是否办好户口手续的事,谁也没有要她出示盖了印记的户口登记卡。但是她已经有好几次觉得居民小组长格拉菲拉在监视她。格拉菲拉是个机敏的高鼻子女人,总是亲亲热热的,说话总是用甜甜的、透着虚伪的语调。叶尼娅每次碰到格拉菲拉,看到她那又亲热又阴沉的黑眼睛,总是感到害怕。她似乎觉得,在她不在家的时候,格拉菲拉就用配好的钥匙打开她的房门,搜查她的证件,抄录她申报户口的申请书,看她的信件。

叶尼娅尽可能悄没声地推开大门,踮着脚在走廊里走,很怕碰见居民小组长。说不定居民小组长会对她说:

“您干吗破坏法纪,要我替您担责任?”

早晨,叶尼娅来到里津的办公室,对他说了说在户籍股又碰钉子的事。

“请您帮我买一张去喀山的船票吧,要不然,也许会因为破坏户籍制度送我去开采泥炭呢。”她没再要求他开什么证明,而且说话用的是嘲笑和恼怒的口气。

这个低声细语的高大的漂亮男子望着她,因为自己的胆小怕事感到羞惭。她经常感觉到他那恋恋不舍的目光停留在她身上。他望着她的肩膀、大腿、脖子、后脑勺,而她的肩膀和后脑勺也感觉出这种火辣辣的爱恋的目光。但是,看样子,决定文件收发规则的力量是非同小可的。

下午,里津来到叶尼娅面前,一声不响地把开好的证明信放在图纸上。叶尼娅也一声不响地看了看他,眼泪不禁夺眶而出。

“我通过保密部门提了申请,”里津说,“本不抱什么希望,谁知领导一下子就批准了。”

同事们都向她祝贺,说:

“您的苦总算熬到头了。”

她来到公安局。排队等候的人都向她点头打招呼,有些人已经跟她熟识了,他们问她:

“怎么样?……”

有几个声音说:

“您进去吧,不用排队了……您这事一会儿就能办好,干吗还要等两个钟头?”

她觉得,那办公桌,那漆了仿木褐色粗花纹的保险柜也不再那样阴森、带着官气了。格里申看着叶尼娅那匆忙的手指头把所需要的证明信放到他面前,微微地、满意地点了点头,说:

“好吧,您把身份证、证明信留下,三天后在接待时间在收发室等候结果。”他的声音还是和平常一样,但是叶尼娅觉得格里申那明亮的眼睛很亲切地笑了笑。

她一面往家走,一面想,原来格里申也和所有的人一样,也会做好事,也会笑。原来他不是毫无心肝的人。她原来把这位户籍股长想得那样不好,现在她觉得不好意思起来。

过了三天,一只涂了暗红色指甲油的女人的手从小窗口里把身份证连同整整齐齐夹在里面的证明信递给她。叶尼娅看了看清清楚楚写在上面的批示:“因此人与该住处无固定关系,不予办理户口登记手续。”

“狗崽子!”叶尼娅大声叫起来。她再也忍不住,又大声叫道:“简直是捉弄人,存心折腾人,这家伙!”

她大声叫着,摇晃着不管用的证件,对着排队的人们,希望得到他们的支持,但是她看到,他们都转过脸去,躲开她。一时间她心里泛起一股要拼命的情绪,绝望和发疯的情绪。一九三七年,在索科尔尼基的布特尔监狱里,许多妇女站在昏暗的监狱大厅里,排队等候探望失去通信自由的罪犯,那时候有些悲痛绝望得发了疯的妇女就是这样喊叫的。

站在走廊里的一名民警抓住叶尼娅的胳膊把她往门外推。

“放开我,别动我!”她抽出胳膊,把他推开。

“女公民,”他用嗄哑的声音说,“别叫啦,要不然会判十年徒刑!”

她觉得,民警的眼睛里闪过一丝恻隐和怜悯的神情。

她快步朝大门口走去。大街上摩肩擦背地走着许多人,他们都办过了户口登记手续,都有定量供应卡……夜里她梦见大火,她朝一个趴在地上的伤员俯下身去,她想把他背起,并且知道这是克雷莫夫,虽然没看到他的脸。她醒来后,又惊愕,又沮丧。“他能快点儿来就好啦。”她一面穿衣服,一面想道。并且嘟哝说:“帮助我吧,帮助我吧。”她非常非常想看到的,不是夜里她要救护的克雷莫夫,而是诺维科夫,非常想看到他还是今年夏天她在斯大林格勒看到的那种样子。

像这样没有户口,没有供应卡,见了看院人、房管员、居民小组长总感到提心吊胆的日子,实在叫人受不了。叶尼娅总是趁大家都睡了才上厨房去,早晨去洗脸尽量赶在大家都醒来之前。每次邻居们跟她说话,她的声音变得温和得有些过分,极不自然,很像浸礼派修女的声音。

这天下午,叶尼娅写好了离职申请书。

她听说,在户籍股拒绝办理户口登记手续之后,来过一名民警,送来一张限三天内离开古比雪夫的批示。批示的正文中说:“破坏户籍制度者,理应……”叶尼娅不希望“理应”,要她离开古比雪夫,她就离开好啦。她一想到可以不再看到格里申,不再看到格拉菲拉和她那柔和得像烂橄榄一样的眼睛,不再苦恼,不再担惊害怕,心里马上就觉得轻松了。她不想违抗法律,她要服从法律。

等她写好了离职申请书,正要去交给里津的时候,有人叫她去接电话,是里蒙诺夫打来的。

他问她,明天晚上她是不是空闲,从塔什干来了一个人,说了一些那里的情形,挺有意思,还带来了阿列克谢·托尔斯泰的问候。于是她又感受到另一种生活的气氛。

叶尼娅尽管不准备说,可还是对里蒙诺夫说了说有关户口的事。

他听她说,也不插话,后来他说:

“竟有这种事,真有意思:古比雪夫有爸爸的街道,可是不准女儿落户口,要把女儿撵出去。有趣。有趣。”

他略作思索,又说:

“这样吧,叶尼娅,您的离职申请书今天不要交,晚上我要参加州委书记召开的会议,我把您的事情说一说。”

叶尼娅道了谢,但是她以为里蒙诺夫把话筒一放,马上就会把她的事情忘了。不过她还是没有把离职申请书交给里津,只是问他,能不能通过军区司令部给她弄一张去喀山的船票。

“这倒好办,”里津说,并且把两手一摊,“就是公安机关难说话。有什么办法呢,古比雪夫实行一套特殊的制度。他们有专门指示。”

他问她:

“今天晚上您有时间吗?”

“没空,有事。”叶尼娅生气地说。

她一面往家里走,一面想,她很快就要看到妈妈、姐姐、姐夫、娜佳了,她在喀山一定会比在古比雪夫好些。她很奇怪,为什么她这样伤心,为什么一进公安局就吓得发呆。不给办户口手续,就去它的吧……如果诺维科夫有信来,就请邻居们转往喀山去好啦。

早晨,她刚来上班,就叫她去接电话。有一个很有礼貌的声音请她上市公安局户籍股办理户口手续。

二十五

叶尼娅结识了住在这座宅子里的一位邻居—沙尔戈罗茨基。每次沙尔戈罗茨基突然转头的时候,似乎他那老大的、像雪花石膏一般的头就要从细细的脖子上掉下来,咚的一声落到地上。叶尼娅发现,老头子脸上那苍白的皮肤泛着柔和的蓝色光泽。叶尼娅很喜欢这种皮肤的蓝与眸子的蓝色冷光相搭配;老头子是高等贵族出身,她一想到恰好可以用表示高贵的蓝色来画老头子,就觉得十分好笑。

弗拉基米尔·安德列耶维奇·沙尔戈罗茨基在战前的生活不如战争时期。现在他有一些活儿干了。苏联情报局约他写一些短文,写德米特里·顿斯科伊、苏沃洛夫、乌沙科夫,写俄罗斯军人的光辉传统,写十九世纪的诗人,如丘特切夫、巴拉丁斯基……

沙尔戈罗茨基告诉叶尼娅,从母系来说,他是罗曼诺夫王朝之前一支古老的公爵世家的后裔。他年轻时在省地方自治局任职,在地主子弟、乡村教师和年轻神甫们中间鼓吹彻底的伏尔泰主义和恰达耶夫思想。

他对叶尼娅说过他同省首席贵族的谈话。是四十四年以前的事了。

“您是俄罗斯一支古老世家的代表,可是居然向庄稼汉鼓吹,说人类起源于猴子。庄稼汉会问您:大公们是不是?皇太子是不是?皇后是不是?皇上本人是不是?……”

沙尔戈罗茨基继续进行思想宣传,结果他被流放塔什干。一年后他得到赦免,于是他出国到了瑞士。在瑞士他遇到很多革命活动家。布尔什维克、孟什维克、社会革命党人、无政府主义者都知道这位古怪的公爵世家后裔。他参加辩论会、晚会,和一些人谈得很愉快,但是他不赞成任何人的主张。就在这时候,他和一个犹太大学生李别茨成了好朋友,李别茨是一个留着黑色胡须的崩得[24]分子。

第一次世界大战之前不久,他回到俄国,住在他自己的庄园里,有时在《下诺夫哥罗德报》发表历史题材和文学题材的文章。

他不善经营家产,庄园由母亲管理。

沙尔戈罗茨基是唯一一个庄园未被农民触动的地主。贫农委员会甚至分给他一大车木柴和四十棵大白菜。他整日坐在家里唯一生了炉子、装了玻璃的房间里,读书,写诗。有一首诗他还念给叶尼娅听过。这首诗题为《俄罗斯》:

放眼四望,无虑无忧。

大平原,无边无沿。

老鸦悲怆地啼叫。

玩乐。大火。隐秘。

麻木不仁。

处处别具一格。

又惊人地雄伟。

他用心地念着一个一个的字,停顿、转折处都念得很清楚,长长的眉毛扬得高高的,然而他那宽大的额头并不因为扬起眉毛而显得小些。

一九二六年,沙尔戈罗茨基讲授起俄罗斯文学史。他抨击杰米扬·别德内,赞扬费特[25],参加当时非常风行的关于生活的真和美的辩论会。他声称自己反对任何国家形式,声称马克思主义是有局限性的学说,谈俄罗斯精神的可悲命运,直到又一次免费去了塔什干。他住在那里,一直不理解地理位置的转换在理论辩论中的作用。直到一九三三年底,他才得到允许迁到萨马拉他的姐姐那里去。他姐姐叶连娜·安德列耶芙娜是战前不久才死的。

沙尔戈罗茨基从来不请别人到自己屋里去。但是有一次叶尼娅到这位公爵后裔的住处看了看:书和旧报纸堆在角落里像山一样,一张张旧椅子摞在一起,几乎抵到天花板,镶了镀金框的画像摆在地板上。在蒙了红丝绒的沙发上放着一床皱皱巴巴、露出棉絮的棉被。

这是一个和善的人,在现实生活中没办法的人。通常大家都说这样的人有“孩子般的心灵、天使般的善良”。但是他可以默诵着他心爱的诗句,无动于衷地从伸手向他乞讨的饥饿的孩子或衣衫褴褛的老妪身边走过。

叶尼娅听沙尔戈罗茨基说话,常常想起自己的第一个丈夫,可是这位费特和弗拉基米尔·索洛维约夫的一贯崇拜者与共产国际战士克雷莫夫太不相像了。

叶尼娅感到奇怪的是,克雷莫夫跟沙尔戈罗茨基老头子一样是俄罗斯人,但对俄罗斯美丽的风光,对俄罗斯民间故事和费特、丘特切夫[26]的诗竟毫无兴趣。克雷莫夫从小就看重的俄罗斯生活中的一切,他认为在俄罗斯头等重要的一些人物,沙尔戈罗茨基却毫不感兴趣,有时甚至有些敌视。

对于沙尔戈罗茨基来说,费特是上帝,首先是俄罗斯的上帝。对于他来说,关于好汉菲尼斯特的故事和格林卡[27]的组歌《彷徨》都是神奇的。而且,不管他多么赞赏但丁,他仍然认为但丁作品中没有俄罗斯音乐和诗歌那种神奇的魅力。

克雷莫夫却认为杜勃罗留波夫和拉萨尔,车尔尼雪夫斯基和恩格斯之间没什么区别。他认为,马克思高于一切俄罗斯天才人物,贝多芬的英雄交响曲毫无疑问胜过俄罗斯的音乐。也许只有涅克拉索夫是例外。他认为涅克拉索夫是全世界第一位诗人。有时叶尼娅觉得,沙尔戈罗茨基不仅可以帮助她认识克雷莫夫,而且可以帮助她看清她与诺维科夫将来的关系。

叶尼娅很喜欢跟沙尔戈罗茨基谈话。往往是从令人不安的战况谈起,然后沙尔戈罗茨基就议论起俄罗斯的命运。

“俄罗斯贵族,”他说,“是有罪于俄罗斯的,叶夫根尼娅·尼古拉耶芙娜。但他们也珍爱着俄罗斯。第一次世界大战,我们不应该得到丝毫宽恕。傻瓜,蠢货,饱食终日的饭桶,拉斯普京[28],米亚索耶多夫上校,椴树林荫道,逍遥自在的生活,没有烟囱的农舍,树皮鞋……一律完蛋。我姐姐有六个儿子死在加里西亚和东普鲁士,我大哥又老又病,也在战斗中牺牲了,但是历史不给他们算上这些……应该算呀。”

叶尼娅常常听他评论文学,他的观点与现在的观点完全不同。他认为费特在普希金与丘特切夫之上。他对费特熟悉的程度,自然没有一个俄罗斯人能比得上,也许费特生前能记得的关于自己的事,还没有沙尔戈罗茨基知道的多。

他认为列夫·托尔斯泰太实际了,虽然承认他有诗意境界,却不看重他。他是看重屠格涅夫的,却认为屠格涅夫是一位不够深刻的天才作家。在俄罗斯小说家中,他最喜欢果戈理和列斯科夫[29]。

他认为,摧残俄罗斯诗歌的祸首是别林斯基和车尔尼雪夫斯基。他对叶尼娅说,除了俄罗斯诗歌,他还爱三样东西:糖、太阳和睡觉。

他问道:

“我还没看到我的任何一首诗得到发表,难道我能死吗?”

有一天,叶尼娅在下班回家的路上遇到里蒙诺夫。他拄着疙疙瘩瘩的拐杖在街上走,敞着皮大衣,一条鲜艳的方格围巾从脖子上耷拉下来。这个头戴名贵的海狸皮帽的高大的人在古比雪夫的人群中显得非常奇怪。

里蒙诺夫陪叶尼娅走到门口。她请他进去喝杯茶。他凝神看了看她,说:

“好吧,谢谢,不管怎么说,为了户口的事,您应该请我喝两杯。”于是一面喘着粗气,一面上楼。

里蒙诺夫走进叶尼娅的小小的房间,说:

“唔,唔,这儿对于我这样胖大的身体来说,是很窄小的,不过,对于思想,也许是很宽敞的。”

他忽然用一种极不自然的语调和她谈起来,谈起自己的爱情理论和男女关系。

“维生素缺乏症,精神上的维生素缺乏症!”他喘着粗气说。“您要知道,这是一种很强的饥饿,就像非常需要盐的公牛、母牛和麋鹿那样。我自己身上没有的,我的家里人、我的妻子身上没有的,我就在我所爱的人身上找。妻子是维生素缺乏症的根源!于是男人就渴望在自己所爱的女人身上找到几年几十年在自己妻子身上找不到的东西。您明白吗?”

他抓住她的胳膊,抚摩起她的手掌,然后又抚摩她的肩膀,摸她的脖子、脑后。

“您明白我的意思吗?”他用甜蜜的口吻问道。“非常简单嘛。精神上的维生素缺乏症!”

叶尼娅用冷笑和发窘的眼睛看着他那指甲修剪得光滑的白白的大手从她的肩膀溜到胸脯上,就说:

“看起来,维生素缺乏症不只是精神上的,也是肉体上的呢。”又用老师教训一年级小学生的口吻说:“别拉拉扯扯,真的,不准。”

他惊慌地看了看她,不过并不羞惭,倒是笑了起来。她也和他一起笑起来。

他们一面喝茶,一面谈艺术家萨里扬。沙尔戈罗茨基老头子来敲门了。

原来,里蒙诺夫早就从有些人的手稿和档案馆藏的信札中知道沙尔戈罗茨基的名字。沙尔戈罗茨基没读过里蒙诺夫的作品,但也知道他的名字。报纸列举专写历史军事题材的作家时,常常出现这个名字。

他们谈了起来,一感觉到有共同语言,便兴奋起来,高兴起来,在他们的谈话中不时出现一些名字,如索洛维约夫、梅列日科夫斯基[30]、罗扎诺夫、吉皮乌斯、别雷[31]、别尔嘉耶夫、乌斯特里亚洛夫、巴尔蒙特[32]、米留可夫[33]、叶夫列伊诺夫[34]、列米佐夫[35]、维亚切斯拉夫·伊万诺夫[36]。

叶尼娅心想,这两个人好像把早已沉没的一个书籍、绘画、哲学体系和戏剧场景的世界从海底捞了出来。

里蒙诺夫忽然把她的这一想法说出口来:

“咱们好像把早已沉没的大西洲从海底捞出来啦。”

沙尔戈罗茨基伤感地点点头,说:

“是啊,是啊,不过您是俄罗斯的大西洲的考察者,我却是大西洲的居民,跟大西洲一起沉到了大洋底层。”

“这没什么,”里蒙诺夫说,“战争已经把一些人从大西洲捞到水面上来啦。”

“是啊,”沙尔戈罗茨基随口说,“结果共产国际的创造者再也想不出别的好法子,只会重复说:俄罗斯土地是神圣的。”

他笑了笑。

“别着急,等战争胜利了,那时候国际主义者们就要说:‘我们的俄罗斯祖国是全世界的首领。’”

奇怪的是,叶尼娅感觉到,他们谈得这样热烈,这样没完没了,这样俏皮,不仅是因为高兴他们的相遇,不仅是因为找到了共同感兴趣的话题。她明白,他们(一个已经完全老了,一个也早已上了年纪)一直都能感觉到她在听他们说话,他们都很喜欢她。这有多么奇怪呀。还有,奇怪的是,他们谈话她一点也不感兴趣,甚至觉得可笑,可同时又并非完全不感兴趣,而是有几分愉快。

叶尼娅望着他们,心想:“了解自己是不可能的……为什么我为过去的生活这样难过?为什么我这样怜悯克雷莫夫?为什么我一个劲儿地想着他?”

就像过去与克雷莫夫来往的那些共产国际的德国人和英国人使她非常反感一样,现在沙尔戈罗茨基用嘲笑的口气说起国际主义者,她听着也很厌烦、很反感。就连里蒙诺夫的维生素缺乏论也不能帮她理清头绪。再说,这类事也跟理论无关。

她忽然觉得,她一直想着克雷莫夫,一直为他担心,仅仅是因为她在想念另一个人,但那个人她几乎完全没有想起来。

“难道我真的在爱他?”她惊讶地想。

二十六

夜里,伏尔加河上空的黑云散尽。被山谷里浓浓的夜色劈开的一座座山冈,在星空下缓缓荡漾着。

有时流星在天空划过,于是柳德米拉不出声地说:

“让托里亚活着吧。”

这是她唯一的祝愿。她对苍天再也没有别的要求了。

当年她还在数学物理系上学的时候,就在天文研究所做过计算员。那时候她听说,流星在各个月份成群地迎着地球流动,有英仙流星群、猎户流星群,好像还有双子流星群、狮子流星群。她已经忘记,在十月、十二月跟地球相会的是哪些流星群了。但是让托里亚活着吧!

维克托责怪她,说她不爱帮助人,说她对他家的人不好。他认为,如果柳德米拉愿意的话,他母亲就会跟他们住在一起,不会留在乌克兰了。

当维克托的堂兄从集中营里放出来,即将被送往流放地的时候,柳德米拉不愿意让他留宿,怕房管所知道这事。她知道:母亲至今耿耿于怀,父亲病危时,柳德米拉正住在加斯普拉休假,等她度完假赶回莫斯科,已经是下葬后第二天了。

母亲有时和她谈起米佳,为他的事情担心害怕。

“他是一个老实孩子,一辈子都是这样。居然说他从事间谍活动,说他谋杀卡冈诺维奇和伏罗希洛夫……简直是荒唐,胡说八道!什么人要这样造谣?是什么人要陷害忠实、正直的好人?”

有一天她对母亲说:

“你不能完全为他担保。没罪的人是不会抓起来的。”

现在她想起了当时母亲看她的那种目光。

有一次她对母亲说到米佳的妻子:

“我一辈子都讨厌她,说实在的,现在我还是非常讨厌她。”

现在她也想起了母亲的回答:

“可是你要知道,做妻子的因为不检举丈夫而被判十年徒刑,这说明了什么!”

随后她又回忆起,有一次她在街上捡到一条小狗,带回家中,可是维克托不愿意收养这条小狗,她便大声对他说:“你这人真冷酷!”

他这样回答她:

“唉,我的柳德米拉呀,我不希望你年轻漂亮,只希望你的善良心肠不只是对猫和狗。”

现在她坐在甲板上,第一次不袒护自己,不责怪别人,回想着一生中听到的一次次责难的话……有一次丈夫打电话时笑着对人说:

“自从我们家养了一只小猫,我能听到妻子亲热的声音了。”

有一次,妈妈对她说:

“柳德米拉,你怎么不肯可怜乞丐呢,你想想看:这是吃不饱的人向你吃饱的人乞讨呀……”

但是她并不吝啬。她是好客的,她做的一手好菜,在朋友们中间是出了名的。谁也看不见这天夜里她坐在甲板上哭。就算她心肠硬好了,她把所学的东西全忘了,她一点用处也没有,谁也不会喜欢她了。她已经发胖,头发也灰白,又有高血压,丈夫不爱她了,所以才觉得她冷酷无情。但是只要托里亚活着就行!她准备什么都承认,家里人认为她不对的地方,她都认错、改正,只要托里亚活着就行!

为什么她一直记着自己的第一个丈夫呢?他在哪儿?怎么能找到他呢?为什么她没有给他在罗斯托夫的姐姐写信?现在想写也不行了,那里有德国人。他姐姐如果知道托里亚的情况,会告诉他的。

轮机轰鸣,甲板颤动,水花拍溅,天空的星光全混合到一起,融汇到一起,于是柳德米拉睡着了。

黎明渐渐近了。夜雾在伏尔加河上飘荡,似乎一切有生命的东西都沉没在雾中。忽然跃出一轮红日,好像又迸发出希望。蓝天倒映在水中,阴郁的秋水呼吸起来,太阳也好像在浪花上雀跃。岸坡上夜里落了厚厚的一层白霜,红色的枫树在白霜里显得分外悦目。晨风吹来,雾气消散,世界变得像玻璃一般明净剔透。不论是明亮的朝阳还是蓝天碧水,都没有一丝暖意。

大地是辽阔的,大地上的森林看去也是无边无际的,其实既能看到森林的头,又能看到森林的尾,可大地是无穷无尽的。

像大地一样辽阔、一样长久的,是痛苦。

她看到坐在一等舱里的人民委员会领导干部,穿着草绿色大衣,戴着灰色羊羔皮军帽。在二等舱里坐的是显要们的妻子和丈母娘,穿着打扮都与身份相称,似乎妻子们有妻子们的特别服饰,丈母娘和妈妈们也有自己的特别服饰。妻子们穿皮袄,戴白色长绒毛头巾;丈母娘和妈妈们穿蓝呢子皮袄,黑色羊羔皮翻领,咖啡色头巾。跟她们在一起的孩子们都流露着苦闷和不满的神情。从舱房窗户里可以看到这些乘客带了很多吃的东西。柳德米拉经验丰富的眼睛很容易看清装在各种容器里的东西。有蜂蜜,有炼过的油,装在一个个罐子坛子里,用火漆封了口的黑色大瓶里,顺着伏尔加河,朝下游而去。有些高等乘客在甲板上散步,从他们谈话的片断可以听出来,他们最关心的是从古比雪夫开往莫斯科的火车。

柳德米拉觉得,那些高等女乘客看到坐在过道里的红军士兵和尉官们,表情都很冷漠,好像她们都没有儿子和兄弟在前方。

在播送苏联情报局的晨间新闻的时候,她们并不跟那些睡眼惺忪的战士和水手一起聚在喇叭下面听,而是走来走去干自己的事情。

柳德米拉从水手们那里打听到,这艘船是包给一些党政干部及其家属的,他们要经过古比雪夫回莫斯科,军事机关命令这艘船在喀山停靠,上一部分军队和普通乘客。原定的合法乘客们大闹了一场,反对让军人上船,还打电话给国防委员会特派员。

这些开赴斯大林格勒的红军战士,竟然觉得自己挤了合法的乘客,脸上露出歉疚的神气,令人感到说不出的奇怪。

柳德米拉觉得,高等女乘客们那种心安理得的眼神特别使人难以忍受。老奶奶们把孙子唤到跟前,一面继续说话,一面很熟练地把糖果往孙子们嘴里塞。等到从船头的一个舱里走出一个穿黄鼬皮皮袄的小个子老太太,带着两个孩子在甲板上玩儿,女乘客们都慌不及待地向她鞠躬、微笑,而在那些政治活动家们的脸上则出现了亲切和诚惶诚恐的表情。

如果现在广播电台宣布开辟了第二战场,列宁格勒包围圈已经突破,他们谁也不会动一下;但如果有人告诉他们,莫斯科列车的国际车厢已经取消,一切战争大事就会被争购软卧票和硬卧票的劲头儿淹没。

真奇怪呀!柳德米拉也穿着灰羊羔皮袄,戴着长绒毛头巾,论服装也跟一等舱、二等舱的乘客差不多。不久前她也曾争着购买卧铺车票;维克托到莫斯科出差,没买到软席票,她还生气呢。

她对一位炮兵中尉说,她的儿子也是炮兵中尉,受了重伤,现在躺在萨拉托夫军医院里。她跟一个有病的老奶奶谈到玛露霞和薇拉,谈到身在沦陷区的婆婆。她的痛苦,跟这甲板上的痛苦气氛,跟那种总是牵连着军医院、前线坟地与乡村农舍、无名空地上没有门牌的棚屋的痛苦,是一样的。她离家时没有带茶杯,没有带面包;似乎她一路上不需要吃,也不需要喝。

但是,从早晨起,她在船上就非常想喝水,她知道,她要受罪了。第二天,红军战士们和船上司炉商量好,在机器舱里煮了一锅麦粒儿汤,把柳德米拉叫去,给她盛了一饭盒汤。

柳德米拉坐在空箱子上,用别人的饭盒和调羹喝起热汤。

“这汤好极啦!”一名炊事兵对柳德米拉说。因为她没有作声,炊事兵又问她:“怎么,不好吗?不是浮着一层油吗?”

红军战士请她喝汤,又希望她夸汤好喝,她可以感受到战士的大方和朴实。

一名战士的自动步枪出了毛病,弹簧塞不进去,就连带红星勋章的准尉也没办法,她却帮着把弹簧塞了进去。

柳德米拉听了几名炮兵尉官的争论,她拿起铅笔,帮他们解了一道三角公式。

解出公式以后,一名原来喊她“女公民”的中尉忽然问起她的名字和父名。到夜里,柳德米拉依然在甲板上徘徊。

河上弥漫着冰一般的寒气,下游来的狂风从黑暗中冲来。头顶上星光闪烁;高悬在她的不幸的头上的、由火与冰构成的无情的天空,既不能给人安慰,又不能使人安宁。

二十七

轮船抵达战时临时首都之前,船长接到命令,要继续往前开,开往萨拉托夫,接运萨拉托夫军医院的伤员。

坐在一、二等舱里的乘客开始准备下船了。他们把提箱、公文包拿出来,放到甲板上。

开始看到工厂的轮廓,一座座铁皮顶的楼房、棚屋,似乎船尾的水声也变了,轮机声也变得更惶惶不安了。

然后,宽阔的萨马拉河开始慢慢出现。河水有灰色的、红色的、黑色的,有时像光闪闪的碎玻璃,有时裹在一股股工厂与火车头喷出的灰烟之中。

在古比雪夫下船的乘客站到了船舷边。

下船的人并不彼此道别,也不向留下的人点头致意。他们在路上没有交朋友。

一辆“齐斯—101”牌的小汽车等候着穿黄鼬皮皮袄的老奶奶和她的两个孙子。一个穿将军呢大衣的黄脸男子向老奶奶行了一个军礼,又跟两个孩子握了握手。

过了几分钟,带着孩子、提箱和公文包的乘客们消失了,就好像本来就没有他们似的。

轮船上只剩下许多军大衣和棉军装。

柳德米拉觉得,这些人都是由共同的命运、劳动和痛苦联结在一起的,现在她在这些人当中,呼吸起来就轻松些、痛快些了。

可是,她错了。

二十八

在萨拉托夫迎接柳德米拉的是粗暴和冷酷。

她一踏上码头,就和一个身穿军大衣的醉汉相撞,醉汉打了一个趔趄之后,一把把她推开,又骂了一句脏话。

柳德米拉顺着石子铺砌的很陡的岸坡往上爬,爬了一会儿,停了下来,喘着粗气,回头看了看。那轮船在下面,在一个个灰色的码头货栈中间显得很白。轮船好像知道她在向它告别,发出低低的、断续的汽笛声,好像在说:“你走吧,走吧!”于是她走了。

在上电车的时候,一些年轻女子一声不响地拼命推挤老年人和病弱的人。有一个头戴红军帽的盲人,看样子是从军医院出来不久的,还不会摸索着单独行动,两只脚急急慌慌地倒换着,拿小棍儿在面前直捣。他像个孩子一样急切地抓住一个不怎么年轻的妇女的衣袖。那妇女把胳膊一抽,朝旁边跨了一步,钉了铁掌的靴底在石子路面上叮当响了两声。他还要去抓她的袖子,并且连忙解释说:

“请帮我上车,我是刚从军医院出来的。”

那妇女骂了一声,把瞎了眼的伤兵一推,那伤兵失去平衡,一屁股坐到马路上。

柳德米拉看了看那妇女的脸。

这种无人性的表情是从哪儿来的?来自什么?是来自她在童年经历过的一九二一年的饥荒?来自一九三〇年的大批大批的死亡?还是来自穷困艰难的生活?

那盲人愣了一会儿,然后一下子站起来,用鸟叫般的声音叫喊起来。他的帽子歪到了一边,无可奈何地摇晃着棍子,他那一双瞎眼,大概也清楚地看见了自己的窘境。

盲人拿棍子在空中敲打着,在这种乱摇乱打中,表达着他对冷酷的明眼人的世界的痛恨。人们推搡挨挤着往车上爬,他站在那里又哭又叫。柳德米拉怀着希望和挚爱,把他们联结为一个辛劳、贫穷、善良和痛苦的大家庭的这些人,就好像商量好了似的,坚决不做人道的事情。他们似乎商量好了要推翻一种说法,这种说法就是:穿油污衣裳、在劳动中弄黑了手的人,心肠必定是善良的。

柳德米拉的心触到一种令人难受的、黑沉沉的东西,就好像来到俄罗斯那数千里的贫瘠土地上,感到寒冷与黑暗,这是置身现实生活的冻土带时的无可奈何。

柳德米拉问女售票员,应该在哪儿下车。女售票员冷冷地说:

“我已经说过了。你聋了吗?”

有些乘客站在电车通道上。问他们是不是要下车,他们也不回答,像石头一样,动也不动。

过去柳德米拉曾经上过萨拉托夫女子中学初级预备班。冬天的早晨,她坐在饭桌旁,悠荡着两条腿,喝着茶,她心爱的父亲给她往热烘烘的白面包上抹奶油,灯光映照在茶炊圆圆的肚子上。她不愿意离开父亲温暖的手,不愿丢下热烘烘的面包,不愿离开热气腾腾的茶炊。似乎那时在这座城市里没有寒风,没有饥饿,没有自杀的人,医院里没有奄奄一息的孩子,只有温暖,温暖,温暖。

她的大姐索菲亚死于喉炎,就葬在这里的坟地。妈妈给大姐取名索菲亚,为的是纪念因为谋刺沙皇而被处死的女革命家索菲亚·里沃菲娜·佩罗夫斯卡娅。爷爷好像也葬在这里的坟地。

她来到一座三层的学校大楼跟前,这就是托里亚所在的军医院。

门口没有岗哨。她觉得这是好兆头。她感觉到医院里的空气,气味是那样浓重,就连冻得要死的人也不会喜欢这里的温暖,宁愿离开这里再上寒冷的地方去。她从厕所旁边走过,门口还挂着过去的牌子:“男生厕所”、“女生厕所”。她经过走廊,厨房里的气味朝她扑来。她又往前走,透过蒙了一层水汽的玻璃看到院子里堆着不少长方形的棺材。她又像在家里拿着未打开的信那时候一样,心想:“天啊,万一已经死了呢。”可是她放大了步子又朝前走去,走上灰灰的地毯,从一个个床头小柜和她所熟悉的天门冬和蓬莱蕉中间穿过,来到一个门口,门口挂着“四年级”的牌子,并排挂着手写的牌子:“病历室。”

柳德米拉抓住门把手。阳光穿过乌云,射在窗户上,四周一下子都亮了。过了几分钟,爱说话的管理员一面在被阳光照得亮闪闪的长匣子里翻着病历卡,一面对她说:

“噢,噢,就是说,沙波什尼科夫,阿……哦……阿纳托里·维……噢……您很幸运,没有碰到我们的警卫长。不脱大衣,他要是看见了,够您受的……噢,噢……就是说,沙波什尼科夫……就是,就是,就是他,中尉,不错。”

柳德米拉看着他的手从长长的胶合板匣子里抽出卡片,她似乎站到了上帝面前,等候上帝告诉她是死是活,可是她一时之间呆住了,弄不清她的儿子是死了还是活着。

二十九

柳德米拉来到萨拉托夫的时候,给托里亚做过上一次手术,即第三次手术之后,已经过了一个星期。做这次手术的是二级军医麦捷尔。手术又复杂,时间又长。托里亚有五个多钟头处在全身麻醉状态中,两次静脉注射安眠朋钠。军医院的军医和医科大学的临床医生中,都没有人在萨拉托夫做过类似的手术,只见过文字材料,美国一份军事医学杂志在一九四一年发表过类似手术的记载。

因为这项手术特别复杂,在做过例行的X 光检查之后,麦捷尔医生曾经和托里亚进行过长时间的、坦率的交谈。他向托里亚解释了重伤之后在他机体内发生的病理变化的性质。同时医生也坦率地说了手术中可能出现的危险。他说,会诊的医生的意见并不一致,老医师罗季奥诺夫就反对这次手术。托里亚向麦捷尔医生提了两三个问题,略作思索之后,就在X 光室里表示同意做手术。为这次手术做准备,用了五天时间。

手术从上午十一点开始,到下午四点多钟才结束。在做手术的时候,军医院院长、军医季米特鲁克也在场。在场观察手术的医生们都认为,手术做得非常漂亮。

麦捷尔医生在手术台边当机立断,正确地解决了事先未料到的以及文字记录中不曾提到的难题。

手术时病人的状况是令人满意的,脉搏正常,没有减弱。下午两点钟左右,已经不年轻的、胖大的麦捷尔医生感觉体力不支,只好暂停几分钟。内科医生给他注射了一针戊酸薄荷脑脂,之后麦捷尔医生再也没休歇,一直把手术做完。可是,手术结束后不久,托里亚刚刚被送进隔离病房,麦捷尔医生就心绞痛发作,情况很严重。只有一再地注射樟脑剂,服用硝化甘油,到夜里才把心绞痛压下去。显然,心绞痛是神经紧张和健康欠佳的心脏超负荷工作引起的。

值班护士捷连季耶娃遵照指示观察托里亚中尉的病情。克列斯托娃医生走进病房,摸了摸尚处在昏迷状态的托里亚的脉搏。病人的情况很好,克列斯托娃对护士说:

“麦捷尔把沙波什尼科夫中尉救活了,可是麦捷尔自己差点儿送命。”

护士捷连季耶娃说:

“噢嘿,万一光是中尉托里亚活下来,那才够受呢!”

托里亚呼吸几乎没有声音。他的脸一动也不动,细细的手臂和脖子就像是小孩子的,苍白的皮肤上还保留着战地作业和草原行军中晒黑的痕迹,就像隐隐约约的影子。托里亚的状况介乎昏迷和睡梦之间:一方面是麻醉药的力量尚未完全消退,一方面是体力和精力受到巨大消耗。

托里亚迷迷糊糊地吐出一些不相关的词儿,有时也说出连贯的句子。捷连季耶娃觉得他好像很快地说了一句:

“你没看到我这个样子,太好了。”

说过这一句以后,他不作声了,两个嘴角耷拉下来,就好像他在昏迷中不出声地哭了。

晚上八点左右,他睁开眼睛,并且很清楚地说要喝水,护士一见这情形,非常高兴,非常惊讶。她告诉他,他现在不能喝水,又告诉他,手术十分成功,完全可以复原。她问他感觉如何,他回答说,背部和腰侧都不怎么疼痛。

她又试了试他的脉搏,往他的嘴上和额头上敷了湿毛巾。

这时候卫生员麦德维杰夫走进病房,说外科主任普拉托诺夫医生打电话找护士捷连季耶娃。捷连季耶娃来到值班室里,拿起话筒,向普拉托诺夫汇报说,病人已经醒了,就一个经过大手术的病人来说,情况完全正常。

护士捷连季耶娃要求派人接替她,她要上市军委会去,因为给她丈夫的领款证的地址写错了。普拉托诺夫答应让她去,但叫她继续观察一会儿,等会儿普拉托诺夫亲自来接替她。

护士捷连季耶娃回到病房。病人依然躺着未动,还是她离开时那个样子,但脸上的痛苦表情不那么强烈了:嘴角抬上去了,脸色平静,似乎在笑。看样子,一直很痛苦的表情使托里亚的脸显得苍老,现在这一副笑脸使护士捷连季耶娃感到吃惊:那瘦小的脸,那苍白而饱满、微微撅起的嘴唇,没有一丝皱纹的高高的额头,似乎不是属于一个成年人,甚至也不属于一个大孩子,而是属于一个小孩子的。她问他感觉如何,他没有回答,看样子,是睡着了。

捷连季耶娃又看了看他脸上的气色,有点儿不放心。她抓起他的手,没有摸到脉搏,手只是多少有一点儿热乎,这是勉强能感觉到的余热,就好比前一天生的炉子,早已熄灭,但到早晨还保留着一点儿微热。

尽管护士捷连季耶娃一直生活在城市里,可是她跪了下来,为了不惊动活着的人,轻轻地、像农村妇女那样哭号起来:“我们的亲人呀,最最心爱的人呀,你怎么就走了呀?”

三 十

军医院里已经知道沙波什尼科夫中尉的母亲来了。接待死者母亲的是军医院政委、营级政委希曼斯基。他是一个漂亮男子,听口音可以知道他是波兰出生的。他皱着眉头等待柳德米拉到来,他以为她必然要流泪,也许还会昏过去。他用舌头舔着刚长出来的胡子,为死去的中尉、为死者的母亲难过,并且因此也生起中尉和他妈妈的气:如果每一个死去的尉官的妈妈都需要接待,神经怎么能受得了呀?

希曼斯基请柳德米拉坐下,在开始谈话之前,先递给她一杯水。于是她说:

“谢谢您,我不渴。”

她听他谈了手术前会诊的情形(这位政委认为没必要说有一人曾经反对做手术),谈了这次手术的困难,谈了这次手术进行得很好;又说,医生们认为,对于沙波什尼科夫中尉这样的重伤,应该做这种手术。他说,沙波什尼科夫死于心脏麻痹,经过三级军医鲍尔德廖夫病理解剖,得出结论:这次突然变化,医生是无法预测,也无法排除的。

接着政委又说到,军医院来的病人成百上千,可是很少有人像沙波什尼科夫中尉这样受到医护人员喜爱。他又自觉,又文雅,又有礼貌,总是不好意思提什么要求,怕麻烦医护人员。

希曼斯基说,一个做妈妈的,养育出这样一个忠诚无私地把生命献给祖国的儿子,应当感到自豪。

然后,希曼斯基问她,对医院领导有没有什么要求。

柳德米拉说,占用政委不少时间,请多原谅,接着她从小包里抽出一张纸,念起自己的要求。

她要求把儿子的埋葬地点告诉她。

政委一声不响地点了点头,并在小本子上记下来。

她希望和麦捷尔医生谈一谈。

政委说,麦捷尔医生听说她来了,也很想和她见见面。

她要求见见护士捷连季耶娃。

政委点点头,又在小本子上记了一下。她要求把儿子的遗物给她,作为纪念。

政委又记了记。

然后她要求把她给儿子带来的礼物转送给别的伤员,接着就把两罐鲱鱼罐头和一包糖果放到桌子上。

她的眼睛和政委的眼睛相遇。政委的眼睛遇到她那蓝蓝的大眼睛的光芒,不由得眯缝起来。

希曼斯基请柳德米拉第二天上午九点半到医院来,她所有的要求都不成问题。

政委看了看已经关上的门,看了看柳德米拉要求转送其他伤员的礼物,他摸了摸自己手上的脉搏,没有找到脉搏,就把手一挥,喝起水来,这水便是开始谈话前请柳德米拉喝的那一杯。

三十一

似乎柳德米拉没什么空闲时间。夜里她在大街上走来走去,在公园里的长椅子上坐了坐,到车站里面暖和了一阵子,就又迈着郑重其事的快步子在空荡荡的大街上来来回回地走。

她所要求的事,希曼斯基全给办了。

上午九点三十分,护士捷连季耶娃来见柳德米拉。

柳德米拉请她说说她所知道的有关托里亚的一切。

柳德米拉穿上罩衫,和捷连季耶娃一同登上二楼,从她儿子当时进手术室经过的走廊走过,在一个单间病房的门前站了一会儿,看了看这天上午空出来的病床。护士捷连季耶娃一直走在她旁边,用手帕揩着鼻子。柳德米拉又下到一楼,捷连季耶娃便和她分开了。不久,接待室里进来一个人,白头发,胖大的身子,黑黑的眼睛下面有两个黑黑的圈儿。麦捷尔医生浆过的白罩衫跟他那黑黑的脸和睁得老大的黑眼睛相比,显得很白很白。

麦捷尔对柳德米拉说了说,为什么罗季奥诺夫教授反对做这次手术。柳德米拉想问的事,他似乎全猜到了。他对她说了说手术前他和托里亚谈的话。他很理解柳德米拉的心情,一丝不苟、毫不隐瞒地讲了一遍手术过程。

然后他说,他对中尉托里亚有一种特殊感情,几乎是一种父爱。在这位医生低沉的声音中,有一种碎玻璃碴一样的声音又尖细又悲戚地响起来。她第一次看了看他的手,那是一双很特别的手,似乎不是长在这个眼神悲戚的人的身上的。那手粗大而沉重,手指头黑黑的,粗实有力。

麦捷尔把一双手从桌上抽回去。他似乎在念她心中的想法,说:

“能做的事,我全做了;但结果是,我的手加快了他的死亡,而没有战胜死亡。”他又把一双手放到桌子上。

她明白,麦捷尔说的一切都是事实。他说的有关托里亚的每一句话,她都非常希望听,但每一句都让他痛苦又难受。可是,他这些话里还有一种很难受的沉重感。她觉得,麦捷尔医生希望和她见面不是为了她,而是为了他自己。这使她心中对麦捷尔产生了不好的感觉。

在麦捷尔医生要走的时候,她说,她相信他为了挽救她的儿子,能做的事全做了。他沉重地喘了一口气。她感觉到,她的话使他轻松了。这样她又明白了,他因为感到自己有权从她嘴里听到这样的话,所以希望和她见面,于是和她见面了。

她带着责备的意味在心里想道:“难道还要从我这里得到安慰吗?”

麦捷尔走后,柳德米拉便朝戴皮帽的警卫长走去。他向她行了一个军礼,用嗄哑的声音报告说,政委指示用小汽车把她送到安葬的地方去,小汽车还要等十分钟才来,因为有人用车到票证发放处送文职人员名单去了。中尉托里亚的东西已经收拾好了,最好是从坟地回来后再带走。

柳德米拉提出所有的要求全做到了,而且一丝不苟,不打折扣,就像执行军令一样。不过,从政委、护士、警卫长对她的态度中可以感觉出来,这些人也想从她这里得到宽恕和安慰。

政委因为医院里常常死人,感到自己有责任。在柳德米拉来医院之前,他并没有为此感到不安。医院嘛,总是要死人的,尤其是在战争时期。医疗服务工作的组织安排,并未引起上级领导的责难。经常使他受批评的是政治工作做得不够,没有很好地报导伤员的顽强精神。

部分伤员不相信战争能胜利,还有一部分政治落后的伤员,对集体农庄制度抱有敌对情绪,恶意攻击,他跟这些斗争不够坚决。在医院里还有一些伤员传播军事机密的事件。

军区卫生部政治处曾经把希曼斯基叫了去,告诉他,如果特别处再次汇报说医院思想混乱,就要把他调到前方去。

现在政委见到死去的中尉的妈妈,感到非常羞愧,因为昨天死了三名伤员,可是昨天他还洗了淋浴,让炊事员用炖好的酸白菜给他做了可口的下酒菜,喝了从市商业局弄来的一小桶啤酒。护士捷连季耶娃见到死去的中尉的妈妈也感到羞愧,因为她的丈夫是军事工程师,在集团军参谋部工作,没有上过前方,她的儿子比托里亚还大一岁,却在飞机工厂设计处工作。警卫长羞愧的是,他是一名基干军人,却在后方医院工作,他还把一匹上等的华达呢衣料和一双精制的毡靴寄回家,可是死去的中尉留给妈妈的只有棉军装。

经管死去伤员的殡葬事务的司务长,厚嘴唇,大耳朵,他在陪同柳德米拉前往坟地的时候,也感到羞愧。棺材都是用薄薄的废木板钉成的。死者只穿着内衣入殓。普通士兵的棺材排得十分拥挤,都成为合葬的坟墓。坟上的墓碑都是未刨光的木牌,文字写得歪歪扭扭,而且是用容易褪色的颜料写的。当然,师卫生营里的死者都是直接埋进坑里,连棺材都没有呢,木牌上的字是用变色铅笔写的,一下雨就冲掉。还有那些死在战斗中,死在森林里、沼地上、山沟里、旷野上的人,还常常得不到安葬呢,埋葬他们的往往是沙土、枯叶、风雪。

但是,当这位妇女跟他一起坐在汽车里,问他怎样安葬死者,问他是不是合葬,给死者穿什么服装,在坟地上是否致悼词的时候,他还是因为棺材木料太差而感到羞愧。

他感到不好意思,还因为他在出来之前曾跑到军需仓库一个朋友那里去,喝了一小罐加水的药用酒精,还就着大葱吃了一块面包。使他感到难为情的,是汽车里充满了他呼出来的酒气和大葱气味,可是,不论他多么难为情,不呼吸是不行的。

他愁眉苦脸地望着挂在司机前面的反光镜。在这四四方方的小镜子里映照出司机那一双带笑的、使司务长感到惭愧的眼睛。

“司务长,你喝醉啦!”司机那一双年轻而快活的眼睛不客气地说。

所有的人在牺牲了儿子的母亲面前都感到羞愧,而且,不论人类历史多么长久,想对她说明自己无愧,都是徒然的。

三十二

劳动营的士兵们正从卡车上往下卸棺材。他们不声不响,不慌不忙,可以看出他们干这种活儿已经熟练和习惯了。一个人站在车斗里,把棺材推到边沿上,另一个人用肩膀接住,往外一拖,又一个人不声不响地走过来,用肩膀接住棺材的另一边。他们咯吱咯吱地踩着上了冻的土地,把棺材抬到宽大的合葬坟里,贴着坟坑的边放好,又回到卡车跟前。等到卸空了的卡车回城里去了,士兵们便在墓穴旁的棺材上坐下来,拿出一叠废纸和一丁点儿烟丝卷烟卷儿。

“今天好像空闲些。”一个士兵说着,用装配得很好的打火家什打起火来—细绳的火绒塞在铜弹壳里,火石嵌在里面。这个士兵把火绒摇了两下,就冒出烟来。

“司务长说,今天就一汽车,再没有了。”另一名士兵说着,喷了一大口烟,抽起烟卷儿。

“那咱们可以封坟啦。”

“过一会儿当然好些,他还要拿名单来,要检查。”另一名没抽烟的士兵说着,从口袋里掏出一块面包,打了打灰,又轻轻吹了吹,便吃起来。

“你跟司务长说说,让他给咱们发铁钎。这地冻了好几尺厚,明天还要挖新坟,像这样的地用铁锹能挖得动吗?”

刚才在打火的那一名士兵,用手叭叭拍了两下,把木头烟嘴里的烟灰拍出来,又轻轻地拿烟嘴在棺材盖上敲了敲。三个人都没有说话,好像在听什么。什么声音也没有。

“听说,要给劳动营发干粮了,是真的吗?”吃面包的士兵说。他把嗓音放得很低,为的是不打搅棺材里的死者,知道他们对这些话不感兴趣。

另一个抽烟的士兵把烟灰从长长的芦苇烟嘴里吹出来,又对着亮光朝烟嘴里看了看,摇了摇头。

还是没有什么声音。

“今天天气不坏,就是有风。”

“听,汽车来了,这一下子咱们要干到中午了。”

“不对,这不是咱们的大汽车,是小汽车。”

从小汽车里走出他们熟悉的司务长,接着出来的是一位戴头巾的妇女。他们朝铁栏杆那边走去,在上个星期之前都是在那里埋死人,后来因为已经没有地方,就不在那里挖坟了。

“埋葬军人,没有一个人送葬,”一名士兵说,“在和平时期,你要知道,一口棺材,后面上百人捧着鲜花。”

“也有人哭这个人的。”一名士兵用厚厚的长圆形指甲很有礼貌地敲了敲棺材板,指甲因为干活儿磨得像海边石子一样光溜。“只不过那些眼泪咱们看不到……瞧,司务长一个人来了。”

他们又抽起烟来,这一次三个人都抽了。司务长走到他们跟前,和善地说:

“同志们,咱们都抽烟,谁又替咱们干活儿呢?”

他们一声不响吐出三个烟团儿,接着,刚才打火的那个士兵说:

“你也抽一口吧,听,咱们的卡车又来了。我从马达声能听出来。”

三十三

柳德米拉走到一个坟包前面,念了念写在胶合板上的儿子的姓名和军衔。她清楚地感觉到,在头巾下面的她的头发动了起来,不知是谁的冰冷的手指头在拨弄她的头发。

左边,右边,直到栏杆边,全是灰灰的坟包,没有青草,没有鲜花,只有插在坟土里的一根根木杆。木杆顶上钉着胶合板,上面写着一个人的姓名。胶合板有许多,密密麻麻,全都是一个样子,很像田野里长得很茂盛的庄稼。

她现在终于找到了托里亚。有多少次,她拼命猜想,他在哪儿,在干什么,想什么,他是倚着战壕的土壁打瞌睡,还是在路上走,是不是一只手端着茶缸、另一只手拿着糖块喝茶,是不是冒着枪林弹雨在田野上奔跑……她很希望跟他在一起,他需要有妈妈—她可以给他斟茶,对他说:“再吃块面包吧。”她给他脱鞋,给他洗磨出泡的脚,给他脖子上围围巾……每次他走了,她都无法找到他。现在她终于找到了托里亚,可是他已经不需要她了。

稍远处可以看到革命前的一些坟墓,坟前还有大理石十字架。那些十字架就像是一群谁也不要、跟谁也没有关系的老头子—有些歪倒在一旁,有些软弱无力地靠在树上。

天空好像是真空的,好像有人把空气抽光了,头顶之上,空空荡荡,只有干燥的灰尘。可是无声无息然而马力强大的气泵还在抽天空的空气,不停地抽着,抽着。柳德米拉觉得不仅已经没有天空,而且没有信念,没有希望,在巨大的没有空气的天地间只剩下灰灰的冻土块垒成的一个小小的土丘。

一切活着的,母亲,娜佳,维克托的眼睛,战报,一切都不再存在了。

活着的,成了死的了。世界上只有托里亚活着。可是,周围多么静呀。他是不是知道她来了……

柳德米拉跪下来,为了不惊扰儿子,轻轻地把写着儿子姓名的胶合板扶正。她记得,过去她送他上学的时候,给他理衣领,他总要生气。

“瞧,我来了,你也许在想,怎么妈妈还不来……”

她说起话来,声音小小的,怕栏杆外面有人听见。

公路上奔驰着汽车,黑糊糊的、花岗岩般的卷地的风雪在旋转,茫茫一片,在柏油路面上又绕圈儿,又打旋儿……背着口袋的人、提着牛奶桶的女人都穿着军靴,橐橐地走着,身穿棉袄、头戴棉军帽的孩子们跑着去上学。

但是她觉得这到处在活动的世界只是一种模模糊糊的幻景。

多么静啊。

她和儿子在说话,回忆着他过去生活中的细节,于是这些仅仅存在于她的记忆中的往事充满了天地间,到处是孩子的声音、眼泪,翻看小人书的沙沙声,小调羹敲打白碟子边儿的响声,自己装配的收音机的咝咝声,滑雪板的哧哧声,别墅池塘里船桨的划水声、剥开糖果纸的沙沙声,闪来闪去的孩子的脸、肩膀、胸膛。

他的眼泪、苦恼,他的好的、不好的行为,都因为她的绝望而复活了,一切如在眼前,好像可以触摸到。

她不是回忆死去的儿子,而是为他的实际生活操起心来。

干吗要在这么弱的灯光下通宵看书呀。这么年轻就开始戴眼镜,以后怎么办啊……

瞧,他就穿着薄薄的布衬衣躺在这儿,光着脚,怎么不给他盖被子,这地冰凉冰凉的,到夜里还有老厚的霜呢。

柳德米拉鼻子里忽然涌出鲜血。头巾都湿透了,沉甸甸的。她头晕,眼睛发黑,有一会儿她觉得就要昏过去。她闭上眼睛。等她把眼睛睁开,在她的悲痛中复活的世界已经消失,只有被风卷起的灰色尘土在坟墓上面盘旋着;好像是一会儿这座坟,一会儿那座坟,冒起灰烟。

奔流在坚冰之上、把托里亚从黑渊中托出来的那股仙水流走了,消失了;在母亲的绝望中出现的那个世界,一时间冲破现实的桎梏、要取代现实的那个世界,又不见了。她的绝望好像变成了上帝,让儿子从坟墓里站起来,让空中布满新的星星。

在过去的这几分钟里,世界上只有托里亚活着,其余的一切都有赖于他。但是,母亲的强大力量不能长久地使大量的人群、大海、道路、土地和城市服从死去的托里亚。

她把头巾按到眼睛上,眼睛是干的,头巾却被血湿透了。她觉得她的脸上沾满黏糊糊的血。她弯着腰坐着,渐渐平静下来,不由得在思想上迈着小小的起步,开始承认托里亚不在人世。

医院里的人见她这样平静,听到她提的问题,都感到吃惊。他们不知道,她还没有意识到他们已经很清楚的事实,没有意识到托里亚已经不在人世。她对儿子的感情太强烈了,以至于既成事实的威力丝毫不能动摇这种感情,所以他还继续活着。

她已经失去理智,谁也没看出这一点。她终于找到了托里亚。就好像老猫找到已死的小猫,又高兴,又拿舌头舔。

她的心还要经历长时间的痛苦,直到几年、也许几十年之后,慢慢地、一块石头一块石头地堆起自己的坟包,在心里清醒地感觉到永远失去了儿子,才会在既成事实的威力面前屈服。

劳动营的士兵干完活儿,已经走了。太阳就要落山,坟地上的胶合板投出了长长的影子。只剩柳德米拉一个人。

她想,应该把托里亚的死讯通知亲属们,通知在集中营里的他的父亲。一定要通知父亲。要通知亲生父亲。托里亚在手术之前想些什么呢?他吃得怎样呢?还用调羹吃饭吗?他是不是有时也侧着睡呢?还是仰着睡?他喝水喜欢加柠檬和糖呀。现在他是怎样躺着的?头发理过没有?

大概由于心里的痛苦过于沉重,周围的一切变得越来越黑沉了。

她突然想到,自己的痛苦永无尽期;将来维克托会死,她的女儿的后代们也会死。她会一直痛苦下去。

在悲痛过分沉重,内心支持不住的时候,现实与柳德米拉心中浮现的世界,界限再次消失了,她的爱打退了永恒。

她想,干吗要把托里亚的死讯通知他的生父,通知维克托和所有亲属?要知道,情况还完全不能肯定呀。最好是等一等,也许,还能好转呢。

她小声说:

“你也不必告诉任何人,情况还一点不清楚呢,还会好起来呢。”

柳德米拉拿大衣襟盖住托里亚的腿。她又从头上摘下头巾,盖住儿子的肩膀。

“上帝,可不能这样,怎么能不盖被子。哪怕把腿盖一盖也好。”

她想得出神了。在迷迷糊糊的状态中继续同儿子说话,责备他写信写得那样短。她渐渐清醒,给儿子拉了拉被风吹到一边去的头巾。

她跟儿子两个人在一起,谁也不打搅他们,多么好呀。谁也不喜欢他,都说他不漂亮:嘴唇又厚,又往上翻。都说他行动古怪,动不动就生气,发火。同样,谁也不喜欢她,家里人光看她的缺点……我的可怜的孩子,我的腼腆的、不漂亮的好儿子呀……只有他喜欢我,现在,在这黑夜里,在坟地上,只有他和她在一起,他再也不离开她,等她成了一个没人要的老婆子,他还会爱她……他是一个多么不圆滑的人啊。从来不要求什么,又羞怯,又可笑;一位女教师说,他在学校里成了取笑的对象;大家逗他,捉弄他,他就像小孩子一样哭起来。托里亚呀,托里亚,可别丢下我一个人。

后来,天亮了。伏尔加彼岸的草原上升起冷冷的红光。汽车吼叫着从大路上驶过。

精神狂乱的状态过去了。她坐在儿子坟前。儿子的身体被黄土埋了。儿子没有了。她看到自己肮脏的手指,看到铺在地上的头巾,她的两腿麻木了,觉得她的脸也弄脏了。她的喉咙里发痒。

她对一切都冷漠了。如果有人告诉她,说战争结束了,说她的女儿死了,她会无动于衷。如果旁边有一杯热牛奶,有一块热面包,她连动都不会动,手也不会伸一下。她坐在地上,既不操心,又无思虑。一切都无所谓,什么都不需要。只有不肯休歇的痛苦紧压着她的心,冲打着她的两边鬓角。医院里的人、穿白衣的医生说起托里亚的事,她看到他们那张开又合上的嘴,却没有听见他们说的是什么。地上有一封信,是从大衣口袋里掉出来的,是军医院给她的那一封,她也不想捡起来,抖一抖上面的灰土。她无意识地想起,托里亚两岁的时候,蹒跚地追赶在地上跳来跳去的蟋蟀,耐心地、毫不泄气地跟在蟋蟀后面走来走去;又想起她没有问护士,托里亚在生命的最后一天,在手术前的那个早晨是怎样躺着的,是侧着身,还是仰着。

她看到了晨光,她不可能看不到啊。

忽然她想起:托里亚满三岁了,那天晚上家里人吃着甜点心,托里亚还问:

“妈妈,为什么天黑了?今天是生日呀。”

她看到树枝,看到在阳光下闪亮的光滑的石头墓碑,看到写着儿子姓名的胶合板,字有大有小,稀密不匀。她没有想法,她没有心思了。她什么也没有了。

她站起身来,捡起那封信,用麻木的手抖了抖大衣上的小土块,又拍了拍,擦了皮鞋,拿起头巾,抖了老半天,一直抖到头巾又成了白的。她把头巾系在头上,用头巾边儿擦了擦眉毛上的灰土,擦去嘴上和下巴上的血。她朝坟地大门口走去,不回头,不慢也不快。

三十四

回到喀山以后,柳德米拉就渐渐消痩,越来越像她学生时代照的相片。她上供应商店买东西,烧饭,生炉子,擦地板,洗衣服。她觉得秋天的日子太长,怎么也没办法打发过去。

从萨拉托夫回来的那一天,她就向家里人说了这次外出的情形,说了她想过自己有一些对不起家里人的地方,说了她去军医院的情形,又把包着儿子被炮弹片炸碎的血衣的小包打了开来。在她说这些事的时候,弗拉基米罗芙娜在重重地喘气,娜佳在哭,维克托的手发抖,他都无力端起桌上的茶杯。这时来看她的玛利亚的脸也变得煞白煞白的,嘴巴半张着,眼睛里也出现了痛苦的神情。只有柳德米拉平静地说着,两只发亮的蓝眼睛睁得大大的。她一向是个十分喜欢争论的人,现在她跟谁也不争论了。以前如果有人说怎样可以到车站去,柳德米拉就会又生气又着急地抬起杠来,说根本不是走那几条街,也不是坐那几路电车。有一次维克托问她:

“柳德米拉,每天夜里你是在和谁说话?”

她说:

“我不知道,也许是做梦。”

他再也没有问她,但是他对岳母说,柳德米拉几乎每夜都要打开箱子,把被子铺在角落里一张沙发上,心事重重地在小声说话。

“我有这样一种感觉:白天她跟我、跟娜佳、跟您在一起,似乎是在梦里;到夜里她说起话来就有了精神,就像战前一样,”他说,“我觉得她好像病了,渐渐变成另外一个人了。”

“我不知道,”亚历山德拉·弗拉基米罗芙娜说,“我们都在受苦。都一样,又各有不同。”

他们的谈话被敲门声打断。维克托站起身来。但柳德米拉在厨房里高声说:

“我去开。”

家里人不明白是怎么回事儿,但却发现,柳德米拉从萨拉托夫回来以后,每天都有好几次去翻信箱,看有没有信来。

每当有人来敲门,她都要急急忙忙去开门。

现在,又听到她急匆匆的、几乎是在跑的脚步声,维克托和岳母交换了一下眼色。他们听到柳德米拉很生气的声音:

“没有,今天什么也没有,你们别总来,两天前我已经给你们半公斤面包了。”

三十五

维克托罗夫中尉被召到团部,去见歼击机飞行团预备队的指挥官,萨卡布卢卡少校。值日参谋维里卡诺夫告诉他,团长乘飞机到驻在卡里宁区的空军集团军司令部去了,傍晚才能回来。维克托罗夫问为什么叫他来,维里卡诺夫挤挤眼睛,说,可能跟在食堂里酗酒、打架有关。维克托罗夫朝防雨布加棉被做成的帷幔里面望了望,听到有打字机在响。办公室主任沃尔康斯基看到维克托罗夫,就猜到他要问什么,便说:

“没有,中尉同志,没有信。”

文职女打字员列诺奇卡回头看了看中尉,又瞟了瞟面前的小镜子,这是已经牺牲的飞行员杰米道夫从一架击落的德国飞机上缴获了送给她的。她扶了扶军便帽,推了推压在正在打的表单上的小尺子,继续打起字来。

这位长脸的中尉竟也向办公室主任问这个问题,惹起她同样的苦恼。

维克托罗夫在回机场的路上,拐弯朝树林边走去。这个团退出战斗休整以来,已经有一个月了,这期间主要是补充物资,接收新的飞行员。一个月之前,维克托罗夫觉得这人迹罕至的北方是奇特的。那苍莽的森林,陡峭山冈间弯曲的急流,枯枝败叶和菌类的气息,林海不绝于耳的飒飒声,日日夜夜使他心神不安。

在飞行的时候,他常常觉得地上的气味进入了机舱。这里的森林、湖泊散发着战前他在书上读到的古代罗斯生活的气息。在这儿,森林和湖泊之间有古老的驿道,过去曾用这些笔直的树干建造房屋、教堂,制作船桅。灰狼曾在这里出没。阿廖努什卡[37]坐在河岸上哭泣(就是维克托罗夫现在去军人服务社食堂经过的河岸)。古老的生活已经沉寂,荡然无存了。他觉得,这逝去的古代是天真、单纯和年幼的,不仅是深闺的少女,就连白胡子的商人、助祭和长老们,都比这些精明世故的小伙子们,比萨卡布卢卡少校的空军集团军的飞行员们年轻一千岁;这些人是从高速汽车、自动炮、柴油机、电影和无线电的世界来到这森林里的。逝去的幼年时代的标志,就是奔流在花花绿绿的陡岸之间,在绿树与红蓝花团中的湍急而纤瘦的伏尔加河……

有许多尉官、军士和没有军衔的小伙子走在战争的道路上。他们抽定额配给的烟,用白调羹和铝盆子吃饭,在车厢里玩“捉傻瓜”,到城市里就吃冰棒,一面咳嗽,一面喝他们分到的一点酒。他们写信不能超过规定次数,他们对着战地电话喊叫,射击,有的开炮,有的放枪,有的驾驶T—34坦克,踩油门,呐喊……

土地在脚下咯吱咯吱直响,又有弹性,就像旧弹簧垫子—这是枯叶,上面的几层又轻又脆,尽管已枯死,但依然片片不同。下面则是多年前的枯叶,已经合成松软的褐色的一片—这是生命的灰烬,这生命曾经发出幼芽,在雷雨中飒飒作响,又闪着笑眼迎接雨后的阳光。几乎没有重量的腐烂树枝在脚下碎裂。静静的阳光射在林中土地上,被树叶划成斑斑点点。林中的空气浓稠,凝止不动;习惯了空中旋风的歼击机飞行员特别会感觉到这一点。晒热的潮湿树木散发着清新的木头气息。但是枯树朽枝的气味比活着的树木更强烈。在有枞树的地方,浓烈的松节油气味胜过一切味道。山杨甜得发腻,赤杨又苦又涩。森林过的是独立生活,跟其他世界不相干,维克托罗夫觉得自己好像进了一座房子,里面的一切和外面都不一样:气味不一样,射进来的光线不一样,声音在里面响起来也跟外面不一样。一个人在森林里,总觉得自己不大习惯,就像在生人面前。在底下透过高高的、厚厚的林中空气层朝上面张望,就像站在湖底;树叶飒飒响,那哧啦哧啦、往军便帽的帽徽上乱缠的蛛丝,就像挂在水面与湖底之间的水藻。似乎那些横冲直撞的大头苍蝇,无精打采的蚊子,像鸡一样在枝桠中间穿来穿去的松鸡,尽管长着翅膀,可是永远也飞不到森林上面去,就像鱼不会游到水面之上。喜鹊有时一下子飞到山杨树顶上,可是马上就又钻进枝丛里,就像鱼有时猛地一跃,白肚皮在阳光里闪一下,可是马上又钻进水里。在幽暗的林底,那挂满渐渐消散的蓝色、绿色露珠儿的青苔多么奇怪呀。

从静谧幽暗的林底,忽然来到明亮的林中空地,马上一切都不同了:暖烘烘的土地,晒热的刺柏的气息,流动的空气,耷拉着头的风铃草(那老大的风铃花像用紫金铸成的),还有长在黏黏的茎上的野石竹。心里顿时轻松起来;来到林中空地,就像不幸的生活中出现了幸福的一天。好像那些黄色的蝴蝶、蓝黑色的油亮的甲虫、在草丛里沙沙爬的蚂蚁,已经不是各顾自己,而是大家一起干着共同的活儿。缀满细小叶片的桦树枝轻拂着人脸。草蜢蹦来蹦去,把人当成树干,往人身上直撞,趴到人的腰带上,不慌不忙地蹲在上面,绿色的大腿鼓着劲儿,山羊脸上眼睛瞪得圆圆的。还有迟开的野莓花儿,晒热的纽扣和皮带扣环……大概,这林中空地上空从来不曾有“U—88”,不曾有“海因克尔”夜袭机飞过。

三十六

夜里他常常想起在斯大林格勒医院里过的那几个月。他不记得汗湿的衣裳、咸得使人恶心的水,不记得那使人受不了的恶浊气味。他觉得在军医院的那些日子是幸福的。现在,在这森林里,他听着树木的沙沙声,心想:“难道我听到了她的脚步声?”

难道有过这样的事?她抱着他,抚摩他的头发,她哭着,他吻她那湿湿的、咸咸的眼睛。

有时维克托罗夫想,他可以驾着“雅克”上斯大林格勒去,不过几个小时,可以在梁赞[38]加加油,然后上恩格斯城去,他有一个熟识的小伙子在那儿做值班主任。以后要枪毙就枪毙好啦。

他常常想起他在一本旧书上读到的一段故事:舍列梅捷夫[39]元帅的儿子们把十六岁的妹妹嫁给多尔戈卢基公爵。姑娘在婚前好像只见过他一回。姑娘的哥哥们给妹妹送了大量的陪嫁,送的银子装满三间屋子。结婚后第二天,彼得二世被杀。多尔戈卢基公爵是他的亲信,也被抓起来押往北方,关在一座木塔里。有人告诉新娘,说她可以不受这一婚姻约束,因为她跟丈夫总共生活了两天。但是她不听劝说,跟丈夫前去,住到偏僻的林区一座木屋里。一连十年,每天她都要到多尔戈卢基所在的木塔跟前去。有一天早晨,她看到木塔的小窗户开着,门也没有上锁。年轻的公爵夫人朝街上跑去,见到每一个人,不论是庄稼汉,还是士兵,她都跪下来哀求,问她的丈夫在哪儿。有人告诉她,她的丈夫被押到下诺夫哥罗德去了。她于是步行前往,一路上吃了很多苦。到了下诺夫哥罗德,她听说多尔戈卢基被分尸了。她决定进修道院,便前往基辅洞窟修道院。在要成为修女的那一天,她在第聂伯河岸边走来走去很久。但她不是俗念未灭,而是在那之前要把指头上的结婚戒指取下来,她却舍不得……她在河岸上徘徊了好几个钟头,后来,等到太阳就要落山了,她才把戒指从手指上摘下来,扔到河里,便朝修道院大门口走去。

这位空军中尉,这位保育院出身的斯大林格勒发电站机械车间钳工,老是想着多尔戈卢基公爵夫人的一生。他走在森林里,常常活灵活现地想象着:他已经死了,已被埋葬,那架被德国人击落的飞机,半截扎在土中,已经锈烂了,散架了,四周长满了青草,薇拉·沙波什尼科娃常常在这儿走来走去,有时停下来,走下岸坡,走到伏尔加河边,凝望河水……在两百年前,年轻的多尔戈卢基公爵夫人就曾在这里走过,有时走到林中空地,用手拨开缀满红色野果的树棵子,从野麻丛里穿过。他顿时觉得又难过,又痛苦,又失望,又甜蜜。

穿破军装的窄肩膀中尉在森林里走着。在难忘的时代里,有多少这样的人被遗忘了啊。

三十七

维克托罗夫还没有走到机场,就看出一定是发生了什么重要情况。许多加油车在夏天的田野上东奔西跑,机场维修营的机械师和发动机修理工围着停在掩护玻璃罩下的飞机忙活着。平时一声不响的电台发动机又清楚又起劲地嗒嗒响着。

“坏了。”维克托罗夫心里说着,加快了脚步。

马上就证实了他的猜测。腮上带着红色烫伤疤痕的上尉索洛马津一见到他就说:

“有命令,咱们要出发了。”

“上前方吗?”维克托罗夫问。

“不上前方,上哪儿去?”索洛马津说过这话,便朝村子走去。

看样子,他的情绪很坏,他和女房东的关系不同一般,现在大概是急急忙忙找她去了。

“索洛马津要分家啦:把房子给老娘们儿,老牛自己带着。”维克托罗夫旁边有一个熟悉的声音说。这是叶列玛中尉,从小路上走来,他常常跟维克托罗夫搭档飞行。

“叶列玛,调咱们上哪儿?”维克托罗夫问。

“可能是西北战线要反攻了。师长乘着‘艾尔—5’来了。我有一个驾驶‘道格拉斯’的朋友在空军军部里,可以问他。他什么都知道。”

“有什么好问的,不问也会知道。”

不仅团部的人和机场的飞行员们紧张起来,村子里也开始惶惶不安。团里最年轻的飞行员,黑眼睛、厚嘴唇的科罗尔少尉捧着浆洗熨好的衣服从街上走来,衣服上面还放着小甜饼和一包果干。

科罗尔的女房东是两个独身的老奶奶,常常给他做甜饼吃,大家都拿他开玩笑。每次他出来执行任务,两位老奶奶都要来机场,在半路上迎他。一个高高的,身子笔直,另一个是驼背,他走在她们中间,又生气,又难为情,像一个娇惯的孩子。飞行员们说,科罗尔跟一个惊叹号、一个问号走在一起。

飞行大队长万尼亚·马尔丁诺夫穿了军大衣从屋里走出来,一只手拎着提箱,另一只手拿着崭新的制帽,他怕弄皱了,没有放到提箱里。房东的红头发女儿没戴头巾,披着一头自己卷的卷发,在后面用那样的目光看着他,见到这种目光,再猜测她和他的关系,就是多余的了。

一个有点儿瘸腿的男孩子向维克托罗夫报告说,跟他住在一起的指导员戈卢普和中尉沃夫卡·斯科特诺伊已经带着东西走了。

维克托罗夫在几天以前才搬到这一家来。在这之前,他和戈卢普住在一个很坏的女人家里。那女人额头凸起,一双黄眼睛鼓鼓的。谁看到这双眼睛,都觉得不舒服。

为了不让他们住下去,她往屋子里放浓烟,有一天还偷偷地往他们的茶里撒灰。戈卢普劝维克托罗夫把这个女人的事写成报告递到团政委,但是维克托罗夫不愿写报告。

“让她害霍乱死掉。”戈卢普骂了一句,也就算了。

他们搬到另一家,觉得这一家简直是天堂。可是这天堂他们却不能久住了。

维克托罗夫很快也背着背包,拎着塞得满满的手提箱,从一座座足有二层楼高的灰色房屋前面走过。瘸腿的男孩子在旁边蹦跳着,拿维克托罗夫送给他的战利品手枪皮套朝母鸡瞄准,朝盘旋在森林上空的飞机瞄准。他从先前住的房子前面走过,透过模模糊糊的窗玻璃看到那个坏女人的一动不动的脸。每次她挑着两桶水从井上回来,停下来休息的时候,谁也不搭理她。她没有牛,也没有羊,屋顶下也没有燕子。戈卢普打听过她的情况,想弄清她的富农阶级根源,谁知她却出身贫苦家庭。妇女们说,她在丈夫死后好像是疯了。有一次在深秋天凉的时候,她跑到湖里,在水里呆了一昼夜。几个男子汉把她硬拖了上来。可是妇女们说,她在丈夫死之前甚至在出嫁之前,都不爱说话。

这会儿维克托罗夫走在这个林区村庄的街道上,再过几个钟头,他就要飞走,永远离开这儿了。这飒飒响的森林,村庄,麋鹿常常光临的菜园,还有这蕨草,金黄的松脂,杜鹃,他都看不到了。这些老头儿、小姑娘他也再见不到了。再也没有人给他讲当年怎样实行集体化的事,没有人给他讲狗熊抢夺妇女们的马林果篮子,还有小孩子用光脚板踩蛇头的故事了……再也见不到这个又奇特又平常的村庄,这村庄一切都跟森林有关,正如他出生和成长的工人村,一切都跟工厂有关。

然后飞机又要着陆,转眼间又要出现新的机场,出现农村或者工人村,出现另一些老年人、小姑娘,他们有他们的伤心事和开心事,有受伤而秃了鼻子的猫,又可以听到另外一些人叙述往事,叙述全面实行集体化的事,又会有另外一些好的或不好的房东。

美男子索洛马津到了新的环境里,又会在闲暇时间戴起漂亮的军帽,在大街上溜达,弹着吉他唱歌儿,叫姑娘们心醉。

团长萨卡布卢卡少校,一张古铜色的脸,白头顶刚刚剃过,胸前晃着五颗红旗勋章,倒换着两条弯弯的腿,向飞行员们宣读准备战斗的命令。他说,今晚在掩蔽所里过夜,出发次序在起飞前在机场上宣布。

然后他又说,指挥部命令不准离开机场的掩蔽所,违反军令,严惩不贷。

“不能在天上睡觉,所以要在起飞前好好睡一觉。”他解释说。

团政委别尔曼接着讲话。他很高傲,大家都不喜欢他,虽然对于飞行上的事他能说得头头是道。在处理飞行员穆欣那件事情之后,大家就特别讨厌他了。穆欣和漂亮的女电报员丽达沃伊诺娃谈恋爱。大家都很赞成他们这段恋情:一有空他们就相会,上河边散步,总是手挽着手走在一块儿。大家甚至都不取笑他们了,他们的关系已经非常明朗。

忽然有一种说法传了开来,这一说法出自丽达之口,是她对一位女友说的,又由女友传遍了全团:在一次外出散步的时候,穆欣强奸了她,还曾经拿手枪威胁她。

别尔曼听到这桩事以后,暴跳如雷,而且表现出极大的积极性。穆欣被法庭审问了十天,并且被判了死刑。

在执行枪决之前,空军集团军军委委员阿列克谢耶夫空军少将来到团里,开始调查穆欣的案情。丽达弄得将军非常难为情;她跪在他面前,恳求他相信,有关穆欣一案全是胡编乱造。

她对他说了事情的全部经过:她和穆欣躺在林中空地上,接了一会儿吻,后来她睡着了,穆欣要跟她开开玩笑,悄悄把手枪伸到她的两个膝盖中间,朝土里开了一枪。她惊醒了,叫了起来,于是穆欣又跟她接起吻来。她把这事儿对女友说了,可是从女友嘴里往外一传,事情就十分可怕了。在这件事情中,只有一点是真实的,那就是,她跟穆欣的爱情是极其纯真的。事情很顺利地解决了,判决取消了,穆欣调到了另一个团里。

从那时起,大家就更不喜欢别尔曼了。

有一次索洛马津在食堂里说,俄罗斯人是不会干这种事儿的。

有一个飞行员,好像是莫尔恰诺夫,说所有的民族中都会有坏人。

“就比如科罗尔,是犹太人,跟他搭档飞行就很好。在执行任务时知道有这样一个朋友在后面,心里就觉得踏实。”万尼亚·斯科特诺伊说。

“科罗尔算什么犹太人?”索洛马津说。“科罗尔是咱们的小伙子,我在飞行中对他比对自己都信得过。他在勒热夫把紧跟在我后面的一架德国飞机扫掉了。多亏波里亚·科罗尔,我有两次甩脱盯住我的该死的敌机。你知道,我打起仗来,也是不要命的。”

“这是怎么一回事儿,”维克托罗夫说,“如果一个犹太人很好,你就说,他不是犹太人。”

大家都笑起来。索洛马津说:

“好啦,穆欣被别尔曼安上枪毙罪名的时候,他才不觉得好笑呢。”

这时候科罗尔走进食堂,有一个飞行员很同情地问他说:

“我问你,波里亚,你是犹太人吗?”

科罗尔有点儿难为情,回答说:

“是的,是犹太人。”

“是真的吗?”

“完全是真的。”

“行过割礼吗?”

“滚你的蛋。”科罗尔回答说。大家又笑起来。

等飞行员们从机场回村子去,索洛马津和维克托罗夫走在一起。

“你要知道,”索洛马津说,“你不该说那话。我在肥皂厂工作的时候,找碴儿整人的人不少,一个个都是领导。我看够了那些家伙。”

“你啰唆什么,”维克托罗夫耸耸肩膀,“你以为我是他们那种人吗?”

别尔曼说,飞行员生活的新时期开始了,预备队的生活结束了。这些话不用他说大家也明白,但大家还是注意听着,听听他的话里有没有什么暗示,本团是不是还留在西北战线,是调到勒热夫一带,还是调到西线或南线?

别尔曼说:

“所以,战斗飞行员必须具备的第一点素质,是熟悉装备,熟悉得能够操纵自如;第二点,热爱自己的飞机,要像爱母亲、爱姐妹一样;第三,要勇敢,勇敢就是火热的心加冷静的头脑;第四,要有同志感情,这种感情是我们整个苏维埃生活培养出来的;第五,在战斗中要有献身精神!成功就在于编队飞行技能!要紧跟机长!一个好的飞行员,就是在地面上也要常常思考,分析、研究上一次战斗:‘嗯,这样会好些!嗯,不该那样!’”

飞行员们装做很感兴趣地看着政委,一面小声说着话儿。

“也许,是叫咱们护送运输机往列宁格勒送吃的东西?”索洛马津说。他有女朋友在列宁格勒。

“是不是去莫斯科方向?”莫尔恰诺夫说。他家里的人都在昆采沃。

“也许,要上斯大林格勒呢?”维克托罗夫说。

“算啦,不一定。”斯科特诺伊说。

他们团上哪儿,对他都无所谓,因为他家的人都在敌占区乌克兰。

“波里亚,你想上哪儿去?”索洛马津问道。“是不是上你们犹太人的首府别尔基切夫去?”

科罗尔那双黑黑的眼睛气得一下子完全黑沉下来,他很清楚地骂了一句娘。

“科罗尔少尉!”政委喝道。

“是,政委同志……”

“不要作声……”

其实科罗尔已经不作声了。

换做是萨卡布卢卡少校,他本来就是一个骂娘的行家,遇到飞行员当着领导的面骂娘,他不会管的。他每天早晨都对自己的通信员叫喊:“马秋金……你他妈的……”然后和和气气地说:“把手巾给我拿来。”

可是,团长知道政委那善于罗织罪名的作风,所以不敢马上把科罗尔放过。如果放过了,别尔曼会写报吿,说萨卡布卢卡在全体飞行员面前不维护政治领导人的威信。别尔曼已经向政治部写过报告,说萨卡布卢卡在预备队期间干私活儿,和团部里的人一起喝酒,和当地的女畜牧师叶尼娅·邦达列娃有不正当关系。

所以团长绕着弯子开始了。他很威严地嗄声喝道:

“科罗尔少尉,怎么站的?上前两步走!干吗那么吊儿郎当?”

接着他继续虚张声势。

“戈卢普指导员,您向政委汇报一下,为什么科罗尔破坏纪律。”

“少校同志,请允许我报告,他是和索洛马津争吵,至于为什么,我没听见。”

“索洛马津上尉!”

“有。少校同志。”

“您来汇报。不是向我!向政委汇报!”

“政委同志,让我汇报吗?”

“汇报吧。”别尔曼点了点头,对索洛马津连看也没看。他感觉出来,团长还是在坚持自己那一套。他知道,萨卡布卢卡不论在地上还是在空中,都特别狡猾。在空中,他能比谁都快地判断出敌人的目的和战术,以诡诈战胜敌人的诡诈。在地上,他懂得领导强中有弱,下属弱中有强。如有必要,他可以装装样子,装成一个憨大,听到蠢人说的很蠢的俏皮话也可以凑趣,可以哈哈大笑。他能把天不怕地不怕的飞行员们掌握在手心里。

在担任预备队期间,萨卡布卢卡对农业,主要是对饲养家畜家禽表现出很大的兴趣。他也搞起果品加工:用马林果制果子露酒,腌蘑菇,晒蘑菇。他做的饭菜出了名,有许多团长喜欢在空闲时间驾飞机上他这儿来,又吃又喝。但这位少校不认为这是白慷慨。

别尔曼知道这位少校还有一个特别难对付的特点,那就是:尽管他又精明,又谨慎,又狡猾,然而同时又几乎是个疯子,一旦硬干起来,连命都不顾。

“跟领导争论,简直就像……跟风作战。”他对别尔曼说。他会忽然不顾一切地干起有损切身利益的事,政委只有叹气。

有时两个人情绪都很好,他们就聊天,就你朝我、我朝你挤眼睛,互相拍肩膀或者拍肚子。

“嘿,我们的政委真是个精明汉子。”萨卡布卢卡说。

“嘿,我们的英雄少校真棒。”别尔曼说。

萨卡布卢卡不喜欢政委那种假殷勤,不喜欢他把每一句不小心的话都要写进报告的那股积极劲儿。他嘲笑别尔曼见了漂亮姑娘就眼馋,嘲笑他喜欢吃炖鸡而不喜欢喝酒。别尔曼对别人的生活条件漠不关心,却善于为自己创造舒适的生活条件,他就更加不满。他佩服别尔曼的聪明,佩服他为了事业敢于同领导冲突,佩服他的勇气—有时候似乎别尔曼自己也不知道,他会很轻易地丢掉性命。

这会儿,这两个人在准备率领空军集团军奔赴前线的时候,彼此侧眼看着,听着索洛马津上尉陈述:

“政委同志,我应该直说,科罗尔破坏纪律,这是我的过错。我嘲笑他,他忍着忍着,后来就忍不住了。”

“您对他说什么来着,您向政委说说。”萨卡布卢卡打断他的话。

“刚才同志们都在猜,咱们团上哪儿去,上哪条战线去,我就对科罗尔说:你想不想上你们的首府别尔基切夫去?”

飞行员们都看着别尔曼。

“我不懂,上什么首府?”

别尔曼说过这话,忽然明白了。

他有点儿难为情,大家都感觉到了这一点,而团长特别吃惊的是,这事儿竟出在一个像剃刀刃一样锐利的人身上。可是,使人惊讶的事儿还有呢。

“这是怎么搞的?”别尔曼说。“科罗尔,大家都知道,索洛马津是新鲁扎区多罗霍沃村人。如果您对索洛马津说,他想上多罗霍沃村作战,他就该因为这样打您的嘴巴吗?这真是很奇怪的乡土道德标准,跟共青团员称号很不相称。”

他总是说一些耸人听闻的话。大家都明白,索洛马津是想逗科罗尔生气,科罗尔果然生气了,可是别尔曼却满有把握地向飞行员们解释,是科罗尔没有摆脱民族主义偏见,他的行为是藐视各民族友谊,说科罗尔不应当忘记,法西斯正是利用民族主义偏见为所欲为。

别尔曼说的话本身是正确的。他这会儿用激动的语调说的思想,来自革命,来自民主。但这会儿别尔曼的着力点,不是他为了思想,而是让思想为他,为他今天颇有问题的用心服务。

“同志们,你们看,”政委说,“哪儿思想不正确,哪儿就没有纪律。今天科罗尔的行动就说明了这个问题。”

现在政委把科罗尔的行动同政治问题联系起来,萨卡布卢卡自然是不能干预的。萨卡布卢卡知道,任何一个战斗指挥员任何时候都不敢干预政工机关的行动。

“同志们,就是这么回事儿。”别尔曼说。为了加重自己谈话的分量,他停顿了一会儿,才又说下去:“出现这种不成体统的事,责任在犯错误的本人,但我这个团政委也有责任,因为我没能帮助飞行员科罗尔清除思想上的落后的、丑恶的、民族主义的东西。问题比我一开始设想的要严重些,所以我现在还不能处罚科罗尔的违纪行为。但是我要把教育科罗尔少尉的任务承担下来。”

大家动了动,坐舒服些,都觉得事情过去了。科罗尔看了看别尔曼,在他的目光中有一种异样的神情,别尔曼一看到这种神情,皱了皱眉头,抖了抖肩膀,并且转过脸去。晚上,索洛马津对维克托罗夫说:

“你瞧,廖尼亚,他们总是这样,一个个多么深奥呀。这事儿要是出在你或者万尼亚·斯科特诺伊身上,肯定被别尔曼送到惩戒分队去了。”

三十八

晚上,飞行员们在掩蔽所里都没有睡,躺在铺上抽烟,谈话。斯科特诺伊吃晚饭时喝了不少告别酒,这会儿不住地在哼歌儿:

飞机打着螺旋飞翔,

吼叫着飞向大地胸膛,

不要哭,好姑娘,不要悲伤,

从此永远、永远把我遗忘。

维里卡诺夫还是憋不住,说漏了嘴,于是大家都知道了,本团要转移到斯大林格勒附近。

一轮明月升到森林上空,树木中间出现了晃晃不定的光斑。离机场两公里的那个村子,好像是躲在灰堆里,黑糊糊的,一点声息也没有。坐在掩蔽所门口的一些飞行员观赏着这美妙的、布满地标的世界。维克托罗夫望着“雅克”机翼和机尾投出的淡淡的月光阴影,也跟着斯科特诺伊哼唱起来:

用手把骨架抬起,

从飞机底下掏出我们,

一架架飞机盘旋上升,

送我们最后一程。

躺在铺上的飞行员们在聊天。黑暗中看不清说话的人,但是听声音就知道是谁,所以不用呼唤名字,只凭着声音回答或提问。

“杰米多夫自己请求任务,他不飞就受不了。”

“你还记得吧,在勒热夫的时候,我们掩护轰炸机,八架飞机一齐朝他扑过去,他从容应战,坚持了十七分钟。”

“是呀,拿一架歼击机换一架‘容克’,是划算事儿。”

“他一面飞,一面唱。我每天都能记住他唱的一两支歌儿。他也唱过维尔津斯基的歌。”

“这个莫斯科人有两下子!”

“是啊,他在飞行中肯照顾别人。总是照顾落后的同志。”

“你还没有真正了解他呢。”

“我了解他。在飞行中最能看清同飞的搭档。他的一切都向我表露出来了。”

斯科特诺伊唱完一支歌,大家都静下来,等着他再唱另一支。可是他没有再唱。

斯科特诺伊说了一句流行于各个机场的谚语,说的是飞行员的生命短得好比小孩子的衣裳。

大家谈起德国人。

“认出德国佬也不难,一下子就可以判断出来,哪一架厉害,哪一架顽强,哪一架想捉呆瓜,从后面咬尾巴,哪一架专找½在后面的。”

“总的说,他们配合不怎么紧密。”

“可不能这样说。”

“德国佬见到受伤的就拿牙紧紧咬住,见到厉害的就逃跑。”

“要是一架对一架,就算是双头的,我肯定能把它打掉!”

“你别见怪,要是依着我,因为你打掉一架‘容克’,才不会授给你什么勋章。”

“空中撞击—是俄罗斯人的天性。”

“我有什么好见怪的,你又不能把我的勋章取消。”

“是啊,关于撞击我早就有一种想法……我还可以拿螺旋桨来撞。”

“追赶中的撞击,才真够劲儿!把它赶着朝地上冲击,叫它撞个粉碎!”

“听说,团长要用‘道格拉斯’把母牛和母鸡都带上,是吗?”

“反正这些东西全都宰啦,用盐腌起来了。”

有一个人拉长声音用若有所思的语调说:

“现在我要是带着姑娘上豪华俱乐部去,还难为情呢。已经不习惯啦。”

“不过,索洛马津不会难为情。”

“你是不是羡慕呀,廖尼亚?”

“羡慕这种事,不是羡慕这个对象。”

“我明白。绝对相信。”

然后大家回忆起勒热夫的战斗,那是转为预备队之前的最后一次战斗。那一次七架歼击机跟敌人的一大群“容克”轰炸机和护航战斗机相遇。大家似乎都是各说各的,但又像是都在说同一件事。

“起初有森林做背景,看不见它们;等它们飞高,马上就看见了。分三个高度飞行。我立刻认出是‘容克—87’:腿儿跷着,鼻子是黄的。于是我坐得舒服些:好,来吧!”

“我起初还以为那是高射炮炮弹爆炸呢。”

“阳光对这种事儿显然是有利的。我从阳光方向朝德国佬冲去。我是左侧僚机[40]。一下子被甩开三十公尺。跟上去不难,飞机很听话。我朝一架‘容克’开了火,把它打得冒了烟,可是这时候有一架敌人的歼击机,长长的,像一条黄鼻子狗鱼,转弯来打我,可是晚了。我看到它朝我开火了,一道青青的印子。”

“我看见我射出的青印子一直抵到那架飞机黑色的机翼。”

“你好得意呀!”

“我小时候放风筝,我爸打我。我进工厂以后,工余时间常常跑七公里上航空俱乐部去,累得要命,可是一次表演都不放过。”

“喂,你听我说说。德国佬一下子把我打着了火:油箱、输油管都烧着了。里面着了起来。到处是浓烟!另外又打中了我的护罩,把眼镜打碎,护罩上的玻璃乱飞,流起了眼泪。你猜我怎样—我一下子钻到它底下,又一把把眼镜扯下来!索洛马津掩护了我。我着了火,可是不害怕,没工夫害怕!我仍旧坐着,身上没着火,靴子烧坏了,飞机烧坏了。”

“眼看着咱们要被打掉了。我又转了两个圈儿,有一架敌机要同我较量。我没理会,赶去打另外的敌机,解救被追击的同志。”

“嗬,当时我已经带了不少窟窿,被打得像一只老山鹑一样啦。”

“我朝那个德国佬冲了十二次,把他打得冒烟了!我看到他的头乱摇,可见已经不行啦!在二十五公尺的距离我开了炮,把他打了下去。”

“是的,总的应该说,德国佬不喜欢在同一水平线上作战,总是尽可能飞到垂直线上。”

“怎么能这样说?”

“怎么样?”

“这事儿谁不知道?就连农村姑娘都知道:德国佬这是躲避急转弯攻击。”

“唉,真该把勒热夫掩护好一点儿,那儿的人真好呀。”

后来安静下来,有一个人说:

“明天天一亮咱们就要走啦,只有杰米多夫一个人留在这儿啦。”

“好啦,同志们,不管怎样,我要上储蓄所去,要到村子里去一趟。”

“去告别吗?”

深夜,周围的河流、田野、森林,一切是那样宁静,那样美好,似乎世界上不可能有仇敌、叛卖、衰老,只有幸福的爱情。云彩涌向明月,明月在灰色云雾中飘动,青烟遮住大地。在这样的夜里,有多少人在掩蔽所里过夜。在森林边上,在木栅栏旁,闪动着一方方白色的头巾,不时响起清脆的笑声。树木在寂静中轻轻抖着,想必是在梦中受了惊吓。河水有时轻轻低语一会儿,接着又无声无息地流起来。

恋人们最痛苦的时刻来到了。这是离别的时刻,是决定命运的时刻:有的今天在哭,明天就会被忘记;有的被死神永远分开;有的会得到命运的青睐,还会相见。

但是,早晨到了。发动机隆隆响起来,飞机扇起的平刮的风把惊慌的青草压倒在地上,成千上万的露珠儿在阳光下颤动……一架架战斗机飞向蓝天,把小炮和机枪带上天空,在天空盘旋,等待伙伴们编队飞行……

昨天夜里似乎还是无边无垠的林区,如今渐渐离开,在蓝天里渐渐沉没……

看得见一个个小盒子似的房屋、小方块似的菜园,房屋和菜园向后滑去,在机翼下渐渐消失……那青草萋萋的小路看不见了,杰米多夫的坟也看不见了……走吧!森林也哆嗦了几下,在机翼下滑走了。

“你好,薇拉!”维克托罗夫默念着。

三十九

早上五点钟,值日囚犯把一个个囚犯唤醒。外面夜色依然黑沉。棚屋里有通宵不熄的电灯照耀着。这样的灯在监狱、铁路枢纽站和城里医院的急诊室都有。

成千上万的人一面咳嗽、吐痰,一面穿棉裤,缠脚布,在腰侧、脊梁、脖子上搔痒。

睡在上铺的人穿好衣服下来,有时脚会碰到坐在下铺的人头上,下铺的人也不骂娘,而是一声不响地把头朝旁边一歪,用手把上面的脚推开。

夜里唤醒这么多人,裹脚布闪来闪去,人头、脊背不住地晃动,烟气腾腾,电灯光明晃晃的,这一切显得极不正常。几百平方公里的原始森林在寒夜里静静地沉睡,可是劳改营里已经到处是人,到处在活动,到处是烟雾、灯光。

上半夜一直在下雪,雪堆把棚屋的门堵住,把通往矿井的大路埋住……

矿井的汽笛慢慢叫起来,也许,密林深处的狼也跟着那粗壮而凄厉的汽笛声嚎起来了。警犬在劳改营的田野上嘶哑地吠着,拖拉机隆隆响着清扫通往矿区大楼的道路,押队兵彼此呼唤着……

雪花飘到探照灯光中,晶亮晶亮的,显得十分柔和悦目。在广阔的劳改营田野上,在乱糟糟的狗吠声伴奏下,开始点名了。押队兵那伤了风的嗓门儿又嘶哑又激昂……巨大的人流朝矿井涌去,一片咯吱咯吱的皮鞋声和毡靴声。守望塔瞪着巨大的独眼,盯着周围的一切。

笛声依然呼啸着,有远的,也有近的,这是北方的混合乐队。这声音回荡在寒冷的克拉斯诺亚尔斯克土地上,在科米自治共和国上空,在马加尔,在苏维埃港,在科雷马边区的雪野上,在楚科奇冻土地带,在摩尔曼斯克北部和北哈萨克的劳改营里……

伴随着汽笛声,伴随着铁撬棍敲击铁轨的声音,人们前去采掘索里卡姆斯克的钾、里杰罗夫和巴尔喀什的铜、科雷马的镍和铅、库兹涅茨和萨哈林的煤炭,人们前去铺设穿过北冰洋岸永久冻土带的铁路、科雷马的无接缝线路,前去砍伐西伯利亚、北乌拉尔、摩尔曼斯克和阿尔罕格尔边区的森林。

在原始林区各处,边远建设劳改营大队新的一天,就在这风雪交加的夜晚时刻开始了。

四 十

夜里,囚犯阿巴尔丘克觉得一阵烦恼。不是那种习惯了的、劳改营里常有的愁思绵绵的烦恼,而是火烧火燎的烦恼,就像疟疾发作那样,使人要叫起来,要从床铺上跳下来,用拳头打自己的两鬓,捶自己的脑壳儿。

早晨,囚犯们急急忙忙而又很不情愿地准备去上工的时候,在阿巴尔丘克的邻铺,煤气工长,原内战时期的旅长,长腿涅乌莫里莫夫问道:

“夜里你翻来翻去干吗?梦见老娘们儿啦?还嗷嗷地叫。”

“你就知道老娘们儿。”阿巴尔丘克回答说。

“我以为你在梦里哭呢。”另一个邻铺上的人说。他叫莫尼泽,有点儿傻头傻脑,原是青年共产国际的委员。“我本来想把你唤醒呢。”

阿巴尔丘克在营里的另一个好友、医士阿布拉姆·鲁宾什么也没有发现,在他们朝又冷又黑的门外走的时候,他说:

“你可知道,夜里我梦见了尼古拉伊凡诺维奇·布哈林,好像他来到我们红色教授学院,他很快活,精神抖擞,延琴曼的理论引起了激烈的争论。”

阿巴尔丘克来到工具库干活儿。他的助手巴尔哈多夫是为了抢劫杀死一家六口人的罪犯,现在正用做框子剩下来的雪松木片生炉子。阿巴尔丘克在整理木箱里的工具。他觉得,那些寒光闪闪的锋利的锉刀与旋刀,唤起了他在夜里产生的感觉。

这一天和以往的日子没有什么不同。会计一大早就送来技术科批准的各边远劳改营分部的申请报告。应该把材料和工具拣出来,装进箱子,编制相应的清单。有些东西是不成套的,需要编制特别交接单。

巴尔哈多夫像往常一样,什么活儿也不干,没办法叫他干。他来到工具库里,只是解决吃的问题。今天一大早他就在锅子里煮土豆白菜汤。担任第一大队通信员的原哈尔科夫药学院拉丁语教授跑到巴尔哈多夫跟前,哆哆嗦嗦地伸出红红的手指头,往桌上撒了一把肮脏的小米。不知为什么事,他给巴尔哈多夫这样的报酬。

下午,阿巴尔丘克被叫到财务处,因为在统计表上有些数字不对头。财务处副处长训斥他,还说要报告上级。他听到这些吓唬,心里觉得憋得慌。助手不帮忙,他一个人干不了那么多事情,可是他又不敢告巴尔哈多夫的状。他很劳累,很怕丢掉管理仓库的活儿,又要到矿上去,或者去伐木。他已经白了头,没有多大力气了……大概他就是因为这样才烦恼—他的一生已经消失在西伯利亚的冰层下。

等他从财务处回来,巴尔哈多夫在睡觉,头底下枕着毡靴,看样子,是其他犯人给他送来的;他的脑袋旁边放着已经空了的锅子,腮上粘着他捞来的小米。

阿巴尔丘克知道,巴尔哈多夫有时把仓库里的工具弄出去,很可能,这毡靴就是仓库里的东西换来的。有一天,阿巴尔丘克发现少了三把锉刀,就说:

“在卫国战争时期偷窃紧缺的钢材,怎么不知道羞耻……”

巴尔哈多夫回答说:

“你这狗虱子,闭嘴!要不然你等着瞧!”

阿巴尔丘克不敢直接唤醒他,就叮叮当当地整理锯条,又咳嗽,又把小锤掉在地上。巴尔哈多夫醒了,带着心安理得和不满意的神气注视着他。后来巴尔哈多夫低声说:

“昨天一列军车里下来的一个小伙子说,有些劳改营比湖泊地区的劳改营还不如呢。犯人都带着镣铐,半个脑袋剃得光光的。没有姓名,只有编号缝在胸前,缝在膝盖上,背后还缝着犯人标记。”

“胡扯。”阿巴尔丘克说。

巴尔哈多夫带着向往的神气说:

“应当把所有的政治坏分子弄到那儿去,首先应当把你这个家伙弄去,免得把我弄醒。”

“对不起,巴尔哈多夫先生,我打搅您了。”阿巴尔丘克说。

他非常怕巴尔哈多夫,但有时候也压抑不住心头的怒火。

在换班时间,满身黑炭粉的涅乌莫里莫夫来到仓库里。

“竞赛怎么样?”阿巴尔丘克问道。“大家都参加了吗?”

“竞赛是展开啦。打仗需要煤炭嘛,这大家都知道。今天把标语贴到了文教处:突击劳动,支援祖国。”

阿巴尔丘克叹了一口气,说:

“你要知道,应该写一部描述劳改营里的烦恼的著作。有时烦恼使人感到沉重,有时烦恼来势凶猛,有时烦恼使人气闷,叫人喘不上气来。可是还有一种烦恼很特别,既不沉重,也不凶猛,也不使人气闷,而是撕心裂腑,就像深水怪物要把海洋搅翻。”

涅乌莫里莫夫苦笑了一下,不过他露出来的不是雪亮的白牙,他的牙齿已经坏了,和煤炭一样颜色了。

巴尔哈多夫走到他们跟前。阿巴尔丘克回头看了看,说:

“你老是这样悄没声地走路,冷不丁来到我跟前,我都哆嗦起来啦。”

巴尔哈多夫是个不爱笑的人,带着很操心的神气说:

“我要上粮食仓库去一下,你没意见吧?”

他走后,阿巴尔丘克对自己的朋友说:

“夜里我想起前妻生的儿子。他大概已经上前方去了。”

他凑到涅乌莫里莫夫耳朵跟前,说:

“我希望我的儿子成为一个很好的共产党员。我在想,我会见到他的,我要对他说:记住,你爸爸的遭遇是很偶然的,算不了什么,党的事业是神圣的事业!是合乎时代最高要求的!”

“他姓你的姓吗?”

“不,”阿巴尔丘克回答说,“我原来认为,他可能会长成―个市侩。”

昨天傍晚和夜里,他想过柳德米拉,很希望见到她。他翻阅残破的莫斯科的报纸,说不定能看到“中尉托里亚·阿巴尔丘克”呢,那样他就会清楚,儿子想姓他的姓了。

他生平第一次希望有人怜惜他。他想象着,他怎样走到儿子跟前,激动得连气都喘不上来,拿手指着自己的喉咙,表示说不出话来。托里亚会把他抱住,他会把头放到儿子胸前,哭起来,毫不难为情,尽情地哭,哭。他们会站上很久,儿子比他高一个头……

儿子一直想着父亲。他找到父亲的同志们,向他们打听当年父亲参加革命斗争的情形。托里亚会说:“爸爸,爸爸,你的头发完全白啦,你的脖子多么细,皱纹好多啊……你一直斗争了这么多年,你进行的是伟大而孤单的斗争呀。”

在审讯的时候,给他吃了三天咸菜,却不给他水喝。还要打他。

他明白,主要的不是要他招供破坏行为和间谍行为,也不是要他诬陷别人。关键是要他怀疑他终生为之奋斗的事业的正确性。在审讯的时候,他觉得自己好像落到了匪徒手里,只要能见到审讯科长,这些审讯他的匪徒就会被抓起来。

但是,过了一些时间,他看出来,问题不仅仅在于几个暴徒。

他了解了羁押犯人的军用列车和轮船统舱,各有各的规矩。他看到过,一些刑事犯不仅输掉别人的东西,而且输掉别人的性命。他见过下流无耻,见过卑鄙的出卖。他见过刑事犯的野蛮行为,那是疯狂的、血腥的、极其残酷的。他见过得势的正统派与不得势的正统派之间可怕的派系斗争。

他说:“抓人是不会冤枉的。”他认为,只有极小的一部分人,包括他在内,是抓错了的,其余的都是罪有应得,是正义的利剑惩罚革命的敌人。

他见过阿谀奉承、背信弃义、唯唯诺诺、残酷无情……他把这些东西叫做资本主义遗毒,他认为这些东西只有那些遗老遗少、白军军官、富农分子、资产阶级民族主义者身上才有。

他的信仰是不可动摇的,他对党是无限忠诚的。

涅乌莫里莫夫就要离开仓库的时候,忽然说:

“哦,我忘啦,刚才有一个人问你来着。”

“哪儿来的人?”

“昨天军车上下来的。正在分配他们工作。有一个人问起你。我说:‘凑巧我知道,我跟他铺挨铺已经睡了有三年多。’他对我说了他的姓名,可是我一下子就忘啦。”

“他是什么样子?”阿巴尔丘克问。

“噢,模样儿够寒碜的,鬓角上还有一道伤疤。”

“啊哈!”阿巴尔丘克叫起来。“莫不是马加尔呀?”

“就是,就是。”

“这是我的老同志,我的老师,是他发展我入党的。他问什么来着?他说了一些什么?”

“问的是一般的话,问你判了几年。我说:报了五年,批下来是十年。现在咳嗽起来,有可能提前获释。”

阿巴尔丘克没有听涅乌莫里莫夫说话,而是一遍又一遍地叫着老同志的名字:

“马加尔,马加尔……他有一段时期在全俄肃反委员会工作。这是一个很特别的人,真的,很特别。他对同志什么都舍得,冬天可以脱下自己的大衣,可以把最后一块面包送给同志。又聪明,又有学问。是地道的无产阶级出身,是刻赤[41]渔民的儿子。”

他回头看了看,俯身对涅乌莫里莫夫说:

“你记得,咱们说过,劳改营里的共产党员应该建立起组织,帮助党。阿布拉姆·鲁宾曾经问:‘让谁当书记呢?’现在有了,就是他。”

“可我还是推选你,”涅乌莫里莫夫说,“我不了解他。你要是想找他,刚才有十辆汽车装着人到各分部去了,大概他也去了。”

“没什么,能找到他的,啊,马加尔,马加尔。就是说,他问我了吗?”

涅乌莫里莫夫说:

“我差点儿忘了我是来干什么的。给我一张白纸。瞧我的记性真差。”

“要写信吗?”

“不是,要向谢苗·布琼尼写申请书,要求上前线去。”

“不会让你去的。”

“布琼尼还记得我呢。”

“不会让政治犯上军队里去。咱们的煤矿可以多出一些煤炭,战士们也会因此感谢咱们,也可以说尽到自己的力量啦。”

“我还是希望上军队里去。”

“这种事儿布琼尼也没办法。我还给斯大林写过信呢。”

“布琼尼也没办法?你真是开玩笑!还是你舍不得一张纸?我的限额用纸已经用完了,文教处又不给我。要不然我不会向你要。”

“好吧,我给你一张。”阿巴尔丘克说。

他还有几张纸,是未经批准存下的。文教处发纸是有数的,而且以后还必须说明纸是怎么用了的。晚上,棚屋里的情形一如往常。原近卫重骑兵团军官东古索夫老头子眨巴着眼睛,没完没了地说着传奇故事。犯人们仔细听着,搔着痒痒,带着赞赏的神气晃着脑袋。

东古索夫随心所欲地编造着荒诞离奇的故事,把一些熟悉的女舞蹈家、阿拉伯的劳伦斯,把三个火枪手和凡尔纳“鹦鹉螺”号潜艇的事都编了进去。

“等一等,等一等,”有一个听众说,“她究竟怎样跨过波斯国境的?你昨天说,她被奸细毒死啦。”

东古索夫停了一会儿,和善地看了看挑毛病的人,就又很起劲地说起来:

“娜金其实并没有死。一位西藏医生往她那半张开的嘴里滴了几滴高山仙草熬出来的药水,又把她救活了。到第二天早晨她就能起来,不用别人搀扶,可以在屋里走动了。她的体力渐渐恢复了。”

大家听了他的解释,都很满意。

“明白啦……再说下去吧。”大家说。

在角落里,一些人在哈哈大笑,在听蠢头蠢脑的老工长、德国人加秀琴柯拉长了声音说下流的顺口溜。

有的顺口溜十分好笑,听众一直笑得没了劲儿。有一个害疝气的莫斯科记者和作家,是一个善良、聪明而腼腆的人,正慢慢地嚼着烤干的白面包,这是妻子寄来的,他昨天才收到。看样子,他吃着又香又脆的干面包,想起了过去的日子—他的眼里含着泪水。

涅乌莫里莫夫正在跟一个坦克手争论。坦克手进劳改营,是因为出于卑劣的动机,杀人行凶。他为了给大家解闷,嘲笑骑兵,涅乌莫里莫夫气得脸发了白,大声对他说:

“你可知道,在一九二〇年,我们凭马刀干过一些什么样的事!”

“我知道,你们拿马刀杀过偷来的母鸡。一辆坦克就可以把你们整个骑兵第一集团军打退。你们的国内战争无法跟卫国战争相比。”

年轻的小贼科尔卡·乌加罗夫缠着阿布拉姆·鲁宾,要拿一双脱了掌的破运动鞋换他的皮鞋。

鲁宾觉得要倒霉,神经紧张地打着呵欠,环视着周围的人,寻求支持。

“你这小气鬼,小心点儿,”像一只灵活的黄眼野猫似的科尔卡说,“该死的东西,你小心点儿,别惹我发火。”

后来科尔卡说:

“你为什么不准我病假?”

“你很健康嘛,我不能同意。”

“你同意不同意?”

“科尔卡,我向你保证,我很希望准你请假,但是我不能。”

“你同意不同意?”

“你要知道我的难处。难道你以为,我能批……”

“好啦。算啦。”

“别急,别急嘛,你要了解我的难处。”

“我了解。现在该你了解我了。”

什捷金格是完全俄罗斯化了的瑞典人,大家都说他是真正的间谍。他正在文教处发给他的一块硬纸板上作画,他的眼睛离开画一小会儿,看了看科尔卡,看了看鲁宾,摇了摇头,又转过头去作画。画名叫《原始森林妈妈》。什捷金格不怕刑事犯人,不知道为什么,刑事犯们都不敢碰他。

等科尔卡走开以后,什捷金格对鲁宾说:

“阿布拉姆,你的做法很不聪明。”

白俄罗斯人科纳舍维奇也不怕刑事犯。他在进劳改营之前,在远东做航空技师,在太平洋舰队里获得重量级拳击冠军称号。刑事犯们都很敬重他,但是他从来不曾为受刑事犯欺负的人打抱不平。

阿巴尔丘克慢慢地在两层架铺中间的狭窄通道上走着,又烦恼起来。百米长的棚屋的那一头沉没在马合烟[42]的烟气中。每次他都觉得,等走到棚屋的尽头,会看到一点新的东西,可是走到尽头,一切都还是老样子,还是那装着洗脸木槽的过道,刑事犯在木槽下面洗裹脚布,还是挂在石灰墙上的拖把,还是那油漆木桶,铺上还是露着刨花的褥垫,还是不高不低的嗡嗡说话声,还是一张张枯瘦的、一样颜色的囚犯脸。大多数囚犯坐在铺上等待就寝信号,谈女人,谈菜汤,谈切面包的人弄鬼,谈自己给斯大林的信和给苏联最高检察院的申诉书的遭遇,谈新的采煤和运煤定额,谈今天的寒冷和明天的寒冷。

阿巴尔丘克慢慢走着,听着谈话的片断。他觉得,这种一模一样、没完没了的谈话要在押送站、军车上、劳改营的棚屋里,在成千上万的人中间持续很多年,年轻的都要谈女人,年老的都要谈吃的。等到老头子如饥似渴地谈起女人,年轻小伙子谈起不受限制的好吃的东西,那就特别糟了。

阿巴尔丘克从加秀琴柯坐的铺旁边经过时,加快了脚步。一个老人,他的妻子已经有儿孙们唤“妈妈”、“奶奶”了。他受到这样的待遇,这待遇太可怕了。

就寝号快点儿响起来吧,快点儿躺到铺上,拿棉袄蒙住头,什么也看不见,什么也听不见。

阿巴尔丘克朝门口看了看—也许马加尔来了呢。阿巴尔丘克要求求大组长,让他们睡在一起,他们每夜都可以长谈,推心置腹地谈,因为他们是两个共产党员,是老师和学生。

棚屋的头面人物,采煤队队长佩列克列斯特、巴尔哈多夫、棚屋大组长萨罗科夫在一个铺上举行小小的宴会。佩列克列斯特的狗腿子、原来管计划的日里亚波夫担任跑堂,将一块手巾铺在凳子上,摆放奶油、鲱鱼、点心—这都是佩列克列斯特队里的人孝敬的贡品。

阿巴尔丘克从头面人物的铺边走过,觉得自己的心紧张得停止了跳动:说不定他们会喊他,叫他吃一点儿呢。他真想吃点儿好吃的呀。巴尔哈多夫真没有良心!他在仓库里想干什么就干什么,阿巴尔丘克也知道他偷钉子,偷了三把锉刀,但是在值班时什么也没说……现在他完全可以招呼一声:“喂,主管,来跟我们坐一会儿吧。”阿巴尔丘克很瞧不起自己,觉得自己不仅想吃,而且还有一种感情在作祟,这是一种很卑微、很下贱的囚犯感情:很想在厉害角色的圈子里坐一会儿,随便跟佩列克列斯特谈一谈,佩列克列斯特可是偌大的劳改营听到名字都发抖的人物。

阿巴尔丘克想起了自己—下贱。马上又想到巴尔哈多夫—下贱。

没人喊他,却喊了涅乌莫里莫夫。于是这位骑兵旅长、获得两颗红旗勋章的英雄龇着褐色的牙齿,笑嘻嘻地朝他们的床铺走去。这个笑嘻嘻地去参加几个贼的宴会的人,二十年前曾经率领几个骑兵团为实现世界共产主义战斗过……

他今天干吗对涅乌莫里莫夫谈起托里亚,谈自己的心事?

不过他也为共产主义战斗过,他也在库兹巴斯工地上,在自己的办公室里向斯大林做过汇报。当他低着头,装做若无其事的样子从蒙了肮脏的绣花手巾的凳子旁边走过时,也曾经希望他们喊他。

阿巴尔丘克走到莫尼泽的床铺边,莫尼泽一面补袜子,一面说:

“今天佩列克列斯特对我说:‘你要小心,我要拿拳头敲你的脑袋,我要汇报你,还算便宜你,你是最坏的叛徒。’”

坐在邻铺上的鲁宾说:

“这还不是最糟的呢。”

“是的,是的,”阿巴尔丘克说,“你看到他们把旅长喊过去,旅长那股高兴劲儿吗?”

“他们没喊你,你不痛快了吧?”鲁宾说。

阿巴尔丘克恼羞成怒,说:

“你看看自己的灵魂吧,别忙着说我。”

鲁宾像鸡那样半闭起眼睛,说:

“我吗?我连不痛快也不敢。我是最低下的一类,没人理睬。我和科尔卡的谈话,你没听见吗?”

“不是那么回事儿,不是。”阿巴尔丘克把手一挥,站了起来,又顺着床铺之间的通道朝那张凳子走去,又听到那没完没了的谈话。

“甜菜猪肉汤天天有,不光是过节。”

“她的乳房才滑溜呢,你恐怕都不信。”

“哥儿们,我不讲究,有羊肉泡饭就行啦,干吗要你们的沙拉凉拌菜……”

阿巴尔丘克又回到莫尼泽的铺前,坐下来,听别人谈话。

鲁宾说:

“我不明白他的意思,为什么他说:‘你可以做眼线。’他说的是告密者,比如说,向侦缉人员暗地汇报。”

莫尼泽一面继续补袜子,一面说:

“去他娘的吧,告密—是顶下贱的事。”

“怎么会告密呢?”阿巴尔丘克说。“你是共产党员嘛。”

“他这共产党员跟你一样,”莫尼泽说,“是过去的话了。”

“我不是过去的,”阿巴尔丘克说,“你也不是过去的。”

鲁宾又使他恼了,因为说出了应有的怀疑,应有的怀疑往往比不应有的怀疑更刺激人,更叫人受不了。

“这不是党员不党员的问题。一天喝三次玉米泔水汤,大家都喝够了。我也恨死了这种汤。你这一点我赞成。不赞成的是你夜里和白天两副面孔。我和科尔卡的谈话,你听见了吗?”

“头朝下,腿朝上啦!”莫尼泽说过这话,就笑了起来。可能因为再没什么好笑的了。

“你怎么,以为我只有动物本能啦?”阿巴尔丘克问道。他觉得自己简直憋不住要把鲁宾揍一顿。

他又霍地站起来,在屋里走起来。

当然,他吃够了玉米糊。多少天以来,他都在猜想着十月革命节的伙食:会不会有肉丁炒白菜、通心粉汤、杂烩?

当然,很多事情要取决于侦缉人员。好一点儿的差事,比如管澡堂,切面包,是不容易弄到手的。他可以在实验室工作,穿白大褂子,干自在活儿,跟刑事犯们不发生关系,他也可以在计划处工作,可以领导煤矿……可是鲁宾不对。鲁宾想侮辱他,鲁宾泄他的气,在他身上寻找下意识地悄悄出现的东西。鲁宾就喜欢钻空子。

阿巴尔丘克一辈子痛恨圆滑,痛恨两面派和社会异己分子。

他过去的精神力量、他的信心,在于他能使用法庭的权力。他怀疑妻子,就和她离了婚。他不相信她能够把儿子教育成一个坚定的战士,就不让儿子用他的名字做父称。他常常痛斥摇摆不定的人,瞧不起爱发牢骚的人和意志薄弱、信念不坚定的人。他曾经把库兹巴斯工地上一些想家、不安心的莫斯科工程技术人员交付法庭。他把四十名离开工地跑回农村的工人判了刑。他还和钻营市侩的父亲断绝了关系。

做一个坚定不移的人,是幸福的。每一次把人送交法庭,他都可以证实自己的精神强大,证实自己是典范,证实自己的纯洁。他从中得到乐趣,增强信心。他从不躲避党的动员号召。他自愿不领取党员最高月工资。他天天穿着很平常的制服和靴子去上班,参加人民委员部委员会议,上戏院。有时党派他去休养,他就穿这套服装在雅尔塔的海边散步。他希望一切都像斯大林。

他失去使用法庭的权力,就失去自己的本色。鲁宾感觉到这一点。几乎每天他都要在话里指出他的软弱、他的怯懦,指出悄悄进入劳改犯心中的一些可怜的愿望。

前天他就说:

“巴尔哈多夫拿仓库里的钢材把有的坏家伙喂饱啦,可是我们的大英雄连一声也不哼。就连小鸡也想活呢。”

当阿巴尔丘克准备责备别人的时候,他感到自己也会被责备,就会动摇起来,觉得灰心丧气,便失去自己的本色。

阿巴尔丘克在一个床铺旁边站下来。老公爵多尔戈卢基正在这里和经济学院的年轻教授斯捷潘诺夫说话。斯捷潘诺夫在劳改营里一向表现很高傲,营队领导人走进棚屋巡视,他都不肯站起来,常常公开发表反政府观点。他感到自豪的是,他和许多政治犯不同,他被关押是因为这样的事情:他写了一篇题为《列宁和斯大林的国家》的文章,让学生传阅。不知是读到这篇文章的第三个还是第四个学生把他告发了。

多尔戈卢基是从瑞典回到苏联的。去瑞典之前,他在巴黎住了很久。他想念祖国,就回来了。回国一个星期之后,就被捕了。他在劳改营里常常祷告,结识了一些教徒,并且写一些内容神秘难懂的诗。

这会儿他就在给斯捷潘诺夫念诗。

阿巴尔丘克将肩膀靠在上铺与下铺之间钉的十字形木板上,听他念诗。多尔戈卢基半闭着眼睛在念,干裂的嘴唇哆嗦着。他那不高的声音也哆嗦着,并带有干裂声。

是我自己选定了降生年月、时间、国家、民族和地点,

为的是经受所有的苦难,

经受良心、水和火的洗礼。

我向下落去,掉进了深渊黑洞,

落到比什么都低的地方,在臭脓、粪堆里,

启示录中的野兽—

我信心不改!

我相信最高权柄的公正,

是它解放了古老的自然力量,

我在烧焦的俄罗斯腹地,我要说:

你这样决断,是对的!

要想变得钻石般坚硬,

必须炼透整个的人生。

如果熔铁炉里的柴炭不够,

上帝呀,请用我的血肉!

他念完之后依然半闭着眼睛坐在那里,嘴唇依然无声地翕动着。

“胡诌,”斯捷潘诺夫说,“颓废派!”

多尔戈卢基用没有血色的苍白的手朝四周指了指。

“你们瞧,车尔尼雪夫斯基和赫尔岑把俄罗斯人引导到哪儿来了。你们可记得,恰达耶夫在第三封哲学通信里写的是什么?”

斯捷潘诺夫用教师教导学生的口吻说:

“您的神秘的愚昧,就跟有些人要建立这种劳改营一样,我都十分讨厌。不论是您,不论是他们,都忘记了俄罗斯还有一条路,一条最自然的道路:民主和自由的道路。”

阿巴尔丘克和斯捷潘诺夫争论过不止一次了,可是现在他不想插嘴,不想把斯捷潘诺夫说成敌人,说成持不同政见者。他走到角落里,有些洗礼派教徒正在这儿祷告,他听了听他们的嘟哝。

这时候响起大组长萨罗科夫的响亮的声音:

“起立!”

大家一齐站起来,上司走进了棚屋。阿巴尔丘克侧眼看着虚弱不堪的多尔戈卢基那苍白的长脸,看着他两手紧贴裤缝站在那里,嘴唇还在嘟哝着,大概还在念他的诗。斯捷潘诺夫坐在旁边。他像往常一样,目无领导,不服从本棚屋明明白白的内部规章。

“搜查啦,搜查啦。”囚犯们小声说。

但是没有搜查。两名头戴红蓝制帽的年轻看押兵从床铺中间走过,一面打量着囚犯们。其中一名士兵走到斯捷潘诺夫跟前,说:

“教授,你坐着呀,你是怕把什么东西冻坏呀。”

斯捷潘诺夫转过他那翘鼻子的宽宽的脸,用鹦鹉似的响亮的声音很不自然地回答说:

“长官先生,请您对我称‘您’,我是政治犯。”

夜里,棚屋里发生了严重事件:鲁宾被杀死了。

凶手趁被害者睡觉的时候,拿一个大钉子插到他的耳朵里,然后用力一砸,把钉子楔进脑子里。有五个人,包括阿巴尔丘克在内,被侦缉人员传去。看样子,侦缉人员感兴趣的是钉子的来历。这种钉子才进库不久,生产部门还不曾领用。

在洗脸的时候,巴尔哈多夫在木槽边和阿巴尔丘克站在一起。巴尔哈多夫朝他转过湿漉漉的脸,一面舔着嘴上往下流的水滴,一面小声说:“该死的东西,你记住,你要是去告发,我一点也没有事儿。可是今天夜里我就收拾你,狠狠收拾你,叫全营都知道厉害。”

他用毛巾把脸擦干以后,拿平静的眼睛看着阿巴尔丘克的眼睛,看到眼睛里的神气正是他希望看到的,便握了握阿巴尔丘克的手。

在食堂里,阿巴尔丘克把自己的一钵子玉米糊送给了涅乌莫里莫夫。涅乌莫里莫夫哆嗦着嘴唇说:

“真是野兽。把我们的阿布拉姆害死啦!多么好的一个人呀!”

他说着,把阿巴尔丘克的玉米糊端到自己面前。

阿巴尔丘克一声不响地站起来,离开饭桌。

在走出食堂的时候,大家纷纷让路,佩列克列斯特往食堂里来了。他在跨门槛的时候,把身子弯了弯,因为劳改营的门都没有他的个头儿高。

“今天是我的生日。来我这儿玩吧。咱们喝两杯。”

多么可怕!有几十个人听到了夜里的凶杀,看见一个人走到鲁宾的床铺边。

如果有人一下子爬起来,把全屋的人喊起来,会怎么样呢?几百个强壮的男子汉团结起来,两分钟就会把凶手制服,会救活一个同伴。但是谁也不抬头,谁也不叫喊。杀一个人,就像杀一头羊一样。大家都躺着,装做睡着了,拿棉袄蒙住头,尽可能不咳嗽,尽可能不去听受害者在昏迷中挣扎。

多么低三下四,多么驯顺啊!可是他当时也没有睡着,也没有作声,拿棉袄把头蒙住。他很明白,驯顺不是微不足道的小事,驯顺来自经验,来自对劳改营规律的了解。如果大家都起来,把凶手制住,带刀的人还是比不带刀的人厉害。全屋的力量是一时的力量,而刀永远是刀。

阿巴尔丘克想着面临的审讯:侦缉人员一定会要他的口供的,他在棚屋里一夜没有睡,早晨也没有洗脸,准备着挨折腾,他不朝矿井方向去,不去上棚屋的厕所,怕有人突然扑过来拿麻袋蒙住他的头。

是的,不错,夜里他是看见一个人朝鲁宾走去。他听见鲁宾在哼哧,听见鲁宾死前手和脚在床铺上乱扑乱蹬。

侦缉人员米沙宁大尉把阿巴尔丘克叫到办公室里,把门关上,说道:“您坐吧,犯人。”

他先提了几个简单的问题,对这样的问题政治犯一般都能很快、很准确地回答。

然后他抬起疲惫的眼睛,看着阿巴尔丘克,早就知道这个有经验的囚犯很怕同棚屋的人报复,永远不会说出钉子是怎样落到凶手的手里的,所以对阿巴尔丘克打量了一阵子。

阿巴尔丘克也看着他,打量着大尉那年轻的脸,他的头发和眉毛,鼻子上的雀斑,心想,这位大尉比他的儿子至多大两三岁。

大尉提了一个问题,正是为这个问题把阿巴尔丘克传来的,在这之前已经有三名被审讯者不肯回答这个问题了。

阿巴尔丘克好一阵子没有作声。

“你怎么,聋了吗?”

阿巴尔丘克还是没有作声。

他多么希望这位侦缉人员说:“你听着,阿巴尔丘克同志,你是共产党员。今天你在劳改营里,明天咱们就要在一个组织里共同缴纳党费。你帮帮我的忙吧,同志要帮助同志,党员要帮助党员。”即使这不是真心实意的,只是采取一种例行的侦讯手段。

可是米沙宁大尉却说:

“您睡着了还是怎的?那我马上来把您唤醒。”

但是阿巴尔丘克却不用唤就醒了。

他用嗄哑的声音说:

“钉子是巴尔哈多夫从库里偷出来的。不光是钉子,他还从仓库里偷了三把锉刀。依我看,杀人的是科尔卡·乌加罗夫。我知道,巴尔哈多夫把钉子给了他,他有好几次说要杀死鲁宾。昨天他还说的,因为鲁宾没有准许他请病假。”

然后他接过递给他的一支纸烟,说:

“侦缉员同志,我认为,向您说出这件事,是我这个党员的责任。鲁宾同志是一位老党员。”

米沙宁借火给他把烟点着了,就一声不响地很快地记起来。然后他用温和的口吻说:

“犯人,您要知道,任何关于党员的话您都不应该说。您不能称呼同志。对于您来说,我是首长。”

“对不起,首长。”阿巴尔丘克说。

米沙宁对他说:

“几天之内,我还在进行调查,您不会出什么事。过几天以后再说。可以把您调到别的劳改营里去。”

“不必,首长,我不怕。”阿巴尔丘克说。他朝仓库里走去,知道巴尔哈多夫什么也不会问他。巴尔哈多夫会一个劲儿地盯着他,时刻注意他的动作、眼神、咳嗽,从中弄清情况。

他终于恢复了自己的本色,他十分高兴。

他又能行使法庭的权力了。他一想到鲁宾,就觉得遗憾,昨天他竟没有对他说出自己的不祥的预感。

三天过去,马加尔还是没有来。阿巴尔丘克上矿务局去打听他,阿巴尔丘克熟悉的几个文书在任何一本册子里都找不到马加尔的姓名。

晚上,在阿巴尔丘克知道命运已经把他们分开的时候,满身白雪的卫生员特留菲列夫来到棚屋里,一面捋眉毛上的冰凌,一面对阿巴尔丘克说:

“告诉您,我们卫生所来了一名犯人,他请您上他那里去。”

特留菲列夫又说:

“最好现在我带您去。您向大组长请个假。要不然我们这些犯人可不讲什么情理,马上就会找你的麻烦,等到把你收拾了,你再讲理由就晚啦。”

四十一

卫生员领着阿巴尔丘克来到卫生所的走廊。这里有一种特别的、和棚屋里不同的坏气味。他们在昏暗中朝前走着,看到堆在一起的许多担架,还有捆成许多捆的旧棉衣,看样子,是等着送去消毒的。

马加尔躺在隔离室里。这是一间木板墙小屋,里面有两张铁床几乎挨在一起。进隔离室的一般都是害了传染病或者快要死的病人。细细的床腿像是铁丝做的,却没有压弯的迹象,从来没有胖子睡这样的床。

“别坐这儿,别坐这儿,右边坐。”

响起一个声音。那声音极其熟悉,阿巴尔丘克一下子觉得似乎没有白发,没有被关押,又是自己终生依靠、终生为之奋斗的一切了。

他打量着马加尔的脸,满怀激动、一字一顿地说:

“你好,你好,你好……”

马加尔怕控制不住自己的激动心情,故意很平淡地说:

“坐吧,就坐在我对面的床上。”

他看到阿巴尔丘克打量旁边床铺的目光,又说:

“你不会打扰他的,他已经不怕打扰了。”

阿巴尔丘克俯下身去,为的是看清老同志的脸,接着又回头看了看盖着的死者,问:

“他死了很久了吗?”

“两个多钟头以前死的,卫生员暂时还没有动他,等医生来。这样好些,要不然换一个活的来,咱们说话就不方便了。”

“这话对。”阿巴尔丘克说。他没有问他非常想问的一些问题:怎么样,你是受布勃诺夫[43]牵连,还是因为索科尔尼科夫[44]案件?判了你几年?你在弗拉基米尔或者苏兹达利的政治犯隔离室呆过吗?主持审讯的是特别机构还是军事委员会?你自己签字了吗?

他回头看了看盖着的尸体,问:

“他是什么人?怎样死的?”

“死于劳改营,是个富农分子。他老是在唤一个娜斯佳的名字,一直想离开这儿上什么地方去……”

阿巴尔丘克在昏暗中渐渐看清了马加尔的脸。他几乎认不出他了,变化太厉害了,竟成了一个垂死的老头子!

他感到自己的后背碰到了死者那弯着的僵硬的胳膊,觉得马加尔在看着自己,心里就想:“恐怕他也在想,‘简直认不出他了。’”

可是马加尔却说:

“先前他一个劲儿嘟哝,好像是‘霍……霍……霉……’,现在我才明白,他这是要喝水。茶杯就在旁边,真应该满足他最后的要求。”

“瞧,死人还是妨碍咱们了。”

“那当然了。”马加尔说。阿巴尔丘克听到了他熟悉的激动的语调,马加尔开始谈严肃的话题时往往是这样。“因为我们谈他,实际上是谈自己。”

“不,不是!”阿巴尔丘克抓住马加尔滚烫的手,紧紧握着,又抱住他的肩膀,不出声地哭起来,哭得浑身打哆嗦,憋得喘不过气来。

“谢谢你,”他含混不清地说,“谢谢你,谢谢,好同志,好朋友。”

他们两个人都哼哧哼哧喘着气,有一阵子没有说话。他们呼出的气汇合到一起,阿巴尔丘克觉得,汇合到一起的不仅是他们呼出的气。

马加尔首先开口说:

“听我说,听我说,朋友,这是我最后一次这样称呼你了。”

“别这样说,你会活下去的!”阿巴尔丘克说。

马加尔在床上坐起来。

“我非常不希望这样说,但是应该说。你也听着,”他对死者说,“这和你,和你的娜斯佳有关系。这是我最后一项革命任务,我一定要完成!阿巴尔丘克同志,你是特殊气质的人。而且我们当年相遇也是在特殊的时候,我觉得,那是我们的最好的时候。现在我要告诉你……我们错了。我们的错误造成了这样的结果,瞧……我们应该请求他原谅。让我抽一支烟。后悔已经晚啦。任何后悔都不能补偿过失。这是我要对你说的。这是第一点。再说第二点。我们不懂得自由。我们压制了自由。马克思也不珍视自由。自由是根本,是目的,是基础的基础。没有自由就没有无产阶级革命。以上我说了两点,再说第三点。我们在劳改营和原始林里经受苦难,可是我们的信仰比什么都坚强。这不是坚强,是懦弱,是保全自身。在铁丝网外面,要保全自身,就得多变,要不然就要死亡,就要进劳改营。共产党人制造偶像,戴肩章,穿制服,信奉民族主义,压制工人阶级,将来必然还要像黑色百人团[45]那样……在这里,在劳改营里,要保全自身,就不能改变:如果不想死的话,在劳改营里几十年都别改变……这是一个铜板的两面……”

“别说啦!”阿巴尔丘克叫起来,把握紧的拳头凑到马加尔的面前。“你受不住啦!你垮啦!你说的话全是胡说八道。”

“如果那样,倒是好;但我不是胡说。我是又一次召唤你!就像二十年前那样!如果我们不能作为革命者活下去,那我们就死,像这样活着比什么都不如。”

“够啦,别说了!”

“请原谅我。我懂。我像一个老妓女,为失去的贞节痛哭。不过我要告诉你:记住吧!好朋友,请原谅我……”

“原谅?你我真应该像这个死人一样,早几个钟头死去,活不到这次见面……”阿巴尔丘克已经站在门口,又说:“我还要上你这儿来……我要给你修复头脑,现在我要做你的老师了。”

第二天早晨,卫生员特留菲列夫在劳改营的大院子里碰到阿巴尔丘克。特留菲列夫用爬犁拉着一桶牛奶,牛奶桶用绳子捆在上面。奇怪的是,在这北极圈里,他的脸上竟出了汗。

“你的朋友不能喝牛奶了,”他说,“昨天夜里他上吊了。”

报告消息叫人吃一惊,是挺快活的事,所以这位卫生员带着友好而得意的神气望着阿巴尔丘克。

“有遗书吗?”阿巴尔丘克问,并且倒吸了一口凉气。他觉得,马加尔一定会有遗书的,说昨天的事,是他一时心血来潮。

“干吗要写遗书?不论写什么,都要落到侦缉人员手里。”

这一夜,是阿巴尔丘克一生中最难熬的一夜。他一动不动地躺着,咬紧牙齿,睁大了眼睛,望着墙上捻死臭虫留下的一个个黑点。

他想起他不准姓他的姓的儿子,呼唤起儿子:

“现在我就剩下你了,只有你是我的希望。瞧,我的朋友和老师马加尔想杀死我的理智、我的志向,结果他自杀了。托里亚呀,托里亚,我在人世上就只有你一个了。你能看到我吗,能听到我的话吗?将来你能不能知道,你的父亲在这天夜里没有屈从,没有动摇?”

周围的人都在睡觉,睡得很熟,声音很大、很不好听,空气很重浊、很窒闷,有的打鼾,有的嘟哝,有的在梦里叫,有的咬牙,有的拉长声音呻吟和呼喊。

阿巴尔丘克忽然在铺上欠起身来,他觉得好像旁边有个阴影闪了一下。

四十二

一九四二年夏末,克莱斯特[46]的高加索集团军群占领了迈科普附近苏联最早开发的一个油田。德国军队进入挪威的北角和希腊的克里特、芬兰北部和拉芒什海峡[47]沿岸。热带作战的大元帅艾尔文·隆美尔驻扎在离亚历山大八十公里的地方。在厄尔布鲁士山[48]顶上,山地军竖起了带有纳粹党徽的旗帜。曼施坦因得到命令,要把巨炮和新式火箭炮推向布尔什维克的堡垒列宁格勒。本来持观望态度的墨索里尼已经在制订进攻开罗的计划,练习骑坐阿拉伯马。寒带作战的季特尔驻扎在任何一个欧洲侵略者都没有到过的北纬地带。巴黎、维也纳、布拉格、布鲁塞尔都成了德国的省城。

实现国家社会主义党最残酷计划的时刻来到了,这一计划的目的在于消灭人,消灭人的生命和自由。法西斯党的头目们四处散布谎言,说是斗争的紧张迫使他们不能不如此残酷。事实正好相反,危险会使他们清醒。如果对自己的力量缺乏信心,他们就会有所收敛。

等到法西斯完全相信已经取得最后胜利的那一天,全世界就会倒在血泊里。如果世界上不再有反法西斯的武装,刽子手们也不会就此收手的。因为法西斯的主要敌人就是人。

一九四二年秋天,帝国政府通过了一系列惨无人道的法律。特别是一九四二年九月十二日,在国家社会主义党的军事胜利到达顶峰之时,居住在欧洲的犹太人被取消法律保护权,由秘密警察管制。

法西斯党的领导和希特勒本人决意完全消灭犹太民族。

四十三

索菲亚·奥西波芙娜·列文顿有时想想过去的事:苏黎世大学五年的生活,巴黎和意大利的夏季旅游,音乐学院的音乐会,中亚山区的考察,从事了三十二年的医务工作,她喜欢的菜肴,跟自己的生活密切相关的朋友们(有艰难的日子,也有愉快的日子),习惯了的电话铃声,习惯了的话语,打纸牌,留在她莫斯科住处的东西。

她也常常想起在斯大林格勒的那几个月,想起亚历山德拉·弗拉基米罗芙娜、叶尼娅、谢廖沙、薇拉、玛露霞。越是和她亲近的人,如今离她越远。

有一天快到黄昏时候,军用货车停在离基辅不远的一个枢纽站的备用线上,她在锁上的车厢里捉自己领口上的虱子,旁边有两个上了年纪的妇女很流利地小声说着犹太话。这时候她特别清楚地意识到她,索菲亚·奥西波芙娜·列文顿,少校军医,面临的真实处境。

这些人的主要变化,是对自己的特殊气质和个性的感觉减弱了,对命运的感觉增强了。

“我,我,我究竟是什么人?实实在在是什么人?”索菲亚·奥西波芙娜想道。“是那个小小的、流鼻涕的、又怕爸爸又怕奶奶的小姑娘,还是那个发胖、脾气暴躁、戴领章的军医,还是这样一个长虱子的脏老婆子?”

幸福的希望没有了,但是出现了许许多多想法:把虱子消灭……凑到门缝儿上,呼吸呼吸新鲜空气……解解小便……洗洗脚,哪怕洗一只脚……还有,浑身都想喝水。

刚把她推进车厢里,她觉得昏暗的车厢里漆黑一团,她朝四下里看了看,听见低低的笑声。

“是疯子在这儿笑吗?”她问。

“不是,”一个男子的声音回答说,“在这儿说笑话呢。”

有一个人伤心地说:

“又一个犹太女人到我们这遭殃的车上来啦。”

索菲亚·奥西波芙娜站在车门口,眯着眼睛,为的是适应黑暗,回答别人的问话。她马上陷入一种不习惯的氛围中:这儿除了哭声、呻吟和臭气,还有从童年时代就已遗忘了的语言、口音……

索菲亚想往里走走,但是走不过去。她在黑暗中摸到一条穿短裤的细细的腿,就说:

“对不起,好孩子,我把你碰疼了吗?”

但是这孩子没有回答她。她在黑暗中说:

“大娘,您是不是让您的孩子挪挪地方?我总不能一直站着呀。”

在角落里有个男子用歇斯底里的演员般的声音说:

“应该早点儿打个电报来,那样就可以给您安排一个带浴室的房间。”

索菲亚清清楚楚地说:

“浑蛋!”

有一个女人,她的脸在昏暗中已经露出来了,她说:

“靠着我坐吧,这儿地方有的是。”

索菲亚·奥西波芙娜感觉出她的手指头在轻轻地、快速地抖动。

这是她从小就熟悉的世界,是犹太小镇的世界;她感觉出这个世界的一切变化有多么大。

这节车厢里有合作社的工人,有无线电技工,有师范学院的女学生,有工会学校的教师,有罐头厂的工程师,有畜牧工作者,还有一位担任兽医的姑娘。以前小镇上没有这样一些职业。但是,要知道索菲娥·奥西波芙娜没有变,她依然是当年又怕爸爸又怕奶奶的那个样子。也许,这新的世界也依然未变?可是,不管怎么说,还不是一样。犹太人的小镇,不论是新是旧,反正是朝坡下滚去,将滑向无底深渊。

她听到有一个年轻的女子声音说:

“现在的德国人都是野蛮人,他们都不知道海涅是什么人。”

另一个角落里,一个男子声音用嘲笑的口吻说:

“结果这些野蛮人把咱们当牲口装进火车里。咱们知道海涅又有什么用?”

大家向索菲亚·奥西波芙娜打听前线的情况,因为她说的全是不好的消息,有人就对她说,她所知道的消息是不可靠的;于是她明白了,在这牲口车厢里有自己的战略,这战略的根据是强烈的生存愿望。

“难道您不知道,希特勒收到了最后通牒,要他立即释放所有犹太人?”

是的,是的,当然是这样。等到任人宰割的痛苦和不祥预感变为剧烈的恐怖的时候,人往往求助于毫无根据的乐观,麻醉自己。

对索菲亚·奥西波芙娜的兴趣很快就过去了。她也和大家一样,成了一个不知道被弄到哪里去、不知道被弄去干什么的同路人。谁也不问她的名字和父称,谁也记不住她的姓。

索菲亚·奥西波芙娜甚至感到奇怪:走倒退的道路,从人回到肮脏、可怜、失去名字和自由的牲口,只需要几天工夫:而从动物到人的路,却走了几百万年。

她很惊讶,人类遭受这样大的灾难,却依然时时刻刻操心生活琐事,依然因为一些小事彼此闹意见。

有一个上了年纪的女人小声对她说:

“医生,你瞧瞧那位阔太太,她坐在门缝儿跟前,就好像只有她的小孩子需要呼吸新鲜空气。太太是上咸湖去呢。”

夜里火车停过两次,大家很留心地听着警备队咯吱咯吱的脚步声,听着杂乱不清的俄语和德语。

在夜晚的俄罗斯小站上听到歌德的语言,显得非常可怕,但是听到德国警备队中有俄罗斯人说起俄语,更使人感到毛骨悚然。

天快亮的时候,索菲亚·奥西波芙娜和大家一样饿得难受,并且幻想能喝到一口水。她的幻想极其微小,极不大胆,她想象着有一个压得凹凸不平的罐头盒子,里面还剩一点儿热乎乎的水汁儿。她用又快又短促的动作搔了搔痒,就像狗抓弄跳蚤那样。

现在索菲亚·奥西波芙娜觉得似乎懂得了生活与生存的区别。生活已经结束了,完了,可是生存依然继续着。虽然这种生存是可怜的、毫无意义的,但是一想到横死,心里就感到十分可怕。……

下起雨来,有些雨滴从装了铁栏的小窗户里飞进来。索菲亚·奥西波芙娜从自己的衣襟上撕下一条布边儿,身子朝车厢壁挪动了下,凑到有一条不大的缝隙的地方,把布条塞到缝隙外面,等着布条浸透雨水。然后她把布条抽回来,嚼起凉丝丝、湿漉漉的布条。这时在靠近车厢壁的地方以及车厢角落里,有些人也开始撕布条了,索菲亚·奥西波芙娜感到很得意:这取雨水、喝雨水的方法是她发明的。

夜里索菲亚·奥西波芙娜碰着的那个男孩子坐在离她不远的地方,看着一些人把布条塞到车门底下的缝儿里。她在朦胧的光线中看到了他那瘦小的脸和尖尖的鼻子。看样子,他有六七岁。索菲亚·奥西波芙娜心想,她来到车厢里这么长时间,还没有人跟这孩子说过话,他也一动不动地坐着,没有和别人说过一句话。她把湿布条递给他,说:

“好孩子,给你。”

他没有作声。

“接着吧,接着吧。”她说。

他犹犹豫豫地伸出手来。

“你叫什么名字?”她问。

他小声回答说:

“达维德。”

坐在旁边的一个叫穆霞·鲍里索芙娜的女人说,达维德从莫斯科来看他的外婆,打起仗来,他不能回到妈妈身边了。外婆死在隔离区里,达维德的姨娘列维卡·布赫曼就跟有病的丈夫在这个车厢里,甚至不让这孩子坐在她身边。

到傍晚时候,索菲亚·奥西波芙娜已经听说不少事情,听到不少争论,她自己也说,也参加争论。她对交谈者说:

“犹太兄弟姐妹们[49],我来跟你们说说。”

许多人盼望着快点儿到地方下车,以为这是把他们送到集中营去,到集中营里每个人都可以根据自己的专长干活儿,有病的人可以住伤残病房。大家几乎一刻不停地谈论着这些。可是心里依然在暗暗地害怕,在不出声地哭号。

索菲亚·奥西波芙娜从别人说的事情中了解到,人身上不仅仅是人性的东西。有人对她说,有一个女人把瘫痪的姐姐放到木盆里,在冬天的夜里拖到外面去,把姐姐冻死了。有人告诉她,有些母亲杀死了自己的孩子,在这个车厢里,就有这样一个女人。还有人说,有些人就像老鼠一样,成年累月地住在下水管道里,吃的是脏东西,只要能活着,吃什么苦都行。

犹太人在法西斯的统治下生活是可怕的,犹太人既不是圣人,也不是坏蛋,他们是人。当索菲亚·奥西波芙娜望着小小的达维德的时候,她心中产生的对人的怜悯感情特别强烈。小达维德照常不说话,一动不动地坐着。有时从口袋里掏出一个揉破了的火柴盒,对着火柴盒看一阵子,然后又藏进口袋。索菲亚·奥西波芙娜有几个昼夜一点没有睡,她不想睡。这一夜她也是坐在又黑又臭的车厢里没有睡。她忽然想道:“这会儿叶尼娅·沙波什尼科娃在哪儿呀?”她听着人们的呓语和叫声,心想,这些睡着了的、发狂的脑袋里这会儿一定活灵活现地发生了言语难以表达的可怕情景。如果一个人还能活在世上,将来希望知道过去的事的话,怎样才能保留、才能记下这些情景?……

“兹拉塔!我的兹拉塔!”有一个男子带着哭声喊道。

四十四

……在瑙姆·罗森贝格的四十岁的头脑里正在进行着他习惯了的统计工作。他一面在路上走,一面算:前天110,加上昨天61,再加上前五天的612 ,共计783 ……可惜他没有计算男人、女人、儿童的分类数字。女人烧起来比较容易。这个有经验的劳工在焚尸的时候,总是把出灰多的干瘦的老头子跟女人的尸体摆在一起。现在马上就要命令他们离开大路,拐个弯往前走了—一年前对那些人就是这样下命令的。他们现在把那些人的尸体挖出来,再用绳子拴着钩子从坑里往外拖。有经验的劳工可以从一个一个的坟包判断出坟坑里有多少尸体:五十,一百,二百,六百,一千……这里的监督[50]艾尔弗要他们管尸体叫“具”,一百具,二百具,可是罗森贝格管他们叫人,被杀的人,被杀的小孩子,被杀的老头子。他是在心里这样叫,要不然监督就要送他一粒枪子儿。可是他嘴里老是在嘟哝:“被杀的人呀,你从坑里出来吧……小家伙呀,别扯着妈妈啦,你们就要在一块儿,想分开也分不开啦……”要是问他:“你在那儿嘟哝什么?”他就说:“我什么也没有说,您也许觉得我在说话。”他还是在嘟哝,他在作斗争,这是他小小的斗争……前天有一个坑,里面8个人。监督叫起来:“真不像话,20人的劳工队只焚化8具!”他说得很对。可是如果一个小村子里只有两户犹太人,又有什么办法呢?命令总归是命令:要把所有的坟都挖开,把所有尸体都烧掉……现在他们离了大路,在草地上走了,终于,在碧绿的林中草地上第115次出现了灰色的土包—坟墓。8 个人挖坟,4个人伐橡树,锯成人体长的木条,两个人用斧头把木条劈开,两个人把引火的干木板和汽油桶从大路上往这里抬,4个人清理架火堆的地方,挖出灰的沟—还要看好风从哪边来。

一会儿尸臭的气味就压倒林中的腐叶味。警备队又笑,又骂,捂起鼻子,监督直吐唾沫,躲到林边去。劳工们扔下铁锹,拿起钩子,拿破布把嘴和鼻子蒙住……“您好,老大爷,您又要见见阳光啦;您可真够重的……啊,一个妈妈带三个孩子,两个男孩,一个已经上学了,一个女孩有三岁吧,有佝偻病……没关系,现在不怕了……别拿手扯住妈妈不放,孩子,你妈妈哪儿也不去啦……”监督在林边大声问:“有几具?”罗森贝格回答:“19……”底下是在心里说的:“19个被杀的人。”大家都在骂:花了半天工夫,才这么一点点儿。可是上星期挖开一个坟,一下子就是200个妇女,而且全是年轻的。当揭去上面一层土的时候,坟里冒出灰色的热气,警备队笑着说:“这些娘儿们还热乎呢!”他们往一道道通风的土沟上放一层干木柴,然后放橡木条,橡木条会变成很耐烧的火炭,然后放被杀的女人,再放一层木条,然后又放被杀的男人,再放一层木条,然后又放分不清男女的尸体碎块,然后浇汽油,然后往中间放一枚燃烧弹。然后监督下口令,焚化工们齐声歌唱,警备队员们脸上早就浮现出笑容。大火堆烧起来。然后把骨灰送进坑里。一切又静下来了。原来就很安静,现在又安静了。接着,他们被带进树林,在绿草地上没有看到坟包。监督命令他们挖坑:四米长,二米宽。他们都懂了,他们已经完成任务:89个村子,加18个小镇,加4个工人村,加2个区中心,加3个国营农场,其中两个是谷物农场,一个是奶牛场,总共116个居民点,这些劳工已经挖完116个坟……会算账的罗森贝格在给自己和其他劳工挖坟坑的时候,一面计算着:最后一个˜Ÿ期是783,在这之前的3个10天共计焚尸4826。前后相加,总数是:5609。他算来算去,时间在算账中不知不觉地过着,他算起尸体,不,人体的平均数;5609除以坟墓数116,得数是:每座合葬坟埋人48.35,去掉尾数,即:每座坟埋48人。如果再算一算,20名劳工干了37天,那么,每名劳工平均……这时候警备队长喊道:“整队!”监督艾里弗发出响亮的命令:“正前方,齐步走!”但是他不愿进坟墓。他跑了,跌倒了,爬起来又跑,他懒懒地跑,他会算账,却不会跑,但是他没被打死。他躺在林中草地上,这里很安静,他既没有想头顶上的青天,又没有想他的兹拉塔,兹拉塔在被杀的时候已经有六个月的身孕了。他躺着,计算着挖坑时没有计算好的数字:20名劳工,37天,平均每人每天焚尸……这是第一;第二,应该算算每人用柴多少;第三,应当算算每一个被杀的人平均用多少时间焚烧……

过了一个星期,他被警察抓住,送进隔离区。

现在在这车厢里,他还在一个劲儿地嘟哝,计算,又乘,又除。要做年终决算!他要报给国家银行会计主任布赫曼。夜里,在梦中,痛楚的泪水忽然挣脱蒙在头脑和心上的疮痂,涌了出来。

“兹拉塔!我的兹拉塔!”他呼唤道。

四十五

她的房间窗户对着隔离区的铁丝网。图书管理员穆霞·鲍里索芙娜夜里醒来,掀开窗帘的一角,看见两名士兵拖着一挺机枪,擦得发亮的枪管闪着斑斑点点的青色月光,走在前面的一名军官的眼镜也闪着光。她听到低低的马达声。有汽车熄了车灯向隔离区开来,沉重的夜晚的灰土银光闪闪,在车轮周围打着圈圈儿,一辆辆汽车就像神仙的车多一样,在云雾中前进。

在这月色之下,当党卫军和保安队、乌克兰警察部队、附属部队、帝国保安局预备队的汽车队开到沉睡的隔离区大门口的时候,一个女子估量着二十世纪的这场厄运。

月光,武装队伍雄赳赳的整齐步伐,巨大的卡车的黑影,墙上挂钟的嘀嗒声,搭在椅子上的上衣、文胸、袜子,屋里暖烘烘的气息—一切无法结合的事物都融合在一起了。

四十六

一九三七年被捕后死去的老医生卡拉西克的女儿娜塔莎,在车厢里不时地试着唱歌。有时她在夜里也唱,但是人们并不生她的气。

她一向很腼腆,说话总是低垂着眼睛,声音几乎听不到,平时串门儿也只是上最亲近的人家去,看到一些姑娘有胆量在晚会上跳舞,她总是感到惊讶。

在挑选应予消灭的人时,没有把她算在手艺人和医生之列,这些人是留下性命的,因为还有点用;一个憔悴不堪、白了头发的姑娘活着没什么用处。

一个警察推搡着把她带到集市上一个灰土包跟前,那儿站着三个醉醺醺的人,其中一个是现在的警察局长,她战前就认识,那时他是一个铁路仓库的守卫队长。她甚至不明白,正是这三个人在裁决人的生与死。警察猛地一推,把她推到乱哄哄的人群里,这是一千多个被认为活着无益的女人、孩子和男人。

然后他们冒着此生最后一次暑热朝飞机场走去,看着大路两旁落了一层灰土的苹果树,最后一次尖声高叫,撕自己身上的衣服,祈祷。娜塔莎一声不响地走着。

她从来没想到,人的血在阳光中那样鲜红。有时叫声、枪声、呼吸声停息一小会儿,这时便可以听见坑里咕咕的流血声,鲜血在白白的人体上奔流着,就像流在白白的石头上。

然后发生的事就不值得可怕了:自动步枪的扳机轻轻扣动,刽子手的脸色很平常、不凶狠,而且杀人已经杀累了,正在耐心地等着她怯生生地往他跟前走,等着她站到咕咕流血的大坑边上。

夜里,她拧干浸透了血的小褂,回到城里—死人是不会从坟里走出来的,就是说,她还活着。

当娜塔莎走过一户户人家朝隔离区走的时候,她看到广场上在举行游艺会,管弦乐队在演奏她一向喜欢的一支悲伤的、带有幻想意味的华尔兹舞曲,在朦胧的月光和灯光下,在灰尘飞扬的广场上旋转着一对对舞伴,有姑娘,有士兵,脚步摩擦声与音乐声混合到一起。憔悴不堪的姑娘这时候高兴起来,并且有了信心,于是她唱了又唱,轻轻地唱,预感到有幸福在等待着她,有时候,如果没有人看到的话,甚至想要跳几步华尔兹呢。

四十七

战争开始后的一切事情,小达维德都记不清楚了。但是有一天夜里,车厢里这孩子的脑海里出现了不久前经历的一件事情。

一天晚上,外婆领着他上布赫曼家去。天空繁星点点,天边十分明亮,呈现出黄绿色,牛蒡叶子拂在腮上,就好像是什么人的凉丝丝、潮乎乎的手掌。

人们躲在阁楼上的夹层墙里。房顶的黑铁皮白天晒得烫人。有时阁楼上充满灯油气味。隔离区的大火在燃烧。白天大家都躲藏着,一动不动地躺着。布赫曼的女儿斯维特兰娜很单调地哭着。布赫曼有心脏病,白天大家把他当作死人,到夜里他吃饭,跟老婆吵嘴。

忽然狗叫起来。听到外语说话的声音:“阿斯塔!阿斯塔!犹太人在哪儿?”[51]头顶上响起轰隆轰隆的声音。德国人从天窗爬上房顶。后来,德国人钉了铁掌的靴子在铁皮房顶上踩起的轰隆声停息了。在墙脚下可以听到轻轻的、有用意的敲打声—有人敲墙传递信息。里面的人静了下来,是一种紧张的寂静,肩头和脖子上的肌肉哆嗦着,由于紧张,眼睛瞪得老大,牙齿龇露着。

小斯维特兰娜在轻轻的敲墙声中又哼起了没有歌词的诉怨曲。小姑娘的哭声忽然断了。达维德回头朝她看了看,却看到斯维特兰娜的妈妈列维卡·布赫曼的发狂的眼睛。

在这之后,有一两次他眼前刹那间浮现出这双眼睛和那小姑娘像布娃娃一样耷拉到后面的头。

可是战前的事他却记得很清楚,常常想起来。在这车厢里,他像个老头子一样,一个劲儿地想着过去,珍惜过去,玩味过去。

四十八

十二月十二日,达维德过生日的那一天,妈妈给他买了一本带画的书。在林中空地上有一只灰色的小羊羔,周围黑压压的森林显得特别凶恶。在黑褐色的树干和毒蘑菇丛中,可以看到一只狼的红红的、龇着牙的大嘴和绿色的眼睛。

只有达维德知道小羊羔一定要遭殃。他拿拳头敲桌子,拿手掌捂着林中空地,不让狼看见,但是他明白,他救不了小羊羔。

夜里他喊:

“妈妈,妈妈,妈妈!”

妈妈醒来,朝他走来,就像漆黑的夜里飞来一片云彩。他幸福地打起呵欠,觉得世界上最强大的力量保护着他,不再怕这黑压压的夜晚的森林。

等他长大了一些,他又害怕起《热带丛林之书》里的红狗。有一天夜里,屋里好像到处都有这种红色的猛兽,达维德就光着脚踩着五斗柜拉开的抽屉跨过去,钻到妈妈被窝里。

有一次他发高烧,反反复复做着同一个梦:他躺在海边沙滩上,小指头般细小的海浪冲得他的身体痒痒的。忽然在天边冒起一座蓝蓝的、无声无息的水山,水山越来越高大,并且飞快地朝他冲来。达维德躺在热乎乎的沙滩上,蓝黑色的水山朝他压过来。这比狼和红狗更可怕。

早晨,妈妈去上班。他走到黑黑的楼梯上,往一个蟹肉罐头空盒子里倒一碗牛奶,有一只尾巴细长、鼻子灰白、眼睛流泪的讨饭的猫是知道来喝的。有一天,邻居家一位大婶说,天亮时候来了几个人,带着一个小箱子,把讨人嫌的讨饭猫弄到研究所去了。

“我上哪儿去找那个研究所?这根本做不到嘛,你忘掉那只倒霉的猫吧,”妈妈看着他那恳求的眼神说,“你以后在人世上怎么过呀?心肠不能这样软。”

妈妈想把他送进儿童夏令营,他哭,央求不去,绝望地扬着手臂叫道:

“我可以去外婆家,就是不去那个营!”

他妈妈带着他到乌克兰找外婆,他在火车里几乎什么也不吃:在人前吃熟鸡蛋,或者撕开浸油的包装纸吃肉饼,他觉得很不好意思。

妈妈陪达维德在外婆家里住了五天,就准备回去上班。他跟妈妈分别的时候,没流眼泪,只是使劲儿搂住妈妈的脖子,妈妈说:

“傻孩子,搂得我喘不上气来啦。这儿有这么多便宜的草莓,过两个月我再来接你回去。”

外婆罗莎家门口就有一个公共汽车站,这一条线的公共汽车是从城里开往皮革工厂的。去世的外公原是一位崩得分子,是一个有名的人物,过去住在巴黎。外婆因此受到尊敬,也因此常常失去工作。

从开着的窗户里可以听到无线电广播:“基辅广播电台开始播音……”

白天大街上空空荡荡,有时制革专科学校的男女学生们从大街上走过,隔着街互相叫喊:“别拉,你考及格了吗?”“雅什卡,你来复习马克思主义!”这时候大街上才热闹起来。

傍晚时候,皮革厂工人们,商店店员们,还有市广播站修理工索洛卡纷纷回家。外婆在一家门诊所基层工会工作。

外婆不在家,达维德也不觉得寂寞。

外婆家旁边,有一处没有主儿的老果园,苹果树已经老得不结苹果,老山羊在里面吃草,带记号的母鸡在里面打食儿,蚂蚁不声不响地在小草上爬。城里的鸟儿乌鸦和麻雀在果园里闹闹嚷嚷,十分得意,达维德叫不出名字的一些田野的鸟儿飞进果园里,感到十分胆怯,就像羞涩的乡下姑娘。

他听到了很多新词儿:gletchik,dikt,kalyuzha,ryazhenka,ryaska,puzhalo,lyadache,koshenya。[52]他听出这些词儿和他听惯了的母语又一样又不一样。他听到了犹太话。他感到惊讶的是,妈妈和外婆当着他的面也说起犹太话。他从来没有听到妈妈说过这种他不懂的话。

外婆带他走亲戚,来到她的胖外甥女列维卡·布赫曼家。达维德看到屋里有很多编织的白色窗帘,十分吃惊。身穿制服、脚蹬皮靴的国家银行会计爱德华·伊萨科维奇·布赫曼走了进来。

“哈伊姆,”列维卡说,“这是咱们从莫斯科来的客人,拉娅的孩子。”又转身对达维德说:“来,见见爱德华姨父。”

达维德向这位会计主任问道:

“爱德华姨父,为什么列维卡姨妈管您叫哈伊姆?”

“哦,这问题有意思,”爱德华说,“难道你不知道,在英国哈伊姆就是爱德华?”

过了一会儿,有一只猫在门上乱抓起来,等到猫终于把门抓开,就看到屋里有一个小姑娘无精打采地坐在瓦罐上。

礼拜天达维德跟着外婆到市场上去。他在路上看到的有披黑头巾的老奶奶,有睡眼惺忪、愁眉苦脸的女列车员,有带蓝提包或红提包的神气活现的当地领导人的夫人,有穿高筒靴的农村妇女。

一些乞讨的犹太人用气势汹汹的粗大嗓门儿叫喊着,似乎别人对他们施舍不是出于怜悯,而是由于害怕。在石子铺的马路上奔驰着集体农庄的吨半货车,装着一袋袋的土豆或麦麸,一笼笼的母鸡,母鸡在汽车颠簸的时候咕咕乱叫,就像一群病弱不堪的老奶奶。

最使他注意、使他难受和害怕的是肉货摊。达维德看到,有人从大车上拖下宰好的黄牛,那死牛半张着苍白的嘴唇,脖子上那弯弯的白毛沾满了血。

外婆买了一只很嫩的花母鸡,提着鸡腿,鸡腿用白布条子捆着。达维德在旁边走,老想拿手帮助鸡把没有劲儿的头抬高一点儿。他很吃惊,外婆怎么这样狠心。

达维德想起了妈妈说过的一句他原来不懂的话。妈妈说,外公祖上都是知识分子,外婆祖上都是店主和买卖人。大概就因为这样,外婆对鸡一点也不心疼。

他们走进一个小院子,一个戴小圆帽的小老头儿迎着他们走出来,外婆跟他说起了犹太话。老头儿把鸡抓在手里,嘟哝起来,花母鸡信任地咕哒咕哒叫了几声,然后老头儿做了一点儿什么,那动作又快又利索,但是似乎又很可怕,紧接着他把鸡隔着肩膀一扔,那鸡便扑打着翅膀跑起来,达维德看到那鸡已经没有头,跑的只是没有头的身子,老头儿已经把鸡宰了。那鸡身子跑了几步,便倒在地上,用有劲的嫩爪子乱抓土地,过一会儿就不动了。

到夜里,这孩子觉得,那些死黄牛和被宰的小牛犊身上的潮湿气味钻进屋里来了。

住在画上的森林里的死神,原先是在画上的狼偷偷走向画上的小羊的地方,在这一天从画上下来了。他第一次感觉到,他也会死,不是像画上那样死,而是实实在在、真真切切的死。

他才知道,妈妈将来也会死的。来找他和她的死神不是从画上的森林,不是从黑压压的枞树丛里来,而是从这空气中、从生活中、从家里来,想躲也躲不开。

他对死的感触是那样深、那样真切,这样的感触只有小孩子和伟大的哲学家才会有,伟大哲学家的思维力之强和小孩子感情的单纯与强烈,是差不多的。

那坐垫已破、上面重新钉了胶合板的椅子,那厚实的衣橱,散发着一种宁静的、亲切的气味,就像外婆的头发和衣服上的气味。这儿的夜晚是暖和的,表面上很宁静。

四十九

在这个夏季,他的生活离开了拼字方块,离开了画在识字课本上的图画。他看到,公鸭子那黑黑的翅膀泛着多么好看的蓝色光泽,鸭子笑起来和叫起来多么好玩,多么好笑。枝丛里闪烁着白色的甜樱桃,他顺着疙疙瘩瘩的树干爬上去,爬到樱桃跟前,一伸手就摘下来。牛犊拴在空地上,他走过去,拿糖块喂牛犊;小牛犊看到胖乎乎的男孩那可爱的眼睛,快活得惊呆了。

红头发的佩契克走到达维德跟前,说:

“咱们来干一架!”

外婆院子里住的犹太人和乌克兰人彼此十分相像。帕尔丁斯卡娅老奶奶来到外婆屋里,慢悠悠地说:

“罗莎·努西诺芙娜,您觉得怎么样,索尼娅上基辅去啦,又跟丈夫和好啦。”

外婆把胳膊一扬,笑着回答说:

“噢,您又看着笑话了。”

达维德觉得这儿的世界比基洛夫街上更好,更可爱。在基洛夫街上的时候,在小小的沥青院子里经常有一个姓德拉科—德拉康的浓妆艳抹的卷发老太太带着卷毛狗在散步,每天早晨大门口都停着一辆“吉斯—101”小汽车,一个戴夹鼻眼镜的女邻居,抹口红的嘴上叼着香烟,对着公用煤气炉一个劲地嘟哝:

“你这托洛茨基分子,把我炉盘上的咖啡推过来。”

妈妈那天夜里领着他出了车站。他们顺着洒遍月光的石子铺的大街往前走,经过一座白色的天主教堂,在神龛里站着瘦削的弯腰戴着荆冠的耶稣,个头像个十二岁的男孩,又经过妈妈过去上过的专科学校。

过了几天,在星期五的傍晚,达维德看到一些老头子在一片金色灰尘中朝犹太教堂走去,那灰尘是光脚的足球队员在空地上蹚起的。

这儿的乌克兰式白房子,咯吱咯吱的水井吊杆,黑白相间的祈祷服上使人眼花缭乱的表现圣经故事的古老纹饰,这一切糅合在一起,就产生了惊人的美。这儿有《民间歌手》[53],有普希金和托尔斯泰的书,有物理课本,有《共产主义运动中的“左派”幼稚病》,有国内战争时期跑来的鞋匠和裁缝的儿子,有区委指导员,有区工会理事会的斗士和宣传员,有汽车司机,有侦讯处的侦查员,有马克思主义讲解员。

达维德来到外婆家以后,才知道妈妈是很不幸的。首先告诉他这一点的是拉赫莉阿姨,是一个胖胖的女人,两腮通红通红的,就好像老是在害臊。她说:

“扔掉你妈妈这样好的女人,实在是罪过。”

过了一天,达维德已经知道,他的爸爸上一个俄罗斯女人那儿去了,那女人比他大八岁,他在音乐厅每月挣两千五百卢布,妈妈不要赡养费,仅仅靠自己每月挣的三百一十卢布生活。

达维德有一天把装在火柴盒里的一个蚕茧拿给外婆看。

可是外婆说:

“嘿,你留这脏东西干啥,快点儿扔了。”

迖维德有两次跑到货车站,看着往车厢里装牛、羊和猪。他听到老牛哞哞直叫,不知是在诉苦,还是在祈求怜悯。达维德心里很害怕,可是穿着又脏又破的服装的铁路工人在车厢旁边走来走去,也不转过疲惫的瘦脸去看看哞哞叫的老牛。

达维德来了一个星期之后,外婆的邻居、农机厂钳工拉萨尔·扬凯列维奇的妻子杰波拉生下头生儿子。去年杰波拉到科雷马去探望姐姐,在雷雨时候受到电击;她像死人一样躺了两个钟头,后来被救活了,今年夏天就生了孩子。她十五年来一直没有孩子。这是外婆对达维德说的。外婆又说:

“大家都是这么说的,可是,不光是这样:去年医生还给她做过手术。”

有一天,外婆带着达维德看望这家邻居。

“嗯,拉萨尔。嗯,杰波拉。”外婆看了看躺在衣服篮子里的两脚动物。她说话带着一种很严厉的口气,好像警告孩子的父亲和母亲对待这出现的奇迹不能马虎。

在铁路旁边的一座小屋里住着索尔金娜老太婆和两个儿子,两个儿子都是又聋又哑的理发匠。邻居都很怕他们。

“他们不喝酒的时候,挺老实,”帕登斯卡娅老奶奶对达维德说,“等他们一喝了酒,就要打架,又嚷嚷,又拿刀子,窜来窜去,跟野马一样!”

有一次外婆叫达维德去给图书管理员穆霞·鲍里索芙娜送一小罐酸奶油……她那间屋子非常小。桌上有一只小碗,墙上钉着小小的书架,书架上有一本一本的小书,小床上面挂着一张小小的照片。照片上是妈妈和襁褓中的达维德。达维德看到照片,穆霞·鲍里索芙娜脸红了,并且说:

“我跟你妈妈是同桌同学呢。”

他给她念了关于蜻蜓和蚂蚁的寓言故事,她也小声给他念了一首诗的开头:

“看到砍伐森林,萨沙哭了……”

早晨,院子里闹哄起来:索洛蒙·斯列波依家里一件皮袄,已经撒了香料、包起来准备过夏天的,夜里被偷了。

外婆一听说斯列波依家的皮袄被偷,就说:

“谢天谢地,应该让这强盗倒倒霉。”

达维德听说,斯列波依是一个喜欢告密的人,在取消旧币和金卢布的时候,他出卖了很多人。在一九三七年他又出卖了一些人。在他出卖的人当中,有两个被枪决,一个死在监狱的医院里。

夜晚可怕的沙沙声、无辜的鲜血和鸟儿的歌声—这一切合成惊心动魄的、乱糟糟的一团。达维德要理解这一切,还得过几十年。但是他的小小的心灵却日日夜夜感受到那动人的美和可怕。

五 十

为了宰杀害了传染病的牲口,要做一系列准备工作:把牲口运送和集中到屠宰点,给屠宰工人作指示,开挖壕沟和大坑。

居民们帮助政府把染病的牲口送往屠宰点,或者帮助捕捉跑散的牲口。他们这样做不是因为痛恨牛犊或老牛,而是出于自我保全。

在大规模屠杀人的时候,一般的人对于要被消灭的老人、妇女和儿童同样没有切齿的痛恨。所以,要进行大规模的消灭人的运动,必须进行特殊的准备。在这方面,光有自我保全的心态是不够的,还必须唤起一般人的憎恶和仇恨。

对乌克兰和白俄罗斯犹太人的种族灭绝,正是在这种憎恶和仇恨的气氛中进行的。当年,也是在这块土地上,斯大林煽动起群众的痛恨,推行了消灭富农阶级的运动和残杀托洛茨基—布哈林分子的运动。

经验证明,在这样的运动中大多数人对政府的指示只是盲目服从,也有少数人是为运动摇旗呐喊、制造气氛的。其中有残忍成性、幸灾乐祸的糊涂虫,也有抱着个人目的和打算的,想要捞到别人的财物、住房和职务空缺。大多数人心里害怕大规模的残杀,然而他们尽量不露声色,不仅是对最亲近的人,而且对自己隐瞒真实的心情。一有煽动种族残杀的大会,这些人就坐满了会场。不论这样的大会开多少次,不论会场上有多少人,几乎没有什么人破坏一致默认的事。要是一个人面对被怀疑的疯狗,看到疯狗祈求的目光而没躲开,并且让疯狗住到自己和妻子儿女同住的家里去,这样的事就更少了。不过,这样的事总归还是有的。

二十世纪上半叶在历史上将占有特殊地位,因为它是伟大科学发明的时代,革命的时代,巨大的社会变革的时代和两次世界大战的时代。

但是,二十世纪上半叶将以普遍残杀各阶层犹太人的时代进入人类历史,而这一残杀运动还有种族和社会理论的根据。当代现实抱着不难理解的谨慎态度,对此讳莫如深,保持沉默。

在这个时期暴露出来的人类天性最惊人的一个特点就是顺从。有时候,前往行刑的地方要排很长的队,等待被杀的人就自动排队。有时候,等待受刑要从早晨等到深夜,在长长的炎热的一天中,已经知道这件事的母亲提前带着水和面包为儿子准备着。成千上万的无辜者感觉到自己快要被逮捕了,提前把衣服和手巾包好,提前和家里人告别。千百万人住在巨大的集中营里,这些集中营不仅是他们自己建造的,而且自己看守着。

不是一万、两万人,甚至也不是几千万人,而是无数的芸芸众生成为旁观者,看着顺从的无辜者被杀害。他们不只是顺从的旁观者,等到要他们做表决的时候,他们会众口一声地表示赞成大规模的屠杀。这种大量的人的顺从,是新发现的一种意外。

当然,也有反抗,也有人英勇、顽强,也有起义,也有自我牺牲。有的人为了挽救毫不相干的陌生人,献出了自己和家人的生命。可是,群众性的顺从总归是无可争辩的事实!

这种顺从说明什么呢?是不是说明在人的天性中忽然出现了新的特点?不是。这种顺从说明有一种新的可怕的力量对人的影响。极权社会的超级暴力,足以造成所有大陆上人类灵魂的麻痹。

甘心为法西斯效劳的人会把只能使人遭殃和灭亡的奴性称作唯一和真正的美德。出卖国家民族的人一面承认人类感情,一面说法西斯的种种暴行是最高形式的人道主义,赞成把人分为高雅的、体面的人和不高雅、不体面的人。自我保全的欲望,就表现在生存本能与良心的相互妥协。

一些影响遍及世界的思想所具有的麻醉力量,支持着生存的本能。这样一些思想号召:为了祖国的伟大前途,为了人类幸福,为了民族、阶级的幸福,为了人类的进步事业,为了达到伟大的目的,不惜任何牺牲,不惜采取任何手段。

除了一些伟大思想的麻醉力量,跟生存本能一同起作用的还有第三种力量,这就是对于强大国家机器不受限制的强权,对于已成为国家日常生活基础的残杀的恐惧。

极权国家的强权是如此巨大,以至于它不再是手段,而变成了神秘的宗教崇拜的对象。

要不然怎样解释一些有思想有知识的犹太人的说法呢?他们说,为了人类幸福必须杀尽犹太人,他们认识到这一点,愿意把自己的孩子领到屠杀点去,为了祖国的幸福,他们愿意作出牺牲,就像圣经上的亚伯拉罕那样。

要不然怎样解释一位农民出身的才智双全的诗人的作为?他怀着真挚的感情写了一首长诗,歌颂农民受苦受难的血腥时期,正是那个时期吞噬了他那忠厚、纯朴、干了一辈子庄稼活儿的父亲。

法西斯制服人的手段之一,就是使人完全地,或者近乎完全地丧失理性。人们不相信会被消灭。说来奇怪,已经站在坟坑边上,竟是那样乐观。在极不明智的,有时是不可告人的、可鄙的希望的基础上产生的顺从,也是见不得人的,有时甚至是可鄙的。

华沙起义、特雷布林卡集中营的起义、索比波尔集中营的起义、炉工们的暴动和起义,都是由于完全失去了希望。

但是,真实、彻底的绝望引起的不仅是起义和反抗,也能使一些人产生正常人不能理解的早作刀下鬼的渴望。有些人就为了走向血淋淋的埋人坑的先后而争吵,还能听到兴奋的、激昂的、几乎是狂喜的叫喊声:“犹太弟兄们,不要怕,没有什么可怕的,再有五分钟就行了!”

希望能产生顺从,失望也能产生顺从,因为同命运的人们的性格各不相同。

需要想想人们遭受的苦难和折磨,才能理解为什么有些人认为早点儿被杀是幸运的。很多人应该想想这一点,特别是那些喜欢教导人的人,他们常常教导人在艰难境况下应当怎样进行斗争,可惜这些说空话的导师都很幸运,想象不出那样的境况。

明白了人对于强权暴力的顺从,还必须做出最后的结论,这样的结论对于理解人、理解人的未来是有意义的。

人的天性会不会起变化,在极权暴力作用下会不会变异?人会不会失去生来就有的对自由的渴望?人的命运、极权国家的命运就在对这一问题的回答中。人如果改变了天性,国家独裁制必然会取得世界性的永久的胜利;人追求自由的愿望不改变,就是对极权国家宣判死刑……

人类渴望自由的天性是消灭不了的,可以压抑,但无法消灭。极权政治不能不使用暴力。如果离开暴力,极权政治就会完蛋。经常或者不断使用的超级暴力,露骨的或者经过伪装的超级暴力,是极权政治的基础。人不会自愿地放弃自由。我们时代的曙光、未来的曙光就在这一结论中。

五十一

电子计算机能进行数学计算,能记下历史事件,能下棋,能翻译书籍。电子计算机快速计算数学习题的能力超过了人,其记忆力也是无可比拟的。

根据人的模式和行动创造机器的科学,其发展有无极限?显然,没有这种极限。

可以想象出未来几个世纪和几十个世纪的机器。它可以听音乐,欣赏绘画,而且它自己能够作画,作曲,写诗。

它的完善有极限吗?能否与人媲美,甚至超过人?

机器模仿人,将要求电子学不断有新的发展,电子元件的重量和体积不断更新。

回忆童年……高兴时流泪……离别时伤心……热爱自由……心疼生病的小狗……疑神疑鬼……母爱的抚慰……考虑死亡……悲伤……交朋友……同情弱者……突然萌生的希望……准确的猜测……忧愁……无缘无故的快乐……无缘无故的慌乱……一切,一切,机器都能做到!但是,即使渐渐能代替一个最普通、最平常的人的智慧和心灵,不断增加的机器的负荷,整个地球的土地都将容纳不了。

法西斯竟消灭了几千万人。

五十二

在乌拉尔林区小村中一个宽敞、明亮、整洁的房间里,坦克军军长诺维科夫和政委格特马诺夫正在看接到出发命令的各旅旅长的报告,快要看完了。

一连几昼夜不眠的工作换来宁静的时刻。

就像在类似的情况下一样,诺维科夫总觉得他们的时间不够,无法完全、充分地掌握教学大纲规定的内容。但是,学习阶段—掌握坦克发动机和传动部分操作规程、掌握大炮技术、使用光学瞄准器和无线电通信设备的阶段,已经结束了;操纵火力,判断、选择和确定目标,选择射击方法,确定开火时刻,观察爆炸点,校正目标、变更目标等项训练全结束了。

今后的教员将是战争,战争会很快地把人教会,还会督促落后者,弥补不足。格特马诺夫朝两个窗户之间的小橱探过身子,拿指头敲着小橱,说:

“喂,伙计,出来吧。”

诺维科夫把橱门打开,拿出一瓶白兰地,把两只蓝色的厚玻璃杯斟满了。

这位军长一面考虑着,一面说:

“咱们为谁干杯呢?”

诺维科夫自然知道应该为谁干杯,所以格特马诺夫也问:

“你说该为谁?”

诺维科夫犹豫了一下子之后,说:

“来,政委同志,为咱们率领作战的同志们干杯,愿他们少流血。”

“很对,首先要关怀各负责干部,”格特马诺夫随口说,“来,为咱们的小伙子们干杯!”

他们碰了杯。

诺维科夫带着掩饰不住的抢先心情又斟了两杯,说:

“为斯大林同志干杯!为了不辜负他的信任。”

他看到隐藏在格特马诺夫那亲切而留神的眼睛里的冷笑,便责备起自己,心想:“唉,太着急啦。”

格特马诺夫和善地说:

“是的,不错,为他老人家,为咱们的父亲干杯。咱们要在他的率领下打到伏尔加河边。”

诺维科夫看了看政委,可是,从这个四十岁的聪明人颧骨突出的微笑的大脸上,从他那又快活又厉害的眯细的眼睛里又能看出什么呢?格特马诺夫忽然谈起军参谋长涅乌多布诺夫将军:

“是一个好人,一个很好的人。一个布尔什维克。一个真正的斯大林主义者。有丰富的领导工作经验。有坚强的毅力。我记得他在一九三七年的情形。叶若夫[54]派他主持军区的肃反。我当时也担任很重要的工作。可是谁也没有他那样的魄力。雷厉风行,毫不手软,说枪毙就枪毙,不次于乌尔里赫,没有辜负叶若夫同志的信任。应当现在马上把他请来,要不然他还要生气呢。”

在他的口气中仿佛有不满意肃反斗争的意味,据诺维科夫所知,他也曾参加肃反斗争。于是诺维科夫又看了看他,还是什么也看不出来。

“是啊,”诺维科夫慢慢地、很不利落地说,“那时候有些人的做法很不对头。”

格特马诺夫把手一挥。

“今天收到总参一份战报,情况很严重:德国人已经接近厄尔布鲁士,在斯大林格勒眼看着就要把我军逼到水里。我要坦率地说:我们杀自己人,消灭大量干部,我们的厄运就是这些事造成的。”

诺维科夫一下子就对格特马诺夫产生了信任感,说:

“是啊,这些同志杀害了不少有才能的好人,政委同志,在军队里造成的不幸的事太多了。就比如军长克里沃卢契科在审讯中被打坏一只眼睛,他又用墨水瓶把侦讯员的脑袋打碎。”

格特马诺夫点点头,表示有同感,又说:

“贝利亚同志很器重咱们的涅乌多布诺夫。贝利亚同志是不会看错人的,他可是一个聪明人,确实聪明。”

“是的,是的。”诺维科夫在心里慢悠悠地想道,却没有说出口来。

他们沉默了一会儿,倾听着隔壁不太高的说话声:

“胡说,这是我们的袜子。”

“就算你们的吧,少尉同志,不过您怎么,迷糊啦?”接着又把“您”换成“你”,说:“你往哪儿放?别动,这是我们的衬领。”

“副指导员同志,你拿去看看,这哪儿是你们的?”这是诺维科夫的副官和格特马诺夫的办事人员在洗过衣服以后分检首长的衣物。

格特马诺夫说:

“我一直在观察他们这两个家伙。那一天咱们到法托夫营里去看射击演习,我和您在前面走,他们跟在后面。过小河沟的时候,我踩着小石头走过去,您跳过去,为了不踩到泥巴,把一条腿一蹬。我看到:我的办事人员也踩着小石头走过去,您的副官也跳过去,而且也把一条腿一蹬。”

“喂,两位勇士,别吵啦!”诺维科夫说。

隔壁房间里马上安静下来。

涅乌多布诺夫走了进来。他脸色苍白,宽阔的额头,密密的头发白了不少。他打量了一下酒杯和酒瓶,把一叠文件放到桌上,向诺维科夫问道:

“上校同志,咱们该对第二旅参谋长怎么办?米哈廖夫过一个半月才能回来。我收到军区医院的诊断结论啦。”

“他没有了肠子,胃也去掉了一部分,怎么能做参谋长呀?”

格特马诺夫说过这话,斟了一杯白兰地,递给涅乌多布诺夫。

“将军同志,趁着肠子还在,喝一杯吧。”

涅乌多布诺夫扬起眉毛,带着询问的神气用淡灰色的眼睛看了看诺维科夫。

“请吧,将军同志,请吧。”诺维科夫说。

他很不满意格特马诺夫那种自以为处处可以当家作主的作风。格特马诺夫好像自信有权在讨论技术问题的会议上发表长篇大论,其实他根本不懂什么技术。格特马诺夫还常常拿别人的酒招待客人,让客人在别人的床上休息,看别人桌上的文件,认为自己有权这样做。

“是不是暂时派巴桑戈夫少校代理参谋长?”诺维科夫说。“他是一位精明能干的指挥员,在沃伦斯基新城战役中就参加过坦克战斗。政委没有意见吧?”

“意见当然没有,”格特马诺夫说,“我怎么会有意见……不过,倒是有一点想法:第二旅上校副旅长是亚美尼亚人,现在又想让一个卡尔梅克人做他们的参谋长。要知道第三旅参谋长,那个叫利夫希茨的,也是卡尔梅克人。我们离了卡尔梅克人就不行吗?”

他看了看诺维科夫,然后又看了看涅乌多布诺夫。

涅乌多布诺夫说:

“说心里话,按家常道理来说,您这话是对的,不过马克思主义要咱们从另外一个角度来看待这个问题。”

“要紧的是,这个同志怎样打德国人,这就是我的马克思主义,”诺维科夫说,“至于他的父亲是在哪儿祷告,是在天主教堂,还是在清真寺……”他想了想,又说:“还是在犹太教堂,我都不管……我认为,在战争中最要紧的是射击。”

“是的,是的,正是这样,”格特马诺夫快活地说,“在坦克军里咱们还管什么清真寺和犹太教堂?反正咱们是保卫俄罗斯。”

忽然他阴沉下脸,发狠地说:

“说实在话,够啦!简直叫人受不了!为了各民族友谊,咱们总是拿俄罗斯人当牺牲品。少数民族的人,只要能认识几个字母,我们就要把他们选为人民委员。咱们俄罗斯人,哪怕浑身是本事,都得让开,让路给少数民族的人!伟大的俄罗斯民族倒变成了小民族。我赞成各民族友好,但是不赞成这样的做法。够啦!”

诺维科夫想了想,看了看桌上的文件,拿手指甲敲了一会儿酒杯,说:

“怎么,我是对卡尔梅克族人抱有特别的好感,压制俄罗斯人了吗?”

他转过脸朝着涅乌多布诺夫,说:

“好吧,请您发命令:任命萨佐诺夫少校为第二旅代理参谋长。”

格特马诺夫用不高的声音说:

“萨佐诺夫是一位出色的指挥员。”

诺维科夫本来是学会了做一个粗暴、威风和强硬的人的,这会儿却又感到自己在政委面前缺乏自信……

“好啦,好啦,”他在心中安慰自己说,“我不懂政治。我是无产阶级的军事专家。我管不了那许多:只管打德国佬。”但是,尽管他也常常在心里嘲笑格特马诺夫不懂军事,承认自己在他面前感到胆怯却是很不愉快的。

格特马诺夫老大的脑袋,一头乱发,个头儿不高,肩膀却很宽阔,肚子很大,但十分敏捷,说话声音不高,爱说爱笑,精力异常充沛。

尽管他从来没有上过前线,可是在各旅里谈到他时,都说:

“噢,我们的政委很有战斗经验!”

他很喜欢召开红军官兵大会;大家很喜欢听他讲话,他讲话很随便,很风趣,有时还说些粗话。他走路有些蹒跚,常常拄着手杖,如果有坦克兵忘记向他行礼,他就在坦克兵面前站下来,拄着手杖,摘下帽子,像乡下佬那样鞠一个九十度的大躬。

他爱发火,不喜欢听反对意见。要是有人和他争论,他便阴沉着脸,鼻子里直哼哧。有一次他发了火,抡起拳头,照着重坦克团参谋长古宾科夫轻轻地打了一拳。古宾科夫是个很固执的人,同志们说他“原则性强得可怕”。

格特马诺夫手下的办事人员一提到这位固执的大尉,就用责备的口气说:

“这家伙把我们政委气坏啦。”

格特马诺夫对那些经历过战争初期艰难日子的人毫无敬意。有一次他谈起诺维科夫很器重的第一旅旅长马卡罗夫,说:

“我要打掉他一九四一年那一套!”

诺维科夫没有作声,虽然他很喜欢和马卡罗夫谈论战争初期那些可怕而又吸引人的日子。

格特马诺夫的见解之大胆、尖刻,似乎恰恰是涅乌多布诺夫的对立面。这两个人尽管非常不相像,但因为也有某种永远一致的地方,所以团结得很好。

诺维科夫看到涅乌多布诺夫不露表情然而凝神注视的目光,听到他圆滑的措辞和总是平心静气的语调,就感到纳闷。

可是格特马诺夫却哈哈笑着说:

“我们很幸运,德国佬一年来对庄稼汉造的孽,比共产党二十五年来造的孽还多。”

有时忽然冷笑着说:

“没说的,咱们的老爷子就喜欢让人说他英明伟大。”

这种大胆并不能感染别人,倒是会引起别人担心。

战前格特马诺夫领导一个州,常常就耐火砖的生产问题和煤炭研究院分院如何进行科学研究的问题作报告,常常谈本市面包工厂的生产质量,谈刊登在地方丛刊上的小说《蔚蓝色的火》中的谬误,谈车辆的修理问题,谈州商业局货栈商品的仓储管理水平低下,谈集体农庄养禽场流行的鸡瘟。

现在他又很有把握地在谈燃料的质量、发动机损耗率、坦克战战术、坦克与步兵和炮兵协同进攻敌方永久性防御工事、行军时的坦克、战场救护、密码电报、坦克手的作战心理、每个坦克组内部和坦克组关系的特点、坦克的抢救与大修、受损的坦克如何从战场上转移。

有一天,诺维科夫和格特马诺夫来到法托夫大尉的营里,在获得全军射击第一名的一辆坦克旁边站了下来。这辆坦克的坦克手在回答首长的问题的时候,轻轻地用手掌在坦克的装甲钢板上抚摩着。格特马诺夫问坦克手,得到第一名是不是很难。这名坦克手一下子就来了精神,说:

“不,没什么难的。我太喜欢它了。我从乡下一进学校,一看到坦克,就喜欢得不得了。”

“一见钟情嘛。”格特马诺夫说着,笑了起来。在他的宽厚的笑中,似乎有不赞成小伙子对坦克这种可笑的爱的意味。

诺维科夫此刻觉得自己也有这个短处,因为他爱坦克也爱得不高明。不过他并不想跟格特马诺夫谈谈这种不高明的爱的水平,而且,当格特马诺夫换成严肃的神气,用教导的口吻对坦克手说“好样儿的,爱坦克是一种了不起的力量。正因为你爱自己的坦克,所以才取得成就”的时候,诺维科夫用嘲笑的口吻说:

“实际上,坦克有什么可爱的?坦克是很大的目标,打坦克比什么都容易,响声比什么都大,自己暴露自己,驾坦克的人能叫坦克响声震昏。开起来颠簸得厉害,既不能好好地观测,又不能好好地瞄准。”

格特马诺夫当时微微一笑,看了看诺维科夫。这会儿,格特马诺夫一面斟酒,一面也微微一笑,看了一眼诺维科夫,说:

“咱们的路线要经过古比雪夫。咱们的军长可以有机会和什么人见见面啦。咱们来干一杯,祝贺这次相会。”

“拿我开心,岂有此理!”诺维科夫在心里说。他觉得自己的脸像小孩子那样通红通红的了。

战争开始的时候,涅乌多布诺夫正在国外。只是在一九四二年初回莫斯科,到国防人民委员部报到以后,他才看到莫斯科河南岸的街垒和防坦克菱形拒马,听到空袭警报的笛声。

涅乌多布诺夫和格特马诺夫一样,从来不向诺维科夫询问有关战争的事情,也许是怕暴露自己在军事上的无知。

诺维科夫思索着这位军参谋长的一生,一直想弄清他是凭什么资格成为将军的。涅乌多布诺夫的生平在履历表里反映得清清楚楚,就像映照在塘水里的小白桦树。

涅乌多布诺夫的年纪比诺维科夫和格特马诺夫都大。在一九一六年因为参加布尔什维克小组就进了沙皇的监狱。

国内战争以后,他响应党的号召在政治保卫总局[55]工作过一个时期,后来在边防军工作,又被送到军事学院学习,学习期间担任年级党组织书记……后来又在党中央军事部、国防人民委员部中央机关工作。

战前他两次出国。他是上级任命的工作人员,属于特别登记的人员,以前诺维科夫不十分明白这有什么意义,不明白上级任命的工作人员有什么与众不同,有什么了不起。

从申报军衔到得到军衔,一般都要经过很长时间,涅乌多布诺夫的军衔从申报到批准却快得出奇,好像国防人民委员部就等着批他的申报材料呢。履历表具有很奇怪的特点:它能说明人的一生中所有的秘密,说明成功与失意的原因,可是,过了一阵子,在新的情况下,结果却什么也不能说明了,相反,倒是掩盖了实质。

战争用自己的眼光重新审查了履历表、自述、鉴定、奖状……所以上级任命的涅乌多布诺夫成了上校诺维科夫的下属。

涅乌多布诺夫明白,等战争结束,这种不正常的状况也会结束的……

他带了猎枪来到乌拉尔,军里所有喜欢打猎的人都惊得发了呆,诺维科夫说,大概沙皇尼古拉当年就是用这样的猎枪打猎的。这支猎枪是涅乌多布诺夫在一九三八年凭一张领物证领到的,他还凭领物证从特别仓库领到家具、地毯、瓷器和别墅。

不论谈战争,谈德拉戈米罗夫将军的著作《集体农庄》,谈中华民族,谈罗科索夫斯基将军的人品,谈西伯利亚的气候,谈俄罗斯大衣呢的质量,或者谈金发女子比黑发女子漂亮,他的见解都不超出规格。

很难理解,他这是拘谨,还是真实内心的表露。

有时在吃过晚饭之后,他的话多起来,说起揭露反革命破坏者的事,这些破坏者活动在最使人意想不到的部门:生产医疗器械的工厂、生产军鞋的车间、食品厂、地方的少年宫、莫斯科赛马场的马棚、特列季亚科夫美术馆。

他的记性特别好。看样子,他读了很多书,列宁和斯大林的著作他读了很多遍。在争论的时候,他常常说:“斯大林同志在十七次党代表大会上就说过……”于是他从中引出一段话。

有一天格特马诺夫对他说:

“引文归引文。书上讲的话多着呢。书上说:‘我们不要别人的土地,自己的土地我们一寸也不让。’我们的土地不是已经让德国人占了吗?”

可是涅乌多布诺夫耸耸肩膀,就好像侵占着伏尔加河的德国人跟一寸土地也不让的话一点也不相干似的。

忽然,一切都消失了,坦克、战斗条令、射击、森林、格特马诺夫、涅乌多布诺夫……都隐没了。啊,叶尼娅!难道他能再看见她吗?

五十三

诺维科夫觉得很奇怪,格特马诺夫看完了家信之后竟说:

“我老婆可怜咱们呢,因为我在信里对她说了说咱们这儿现在的生活条件。”

政委以为很艰苦的生活,诺维科夫却觉得很阔气,觉得过起来有愧。

他起初自己选了一套住房。有一次他在下旅里去的时候说,他不喜欢房东家的大沙发,等他回来,沙发换成了木靠背的安乐椅,而且他的副官维尔什科夫还不放心,不知道军长是否喜欢这张安乐椅。

炊事员也常常问:“上校同志,汤怎么样?”

他从小就喜欢动物。现在他的床底下就住着刺猬,到夜里剌猬就吧嗒吧嗒地拿小爪儿敲着地面,大模大样地在屋里到处跑。修理工还做了一个带有坦克标记的笼子,笼子里有一只小小的花老鼠,夜里就在里面嗑花生。小花鼠很快就和诺维科夫混熟了,有时就坐在他的膝盖上,拿孩子般的又信任又好奇的小眼睛看着他。副官维尔什科夫、炊事员奥尔列涅夫、吉普车司机哈里托诺夫,大家对这些小动物都很关心,很爱护。

诺维科夫觉得这都不是微不足道的小事。战前他把一只小狗带进领导干部住的一座楼房里,小狗咬坏了邻居一位上校夫人的鞋子,半个钟头撒了三泡尿,弄得公共厨房里一些人大叫大嚷起来,诺维科夫只好马上把狗送走。

出发的日子到了,一个坦克团团长和该团参谋长之间的复杂的纠纷还是没有解决。出发的日子到了,和出发的日子一起来到的是种种操心事:油料问题,路上的给养问题,上军车的次序问题。

今天就要有一些步兵和炮兵团队同时出发,朝铁路方向开去,诺维科夫一想到就要和步兵、炮兵的领导人配合共事,心里激动起来。他还十分激动地想着一个人,他要在那人面前立正站定,说:“上将同志,请允许我报吿……”

出发的日子到了,没有来得及见哥哥和侄儿。原来心想,来到乌拉尔,哥哥就在跟前了,谁知竟没有时间去看看。

现在已经向他这位军长报告了各旅的行动,报吿了装运重型坦克的车辆问题,还报告说,已经把刺猬和小花鼠放归森林。

当家作主,要对每一样小事负责,关照每一处细小的地方,是很不容易的。现在坦克都已经各就各位了。可是,制动器是否装好了?是不是挂上了一档?炮塔上的炮口是不是朝前?舱口的盖是不是盖紧?是不是准备了木头块垫坦克,防止车厢颠簸?

“喂,咱们临走来打打牌吧。”格特马诺夫说。

“我没意见。”涅乌多布诺夫说。

但是诺维科夫想出去走走,一个人待一会儿。

在这静静的傍晚时分,空气格外清爽,就连最微小、最不惹眼的东西都显得极其清楚。从烟囱里冒出来的一股股的烟,不绕圈儿,垂直地向上升去。劈柴在行军灶里噼噼啪啪地响着。街心里站着一个黑眉毛的坦克手,一位姑娘抱住他,把头放在他的胸前,哭了起来。一些人把箱子、提包、套了黑套子的打字机从军部的房子里往外搬。通信兵在拆通向各旅部的电话线,把又黑又粗的电线绕成圈儿。军部的一辆坦克停在棚子外面,喘着粗气,冒着白烟,不时地突突响几声,准备出发。坦克兵在往新的货运“堡垒”里加油,揭下舱口盖上绗得密密实实的罩布。四周依然静悄悄的。

诺维科夫站在台阶上,四下里看了看,忙乱和操心离开他,跑到一边去了。

太阳快落山的时候,他乘的吉普车驶上去车站的大路。

坦克纷纷从森林里开出来。

结了冰的土地被坦克轧得咯吱咯吱直叫。夕阳照耀着远处枞树林的树顶,卡尔波夫中校的那个旅正从那边开过来。马卡罗夫旅正在小白桦林中行进。坦克兵们拿树枝掩护着钢甲,仿佛那枞树枝和白桦枝叶跟坦克的钢甲,跟马达的隆隆声、履带的银光闪闪的轧轧声,都是一块儿诞生的。

军人们看到出发上前线的后备队,都会说:“要举行婚礼啦!”

诺维科夫让吉普车开到路边上,看着一辆辆坦克从他身边开过去。

他们在这儿闹出多少事情啊,多少奇怪的、可笑的事情!什么样的重大事故没向他报告过呀……在一次军部营里开早饭,在菜汤里发现了一只青蛙……上过十年级的少尉罗日杰斯文斯基在擦枪的时候走了火,打伤了一个同志的肚子,误伤同志之后,少尉罗日杰斯文斯基竟自杀了。摩托化步兵团的一名战士拒绝宣誓,说:“宣誓只能在教堂。”

蓝灰色的轻烟挂在路边的树枝上。

在这些盔形皮帽底下的一个个头脑里有许许多多各种各样的想法。其中有跟全体人民一致的,如痛恨战争,热爱自己的土地;但也有惊人的不一致,正因为不一致,人类的一致才显得美好。

天啊,我的天啊……穿黑色坦克服装、腰系宽皮带的小伙子有多少啊。领导挑选的都是宽肩膀、小个头儿的小伙子,为的是爬进爬出坦克方便,在里面活动起来也方便。在他们的履历表上所填写的出身、出生年月、毕业的学校、拖拉机手训练班,有多少全都一样啊。一辆辆扁平的“T—34”绿色坦克汇合到一起,舱口的盖子全都开着,绿色的钢甲上全都系着防雨布。

有的坦克手唱着歌儿;有的坦克手半闭起眼睛,怀着恐惧和不祥的预感;有的在想家;有的在吃面包就香肠,一心想着香肠;有的张着嘴,聚精会神地辨认树上的是不是鸡冠鸟;有的还在担心,昨天说了一句很不礼貌的话,是不是得罪了同志;有的有气未消,想着点子,一心想叫跟自己作对的、行进在前面的坦克手吃吃拳头;有的在心里作诗,抒发告别秋日森林时的惆怅;有的想着姑娘的酥胸;有的心疼小狗,知道小狗就要被抛弃在空荡荡的驻地上了,刚才小狗还扒到坦克钢甲上,恋恋不舍地摇着尾巴;有的想着到森林里去,一个人盖间小屋子,吃野果,喝泉水,光脚走路,该有多么惬意;有的在考虑,是不是装病,躲到什么地方的医院里去;有的在默念小时候听来的故事;有的想起姑娘的情话,不再因为永别而伤心,倒是感到幸福;有的想着将来:战后能做一个食堂经理,就太好啦。

“唉,弟兄们……”诺维科夫心里说。

他们都看着他。大概他是在检查他们的军装是否整齐。他也可能在听马达的声音,根据马达声判断驾驶员和机械师是否有经验。他在注视,坦克与坦克、分队与分队之间是否保持着应有的距离,莽撞的小伙子们是否会争先恐后。

他看着他们,就像他们看着他一样,他们的心事,他也有:他又想格特马诺夫自作主张打开的那瓶白兰地,又想到涅乌多布诺夫这个人多么难以相处,又想再也不能在乌拉尔打猎了,最后一次打猎毫无收获,胡乱打枪,大口喝酒,闹了不少笑话……他又想到,他就要看到他爱了很多年的女人了……六年前听说她嫁了人的时候,他写了一个简短的报告:“请长假。附件:手枪10322号。”他当时在尼科利斯克—乌苏里斯基的部队里。幸亏他没有扣扳机……

这里面有腼腆的,有郁郁寡欢的,有喜欢笑的,有冷漠的,有深思熟虑的,有色鬼,有不得罪人的自私自利者,有流浪汉,有吝啬鬼,有喜欢冷眼旁观的人,有老好人……现在他们都为了共同的正义事业奔赴战场。这个道理是如此简单,要谈它似乎是多余的了。不过,有些最应该处处从这一点出发的人,偏偏最容易忘记这个最简单的道理。

历来争论着一个问题:人是不是为星期六活着?答案就在这里面的什么地方。想着靴子,想着被扔掉的小狗,想着偏僻小村子里的房子,痛恨夺去心头所爱的同志……这些思想多么渺小啊。可是,人生的实质就在这里面。

人与人是否联合,这种联合是否有意义,决定于是否能达到唯一的主要目的,这主要目的就是:为人们争取权利,做各自不同的人、各有特性的人,各人有各人独立的感情,都能独立地思考,独立地生活在世界上。

为了争取、保卫和扩大这一权利,人们必须联合起来。而这却产生了可怕的、很难打破的偏见:这种以民族、上帝、党、国家为名义的联合,说这是人生的目的,而不是手段。不对,不对,不对!为了人,为了人的微不足道的特性,为了使人拥有这些特性的权利—才是人在为生活而斗争中唯一、真正和永久的目的。

诺维科夫觉得他们能行,凭他们的力量、意志、智慧,能够在战斗中战胜敌人。这里面有大学生、十年级中学生,有旋工、拖拉机手、教师、电工、汽车司机,有性格暴躁的,有和善的,有倔犟的,有爱笑的,有喜欢唱歌的,有拉手风琴的,有谨慎的,有慢性子的,有莽撞的,这许许多多来自人民的小伙子的不可量度的智慧、勤劳、勇气、心计、本领、狠劲儿,他们的精神力量就要汇合到一起,合成一股力量,就一定能胜利,因为这股力量太大了。

他们或是这个,或是那个,或在中央,或在侧翼,或今天,或明天,一定会以自己的力量击溃敌人……战斗的胜利正是来自他们,他们在灰尘与硝烟中夺得胜利,只有他们能够思考、能够展开活动,冲锋和攻击比敌人早一点点儿、准确一点点儿,比敌人更乐观、更刚强。

一切都靠他们,这些驾驶坦克、操纵大炮和机枪的小伙子是战争的主要力量。

不过问题还在于所有这些人的精神财宝是否联结到一起,是否能汇成一股力量。

诺维科夫一遍又一遍地望着他们,可是心中有一股幸福的感觉,感觉有把握能得到一个女人的爱,这种感觉越来越强:“她一定会是我的,一定是我的。”

五十四

这是一些多么不平常的日子呀。

克雷莫夫觉得,历史书不再是书,而是进入了生活,与生活混合在一起了。

他感到天空和斯大林格勒的云彩颜色特别鲜明,照射在水上的阳光特别耀眼。这种感觉使他想起童年时候,那时候初雪的景致、夏日的雨点和彩虹都使他充满幸福的感觉。几乎所有的生灵,渐渐习惯了生活中的奇事,也就一年一年地渐渐失去这种奇妙的感觉。

克雷莫夫认为当代生活中一些错误和荒谬的情形,在斯大林格勒这里是感觉不到的。他想:“在列宁时期,就是这个样子的。”

他觉得,这儿的人待他很不一样,比战前一些人待他好些。他不觉得自己是时代的弃儿,依然像被包围时期那样。不久前他还在伏尔加河对岸很带劲儿地准备作报告,并且认为政治部调他做宣讲员是很自然的。

可是现在,他心里有时出现一种难堪的、受辱的感觉。为什么撤去他的战斗部队政委的职务?他干得似乎不比别人差,比很多人都强……

在斯大林格勒,人与人的关系都很好,在这块洒满鲜血的黄土坡上,处处可以感觉到平等和人的尊严。

在斯大林格勒,几乎人人都关心战后的集体农庄的体制问题和伟大的人民和政府之间将来的关系问题。红军的战斗生活,战士们拿着锹挖土,用菜刀刮土豆,或者拿着军营鞋匠使用的修鞋刀干活儿—似乎都和战后国内外人民的生活有直接关系。

几乎所有的人都相信,善良终将战胜。不吝惜自己鲜血的正直的人们一定能建设美好的、公道的社会。表露出这种感人的信心的人,认为自己未必能活到和平时期,每天都因为自己还能从早上活到晚上感到惊讶。

五十五

一天傍晚,克雷莫夫做过又一次报告之后,来到师长巴秋克中校的掩蔽所里。掩蔽所在马马耶夫冈的斜坡上,紧靠着班内山沟。

巴秋克的个头儿不高,一张被战争折磨得痛苦不堪的战士的脸。他见克雷莫夫来了,十分高兴。吃晚饭的时候,巴秋克的桌上摆了挺好的肉冻和滚热的面饼。巴秋克一面给克雷莫夫斟酒,一面眯起眼睛说:

“我一听说您来给我们作报告,就想您先到哪儿呢,先到罗季姆采夫那儿去,还是先到我这儿来。结果,您还是先到罗季姆采夫那儿去了。”

他哼哧两声,笑了笑:

“我们在这儿,就像住在乡下一样。到晚上一安静下来,就跟邻居们打电话聊天:你吃的什么,有谁上你那儿来啦,你要上谁那儿去,首长对你说什么来着,谁那儿澡堂好,报上报道什么人啦?报纸不报道我们,一个劲儿报道罗季姆采夫,从报上看,就好像只有他一个人在斯大林格勒作战。”

巴秋克拿好东西招待客人,自己却只是喝茶吃面包,看来他对好吃的东西不感兴趣。

克雷莫夫看到,那安详的动作和乌克兰式的缓慢语调,与巴秋克流露出来的一些不愉快的想法很不相称。克雷莫夫觉得难过的是,巴秋克没有向他提出任何一个与报告有关的问题。报告似乎没有接触到巴秋克真正关心的事。

巴秋克说了说战争刚开始时候的事,克雷莫夫听了十分吃惊。在大家都从边境撤退的时候,巴秋克率领自己的一团人向西开去,要堵住德国人的渡口。正在公路上向后撤退的高级首长却以为他是想向德国人投降。立即就在公路上进行审讯,所谓审讯就是骂娘和歇斯底里的喝叫,接着就下令把他枪毙。在最后一分钟,他已经站到一棵树跟前,手下的士兵把他抢了出来。

“是啊,”克雷莫夫说,“中校同志,情形很严重呀。”

“我的心脏没被打穿,”巴秋克说,“不过还是落得一点儿毛病,算我的成绩吧。”

克雷莫夫带着几分演戏般的语气说:

“听见雷恩卡的枪声吗?这会儿戈罗霍夫是在干什么事情吧?”

巴秋克侧眼看了看他。

“他干什么?大概是在玩捉‘傻瓜’。”

克雷莫夫说,他听说在巴秋克这里要开一个狙击手会议,他很有兴趣参加这个会议。

“噢,当然会有兴趣,怎么会没有兴趣。”巴秋克说。

他们谈起前线的情况。巴秋克担心的,是德国人夜里悄悄地在北段集结兵力。

等到狙击手们聚集在师长的掩蔽所里,克雷莫夫才知道这些烙饼是为谁准备的。这些身穿棉袄,又腼腆、又拘谨、又矜持的人纷纷坐到靠墙和桌子周围的长凳上。新来的人就像工人放下铁锹和斧头那样,轻轻地把步枪和自动枪放在角落里,尽量不弄出响声。

著名的神枪手扎伊采夫的脸很好看,像平常人一样,是一个可爱、温和的农村小伙子。但是等他转过头来,并且皱起了眉头,便露出十分刚强的相貌。

克雷莫夫想起战前偶然留下的一个印象:有一次,他在一个会上注视着自己的老朋友,忽然看到他那一向显得十分刚强的脸完全变了样子:眼睛眨巴着,鼻子耷拉下去,嘴巴半张着,再加上那小小的下巴,构成了一幅优柔寡断和懦弱的画像。

和扎伊采夫坐在一起的是迫击炮手别兹季科,窄窄的肩膀,一双深棕色眼睛总是带笑,还有一个是乌兹别克小伙子苏列伊曼·哈里莫夫,像小孩子一样撅着厚厚的嘴唇。炮兵狙击手马采古拉一个劲儿地拿手帕揩额头上的汗,他像一个拖家带口的人,他的性格似乎跟可怕的狙击方面的事没有任何共同之处。

来到掩蔽所里的其余的狙击手,有炮兵中尉舒克林,有托卡廖夫、曼茹里亚、索洛德基,全都像腼腆而羞涩的小伙子。

巴秋克向狙击手们询问着,低着头,很像一个好学的学生,而不是一个经验丰富、老谋深算的斯大林格勒战场上的指挥员。

当他和别兹季科说话的时候,所有坐在这儿的人的眼睛里都出现了快活的神气,似乎在等待好笑的事。

“喂,别兹季科,咋样?”

“昨个儿我闹得德国佬够呛,中校同志,您已经知道啦,今个儿早晨,我打死五个德国鬼子,用了四发迫击炮弹。”

“是啊,可这还比不上舒克林,他一门炮打了十四辆坦克。”

“他打一门炮,因为他的炮兵连就剩一门炮啦。”

“他打坏了德国佬的碉堡呢。”漂亮的小伙子布拉托夫说了一句,脸就红了。

“我觉得那不过是普通的掩蔽所。”

“是啊,掩蔽所,”巴秋克说,“今天一颗迫击炮弹把我的门打掉啦。”又转身朝着别兹季科,带着责备的口气用乌克兰语说:“打得这么准,我还以为是狗崽子别兹季科打的呢。”

特别腼腆的炮兵眀„准手曼茹里亚抓起一张饼子,小声说:

“中校同志,这面饼真好。”

巴秋克拿一颗子弹敲着茶杯,说:

“好啦,同志们,咱们言归正传。”

这是一次生产会议,就像工厂里、田野宿营地上常常召开的那种会议。但坐在这儿的不是织布工,不是面包工,不是裁缝,谈的也不是烤面包,不是打谷。

布拉托夫说,他看到一个德国人搂着一个女人在路上走着,他迫使他们趴到地上,在打死德国佬之前,让他们爬起来三次,后来又迫使他们趴下,子弹打得离他们的脚两三厘米的地方直冒烟。

“等他一站起来,我一枪把他打死,他就十字交叉倒在那女人身上了。”

布拉托夫懒洋洋地说着,他说得使人震惊,因为士兵们从来没有说过这样使人震惊的事。

“好啦,布拉托夫,不要胡吹。”扎伊采夫插话说。

“我没有胡吹,”布拉托夫不解地说,“今天我一共打死七十八个。政委同志决不会叫人胡吹,你瞧,这是他签的字。”

克雷莫夫本想加入谈话,很想说,在布拉托夫打死的德国人中可能有工人、革命者、国际主义者……应该记住这一点,要不然就会成为极端民族主义者。但是他没有说出口。因为这种思想对作战没有好处,不能武装军队,倒是会瓦解武装。

口齿不清、面色灰白的索洛德基说了说他昨天怎样打死八个德国佬。然后他又说:

“我是乌曼的集体农庄庄员,法西斯在我们村子里造了许多孽。我自己也流了一些血,受了三次伤。所以我不再做农民,做起了狙击手。”

愁眉苦脸的托卡廖夫说了说怎样选择好地点,监视德国人取水和去厨房必经的道路,然后又顺便说:

“我老婆来信说,很多人在莫扎伊城外被抓去杀了,我儿子也被杀了,因为我给他取了一个和列宁相同的名字—弗拉基米尔·伊里奇。”

哈里莫夫激动地说:

“我从来不着慌,等心定了,我才开枪。我来到前方,有个好朋友古罗夫中士,我教他说乌兹别克语,他教我说俄语。德国佬把他打死了,我打死十二个德国佬。我摘了一个军官的望远镜,挂在自己的脖子上:政治指导员同志,我是照你的吩咐做的。”

狙击手们创造的这些数字还是使人觉得震惊。克雷莫夫经常嘲笑神经衰弱的知识分子,嘲笑叶尼娅和维克托·施特鲁姆一听到富农分子在集体化时期遭殃就唉声叹气。他常常对叶尼娅说起一九三七年的事:

“消灭敌人并不可怕;可怕的是自己人杀自己人。”

现在他很想说说,消灭白党分子、孟什维克和社会革命党歹徒,以及消灭富农,他一向不手软,他对革命的敌人从没有任何恻隐之心,不过,在消灭法西斯的同时,把许多德国工人打死,不应该感到高兴。听着狙击手们的话,还是感到可怕,虽然他们都知道他们干这些事为的是什么。

扎伊采夫说起他很多天以来在马马耶夫冈脚下同一名德国狙击手的较量。德国狙击手知道扎伊采夫在注视着他,他也在注视着扎伊采夫。他们的本领大致相当,谁也没有打到谁。

“昨天他打倒了我们三个人,我坐在小棚子里,一枪也没有发,他最后一枪打出来,打中了,一名弟兄把胳膊一伸,侧着身子倒下了。他们那边走出来一个兵,手里拿着一摞纸,我坐着,看着……我明白,他知道这儿有狙击手,一定会打死他们那个兵,可是那个兵走过去了。我知道,他看不到他打倒的那个战士,他很想看一看。静了一阵子。又有一个德国佬提着水桶跑过去,我还是没有动。又过了十几分钟,他慢慢欠起身来,站了起来。我一下子站了起来……”

扎伊采夫沉浸在当时的情景中,在桌子旁边霍地站了起来,在他脸上闪现过的一种特别的、刚强的表情,现在成了他的唯一的、主要的表情,他已经不是一个和善的大鼻子小伙子,在他那鼓起的鼻孔、宽宽的额头、充满凌厉逼人的必胜神情的眼睛中,有一股狮子般的强硬而凶狠的杀气。

“他认出我来,明白了。我也开枪了。”

有一阵子鸦雀无声。昨天响过那一枪之后大概就是这样寂静,而且似乎听到了那个德国狙击兵倒下去的响声。巴秋克忽然朝克雷莫夫转过脸来,问:

“怎么样,感兴趣吗?”

“很好。”克雷莫夫只是回答了一声,再也没有说什么。

克雷莫夫留在巴秋克的掩蔽所里过夜。巴秋克咕哝着嘴巴,数着心脏病药水的滴数往杯子里倒,然后又往杯子里倒水。

他一面打着呵欠,一面对克雷莫夫说师里的事情,不是说战斗情况,说的是各种各样生活中的事。

克雷莫夫觉得,巴秋克说的一切,都和战争一开始巴秋克遭遇的那件事有关系,他的思想一直牵挂着那件事。

自从克雷莫夫来到斯大林格勒,就一直有一种奇怪的感觉。有时他觉得自己进入一块非党的天地里。有时恰恰相反,他觉得呼吸到了革命初期的空气。

克雷莫夫忽然问道:

“中校同志,您入党很久了吧?”

巴秋克说:

“怎么,政委同志,您觉得我掌握的路线不对头吗?”

克雷莫夫没有立即回答。他对这位师长说:

“您要知道,我是个还算不错的党的报告员,常常在工人大会上作报告。可是在这儿我一直有一种感觉:是别人在开导我,不是我开导别人。事情就是这么奇怪。是的,这就是谁掌握着路线,谁被路线掌握着。我本来想加入你们的狙击手们的谈话,进行一点纠正。可是后来我想,圣人面前夸学问,自讨没趣儿。不过说实在的,我没有插嘴,也不光是因为这一点。政治部就是要报告员使士兵们认识到,红军是复仇的军队。可是我却要从无产阶级立场谈什么国际主义。主要的是鼓起群众的愤怒来反对敌人嘛!要不然就会像童话里说的那个糊涂蛋一样:本来是来参加婚礼的,却念起追荐亡灵的经文……”

他想了想,又说:

“而且也是习惯……党一般都是鼓起群众的仇恨和愤怒,使他们去打击敌人,消灭敌人。在我们的事业中用不着基督式的人道主义。我们苏维埃的人道主义是严酷无情的……我们不讲客气……”

他想了想,又说:

“当然,我指的不是毫无根据就要把您枪毙那样的事。在一九三七年也常常有杀自己人的事,这些事是我们的不幸。现在德国人侵入工人和农民的国家,那就来吧!战争毕竟是战争!他们是罪有应得。”

克雷莫夫等待巴秋克说话,可是巴秋克没有作声,不是因为他听了克雷莫夫的话感到无法回答,是他睡着了。

五十六

“红十月”工厂的炼钢车间里,许多身穿棉军服的人在昏暗中来回穿梭,外面不时传来啪啪的枪声,火光乱闪,空气中硝烟弥漫,像灰尘,又像雾。

师长古里耶夫命令各团把指挥所设在几座炼钢炉里,这些炉子不久前还在炼钢。克雷莫夫觉得,这些坐在炼钢炉里的都是些特殊人物,他们的心确实是用钢铁打成的。

在这里已经能听到德国人皮靴的走动声。不仅听得到清晰的口令声,而且能听到轻微的咔嗒声和叮当声,那是德国人在给自动步枪上子弹。

当克雷莫夫缩着头爬进步兵团指挥所所在的炼钢炉炉口,他的手感触到几个月来尚未冷却、隐藏在耐火砖里的余热时,他突然感到有些胆怯—他觉得,伟大的抗战的秘密就要向他打开了。

他在昏暗中看到一个蹲着的人,看到他那宽宽的脸,听到那和悦的声音:

“瞧,客人上我们的皇宫里来啦,欢迎欢迎。快把酒拿来,再煎几个鸡蛋当下酒菜。”

在这又黑又闷、到处是灰尘的地方,克雷莫夫忽然产生一个想法:他永远不会对叶尼娅说,他钻进斯大林格勒的炼钢炉之后,是怎样想起她的。以前他一直想摆脱她,忘掉她。可是现在如果她寸步不离地照料他,他也由她了。即使这妖魔也爬进炼钢炉里来,他也不能躲着她了。

当然,一切都非常简单。谁需要时代的弃儿?他几乎成了残废,成了废物,成了吃退休金的人!她的离开,说明和证实了他这一生已经完全没有希望。就是在这里,在斯大林格勒,他也没有驰骋沙场,做点真正的事情……

这天晚上,克雷莫夫在炼钢车间里做过报告之后,和古里耶夫将军聊了起来。古里耶夫没有穿制服上衣,不时用手帕揩着红红的脸,用嗄哑的大嗓门儿向克雷莫夫敬酒,用同样的嗓门儿在电话里向各团团长发指示,用同样的嗓门儿训斥炊事员烤羊肉烤得不地道,并且给友邻部队师长巴秋克打电话,问他,在马马耶夫冈上是不是打到了山羊。

“咱们的人,总的说,都是快活人,都是好人,”古里耶夫说,“巴秋克是一个聪明男子汉,拖拉机场的若卢杰夫将军是我的老朋友。在‘街垒’工厂的古尔季耶夫上校也是一个很好的人,不过太像一个和尚,滴酒不沾。当然,我这样说不对。”

后来他就对克雷莫夫说起来,谁也不像他这样,战斗减员这样厉害,每个连队只有六至八人;敌人从他这里过河,比任何地方都难,有时从汽艇上撤下去的人有三分之一是负伤的。打得这样漂亮的,只有在雷恩卡的戈罗霍夫。

“昨天崔可夫把我的参谋长舒巴叫了去,因为他报告前沿阵地变动情况不大准确,所以我们这位舒巴上校无精打采地回来了。”

他看了看克雷莫夫,又说:

“您也许在想,我会骂娘了吧?”然后笑起来。“骂娘算什么?我天天骂他的娘。整个前沿阵地我都骂遍了。”

“是啊。”克雷莫夫拉长声音说。这个“是啊”的意思,显然,是人的尊严在斯大林格勒这块土坡上并不经常被看重。然后古里耶夫议论起报纸的作家们为什么写不好战争。

“这些狗崽子躲得远远的,什么也看不到,坐在伏尔加那边的大后方,在那里写。谁招待得好些,他们就写谁。瞧,列夫·托尔斯泰写的《战争与和平》。人们读了一百年,今后还要读一百年。为什么?因为他亲自参加,亲自战斗过,所以他知道应该写什么人。”

“对不起,将军同志,”克雷莫夫说,“托尔斯泰没参加过那一次卫国战争[56]呀。”

“‘没参加过’是什么意思?”将军问。

“意思很简单,就是没参加过,”克雷莫夫说,“和拿破仑打仗的时候,托尔斯泰还没有出生呢。”

“还没有出生吗?”古里耶夫反问了一遍。“怎么会没有出生呢?嗯?您是怎么算的?”

于是他们忽然很激烈地争论起来。这是克雷莫夫到这里作报告以来发生的第一次争论。他感到吃惊的是,他怎么也不能把对方说服。

五十七

第二天,克雷莫夫来到“街垒”工厂,古尔季耶夫上校的西伯利亚步兵师驻守在这里。

他越来越怀疑他的报告是不是有用。有时他觉得,大家听他的报告完全出于礼貌,就好像不信教的人在听老神甫布道。不错,大家都欢迎他来,但他明白,大家欢迎他,是出于人情,而不是欢迎他作报告。他也成了那些舞文弄墨、游手好闲妨碍别人战斗的军队政工人员之一。只有那些不询问、不解释、不做冗长的汇报、不进行宣传,而是参加战斗的政工人员,才是真正称职的。

他想起战前在大学里教马列主义的情形,像钻研宗教语录那样钻研《联共(布)党史简明教程》,他和学生们都觉得枯燥得要命。

但是在和平时期这种枯燥乏味的事属于常规,是免不掉的。在这里,在斯大林格勒,干这种事就很荒唐、没有必要了。这有什么意思呢?

克雷莫夫在师部的掩蔽所门口碰到古尔季耶夫,却没有认出这个瘦瘦的人就是师长,他穿着毡靴,披着不合身的士兵短大衣。

克雷莫夫在宽敞而低矮的掩蔽所里作报告。自从他到斯大林格勒以后,从来没有像这回这样猛烈的炮声。他只好一直不停地大声叫喊。

师政委斯维林是一个很会说话的人,声音洪亮,富于风趣。在报告开始之前,他说:

“为什么要限定听报告必须是高级指挥人员?来,地形测绘员同志们,警卫连没有事的战士们,不值班的电话员和通讯员同志们,都来听听国际形势报告!报告以后放电影。跳舞跳个通宵。”

他朝克雷莫夫挤了挤眼睛,好像在说:瞧,还是有办法的,这样对您对我们都很好。

克雷莫夫看到古尔季耶夫望着开玩笑的斯维林笑了笑,又看到斯维林帮着古尔季耶夫提了提披在肩上的大衣,发现这个掩蔽所里洋溢着一种很好的友谊气氛。

不过,斯维林眯起已经够小的眼睛,打量了一下参谋长萨夫拉索夫,萨夫拉索夫却带着很不悦很不满的表情气嘟嘟地朝斯维林看了一眼,于是克雷莫夫又了解到,在这个掩蔽所里,不光是友谊和同志气氛。

师长和政委听过报告以后,因为集团军司令员有急事找他们,很快就走了。克雷莫夫和萨夫拉索夫聊起来。看样子,这个人性格又乖僻,又暴躁,虚荣心又重,心胸又狭窄。他有许多地方很不好,如爱虚荣,暴躁,议论人时那种尖酸刻薄的嘲笑态度。

萨夫拉索夫望着克雷莫夫,滔滔不绝地说:

“在斯大林格勒,不论你到哪个团里去,都会看到在团里团长是老大,团长说了算数!这是对头的。在这儿不看大叔有几头牛,只看一点—看头脑……有头脑吗?有就好啦。用不着那些不管用的东西。可是在战前怎么样?”他笑嘻嘻地拿黄眼珠直盯着克雷莫夫的脸。“您要知道,我最讨厌政治。什么左倾啦,右倾啦,机会主义啦,理论家啦。我看不惯那些唱赞歌的人。可是,虽然我不问政治,还有十来次想把我干掉。好在我不是党员,不过有时说我酗酒,有时说我乱搞女人。怎么,要我装得一本正经?我不会。”

克雷莫夫想对萨夫拉索夫说,他克雷莫夫在斯大林格勒,命运也没有好转,依然荡来荡去,没有真正的事情可干。为什么罗季姆采夫师的政委是瓦维洛夫,而不是他呢?为什么党对斯维林比对他更信任呢?要知道,实际上他又聪明,目光又远,党的经验更丰富,也有足够的胆量,在必要的情况下,也有足够的狠心,手决不会发抖……而且,说真的,他们和他相比,只是刚开始识字的学生!……你们的时代过去啦,克雷莫夫同志,滚开吧。

这位黄眼睛的上校挑动了他的思绪,挑动了他的怒火,使他的心乱了。

天啊,还有什么疑问,他的一生垮了,日暮途穷了……当然,主要的不是叶尼娅看到他在物质方面毫无办法。她不在乎这个。她是一个纯洁的人。她不爱他啦!不走运的人、垮台的人是不会有人爱的。一个不荣耀的人。是的,是的,他已经被打入另册……再说,她纯洁是纯洁,物质条件对她也不是毫无意义的。比如,她就不会嫁给一个穷艺术家,哪怕她把他乱涂的画也看做天才的作品……

克雷莫夫有许多这一类的想法可以对这位黄眼睛上校说说,但他只能在心里赞同这一点,嘴上不能苟同。

“您怎么啦,上校同志,您把事情简单化了。战前也不光是要看大叔有几头牛。挑选干部也不是单凭业务能力。”

战争不让他们谈论战前的事情。轰隆一声爆炸的巨响,从硝烟与灰尘中冒出一名神情焦急的大尉。师部接到团里打来的电话,德国坦克朝该团团部开了火,德国步兵紧跟在坦克后面冲进了重炮营指挥人员所在的石砌楼房;指挥人员据守二楼,和德国人展开搏斗。坦克烧着了旁边一座木头楼房,伏尔加河上吹来的大风吹得火苗朝团长恰莫夫的指挥所直扑,恰莫夫和团部的人都呛得喘不上气,决定转移指挥所。但是,在炮火下,在对准了恰莫夫团的一挺挺重机枪的火力控制下,在大白天转移指挥所是很难的。

这一切同时发生在该师的防御地段上。有的请示对策,有的请求炮火支援,有的请求准许转移,有的在报告战况,有的要了解情况。每个人都有自己的事,所有的人只有一点是共同的,那就是都在操心生与死的问题。

等到多少安静下来,萨夫拉索夫向克雷莫夫问道:

“政委同志,趁师长和政委上司令部还没有回来,咱们是不是先吃饭?”

他不遵守师长和政委定的规矩,照样喝酒。所以他要单独吃饭。

“古尔季耶夫是很好的战将,”有些醉意的萨夫拉索夫说,“他有文化,忠实可靠,但有一点很糟:他是一个可怕的苦行僧!办起修道院来啦。可是我见了姑娘就馋得要命,像蜘蛛一样,粘住就不放,我就喜欢这种事儿。在古尔季耶夫面前,连个笑话都别想说。不过,跟他在一起配合作战,总的说还是很合拍子的。可是政委就很不喜欢我,虽然论天性他这个修道士跟我差不多。您以为,斯大林格勒使我老了吗?那是我这些朋友们老了。我在这儿却相反,倒是过好。”

“我也是政委这种类型的呀。”克雷莫夫说。

萨夫拉索夫摇了摇头。

“你又是,又不是。问题不在于这酒,而是在于这个……”

他先用手指头敲了敲酒瓶,然后又敲了敲自己的额头。

师长和政委从崔可夫的指挥所回来的时候,他们已经吃完了饭。

“有什么新情况吗?”古尔季耶夫打量了一下桌子,又快又严厉地问道。

“咱们的联络科长受伤了,德国人冲进来跟若卢杰夫打起来,恰莫夫和米哈廖夫的楼房被打着了火。恰莫夫被烟呛得够受,不过总的说,没什么特殊情况。”萨夫拉索夫回答说。

斯维林望着萨夫拉索夫喝得通红的脸,拉长了声音很亲热地说:

“上校同志,咱们喝吧,再喝点。”

五十八

师长向团长别廖兹金少校询问“6—1”号楼房的情况:是不是最好把人从里面撤出来?

别廖兹金建议师长不要把人撤出,虽然楼房有被包围的危险。楼房里有对岸炮兵部队的观测点,可以提供有关敌人的重要情况。楼房里还有一个工兵排,可以阻止敌人坦克的运动。敌人在消灭这个据点以前,未必会发动总攻,他们的活动规律是大家都清楚的。只要能得到一定的支援,“6—1”号楼房可以支持很久,就可以打乱德国人的部署。因为联络人员只能在夜间难得的时刻到达被困的大楼,电话线又一直无法修复,所以最好派一名无线电报话员过去。

师长同意别廖兹金的意见。夜里政治指导员索什金带领一组士兵进入“6—1”号楼房,给楼房守卫者带去几箱子弹和手榴弹。同时,索什金还将一位报话员姑娘和从联络点弄来的一部报话机带到了“6—1”号楼房。

政治指导员天快亮时返回团部,说守卫队队长拒绝写书面汇报,他还说:“我们没工夫搞这些乱七八糟的文字玩意儿,我们要报吿就向德国佬报告。”

“反正他们那儿一切都跟别处不一样,”索什金说,“大家都怕这个格列科夫,他跟他们称兄道弟,横七竖八地躺在一起,他也在他们中间,他们称他‘你’,喊他的小名。团长同志,那不是一个排的军人,是一群乌合之众。”

别廖兹金摇着头问道:

“拒绝写汇报?这个粗野汉子!”

后来,团政委皮沃瓦罗夫谈起一些指挥员的游击作风。

别廖兹金心平气和地说:

“游击作风怎么啦?有主动性,有独立性,很好。我有时候就在幻想:顶好我也落进包围圈里,暂时摆脱一下这些烦琐的公文游戏。”

“恰好,现在又要玩公文游戏了,”皮沃瓦罗夫说,“您要写一份详细的报告,我去交给师政委。”

师部里把索什金报告的问题当做一件严肃的事情来对待。

师长吩咐皮沃瓦罗夫搞一份有关“6—1”号楼情况的详细报告,并且要扭转格列科夫的思想。师政委马上向集团军军委委员和政治部主任汇报了这个政治思想上的严重问题。

对索什金报告的问题,集团军司令部比师里看得更为严重。师政委得到指示,要立即把被困的楼房里的问题抓一抓。担任集团军政治部主任的旅级政委向担任前总政治部主任的师级政委写了紧急报告。

报话员姑娘卡佳·文格罗娃夜里进入“6—1”号楼。早晨,她来见这座楼的头头儿格列科夫。格列科夫一面听这个有点儿驼背的姑娘的报告,一面凝视着她那慌乱、胆怯,同时又带有嘲笑神气的眼睛。

她的嘴很大,嘴唇的血色很淡。格列科夫等了好几秒钟,没有回答她的问题:“我可以走吗?”

在这几秒钟里,在他的头脑里出现了一些与军事无关的想法:“真的,很漂亮……腿也很好看……她还怕呢……看样子,是个娇生惯养的姑娘。她有多大,顶多十八岁。我的小伙子们可别跟她乱搞……”

在格列科夫头脑里闪过的这些念头,到末了忽然变成这样的想法:“在这儿谁说了算,谁在这儿闹得德国佬晕头转向?”然后他回答她的问话:

“姑娘,您上哪儿去?就陪着您的报话机好啦。咱们有办法。”

他用手指头敲着报话机,侧眼看了看天上,德国轰炸机在天上吼叫着。

“您是莫斯科来的吧,姑娘?”他问道。

“是的。”她回答说。

“您请坐,我们这儿很随便,不讲究。”

姑娘朝一旁走去,碎砖块在她的靴子下面咯吱咯吱响着,阳光照在机枪筒上,照在格列科夫缴来的黑黑的手枪上。她蹲下来,看着堆在断墙脚下的军大衣。有一会儿她觉得很奇怪的是,这情景她怎么一点也不感到奇怪。她知道,对着墙豁口的机枪是“杰格佳廖夫”型的;知道缴获的“瓦尔德”式手枪弹夹里装八颗子弹,知道这种手枪发射力强,但准确性差;知道堆在角落里的大衣是死者留下的,知道死者都埋得不深,因为焦土气味中混杂着一种她已经闻惯了的气味。昨天夜里交给她的报话机跟她在科特卢班冈脚下使用的报话机差不多,接收刻度盘一样,开关也一样。她想起她在野外的时候,眼睛盯着电流表上蒙了尘土的玻璃,不住地撩着从船型军帽里溜出来的头发。

谁也不和她说话,这楼房里的狂暴而可怕的生活似乎跟她无关。但是在一个白头发的人(她从别人的话里知道他是迫击炮手)骂了几句脏话的时候,格列科夫便对他说:

“老爹,这像话吗?这儿有咱们的姑娘。说话要规矩点儿。”

卡佳打了一个寒噤,不是因为老头子的脏话,而是因为格列科夫的目光。

她感觉出来,虽然大家都不和她说话,可是她的到来,使楼房里气氛紧张了。似乎她的皮肤都感觉出周围的紧张气氛。即使在俯冲轰炸机啸叫,炸弹在很近的地方爆炸,碎砖乱飞的时候,这种气氛依然存在。

她对轰炸,对炮弹片的啸声总算有点儿习惯了,不怎么慌张了。可是她在感到男人们火辣辣地盯着她时产生的感觉,依然常常使她心慌意乱。昨天傍晚电话员姑娘们就可怜起她来,说:“哎呀,你到那里面才可怕呢!”

夜里,一名通信员把她带到团部。在这儿已经特别感到敌人的接近、生命的脆弱。人似乎成了极容易打碎的东西,这会儿还在,过一会儿就没有了。

团长很伤心地摇了摇头,说:

“怎么能把孩子们送到前线来?”

过一会儿,他说:

“别怕,好孩子,如果有什么情况,就通过报话机直接向我报告。”

他说这话的语调那样和善,那样亲热,卡佳听了差点儿掉下泪来。

然后另一名通信员把她带到营部。那儿在放留声机,红头发的营长请卡佳喝酒,并且请她在《中国小夜曲》的乐曲声中和他一起跳舞。营里有一种恐怖的气氛,卡佳觉得,营长喝酒不是为了快活,而是为了压一压承受不了的恐怖,忘记自己像玻璃一样易碎。

这会儿,她坐在“6—1”号楼里一堆碎砖上,不知为什么并不感到恐怖,而是在想着自己童话般美好的战前生活。

被困在楼房里的官兵显得特别坚定,有信心,他们这种信心很能感染人。著名的医生、轧钢车间的熟练工人,剪裁贵重呢料的剪裁师,救火队员,在黑板前讲课的老教师,都有这种令人心安的自信。

战前,卡佳觉得自己注定要过不幸的生活。战前,她认为女伴们坐公共汽车是摆阔气。她觉得就连平民饭馆里走出来的都是很不平常的人,有时她跟在从平民饭馆里涌出来的人群后面,听他们说话。有一次她放学后回到家里,很得意地对妈妈说:

“你可知道今天怎么啦,同学请我喝果汁汽水,真正的果汁,味道就像真正的黑醋栗。”

妈妈每月工资四百卢布,扣除所得税和文化税,扣除建设公债,她们靠剩下的几个钱生活是很不容易的。她们不添置新东西,把旧衣服改了穿,邻居们凑钱雇女工玛露霞打扫公用的地方,她家不参加,轮到她家打扫的日子,卡佳就擦地板,倒垃圾桶。她家的牛奶不请人送,而是到国营商店去取,每天要排很长时间的队,但这样每月可以节省六卢布;有时国营商店不供应牛奶,卡佳妈妈傍晚时候就到市场去买,卖牛奶的因为急着要赶火车,卖的价钱比早晨便宜,几乎和国家的价钱一样。她们从来不坐公共汽车,因为票价太贵,有时如果要走很远的路,她们就坐电车。卡佳也不上理发馆,妈妈自己给她理发。衣服当然都是自己洗,用的电灯也很不亮,只比公用场地的电灯多少亮一点点儿。她们做饭要做够三天吃的。她们一般都是用菜汤下饭,有时候素油炒饭,有一次卡佳喝了三碟子菜汤,就说:“嘿,今天我家吃三个菜了。”

妈妈不提她们跟爸爸在一起时是怎样生活的,那时候的事卡佳已经不记得了。只是有时候,妈妈的好友薇拉·德米特里耶芙娜看到她们母女做饭,会说一句:“啊,我们当年也有过好日子。”

可是妈妈一听就生气,所以她们过去究竟怎么样,薇拉·德米特里耶芙娜也不多说。

有一次卡佳在衣柜里发现爸爸的一张照片。她是第一次在照片上看到他的面孔,好像有人悄悄告诉她什么,她马上就明白了,这是她爸爸。照片背面写着:“莉达:我生在穷家,我们相亲相爱,死而无怨。”她什么也没有对妈妈说,但是放学回来,常常拿出照片,对着爸爸那黑黑的,她觉得似乎很忧伤的眼睛看上很久。

有一天她问:

“现在爸爸在哪儿?”

妈妈说:

“不知道。”

等到卡佳要参军了,妈妈才第一次跟她谈起爸爸,卡佳才知道爸爸在一九三七年被捕,知道他再婚的事。

她们一夜没有睡,谈了一夜。什么都谈。一向善于隐忍的妈妈跟女儿谈了丈夫怎样把她抛弃,谈她怎么嫉妒,怎样受辱、受欺负,谈她的爱、她的怜惜心。卡佳感到十分惊讶:人的心灵世界竟有这样广大,相形之下,轰轰烈烈的战争简直算不上什么了。早晨,她向妈妈告别。妈妈把卡佳的头搂到自己怀里,把背包给她套到两肩上。卡佳说:

“妈妈,我也是生在穷家,我们相亲相爱,死而无怨……”

后来妈妈轻轻推了推她的肩膀,说:

“该走啦,卡佳,走吧。”

于是卡佳走了,就跟此时此刻成千上万的年轻人和成年人一样,她离开了妈妈,离开了家,也许从此不再回来,也许回来已成了永远告别了自己的不幸而可爱的童年时代的另一个人。这会儿她在斯大林格勒,跟这座楼里的头头儿格列科夫坐在一起,望着他的大头,望着他的厚嘴唇和阴沉的脸。

五十九

她来的第一天,有线电话接通了。

这位无线电报话员姑娘因为老半天无事可干,再加上还没有和“6—1”号楼里的人打成一片,所以格外苦闷。

但是,来到“6—1”号楼里的这第一天,为她接下来的生活做了很多准备。

她了解到,在打得残破不堪的二楼设有炮兵观测点,可以向对岸发送情报,二楼的头头儿是一名中尉,穿着肮脏的军装,戴的眼镜老是从翘鼻子上往下溜。

她了解到,那个爱发火、爱说脏话的老头子是从民兵里来的,因为自己有了迫击炮长的称号,感到很神气。在高墙与一堆碎砖之间的那些人是工兵,其中的头头儿是一个胖子,走起路来皱着眉头,嘴里咯咯响,好像脚上长了鸡眼。

掌管楼房里唯一一门大炮的是一个穿水兵服的秃子。他姓科洛密采夫。卡佳曾经听到格列科夫喊他:

“喂,科洛密采夫,你睡过头啦,把天大的好事儿耽误了。”

掌管步兵和机枪的头头儿是一名浅色胡子的少尉。他的脸虽然有一圈胡子,却显得特别年轻,也许他自己以为,留胡子可以显得有三十岁,像个上了年纪的人。

下午,大家拿东西给她吃。她吃面包,就羊肉灌肠。后来她想起军装口袋里还有水果糖,便悄悄地把一块糖放进嘴里。吃过东西以后,她就想睡觉,虽然四周枪声很近。她睡着了,在睡梦中依然咂摸着糖,依然很烦恼、苦闷,等待着灾难降临。忽然她听到唱歌的声音。她没有睁眼睛,字字都能听得很清楚:

往日的伤心事在我胸怀,

像酒,越陈越厉害……

在夕阳的余晖照亮的石头天井里,站着一个肮脏的、头发蓬乱的小伙子,手里拿着一本小书。红色的碎砖堆上坐着五六个人,格列科夫躺在大衣上,拿拳头支着下巴。有一个像格鲁吉亚人的小伙子在听着,露出不信任的神气,好像在说:“算啦,别想拿这一套收买我。”

附近有一颗炮弹爆炸,冒起一团红红的砖灰,似乎这团团乱转的是童话里的烟雾,坐在红色砖堆上的人和他们在红雾里的武器,似乎是在《伊戈尔远征记》[57]里描写的那个可怕的时日。姑娘的心忽然颤抖起来,因为她产生了一种荒唐的信心,相信有幸福等待着她。

第二天。这一天发生了一件事,惊动了已经习惯了一切的楼里的人们。

二楼的负责人是巴特拉科夫中尉。他手下有一名测绘计算员和一名观测员。一个是垂头丧气的兰巴索夫,一个是机灵而忠厚的蓬丘克。蓬丘克是一个很古怪的、一天到晚自己对着自己笑的戴眼镜的中尉。卡佳一天能看到他们好几次。

在安静的时候,从楼板上的豁口能在下面听见他们的声音。

兰巴索夫在战前养过鸡,常常和蓬丘克谈起鸡的聪明和狡诈的本性。蓬丘克趴在炮队镜上,像唱歌一样拉长声音报告着:

“注意:从面包厂方向开来一队汽车……中间有一辆坦克……出来的德国佬有一营人……像昨天一样,有三个地方冒炊烟,一些德国佬带着锅盆……”

他观察到的一些情况有时没有什么军事意义,只是一些生活趣事。这时候他就唱:

“注意……一个德国军官带一条狗出来玩啦,狗闻到什么味道,朝前跑啦,好像那是一条母狗,那公狗站住,在闻呢。那边有两个德国兵,一个掏出烟盒,抽起烟来,另一个直摇头,好像是说:我不抽……”

忽然蓬丘克用同样的唱歌的腔调报告说:

“注意……操场上有很多人……有人拿着乐器……很多人围着他们,还堆了很多柴……” 后来,他停了很久,又用十分难受但是仍然拉得很长的声音说:“注意,中尉同志,拉出一个女人来,女人穿着小褂,在叫呢……把女人捆在柱子上啦……注意,中尉同志,又拉出一个小孩子,也捆在柱子上啦……中尉同志,好像两个德国佬在从桶里往外倒汽油……”

巴特拉科夫通过电话把这一情况通知了对岸。

他趴在炮队镜上,用自己的卡卢加地方口音,学着蓬丘克的语调,大声叫道:

“喂,注意,同志们,乐队在烟火里演奏呢……开火!”

他厉声喊叫起来,并且转过身朝向对岸。

但是对岸没有动静……过了几分钟,重炮团集中火力猛轰行刑的地方。操场被一团团硝烟和灰尘罩住。

几个小时之后,通过侦察员克里莫夫了解到,那是德国人要烧死一个茨冈女子和一个小孩子,因为怀疑他们从事间谍活动。头天晚上,克里莫夫把两件脏衣服和裹脚布留给一个老太婆,说定第二天去取洗好的衣服。他想向老太婆了解一下茨冈女子和小孩子的情况—是苏军炮弹把他们打死了呢,还是被德国人烧死了。老太婆是跟孙女和一头山羊一起住在地窖里的,克里莫夫穿过瓦砾堆顺着他还记得的小路朝前爬去,可是苏军夜间轰炸机在地窖所在的地方扔下一颗重磅炸弹,老太婆、小孙女、山羊、克里莫夫的衣服和裹脚布全不见了。他只是在炸裂的木头和石灰碎块之间发现一只肮脏的小猫。小猫很老实,既没有什么要求,又不抱怨,认为这轰炸声、饥饿和战火是世间正常的事情。

克里莫夫一直不明白,为什么自己一下子把小猫装进衣服口袋里。

“6—1”号楼里人与人之间的关系使卡佳感到吃惊。侦察员克里莫夫在向格列科夫报告的时候,不是按规矩站着,而是跟他坐在一块儿,他们说话就像同志跟同志说话。克里莫夫抽烟就找格列科夫借火。

克里莫夫报告完了之后,走到卡佳跟前,说:

“姑娘,瞧,世界上有些事儿多可怕呀。”

她叹了一口气,感觉到他那火辣辣的眼睛在望着她,顿时脸红了。

他从口袋里拿出小猫,放在卡佳身边的碎砖上。

这一天有十来个人走到卡佳跟前,他们都和她谈小猫,谁也没有提起那个茨冈女子的事,虽然那件事使他们心里很不安宁。有些人想坦率地跟卡佳谈谈感情问题,谈起来却用的是嘲弄和粗暴的口气。有些人干脆利落想跟她睡睡觉,谈起来却十分客气,彬彬有礼。

小猫哆嗦起来,浑身都在颤抖,看样子,是受了震伤。

老迫击炮长皱着眉头说:

“干脆把它打死好啦。”

可是他马上又说:

“你还是逮逮它身上的虼蚤吧。”

另一名担任迫击炮手的黑红脸膛的民兵琴佐夫劝卡佳:

“姑娘,把这讨厌东西扔掉吧。要是西伯利亚猫就好啦。”

工兵里亚霍夫薄薄的嘴唇,阴沉着脸,一脸凶相。只有他真正对猫感兴趣,而对报话员姑娘的美貌无动于衷。

“我们在野外的时候,”他对卡佳说,“有沙沙声冲我来,我想,这是要落地的子弹。谁知是一只兔子。它一直跟我坐到天黑,等到安静了,它才走了。” 他说:“您虽然是姑娘,可还是知道这是侦察机在伏尔加河上飞,在打一百八十毫米的炮,在打火箭炮。兔子却很傻,什么也不知道。分不清迫击炮和榴弹炮。德国佬放照明弹,兔子就吓得打哆嗦,又没法儿给它解释。所以这些畜生都很可怜。”

她感到对方是严肃的,所以也很严肃地回答说:

“我不完全同意您的说法。比如说,狗就能认得飞机。我们驻扎在一个村子里,那儿有一条狗叫‘凯尔逊’,我们的飞机来了,它躺在那儿,连头也不抬,可是敌机一来,它立刻就找地方躲起来。它分得才清楚呢。”

空气抖动起来,因为空中响起可怕的刺耳响声,这是德国的十二筒火箭炮开炮了。炮弹轰鸣,黑烟和红砖灰混合到一起,石块到处乱飞。过了一分钟,等到灰土渐渐落下来,卡佳和里亚霍夫又继续他们的谈话,就好像他们不曾趴到地上。显然,被困孤楼里的人们的自信心也传染了卡佳。似乎他们都相信,在被打成了瓦砾场的楼房里,一切一切,包括钢铁和石头,都很脆弱,都很容易打碎,只有他们是例外。

一排机枪子弹呼啸着从他们坐的豁口旁边飞过,紧接着又是一排子弹。里亚霍夫说:

“春天我们驻扎在圣山城外。头顶上常常有子弹的啸声,却听不见枪响,真叫人莫名其妙。原来,那是椋鸟学会了模仿子弹的声音……我们有一位上尉连长也常常弄得我们惊慌起来,他学子弹声音才像呢。”

卡佳说:

“我在家里的时候就想象战争是什么样子:孩子们在哭叫,大家都在火里,猫在乱跑。我来到斯大林格勒一看,果然就是这个样子。”

一会儿,留大胡子的祖巴廖夫走到卡佳跟前。

“怎么样,”他关切地问,“长尾巴的小家伙还活着吗?”他掀起盖在猫身上的一块裹脚布。“噢,多么可怜呀,多没精神呀。”他嘴里说着,眼睛里露出馋涎欲滴的神气。

晚上,在短时间的战斗之后,德军向“6—1”号楼的侧翼推进了一小段距离,用机枪火力切断了楼房与苏军防御阵地之间的道路。通往步兵团团部的电话线也被切断了。格列科夫下令打一条通道,从地下室通向离楼房不远的一条地道。

“有炸药。”肥胖的司务长一只手端着茶缸,另一只手拿着一小块糖,对格列科夫说。

楼房里的一些人很随便地坐在基墙边的一个大坑里,说着话儿。大家都很忿怒地想着烧死茨冈女子的事,但是依然没有谁说起这事。似乎这些人对身陷重围这事漠不关心。

卡佳觉得这种镇静非常奇怪,但是这镇静却很能征服人,在这些十分自信的人中间,就连可怕的字眼“被围”,她觉得也不可怕了。等到机枪就在旁边嗒嗒响起来,格列科夫高喊“打呀,打呀,他们来啦”的时候,她也不怕了。等到格列科夫说“想用什么就用什么。手榴弹,刀,铁锹。打,打,狠狠地打”的时候,她也不害怕了。

在安静的时候,楼房里的人就详细地、不慌不忙地讨论起姑娘的相貌。巴特拉科夫似乎不是这方面的行家,而且是近视眼,然而在讨论卡佳的美的时候常常提出很精到的见解。

“我认为姑娘的胸脯是最要紧的。”他说。

炮兵科洛密采夫和他争论,他就像祖巴廖夫说的,“发表长篇论文”。

“喂,你们好像谈起猫来啦?”祖巴廖夫问。

“不行吗?”巴特拉科夫说。“就连老头子还拿人当猫谈呢。”

老迫击炮长吐了一口唾沫,拿手掌搓着胸脯,说:

“都说这姑娘很漂亮,她的漂亮究竟在哪儿?你们说说看。”

他听到有人暗示说,格列科夫很喜欢这姑娘,特别生气。

“依我看,这个卡佳实在不咋样,经不住细看。两条腿那样长,跟仙鹤一样,屁股没有屁股。眼睛老大,像牛眼睛,这算什么姑娘?”

琴佐夫反驳说:

“你就喜欢大屁股娘们儿。你这是老眼光,是革命以前的眼光。”

科洛密采夫专爱说脏话、下流话,那老大的秃头里装着许多古怪的想法,灰灰的眼睛笑嘻嘻地眯缝着,他说:

“这姑娘还是不错的,不过我有我的特别胃口。我喜欢小小的,像亚美尼亚和犹太妞儿那样的,大眼睛,短头发,又灵活,又麻利。”

祖巴廖夫若有所思地望了望被探照灯光划破的黑黑的天空,低声说:

“还不知道这事儿究竟怎么样呢。”

“你是说,她究竟喜欢谁吗?”科洛密采夫问。“她喜欢格列科夫,这是肯定的。”

“不,不一定。”祖巴廖夫说过这话,从地上拿起一块断砖,使劲扔到一边。

大家看了看他,看了看他的大胡子,一齐哈哈笑了起来。

“你凭什么叫她喜欢,凭大胡子?”巴特拉科夫问道。

“凭唱歌!”科洛密采夫说。“现在广播:有步兵要唱歌啦。他唱,她就把他的歌声广播出去。恰好是一对儿!”

祖巴廖夫打量了一下昨天晚上念诗的小伙子。

“你怎么样?”

老迫击炮长用争吵的口气说:

“他不说话,就是说,他不愿说话。”

又用父亲责备儿子不该听大人说话的口气说:

“你顶好到地下室里去,趁这会儿安静,好好地睡一会儿。”

“这会儿在地下室里安齐费罗夫准备用炸药炸通道呢。”巴特拉科夫说。

这时候格列科夫在口述报告,由卡佳向外发送。

他向集团军司令部报吿说,据各方面观察,德军正准备进行突击,据各方面情况判断,这次突击方向是拖拉机工厂。他只是没有报告,据他判断,他和手下弟兄们所据守的楼房正是德军突击目标的中心。但是看着姑娘的脖子,看着她的嘴唇和耷拉着的睫毛,他想象到,而且是活灵活现地想象到,这细细的脖子断了,像珍珠一样白的颈脊骨从破烂了的皮肤里露了出来,这玻璃球般大眼睛上的睫毛和没了血色的嘴唇都像是用落满尘土的灰色橡胶做成的了。

他真想抱住她,趁他和她都还活着,还没有被消灭,趁这个年轻姑娘还是这样美,他要享受一下她的温暖、她的青春活力。他觉得,单是因为他对姑娘的怜悯,也要把她抱住,但是,血液在耳朵里腾腾直跳,朝两边鬓角直冲,难道是因为怜悯吗?

司令部没有马上回答。

格列科夫伸了个懒腰,骨头舒舒服服地响了几声,大声地舒了一口气,心里想:“好的,好的,等天黑了再说。”接着又很亲热地问道:

“克里莫夫带回来的小猫怎么样啦,好些了吗,结实了吗?”

“哪儿会结实。”卡佳回答说。

卡佳一想到茨冈女子和小孩子在火里的情形,她的手指头就发抖,她侧眼朝格列科夫看了看,看他是不是发觉这一点。

昨天她觉得,“6—1”号楼里的人谁也不会跟她说话的,可是今天在她吃饭的时候,有一个手持自动步枪的大胡子从她身边跑过,像老朋友一样对她喊道:

“卡佳,多吃多长肉!”并且用手比划着,怎样拿调羹在饭盒里吃饭。

她看到昨天念诗的那个小伙子用防雨布搬迫击炮弹。还有一次,她一回头又看到他,他站在开水锅边,她知道他是在看她,所以她打量了他一下,他却赶紧转过脸去。

她已经在猜想,明天谁会拿信和照片给她看,谁会叹着气一声不响地看她,谁会对她说他不相信女人的爱情,今后再也不谈恋爱,谁会给她送礼物,给她半壶水或一把白糖。那个大胡子步兵可能会爬过来摸她。

终于司令部回答了,卡佳把司令部的话转告格列科夫:

“命令你们每天十九时正进行详细汇报……”

忽然格列科夫打了一下她的手,把她的手掌从开关上拨下来,她吓得叫起来。他笑了笑,说:

“一块炮弹皮落在报话机上啦,什么时候格列科夫需要,再把报话机修好。”

姑娘慌乱地看着他。

“请原谅,亲爱的卡佳。”格列科夫说着,抓住她的手。

六 十

凌晨时候,别廖兹金团部向师部报告说,被困在“6—1”号楼里的人打通了与工厂的水泥地道相接的地下通道,进入了拖拉机厂的车间。师部值班参谋将此事报告了司令部,司令部里的人报告了克雷洛夫将军,克雷洛夫命令找一个楼里出来的人到他这儿来,以便查问有关情形。值班参谋便挑了一个小伙子,由联络官领着朝司令部走去。他们顺着山沟朝岸边走,小伙子一路上眼睛转来转去,不住地问这问那,心里很不踏实。

“我要回去。我只是为了把地道摸清楚,好把伤员抬出来。”

“没关系,”联络官回答说,“你现在去见的官比你们的官大,他怎样吩咐,你就怎样做好啦。”

在路上,小伙子对联络官说,他们已经在“6—1”号楼里蹲了两个多星期,有些天他们只能吃堆在地下室里的一些土豆,喝水就喝暖气锅炉里的水,把德国人弄得够呛,德国人几次派人来谈判,说要把被围困的人放出来,放到工厂里去,可是,大楼里的指挥员(小伙子管他叫“楼长”)命令所有的火器一齐开火,算是对他们的回答。等他们来到伏尔加河边,小伙子趴下,喝起水来,等喝足了水,又把棉袄上的水滴小心地刮到手心里,拿舌头舔了舔,就像饥饿的人舔面包碴儿一样。他说,暖气锅炉里的水都臭了,头几天大家喝了那水都闹肚子,楼长吩咐把锅里的水烧开了再喝,这样就不闹肚子了。

然后他们一声不响地又往前走。小伙子倾听着夜间轰炸机的隆隆声,望着红的绿的信号弹和一道道子弹与炮弹曳光装饰得色彩缤纷的天空。他看了看尚未熄灭的市区大火那疲惫无力的火苗,看了看大炮发射时的白光和重型炮弹在伏尔加河里爆炸掀起的青色浪花,不禁渐渐放慢了脚步,直到联络官喊他:“走吧,走吧,快点儿!”

他们在岸边乱石丛里走着,一颗颗迫击炮弹在头上呼啸而过,岗哨不时地呼喊他们。后来他们顺着一条小路朝坡上走,经过弯弯曲曲的巷道,经过一座座挖进土山里的掩蔽所,一会儿走在黄土台阶上,一会儿走过木板搭的小桥,到末了来到一个拉了铁丝网的通道口—这便是第六十二集团军指挥所。联络官紧了紧腰带,便顺着交通壕朝军委掩蔽所走去,用来造掩蔽所的圆木特别结实。

哨兵去找副官。有一小会儿,从半开着的门里射出柔和的电灯灯光,那是一盏带灯罩的台灯。

副官打了一下手电,问过小伙子的姓名,便吩咐他等一会儿。

“等会儿我怎么回去呀?”小伙子问道。

“没关系,有嘴巴,就不怕迷路。”副官说过这话,又用严肃的口气说:“你们到门道里来,要不然挨了迫击炮弹,将军还要我负责任呢。”在暖和而昏暗的过道里,小伙子坐在地上,侧着身子往墙上一靠,就睡着了。

有一只手使劲把他摇晃了两下。他正迷迷糊糊地做着梦,在梦里既听到若干天来战场上凄惨的叫声,又听到早已不存在的自己家里的柔声细语,这时候一个很严厉的声音闯入他的梦境:

“沙波什尼科夫,快去见将军……”

六十一

谢廖沙·沙波什尼科夫在司令部警卫队的掩蔽所里过了两个昼夜。司令部的日子使他感到苦闷,他觉得这儿的人一天到晚没有事干,闲得难受。

他想起战前他怎样和奶奶一起在罗斯托夫等了八个钟头,等待开往索契的火车,他觉得今天的等待很像那一次等待换车。后来他觉得,把去“6—1”号楼比作去索契疗养院,简直好笑。他要求司令部少校警卫队长放他走,但是警卫队长没得到将军的指示,不敢让他走。将军把沙波什尼科夫叫去后,只问了两个问题,就中断了谈话去接电话了。警卫队长决定暂时不让小伙子走。说不定将军还要再叫他去呢。

警卫队长一走进掩蔽所,就看到小伙子看着他,便说:

“好的,我记着。”

有时候小伙子恳求的目光使他生起气来,他就说:

“你在这儿有什么不好?有什么好吃的,给你吃什么。这儿又暖和。干吗要急着回去叫人家打死?”

当一天到晚炮火连天,一个人整个沉入战争的大锅里的时候,他往往无法理解、无法看到自己的生活;他需要朝旁边哪怕跨上一步。这时就像站到了岸上,能看到整条大河,就会想:难道我刚才就在这疯狂的水里,在浪涛里游过来的吗?

谢廖沙觉得原来在民兵团里的那段生活是很平静的:夜晚在黑沉沉的草原上放哨,远方天空闪着火光,民兵们在闲聊。

总共只有三个民兵进入拖拉机厂的居住区。波里亚科夫很不喜欢琴佐夫,说:

“整个民兵团就剩下一老一小,再加一个糊涂虫。”

“6—1”号楼里的生活遮没了过去的一切。尽管这种生活是令人难以想象的,但却是唯一的现实,而过去的一切都成了虚幻。只是有时候在夜里,脑海里出现奶奶那灰白的头,出现姑姑叶尼娅那带笑的眼睛,一向被慈爱浸润着的心就紧缩起来。

进入“6—1”号楼的头几天,他心里想:如果格列科夫、科洛密釆夫、安齐费罗夫等人忽然闯入他的日常生活,那会是十分奇怪和荒诞的。可是他现在有时候却觉得,如果他的姑姑们、他的表妹和姑父维克托闯入他今天的生活,那就太可笑了。

啊,奶奶听到谢廖沙这样会骂娘,准会吓一大跳……

格列科夫!

真不明白,是专门挑选了一些稀奇特别的人到“6—1”号楼里来,还是一些普通人一进这座楼就变得很特别了……

民兵队长克里亚金如果在这儿当领导,一天也干不了。还有琴佐夫,虽然大家都不喜欢他,却依然待下去了。但是他已经不像在民兵团里那样,已经改掉了行政机关的习性。

格列科夫!真是个刚强、勇敢、威风,却又那么平常的奇妙人物。他记得战前小孩子穿的鞋什么价钱,清洁工和钳工拿多少工资,在他叔叔所在的集体农庄里每个劳动日能分到多少粮食和钱。

有时他谈起战前军队里的清洗,谈起授军衔的情形,谈起分配住房时怎样走后门,还谈到在一九三七年有些人写了几十次秘密报告,揭发臆造的人民敌人,因而得到将军官衔。

有时候,他的力量似乎在于他的狮子般的勇猛,在于他天不怕地不怕的乐观,他就是那样天不怕地不怕地从墙豁口里跳出去,高声喊着“狗杂种们,叫你们尝尝厉害的!”拿手榴弹朝攻上来的德国佬扔去。有时候,他的力量又似乎在于他的纯朴随和,在于跟大楼里的人们的友谊。

他在战前的生活没有什么引人注目的地方。他在矿业中学上过十年级,后来当建筑技术员,后来成为驻扎在明斯克附近的一支部队的步兵大尉,在野外和军营里指导操练,进过明斯克的训练班,晚上看书,喝酒,看电影,和朋友们打牌,和妻子吵嘴,妻子吃醋完全是有根据的,因为他和当地许多大姑娘小媳妇有关系。这一切都是他自己说的。于是他一下子在谢廖沙的心目中,而且不只是谢廖沙的心目中,成为英雄,成为敢做敢当的好汉。

谢廖沙周围来了许多新人,挤走了他心中最亲近的人。

炮兵科洛密采夫原是基干水兵,在军舰上服务,三次在波罗的海落水。

谢廖沙很喜欢科洛密采夫常常用鄙夷的口气谈起那些不能用鄙夷的口气议论的人,而对学者和作家却表现出不同一般的尊敬。在他看来,所有当官的,不论是什么职位和头衔,跟秃顶的洛巴切夫斯基[58]或者病歪歪的罗曼·罗兰相比,都不算什么。

有时科洛密采夫谈起文学。他完全不像琴佐夫那样谈文学的教育意义和爱国主义。他很喜欢一位作家,不知是美国的,还是英国的。尽管谢廖沙从来没有读过这位作家的作品,科洛密采夫也忘记了这位作家的名字,但是谢廖沙相信他的作品很好,因为科洛密采夫常常津津有味、兴高采烈地夸奖他的作品,而且高兴得直骂娘。

“我为什么喜欢他?”科洛密采夫说。“因为他不教训我。男子汉找娘们儿,找娘们儿就是找娘们儿;当兵的喝醉了,喝醉了就是喝醉了;老头子的老伴儿死了,都写得实实在在。又好笑,又可怜,又有趣,反正不知道人为什么活着。”

侦察员瓦夏·克里莫夫和科洛密采夫很要好。

有一次谢廖沙和克里莫夫潜入德军阵地,爬过铁路路基,爬到德国炸弹炸出的一个大坑边,坑里坐着德军一挺重机枪的几个机枪手和一名观测军官。他们贴在坑边上,观看德国兵的生活情形。一个小伙子解开上衣,把一块红方格手帕塞到衬衣领子里,刮起胡子。谢廖沙听到那沾满灰尘的硬扎扎的胡子在剃刀底下哧啦啦直响。另一个德国兵在吃扁平罐头盒子里的食品,谢廖沙在很短的一瞬间望着他的大脸,那张脸上流露出心满意足的神情。那名观测军官在上手表。谢廖沙真想用低低的声音(免得把他吓坏)问问他:“喂,请问,什么时间啦?”

克里莫夫把手榴弹的导火索一拉,将手榴弹扔进坑里。尘土在空中还没有落下,克里莫夫又扔出第二颗手榴弹,并且在爆炸之后立即跳进坑里。德国人全都死了,就好像在一分钟之前也不曾生活在世界上。克里莫夫被硝烟和灰尘呛得打着喷嚏,一面搜索他用得着的东西。他拿起望远镜,卸下重机枪的枪栓,从军官的热乎乎的手上捋下手表,又把机枪手的证件从军装口袋里小心翼翼地掏出来,免得沾上血。

他把得到的战利品交了公,说了说事情的经过,请谢廖沙给他倒水洗了洗手,便挨着科洛密采夫坐下来,说:

“现在咱们来抽支烟。”

这时候,曾经说自己是“安分守己的梁赞老百姓,喜欢钓鱼”的别尔菲里耶夫跑来了。

“喂,克里莫夫,你干吗在这儿坐着?”别尔菲里耶夫喊道。“楼长找你,还要再上德国人住的楼房里去一趟。”

“马上就去,就去。”克里莫夫用歉疚的语调说着,就开始收拾自己的家当:一支自动步枪和一帆布袋的手榴弹。他收拾这些东西很小心,似乎很怕把它们碰疼了。他对很多人称“您”,从来不骂娘。

“你不是洗礼派教徒吧?”有一次波里亚科夫老头子问他,虽然他已经打死一百一十个人了。

克里莫夫不是寡言少语的人,特别喜欢聊自己的童年。他父亲是普济洛夫工厂的工人。克里莫夫自己是万能车工,战前在工厂技术学校当教师。克里莫夫说,技术学校里有一个学生被一颗螺丝钉卡住,喘不上气来,脸发了青,克里莫夫赶去抢救,拿平口钳把螺丝钉从学生喉咙里拔了出来,谢廖沙听了觉得十分好笑。

但是有一次谢廖沙看见克里莫夫喝了不少缴获来的酒,他的样子很可怕,格列科夫见到他似乎都有点儿胆怯了。

“6—1”号楼里最邋遢的人是巴特拉科夫中尉。他从来不刷洗靴子,走起路来就有一个靴后跟吧嗒吧嗒直响,别人不用转头,就知道这位炮兵中尉来了。不过他每天都要用一块麂皮把眼镜擦几十次,镜片度数不适合他的视力,所以他老以为灰尘和硝烟把他的镜片弄模糊了。克里莫夫好几次摘下被打死的德国人的眼镜送给他。可是他很不走运:眼镜框很漂亮,镜片却不合适。

战前巴特拉科夫在技术学校教数学,其特点是自信心很强,常常用傲慢的语调说学生水平太低。

他曾经出数学题考谢廖沙,谢廖沙丢了脸。大家都笑起来,说要让谢廖沙留级,待到明年。

有一天空袭的时候,敌机像发了疯的锤工,用沉重的大锤砸在泥土、石头和钢筋上。格列科夫看到巴特拉科夫坐在残破的楼梯上,在读一本书。格列科夫说:

“德国佬什么也搞不到。他们拿这样的傻瓜有什么办法?”

德国人所干的一切,非但没有让“6—1”号楼里的人感到恐怖,倒是引来他们的嘲笑和轻蔑。

“嘿,德国佬上劲儿啦。”

“瞧,瞧,这些下流坯想的好主意……”

“真是笨蛋,瞧你把炸弹扔到哪儿去啦?”

巴特拉科夫和工兵排长安齐费罗夫很要好。安齐费罗夫四十岁上下,喜欢谈自己的慢性病,前线上这种现象是少见的。胃溃疡和神经根炎,在炮火下一般都能自动痊愈。

不过在斯大林格勒鏖战中安齐费罗夫依然经受着很多疾病的折磨,疾病已经在他胖大的身体中扎了根。德国医生没有治好他的病。

这个长着圆滚滚的秃头、圆脸和圆眼睛的人,在浑身被可怕的战火照得通亮的时候,依然悠闲自在地跟他手下的工兵们一起喝茶,那样子真是古怪离奇。他一般都是光着脚坐着,因为他脚上有鸡眼,一穿鞋就难受;他常常不穿制服,因为总觉得很热。他爱用一个蓝花碗喝滚热的茶,一面拿大手帕擦秃头上的汗,又叹气,又笑,朝茶碗吹气,头上缠着绷带的战士里亚霍夫时不时地用一个熏黑的大茶壶往茶碗里倒烧得滚开的陈水。有时安齐费罗夫不穿靴子,脚被硌得哼哧着,爬到碎砖堆上去,看看周围的情形。他光脚站着,不穿军服,不戴军帽,就像一个农民在狂风暴雨时候走出来站到门槛上,要看一看自己院子里的家当。

战前他担任工程主任。现在他的建筑经验用到了相反的方面。他的脑子时时在考虑如何破坏房屋、墙壁和地下工事。巴特拉科夫和他谈的主要话题是哲学问题。安齐费罗夫因为自己从建设转向破坏,所以很需要思考思考这种不寻常的转变。

有时候他们的谈话从哲学的高度出发,比如,人生的目的是什么,外星世界有没有苏维埃政权,男人的脑力结构在哪些方面胜过女人的脑力结构,然后谈话转向日常生活方面。

在这儿,在斯大林格勒的瓦砾堆里,一切都不同了,就连人们需要的智慧也常常在呆头呆脑的巴特拉科夫这边。

“说真的,老弟,”安齐费罗夫说,“多亏了你,我开始明白一些事情了。可是以前我还以为我彻底了解全部奥妙:谁需要半斤酒加小菜,谁需要汽车轮胎,谁需要票子。”

巴特拉科夫当真以为正是他和他的一些含混不清的见解,而不是斯大林格勒,使安齐费罗夫对人们有了新的认识,所以用居高临下的口气回答说:

“是啊,老兄,可以说,咱们是相见恨晚呀。”

在地下室里住的是步兵,他们多次打退德军的进攻,并且响应格列科夫响亮的号令进行反击。

指挥步兵的是祖巴廖夫少尉。战前他在音乐学院学声乐。有时他在夜里悄悄走到德国人盘踞的楼房跟前唱起来,有时唱《春天的气息,不要把我惊醒》,有时唱一段连斯基咏叹调。

别人问他,为什么要爬到碎砖堆上冒着被打死的危险唱歌儿,他从来不肯回答。也许他是要在这日日夜夜充满尸臭气的地方,不仅向自己和同志们,而且也向敌人显示,强大的毁灭性力量永远无法战胜美好的生命力。

如果不知道格列科夫、科洛密采夫、波里亚科夫、克里莫夫、巴特拉科夫和大胡子祖巴廖夫,能算是生活吗?

奶奶过去常说,头脑简单的干活儿的人都是好人,一直生活在知识分子环境中的谢廖沙认为奶奶的说法显然是很对的。

可是聪明的谢廖沙还是发现了奶奶的错误,这错误就是:她总认为干活儿的人头脑都是简单的。

“6—1”号楼里的人头脑并不简单。有一天,格列科夫说的一番话就使谢廖沙大吃一惊:

“不能把人当绵羊来领导。列宁那样聪明,就连他也不懂得这一点。所以要革命,为的就是不要任何人领导人。可是列宁却说:‘以前领导你们的人糊涂,我会做明智的领导。’”

谢廖沙从来没听到有人这样大胆,敢指责内务部里的人,指责他们在一九三七年杀害了成千上万无辜的人。

谢廖沙从来没听到有人带着这样沉痛的心情谈论普遍实行集体化时期农民所遭受的痛苦与灾难。有关这些问题的主要发言人是楼长格列科夫,不过科洛密采夫和巴特拉科夫也常常谈这些事。

这会儿,谢廖沙在司令部的掩蔽所里,觉得在“6—1”号楼以外度过的每一分钟都长得使人难受。听着人们谈论值班,谈论各部门领导的召见,觉得不可思议。

他想象这会儿波里亚科夫、科洛密采夫和格列科夫在干什么。

晚上,寂静的时刻,大家又在谈报话员姑娘了吧。

格ˆ—科夫要是下了决心,什么也阻止不住他,就是佛祖,甚至崔可夫,都对他没有办法。

“6—1”号楼里的人都是极好的人,是刚强、勇敢的人。大概今天夜里祖巴廖夫又唱歌了……她一定是在无精打采地坐着,等待着自己的厄运呢。

“我要杀人!”他在心里喊道,但没弄清他要杀谁。

他哪儿行啊,他还从来没有吻过姑娘呢,可是那些家伙是老手,当然会欺骗她,玩弄她。

他听到不少艳史,说的是有些护士、女电话员、女测距员、女仪表员、女学生很不情愿地成为一些团长和炮兵营长的“野味”。他对这些艳史不欣赏,不感兴趣。

他看了看掩蔽所的门。他先前为什么没有想起,他可以谁也不问,站起来就走呢?

他站起来,开了门,走了出来。

就在这时候,有人给司令部值班参谋打来电话,说是根据政治部主任瓦西里耶夫指示,要让被困的楼房里出来的战士立即去见政委。

达佛尼斯和克洛伊[59]的故事所以永远能打动人心,并不是因为他们的爱情发生在蓝天之下,葡萄藤蔓丛中。

达佛尼斯和克洛伊的故事在各种地方重演着,不论是带有炸鳕鱼气味的窒闷的地下室,在集中营的棚屋,在机关会计室的算盘声中,还是在纺纱车间的灰尘里。

这故事又发生在瓦砾堆里,在德国轰炸机的隆隆声中,在人们不是用蜜糖,而是用烂土豆和旧锅炉里的水滋养自己肮脏的、汗淋淋的身体的地方,发生在没有了安宁和寂静,只有打碎的石头、轰隆声和臭气的地方。

六十二

在斯大林格勒发电站担任门卫的安德列耶夫老头子收到从列宁斯克捎来的一封信,是儿媳妇写来的。儿媳妇在信里说,婆婆害肺炎死了。

得到老伴去世的消息以后,安德列耶夫打不起精神了,很少上斯皮里多诺夫那儿去,每天傍晚都坐在工人宿舍的门口,望着一闪一闪的炮火和愁云密布的天上晃动着的探照灯光。宿舍里的人有时候找他说话,他却一声不响。说话的人以为老头子耳朵背了,便用更高的声音把话重说一遍。安德列耶夫就阴沉地说:“听见啦,听见啦,我没有聋。”就又不作声了。老伴的死对他震动很大。他的生活反映在妻子的生活中,他遇到的好事、坏事,他的快活心情、悲伤心情都保存和反映在老伴的心中。

在狂轰滥炸,重磅炸弹到处爆炸的时候,安德列耶夫老汉望着发电站各车间之间冒起的一股股灰尘和硝烟,心里想:“我那老伴儿能看看就好啦……嘿,瞧,好家伙……”

可是这时候她已经不在人世了。

他觉得,被炸弹和炮弹炸坏的房屋残骸,被炸得坑坑洼洼的院子,一堆堆的黄土和扭七歪八的钢铁,着了火的油库那苦涩、潮湿的浓烟和黄黄的、火龙般的慢慢爬动的火焰—都是他的生命的表现,是他的残生的象征。

难道他当年曾经坐在明亮的房间里,吃早饭准备上班,妻子站在他身旁看着他:该不该为他添饭?是啊,他只有孤单单地死去了。他忽然想起年轻时候的她,胳膊晒得黑黑的,眼睛里洋溢着快活的神气。算啦,他也要死的,而且时间不远了。

有一天晚上,他踩着咯吱咯吱响的木头台阶,慢慢地走进斯皮里多诺夫的掩蔽所。斯皮里多诺夫看了看老头子的脸,说:

“老人家,身体不舒服吗?”

“斯捷潘·费多罗维奇,您还年轻,”安德列耶夫回答说,“您的力气小些,您要多保重。我的力气有的是,我一个人能走得到。”

这时候正在洗锅的薇拉没有立即明白老头子的意思,回头看了看他。

安德列耶夫不需要任何人的同情,希望转换话题,就说:

“薇拉,您该走了,这儿又没有医院,只有坦克和飞机。”

她笑了笑,摊开湿漉漉的两只手。

斯皮里多诺夫很生气地说:

“就连一些不认识她的人都说这话。不论谁看到她,都说,应该转移到左岸去。昨天集团军军委委员来了,来到我们的掩蔽所里,看了看薇拉,什么也没说,可是等他坐上汽车,却骂起我来:您怎么,没做过父亲吗,是不是想让我们用装甲快艇把她送过河去?我能说什么呢:她不愿意,就是不愿意。”

他说得很快、很流畅,就好像天天在争论同一个问题的一些人那样。安德列耶夫老头子望着早就绽了线的上衣袖子没有作声。

“在这儿简直收不到什么信。”斯皮里多诺夫又说。

“这算什么军邮。我们在这儿待了这么久,没收到过岳母、叶尼娅、柳德米拉一封信。托里亚在哪儿,谢廖沙在哪儿,谁又能知道?”

薇拉说:

“他老人家收到信啦。”

“他收到的是死讯。”

斯皮里多诺夫对自己的话感到害怕。他十分激动地说起来,一面用手指着掩蔽所矮矮的墙壁,指着遮住薇拉的床的布幔:

“瞧她在这儿是怎么住的,她总是姑娘,是女的,这儿天天有男子汉挤来挤去,白天是这样,晚上也是这样,时而是工作人员,时而是卫队,人挤得满满的,又嚷嚷,又抽烟。”

安德列耶夫说:

“您就可怜可怜快要生的孩子吧,在这儿孩子就完啦。”

斯皮里多诺夫对薇拉说:

“你想想看,万一德国人冲进来呢!那时候怎么办?”

薇拉没有作声。她自己相信,维克托罗夫会走进炸坏的发电站大门的,她会老远看到他穿着飞行服、软底靴,挎着图囊走来。

她常常走到公路上,看他是不是来了。乘车经过的战士们常常对她喊:

“喂,姑娘,你等谁呀?坐到我们车上来吧。”

她一时间也快活起来,就回答说:

“你们的汽车经不住人坐。”

在苏军飞机飞过的时候,她凝望着低低地飞行在发电站上空的一架架歼击机,似乎她就要认出维克托罗夫来了。

有一天,有一架歼击机在发电站上空飞过时摇了摇翅膀,薇拉就叫了起来,并且像一只失望的小鸟一样打着趔趄向前奔去,跌倒在地上。跌过这一跤之后,她的腰疼了好几夜。

月底,她看到在发电站上空进行的一场空战。这场空战不分胜负。苏军飞机进入云层中,德军飞机转过头朝西飞去。薇拉站着,望着没有了飞机的天空,她那瞪得老大的眼睛里还流露着极其紧张的神情,一名装配工从院子里走过,看见她这种神情,说:

“斯皮里多诺娃同志,您怎么啦,是不是受伤了?”

她相信,她就会在这儿,在发电站和维克托罗夫见面,但是她觉得,如果把这一点告诉爸爸,命运之神就会怪她沉不住气,不让他们见面了。有时候她这种信心十分强烈,以至于匆匆忙忙地烙起面粉加土豆粉饼子,匆匆忙忙地扫地,收拾东西,擦洗脏鞋……有时她和爸爸坐在一起,忽然侧耳倾听一阵子,说:“等一等,我出去一下子。”便披起大衣,从掩蔽所里走出去,四处张望,看看有没有飞行员站在外面,是不是有人在问,怎样可以找到斯皮里多诺夫父女。

她一次也没有想过、一分钟也没有想过他会忘记她。她相信,维克托罗夫也和她一样,日日夜夜在急切地、深深地想念着她。

德军的重炮几乎每天都在轰击发电站。德国人的技术很好,试射、发炮都很准,炮弹打在车间的墙壁上,一阵一阵的爆炸声震颤着大地。常常飞来一两架零散的轰炸机,投掷炸弹。有的敌机贴着地面飞,在从发电站上空飞过时,拿机枪扫射。有时在远处的山冈上出现德军的坦克,这时能清楚地听到机关炮的嗒嗒声。

斯皮里多诺夫似乎已经习惯了炮击与轰炸,发电站的其他工作人员好像同样也习惯了。不过,不论是他还是他们,习惯归习惯,同时却渐渐失去积蓄起来的精神力量。有时斯皮里多诺夫就感到疲惫无力,很想躺到床上,拿棉袄把头蒙上,静静地躺着,一动也不动,也不睁眼睛。有时他拼命地喝酒。有时他想跑到伏尔加河岸上,渡过河去,在对岸的草原上走一走,再不回头看这发电站,宁愿蒙受当逃兵的羞耻,只要不再听到德军炮弹和炸弹的可怕的呼啸声。有一次,他通过附近的六十四集团军司令部的高频电话和莫斯科通话,副人民委员说:

“斯皮里多诺夫同志,转达莫斯科方面的敬意,向您领导的英雄集体致敬。”

这时他感到很难为情:哪儿谈得上英雄呀?此外,还一直有一种传闻,说是德军正准备对发电站进行密集袭击,要用巨型的炸弹把发电站摧毁。听到这些传闻,手脚都发冷。白天,眼睛一直瞅着灰色的天空,看是不是有敌机飞来。夜里,他有时忽然跳起来,因为仿佛听到越来越近的大批敌机沉闷而密集的隆隆声。胸前和背后常常吓出冷汗。

显然,不只是他一个人神经紧张。总工程师有一天对他说:

“一点力气也没有啦,好像有什么妖魔鬼怪跟着我,我常常看着公路,想:能跑掉就好啦。”

党委书记尼古拉耶夫晚上到他这儿来,说:

“给我拿酒来,这些天我离了这种防弹剂就睡不着觉。”

他一面给尼古拉耶夫斟酒,一面说:

“真是‘活到老,学到老’。应当学会一门技术,能够轻而易举地把设备转移,要不然,你瞧,涡轮机留在这儿,咱们也只好陪着。别的工厂的人早就在斯维尔德洛夫斯克大街上溜达了。”

有一天,他在劝薇拉走的时候说:

“我真不理解,我们这儿的人天天上我这儿来,拿出种种理由要求离开这儿,可是我实心实意劝你走,你却不走。要是准许我走的话,我一分钟也不耽搁。”

“我因为你才留在这儿,”她粗声粗气地回答说,“没有我,你会变成酒鬼。”

不过,当然,不能说斯皮里多诺夫一味地在德军炮火面前发抖。发电站的人也很勇敢,也担负着艰巨的工作,也笑,也说笑话,对于严峻的命运也有满不在乎的感觉。

薇拉一直在为孩子担心。孩子生下来会不会健康?她住在这闷人的、充满烟气的地下室里,每天大地都被炸得不住地颤动,这对孩子有没有影响?近来她常常觉得恶心,头晕。她这个当母亲的天天看到的是瓦砾堆、战火、被炸得坑坑洼洼的大地、盘旋在灰色天空的黑十字飞机,会生出多么悲伤、胆小、忧愁的孩子?也许,孩子甚至能听见可怕的爆炸声,也许,听到炸弹呼啸声,那蜷缩着的小小身体连动也不敢动,小小的头缩进肩膀里了。

常常有身穿肮脏油污的大衣,腰系士兵帆布带的人从她身边跑过,一面跑一面挥手,微笑,喊叫:“薇拉,日子过得怎样?薇拉,想我吗?”她感觉到大家对她这个未来的母亲的亲热。也许,小东西也能感觉到这种亲热,他的心将是纯洁而善良的。

她有时候到机械车间去,现在这里在修理坦克,过去维克托罗夫曾经在这里工作过。她在猜:哪儿是他的车床呀?她使劲儿想象他穿着工装或者飞行服的样子,但是他却总是穿着军医院的伤员服出现在她的脑海里。

在车间里,不仅是发电站的工人,而且集团军基地的坦克手们也都认识她。她却无法辨别他们,因为干活儿的工人和干活儿的军人十分相像,都是穿着油糊糊的棉袄,戴着皱巴巴的帽子,手都很脏。

薇拉时时刻刻想着维克托罗夫,想着孩子,日日夜夜都感觉到孩子的存在。对于外婆、小姨叶尼娅、谢廖沙和托里亚的担心退到了次要地位,有时她想起他们,也只是感到怅惘罢了。

夜里,她想念母亲,呼唤她,向她诉苦,向她求助,她低语着:

“妈妈,好妈妈,帮帮我吧。”

这会儿她觉得自己软弱无力,一筹莫展,完全不像刚才那样,还很沉着地对父亲说:

“别说了,我不走,哪儿也不去。”

六十三

吃午饭的时候,娜佳随口说:

“托里亚喜欢吃煮土豆,不怎么喜欢吃烤土豆。”

柳德米拉说:

“到明天他正好十九岁零七个月。”

晚上,她说:

“玛露霞要是听说了法西斯在亚斯纳亚波利亚纳[60]的暴行,会多么伤心呀。”

过了一会儿,弗拉基米罗芙娜在工厂里开完大会回来了,维克托帮她脱大衣,她对维克托说:

“维克托,天气真好,空气又干,又冷。你妈妈会说:像葡萄酒。”

维克托回答说:

“妈妈还说酸白菜像葡萄。”

生活在流动着,好像漂游在大海里的大冰块,在寒冷而昏暗的水中游动的水下部分支持着水上部分,水上部分抗击着波涛,听着水的喧嚣与拍溅,散发着寒气……每当朋友家的年轻人进入研究生院,论文答辩,恋爱,结婚,除了祝贺和家长里短的议论之外,往往免不了几声慨叹。

每当维克托听到熟识的人在战争中牺牲,就好像他身上有一部分活的物质死了,脸上的血色也暗淡了。不过死者的声音依然在生活的喧嚣中回荡着。

维克托的思绪和心灵所萦系着的时代是可怕的,它也波及了妇女和孩童。在这段时间里,他家里死了两个妇女、一个小伙子,这小伙子几乎还是孩子。维克托常常想起有一次他听到索科洛夫的亲戚、历史学家马季亚罗夫念的曼德尔施塔姆的两行诗:

捕狼犬的时代向我扑来,

但我不是狼,生来就不是……

不过这时代就是他的时代,他和这时代生活在一起,死后仍然联系在一起。

维克托的研究工作依然进行得很不顺利。

战前早就开始的试验,没有得到理论所预测的结果。

尽管有各种各样的试验数据,尽管有决心打破现有的理论,但依然显得凌乱、不合理,使人丧气。

起初维克托认为,他失败的原因在于试验不完善,缺乏新的仪器设备。他对实验室的工作人员很生气,似乎他们没有把足够的精力放在工作上,只是关心生活琐事。

可是,问题并不在于才华横溢、乐观而可爱的萨沃斯季扬诺夫天天想方设法去弄酒票买酒,不在于无所不知的马尔科夫在工作时间发表长篇议论或者讲解这个或那个院士享受什么样的供应,某某院士的供应要怎样分配给两位过去的夫人和一位现在的夫人,也不在于安娜·纳乌莫芙娜天天唠叨她和女房东的关系。

萨沃斯季扬诺夫的思想很活跃,很清晰。马尔科夫照样很赞赏维克托·施特鲁姆知识渊博,善于进行精密的试验,冷静地进行推理。安娜·纳乌莫芙娜虽然住在寒冷而残破的过道小屋里,工作还是非常勤奋,非常踏实。维克托照样因为有索科洛夫和他在一起工作感到自豪。

不论多么精确地安排试验条件,不论怎样检查测定,不论怎么校正计量器,都不能得出明确的结果。在重金属有机盐在强辐射下受到的影响这一研究中,也出现了混乱现象。有时维克托觉得这种盐粒就像一个毫无礼貌和理性的小矮子,戴着耷拉在耳朵上的小圆帽,脸上搽着红粉,对着理论的严肃面孔不停地做鬼脸,还做着下流动作和轻蔑的手势。参与提出这一理论的是世界上知名的物理学家。数据计算是无可指摘的,德国与英国一些有名的实验室里几十年来积累的试验资料为理论提供了证据。战前不久在剑桥进行过一次试验,可以证实理论所预言的粒子在特殊环境中的反应。那次试验的结果是理论上的重大成就。可是维克托依然觉得那次,那次试验是不够实际的,就像证实相对论所预言的光线进入太阳磁场会出现偏斜的试验。触动这一理论似乎是不可思议的,就好比一名士兵要撕掉元帅的金肩章。

可是小矮子依然在做鬼脸,在做轻蔑的动作,而且没办法叫他老实下来。在柳德米拉去萨拉托夫之前不久,维克托想到,扩大理论探索范围是可能的,当然,这就需要做出两种任意的假设,需要大大加强数学计算。

新的方程式涉及索科洛夫所擅长的一个数学分支。维克托觉得自己在这一数学领域没有足够的把握,便求助于索科洛夫。索科洛夫很快地为扩展理论算出新的方程式。

问题似乎解决了,试验数据不再与理论相矛盾了。维克托为此感到高兴,向索科洛夫祝贺,索科洛夫也向维克托祝贺,可是担心和不满意依然存在。

不久,维克托又苦闷起来。他对索科洛夫说:

“我发现,每天晚上柳德米拉一拿起毛线织补袜子,我的情绪就坏了。这使我想起我和你,我和你在织补理论,粗糙的活儿,毛线的颜色也不一样,是瞎折腾。”

他喜欢摆出自己的疑虑,幸而他不会欺骗自己,因为他本能地感觉到,自我安慰只能导致失败。

扩展理论没有任何好处。理论一旦经过织补,就失去内部的协调,任意的假设会使理论丧失其自主的力量和独立的存在,其方程式会十分复杂,运用起来很不容易,理论就会带有学究式的、空洞的、贫血的意味,仿佛失去了活的肌体。

才能出众的马尔科夫安排了一系列新的试验,得出的结果又与算出的方程式产生了矛盾。为了解释这一新的矛盾,只好提出另一种任意的假设,又要用火柴和碎木片支持理论。

“瞎折腾。”维克托自己对自己说。他明白了,他的做法很不对头。

他收到克雷莫夫工程师一封信。克雷莫夫告诉他,他所订制的仪器的浇铸和磨光工作要推迟一段时间,工厂正忙着生产军用品,看样子,所需要的仪器要比原定时间晚一个半月到两个月才能生产出来。

不过,维克托收到这封信并没有感到难过。他已经不像过去那样急切地等待着新仪器了,不相信新仪器会改变试验结果。有时他非常烦恼,这时很希望快点儿收到新仪器,以便最后证实,大量的扩展的试验资料,是彻头彻尾与理论相矛盾的。

研究方面的不顺利与他的个人伤心事交织起来,一切都变得灰暗,绝望。

这种灰沉情绪持续了好几个星期。他变得很容易生气,对家务琐事似乎有了兴趣,常常过问柴米油盐的事,看到柳德米拉花那么多钱,总觉得惊讶不解。

他关心起柳德米拉和房东家的争执。房东要求增加房租,因为使用了他们家的柴棚。

“你跟房东太太谈得怎么样啦?”他问道。等他听过柳德米拉的叙述,又说:“唉,他妈的,这娘们儿真坏。”

现在他不考虑科学与人类生活的关系,不考虑科学是福还是祸。要考虑这些问题,必须自觉是主人,是强者。然而这些天来他一直感到自己是个一事无成的受雇的徒工。

他似乎再也不能像原来那样从事研究了,他所经受的痛苦使他失去了研究科学的力量。他在脑子里一一回想了一些有名的物理学家、数学家、作家,他们的主要成就都是在青年时代取得的,在三十五岁到四十岁以后,他们已经没有什么了不起的成就了。仅此一点,他们就足以自豪。而他却没有在年轻时做出终生可以回忆的事情,只有坐等老死。为一百年来数学的发展提供了多种途径的伽罗华在二十一岁就死了,爱因斯坦二十六岁就发表了专著《运动物体的电动力学》,赫兹死时不到四十岁。这些人的命运和维克托之间存在的差别,简直有如云泥!

维克托对索科洛夫说,他想暂时停止试验工作。但是索科洛夫认为,应当继续进行试验,等新仪器来了,许多问题可能解决。维克托本来想对他说说刚收到的工厂来信,现在甚至忘记了。

维克托看出来,妻子知道他的研究很不顺利,但是她不跟他谈他的研究。

她不关心他生活中的主要的东西,而把时间用于做家务,同玛利亚聊天,同房东太太争吵,为娜佳做连衣裙,同波斯托耶夫的妻子来往。维克托很生她的气,不了解她的心境。

他觉得,妻子已经恢复了习惯的生活,而她所以做习惯的事情,正因为已经习惯了,不需要什么精力,她的精力已经没有了。

她一面做面条汤,一面谈娜佳的鞋子,因为她做了多年的家务事,所以现在像机器一样做着已经习惯了的事情。他却没有看出来,她虽然像以往一样生活,在生活中却没有感觉了。好比一个行路人,想着自己的心思,在走惯了的路上走着,绕过坑洼,跨过水沟,却没有觉察到有坑洼和水沟。

要想跟丈夫谈他的研究,她需要新的力量、新的精神资源。她没有力量。维克托觉得,她对一切事情的兴趣都还保留着,只是对他的研究没有兴趣了。

柳德米拉在谈到儿子的时候,常常提到一些事,似乎说明丈夫对托里亚不够好,维克托觉得很委屈。她好像是在总结托里亚与继父的关系,而结论总是对维克托不利。

柳德米拉对母亲说:

“托里亚很可怜,有一个时期脸上出了很多粉刺,他很难过,甚至要我找美容师给他弄点儿药膏治一治。可是维克托还一个劲儿地笑话他。”

这的确是事实。

维克托很喜欢逗托里亚。托里亚回到家来,向他问好,他常常把托里亚仔细打量一遍,摇摇头,若有所思地说:

“哎,伙计,你脸上好像出星星啦。”

近来维克托一到晚上不喜欢坐在家里。有时他上波斯托耶夫家里下棋,听音乐。波斯托耶夫的妻子钢琴弹得不错。有时去找喀山的新朋友卡里莫夫。但多半还是去索科洛夫家。

他喜欢索科洛夫家那小小的房间,喜欢殷勤好客的玛利亚那亲切的笑容,尤其喜欢茶余酒后的聊天。每当他很晚串门子回来,一走到家门口,暂时忘却的苦闷又袭上心头。

六十四

维克托没有从研究所回家,而是去找自己的新朋友卡里莫夫,邀他一起上索科洛夫家去。

卡里莫夫是个麻子,相貌很丑。黑皮肤衬得白头发特别白,白头发又使黑皮肤显得特别黑。

卡里莫夫俄语说得十分地道,只有仔细听,才能听出在发音与用词造句方面的细微差异。

维克托过去没有听到过他的名字,但实际上他已经很有名气,而且不只是在喀山。卡里莫夫将《神曲》、《格列佛游记》译成鞑靼语,最近又在译《伊利亚特》。

当他们还不熟识的时候,他们走出大学的阅览室,常常在吸烟室里见面。图书管理员是个衣着马虎,爱抹口红又十分健谈的老太婆,对维克托说了不少有关卡里莫夫的事情。说他是巴黎大学毕业的,在克里木有别墅,战前每年一大半时间在海边度过。战争时期他的妻子和女儿留在克里木,他一直没有她们的音信。老太婆还向维克托暗示,此人一生中有过长达八年的艰难经历,但是维克托却用大惑不解的目光迎接了这一消息。看样子,老太婆也把维克托的情况对卡里莫夫说了。他们还没有认识就彼此了解了,感到很不好意思,每次相遇时不是微笑,倒是皱起眉头。有一次他们在图书馆的前厅里撞了个满怀,两个人同时笑起来,说起话来,才结束了这种尴尬的局面。

维克托不知道卡里莫夫是否对他说的话感兴趣,但在卡里莫夫听他说话的时候,他很有兴趣说话。维克托有过很不愉快的经验,常常碰到一些交谈者,似乎又聪明又机智,实际上呆板得不得了。

有些人,维克托在他们面前连说话都很吃力,声音也变僵硬了,说的话既无意义,又无趣味,有点儿像聋哑盲人了。有些人,在他们面前任何真诚的话都带有做作的腔调。也有些人是多年的相识,但在他们面前维克托感到自己特别孤独。

为什么会这样?途中邂逅的旅伴,邻铺而眠的宿友,或者一次偶然争论的参与者—只要有人在场,他就愿意敞开心扉,不再感到孤独。

他们在一起走着,说着话儿,维克托心想,现在,特别每天晚上在索科洛夫家聊天的时候,他可以一连几个钟头不回想自己的研究了。以前这种情形从来不曾有过,以前他时时想着自己的研究,不论在电车上,在吃饭的时候,听音乐或者早晨洗脸的时候。

也许,他钻进的这个死胡同太气闷了,所以他下意识地要摆脱有关研究的一些想法……

“艾哈迈德·奥斯曼诺维奇,今天工作效率如何?”维克托问道。

卡里莫夫说:

“脑袋一点儿不听使唤。一个劲儿地在想着老婆和女儿,有时觉得一切都会平安无事,会看到她们的,有时会出现一种预感,觉得她们都完了。”

“我了解您。”维克托说。

“我知道。”卡里莫夫说。

维克托心想:奇怪,他和这个人才认识了几个星期,就想对他说说自己对妻子和女儿都不能说的话了。几乎每天晚上都有一些人在索科洛夫家小小房间的饭桌上聚会,这些人在莫斯科未必都见过。

索科洛夫是一个才华出众的人,说话文绉绉的,谈起什么都是长篇大论。很难相信,他出身伏尔加水手之家,会有这样优雅斯文的谈吐。他是一个善良而高尚的人,可是脸上的表情却显得狡猾又严酷。

索科洛夫还有一些地方很不像伏尔加的水手,比如,他滴酒不沾,怕穿堂风,因为怕传染,一个劲儿地洗手,吃面包还要把手指头接触到的那一部分面包皮剥掉。

维克托在宣读他的论文的时候,常常感到惊讶:一个人能这样细致、大胆地思考,这样简洁地表述和证明极其复杂和细微的原理,平常说话竟那样冗长,那样啰唆。维克托和许多在斯文的知识分子环境中长大的人一样,言谈之间倒是喜欢说一些粗话,如“他妈的”、“胡扯”,在和老院士谈话时常常把爱争吵的学者夫人叫做“冤鬼”或者“女魔”。

索科洛夫在战前最不喜欢谈政治。维克托一谈到政治,索科洛夫就沉默下来,不再说话,或者故意换个话题。

他的性格中有一种奇怪的顺从态度,对于集体化时期和一九三七年的许多残酷的事没有任何抱怨。他似乎认为国家的灾祸是自然的灾祸,是上天降下的灾祸。维克托觉得,索科洛夫似乎信仰上帝,而且这种信仰表现在他的研究中,表现在他对当今世界的强者的顺从中,表现在他与别人的个人关系中……

六十五

马季亚罗夫说话平静而从容,他不为那些后来被当做人民敌人和祖国叛徒枪毙了的师长和军长们辩护,不为托洛茨基辩护,但是从他赞扬克里沃卢奇科和杜波夫的口气,从他提到一九三七年被杀害的一些指挥官和政委的名字时不经意流露出的那种尊敬,可以感觉出来,他不相信图哈切夫斯基、布柳赫尔、叶戈罗夫元帅、莫斯科军区司令穆拉洛夫、二级集团军司令列万多夫斯基、加马尔尼克、特宾科、布勃诺夫以及托洛茨基的第一副手斯克良斯基和温什里希特是人民的敌人,祖国的叛徒。

马季亚罗夫谈论这些大事,口气之平静与从容令人不可思议。要知道强大的国家机器篡改了历史,按自己的要求重新发动骑兵,重新任命历史事件的英雄,把真正的英雄抹去。国家有足够的力量,可以使永远无法改变的既成事实重演一番,可以重刻大理石,重铸铜像,可以改变以往的发言,改变文献纪录片上的人的位置。

这真是全新的历史。就连当年幸存下来的人,都要按新的方式考虑过去的生活,把自己从勇士变为懦夫,从革命者变为外国间谍。

听到马季亚罗夫的话,会觉得更为强大的逻辑,真理的逻辑,有朝一日必然会显露它的本来面目。在战前从来没有这样的谈话。有一次他说:

“唉,所有这些人如果活到今天,都会奋不顾身地同法西斯作战,决不吝惜自己的鲜血。真不该把他们杀掉……”

化学工程师弗拉基米尔·罗曼诺维奇·阿尔捷列夫是喀山本地人,是索科洛夫家的房东。阿尔捷列夫的妻子到傍晚时候才下班回家。两个儿子都在前方。阿尔捷列夫在化工厂担任车间主任。他穿着很不讲究,没有皮大衣和皮帽,为了保暖,棉祆外面罩上胶布披风,头上戴一顶油糊糊、皱巴巴的圆帽,去上班的时候把圆帽紧紧扣到耳朵上。

每次他到索科洛夫家来,总是呵着冻得发僵发红的手指头,羞怯地对坐在桌边的人笑着,维克托觉得,好像他不是房东,不是大工厂的大车间的主任,而是一个穷邻居,是寄人篱下的。

就如这天晚上,胡子拉碴、两腮瘪下去的阿尔捷列夫就站在门口,听马季亚罗夫在说话,看样子他是怕踩得地板吱咯响。

玛利亚在前往厨房的时候,走到他跟前,小声对着他的耳朵说了两句话。他吓得直摇头,看样子,是玛利亚请他吃饭。

马季亚罗夫说:

“昨天,有一位上校,是在此地养病的,他对我说,在前线党委会有人对他提出控告,他打了那个中尉一顿耳光。在国内战争时期可没有这样的事。”

“您自己说过,邵尔斯把革命军事委员会的人狠狠打了一顿嘛。”维克托说。

“这是下属打领导部门的人呀,”马季亚罗夫说,“这是不同的。”

“在我们厂里,”阿尔捷列夫说,“厂长对所有的工程技术人员都称‘你’,可是如果你叫他‘舒尔约夫同志’,他就生气,必须喊他‘厂长’。前几天在车间里有一位老技术员得罪了他,他又骂娘又嚷嚷,说:‘叫你干什么,你就干什么,要不然我叫你滚,你就得滚你妈的。’那位老人家已经七十二岁了。”

“工会不说话吗?”索科洛夫问道。

“还说什么工会,”马季亚罗夫说,“工会号召做牺牲:战前准备迎接战争,战争时期一切为了前方,等战后工会又要号召消除战争后果。哪儿会关心老头子的事?”

玛利亚小声问索科洛夫:

“是不是该用茶了?”

“是的,是的,”索科洛夫说,“给我们弄茶来。”

“她动作多么轻悄呀。”维克托在心里说,一面漫不经心地看着玛利亚那瘦削的肩膀,看着她溜进半开着的厨房的门。

“唉,亲爱的同志们,”马季亚罗夫忽然说,“你们可知道,什么是言论自由吗?但愿你们在战后和平的早晨,打开报纸,看不到欢呼的社论,看不到劳动者给伟大的斯大林的信,看不到炼钢工人为庆祝最高苏维埃选举加班加点的报导和美国劳动者在悲惨、失业和穷困中迎接新年的报导,你们猜,在报纸上能看到什么?看到各种各样的信息!你们能想象这样的报纸吗?能提供信息的报纸!你们可以看到:库尔斯克州歉收,对布特尔监狱的制度进行了检查,对于开凿白海至波罗的海的运河正在进行争论,可以看到普通工人发表意见,反对发行新的公债。

“总而言之,你们可以知道国内发生的一切:知道丰收,也知道歉收;知道忘我劳动,也知道撬锁盗窃;知道矿井产量,也知道矿井事故;知道莫洛托夫和马林科夫的分歧;还会看到因为厂长侮辱七十岁的老技术员而引起罢工的报导;可以读到丘吉尔和布吕姆的讲演,而不是他们‘似乎声称’的那一些;可以读到英国下议院辩论的报导;可以知道,昨天在莫斯科有多少人自杀,有多少被撞伤的人被送进外科医院。

“可以知道为什么没有荞麦米,而不是仅仅知道用飞机从塔什干往莫斯科运来了最早的草莓。如果要了解集体农庄每个劳动日分多少粮食,可以看报纸,不必问家里的保姆,不必等到她的侄女从乡下来莫斯科买粮食。是的,是的,尽管如此,苏联人还是苏联人。

“每个人都可以进书店,买书,依然做自己的苏联人,但是可以阅读美国、英国、法国哲学家、历史学家、经济学家、政治评论家的作品。都可以自己分辨,他们哪些地方不对;每个人都可以不要保姆,随意在街上行走。”

恰好在马季亚罗夫结束自己的长篇大论的时候,玛利亚端着茶具走了进来。索科洛夫忽然用拳头在桌上一擂,说:

“算啦!我恳切地、坚决地要求不要再谈这一类的事啦。”

玛利亚半张着嘴,看着丈夫。茶具在她手里叮当响起来,看样子,她的手发抖了。

“瞧,彼得·拉甫连季耶维奇取消了言论自由!言论自由只存在了一小会儿。好在玛利亚·伊凡诺芙娜没有听到这些造反的话。”维克托说。

“我们的制度显示了自己的优越性,”索科洛夫愤慨地说,“资产阶级民主过时啦。”

“不错,显示倒是显示了,”维克托说,“不过,芬兰的过时的资产阶级民主在一九四〇年与我们的集中制相遇,我们竟陷入十分尴尬的境地。我不崇拜资产阶级民主,不过事实毕竟是事实。再说,老技术员的事究竟该怎样解释呢?”

维克托回头看了看,看到正在听他说话的玛利亚凝视的眼睛。

“问题不在芬兰,而在芬兰的冬天。”索科洛夫说。

“哎,算啦,彼得。”马季亚罗夫说。

“可以这样说,”维克托说,“在战争期间,苏维埃国家显示了自己的优越性,也显示了自己的弱点。”

“什么样的弱点?”索科洛夫问。

“比如说,有许多人,本来现在可以参加战斗的,却被关起来了,”马季亚罗夫说,“你们瞧,伏尔加河上打得多激烈呀。”

“不过,这和制度有什么关系?”索科洛夫问道。

“怎么没有关系?”维克托说。“彼得,依您看,难道士官的遗孀一九三七年是自己枪毙自己的吗?”

他又看到玛利亚那凝神注视的眼睛。他心想,他在这场争论中表现实在奇怪:马季亚罗夫一批评国家,他就和他争论;可是索科洛夫一反驳马季亚罗夫,他又批评起索科洛夫。

索科洛夫有时喜欢嘲笑不高明的文章或文理不通的讲话,但是一谈到总的路线,就变得像石头一样坚硬。马季亚罗夫则相反,从不掩饰自己的心情。

“你们认为,我们撤退是由于苏维埃制度不完善,”索科洛夫说,“其实是德囯人给予我们国家的打击太强烈,我们国家能经住这样的打击,恰恰清楚不过地显示了我们的强大,而不是软弱。你们看到巨人投下的影子,会说:瞧,好大的影子。但是你们忘记了巨人本身。要知道,我们的集中制是巨大的原动力的社会发动机,能够产生种种奇迹。已经产生了不少奇迹。今后还会产生许多奇迹。”

“如果国家不需要你,就会把你折腾够,把你和你的思想、计划和文章弄得一钱不值,”卡里莫夫说,“如果你的思想与国家利益相符,就会让你坐上飞毯,青云直上!”

“就是,就是,”阿尔捷列夫说,“我曾经被派到一处特别重要的国防工程去工作了一个月。斯大林亲自过问各车间的生产,不时给主管人打电话。设备是一流的。原料、零件、备件,要什么有什么。生活条件好极了。有浴室,炼乳每天早晨送到家。一辈子我还没过过那样的日子呢。生产上的供应好得不得了!主要是没有什么官僚主义。干什么事都不靠公文来往。”

“老实说,官僚主义的国家机构,就像童话里的巨人一样,都是人安排的。”卡里莫夫说。

“如果在国家的重要国防工程方面能这样完善,那原则上就很清楚:可以在所有的工业中推行这样的制度。”索科洛夫说。

“禁区!”马季亚罗夫说。“这是完全不同的两种原则,不是一种原则。斯大林兴建的工程是国家需要的,而不是人民需要的。需要重工业的是国家,而不是人民。白海至波罗的运河对人民无益。一头是国家需要,一头是人民需要,二者永远不能调和。”

“就是,就是,从这种禁区再往旁边跨一步,就是胡闹,”阿尔捷列夫说,“有时附近的喀山需要我们的产品,可是我们得按计划把产品运往赤塔,然后再从赤塔运回喀山。我们需要装配工,可是我们修建托儿所的贷款没有花完,我们就要把装配工送往托儿所做保育员。集中制真害死人!有的发明者向厂长建议,可以生产一千五百件零件,而不是原计划的二百件,厂长把他撵走,因为厂长正在煞有介事地执行计划,所以别多事。如果生产停顿,所缺的材料可以花三十卢布在市场上买到,那他宁可损失两百万,不肯冒险花三十卢布去买材料。”

阿尔捷列夫很快地拿眼睛扫了扫听他说话的人,又很快地说起来,好像生怕别人不让他说下去。

“工人收入很少,不过根据不同劳动,有所差别。一个售货员的实际所得就相当于一个工程师的五倍。可是领导人、厂长、委员们就知道一点:完成计划!不管你是否饿肚子,是否浮肿,计划都要完成!我们原来的厂长是什马特科夫,他常常在会议上喊叫:‘工厂比亲娘更重要,你们就是脱三层皮,也要把计划完成。谁要是不自觉,我要亲自揭他三层皮。’后来忽然听说,他要调到沃斯克列先斯克去了。我问他:‘厂长同志,生产计划还没有完成,您怎么丢下工厂要走啦?’他毫不掩饰,坦率地回答说:‘噢,您要知道,我的孩子在莫斯科上大学,沃斯克列先斯克离莫斯科近些。再说,到那儿要分给我一套好房子,还有花园,我妻子身体不大好,需要新鲜空气。’所以我感到很奇怪,为什么国家要把工厂交给这样的人,却把工人、党外的著名学者看得不值几个钱。”

“原因十分简单,”马季亚罗夫说,“交给这些人的是比工厂和学校更重要的东西,交给他们的是制度的心脏,是最神圣的东西:产生苏维埃官僚主义的权力。”

“我说的就是这话,”阿尔捷列夫不想把谈话变成说笑话,继续说,“我很爱自己的车间,从不爱惜自己。可是我的心不够狠,不能从活人身上剥三层皮。剥自己的皮还可以,剥工人的皮就有些于心不忍。”

维克托继续保持着他自己也不明不白的态度,但觉得有必要反驳一下马季亚罗夫,虽然他觉得马季亚罗夫说的话都很对。

“您的话有很大的毛病,”他说,“难道在今天,人民的利益和兴建国防工业的国家的利益不相符,不是完全一致吗?我认为,飞机、大炮、坦克是我们的子弟兵需要的,也就是我们每个人的需要。”

“这话完全对。”索科洛夫说。

六十六

玛利亚开始给大家斟茶。大家谈论起文学。

“咱们把陀思妥耶夫斯基忘记啦,”马季亚罗夫说,“图书馆不愿出借,出版社不愿重印。”

“因为他是反动作家呀。”维克托说。

“这话很对,他不应该写《群魔》。”索科洛夫附和说。

可是维克托马上问道:

“您真的认为不应该写《群魔》吗?还不如说,不该写《作家日记》呢。”

“天才作家不需要别人指教,”马季亚罗夫说,“我们的思想体系容不得陀思妥耶夫斯基。马雅可夫斯基就不同。难怪斯大林称他为最优秀的、最有才华的作家。他的情感本身就是国家观念。可是陀思妥耶夫斯基呢,就连他的国家观念本身也是人道主义。”

“如果这样说,”索科洛夫说,“那么,整个十九世纪的文学都不符合我们的思想体系。”

“可不能这样说,”马季亚罗夫说,“比如托尔斯泰,他把人民战争的思想诗化了,现在国家领导的就是人民的正义战争。正如刚才艾哈迈德·奥斯曼诺维奇[61]说的,两种思想相符合,就会乘飞毯直上云端:托尔斯泰的作品又在广播电台广播,又在晚会上朗诵,又出版,领导人又引用。”

“最顺利的是契诃夫,过去的时代和我们的时代都承认他。”索科洛夫说。

“你这话可错了!”马季亚罗夫叫起来,并且拿手掌在桌子上一拍。“我们承认契诃夫,是由于没有真正理解。就像承认在某种程度上师从他的左琴科[62]一样。”

“我真不懂,”索科洛夫说,“契诃夫是现实主义作家。我们反对的是颓废派。”

“你不懂吗?”马季亚罗夫问道。“我可以给你解释。”

“你们别糟践契诃夫吧,”玛利亚说,“他是我最喜欢的作家。”

“玛利亚,你说的很对,”马季亚罗夫说,“彼得·拉甫连季耶维奇,你要在颓废派身上寻找人道主义吗?”

索科洛夫很生气地摆了摆手,表示不再睬他。

但是马季亚罗夫也朝他摆了摆手,他认为最主要的是说出自己的想法,为此就必须让索科洛夫找找颓废派的人道主义。

“个人主义不是人道主义!您混淆了。完全混淆了。您以为颓废派受到打击了吗?胡说。颓废派对国家无害,只是没有用处。我认为,社会主义现实主义与颓废主义没有太大差别。大家都在争论什么是社会主义现实主义。社会主义现实主义是镜子,这镜子对于党和政府提出的问题‘世界上谁最可爱、最好、最伟大?’回答说:‘你,你,党,政府,国家,最好、最可爱。’颓废派对这个问题回答说:‘我,我,我,颓废派,最美、最可爱。’二者差别不太大,社会主义现实主义强调国家的特别重要性,颓废主义强调个人的特别重要性,方式不同,实质是一样,都是陶醉于各自的特别重要性。完美无缺的国家,瞧不起与国家不一致的一切人。颓废派的镶了花边的人,对一切其他的人都极其冷漠,只除了两种人:一种是和他们高谈阔论的人,一种是跟他们卿卿我我的人。从表面上看,个人主义、颓废主义似乎都在为了人而斗争。从实质上说,根本没有斗争。颓废派不关心人,国家也不关心人。在这方面没什么不同。”

索科洛夫眯着眼睛在听马季亚罗夫说话,他感觉到马季亚罗夫马上就要说到根本不能说的东西,就打断他的话,说:

“请问,这和契诃夫有什么相干?”

“说的正是契诃夫。契诃夫和现在的一切就有很大的不同。契诃夫把没有实现的俄国的民主担在自己的肩上。契诃夫的道路就是俄国自由的道路。我们走的却是另一条道路。你们数数看,他写的人物有多少呀。也许只有巴尔扎克使这样众多的人物为社会所认识。而且也未必有这样多!真是可观:有医生、工程师、律师、教员、教授、地主、小店老板、工厂主、家庭女教师、仆人、大学生、大大小小的官吏、牲口贩子、技工、媒婆、教会执事、僧侣、农民、工人、鞋匠、模特儿、管园子的、动物学家、客店老板、猎人、渔夫、娼妓、尉官、士官、艺术家、厨娘、作家、管院子的、修女、士兵、产婆、萨哈林岛的苦役犯人……”

“够啦,够啦。”索科洛夫叫道。

“够啦?”马季亚罗夫用故作威胁的口吻反问道。“不,不够。契诃夫使我们认识了整个的俄罗斯、俄罗斯的各个阶级、阶层、各种年龄的人……但是不仅如此。他使我们认识了这平平常常的许多人,明白吗,俄国的平常人!在他以前从没有人这样说,就连托尔斯泰也没有说,可是他说:我们所有的人首先是人。明白吗?首先是人,人,人!俄罗斯在他以前谁也没有这样说过。他说,最主要的是,人就是人,然后才是僧侣、俄罗斯人、小店老板、鞑靼人、工人。要明白,人的好与坏不是因为他是僧侣还是工人,是鞑靼人还是乌克兰人,人都是平等的,因为都是人。半个世纪之前,持有狭隘的党派观点的人认为契诃夫是停滞时代的代表。然而契诃夫却是最伟大的旗帜的旗手,这面旗帜是在俄罗斯一千年的历史中高高举着的旗帜,是真正的、俄罗斯的、实实在在的民主的旗帜,明白吗,是俄罗斯的人的尊严、俄罗斯的自由的旗帜。因为我们的人道主义总带有宗派色彩,成了不可调和的,残酷的。就连托尔斯泰宣传不以暴力抗恶也受到批判,而其实,他不是从人出发,而是从上帝出发。他认为最重要的是主张善良的思想得到肯定,因为传教的人总是急不可待地强迫人相信上帝,而在俄国为此不惜采取一切手段,刺伤,杀害,在所不顾。

“契诃夫说:让上帝到一边去吧,让所谓伟大的先进思想到一边去吧,首先是人,我们要善良,要关心人,不管什么人,僧侣、庄稼汉、百万巨富的工厂主、萨哈林的苦役犯、饭店的跑堂;首先要尊重人,怜惜人,热爱人,不这样绝对不行。这就叫民主,这就是俄罗斯人民目前还没有得到的民主。

“俄罗斯人一千年来什么都看到了,看到了‘伟大’,也看到了‘超级伟大’,但有一样东西没看到,那就是民主。这也正是颓废派与契诃夫的区别。国家愤恨颓废派,会捶他们的后脑勺,会踢他们的屁股。可是国家却不理解契诃夫思想的实质,所以容许他存在。民主在我们的事业中是没有用场的,当然,这是指真正的、人道主义的民主。”

看样子,索科洛夫很不喜欢马季亚罗夫这一番十分尖锐的话。维克托看出这一点,便带着自己也弄不清来由的满意心情说:

“说得太好了,很对,很有道理。不过请多多原谅斯克里亚宾[63],他好像也属于颓废派,可是我非常喜欢他的乐曲。”

玛利亚正要把一碟子蜜饯放到他面前,他用手做了一个推让的姿势,并且说:

“不用,不用,谢谢,我不要。”

“这是黑醋栗。”她说。

他看了看她那棕色的、微黄的眼睛,问道:

“我对您说过我特别喜欢黑醋栗吗?”

她一声不响地点了点头,含着笑意。她的牙齿不大整齐,嘴唇薄薄的,血色淡淡的。她那苍白而多少有些灰色的脸因为带笑,显得可爱动人。

“如果不是鼻子一直发红的话,她倒是很漂亮,很好看。”维克托在心里说。

卡里莫夫对马季亚罗夫说:

“列昂尼德·谢尔盖耶维奇,怎么能把您对契诃夫的人道主义的颂扬和对陀思妥耶夫斯基的赞美结合到一起呢?陀思妥耶夫斯基认为,在俄罗斯并不是所有的人都一样。希特勒骂托尔斯泰是蠢猪,可是,据说,他把陀思妥耶夫斯基的肖像挂在他的办公室里。我是少数民族,是鞑靼人,出生在俄罗斯,这位俄罗斯作家仇恨波兰人和犹太人,我不原谅他。虽然他是天才作家,我也不能原谅他。在沙皇俄罗斯我们流的鲜血、受的欺骗、遭的浩劫太多了。俄罗斯的伟大作家没有权利中伤异族人,没有权利蔑视波兰人、鞑靼人、犹太人、亚美尼亚人、楚瓦人。”

这位白头发、黑眼睛的鞑靼人带着气愤而傲慢的蒙古人的冷笑口气,对马季亚罗夫说:

“您大概读过托尔斯泰的《哈吉·穆拉特》吧?大概读过《哥萨克》吧?大概读过《高加索俘虏》吧?这些都是这位俄罗斯伯爵写的。跟立陶宛人陀思妥耶夫斯基不一样。鞑靼人有生之日,都要为托尔斯泰祈祷上天。”

维克托看了看卡里莫夫,在心里说:“原来你这样,原来你这样。”

“艾哈迈德·奥斯曼诺维奇,”索科洛夫说,“我非常尊重您对自己民族的感情。但是请原谅,我也因为我是俄罗斯人而感到自豪,请原谅,我喜欢托尔斯泰并不仅仅因为他写鞑靼人写得很好。不知为什么,我们俄罗斯人不能因为自己的民族而自豪,差点儿我们要成为黑色百人团了。”

卡里莫夫站起身来,脸上冒出一层汗珠,他说:

“我要对您说实话,真的。如果有实话可说,我为什么要说假话。早在二十年代大批鞑靼族的精英就被杀害了,文化界知名人士全被杀了,如果没忘记这个,就应该想到为什么《作家日记》会成为禁书。”

“不仅杀你们的人,也杀了我们的。”阿尔捷列夫说。

卡里莫夫说:

“消灭的不光是我们的人,还有我们的民族文化。现在鞑靼的知识分子与那些人相比,等于白丁。”

“是的,是的,”马季亚罗夫用嘲笑的口吻说,“那些人不仅创立了文化,而且创立了鞑靼自己的内外政策。”

“你们现在有自己的国家了,”索科洛夫说,“有大学、中学、歌剧院、书籍、鞑靼报纸,都是革命给予你们的。”

“是的,有国家歌剧院,也有国家。可是抓我们进监狱的也是……”

“不过,要知道,如果抓你们的是鞑靼人,你们也不见得好过些。”马季亚罗夫说。

“可是,如果根本没有人抓,不是更好吗?”玛利亚问道。

“噢,玛利亚,你想得太好啦。”马季亚罗夫说。

他看了看表,说:

“哎呀,时间不早啦。”

玛利亚连忙说:

“列昂尼德,在我家睡吧。我给您支起活动床。”

有一次他对玛利亚诉苦说,每当晚上回到家里,一个人也没有,走进空荡荡的黑屋子,感到自己特别孤单。

“好吧,”马季亚罗夫说,“我没意见。彼得·拉甫连季耶维奇,你不反对吧?”

“不反对,瞧你说的。”索科洛夫说。马季亚罗夫又用开玩笑的口气说:

“男主人说得一点热情也没有。”

大家一齐站起来,开始告别。索科洛夫出去送客人,玛利亚压低声音对马季亚罗夫说:

“真不错,这一次彼得·拉甫连季耶维奇听到这类的话没有躲避。在莫斯科,只要一涉及这方面的事,他就闭上嘴巴,一句话也不说。”

她称呼丈夫的名字和父称“彼得·拉甫连季耶维奇”用的是特别亲热、特别尊敬的语调。她晚上常常为他誊写论文,把他的手稿保存起来,把他随便写的一些字用硬纸裱糊起来。她认为他是伟人,同时又觉得他是无用的孩子。

“我很喜欢那位维克托·施特鲁姆,”马季亚罗夫说,“我真不懂,为什么有人认为他是叫人讨厌的人。”

他又用开玩笑的口吻说:

“玛利亚,我发现,他所有的话都是当着您的面说的,您在厨房里忙活的时候,他舍不得运用他的口才。”

她脸朝门口站着,没有作声,就好像没听见马季亚罗夫的话,过了一会儿才说:

“列昂尼德,您怎么啦,我在他眼里只是微不足道的女人。彼得认为他不厚道,认为他可笑、高傲,因此同事们很不喜欢他,有些人还怕他。可是我就不这样看,我觉得他憨厚。”

“憨厚算不上,”马季亚罗夫说,“他对什么人都挖苦,什么人的话他都不赞成。不过他的思想是活泼的,没有僵化。”

“不,他很憨厚,最没有城府。”

“但是,应当承认,”马季亚罗夫说,“彼得就是现在也不说一句多余的话。”

这时索科洛夫走了进来。他听见了马季亚罗夫的话。

“列昂尼德,我对你有一点要求,”他说,“求你不要教训我,还有,求你在我在场的时候不要谈诸如此类的事情。”

马季亚罗夫说:

“你要知道,彼得,你也不要教训我。我说的话我自己负责,你只管你自己的话好啦。”

看样子,索科洛夫本想用很尖锐的话回答他,但是他忍住了,又从屋里走了出去。

“好吧,也许我还是回家好些。”马季亚罗夫说。

玛利亚说:

“您太让我难过了。您该知道他的心是善良的。他会难过得一夜都睡不好。”

她解释说,彼得·拉甫连季耶维奇的心灵是受过创伤的,他经历过许多事情,一九三七年被抓去受到严厉审讯,审讯以后在精神病院住了四个月。

马季亚罗夫一面听着,一面点头,然后说:

“好吧,好吧,玛利亚,我听您的,不走了。”

忽然他又生起气来,说:

“您这话当然有道理,不过,被抓过的不光是您的彼得。还记得,把我关在卢宾卡,关了十一个月吗?在那段时间里,彼得只给克拉娃打过一次电话。这是对亲妹妹的态度吗?还有,他还不准您给她打电话。克拉娃因为这事十分伤心……也许,他是很伟大的物理学家,不过他的心灵却带有奴性。”

玛利亚拿手捂住脸,一声不响地坐着。

“谁也不了解,不了解我因为这事儿有多么难受。”她小声说。

只有她知道,他多么痛恨一九三七年的事以及普遍推行集体化时的惨无人道,只有她知道,他的心灵有多么纯洁。但也只有她知道,他的思想被束缚得多么厉害,他对政府多么顺从,多么俯首帖耳。

因此他在家里非常任性,像老爷一样,玛利亚为他刷鞋子,天热时为他擦汗,在别墅里散步的时候用小树枝儿为他赶蚊子。

维克托还是大学高年级学生的时候,有一次忽然对一位同班同学说:“真无法看下去,全是甜言蜜语,千篇一律。”他说着,把一张《真理报》扔到地上。

他刚刚说过这话,就害怕起来。他捡起报纸,抖了抖灰尘,非常可怜地笑了笑,很多年之后,他一想起那次低声下气的笑,就觉得脸上火辣辣的。

过了几天,他又把一张《真理报》递给那位同学,很带劲儿地说:

“格里沙,你看看这社论,写得真棒!”

那位同学接过报纸,用怜惜的口吻对他说:

“可怜的维克托胆子太小啦。你以为我会去汇报吗?”

于是,维克托就在那时候发下誓言:要么沉默,不说危险的话,要么,说出来就不怕。可是他没有守住自己的誓言。他常常失去谨慎,一冲动,就“乱说一气”,一说出来,往往又失去勇气,就想方设法扑灭自己烧起的火星。

一九三八年,在布哈林事件之后,他对克雷莫夫说:

“不管怎么说,我是了解布哈林的,我同他交谈过两次:他聪明过人,和蔼可亲,妙语横生,总而言之,是一个非常纯洁、非常有魅力的人。”

可是他看到克雷莫夫那忧郁的目光,就觉得不安起来,马上又说:

“不过,鬼才知道,间谍,暗探,还有什么纯洁和魅力。简直是卑鄙!”

接着他又激动起来,因为克雷莫夫仍然像刚才听他说话时那样,带着忧郁的神气说:

“因为咱们是亲戚,我可以告诉您:说布哈林是暗探,我无法理解,永远无法理解。”

这时维克托忽然愤恨起自己,愤恨那种使人不能做人的力量,大声叫道:

“天呀,我才不相信这种可怕的事!这些事是我一生中的噩梦。为什么他们要承认,为什么要承认呀?”

但是克雷莫夫不再说了,看样子,他觉得已经说多了……

啊,坦率地说话,说真话,这其中有多么神奇、光明磊落的力量呀!有些人因为说了几句大胆的、没有多加考虑的话,付出了多么可怕的代价!

有好几次,维克托夜里躺在床上,仔细听着大街上的汽车声。柳德米拉光着脚走到窗前,撩开窗帘。她看一阵子,等一阵子,然后轻悄悄地(她以为维克托睡着了)回到床上躺下。第二天早晨,她问:

“你睡得怎样?”

“谢谢,很好。你呢?”

“有点儿闷热。我到窗口去过。”

“噢,噢。”

真不知如何表达夜晚这种无罪而又唯恐大祸临头的感觉。

“维克托,记住,你的话万一有一句传到那地方,你就完啦,我和孩子们也完啦。”

还有一天她说:

“我说不出很多道理,不过,看在上帝面上,你听我的,对谁都不要说什么。维克托,咱们生活在可怕的时代,你什么也算不上。记住,维克托,什么都别说,对谁都不要说……”

有时维克托面前会出现一个人的痛楚而困惑的眼神,这人是他从小就认识的,使人感到可怕的不是老朋友的话,而是那种欲言又止的神情,可怕的是,维克托不敢直截了当地问他:“他们传讯你。你是间谍吗?”

他有时想起自己的助手的脸,有一次他当着这位助手的面很轻率地开玩笑说,斯大林在牛顿之前很久就发明了万有引力定律。

“您什么也没有说,我什么也没听见。”年轻的助手爽快地说。

为什么,为什么,为什么要开这种玩笑?不管怎么样,开这种玩笑是愚蠢的,就好比随便乱敲硝化甘油[64]瓶。

啊,自由而爽快地说话的力量呀!这力量就表现在一下子说出来而不害怕。

不论维克托是否了解今日自由交谈的悲惨结果,这些谈话的参与者都是痛恨法西斯、害怕法西斯的……为什么在战争已经打到伏尔加河上,他们都在经受着战争失败的痛苦,战争失败带来可恨的法西斯奴役的时候,仍然没有自由?

维克托一声不响地同卡里莫夫在一起走着。

“很奇怪,”他忽然说,“看外国的描写知识分子的小说,比如海明威的小说,他笔下的知识分子在谈话的时候不停地喝酒。鸡尾酒,威士忌,朗姆酒,白兰地,然后又是鸡尾酒,威士忌,各种牌子的白兰地,俄罗斯知识分子的重要谈话却是在喝茶时进行的。民意派、民粹派和社会民主党人的许多事都是靠一杯上等的清茶谈成的,列宁同战友们商讨伟大的革命也是靠一杯清茶。不错,听说,斯大林倒是喜欢白兰地。”

卡里莫夫说:

“是的,是的,是的。如今的谈话也都是在喝茶的时候。您说得很对。”

“就是,就是。马季亚罗夫真有头脑!真够大胆!他说的那一番叫人十分听不惯的话太有意思了。”

卡里莫夫抓住维克托的胳膊。

“维克托,您是否发现,马季亚罗夫有时把微不足道的事情说得过分严重?使我不放心的就是这一点。要知道,他在一九三七年被捕过,关了几个月,又放出来了。那时候可没有放过任何人。无缘无故是不会放的。明白吗?”

“明白,明白,当然明白,”维克托慢悠悠地说,“他是不是拿话来引话?”

他们在拐弯处分了手,维克托朝自己家走去。

“去他妈的,随他的便吧,”他想道,“真希望像人一样说说话儿,不害怕,什么都谈,痛痛快快地谈,不矫饰,不说假话,什么都不在乎……”

幸亏像马季亚罗夫这样能独立思考的人还有,还没有完全灭绝。而且卡里莫夫在分手时对他说的一番话也没有像往常一样使他心里发冷。

他心想,他又忘记对索科洛夫说说他收到的乌拉尔来信了。

他在黑沉沉、空荡荡的大街上走着。忽然出现了一点想法。他马上毫无疑虑地认识到、感觉到这想法是对的。他发现了对于一些似乎不能解释的核现象的新解释,全新的解释,天堑忽然变成通途。多么简单,多么明了呀!这想法极其可亲,极其可爱,似乎不是他想出的,而是自己随便而轻盈地冒出来的,就像一朵水生的白花儿一下子从静静的湖水中冒了出来,他看到这美丽的花儿,不禁赞赏起来……

他忽然想:偏偏在他根本没有想科学上的事,在他很感兴趣的关于人生的争论成为一个自由的人的争论的时候,在他的话和交谈者们的话受着苦涩的自由约束的时候,出现了这一想法,真是奇怪,真是意外。

六十七

当你第一次看到卡尔梅克草原的时候,当你坐在汽车上,焦虑不安,心事重重,眼睛漫不经心地看着一座座不高的山冈出现又消失,看着山冈缓缓地从地平线后面浮起来又缓缓地游到地平线后面的时候,这生长着一片片羽茅草的草原似乎显得异常寒碜,异常苦闷……达林斯基觉得,似乎只是一座光秃秃的山冈在他面前一次又一次浮起来,只是一段道路弯来弯去,一次又一次钻到汽车轮胎底下。草原上的骑马人似乎也都是一个样子,都是孤孤单单的,尽管骑马人有的是没有胡子的年轻人,有的是白胡子老头儿,有的骑的是黄骠马,有的骑的是青色的快马……

汽车经过一个个村落和放牧点,擦过一座座小屋,小屋都有小小的窗户,窗户里都有密密的天竺葵,就像生长在玻璃缸里一样,看样子,如果把窗玻璃打碎,如水一般的空气就会向周围流淌开去,天竺葵就会干死;汽车擦过一座座圆圆的、抹了黄泥的毡房,穿过一片片毫无生气的羽茅草、一片片带刺的骆驼草、一片片盐土,擦过一头头用小腿踢得灰尘乱飞的绵羊、一堆堆在风中摇曳的野火……

从城里驱车而来,轮胎里充满了带着城市烟尘的空气,这样的人来到草原上,所看到的一切似乎一律是灰色的、寒碜的,一切都是单调的、一模一样的……刺蓬,大蓟,羽茅,菊苣,艾蒿……被漫长的时间巨轮压平展了的一座座山冈散落在大平原上。卡尔梅克东南部的这片草原正在渐渐变成沙漠,沙漠向东扩展,从埃利斯塔向雅什库,直到伏尔加河口和里海岸边……这片草原具有一个惊人的特点:天与地彼此相望时间太久,以至于变得分不出彼此了,就好比在一起过了一辈子的夫妻,到后来十分相像了。很难分清那一丛丛铝灰色的羽茅是生长在寂寞的淡淡的草原蓝天里,还是草原泛起蓝色的天光;有时旋起一阵轻轻的灰尘,就连天和地也分不清了。看着巴茨湖和巴尔曼扎湖那浓重的湖水,就觉得那是盐碱冒到了地面上;而看着那光秃秃的盐碱地,又觉得那不是土地,是湖水……

在十一月和十二月无雪的日子里,卡尔梅克草原上的道路显得很奇怪:依然是干枯的灰绿色野草,大路上依然飞舞着灰尘,真不知道,这草原是太阳晒干的,还是寒风吹干的。

也许因此这儿常常出现海市蜃楼,这时候空气和大地、水和盐碱地的界限模糊了。这种幻景让旅途中饥渴的人遇见,由于想象的操纵和思想的动向再度幻化,灼热的空气会变成蔚蓝色的、轮廓整齐的石头,光秃的大地会像静静的湖水似的晃动起来,一片片的棕榈树一直铺展到天边,火辣辣的阳光和一团团灰尘混到一起,变成庙堂和宫殿的金灿灿的圆顶……人在疲惫的时刻自己也用天和地创造自己的理想世界。

汽车在大路上,在寂寞的草原上不停地奔驰着,奔驰着。

忽然之间,这空荡荡的草原世界以全新的、完全不同的姿态呈现在人的面前……

卡尔梅克草原!你是大自然最古老、最高明的创作,其中没有一丝矫饰的美,没有任何生硬突兀的线条,这儿朴素而凄怆的蓝灰色调可以和雄伟而悲壮的秋日俄罗斯森林媲美,这儿缓缓起伏的岗峦比高加索的高山更动人心魄,这儿的小湖积满了黑郁郁的、宁静的古老的水,似乎比所有的海洋更能表现水的实质。

一切都会过去,可是这暮霭中巨大的、铁球般的、沉甸甸的太阳,这充满野蒿苦味的风,不会被忘记。还有这草原,将不再贫瘠可怜,必将繁茂富饶……

到了春天,草原上生机盎然,到处是郁金香,草原成了海洋,不过不是波涛怒吼,而是繁花似锦。凶恶的骆驼刺也披上绿装,新生的尖刺还是柔软的,还没有变硬……

夏日的夜晚,在草原上可以看到银河系像摩天大楼一样耸立着:底部是蓝色、白色巨石般的星群,顶部是直插苍茫的宇宙穹顶的一个个球状星团……

草原有一个特别了不起的特点。它永远保持这一本色,从不改变:不论冬天或是夏天,不论在黎明时候,还是在黑沉沉的风雨交加或者月明星稀的夜晚,草原总是首先对人说着自由……草原总是让失去自由的人想起自由。

达林斯基走出汽车,看着走上山冈的一个骑马人。那人身穿长袍,腰上扎着绳子,骑在一匹长毛痩马上,正回头望着草原。那是一个老人,一张脸已经像石头一样僵硬了。

达林斯基向老人家呼唤了一声,走到他跟前,把烟盒递过去。老人家很快地在马上转过整个身子,那动作中既有年轻人的灵活,又有老年人的沉着,他打量了一下拿着烟盒的手,然后打量达林斯基的脸,然后打量他腰上的手枪、他那中校级的三道杠杠、他的漂亮的皮靴。然后伸出细细的褐色手指头,那指头又细又小,简直可以叫做小孩子手指头,他拿了一支烟,在空中转悠了一下。

这位卡尔梅克老汉那一张颧骨很高的、像石头一样僵硬的脸一下子全变了,纵横交错的皱纹里露出两只善良而精明的眼睛。这一双栗色的老眼流露出来的目光同时带有试探和信任的神气,看样子,这目光中包含着某种很好的东西。达林斯基不由得快活起来,高兴起来。老汉的马在达林斯基走近时不友好地竖起耳朵,这时也放下心来,好奇地侧过一只耳朵,后来又侧过另一只,随后那大牙齿的嘴巴和圆圆的大眼睛露出了笑意。

“谢谢。”老人家用细细的嗓门儿说。

他拿手掌在达林斯基的肩膀上抚摩了一会儿,说:

“我有两个儿子,都在骑兵师里,一个已经牺牲了,是大儿子。”

他用手比了比,表示大儿子比马头还高。

“另一个儿子,就是小儿子,”他用手比着比马头低些的地方,“是机枪手,得了三个勋章啦。”

接着他又问:

“你家里还有人吗?”

“我母亲还活着,父亲已经死了。”

“唉,真可惜呀。”老人家摇了摇头。达林斯基心想,老人家难过不是出于礼貌,而是听到这位请他抽烟的俄罗斯中校死了父亲,实心实意地表示同情。

后来老人家忽然吆喝一声,大大咧咧地扬了扬手,那马就极其敏捷、极其轻盈地冲下山冈。

这骑马的老人家奔驰在草原上,想着什么呢:是想着儿子,还是想着仍然待在破旧汽车旁边的俄罗斯中校死了父亲的事?

达林斯基注视着骑马飞驰的老人家,觉得太阳穴里不是血在冲打,而是有话要向外冲:“自由……自由……自由……”

他心里不由得充满了对那位卡尔梅克老人家的羡慕。

六十八

达林斯基是奉命长期出差,从方面军司令部到位于左翼边缘的集团军去。方面军司令部的人都认为到这个集团军里去是一项特别苦的差事,最可怕的是缺水,驻地条件差,供应差,距离又远,路又难走。这一部分军队孤零零地驻扎在里海与卡尔梅克草原之间的沙漠里,方面军司令部不了解他们的实际情况,所以把达林斯基派往该地区,交给他许多任务。

达林斯基在草原上走了几百公里之后,觉得烦闷起来。这儿谁也不考虑进攻,被德国人赶到了天边的这支部队似乎已到了绝境……不久前司令部日日夜夜的紧张情形、对于近期发动进攻的揣测、后备兵力的调动,来来往往的密码电报、司令部通讯中心昼夜不停的工作、北方开来的汽车队和坦克队……是不是梦中的事?

达林斯基听着炮兵指挥员和其他兵种指挥员们灰心丧气的话,看着技术装备情况的资料,视察着各炮兵营和炮兵连,望着士兵和指挥员们无精打采的脸,望着人们慢慢地、懒洋洋地在草原灰尘中移动,渐渐染上此地的寂寞与烦闷。他心想,这下俄罗斯到骆驼生活的草原上来了,来到荒芜的沙丘上,疲惫无力地躺倒在贫瘠的土地上,再也爬不起来,站不起来了。

达林斯基来到集团军司令部,来见高级领导人。

在宽敞而幽暗的房间里,有一个圆脸、秃顶、身穿没有领章的军便服的小伙子正在同两个穿军装的女人打牌。这位中校走进来,小伙子和两个戴尉官领章的女人没有放下手里的牌,只是漫不经心地打量了他一眼,依然很带劲儿地喊着:

“不要王牌?J也不要?”

达林斯基等到一局结束,这才问道:

“集团军司令员住在这儿吗?”

其中一个年轻女人回答说:

“他到右翼去了,到傍晚才回来。”

她用老练的军事工作人员的目光打量了一下达林斯基,就问道:

“中校同志,您大概是方面军司令部来的吧?”

“是的。”达林斯基回答过,又轻轻使了个眼色,问:“那么,请问,我可以见见军委委员吗?”

“他和司令员一块儿出去了,傍晚才回来。”另一个女人回答过,又问:“您是从炮兵司令部来的吧?”

“是的。”达林斯基回答说。

达林斯基觉得回答有关司令员情况的第一个女人特别漂亮,虽然看样子她比回答有关军委委员情况的那个女人大得多。这样的女人有时显得非常漂亮,有时候,比如偶然一转头,却显得憔悴,衰老,不好看。这个女人就是这种类型的。她的鼻子很端正,很秀气,眼睛蓝蓝的,很不和善,说明这个女人知道别人以及自己的准确分量。

她的脸显得非常年轻,看起来她顶多二十五岁,可是只要一皱眉头,沉思起来,嘴角上就露出皱纹,下巴底下的皮肤也耷拉下来,看起来就至少有四十五岁了。不过那一双穿着尺寸合适的鞣革皮靴的脚,实在好看。

这些情形要说是得说好一阵子的,可是达林斯基那老练的眼睛一眼就看清楚了。

另一个女人是年轻的,但是已经发胖了,身体很肥大。她的一切分别看来都不怎么美:头发稀稀的,颧骨很宽大,眼睛颜色蓝不蓝、棕不棕;但她却显得很年轻、很有风韵,即使瞎子来到她跟前,也会感觉到她那娴雅的风韵。

这一点达林斯基也是在转瞬间看出来的。

不但如此,他还以某种方式在这一瞬间掂量了回答有关司令员情况的第一个女子和回答有关军委委员情况的第二个女子的分量,并且做出那样一种没有实际意义的选择,男人看到女人时差不多总要做这种选择的。达林斯基一直在操心怎样才能找到司令员,司令员是不是给他提供应有的条件,在哪儿吃饭,在哪儿睡觉,到右翼边缘的师里去的路是不是很远,路是不是难走,这时候他还漫不经心、同时也不是那么漫不经心地考虑了一番:“就这个女的吧!”

这么一来,他就没有马上去找集团军参谋长取所需要的材料,而是坐下来玩牌了。

在玩牌的时候(他是那位蓝眼睛女子的配手)弄清了许多事情:他的配手叫阿拉·谢尔盖耶芙娜,另一位年轻些的女子在司令部医疗站工作,没戴领带的圆脸小伙子名叫沃洛佳,看样子,和司令部的什么人有亲戚关系,所以在军委会食堂做炊事员。

达林斯基马上就觉察到阿拉·谢尔盖耶芙娜是有权势的,这是从进来的一些人对待她的态度上看出来的。看样子,集团军司令员是她的合法丈夫,不过,达林斯基开头以为他们是恩爱夫妻,实际上却根本不是这样。

起初他弄不清楚,为什么沃洛佳对她的态度那样随便。但是后来达林斯基恍然大悟,一下子猜出来:大概,沃洛佳是司令员前妻的弟弟。当然,还不完全清楚,司令员的前妻是否还活着,是不是办理过离婚手续。

年轻的女子克拉芙季娅显然同军委委员不是合法夫妻。阿拉·谢尔盖耶芙娜在对她说话的时候微微流露出傲慢和宽容的语气,那意思似乎是:“当然啦,咱们在一块儿打牌,彼此以‘你’相称,不过,咱们是在参加战争,还得注意一点儿影响。”

但是克拉芙季娅在阿拉·谢尔盖耶芙娜面前也有某种优越感。达林斯基觉得她的优越感大概是这样:虽然我不是合法夫人,而是战时情侣,但我对我的军委委员是忠实的,你虽然是合法夫人,可是你的一些事情我们都知道。你要是敢叫我“破鞋”,那就试试看……

沃洛佳很喜欢克拉芙季娅,他毫不掩饰这一点。他对她的态度大概可以这样来表达:我的爱情是没有希望的,我这个炊事员怎么能跟军委委员比高低……不过,虽然我是炊事员,我是真心诚意爱你的,你自己也能感觉出来;只要能得到你的青睐就行,至于军委委员为什么爱你,我才不管呢。

达林斯基打牌技术很不高明,阿拉·谢尔盖耶芙娜很注意照顾他。她很喜欢这位瘦瘦的中校:他常常说“谢谢您”,在分牌的时候他们的手碰到了,他还慢条斯理地说“对不起”;如果沃洛佳用手指揩鼻涕,然后又用手帕擦手的话,他总要带着发愁的神气看看沃洛佳;别人说俏皮话,他都很有礼貌地笑一笑,他说起俏皮话都要使人捧腹。

听了达林斯基说的一个笑话之后,她说:

“真的,我一下子没有听懂。在这草原上过了这么久,脑子变钝啦。”

她说这话说得很低,好像是要让他明白,或者让他感觉到,他们可以单独谈谈,谈谈只有他们两人能谈的话,那种使人心跳的话,那种特别的、顶顶重要的男人和女人的话。

达林斯基还是常常出错牌,她就给他纠正,而这时候他们玩起另一种牌戏,在这种牌戏中达林斯基就不出错牌了,因为他精于此道……虽然在他们之间,除了说“把小黑桃打出来嘛”、“垫上嘛,垫上嘛,别怕,别舍不得王牌”之类的话以外,什么都没有说,但是她已经了解和看中了他的许多动人之处:又温柔,又刚强,又谨慎,又勇猛,又腼腆……阿拉·谢尔盖耶芙娜所以能感觉到这一切,是因为她暗暗在达林斯基身上观察出这些特点,还因为他很成功地向她显示了这些特点。她也很巧妙地向他显示,她懂得了他的目光,懂得他为什么注视她的笑容、她的手的动作、她的肩膀耸动、她那漂亮的华达呢军便服里面的胸脯、她的脚、她那修得很好看的指甲。他觉得,她的声音拖长得有点儿过分,有点儿不自然,她的笑也比一般的笑时间要长些,为的是让他注意她的清脆的声音、她那雪白的牙齿和腮上的两个酒涡儿……

达林斯基因为忽然出现这样的感情,心中很激动,很不平静。他对这种感情从来不觉得习以为常,每一次都像第一次有这种感情一样。他对待女人的丰富经验没有变为习惯,经验是一回事,迷恋是另一回事。正是这一点说明他是真正的好色男子,不是假的。

结果,这一夜他留在集团军指挥所里。

第二天早晨,他去找参谋长。参谋长是一位寡言少语的上校,既没有问他斯大林格勒方面的情况,也没有打听前线的消息和斯大林格勒西北方的战况。交谈过之后,达林斯基就知道,这位上校参谋长未必能向他提供足够的有关情况,就请他在自己的委派书上签字,决定下连队去。

他坐上汽车的时候,有一种很奇怪的感觉,觉得两手和两脚空空的、轻飘飘的,什么念头、什么希求都没有,觉得十分满足而又十分空虚……似乎周围的一切,似乎昨天他还很喜欢的天空、野蒿和草原山冈已经变得索然无味,不值得一看了。也不想跟司机说话或开玩笑。就连思念亲人,回忆他一向热爱和尊敬的母亲,也变得乏味、冷淡了……想到沙漠里的战斗、俄罗斯边远地区的战斗,也不激动了,他感到无精打采。

达林斯基不时地吐一口唾沫,摇摇头,带着一种困惑而奇怪的口吻说:“这娘们儿……”

这时他脑子里出现了后悔的想法,心想,干这种风流事儿不会有好结果的,又想起过去不知是在库普林的小说里还是在一本翻译小说里看到的话,说是爱情像煤炭,烧起来的时候,热得灼人,冷下来的时候,可以把人弄脏……他甚至很想哭一场,其实不是想哭,是想诉诉苦衷,对什么人说说,他干这事儿是身不由己,是命运让他这个可怜的中校这样对待爱情……后来他睡着了;等他醒来,忽然想道:“如果我不被打死的话,回来的路上一定还要去找阿拉。”

六十九

叶尔绍夫少校下工回来,在莫斯托夫斯科伊床铺前站下来,说:

“那个美国人听到广播,咱们在斯大林格勒英勇抵抗,粉碎了德国人的算盘。”

他皱了皱眉头,又说:

“还有莫斯科方面来的消息,说是解散了共产国际,不知是不是。”

“您怎么,疯啦?”莫斯托夫斯科伊注视着叶尔绍夫那聪明的、像寒冷而有点儿浑浊的秋水似的眼睛,问道。

“也许,那个美国人听错了。”叶尔绍夫说过这话,就用指甲挠起胸膛。“也许正相反,是共产国际扩大了。”

莫斯托夫斯科伊一生中认识不少这样的人,这些人就像电话机的膜片,能灵敏地反映全社会的理想、感情、见解。似乎俄罗斯从来没有一件大事是这些人不了解的。叶尔绍夫便是反映集中营公众思想与见解的这样一个表达者。但是他说的解散共产国际的消息,营里这位有影响的人物却丝毫不感兴趣。

主管过大兵团政治思想教育的旅级政委奥西波夫,对这个消息也漠然视之。奥西波夫说:

“古泽将军对我说:政委同志,由于您的国际主义教育,大家都溃逃啦,应该是用爱国主义精神,用俄罗斯精神教育人民。”

“怎么,还要为了上帝、沙皇、祖国吗?”莫斯托夫斯科伊冷笑道。

“这都是小事,”奥西波夫神经质地打着呵欠说,“这会儿问题不在于正统思想,而是德国人要活剥我们的皮,莫斯托夫斯科伊同志,亲爱的老人家。”

被苏联人叫做安得留沙的那个睡在第三层铺上的西班牙士兵,用英文把“斯大林格勒”写在一块小小的木板上,夜里看着这木板上的字,到早晨就把木板翻过来,不让搜查棚屋的人看到这上面的字。

基里洛夫少校对莫斯托夫斯科伊说:

“以前不赶着我去干活儿的时候,我天天躺在床铺上闲待着。现在我又为自己洗衣服,又嚼松木片治坏血病。”

受惩罚的党卫军分子诨称“快乐的小伙子”(他们在上工的时候总是唱着歌儿) ,他们找苏联俘虏的碴儿找得更厉害了。看不见的联系把集中营棚屋里的人和伏尔加河上的城市连接在一起。

可是大家都觉得共产国际是不起作用的。就在这时候,流亡者切尔涅佐夫第一次走到莫斯托夫斯科伊跟前。

他用手捂着空空的眼窝,谈起美国人偷听到的广播。

莫斯托夫斯科伊高兴起来,他太需要谈谈这个问题了。

“总而言之,这消息很不可靠,”莫斯托夫斯科伊说,“胡说八道,胡说八道。”

切尔涅佐夫扬起眉毛,这空眼窝上扬起的眉毛显得很不好看,露出困惑和神经衰弱的神气。

“为什么?”独眼睛的孟什维克问。“为什么不可靠?布尔什维克先生们创立了第三国际,也是布尔什维克先生们创立了在一个国家实行所谓社会主义的理论。这种统一实际上是胡闹。好比油炸冰块……盖奥尔基·瓦连季诺维奇在他晚年的一篇文章中写道:‘社会主义只有成为世界体系,成为国际体系,才能存在,否则根本不能存在。’”

“是所谓的社会主义吗?”莫斯托夫斯科伊问道。

“是的,是的,所谓的社会主义。苏联的社会主义。”

切尔涅佐夫笑了笑,并看到莫斯托夫斯科伊也笑了笑。他们相视而笑,是因为他们从不友好的话里,从嘲笑而带有敌意的语调中看到了自己的过去。

好像挖开了几十年的沉积层,他们年轻时互相厮杀的利刃露了出来。这次在法西斯集中营里的相会,不仅使他们想起多年的仇恨,也想起青年时代。

这个在集中营里的人,这个敌对分子和异己分子,也熟悉和热爱莫斯托夫斯科伊年轻时熟悉和热爱的东西。是他,而不是奥西波夫,不是叶尔绍夫,还记得第一次党代会期间的许多故事,记得只有他们两个人依然很感兴趣的一些人的名字。他们都很激动地回忆起马克思和巴枯宁的关系,回忆起列宁和普列汉诺夫说的有关温和的火星派和强硬的火星派的话。回忆起已经老眼昏花的恩格斯对待前去见他的俄国社会民主党的年轻人多么亲热,回忆起在苏黎世的柳博奇卡·阿克雪里罗德[65]有多么坏!

独眼的孟什维克觉得自己的所感也正是莫斯托夫斯科伊所感,就苦笑着说:

“很多作家写年轻时代朋友们见面,写得很动人,可是,年轻时代的敌人,像您和我这样经过风风雨雨的白了头发的老家伙,见了面又怎样呢?”

莫斯托夫斯科伊看到切尔涅佐夫的腮上挂着泪水。他们都明白,集中营里的死神能够把多年生活中的一切,把正确、错误、敌视很快地抹平和掩埋。

“是啊,”莫斯托夫斯科伊说,“在漫长的一生中一直跟你作对的人,也不由自主地成为你的生活的参与者了。”

“真奇怪,”切尔涅佐夫说,“在这狼窝里会这样见面。”他忽然又说:“多么奇怪的字眼:小麦,大麦,晴天雨……”

“啊,也是这集中营太可怕了,”莫斯托夫斯科伊笑着说,“与集中营相比,一切都好像很好,就连见到孟什维克也不觉得怎样了。”

切尔涅佐夫伤感地点点头。

“是呀,确实,够您受的。”

“法西斯主义呀,”莫斯托夫斯科伊说,“法西斯主义!这样惨无人道,我真无法想象!”

“您还有什么惊奇的,”切尔涅佐夫说,“您对恐怖手段早应该不觉得稀奇了。”就像一阵风吹跑了他们之间的伤感气氛和友好气氛。他们毫不客气地、恶言恶语地争论起来。

切尔涅佐夫的攻击之所以可怕,因为他说的不完全是无中生有。切尔涅佐夫把苏联建设过程出现的残酷现象和个别错误看作根本的规律性。他直截了当地对莫斯托夫斯科伊说:

“当然,你们都满足于一种看法,认为一九三七年的事是过火了,集体化期间是胜利冲昏头脑,你们敬爱的伟大领袖有点儿残酷和独断独行。然而实质正相反:正如你们常说的,斯大林是今天的列宁。你们总觉得,农村的贫穷和工人的无权是暂时的现象,是发展中的困难。你们这些真正的富农和垄断者,买农民的小麦,五戈比一公斤,再卖给农民,每公斤却卖一卢布,这就是你们的建设的基本原则。”

“就连你们孟什维克,你们这些流亡者都说了:斯大林是今天的列宁,”莫斯托夫斯科伊说,“那我们,也是从普加乔夫到拉辛[66]的历代俄罗斯革命者的继承人。拉辛、杜勃罗留波夫、赫尔岑的继承人不是孟什维克,不是逃亡国外的叛徒,而是斯大林。”

“是的,是的,是继承人!”切尔涅佐夫说。“您知道,在俄国立宪会议自由选举意味着什么吗?是在上千年奴化统治的国家里呀!一千年来,俄罗斯只自由了半年多点儿。我每次想到一九三七年的事,就想起另一项遗产,您该记得第三厅长官苏杰伊金上校,他串通杰加耶夫[67],佯装发动叛乱和平息叛乱,恐吓沙皇,想用这种办法把政权抓到手里。您认为斯大林是赫尔岑的继承者吗?”

“您怎么,真的那么糊涂吗?”莫斯托夫斯科伊问。“您怎么,当真认为不过是苏杰伊金吗?那么,伟大的社会变革,没收剥削者的财产,没收资本家的工厂,没收地主的土地,您没看到吗?这是继承谁的一套,是继承苏杰伊金那一套吗?还有普遍提高文化,还有重工业呢?还有最下等的人,还有工人和农民参与各项社会活动呢?这怎么,都是继承苏杰伊金的一套吗?您真可怜。”

“我知道,知道,”切尔涅佐夫说,“事实不容辩驳,但可以作各种解释。你们的元帅、作家、科学家、艺术家、人民委员都不听命于无产阶级。他们听命于国家。至于那些在车间和田野里干活儿的人,我想,就连您也未必把他们看作当家做主的人。他们又能当什么家,做什么主呀!”

他忽然俯身朝着莫斯托夫斯科伊,说:

“顺便说一句,在所有你们的人当中,我只看得起斯大林。斯大林是你们的泥瓦匠,你们却都怕干脏活儿!斯大林就知道:社会主义要想在单独取得胜利的一个国家里站得住脚,就要靠铁的恐怖手段,靠集中营,靠中世纪对待异端邪说的办法。”

莫斯托夫斯科伊对切尔涅佐夫说:

“先生,这些无耻谰言我们全听说过。不过,我应该坦率地对您说,您说这些话,说得特别无耻罢了。只有一种人,从小就生活在你家里那种地方,后来又被赶出去的人,才会这样诬蔑、这样诽谤。您可知道,这是什么人?……是奴才!”

他直直地看了看切尔涅佐夫,又说:

“说实在的,开头我真想共同回忆一下我们在一八九八年的团结,而不是一九〇三年的分裂。”

“想聊聊还没有把奴仆从家里赶出去那时候吗?”

可是莫斯托夫斯科伊当真火了。

“是的,是的,正是这样!被赶出去的、逃走的奴才!戴白手套的奴才!我们不掩饰,我们不戴手套。我们的手沾满鲜血,我们弄脏了手!这有什么!我们参加工人运动就没有戴普列汉诺夫的手套。你们戴着奴才手套又怎样?你们因为在《社会主义导报》上发表的文章得到几个赏钱?这儿集中营的英国人、法国人、波兰人、挪威人、荷兰人都相信我们!拯救世界靠我们的手!靠红军的力量!红军是自由的军队!”

“是这样吗?”切尔涅佐夫插话说。“一直是自由的吗?”

莫斯托夫斯科伊把两手举到切尔涅佐夫面前,说:

“您瞧瞧这手,没有戴奴才的手套!”

切尔涅佐夫朝他点点头,说:

“记得宪兵上校斯特列里尼科夫吗?他干什么也不戴手套:他就干脆代替被他打得半死的革命者写伪造的坦白认罪书。你们一九三七年的事为了什么?是为了准备同希特勒作战吗?这是斯特列里尼科夫还是马克思教导你们的?”

“您这些臭不可闻的话丝毫不使我觉得奇怪,”莫斯托夫斯科伊说,“您是不会说别的话的。您可知道,我确实感到奇怪的是什么?希特勒为什么把您关在集中营里?关您干什么?希特勒恨我们恨得要命。这是可以理解的。可是希特勒干吗要把您和您这类的人关在集中营里呀?!”

切尔涅佐夫笑了笑,他的脸又变得像开始谈话时那样子。

“这不是,关进来啦,”他说,“而且还不放呢。您给我说说情吧,也许会把我放了。”

但是莫斯托夫斯科伊不想开玩笑。

“您对我们这样仇恨,就不应该蹲在希特勒的集中营里。而且不光是您,还有这样的人。” 他指了指朝他走来的伊康尼科夫。

伊康尼科夫的脸上和手上沾满了泥浆。

他递给莫斯托夫斯科伊几张写满了字的肮脏的纸,说:

“看看吧,也许,明天就要死了。”

莫斯托夫斯科伊把几张纸塞到垫褥底下,气愤地说:

“我是要看看,怎么您要离开这个世界了?”

“您可知道,我听到了什么?咱们挖的基坑,是为了建造毒气工厂。今天已经开始浇灌混凝土地基了。”

“听说有这事儿,”切尔涅佐夫说,“过去还铺过宽轨。”

他回头看了看。莫斯托夫斯科伊心想,切尔涅佐夫关心的,是下工回来的人看到他和一个老布尔什维克谈得多么随便。他大概因为这一点就要在意大利人、挪威人、西班牙人、英国人面前夸耀了。尤其要在苏联战俘面前夸耀。

“这活儿咱们还继续干吗?”伊康尼科夫问道。“还参与制造恐怖吗?”

切尔涅佐夫耸耸肩膀,说:

“您以为咱们这是在英国吗?这八千人要是罢工,在一个钟头之内就会全部被杀害。”

“不,不能干,”伊康尼科夫说,“我不干,不干。”

“如果不干,转眼工夫就把您打死。”莫斯托夫斯科伊说。

“是的,”切尔涅佐夫说,“您可以相信这话,这位同志知道,在没有民主的国家里号召罢工,意味着什么。”

他和莫斯托夫斯科伊争论了一阵子,心绪很乱。他在巴黎自己家里说过多少次的一些话,现在在这希特勒的集中营里说出来,自己觉得很不实际,毫无意义。他听集中营囚犯们谈话,常常听到“斯大林格勒”这个词儿,不管是否合他的心意,现在斯大林格勒是和世界的命运连接在一起了。

一个年轻的英国人向他做了一个胜利的手势,说:

“感谢你们,斯大林格勒挡住了狂飙的飓风。”

切尔涅佐夫听到这话,感到很幸福、很激动。他对莫斯托夫斯伊科说:

“您该知道,海涅说过,只有傻瓜才把自己的弱点暴露给敌人。不过,好吧,我就做做傻瓜,您说得很对,我很清楚你们的军队所进行的斗争的伟大意义。一个俄国社会党人理解这一点是极难极难的,一旦理解了,又高兴,又自豪,同时又难过,又痛恨你们。”

他看着莫斯托夫斯科伊。莫斯托夫斯科伊觉得他那一只正常的眼睛也充满了血。

“不过,难道您就是在这里也没有亲身体验到,人没有民主和自由不能生活吗?您在家里忘记了这一点吧?”切尔涅佐夫问道。

莫斯托夫斯科伊皱起眉头。

“算啦,别再歇斯底里了。”

他回头看了看。切尔涅佐夫心想,莫斯托夫斯科伊是在担心,下工回来的人会不会看到流亡的孟什维克和他谈得多么随便。他大概因为这一点在外国人面前觉得不好意思了。尤其在苏联战俘面前觉得不好意思。

他那血红的空眼窝直直地盯着莫斯托夫斯科伊。

伊康尼科夫拉了拉从二层铺上垂下来的神甫的脚,用蹩脚的法语、德语和意大利语夹杂在一起问道:

“咱们在建毒气工厂了。神甫,我该怎么办?”

加尔季神甫用煤球似的眼睛打量着大家的脸。

“大家都在那儿干。我也在那儿干,”他慢慢地说,“我们是奴仆。上帝会饶恕我们的。”

“这是他的职业。”莫斯托夫斯科伊补充说。

“但这不是您的职业。”加尔季用责备的口气说。

伊康尼科夫马上接着说:

“是啊,是啊,米哈伊尔·西多罗维奇,从你们的观点来看,也是这样,不过我不想宽恕自己的罪过。不能说全怪那些强迫你干的人,你是奴隶,你没有罪,因为你不自由。我是自由的!我建造毒气工厂,我就对不起将来被毒气毒死的人。我可以说‘不干’!如果我有胆量不怕枪杀的话,有什么力量能强迫我干?我要说‘不干’!我不干,我就是不干!”

加尔季的手挨到伊康尼科夫的白头。

“把您的手给我。”他说。

“好啦,现在牧师就要开导因为骄傲而迷途的羔羊了。”切尔涅佐夫说。

莫斯托夫斯科伊听到他这话,也不由得怀着同感点了点头。

但是加尔季没有开导伊康尼科夫,他把伊康尼科夫那肮脏的手拉到嘴唇边,吻了吻。

七 十

第二天,切尔涅佐夫和红军战士巴甫柳科夫聊天,巴甫柳科夫是他结识的少数苏联人之一,现在在医务所做卫生员。巴甫柳科夫对切尔涅佐夫诉说,很快就要把他从医务所赶出去,叫他去挖基坑了。

“这都是党员们搞的,”他说,“他们看不惯我占着一个好位置,认为我是行过贿的。他们当清洁工,厨房、盥洗间里到处都安排他们的自己人。老大爷,您该记得和平时期的情况吧?区委都是自己人,工会也都是自己人。不是吗?在这儿他们也搞自己的一套班子,厨房里都是自己的,好东西给自己人吃。他们供养老布尔什维克,像在疗养院里一样,可是您,就像狗一样,没人理睬,谁也不朝您看一眼。难道这公平吗?您也是给苏维埃政权做牛做马了一辈子嘛。”

切尔涅佐夫很不好意思地告诉他说,自己离开俄罗斯已经二十年了。他已经发现,“侨民”、“国外”这样一些词儿一下子就能使苏联人和他疏远。但是巴甫柳科夫听了切尔涅佐夫的话并没有紧张起来。

他们蹲在一堆木板上。巴甫柳科夫宽鼻子,宽额头。切尔涅佐夫心想,这真是人民的儿子。巴甫柳科夫朝在混凝土塔楼上走来走去的哨兵那边望着,说:

“我没有别的办法,只有参加新编的志愿军,或者装做生病。”

“就是说,为了活命吗?”切尔涅佐夫问。

“我根本不是富农,”巴甫柳科夫说,“也没有做过苦役犯人,不过我对共产党还是很不满意。不能自由地干什么事。种田由不得自己,娶老婆由不得自己,干什么工作由不得自己。人变得像鹦鹉一样。我从小就想自己开一座商店,为的是在里面什么都可以买到。商店里有小吃部,货物齐全,请买吧:想喝烧酒,有烧酒;想吃烤鸭,有烤鸭;想喝啤酒,有啤酒。您猜,我卖东西会怎样?很便宜!我还要在小吃部卖乡下吃食儿。请吧!烤土豆!牛油拌大蒜。酸白菜!您猜,我会卖什么样的小菜:骨头汤!骨头汤在锅里翻滚,请吧,来一碗,加一根骨头,还有黑面包,当然,还有盐。到处是皮椅子,免得生虱子。请坐下,歇会儿,有人服侍你。这事儿只要我一说出来,马上就会把我送到西伯利亚。可是这会儿我想,这样做生意对人民有什么特别不好的呢?我定的价钱一定会比国家低一半。”

巴甫柳科夫侧眼看了看切尔涅佐夫,又说:

“在我们的棚屋里,有四十个小伙子报名参加志愿军啦。”

“为什么?”

“为了一碗菜汤,为了一件大衣,为了不至于干活儿累死。”

“还有什么原因吗?”

“有些人是有想法。”

“什么想法?”

“各种各样的想法。有的是看到在集中营里有人被杀害。有的是受够了农村的贫穷。他们忍受不了共产主义,”切尔涅佐夫说,“这太卑鄙了!”

这个苏联人带着好奇的神气看了看这个侨民,这个侨民也看出他这种带有嘲笑与大惑不解意味的好奇神情。

“可耻,下流,恶劣,”切尔涅佐夫说,“不是算陈年老账的时候。算账也不应该这样算。自己对不起自己。对不起自己的土地。”

他从木板上站起身来,用手弹了弹屁股上的土。

“不可能有人说我热爱布尔什维克,真的,但现在不是时候,不是算账的时候。不要去参加叛徒弗拉索夫的军队。”

他忽然说不出话来,片刻之后又说:

“您听着,同志,别去。”

他因为又像青年时代那样说出了“同志”这个词儿,再也掩盖不住自己的激动,而且也不再掩盖自己的激动,喃喃地说:

“我的天啊,天啊,我能不能……”

……火车驶离站台。周围烟雾腾腾,其中有灰尘,有丁香花香和春季里城市的污水气味,有机车的灰烟,还有车站食堂厨房里冒出来的油烟。

信号灯越来越远,越来越远,可是后来好像在其他绿灯和红灯之间停住不动了。

一个大学生在站台上站了一会儿,朝侧门走去。一个女子也像他一样,感情涌来失去自制,用胳膊搂住他的脖子,吻他的额头、头发……他跨上车,一阵幸福感在心头涌起,头脑晕乎乎的,他觉得这是开始,将是他内容充实的整个一生的开端……

他在离开俄罗斯前往斯拉武塔的路上,一再回想起这个黄昏。他在巴黎的医院里,做完青光眼手术之后,常常想起这个黄昏。在他走进他供职的银行那阴凉而幽暗的门洞时,也常常想起这个黄昏。

关于这一点,像他一样从俄国逃往巴黎的诗人霍达谢维奇写过一首诗:

拄着拐杖浪游,不知为何我想起你;

红轮马车在奔驰,不知为何我想起你;

晚上把蜡烛点起,不知为何我想起你;

不论天上人间,发生何事,我都会想起你……

他真想再走到莫斯托夫斯科伊跟前,问问他:

“您认识娜塔莎·萨顿斯卡娅吗?她还活着吗?这几十年来您一直跟她生活在一块土地上吗?”

七十一

在晚上集会点名时,汉堡窃贼凯泽戴着黄手套,穿着淡黄色的贴口袋方格上衣,兴致很好。他用发音不准的俄语小声唱着歌儿:“假如明天发生战争,假如明天踏上征程……”

他红里透黄的委顿的脸和褐色的无神的眼睛在这天晚上显得十分和善。他雪白而光滑的肥厚手掌和能够把一匹马掐死的手指头,不时拍拍犯人们的肩膀和脊梁。他要杀人也很随便,就好像为了开玩笑使个绊脚把人绊倒。杀过人之后,他那股兴奋劲儿也只能持续不大的一阵子,就好像小猫和一只五月金龟子玩了一会儿。

他杀人多数都是根据突击队头头德罗津哈尔的指示。德罗津哈尔主管东区段的卫生防疫。

干这方面的事情,最困难的是把死者的尸体拖去火化,不过凯泽从来不干这种事,谁也不敢叫他干这种事。德罗津哈尔是有经验的,决不让病人病得非要用担架把他们抬到杀人的地方。

凯泽并不催促要被杀死的人,不对他们恶言恶语,也从没有推来搡去,拳打脚踢。凯泽已经有四百多次登上特种囚室的两级混凝土台阶,总是对接受手术的人特别感兴趣:他很喜欢那种目光,那目光中有恐惧,有焦急,有驯顺,有痛苦,有胆怯,还有注定要死的人看到杀他的人进来时所流露出来的极其好奇的神情。

凯泽干这种事就像吃家常便饭,他自己也不懂,他为什么偏偏喜欢这种家常便饭。特种囚室其实很单调:一个凳子,灰色的石头地面,一根水管,一个水龙头,一段橡皮管,一张小桌,上面摆一个记事本。

操作起来极其简单平常,说起来总是用半开玩笑的口吻。如果操作过程中用了手枪,凯泽就说“往脑袋里塞了一粒咖啡豆”;如果注射了石碳酸,凯泽就说“加了一点儿长生水”。

凯泽觉得既奇怪又简单,咖啡豆和长生水能够揭示人生的秘密。

他那褐色的像用塑料做成的眼睛似乎不是活人的眼睛,像是硬化了的黄褐色松脂……每当他那硬僵僵的眼睛里出现快活的神气,别人都觉得十分可怕,就好像一条鱼一下子游到一颗沉在水里、被沙埋住一半的死树跟前,忽然发现这黑黑的、黏黏的庞然大物还有眼睛、牙齿、触角,觉得十分可怕。

在这集中营里,凯泽有一种优越感,感到自己比住在棚屋里的艺术家、科学家、革命家、将军、传教士都优越。这倒是不在于咖啡豆和长生水。这是一种很自然的优越感,这种优越感使他十分得意。

使他感到得意的不是他那巨大的体力,不是他能不顾一切地去作案,去撬保险柜。他很欣赏自己的精神和聪明,他是令人捉摸不透的,是复杂的。他喜怒无常,似乎不合情理。在春天把秘密警察挑选的一些苏联战俘赶进特种棚屋的时候,凯泽请他们唱他们喜欢的歌儿。

有四个目光悲戚、手臂肿胀的苏联人唱道:

我的苏莉科,你在何方?

凯泽愁眉苦脸地听着,望着站在边上的一个高颧骨的人。凯泽由于敬重歌唱者,没有打断歌唱,但等到歌声一停,他就对高颧骨的人说,他在合唱时没有唱,现在要他独唱。凯泽看到这个人肮脏的军服领口上带有拆掉的领章的痕迹,问道:

“你听懂了吗,少校?”[68]

那人点了点头,表示懂了。

凯泽抓住那人的领口,轻轻摇晃了几下,就像摇晃出了毛病的闹钟那样。那人朝凯泽的颧骨捣了一拳,并且骂了两声。

看样子,这个苏联人要完了。但是这个特种棚屋里的头头儿并没有把叶尔绍夫少校打死,而是把他带到角落里靠窗的一个铺上。这个铺空着,是专门留给凯泽喜欢的人的。就在这一天,凯泽还给叶尔绍夫送来煮熟的鸭蛋,哈哈笑着递给他,说:“吃吧,能让你唱歌好听!”[69]

从那时候起,凯泽对待叶尔绍夫一直很好。同棚屋的人也都很尊敬叶尔绍夫,他除了刚强不屈之外,性格也非常随和开朗。在叶尔绍夫那一次拒绝唱歌之后,有一个当时唱歌的人很生叶尔绍夫的气,那就是旅政委奥西波夫。

“不合群的人。”他说。

也是在那件事情之后不久,莫斯托夫斯科伊就管叶尔绍夫叫思想领袖了。

除了奥西波夫之外,对叶尔绍夫不怀好感的还有一个孤僻、沉默然而了解每个人底细的战俘柯季科夫。柯季科夫是一个没有什么特色的人,声音没什么特色,眼睛、嘴唇也没什么特色。不过,正因为他太没有特色了,这种没有特色似乎倒成了鲜明的特色。

这一天凯泽在晚间点名时的快活表情引起许多人高度的焦虑和恐惧。棚屋里的人总是觉得事情不妙,恐惧、不安和不祥感总是在心里,有时强些,有时弱些。

在晚间点名快要结束的时候,特别棚屋里进来八名营警—是戴着滑稽可笑的小圆帽、缠着黄色臂章的“卡波”。从他们的脸可以看出来,他们吃的不是营里的大锅饭。

他们的头儿是一个浅色头发的高个儿美男子,身穿拆掉了领章的铁灰色军大衣。大衣下面露出锃亮的漆皮靴子,那靴子泛着宝石一样的亮光,因此很像是白色的。

这是营内警察队长凯尼克,是党卫军分子,因为刑事犯罪丢了职务,被关在集中营里。

“起立!”凯泽喊道。

开始搜查。“卡波”们熟练得就像工厂里的工人,敲敲桌子,听听是不是挖空了,抖一抖破布,又快又仔细地摸摸衣服上的缝,检査检查饭盒。

有时他们开开玩笑,用膝盖顶一下某人的屁股,说:“你好。”有时“卡波”们把搜到的字纸、笔记本或保险刀片递给凯尼克看,问他怎样处理。凯尼克把手套一扬,表示这些搜到的东西没有意思。在搜查的时候,囚犯们一直排成队站着。莫斯托夫斯科伊和叶尔绍夫站在一起,望着凯尼克和凯泽。这两个德国人像是铁铸的一般。莫斯托夫斯科依头脑发晕,身子摇晃了几下。他用手指着凯泽,对叶尔绍夫说:

“有这样的人!”

“高等民族嘛。”叶尔绍夫说。他不希望站在近处的奥西波夫听见,凑到莫斯托夫斯科伊的耳朵上说:

“不过我们有些人也够呛!”

切尔涅佐夫虽然没有听清他们的谈话,但也接茬说:

“任何民族都有神圣的权利,都可以有英雄,有神圣的人和卑鄙的人。”

莫斯托夫斯科伊对着叶尔绍夫,但说的话不光是回答他的:

“当然,我们也有坏蛋,不过德国刽子手有一种很独特的神气,只有德国人才会有。”

搜查结束了。发出休息的口令。囚犯们开始往床上爬。

莫斯托夫斯科伊躺下来,把两腿伸直。他想起,他还没有检查一下,搜查之后他的东西是不是全在呢,于是哼哧着欠起身子,开始检查自己的东西。

似乎不是少了围巾,就是少了包脚布。但是他找到了围巾,也找到了包脚布,不过他还是没有放下心来。一会儿,叶尔绍夫走到他跟前,小声说:

“‘卡波’涅泽尔斯基透话说,咱们这个区段的人要拆散,一部分人留在这儿继续受审查,大多数人都到普通集中营里去。”

“那有什么,”莫斯托夫斯科伊说,“管它呢!”

叶尔绍夫在铺上坐下来,声音很轻然而很清楚地说:

“莫斯托夫斯科伊同志!”

莫斯托夫斯科伊用胳膊肘支起身子,看了看他。

“莫斯托夫斯科伊同志,我想干一件大事,要和您谈谈这件事。要是失败了,那就很麻烦!”

他小声说起来,莫斯托夫斯科伊听着听着,激动起来,就好像有一阵清风向他吹来。

“时间很宝贵,”叶尔绍夫说,“如果斯大林格勒被德国人攻下来,很多人又要泄气了。从基里洛夫这样一些人可以看出来。”

叶尔绍夫建议成立一个战俘的战斗团体。他凭记忆说了说纲领要点,就像念文稿一样:

“……加强集中营里的苏联人的团结,加强纪律,清除叛徒,破坏敌人部署,在波兰人、法国人、南斯拉夫人、捷克人中间建立斗争委员会……”

他望着床铺顶上棚屋的昏暗处,说:

“有几个兵工厂的同志,他们告诉我,可以搞武器。咱们的组织会很快扩大。联络几十个集中营,成立许多战斗小组,团结德国的地下工作者,制裁叛徒。最终的目的是全面起义,统一自由的欧洲……”

莫斯托夫斯科伊重复说:

“统一自由的欧洲……啊,叶尔绍夫呀,叶尔绍夫。”

“我不是瞎说。咱们说了,就干起来。”

“我参加。”莫斯托夫斯科伊说。他又一面摇着头,一面重复说:“自由的欧洲……在咱们的集中营里就有一个共产国际分部,分部有两个人,其中一个不是党员。”

“您又懂英语,又懂德语,又懂法语,联系的方式多得很,”叶尔绍夫说,“何必还要共产国际:各国囚犯,联合起来!”

莫斯托夫斯科伊望着叶尔绍夫,说出了他早就忘记的话:

“人民的意志!”

他觉得很奇怪,为什么偏偏会忽然想起这话。

叶尔绍夫说:

“应该跟奥西波夫和兹拉托克雷列茨上校谈谈。奥西波夫是力量很大的人物!不过他不喜欢我,还是您和他谈谈。我今天就和上校谈谈。咱们组成四人小组。”

七十二

叶尔绍夫少校的脑子日日夜夜紧张不懈地工作着。

他在考虑囊括德国所有集中营的地下工作计划,考虑地下组织相互联系的技术问题,记熟各劳动营和集中营以及一些火车站的名称。他考虑编制密码,如何利用营里的文书把一些组织者列入调动名单,使他们可以在各营之间串通。

他的心中充满了幻想。成千上万的地下工作者大力宣传,成千上万的英雄暗地进行活动,可以创造条件武装起义,占领各集中营。起义者可以夺取守卫各营的高射炮,把高射炮变为反坦克炮和反步兵炮。应该事先物色高射炮手,为将来夺取的各门高射炮准备炮手。

叶尔绍夫少校很了解集中营里的情况,知道收买、恐惧所起的作用,知道饥饿的力量,看到过很多人脱下清白的军服,换上叛徒弗拉索夫部队带肩章的蓝大衣。

他见过低三下四、背信弃义、巴结顺从;他见过比恐惧更甚的恐惧,见过一些人在可怕的侦讯官员面前吓得怎样发呆。

这位衣衫破烂的被俘的少校毕竟没有沉醉在幻想中。德国人在东线急速推进的阴暗时期,他用乐观、大胆的话鼓励同志,劝浮肿的人千方百计保重自己的身体。他对强权的鄙视一直未消失,未减弱,一直很强烈。

很多人接触过叶尔绍夫之后,感到他身上有一种令人快活的热情—这是人人需要的、平常又宜人的温暖,燃烧白桦木柴的俄罗斯壁炉发出来的温暖就是这样的。

也许,正是这种感人的温暖,而不光光是才智和胆识,使叶尔绍夫少校成为苏联战俘的头儿。

叶尔绍夫早就明白,莫斯托夫斯科伊是第一个可以信得过的人,可以对他敞开自己的想法。叶尔绍夫睁着眼睛躺在铺上,看着粗糙的木板顶棚,就像在棺材里望着棺材盖,他的心怦怦直跳。

他这一生的三十三年以来,从来没有像在这里,在集中营里这样感到自己的力量。

他在战前过的日子很不好,他的父亲是沃龙涅什省的农民,在一九三〇年被划为富农。这时候他在军队里服务。

他没有和父亲断绝关系。他不能进军事学院,虽然他的入学考试成绩优秀。他好不容易在军事学校毕了业,被分配到区兵役局。他的父亲成了流动人口,这时候带着一家人住在北乌拉尔。叶尔绍夫请了假去看父亲。从斯维尔德洛夫斯克起要乘二百公里的窄轨火车。路两旁是一片片的森林和沼地,一堆堆待运的木材,一道道集中营的铁丝网,一座座棚屋和泥屋,还有高高的看守塔楼,就像一簇簇高脚毒蘑菇。火车两次被拦住,押送队要搜查一名逃犯。夜里火车停在一个会让站上,等待前方开来的火车,叶尔绍夫没有睡,听着警犬的吠叫声、哨兵的哨子声。原来会让站附近就是一座很大的集中营。

叶尔绍夫第三天才到达窄轨铁路的终点站。虽然他的领子上戴着中尉领章,证件和乘车证也都是符合规定的,但在检查证件的时候他还是担心有人会对他说:“喂,把东西带着!”把他带到集中营里去。似乎这地方的空气也被铁丝网关住了。

后来他坐上一辆顺路的吨半汽车,走了七十公里。道路从沼地中间穿过。汽车是“奥格普”国营农场的,叶尔绍夫的父亲就在这个农场干活儿。车上很拥挤,上面坐的都是干活儿的流动人口,被调到一处集中营分场去伐木。叶尔绍夫试着向他们询问,但是他们只用一两个字回答,看样子,是害怕他的军装。

傍晚,汽车来到紧靠林边与沼地边缘的一个小村子。他永远记住了北方集中营沼地上的宁静而柔和的黄昏。在暮霭中,一座座小屋完全成了黑的,似乎是在焦油里煮过的。

他走进一座土屋,晚霞随他一起进来,可是迎接他的是潮气、闷热、穷人的食物、衣服和被窝的气味,热乎乎的烟气……

在黑暗中出现了他的父亲,一张瘦削的脸,一双很好的眼睛,那双眼睛流露出的一种无法描述的神情使叶尔绍夫大吃一惊。

一双又老又瘦的粗糙的手臂搂住儿子的脖子。搂住年轻指挥员脖子的这一双受尽磨难的老人的手不住地抽搐着,从中可以感觉出老人在畏畏怯怯地诉苦,是那样痛苦,那样恳切地求助,所以叶尔绍夫只能用一点来回答这一切:他哭了。

后来他们在三座坟前站了一阵子。母亲是第一个冬天死的,大姐阿纽塔死在第二个冬天,妹妹玛露霞死在第三个冬天。

集中营边沿的坟地和村子连在一起了。茅屋墙脚下、土屋斜面上、坟包上、沼地土丘上生长的都是一样的青苔。妈妈和姐姐、妹妹就要一直待在这片天空之下了,不论是冬天,严寒冻实沼地的时候,不论是秋天,坟地上堆满沼泽里冲来的黑糊糊的冲积物的时候。

父亲和不说话的儿子站在一起,也不说话,后来抬起眼睛,看了看儿子,把两手一摊,说:

“死去的,活着的,你们都原谅我吧,我没有把我爱的人保护住。”

夜里,父亲说起来。他说得很平静,声音不高。他说的事情只能用平静的口气来说,如果痛哭、流眼泪,是说不下去的。

在铺了报纸的箱子上,放着儿子带来的点心,还有一瓶酒。老人家在说,儿子坐在旁边,听着。

父亲说起饥饿,说起乡亲们的死,说起饿疯了的老妇人,说起小孩子,说孩子们的身体变得比三弦琴、比小鸡都轻。说村子里日日夜夜都能听到饥饿的哭叫声,村子里许多人家的门窗都钉死了。

他对儿子说,那年冬天他们坐着破漏的货车在路上走了五十天,一些死去的人在车上跟活人一起待了很多天。他说了说流浪者怎样长途跋涉,女人还要抱着孩子。妈妈也这样跋涉过,在酷暑中走路的时候曾经昏过去。说了说他们在冬天怎样被带到这里,既没有草棚,又没有土屋,他们又是怎样重新过起日子,怎样生篝火,拿树枝落叶当床铺,在锅里熔化雪水,怎样掩埋死者……

“这都是斯大林的主意呀。”父亲说。他的话里没有愤怒,也没有恼恨的意味。老实人谈到强大的、无法改变的命运时,都是这样。

叶尔绍夫探亲回来之后,写了一份申请书给卡里宁,要求格外开恩饶恕他无罪的父亲,要求准许老人家上儿子这儿来。可是申请书还没有到莫斯科,叶尔绍夫就被上级叫了去,因为有信来告发他去乌拉尔的事。

叶尔绍夫被军队开除了。他来到建筑工地,打算挣些钱,再去看父亲。可是不久就从乌拉尔来了一封信,报告父亲的死讯。

战争开始后的第二天,预备役中尉叶尔绍夫便应召进了军队。

在罗斯拉夫利战役中,他接替牺牲的团长,把溃散的人召集起来抗击德军,打退渡河的敌人,保证了统帅部后备重炮部队的撤退。

压在他肩上的担子越重,他的肩膀越是强壮有力。他原来也没想到自己会是一个强者。原来,驯顺与他的天性格格不入。压迫越强,越凶狠,他的斗志越强烈。

有时他问自己:为什么他这样痛恨弗拉索夫分子?弗拉索夫分子的号召书所写的事,正是他的父亲所说的。他知道这都是真实的。但是他知道,这些真实的东西到了德国人和弗拉索夫分子嘴里就成了诬蔑。

他觉得道理很清楚,他和德国人斗争,就是为苏联的自由生活而斗争,战胜希特勒,也就是战胜导致他的父母、姐妹早死的死亡营垒。

叶尔绍夫百感交集—在这儿,履历表失去作用,他成了强者,别人都听他的。在这儿,高级头衔、勋章、特种部队、第一科、人事处、鉴定委员会、区委的电话、政治处副处长的意见,全没有意义了。

莫斯托夫斯科伊有一天对他说:

“这是海涅早就说过的:‘脱去自己的衣服,我们都是光光的身子……’但是,一个人脱去礼服,露出虚弱、可怜的身子,另外一些人却被窄小的衣服束缚着,等他们把衣服脱去,才能看到,原来真正的力量在这儿!”

叶尔绍夫所幻想的,已成为今天要做的事情,于是他进一步考虑:该让谁知道,让谁参加。他凭着自己所了解的一些人的长处和短处,逐一思索、掂量。

谁可以进入地下工作指挥部?在他的脑子里出现了五个名字。有些生活上的小缺点,性格上的小怪癖,一切都从新的角度出现在他的脑海里,微不足道的事如今也重要起来。

古济有将军头衔的威望,但是他优柔寡断,胆小怕事,看样子文化水平也不高,如果有聪明能干的副手和参谋长,他才行。他指望指挥员们服侍他,供养他,而且认为这种服侍是理所当然的,不必感谢。他想念自己的厨师似乎比想念老婆孩子的时候多。他常常谈起打猎,又是野鸭,又是野鹅,回忆在高加索军中打猎的情形,打野猪,打山羊。看来他很爱喝酒,也很爱吹牛。常常谈起年的一些战役,周围的人都是不对的,左邻的将军不正确,右邻的将军也不正确,古济将军永远正确。他从来不会责怪最高军事领导的失误。为人处事圆滑,精细,像一个很世故的小吏。总而言之,如果依照叶尔绍夫的意见,他连一个团也不会交给古济将军指挥,更别说一个军了。

旅政委奥西波夫很聪明。有时他忽然会用嘲笑的口吻说在异国的领土上作战要尽量少流血,流露出很悲观的神气。可是过一个小时之后,他又十分坚决地批评起抱着怀疑态度的人,说教起来。然而到第二天,他又会翕动着鼻孔,说:

“真的,同志们,咱们飞得太高,太远,太快啦,这样是不切实际。”

他说起战争头几个月的失败,说得很有道理,但并不感到痛心,就像一名棋手说起一局败棋。他和人说话很随便,毫不拘束,但他的坦率是假装的,不是真正的同志间的坦率。他真正感兴趣的是跟柯季科夫谈话。

这位旅政委为什么对柯季科夫感兴趣?

奥西波夫经验丰富。善于了解人。这种经验非常有用,地下工作指挥部少了奥西波夫不行。不过他的经验不光可以成事,也可以碍事。有时奥西波夫说起一些著名军事人物的可笑轶事,直呼他们的名字,如:谢苗·布琼尼、安德柳什卡·叶廖缅科。有一天,他对叶尔绍夫说:“图哈切夫斯基、叶尔罗夫、布柳赫尔犯的错误,跟你我一样。”

可是基里洛夫对叶尔绍夫说,在一九三七年奥西波夫担任军事学院副院长时,毫不留情地揭发过几十个人,宣布他们是人民的敌人。他很怕生病,常常摸摸自己的头,把舌头伸出来,侧着眼睛看看,有没有舌苔。看样子,他倒是不怕死。

兹拉托克雷列茨上校是一个郁郁寡欢的老实人,是战斗部队的团长。他认为,最高领导在一九四一年的撤退方面犯了错误。大家都能感觉出他在战斗中的指挥能力和作战能力。他的身体十分强壮,声音也刚强有力,这样的声音才能喝止逃跑,发动进攻。他很喜欢骂娘。

他不喜欢解释,喜欢干脆利落地下命令。很讲义气。可以把饭盒里的菜汤倒给士兵。不过他太粗暴。人们常常能感觉出他的厉害。在工作中都要听他的,他大喝一声,谁也不敢不听。谁也别想糊弄他,他决不马虎。可以和他共事。但是他太粗暴了!

基里洛夫倒是个聪明人,但是思想上有些马马虎虎。什么问题他都能看得出来,可是对一切都懒得去问,睁一只眼,闭一只眼……他对一切很淡漠,对人没什么热心,但是原谅人的缺点和卑劣。他不怕死,有时候还很想死呢。

他说起撤退,说得似乎比谁都有道理。他不是党员,有一次他说:

“我不相信共产党会让人变好。在历史上还没有这样的事。”

他似乎对一切都十分淡漠,但是夜里有时在床上哭,对叶尔绍夫的问话很久没有回答,后来低声说:

“俄罗斯我是很爱的。”

他是一个很容易打交道的人,很随和。有一天他说:

“啊,我多么想听听音乐呀。”

昨天他带着傻笑的神气说:

“叶尔绍夫,您听着,我来念一首小诗。”

叶尔绍夫不喜欢这首诗,但他却记住了这首诗,这首诗也不管好歹钻进了他的脑子:

好同志,在要死的时候,

你不要向人呼救。

最好趁你的血还冒热气,

让我在这血上暖暖手。

别像小孩子,别怕,别悲怆,

你只是被打死,不是受伤。

最好把毡靴脱给我,我还要去打仗。

这诗是不是他自己写的呢?不行,不行,基里洛夫不能进指挥部。他怎么能带动别人呀,他自己也未必能行。

还是莫斯托夫斯科伊!他学识渊博,意志坚强。据说,在审讯中他始终刚强不屈。不过,说也奇怪,没有一个人是叶尔绍夫挑不出毛病的。前几天他就责备过莫斯托夫斯科伊:

“莫斯托夫斯科伊同志,您干吗要跟那些骗子磨嘴皮,比如,跟那个绿眼睛的伊康尼科夫,跟那个逃亡的独眼睛坏蛋,有什么好说的?”

莫斯托夫斯科伊笑了笑,说:

“您以为我的立场动摇了吗?以为我会成为教徒或者‘孟什维克吗’?”

“谁知道呢,”叶尔绍夫说,“是臭东西,最好别去碰。这个伊康尼科夫一直待在咱们的集中营里。一旦德国人把他传去审讯,他就会出卖自己,出卖您,出卖跟他接近的人……”

得出的结论是这样:对于做地下工作,没有理想的人。他需要衡量一个人的长处和弱点。这并不难。但只有根据一个人的本质,才能判断这个人是否合适。本质是无法衡量的,只能推测和感触。于是他就从莫斯托夫斯科伊开始。

七十三

古济少将呼哧呼哧喘着粗气走到莫斯托夫斯科伊跟前。他磕碰着脚后跟,哼哧着,撅着下嘴唇,皮肤的褐色皱褶在脸颊和脖子上哆嗦着—这些动作、姿势、声音都是他从往日肥胖时保留下来的,在他今天这样瘦弱的时候,这¸€切显得十分奇怪。

“您是长辈,”他对莫斯托夫斯科伊说,“我是乳臭未干的孩子,我给您提意见,就好比一名少校教训一位上将。不过我要直说:您不该跟那个叶尔绍夫一起搞什么各民族联合,他是一个底细不明的人。缺乏军事知识。论水平是个尉官,可是一心想当总指挥,想给上校们当当老师。应该离他远点儿。”

“阁下,您这是胡扯。”莫斯托夫斯科伊说。

“当然,是胡扯,”古济哼哧着说,“当然是胡扯。有人告诉我,在普通棚屋里昨天有十二个人报名参加那个什么……俄罗斯解放军。可以算算看,其中有几个是富农?我对您说的不光是我个人的意见,还代表一个很有政治经验的人。”

“这个人也许是奥西波夫吧?”莫斯托夫斯科伊问。

“就算是他。您是搞理论的人,您不了解这里面所有的卑鄙龌龊。”

“您这话可是真奇怪,”莫斯托夫斯科伊说,“您似乎是要告诉我,在这儿只能对人保持警惕性,别的什么都不行了。谁能有这样的先见之明!”

古济静静听着他自己支气管的呼哧声和胸中突突的心跳声,非常痛心地说:

“我看不到自由了,看不到了。”

莫斯托夫斯科伊望着他的背影,使劲用手掌拍了一下膝盖—他恍然大悟,他在搜查时为什么出现了担心和焦虑的感觉:原来伊康尼科夫给他的几张纸不见了。

他在纸上写的是什么呀?也许叶尔绍夫说得对,卑劣的伊康尼科夫参与了暗害活动,把这几张纸塞给了他。他在纸上胡写了些什么呢?

他走到伊康尼科夫床铺跟前。但伊康尼科夫不在这儿,旁边的人也不知道他上哪儿去了。这一切—几张纸不见了,伊康尼科夫不在床铺上—一下子使他明白了:他毫无顾忌地跟这个疯疯傻傻的寻神派教徒交谈,太轻率了。

他和切尔涅佐夫争论过,可是,实在说,连争论也不值得,还有什么好争论的呀。要知道,伊康尼科夫是当着切尔涅佐夫的面把几张纸交给他的,这样一来,既有吿密者,又有见证人了。

他的生命本来是革命事业和斗争所需要的,但是他也可能毫无意义地把生命丢掉。

“真是老糊涂了,竟跟一些渣滓打起交道,就在需要干一番事业,干一番革命事业的时候,偏偏要把自己葬送掉。”他这样想着,心里越来越痛苦不安了。

他在洗东西的地方碰到奥西波夫:这位旅政委就着暗淡的灯光下在铁皮水槽上洗裹脚布。

“碰到您,太好啦,”莫斯托夫斯科伊说,“我要和您谈谈。”

奥西波夫点了点头,回头看了看,在腰侧擦了擦湿漉漉的手。他们就在水泥墙根上坐下来。

“我一直是这么想,处处可能会有人使坏点子。”当莫斯托夫斯科伊谈起叶尔绍夫的时候,奥西波夫这样说。他用自己的湿手掌抚摩了两下莫斯托夫斯科伊的手。

“莫斯托夫斯科伊同志,”他说,“我很佩服您的果敢。您是老布尔什维克,是列宁的战友,对于您不存在年龄问题。您是鼓舞我们所有的人的榜样。”

他小声地说:“莫斯托夫斯科伊同志,我们的战斗组织已经建立起来了,我们决定暂时不对您说这件事,我们是想爱护您的生命,不过,看起来,列宁的战友不服老。我要直率地告诉您:我们不能信任叶尔绍夫。正如大家说的,他的根子不正:富农出身,怀有杀亲之仇。不过我们是现实主义者。目前没有他不行。他现在混得人缘很好。不能不考虑这一点。您比我清楚,党在很长的阶段中怎样善于利用这一类人。不过您应当知道我们对他的看法:能暂时利用,就暂时利用。”

“奥西波夫同志,不论叶尔绍夫走到什么地步,我都不怀疑他。”

可以听到水滴落到水泥地上的声音。

“莫斯托夫斯科伊同志,是这样,”奥西波夫说,“我们没有什么事情需要瞒着您。这儿有莫斯科派来的一位同志。我可以说出他的名字:柯季科夫。这也是他对叶尔绍夫的看法,不仅是我的看法。他的意见对于我们所有的共产党员就是法律,在特殊环境中就是党的命令,斯大林的命令。不过,我们要和您喜欢的那个人,和那位有影响的人物一起工作,决定了,就会那样做。要紧的只是一点:要做现实主义者、辩证唯物论者。不过,用不着我们来教训您。”

莫斯托夫斯科伊没有作声。奥西波夫抱住他,吻了他的嘴唇三下。他的眼睛里涌出泪水。

“我吻您,把您当做我的父亲,”奥西波夫说,“我真想为您祝福,就像小时候妈妈为我祝福那样。”

于是莫斯托夫斯科伊觉得,那种使人难受、使人痛苦的世事复杂的感觉消失了。他又像在年轻时那样,觉得世界是光明的、单纯的,世界上的人分成了自己人和敌人。

夜里,党卫军来到特别棚屋,带走了六个人。其中有莫斯托夫斯科伊。

[1]德国国家社会主义工人党,即纳粹党。

[2]“卡波”(德语:kapo)也是集中营里的囚犯,不一定是犹太人,最后往往也得死,但在集中营里他们会临时担任一些管理其他囚犯的特殊工作。

[3]原文为法语。

[4]同上。

[5]伊万·叶菲莫维奇·彼得罗夫(1896—1958),苏联大将,卫国战争期间敖德萨保卫战和塞瓦斯托波尔保卫战的领导者。

[6]安德烈·伊万诺维奇·叶廖缅科(1892—1970),二战结束时的苏联十大方面军司令员之一,一九四二年底指挥斯大林格勒方面军坚守成功。

[7]扬·库贝利克(1880—1940),捷克著名小提琴家、作曲家,以其精湛的技巧、完美的音准和高贵饱满的演奏风格著称。

[8]弗里德里希·保卢斯(1890—1957),法西斯德国陆军元帅,一九四二至一九四三年指挥第六集团军参与斯大林格勒战役,陷入重重包围后被俘投降。

[9]杰尔查文(1743—1816),俄国杰出诗人,主要作品有颂诗《费丽察颂》《攻克伊兹梅尔要塞》等。

[10]阿克萨科夫(1791—1859),俄国作家,代表作有《家庭记事》《巴格罗夫孙子的童年》等,作品带有自传性质。

[11]济宁(1812—1880),有机化学家,俄国化学学派的领导人。

[12]罗巴切夫斯基(1792—1856),俄罗斯数学家,非欧几何的早期发现人之一。

[13]《马克斯和莫里茨》是德国诗人、画家威廉·布施(1832—1908)于1865年发表的讽刺插图故事,被认为是现代连环漫画的主要先驱之一。阿纳托尔·法朗士(1844—1924)是法国小说家,1921年诺贝尔文学奖获得者。

[14]英国化学家、生理学家威廉·蒲劳脱(1785—1850)于1815年提出,所有物质都是由氢构成的,其他元素的原子量都是氢原子量的整数倍,称为蒲劳脱假说。

[15]杜马(1800—1884)和斯塔斯(1813—1891)分别是法国化学家和比利时化学家。

[16]亥姆霍兹(1821—1894),德国物理学家。出版《能量的保存》一书阐明能量守恒的原理,“亥姆霍兹自由能”以他来命名。他也研究过电磁学,预测了麦克斯韦方程组中的电磁辐射。

[17]维克托的爱称。

[18]维克托的爱称。

[19]普朗克(1858—1947),德国物理学家,量子论创始人。

[20]即赫鲁晓夫。

[21]即贝利亚。

[22]邓尼金和弗兰格尔都是苏联内战时期白军武装头目。

[23]科尔尼洛夫(1870—1918),俄国上将,1917年反革命叛乱的头目。

[24]崩得是俄文译音,意为“联盟”,是“立陶宛、波兰和俄罗斯犹太工人总联盟”的简称。

[25]费特·阿法纳西·阿法纳西耶维奇(1820—1892),俄国诗人,诗作有着俄罗斯古典浪漫主义风格,以其独特的魅力和音乐性征服了当时文坛许多名家。六十年代初创作激情衰退,专事农庄经营,晚年又重新执笔。

[26]费多尔·伊凡诺维奇·丘特切夫(1803—1873),十九世纪俄罗斯著名抒情诗人。哲学观点受谢林唯心主义影响,诗作除刻画自然外,还有热烈的感情和深沉的思考。

[27]米哈伊尔·伊万诺维奇·格林卡(1804—1857),俄罗斯民族乐派作曲家。

[28]拉斯普京(1872—1916),沙皇尼古拉二世的宠臣,东正教“长老”和“神医”。

[29]尼古拉·谢苗诺维奇·列斯科夫(1831—1895),十九世纪俄国小说家,对契诃夫、高尔基等人的小说产生过重大影响。主要作品有《姆岑斯克县的麦克白夫人》《奇人录》《大堂神父》等。

[30]梅列日科夫斯基(1865—1941),俄国诗人、历史小说家、批评家和思想家。1893年发表《论现代俄国文学衰落的原因及新流派》一文,是俄国现代主义的重要里程碑。十月革命前反对沙皇政府,他欢迎二月革命,但反对布尔什维克当政。

[31]别雷(1880—1934),俄罗斯象征主义文学中最有影响力的作家之一,代表作品有长诗《交响曲》、长篇小说《银鸽》《彼得堡》等。

[32]巴尔蒙特(1867—1942),诗人,评论家,翻译家。诗集《在北方的天空下》《在无穷之中》《静》是俄罗斯象征主义的奠基之作。

[33]米留可夫(1859—1943),俄罗斯历史学家,西方派的代表人物。

[34]叶夫列伊诺夫(1879—1953),俄罗斯著名导演、剧作家,戏剧理论家,俄罗斯象征主义的核心人物,二十世纪二十年代离开俄罗斯,侨居巴黎。

[35]列米佐夫(1877—1957),俄罗斯“白银时代”著名现代派作家,二十年代侨居巴黎。

[36]维亚切斯拉夫·伊万诺夫(1866—1949),俄罗斯象征主义诗人、剧作家、哲学家、批评家。

[37]俄罗斯童话《阿廖努什卡和伊万努什卡姐弟的故事》中,孤苦伶仃的阿廖努什卡曾来到林中,坐在河岸哭诉自己的遭遇。

[38]梁赞位于俄罗斯中部联邦管区奥卡河畔,是梁赞州的行政中心。

[39]舍列梅捷夫家族在十七、十八世纪的俄国地位显赫。

[40]僚机(wingman),编队飞行中跟随长机执行任务的飞机。僚机应保持在编队中规定的位置,观察空中情况,执行长机的命令。

[41]刻赤半岛位于克里米亚半岛的东端。刻赤城是重要的港市。

[42]贫民吸的一种劣质烟,由黄花烟草的茎叶制成。

[43]布勃诺夫(1883—1940),苏联党务和国务活动家,军事家,革命家,1929年起任俄罗斯联邦教育人民委员,1940年在大肃反中被捕处决,后平反。

[44]索科尔尼科夫(1888—1939),俄国革命家、经济学家,前苏联政治家。1937年被捕,被判处十年徒刑,在狱中被杀,后平反。

[45]二十世纪初俄国极右翼组织,宣扬极端俄罗斯民族主义,仇外心理和反犹主义,煽动大屠杀。

[46]艾瓦尔德·冯·克莱斯特(1881—1954),法西斯德国陆军元帅,时任苏德战场南翼坦克第一集团军群司令。

[47]即英吉利海峡。

[48]厄尔布鲁士山被认为是欧洲第一高峰,位于俄罗斯西南部大高加索山脉。

[49]原文为犹太语。

[50]原文为德语Scharführer。

[51]原文为德语。

[52]犹太语:水壶,胶合板,胶土,酸奶,浮萍,稻草人,懒惰,小猫。

[53]乌克兰诗人、艺术家塔拉斯·谢甫琴科(1814—1861)的诗集。谢甫琴科的文学作品被视为近现代乌克兰文学的奠基者。

[54]尼古拉·叶若夫(1895—1940),苏联政治人物,斯大林大清洗计划的主要执行者之一,1936年至1938年任苏联内务人民委员(内务人民委员会是苏联斯大林时代的主要秘密警察机构),其间实行残酷清洗。

[55]国家政治保卫总局,拉丁字母转写缩写为OGPU,是1923年至1934年苏联的情报机构。

[56]指一八一二年俄国抗击拿破仑入侵的战争。

[57]《伊戈尔远征记》,俄罗斯古代英雄史诗,著者不详,以十二世纪罗斯王公伊戈尔一次失败的远征为史实依据。

[58]洛巴切夫斯基(1792—1856),俄国数学家、几何学家。

[59]达佛尼斯和克洛伊是希腊神话中两小无猜的牧羊人和牧羊女,历经磨难,终成眷属,是被后人视为楷模的一对天真无邪的情侣。

[60]列夫·托尔斯泰的诞生地。

[61]索科洛夫的名字和父称。

[62]左琴科(1895—1958),苏联著名幽默讽刺作家。

[63]斯克里亚宾(1872—1915),俄国交响乐作曲家、钢琴音乐大师。

[64]一种化学危险品,可因震动而爆炸。

[65]柳博奇卡·阿克雪里罗德(1868—1946),俄国哲学家、艺术家,孟什维克。

[66]普加乔夫、拉辛均为俄国农民起义领袖。

[67]苏杰伊金、杰加耶夫均为十九世纪沙俄密探局官员。

[68]原文为德语。

[69]原文为德语。
